\documentclass[a4paper,11pt]{report}
\usepackage{hyperref}
\usepackage[table]{xcolor}
\usepackage{listings}
\usepackage{lmodern}
\usepackage[left=0.75in, right=0.75in, top=0.75in, bottom=0.75in]{geometry}
\usepackage{graphicx}
\usepackage{amsmath,amssymb}
\usepackage{tikz}
\usepackage{pgfplots}
\usepackage{subfigure}
\usepackage{enumerate}
\usepackage{tcolorbox}
\usepackage{fancyhdr}
\usepackage{setspace}
\usepackage{cancel}
\usepackage{placeins}
\usepackage{multirow}
\usepackage{algorithm2e}
\usepackage{booktabs}
\usepackage{bbding}
\pagecolor{white}
\color{black}

\pagestyle{fancy}

\hypersetup{%
  colorlinks=true,% hyperlinks will be black
  linkbordercolor=red,% hyperlink borders will be red
  pdfborderstyle={/S/U/W 1}% border style will be underline of width 1pt
}

\lstset{
  basicstyle=\footnotesize\rmfamily,
  commentstyle=\mdseries\rmfamily,
  numbers=left,
  numberstyle=\footnotesize\rmfamily,
  stepnumber=1,
  numbersep=5pt,
  backgroundcolor=\color{white},
  showspaces=false,
  showstringspaces=false,
  showtabs=false,
  tabsize=4,
  captionpos=b,
  breaklines=true,
  breakatwhitespace=false,
}

\newcommand{\soln}{\\ \textbf{Solution}: }
\newcommand{\bkt}[1]{\left(#1\right)}

\lhead{MA5892: NMSC}
\chead{End Semester Exam}
\rhead{Roll number: PH15M015}

\begin{document}
\doublespacing
\begin{enumerate}

    \item An integral of the form $\displaystyle \int_{0}^{1} f(x) dx$ was computed
    by two students using trapezoidal rule with end point corrections involving only
    the first derivative. The first student reported the value as $0.8$ using a grid
    spacing of $h$, while the second student reported the value as $0.75$ using a grid
    spacing of $h/2$. A smart, lazy student from NMSC class enters their discussion
    and gives a better answer of the integral with a higher order of accuracy by processing
    the information above. How did he do it? What was his answer? What is the order of
    accuracy of his answer?

    \item Gaussian quadrature:
    \begin{itemize}
        \item Find the first three monic polynomials (i.e., till quadratic) on $[0, 1]$
        orthogonal with respect to the inner product

            \begin{equation*}
            \langle f, g \rangle = \int_{0}^{1} \frac{x}{\sqrt{1 - x^{2}}} f(x) g(x) dx
            \end{equation*}

        \item Use the above to find a quadrature formula of the form

            \begin{equation*}
            \int_{0}^{1} \frac{x}{\sqrt{1 - x^{2}}} f(x) dx = \sum_{i=0}^{n} a_{i} f(x_{i})
            \end{equation*}

        that is exact for all $f(x)$ of degree $3$.

        \item Use the above to evaluate $\displaystyle \int_{0}^{1} \frac{x \sin(x)}
        {\sqrt{1 - x^{2}}} dx$
    \end{itemize}

    \textbf{Solution:}

    Given two functions $f, g \in C[0, 1]$. we define an inner product of these two functions by
    \begin{equation*}
    \langle f, g \rangle = \int_{0}^{1} w(x) f(x) g(x) dx, \hspace{2cm} w(x) > 0
    \end{equation*}
        
    Thus the definition of the inner product depends on the integration interval $[0, 1]$ and a given weight function $w(x)$.

    For the above inner product we also have
    \begin{equation*}
    \langle xf, g \rangle_{w} = \int_{0}^{1} w(x)\ x f(x) g(x)\ dx = \langle f, xg \rangle_{w}
    \end{equation*}

    We create a sequence of polynomials $\phi_{k}(x)$ of degree $k$ for $k = 0, 1, 2, 3, ...$ such that
    $\langle \phi_{i}, \phi_{j} \rangle_{w} = 0$ for all $i \neq j$.

    We know that the Chebyshev polynomials of the first kind $T_{n}(x)$ are orthogonal within the interval $x \in [-1, 1]$ with a weight function 
    $w(x) = \displaystyle \frac{1}{\sqrt{1 - x^{2}}}$,

    \begin{equation*}
    \int_{-1}^{1} \frac{1}{\sqrt{1 - x^2}}\ T_{i}(x) T_{j}(x)\ dx = 
    \begin{cases}
        0 & i \neq j \\
        \pi/2 & i = j \neq 0 \\
        \pi & i = j = 0
    \end{cases}
    \end{equation*}

    We also know that the Chebyshev polynomials of the first kind $T_{n}(y)$ are orthogonal polynomials on $[0,1]$ with respect to the weight function
    $w(y) = \displaystyle \frac{1}{\sqrt{4y - 4y^2}}$,

    \begin{equation*}
    \int_{0}^{1} \frac{1}{\sqrt{4y - 4y^2}}\ T_{i}(y) T_{j}(y)\ dy = 
    \begin{cases}
        0 & i \neq j \\
        \pi/4 & i = j \neq 0 \\
        \pi/2 & i = j = 0
    \end{cases}
    \end{equation*}

    We can transform any finite domain $a \le y \le b$ to the basic domain $-1 \le x \le 1$ with the change of variable
    $y = \displaystyle \frac{1}{2} (b - a)x + \frac{1}{2} (b + a)$.

    For the domain $0 \le y \le 1$, we can write $y = \displaystyle \frac{1}{2} (x + 1)$.
    
    Let us now find the sequence of orthogonal polynomials. This is done by a Gram-Schmidt process. 

    let, $\phi_{0}(x) = 1$

    then, $\phi_{1}(x) = x - B_{1} \phi_{0}(x)$, where $B_{1} = \displaystyle \frac{\langle x \phi_{0}, \phi_{0} \rangle_{w}}{\| \phi_{0} \|_{w}^{2}}$
    
    \begin{equation*}
    \begin{aligned}
    \phi_{1}(y) &= (2y-1) - \displaystyle \frac{\langle (2y-1) \phi_{0}, \phi{0} \rangle}{\langle \phi_{0}, \phi_{0} \rangle}\ \phi_{0}(2y-1) \\
                &= (2y-1) - \displaystyle \frac{\langle (2y-1), 1 \rangle}{\langle 1, 1 \rangle} \\
                &= (2y-1) - \displaystyle \frac{\displaystyle\int_{0}^{1} \frac{1}{\sqrt{4y-y^2}} (2y-1) dy}{\displaystyle\int_{0}^{1} \frac{1}{\sqrt{4y-y^2}} dy} \\
                &= y - \frac{1}{2}
    \end{aligned}
    \end{equation*}

    and $\phi_{k}(x) = (x - B_{k}) \phi_{k-1}(x) - C_{k} \phi_{k-2}(x), \hspace{1cm} k \ge 2$ \\
    with, $B_{k} = \displaystyle \frac{\langle x \phi_{k-1}, \phi_{k-1} \rangle_{w}}{\| \phi_{k-1} \|_{w}^{2}}$ and
    $C_{k} = \displaystyle \frac{\langle x \phi_{k-1}, \phi_{k-2} \rangle_{w}}{\| \phi_{k-2} \|_{w}^{2}} = \frac{\| \phi_{k-1} \|_{w}^{2}}{\| \phi_{k-2} \|_{w}^{2}}$

    \begin{equation*}
    \begin{aligned}
    \phi_{2}(y) &= \big((2y-1) - B_{2}\big)\ \phi_{1}(2y-1) - C_{2}\ \phi_{0}(2y-1) \\
                &= \bigg((2y-1) - \frac{\langle (2y-1) \phi_{1}, \phi_{1} \rangle}{\langle \phi_{1}, \phi_{1} \rangle}\bigg)\ \phi_{1}(2y-1) - \frac{\langle (2y-1) \phi_{1}, \phi_{0} \rangle}{\langle \phi_{0}, \phi_{0} \rangle}\ \phi_{0}(2y-1) \\
                &= (2y-1)^2 - \frac{\langle (2y-1)(2y-1), (2y-1) \rangle}{\langle 2y-1, 2y-1 \rangle}\ (2y-1) - \frac{\langle (2y-1)(2y-1), 1 \rangle}{\langle 1, 1 \rangle} \\
                &= (4y^2 - 4y + 1) - \frac{\langle 4y^2 - 4y + 1, 2y-1 \rangle}{\langle 2y-1, 2y-1 \rangle}\ (2y-1) - \frac{\langle 4y^2 - 4y + 1, 1 \rangle}{\langle 1, 1 \rangle} \\
                &= (4y^2 - 4y + 1) - \frac{\displaystyle \int_{0}^{1} \frac{(4y^2 - 4y + 1)(2y-1)}{\sqrt{4y-4y^2}} dy}{\displaystyle \int_{0}^{1} \frac{(2y-1)(2y-1)}{\sqrt{4y-4y^2}} dy}\ (2y-1) - \frac{\displaystyle \int_{0}^{1} \frac{(4y^2 - 4y + 1)}{\sqrt{4y-4y^2}} dy}{\displaystyle \int_{0}^{1} \frac{1}{\sqrt{4y-4y^2}} dy} \\
                &= (4y^2 - 4y + 1) - \displaystyle \frac{\pi/4}{\pi/2} \\
                &= y^2 - y + \frac{1}{8}
    \end{aligned}
    \end{equation*}
    

   \textbf{Solution:} 

    Let $\langle f, g \rangle_{w} = \displaystyle \int_{a}^{b} w(x) f(x) g(x)\ dx$ (the weighted inner product). By the Gram-Schmidt process, there is a sequence
    $\{\phi_{j}\}$ of orthogonal polynomials where $\phi_{j}$ has degree $n$. $\phi_{n+1}$ has $n+1$ distinct real zeros $x_{0}, ..., x_{n}$ in $[a, b]$. 

    Let $l_{k} (x)$ be the $k$-th Lagrange basis polynomial for these zeros and let
    
    \begin{equation*}
    c_{k} = \int_{a}^{b} l_{k} (x)\ dx
    \end{equation*}    

    The claim is that with this set of $x_{k}$'s and $c_{k}$'s, $\displaystyle \int_{a}^{b} w(x) f(x)\ dx \approx \sum_{k=0}^{n} c_{k} f(x_{k})$ has degree $2n+1$. \\

    Suppose $f \in \mathbb{P}_{2n+1}$. Since $p_{n+1}$ has degree $n+1$, polynomial division gives
    \begin{equation*}
    f = q(x)p_{n+1} (x) + r(x), \hspace{2cm} q, r \in \mathbb{P}_{n}.
    \end{equation*}

    Plugging this expression into the integral,

    \begin{equation*}
    \begin{aligned}
    I &= \int_{a}^{b} (q(x)p_{n+1}(x) + r(x)) w(x) dx \\
      &= \langle q, p_{n+1} \rangle_{w} + \int_{a}^{b} r(x) w(x)\ dx \\
      &= \int_{a}^{b} r(x) w(x)\ dx
    \end{aligned}
    \end{equation*}

    because $p_{n+1}$ is orthogonal to all polynomials of degree $\le n$, which includes $q$. Now plug the expression into the formula:

    \begin{equation*}
    \begin{aligned}
    \text{formula} &= \sum_{k=0}^{n} c_{k} f(x_{k}) \\
    &= \sum_{k=0}^{n} c_{k} q(x_{k}) p_{n+1}(x_{k}) + \sum_{k=0}^{n} c_{k} r(x_{k}) \\
    &= \sum_{k=0}^{n} c_{k} r(x_{k})
    \end{aligned}
    \end{equation*}
 
    Last, we need to establish that $I$ and the formula are equal. Because $r(x)$ has degree $\le n$, it is equal to its Lagrange interpolant through the nodes $x_{0},...,x_{n}$, so

    \begin{equation*}
    r(x) = \sum_{k=0}^{n} r(x_{k}) l_{k}(x)
    \end{equation*}

    Thus, working from the formula for $I$,
    
    \begin{equation*}
    I = \int_{a}^{b} r(x) w(x)\ dx = \sum_{k=0}^{n} \int_{a}^{b} l_{k}(x) w(x)\ dx = \sum_{k=0}^{n} c_{k} r(x_{k})
    \end{equation*}
    
    which establishes equality. To see that the degree of accuracy is exactly $2n+1$, consider
    
    \begin{equation*}
    f(x) = \prod_{j=0}^{n} (x-x_{j})^2
    \end{equation*}

    Note that the nodes $x_{k}$ depend on the degree, so really they should be written $x_{n,k}\ (\text{for} k = 0,...,n)$ for $\phi_{n+1}$. One can show that, unlike with equally spaced interpolation,
    
    \begin{equation*}
    \lim_{x\to\infty} |I - \sum_{k=0}^{n} c_{k} f(x_{n,k})| = 0
    \end{equation*}

    under reasonable assumptions on $f$. Thus, Gaussian quadrature does well when adding more points to reduce error when function values of $f$ at any point are available.

    In summary, let $\{\phi_{j}\}$ be an orthogonal basis of polynomials in the inner product $\langle f, g \rangle_{w} = \displaystyle \int_{a}^{b} w(x) f(x) g(x)\ dx$ and let $x_{0},...,x_{n}$ be the zeros of the polynomial $\phi_{n+1}$ with Lagrange basis $\{l_{k}(x)\}$. Then

    \begin{equation*}
    I = \int_{a}^{b} w(x) f(x)\ dx \approx \sum_{k=0}^{n} c_{k} f(x_{k}), \hspace{2cm} c_{k} = \int_{a}^{b} l_{k}(x) w(x)\ dx,
    \end{equation*}

    called the Gaussian quadrature formula for $w(x)$, has degree of accuracy $2n+1$.

























    \item Comment on using the Newton method to compute the root of the function $f(x)
    = x^{1/3}$, i.e., if your initial guess is $x_{0} = 1$, what would be the value of
    $x_{n}$? Does the method converge to the root we want?

    \item It is given that a sequence of Newton iterates converge to a root $r$ of the
    function $f(x)$. Further, it is given that the root $r$ is a root of multiplicity
    $2$, i.e., $f(x) = (x - r)^{2} g(x)$, where $g(r) \ne 0$. It is also given that the
    function $f$, its derivatives till the second order are continuous in the
    neighbourhood of the root $r$. If $e_{n}$ is the error of the $n^{th}$ iterate,
    i.e., $e_{n} = x_{n} - r$, then obtain

        \begin{equation*}
        \lim_{n \rightarrow \infty} \frac{e_{n+1}}{e_{n}}
        \end{equation*}

    \item \textbf{Bonus question:} What happens to the above if the root $r$ has a
    multiplicity $m$?

    \item Compute $\displaystyle \int_{-1}^{1} e^{-x^{2}} dx$ using the

    \begin{enumerate}
    \item Trapezoidal rule
    \item Trapezoidal rule with end corrections using the first derivative
    \item Trapezoidal rule with end corrections using the first derivative and third
    derivatives
    \item Gauss-Legendre quadrature
    \end{enumerate}

    \begin{itemize}
    \item Perform this by subdividing $[-1, 1]$ into $N \in \{2, 5, 10, 20, 50, 100\}$ 
    panels.
    \item Plot the decay of the absolute error using the above methods.
    \item You may obtain the exact value of the integral up to $20$ digits using
    wolfram alpha.
    \item Make sure the figure has a legend and the axes are clearly marked.
    \item Ensure that the font size for title, axes, legend are readable.
    \item Submit the plots obtained, entire code and the write-up.
    \end{itemize}
        
    \textbf{Program:}
    \lstinputlisting[language=Python]{Scripts/program6.py}

    \begin{figure}[ht!]
    \centering
    \resizebox{0.9\linewidth}{!}{%% Creator: Matplotlib, PGF backend
%%
%% To include the figure in your LaTeX document, write
%%   \input{<filename>.pgf}
%%
%% Make sure the required packages are loaded in your preamble
%%   \usepackage{pgf}
%%
%% Also ensure that all the required font packages are loaded; for instance,
%% the lmodern package is sometimes necessary when using math font.
%%   \usepackage{lmodern}
%%
%% Figures using additional raster images can only be included by \input if
%% they are in the same directory as the main LaTeX file. For loading figures
%% from other directories you can use the `import` package
%%   \usepackage{import}
%%
%% and then include the figures with
%%   \import{<path to file>}{<filename>.pgf}
%%
%% Matplotlib used the following preamble
%%
\begingroup%
\makeatletter%
\begin{pgfpicture}%
\pgfpathrectangle{\pgfpointorigin}{\pgfqpoint{8.000000in}{6.000000in}}%
\pgfusepath{use as bounding box, clip}%
\begin{pgfscope}%
\pgfsetbuttcap%
\pgfsetmiterjoin%
\definecolor{currentfill}{rgb}{1.000000,1.000000,1.000000}%
\pgfsetfillcolor{currentfill}%
\pgfsetlinewidth{0.000000pt}%
\definecolor{currentstroke}{rgb}{1.000000,1.000000,1.000000}%
\pgfsetstrokecolor{currentstroke}%
\pgfsetdash{}{0pt}%
\pgfpathmoveto{\pgfqpoint{0.000000in}{0.000000in}}%
\pgfpathlineto{\pgfqpoint{8.000000in}{0.000000in}}%
\pgfpathlineto{\pgfqpoint{8.000000in}{6.000000in}}%
\pgfpathlineto{\pgfqpoint{0.000000in}{6.000000in}}%
\pgfpathlineto{\pgfqpoint{0.000000in}{0.000000in}}%
\pgfpathclose%
\pgfusepath{fill}%
\end{pgfscope}%
\begin{pgfscope}%
\pgfsetbuttcap%
\pgfsetmiterjoin%
\definecolor{currentfill}{rgb}{1.000000,1.000000,1.000000}%
\pgfsetfillcolor{currentfill}%
\pgfsetlinewidth{0.000000pt}%
\definecolor{currentstroke}{rgb}{0.000000,0.000000,0.000000}%
\pgfsetstrokecolor{currentstroke}%
\pgfsetstrokeopacity{0.000000}%
\pgfsetdash{}{0pt}%
\pgfpathmoveto{\pgfqpoint{1.000000in}{0.600000in}}%
\pgfpathlineto{\pgfqpoint{7.200000in}{0.600000in}}%
\pgfpathlineto{\pgfqpoint{7.200000in}{5.400000in}}%
\pgfpathlineto{\pgfqpoint{1.000000in}{5.400000in}}%
\pgfpathlineto{\pgfqpoint{1.000000in}{0.600000in}}%
\pgfpathclose%
\pgfusepath{fill}%
\end{pgfscope}%
\begin{pgfscope}%
\pgfpathrectangle{\pgfqpoint{1.000000in}{0.600000in}}{\pgfqpoint{6.200000in}{4.800000in}}%
\pgfusepath{clip}%
\pgfsetbuttcap%
\pgfsetroundjoin%
\pgfsetlinewidth{1.003750pt}%
\definecolor{currentstroke}{rgb}{0.000000,0.000000,1.000000}%
\pgfsetstrokecolor{currentstroke}%
\pgfsetdash{{6.000000pt}{6.000000pt}}{0.000000pt}%
\pgfpathmoveto{\pgfqpoint{1.000000in}{5.363882in}}%
\pgfpathlineto{\pgfqpoint{2.365992in}{4.947020in}}%
\pgfpathlineto{\pgfqpoint{3.399326in}{4.734862in}}%
\pgfpathlineto{\pgfqpoint{4.432659in}{4.539989in}}%
\pgfpathlineto{\pgfqpoint{5.798651in}{4.293083in}}%
\pgfpathlineto{\pgfqpoint{6.831985in}{4.109814in}}%
\pgfusepath{stroke}%
\end{pgfscope}%
\begin{pgfscope}%
\pgfpathrectangle{\pgfqpoint{1.000000in}{0.600000in}}{\pgfqpoint{6.200000in}{4.800000in}}%
\pgfusepath{clip}%
\pgfsetbuttcap%
\pgfsetroundjoin%
\definecolor{currentfill}{rgb}{0.000000,0.000000,1.000000}%
\pgfsetfillcolor{currentfill}%
\pgfsetlinewidth{0.501875pt}%
\definecolor{currentstroke}{rgb}{0.000000,0.000000,1.000000}%
\pgfsetstrokecolor{currentstroke}%
\pgfsetdash{}{0pt}%
\pgfsys@defobject{currentmarker}{\pgfqpoint{-0.020833in}{-0.020833in}}{\pgfqpoint{0.020833in}{0.020833in}}{%
\pgfpathmoveto{\pgfqpoint{0.000000in}{-0.020833in}}%
\pgfpathcurveto{\pgfqpoint{0.005525in}{-0.020833in}}{\pgfqpoint{0.010825in}{-0.018638in}}{\pgfqpoint{0.014731in}{-0.014731in}}%
\pgfpathcurveto{\pgfqpoint{0.018638in}{-0.010825in}}{\pgfqpoint{0.020833in}{-0.005525in}}{\pgfqpoint{0.020833in}{0.000000in}}%
\pgfpathcurveto{\pgfqpoint{0.020833in}{0.005525in}}{\pgfqpoint{0.018638in}{0.010825in}}{\pgfqpoint{0.014731in}{0.014731in}}%
\pgfpathcurveto{\pgfqpoint{0.010825in}{0.018638in}}{\pgfqpoint{0.005525in}{0.020833in}}{\pgfqpoint{0.000000in}{0.020833in}}%
\pgfpathcurveto{\pgfqpoint{-0.005525in}{0.020833in}}{\pgfqpoint{-0.010825in}{0.018638in}}{\pgfqpoint{-0.014731in}{0.014731in}}%
\pgfpathcurveto{\pgfqpoint{-0.018638in}{0.010825in}}{\pgfqpoint{-0.020833in}{0.005525in}}{\pgfqpoint{-0.020833in}{0.000000in}}%
\pgfpathcurveto{\pgfqpoint{-0.020833in}{-0.005525in}}{\pgfqpoint{-0.018638in}{-0.010825in}}{\pgfqpoint{-0.014731in}{-0.014731in}}%
\pgfpathcurveto{\pgfqpoint{-0.010825in}{-0.018638in}}{\pgfqpoint{-0.005525in}{-0.020833in}}{\pgfqpoint{0.000000in}{-0.020833in}}%
\pgfpathlineto{\pgfqpoint{0.000000in}{-0.020833in}}%
\pgfpathclose%
\pgfusepath{stroke,fill}%
}%
\begin{pgfscope}%
\pgfsys@transformshift{1.000000in}{5.363882in}%
\pgfsys@useobject{currentmarker}{}%
\end{pgfscope}%
\begin{pgfscope}%
\pgfsys@transformshift{2.365992in}{4.947020in}%
\pgfsys@useobject{currentmarker}{}%
\end{pgfscope}%
\begin{pgfscope}%
\pgfsys@transformshift{3.399326in}{4.734862in}%
\pgfsys@useobject{currentmarker}{}%
\end{pgfscope}%
\begin{pgfscope}%
\pgfsys@transformshift{4.432659in}{4.539989in}%
\pgfsys@useobject{currentmarker}{}%
\end{pgfscope}%
\begin{pgfscope}%
\pgfsys@transformshift{5.798651in}{4.293083in}%
\pgfsys@useobject{currentmarker}{}%
\end{pgfscope}%
\begin{pgfscope}%
\pgfsys@transformshift{6.831985in}{4.109814in}%
\pgfsys@useobject{currentmarker}{}%
\end{pgfscope}%
\end{pgfscope}%
\begin{pgfscope}%
\pgfpathrectangle{\pgfqpoint{1.000000in}{0.600000in}}{\pgfqpoint{6.200000in}{4.800000in}}%
\pgfusepath{clip}%
\pgfsetbuttcap%
\pgfsetroundjoin%
\pgfsetlinewidth{1.003750pt}%
\definecolor{currentstroke}{rgb}{1.000000,0.000000,0.000000}%
\pgfsetstrokecolor{currentstroke}%
\pgfsetdash{{6.000000pt}{6.000000pt}}{0.000000pt}%
\pgfpathmoveto{\pgfqpoint{1.000000in}{5.228140in}}%
\pgfpathlineto{\pgfqpoint{2.365992in}{4.319976in}}%
\pgfpathlineto{\pgfqpoint{3.399326in}{3.899255in}}%
\pgfpathlineto{\pgfqpoint{4.432659in}{3.510101in}}%
\pgfpathlineto{\pgfqpoint{5.798651in}{3.016429in}}%
\pgfpathlineto{\pgfqpoint{6.831985in}{2.649910in}}%
\pgfusepath{stroke}%
\end{pgfscope}%
\begin{pgfscope}%
\pgfpathrectangle{\pgfqpoint{1.000000in}{0.600000in}}{\pgfqpoint{6.200000in}{4.800000in}}%
\pgfusepath{clip}%
\pgfsetbuttcap%
\pgfsetroundjoin%
\definecolor{currentfill}{rgb}{1.000000,0.000000,0.000000}%
\pgfsetfillcolor{currentfill}%
\pgfsetlinewidth{0.501875pt}%
\definecolor{currentstroke}{rgb}{1.000000,0.000000,0.000000}%
\pgfsetstrokecolor{currentstroke}%
\pgfsetdash{}{0pt}%
\pgfsys@defobject{currentmarker}{\pgfqpoint{-0.020833in}{-0.020833in}}{\pgfqpoint{0.020833in}{0.020833in}}{%
\pgfpathmoveto{\pgfqpoint{0.000000in}{-0.020833in}}%
\pgfpathcurveto{\pgfqpoint{0.005525in}{-0.020833in}}{\pgfqpoint{0.010825in}{-0.018638in}}{\pgfqpoint{0.014731in}{-0.014731in}}%
\pgfpathcurveto{\pgfqpoint{0.018638in}{-0.010825in}}{\pgfqpoint{0.020833in}{-0.005525in}}{\pgfqpoint{0.020833in}{0.000000in}}%
\pgfpathcurveto{\pgfqpoint{0.020833in}{0.005525in}}{\pgfqpoint{0.018638in}{0.010825in}}{\pgfqpoint{0.014731in}{0.014731in}}%
\pgfpathcurveto{\pgfqpoint{0.010825in}{0.018638in}}{\pgfqpoint{0.005525in}{0.020833in}}{\pgfqpoint{0.000000in}{0.020833in}}%
\pgfpathcurveto{\pgfqpoint{-0.005525in}{0.020833in}}{\pgfqpoint{-0.010825in}{0.018638in}}{\pgfqpoint{-0.014731in}{0.014731in}}%
\pgfpathcurveto{\pgfqpoint{-0.018638in}{0.010825in}}{\pgfqpoint{-0.020833in}{0.005525in}}{\pgfqpoint{-0.020833in}{0.000000in}}%
\pgfpathcurveto{\pgfqpoint{-0.020833in}{-0.005525in}}{\pgfqpoint{-0.018638in}{-0.010825in}}{\pgfqpoint{-0.014731in}{-0.014731in}}%
\pgfpathcurveto{\pgfqpoint{-0.010825in}{-0.018638in}}{\pgfqpoint{-0.005525in}{-0.020833in}}{\pgfqpoint{0.000000in}{-0.020833in}}%
\pgfpathlineto{\pgfqpoint{0.000000in}{-0.020833in}}%
\pgfpathclose%
\pgfusepath{stroke,fill}%
}%
\begin{pgfscope}%
\pgfsys@transformshift{1.000000in}{5.228140in}%
\pgfsys@useobject{currentmarker}{}%
\end{pgfscope}%
\begin{pgfscope}%
\pgfsys@transformshift{2.365992in}{4.319976in}%
\pgfsys@useobject{currentmarker}{}%
\end{pgfscope}%
\begin{pgfscope}%
\pgfsys@transformshift{3.399326in}{3.899255in}%
\pgfsys@useobject{currentmarker}{}%
\end{pgfscope}%
\begin{pgfscope}%
\pgfsys@transformshift{4.432659in}{3.510101in}%
\pgfsys@useobject{currentmarker}{}%
\end{pgfscope}%
\begin{pgfscope}%
\pgfsys@transformshift{5.798651in}{3.016429in}%
\pgfsys@useobject{currentmarker}{}%
\end{pgfscope}%
\begin{pgfscope}%
\pgfsys@transformshift{6.831985in}{2.649910in}%
\pgfsys@useobject{currentmarker}{}%
\end{pgfscope}%
\end{pgfscope}%
\begin{pgfscope}%
\pgfpathrectangle{\pgfqpoint{1.000000in}{0.600000in}}{\pgfqpoint{6.200000in}{4.800000in}}%
\pgfusepath{clip}%
\pgfsetbuttcap%
\pgfsetroundjoin%
\pgfsetlinewidth{1.003750pt}%
\definecolor{currentstroke}{rgb}{0.750000,0.000000,0.750000}%
\pgfsetstrokecolor{currentstroke}%
\pgfsetdash{{6.000000pt}{6.000000pt}}{0.000000pt}%
\pgfpathmoveto{\pgfqpoint{1.000000in}{5.191594in}}%
\pgfpathlineto{\pgfqpoint{2.365992in}{3.790945in}}%
\pgfpathlineto{\pgfqpoint{3.399326in}{3.120259in}}%
\pgfpathlineto{\pgfqpoint{4.432659in}{2.528961in}}%
\pgfpathlineto{\pgfqpoint{5.798651in}{1.786554in}}%
\pgfpathlineto{\pgfqpoint{6.831985in}{1.237377in}}%
\pgfusepath{stroke}%
\end{pgfscope}%
\begin{pgfscope}%
\pgfpathrectangle{\pgfqpoint{1.000000in}{0.600000in}}{\pgfqpoint{6.200000in}{4.800000in}}%
\pgfusepath{clip}%
\pgfsetbuttcap%
\pgfsetroundjoin%
\definecolor{currentfill}{rgb}{0.750000,0.000000,0.750000}%
\pgfsetfillcolor{currentfill}%
\pgfsetlinewidth{0.501875pt}%
\definecolor{currentstroke}{rgb}{0.750000,0.000000,0.750000}%
\pgfsetstrokecolor{currentstroke}%
\pgfsetdash{}{0pt}%
\pgfsys@defobject{currentmarker}{\pgfqpoint{-0.020833in}{-0.020833in}}{\pgfqpoint{0.020833in}{0.020833in}}{%
\pgfpathmoveto{\pgfqpoint{0.000000in}{-0.020833in}}%
\pgfpathcurveto{\pgfqpoint{0.005525in}{-0.020833in}}{\pgfqpoint{0.010825in}{-0.018638in}}{\pgfqpoint{0.014731in}{-0.014731in}}%
\pgfpathcurveto{\pgfqpoint{0.018638in}{-0.010825in}}{\pgfqpoint{0.020833in}{-0.005525in}}{\pgfqpoint{0.020833in}{0.000000in}}%
\pgfpathcurveto{\pgfqpoint{0.020833in}{0.005525in}}{\pgfqpoint{0.018638in}{0.010825in}}{\pgfqpoint{0.014731in}{0.014731in}}%
\pgfpathcurveto{\pgfqpoint{0.010825in}{0.018638in}}{\pgfqpoint{0.005525in}{0.020833in}}{\pgfqpoint{0.000000in}{0.020833in}}%
\pgfpathcurveto{\pgfqpoint{-0.005525in}{0.020833in}}{\pgfqpoint{-0.010825in}{0.018638in}}{\pgfqpoint{-0.014731in}{0.014731in}}%
\pgfpathcurveto{\pgfqpoint{-0.018638in}{0.010825in}}{\pgfqpoint{-0.020833in}{0.005525in}}{\pgfqpoint{-0.020833in}{0.000000in}}%
\pgfpathcurveto{\pgfqpoint{-0.020833in}{-0.005525in}}{\pgfqpoint{-0.018638in}{-0.010825in}}{\pgfqpoint{-0.014731in}{-0.014731in}}%
\pgfpathcurveto{\pgfqpoint{-0.010825in}{-0.018638in}}{\pgfqpoint{-0.005525in}{-0.020833in}}{\pgfqpoint{0.000000in}{-0.020833in}}%
\pgfpathlineto{\pgfqpoint{0.000000in}{-0.020833in}}%
\pgfpathclose%
\pgfusepath{stroke,fill}%
}%
\begin{pgfscope}%
\pgfsys@transformshift{1.000000in}{5.191594in}%
\pgfsys@useobject{currentmarker}{}%
\end{pgfscope}%
\begin{pgfscope}%
\pgfsys@transformshift{2.365992in}{3.790945in}%
\pgfsys@useobject{currentmarker}{}%
\end{pgfscope}%
\begin{pgfscope}%
\pgfsys@transformshift{3.399326in}{3.120259in}%
\pgfsys@useobject{currentmarker}{}%
\end{pgfscope}%
\begin{pgfscope}%
\pgfsys@transformshift{4.432659in}{2.528961in}%
\pgfsys@useobject{currentmarker}{}%
\end{pgfscope}%
\begin{pgfscope}%
\pgfsys@transformshift{5.798651in}{1.786554in}%
\pgfsys@useobject{currentmarker}{}%
\end{pgfscope}%
\begin{pgfscope}%
\pgfsys@transformshift{6.831985in}{1.237377in}%
\pgfsys@useobject{currentmarker}{}%
\end{pgfscope}%
\end{pgfscope}%
\begin{pgfscope}%
\pgfpathrectangle{\pgfqpoint{1.000000in}{0.600000in}}{\pgfqpoint{6.200000in}{4.800000in}}%
\pgfusepath{clip}%
\pgfsetbuttcap%
\pgfsetroundjoin%
\pgfsetlinewidth{1.003750pt}%
\definecolor{currentstroke}{rgb}{0.000000,0.500000,0.000000}%
\pgfsetstrokecolor{currentstroke}%
\pgfsetdash{{6.000000pt}{6.000000pt}}{0.000000pt}%
\pgfpathmoveto{\pgfqpoint{1.000000in}{5.034711in}}%
\pgfpathlineto{\pgfqpoint{2.365992in}{3.958397in}}%
\pgfpathlineto{\pgfqpoint{3.399326in}{1.710568in}}%
\pgfpathlineto{\pgfqpoint{4.432659in}{0.703932in}}%
\pgfpathlineto{\pgfqpoint{5.798651in}{1.047770in}}%
\pgfpathlineto{\pgfqpoint{6.831985in}{1.056759in}}%
\pgfusepath{stroke}%
\end{pgfscope}%
\begin{pgfscope}%
\pgfpathrectangle{\pgfqpoint{1.000000in}{0.600000in}}{\pgfqpoint{6.200000in}{4.800000in}}%
\pgfusepath{clip}%
\pgfsetbuttcap%
\pgfsetroundjoin%
\definecolor{currentfill}{rgb}{0.000000,0.500000,0.000000}%
\pgfsetfillcolor{currentfill}%
\pgfsetlinewidth{0.501875pt}%
\definecolor{currentstroke}{rgb}{0.000000,0.500000,0.000000}%
\pgfsetstrokecolor{currentstroke}%
\pgfsetdash{}{0pt}%
\pgfsys@defobject{currentmarker}{\pgfqpoint{-0.020833in}{-0.020833in}}{\pgfqpoint{0.020833in}{0.020833in}}{%
\pgfpathmoveto{\pgfqpoint{0.000000in}{-0.020833in}}%
\pgfpathcurveto{\pgfqpoint{0.005525in}{-0.020833in}}{\pgfqpoint{0.010825in}{-0.018638in}}{\pgfqpoint{0.014731in}{-0.014731in}}%
\pgfpathcurveto{\pgfqpoint{0.018638in}{-0.010825in}}{\pgfqpoint{0.020833in}{-0.005525in}}{\pgfqpoint{0.020833in}{0.000000in}}%
\pgfpathcurveto{\pgfqpoint{0.020833in}{0.005525in}}{\pgfqpoint{0.018638in}{0.010825in}}{\pgfqpoint{0.014731in}{0.014731in}}%
\pgfpathcurveto{\pgfqpoint{0.010825in}{0.018638in}}{\pgfqpoint{0.005525in}{0.020833in}}{\pgfqpoint{0.000000in}{0.020833in}}%
\pgfpathcurveto{\pgfqpoint{-0.005525in}{0.020833in}}{\pgfqpoint{-0.010825in}{0.018638in}}{\pgfqpoint{-0.014731in}{0.014731in}}%
\pgfpathcurveto{\pgfqpoint{-0.018638in}{0.010825in}}{\pgfqpoint{-0.020833in}{0.005525in}}{\pgfqpoint{-0.020833in}{0.000000in}}%
\pgfpathcurveto{\pgfqpoint{-0.020833in}{-0.005525in}}{\pgfqpoint{-0.018638in}{-0.010825in}}{\pgfqpoint{-0.014731in}{-0.014731in}}%
\pgfpathcurveto{\pgfqpoint{-0.010825in}{-0.018638in}}{\pgfqpoint{-0.005525in}{-0.020833in}}{\pgfqpoint{0.000000in}{-0.020833in}}%
\pgfpathlineto{\pgfqpoint{0.000000in}{-0.020833in}}%
\pgfpathclose%
\pgfusepath{stroke,fill}%
}%
\begin{pgfscope}%
\pgfsys@transformshift{1.000000in}{5.034711in}%
\pgfsys@useobject{currentmarker}{}%
\end{pgfscope}%
\begin{pgfscope}%
\pgfsys@transformshift{2.365992in}{3.958397in}%
\pgfsys@useobject{currentmarker}{}%
\end{pgfscope}%
\begin{pgfscope}%
\pgfsys@transformshift{3.399326in}{1.710568in}%
\pgfsys@useobject{currentmarker}{}%
\end{pgfscope}%
\begin{pgfscope}%
\pgfsys@transformshift{4.432659in}{0.703932in}%
\pgfsys@useobject{currentmarker}{}%
\end{pgfscope}%
\begin{pgfscope}%
\pgfsys@transformshift{5.798651in}{1.047770in}%
\pgfsys@useobject{currentmarker}{}%
\end{pgfscope}%
\begin{pgfscope}%
\pgfsys@transformshift{6.831985in}{1.056759in}%
\pgfsys@useobject{currentmarker}{}%
\end{pgfscope}%
\end{pgfscope}%
\begin{pgfscope}%
\pgfsetrectcap%
\pgfsetmiterjoin%
\pgfsetlinewidth{1.003750pt}%
\definecolor{currentstroke}{rgb}{0.000000,0.000000,0.000000}%
\pgfsetstrokecolor{currentstroke}%
\pgfsetdash{}{0pt}%
\pgfpathmoveto{\pgfqpoint{1.000000in}{0.600000in}}%
\pgfpathlineto{\pgfqpoint{1.000000in}{5.400000in}}%
\pgfusepath{stroke}%
\end{pgfscope}%
\begin{pgfscope}%
\pgfsetrectcap%
\pgfsetmiterjoin%
\pgfsetlinewidth{1.003750pt}%
\definecolor{currentstroke}{rgb}{0.000000,0.000000,0.000000}%
\pgfsetstrokecolor{currentstroke}%
\pgfsetdash{}{0pt}%
\pgfpathmoveto{\pgfqpoint{7.200000in}{0.600000in}}%
\pgfpathlineto{\pgfqpoint{7.200000in}{5.400000in}}%
\pgfusepath{stroke}%
\end{pgfscope}%
\begin{pgfscope}%
\pgfsetrectcap%
\pgfsetmiterjoin%
\pgfsetlinewidth{1.003750pt}%
\definecolor{currentstroke}{rgb}{0.000000,0.000000,0.000000}%
\pgfsetstrokecolor{currentstroke}%
\pgfsetdash{}{0pt}%
\pgfpathmoveto{\pgfqpoint{1.000000in}{0.600000in}}%
\pgfpathlineto{\pgfqpoint{7.200000in}{0.600000in}}%
\pgfusepath{stroke}%
\end{pgfscope}%
\begin{pgfscope}%
\pgfsetrectcap%
\pgfsetmiterjoin%
\pgfsetlinewidth{1.003750pt}%
\definecolor{currentstroke}{rgb}{0.000000,0.000000,0.000000}%
\pgfsetstrokecolor{currentstroke}%
\pgfsetdash{}{0pt}%
\pgfpathmoveto{\pgfqpoint{1.000000in}{5.400000in}}%
\pgfpathlineto{\pgfqpoint{7.200000in}{5.400000in}}%
\pgfusepath{stroke}%
\end{pgfscope}%
\begin{pgfscope}%
\pgfpathrectangle{\pgfqpoint{1.000000in}{0.600000in}}{\pgfqpoint{6.200000in}{4.800000in}}%
\pgfusepath{clip}%
\pgfsetbuttcap%
\pgfsetroundjoin%
\pgfsetlinewidth{0.501875pt}%
\definecolor{currentstroke}{rgb}{0.000000,0.000000,0.000000}%
\pgfsetstrokecolor{currentstroke}%
\pgfsetdash{{1.000000pt}{3.000000pt}}{0.000000pt}%
\pgfpathmoveto{\pgfqpoint{1.000000in}{0.600000in}}%
\pgfpathlineto{\pgfqpoint{1.000000in}{5.400000in}}%
\pgfusepath{stroke}%
\end{pgfscope}%
\begin{pgfscope}%
\pgfsetbuttcap%
\pgfsetroundjoin%
\definecolor{currentfill}{rgb}{0.000000,0.000000,0.000000}%
\pgfsetfillcolor{currentfill}%
\pgfsetlinewidth{0.501875pt}%
\definecolor{currentstroke}{rgb}{0.000000,0.000000,0.000000}%
\pgfsetstrokecolor{currentstroke}%
\pgfsetdash{}{0pt}%
\pgfsys@defobject{currentmarker}{\pgfqpoint{0.000000in}{0.000000in}}{\pgfqpoint{0.000000in}{0.055556in}}{%
\pgfpathmoveto{\pgfqpoint{0.000000in}{0.000000in}}%
\pgfpathlineto{\pgfqpoint{0.000000in}{0.055556in}}%
\pgfusepath{stroke,fill}%
}%
\begin{pgfscope}%
\pgfsys@transformshift{1.000000in}{0.600000in}%
\pgfsys@useobject{currentmarker}{}%
\end{pgfscope}%
\end{pgfscope}%
\begin{pgfscope}%
\pgfsetbuttcap%
\pgfsetroundjoin%
\definecolor{currentfill}{rgb}{0.000000,0.000000,0.000000}%
\pgfsetfillcolor{currentfill}%
\pgfsetlinewidth{0.501875pt}%
\definecolor{currentstroke}{rgb}{0.000000,0.000000,0.000000}%
\pgfsetstrokecolor{currentstroke}%
\pgfsetdash{}{0pt}%
\pgfsys@defobject{currentmarker}{\pgfqpoint{0.000000in}{-0.055556in}}{\pgfqpoint{0.000000in}{0.000000in}}{%
\pgfpathmoveto{\pgfqpoint{0.000000in}{0.000000in}}%
\pgfpathlineto{\pgfqpoint{0.000000in}{-0.055556in}}%
\pgfusepath{stroke,fill}%
}%
\begin{pgfscope}%
\pgfsys@transformshift{1.000000in}{5.400000in}%
\pgfsys@useobject{currentmarker}{}%
\end{pgfscope}%
\end{pgfscope}%
\begin{pgfscope}%
\definecolor{textcolor}{rgb}{0.000000,0.000000,0.000000}%
\pgfsetstrokecolor{textcolor}%
\pgfsetfillcolor{textcolor}%
\pgftext[x=1.000000in,y=0.544444in,,top]{\color{textcolor}\rmfamily\fontsize{10.000000}{12.000000}\selectfont \(\displaystyle {2}\)}%
\end{pgfscope}%
\begin{pgfscope}%
\pgfpathrectangle{\pgfqpoint{1.000000in}{0.600000in}}{\pgfqpoint{6.200000in}{4.800000in}}%
\pgfusepath{clip}%
\pgfsetbuttcap%
\pgfsetroundjoin%
\pgfsetlinewidth{0.501875pt}%
\definecolor{currentstroke}{rgb}{0.000000,0.000000,0.000000}%
\pgfsetstrokecolor{currentstroke}%
\pgfsetdash{{1.000000pt}{3.000000pt}}{0.000000pt}%
\pgfpathmoveto{\pgfqpoint{2.033333in}{0.600000in}}%
\pgfpathlineto{\pgfqpoint{2.033333in}{5.400000in}}%
\pgfusepath{stroke}%
\end{pgfscope}%
\begin{pgfscope}%
\pgfsetbuttcap%
\pgfsetroundjoin%
\definecolor{currentfill}{rgb}{0.000000,0.000000,0.000000}%
\pgfsetfillcolor{currentfill}%
\pgfsetlinewidth{0.501875pt}%
\definecolor{currentstroke}{rgb}{0.000000,0.000000,0.000000}%
\pgfsetstrokecolor{currentstroke}%
\pgfsetdash{}{0pt}%
\pgfsys@defobject{currentmarker}{\pgfqpoint{0.000000in}{0.000000in}}{\pgfqpoint{0.000000in}{0.055556in}}{%
\pgfpathmoveto{\pgfqpoint{0.000000in}{0.000000in}}%
\pgfpathlineto{\pgfqpoint{0.000000in}{0.055556in}}%
\pgfusepath{stroke,fill}%
}%
\begin{pgfscope}%
\pgfsys@transformshift{2.033333in}{0.600000in}%
\pgfsys@useobject{currentmarker}{}%
\end{pgfscope}%
\end{pgfscope}%
\begin{pgfscope}%
\pgfsetbuttcap%
\pgfsetroundjoin%
\definecolor{currentfill}{rgb}{0.000000,0.000000,0.000000}%
\pgfsetfillcolor{currentfill}%
\pgfsetlinewidth{0.501875pt}%
\definecolor{currentstroke}{rgb}{0.000000,0.000000,0.000000}%
\pgfsetstrokecolor{currentstroke}%
\pgfsetdash{}{0pt}%
\pgfsys@defobject{currentmarker}{\pgfqpoint{0.000000in}{-0.055556in}}{\pgfqpoint{0.000000in}{0.000000in}}{%
\pgfpathmoveto{\pgfqpoint{0.000000in}{0.000000in}}%
\pgfpathlineto{\pgfqpoint{0.000000in}{-0.055556in}}%
\pgfusepath{stroke,fill}%
}%
\begin{pgfscope}%
\pgfsys@transformshift{2.033333in}{5.400000in}%
\pgfsys@useobject{currentmarker}{}%
\end{pgfscope}%
\end{pgfscope}%
\begin{pgfscope}%
\definecolor{textcolor}{rgb}{0.000000,0.000000,0.000000}%
\pgfsetstrokecolor{textcolor}%
\pgfsetfillcolor{textcolor}%
\pgftext[x=2.033333in,y=0.544444in,,top]{\color{textcolor}\rmfamily\fontsize{10.000000}{12.000000}\selectfont \(\displaystyle {4}\)}%
\end{pgfscope}%
\begin{pgfscope}%
\pgfpathrectangle{\pgfqpoint{1.000000in}{0.600000in}}{\pgfqpoint{6.200000in}{4.800000in}}%
\pgfusepath{clip}%
\pgfsetbuttcap%
\pgfsetroundjoin%
\pgfsetlinewidth{0.501875pt}%
\definecolor{currentstroke}{rgb}{0.000000,0.000000,0.000000}%
\pgfsetstrokecolor{currentstroke}%
\pgfsetdash{{1.000000pt}{3.000000pt}}{0.000000pt}%
\pgfpathmoveto{\pgfqpoint{3.066667in}{0.600000in}}%
\pgfpathlineto{\pgfqpoint{3.066667in}{5.400000in}}%
\pgfusepath{stroke}%
\end{pgfscope}%
\begin{pgfscope}%
\pgfsetbuttcap%
\pgfsetroundjoin%
\definecolor{currentfill}{rgb}{0.000000,0.000000,0.000000}%
\pgfsetfillcolor{currentfill}%
\pgfsetlinewidth{0.501875pt}%
\definecolor{currentstroke}{rgb}{0.000000,0.000000,0.000000}%
\pgfsetstrokecolor{currentstroke}%
\pgfsetdash{}{0pt}%
\pgfsys@defobject{currentmarker}{\pgfqpoint{0.000000in}{0.000000in}}{\pgfqpoint{0.000000in}{0.055556in}}{%
\pgfpathmoveto{\pgfqpoint{0.000000in}{0.000000in}}%
\pgfpathlineto{\pgfqpoint{0.000000in}{0.055556in}}%
\pgfusepath{stroke,fill}%
}%
\begin{pgfscope}%
\pgfsys@transformshift{3.066667in}{0.600000in}%
\pgfsys@useobject{currentmarker}{}%
\end{pgfscope}%
\end{pgfscope}%
\begin{pgfscope}%
\pgfsetbuttcap%
\pgfsetroundjoin%
\definecolor{currentfill}{rgb}{0.000000,0.000000,0.000000}%
\pgfsetfillcolor{currentfill}%
\pgfsetlinewidth{0.501875pt}%
\definecolor{currentstroke}{rgb}{0.000000,0.000000,0.000000}%
\pgfsetstrokecolor{currentstroke}%
\pgfsetdash{}{0pt}%
\pgfsys@defobject{currentmarker}{\pgfqpoint{0.000000in}{-0.055556in}}{\pgfqpoint{0.000000in}{0.000000in}}{%
\pgfpathmoveto{\pgfqpoint{0.000000in}{0.000000in}}%
\pgfpathlineto{\pgfqpoint{0.000000in}{-0.055556in}}%
\pgfusepath{stroke,fill}%
}%
\begin{pgfscope}%
\pgfsys@transformshift{3.066667in}{5.400000in}%
\pgfsys@useobject{currentmarker}{}%
\end{pgfscope}%
\end{pgfscope}%
\begin{pgfscope}%
\definecolor{textcolor}{rgb}{0.000000,0.000000,0.000000}%
\pgfsetstrokecolor{textcolor}%
\pgfsetfillcolor{textcolor}%
\pgftext[x=3.066667in,y=0.544444in,,top]{\color{textcolor}\rmfamily\fontsize{10.000000}{12.000000}\selectfont \(\displaystyle {8}\)}%
\end{pgfscope}%
\begin{pgfscope}%
\pgfpathrectangle{\pgfqpoint{1.000000in}{0.600000in}}{\pgfqpoint{6.200000in}{4.800000in}}%
\pgfusepath{clip}%
\pgfsetbuttcap%
\pgfsetroundjoin%
\pgfsetlinewidth{0.501875pt}%
\definecolor{currentstroke}{rgb}{0.000000,0.000000,0.000000}%
\pgfsetstrokecolor{currentstroke}%
\pgfsetdash{{1.000000pt}{3.000000pt}}{0.000000pt}%
\pgfpathmoveto{\pgfqpoint{4.100000in}{0.600000in}}%
\pgfpathlineto{\pgfqpoint{4.100000in}{5.400000in}}%
\pgfusepath{stroke}%
\end{pgfscope}%
\begin{pgfscope}%
\pgfsetbuttcap%
\pgfsetroundjoin%
\definecolor{currentfill}{rgb}{0.000000,0.000000,0.000000}%
\pgfsetfillcolor{currentfill}%
\pgfsetlinewidth{0.501875pt}%
\definecolor{currentstroke}{rgb}{0.000000,0.000000,0.000000}%
\pgfsetstrokecolor{currentstroke}%
\pgfsetdash{}{0pt}%
\pgfsys@defobject{currentmarker}{\pgfqpoint{0.000000in}{0.000000in}}{\pgfqpoint{0.000000in}{0.055556in}}{%
\pgfpathmoveto{\pgfqpoint{0.000000in}{0.000000in}}%
\pgfpathlineto{\pgfqpoint{0.000000in}{0.055556in}}%
\pgfusepath{stroke,fill}%
}%
\begin{pgfscope}%
\pgfsys@transformshift{4.100000in}{0.600000in}%
\pgfsys@useobject{currentmarker}{}%
\end{pgfscope}%
\end{pgfscope}%
\begin{pgfscope}%
\pgfsetbuttcap%
\pgfsetroundjoin%
\definecolor{currentfill}{rgb}{0.000000,0.000000,0.000000}%
\pgfsetfillcolor{currentfill}%
\pgfsetlinewidth{0.501875pt}%
\definecolor{currentstroke}{rgb}{0.000000,0.000000,0.000000}%
\pgfsetstrokecolor{currentstroke}%
\pgfsetdash{}{0pt}%
\pgfsys@defobject{currentmarker}{\pgfqpoint{0.000000in}{-0.055556in}}{\pgfqpoint{0.000000in}{0.000000in}}{%
\pgfpathmoveto{\pgfqpoint{0.000000in}{0.000000in}}%
\pgfpathlineto{\pgfqpoint{0.000000in}{-0.055556in}}%
\pgfusepath{stroke,fill}%
}%
\begin{pgfscope}%
\pgfsys@transformshift{4.100000in}{5.400000in}%
\pgfsys@useobject{currentmarker}{}%
\end{pgfscope}%
\end{pgfscope}%
\begin{pgfscope}%
\definecolor{textcolor}{rgb}{0.000000,0.000000,0.000000}%
\pgfsetstrokecolor{textcolor}%
\pgfsetfillcolor{textcolor}%
\pgftext[x=4.100000in,y=0.544444in,,top]{\color{textcolor}\rmfamily\fontsize{10.000000}{12.000000}\selectfont \(\displaystyle {16}\)}%
\end{pgfscope}%
\begin{pgfscope}%
\pgfpathrectangle{\pgfqpoint{1.000000in}{0.600000in}}{\pgfqpoint{6.200000in}{4.800000in}}%
\pgfusepath{clip}%
\pgfsetbuttcap%
\pgfsetroundjoin%
\pgfsetlinewidth{0.501875pt}%
\definecolor{currentstroke}{rgb}{0.000000,0.000000,0.000000}%
\pgfsetstrokecolor{currentstroke}%
\pgfsetdash{{1.000000pt}{3.000000pt}}{0.000000pt}%
\pgfpathmoveto{\pgfqpoint{5.133333in}{0.600000in}}%
\pgfpathlineto{\pgfqpoint{5.133333in}{5.400000in}}%
\pgfusepath{stroke}%
\end{pgfscope}%
\begin{pgfscope}%
\pgfsetbuttcap%
\pgfsetroundjoin%
\definecolor{currentfill}{rgb}{0.000000,0.000000,0.000000}%
\pgfsetfillcolor{currentfill}%
\pgfsetlinewidth{0.501875pt}%
\definecolor{currentstroke}{rgb}{0.000000,0.000000,0.000000}%
\pgfsetstrokecolor{currentstroke}%
\pgfsetdash{}{0pt}%
\pgfsys@defobject{currentmarker}{\pgfqpoint{0.000000in}{0.000000in}}{\pgfqpoint{0.000000in}{0.055556in}}{%
\pgfpathmoveto{\pgfqpoint{0.000000in}{0.000000in}}%
\pgfpathlineto{\pgfqpoint{0.000000in}{0.055556in}}%
\pgfusepath{stroke,fill}%
}%
\begin{pgfscope}%
\pgfsys@transformshift{5.133333in}{0.600000in}%
\pgfsys@useobject{currentmarker}{}%
\end{pgfscope}%
\end{pgfscope}%
\begin{pgfscope}%
\pgfsetbuttcap%
\pgfsetroundjoin%
\definecolor{currentfill}{rgb}{0.000000,0.000000,0.000000}%
\pgfsetfillcolor{currentfill}%
\pgfsetlinewidth{0.501875pt}%
\definecolor{currentstroke}{rgb}{0.000000,0.000000,0.000000}%
\pgfsetstrokecolor{currentstroke}%
\pgfsetdash{}{0pt}%
\pgfsys@defobject{currentmarker}{\pgfqpoint{0.000000in}{-0.055556in}}{\pgfqpoint{0.000000in}{0.000000in}}{%
\pgfpathmoveto{\pgfqpoint{0.000000in}{0.000000in}}%
\pgfpathlineto{\pgfqpoint{0.000000in}{-0.055556in}}%
\pgfusepath{stroke,fill}%
}%
\begin{pgfscope}%
\pgfsys@transformshift{5.133333in}{5.400000in}%
\pgfsys@useobject{currentmarker}{}%
\end{pgfscope}%
\end{pgfscope}%
\begin{pgfscope}%
\definecolor{textcolor}{rgb}{0.000000,0.000000,0.000000}%
\pgfsetstrokecolor{textcolor}%
\pgfsetfillcolor{textcolor}%
\pgftext[x=5.133333in,y=0.544444in,,top]{\color{textcolor}\rmfamily\fontsize{10.000000}{12.000000}\selectfont \(\displaystyle {32}\)}%
\end{pgfscope}%
\begin{pgfscope}%
\pgfpathrectangle{\pgfqpoint{1.000000in}{0.600000in}}{\pgfqpoint{6.200000in}{4.800000in}}%
\pgfusepath{clip}%
\pgfsetbuttcap%
\pgfsetroundjoin%
\pgfsetlinewidth{0.501875pt}%
\definecolor{currentstroke}{rgb}{0.000000,0.000000,0.000000}%
\pgfsetstrokecolor{currentstroke}%
\pgfsetdash{{1.000000pt}{3.000000pt}}{0.000000pt}%
\pgfpathmoveto{\pgfqpoint{6.166667in}{0.600000in}}%
\pgfpathlineto{\pgfqpoint{6.166667in}{5.400000in}}%
\pgfusepath{stroke}%
\end{pgfscope}%
\begin{pgfscope}%
\pgfsetbuttcap%
\pgfsetroundjoin%
\definecolor{currentfill}{rgb}{0.000000,0.000000,0.000000}%
\pgfsetfillcolor{currentfill}%
\pgfsetlinewidth{0.501875pt}%
\definecolor{currentstroke}{rgb}{0.000000,0.000000,0.000000}%
\pgfsetstrokecolor{currentstroke}%
\pgfsetdash{}{0pt}%
\pgfsys@defobject{currentmarker}{\pgfqpoint{0.000000in}{0.000000in}}{\pgfqpoint{0.000000in}{0.055556in}}{%
\pgfpathmoveto{\pgfqpoint{0.000000in}{0.000000in}}%
\pgfpathlineto{\pgfqpoint{0.000000in}{0.055556in}}%
\pgfusepath{stroke,fill}%
}%
\begin{pgfscope}%
\pgfsys@transformshift{6.166667in}{0.600000in}%
\pgfsys@useobject{currentmarker}{}%
\end{pgfscope}%
\end{pgfscope}%
\begin{pgfscope}%
\pgfsetbuttcap%
\pgfsetroundjoin%
\definecolor{currentfill}{rgb}{0.000000,0.000000,0.000000}%
\pgfsetfillcolor{currentfill}%
\pgfsetlinewidth{0.501875pt}%
\definecolor{currentstroke}{rgb}{0.000000,0.000000,0.000000}%
\pgfsetstrokecolor{currentstroke}%
\pgfsetdash{}{0pt}%
\pgfsys@defobject{currentmarker}{\pgfqpoint{0.000000in}{-0.055556in}}{\pgfqpoint{0.000000in}{0.000000in}}{%
\pgfpathmoveto{\pgfqpoint{0.000000in}{0.000000in}}%
\pgfpathlineto{\pgfqpoint{0.000000in}{-0.055556in}}%
\pgfusepath{stroke,fill}%
}%
\begin{pgfscope}%
\pgfsys@transformshift{6.166667in}{5.400000in}%
\pgfsys@useobject{currentmarker}{}%
\end{pgfscope}%
\end{pgfscope}%
\begin{pgfscope}%
\definecolor{textcolor}{rgb}{0.000000,0.000000,0.000000}%
\pgfsetstrokecolor{textcolor}%
\pgfsetfillcolor{textcolor}%
\pgftext[x=6.166667in,y=0.544444in,,top]{\color{textcolor}\rmfamily\fontsize{10.000000}{12.000000}\selectfont \(\displaystyle {64}\)}%
\end{pgfscope}%
\begin{pgfscope}%
\pgfpathrectangle{\pgfqpoint{1.000000in}{0.600000in}}{\pgfqpoint{6.200000in}{4.800000in}}%
\pgfusepath{clip}%
\pgfsetbuttcap%
\pgfsetroundjoin%
\pgfsetlinewidth{0.501875pt}%
\definecolor{currentstroke}{rgb}{0.000000,0.000000,0.000000}%
\pgfsetstrokecolor{currentstroke}%
\pgfsetdash{{1.000000pt}{3.000000pt}}{0.000000pt}%
\pgfpathmoveto{\pgfqpoint{7.200000in}{0.600000in}}%
\pgfpathlineto{\pgfqpoint{7.200000in}{5.400000in}}%
\pgfusepath{stroke}%
\end{pgfscope}%
\begin{pgfscope}%
\pgfsetbuttcap%
\pgfsetroundjoin%
\definecolor{currentfill}{rgb}{0.000000,0.000000,0.000000}%
\pgfsetfillcolor{currentfill}%
\pgfsetlinewidth{0.501875pt}%
\definecolor{currentstroke}{rgb}{0.000000,0.000000,0.000000}%
\pgfsetstrokecolor{currentstroke}%
\pgfsetdash{}{0pt}%
\pgfsys@defobject{currentmarker}{\pgfqpoint{0.000000in}{0.000000in}}{\pgfqpoint{0.000000in}{0.055556in}}{%
\pgfpathmoveto{\pgfqpoint{0.000000in}{0.000000in}}%
\pgfpathlineto{\pgfqpoint{0.000000in}{0.055556in}}%
\pgfusepath{stroke,fill}%
}%
\begin{pgfscope}%
\pgfsys@transformshift{7.200000in}{0.600000in}%
\pgfsys@useobject{currentmarker}{}%
\end{pgfscope}%
\end{pgfscope}%
\begin{pgfscope}%
\pgfsetbuttcap%
\pgfsetroundjoin%
\definecolor{currentfill}{rgb}{0.000000,0.000000,0.000000}%
\pgfsetfillcolor{currentfill}%
\pgfsetlinewidth{0.501875pt}%
\definecolor{currentstroke}{rgb}{0.000000,0.000000,0.000000}%
\pgfsetstrokecolor{currentstroke}%
\pgfsetdash{}{0pt}%
\pgfsys@defobject{currentmarker}{\pgfqpoint{0.000000in}{-0.055556in}}{\pgfqpoint{0.000000in}{0.000000in}}{%
\pgfpathmoveto{\pgfqpoint{0.000000in}{0.000000in}}%
\pgfpathlineto{\pgfqpoint{0.000000in}{-0.055556in}}%
\pgfusepath{stroke,fill}%
}%
\begin{pgfscope}%
\pgfsys@transformshift{7.200000in}{5.400000in}%
\pgfsys@useobject{currentmarker}{}%
\end{pgfscope}%
\end{pgfscope}%
\begin{pgfscope}%
\definecolor{textcolor}{rgb}{0.000000,0.000000,0.000000}%
\pgfsetstrokecolor{textcolor}%
\pgfsetfillcolor{textcolor}%
\pgftext[x=7.200000in,y=0.544444in,,top]{\color{textcolor}\rmfamily\fontsize{10.000000}{12.000000}\selectfont \(\displaystyle {128}\)}%
\end{pgfscope}%
\begin{pgfscope}%
\definecolor{textcolor}{rgb}{0.000000,0.000000,0.000000}%
\pgfsetstrokecolor{textcolor}%
\pgfsetfillcolor{textcolor}%
\pgftext[x=4.100000in,y=0.351543in,,top]{\color{textcolor}\rmfamily\fontsize{12.000000}{14.400000}\selectfont \(\displaystyle grid\ points\)}%
\end{pgfscope}%
\begin{pgfscope}%
\pgfpathrectangle{\pgfqpoint{1.000000in}{0.600000in}}{\pgfqpoint{6.200000in}{4.800000in}}%
\pgfusepath{clip}%
\pgfsetbuttcap%
\pgfsetroundjoin%
\pgfsetlinewidth{0.501875pt}%
\definecolor{currentstroke}{rgb}{0.000000,0.000000,0.000000}%
\pgfsetstrokecolor{currentstroke}%
\pgfsetdash{{1.000000pt}{3.000000pt}}{0.000000pt}%
\pgfpathmoveto{\pgfqpoint{1.000000in}{0.600000in}}%
\pgfpathlineto{\pgfqpoint{7.200000in}{0.600000in}}%
\pgfusepath{stroke}%
\end{pgfscope}%
\begin{pgfscope}%
\pgfsetbuttcap%
\pgfsetroundjoin%
\definecolor{currentfill}{rgb}{0.000000,0.000000,0.000000}%
\pgfsetfillcolor{currentfill}%
\pgfsetlinewidth{0.501875pt}%
\definecolor{currentstroke}{rgb}{0.000000,0.000000,0.000000}%
\pgfsetstrokecolor{currentstroke}%
\pgfsetdash{}{0pt}%
\pgfsys@defobject{currentmarker}{\pgfqpoint{0.000000in}{0.000000in}}{\pgfqpoint{0.055556in}{0.000000in}}{%
\pgfpathmoveto{\pgfqpoint{0.000000in}{0.000000in}}%
\pgfpathlineto{\pgfqpoint{0.055556in}{0.000000in}}%
\pgfusepath{stroke,fill}%
}%
\begin{pgfscope}%
\pgfsys@transformshift{1.000000in}{0.600000in}%
\pgfsys@useobject{currentmarker}{}%
\end{pgfscope}%
\end{pgfscope}%
\begin{pgfscope}%
\pgfsetbuttcap%
\pgfsetroundjoin%
\definecolor{currentfill}{rgb}{0.000000,0.000000,0.000000}%
\pgfsetfillcolor{currentfill}%
\pgfsetlinewidth{0.501875pt}%
\definecolor{currentstroke}{rgb}{0.000000,0.000000,0.000000}%
\pgfsetstrokecolor{currentstroke}%
\pgfsetdash{}{0pt}%
\pgfsys@defobject{currentmarker}{\pgfqpoint{-0.055556in}{0.000000in}}{\pgfqpoint{-0.000000in}{0.000000in}}{%
\pgfpathmoveto{\pgfqpoint{-0.000000in}{0.000000in}}%
\pgfpathlineto{\pgfqpoint{-0.055556in}{0.000000in}}%
\pgfusepath{stroke,fill}%
}%
\begin{pgfscope}%
\pgfsys@transformshift{7.200000in}{0.600000in}%
\pgfsys@useobject{currentmarker}{}%
\end{pgfscope}%
\end{pgfscope}%
\begin{pgfscope}%
\definecolor{textcolor}{rgb}{0.000000,0.000000,0.000000}%
\pgfsetstrokecolor{textcolor}%
\pgfsetfillcolor{textcolor}%
\pgftext[x=0.944444in,y=0.600000in,right,]{\color{textcolor}\rmfamily\fontsize{10.000000}{12.000000}\selectfont \(\displaystyle {10^{-16}}\)}%
\end{pgfscope}%
\begin{pgfscope}%
\pgfpathrectangle{\pgfqpoint{1.000000in}{0.600000in}}{\pgfqpoint{6.200000in}{4.800000in}}%
\pgfusepath{clip}%
\pgfsetbuttcap%
\pgfsetroundjoin%
\pgfsetlinewidth{0.501875pt}%
\definecolor{currentstroke}{rgb}{0.000000,0.000000,0.000000}%
\pgfsetstrokecolor{currentstroke}%
\pgfsetdash{{1.000000pt}{3.000000pt}}{0.000000pt}%
\pgfpathmoveto{\pgfqpoint{1.000000in}{1.200000in}}%
\pgfpathlineto{\pgfqpoint{7.200000in}{1.200000in}}%
\pgfusepath{stroke}%
\end{pgfscope}%
\begin{pgfscope}%
\pgfsetbuttcap%
\pgfsetroundjoin%
\definecolor{currentfill}{rgb}{0.000000,0.000000,0.000000}%
\pgfsetfillcolor{currentfill}%
\pgfsetlinewidth{0.501875pt}%
\definecolor{currentstroke}{rgb}{0.000000,0.000000,0.000000}%
\pgfsetstrokecolor{currentstroke}%
\pgfsetdash{}{0pt}%
\pgfsys@defobject{currentmarker}{\pgfqpoint{0.000000in}{0.000000in}}{\pgfqpoint{0.055556in}{0.000000in}}{%
\pgfpathmoveto{\pgfqpoint{0.000000in}{0.000000in}}%
\pgfpathlineto{\pgfqpoint{0.055556in}{0.000000in}}%
\pgfusepath{stroke,fill}%
}%
\begin{pgfscope}%
\pgfsys@transformshift{1.000000in}{1.200000in}%
\pgfsys@useobject{currentmarker}{}%
\end{pgfscope}%
\end{pgfscope}%
\begin{pgfscope}%
\pgfsetbuttcap%
\pgfsetroundjoin%
\definecolor{currentfill}{rgb}{0.000000,0.000000,0.000000}%
\pgfsetfillcolor{currentfill}%
\pgfsetlinewidth{0.501875pt}%
\definecolor{currentstroke}{rgb}{0.000000,0.000000,0.000000}%
\pgfsetstrokecolor{currentstroke}%
\pgfsetdash{}{0pt}%
\pgfsys@defobject{currentmarker}{\pgfqpoint{-0.055556in}{0.000000in}}{\pgfqpoint{-0.000000in}{0.000000in}}{%
\pgfpathmoveto{\pgfqpoint{-0.000000in}{0.000000in}}%
\pgfpathlineto{\pgfqpoint{-0.055556in}{0.000000in}}%
\pgfusepath{stroke,fill}%
}%
\begin{pgfscope}%
\pgfsys@transformshift{7.200000in}{1.200000in}%
\pgfsys@useobject{currentmarker}{}%
\end{pgfscope}%
\end{pgfscope}%
\begin{pgfscope}%
\definecolor{textcolor}{rgb}{0.000000,0.000000,0.000000}%
\pgfsetstrokecolor{textcolor}%
\pgfsetfillcolor{textcolor}%
\pgftext[x=0.944444in,y=1.200000in,right,]{\color{textcolor}\rmfamily\fontsize{10.000000}{12.000000}\selectfont \(\displaystyle {10^{-14}}\)}%
\end{pgfscope}%
\begin{pgfscope}%
\pgfpathrectangle{\pgfqpoint{1.000000in}{0.600000in}}{\pgfqpoint{6.200000in}{4.800000in}}%
\pgfusepath{clip}%
\pgfsetbuttcap%
\pgfsetroundjoin%
\pgfsetlinewidth{0.501875pt}%
\definecolor{currentstroke}{rgb}{0.000000,0.000000,0.000000}%
\pgfsetstrokecolor{currentstroke}%
\pgfsetdash{{1.000000pt}{3.000000pt}}{0.000000pt}%
\pgfpathmoveto{\pgfqpoint{1.000000in}{1.800000in}}%
\pgfpathlineto{\pgfqpoint{7.200000in}{1.800000in}}%
\pgfusepath{stroke}%
\end{pgfscope}%
\begin{pgfscope}%
\pgfsetbuttcap%
\pgfsetroundjoin%
\definecolor{currentfill}{rgb}{0.000000,0.000000,0.000000}%
\pgfsetfillcolor{currentfill}%
\pgfsetlinewidth{0.501875pt}%
\definecolor{currentstroke}{rgb}{0.000000,0.000000,0.000000}%
\pgfsetstrokecolor{currentstroke}%
\pgfsetdash{}{0pt}%
\pgfsys@defobject{currentmarker}{\pgfqpoint{0.000000in}{0.000000in}}{\pgfqpoint{0.055556in}{0.000000in}}{%
\pgfpathmoveto{\pgfqpoint{0.000000in}{0.000000in}}%
\pgfpathlineto{\pgfqpoint{0.055556in}{0.000000in}}%
\pgfusepath{stroke,fill}%
}%
\begin{pgfscope}%
\pgfsys@transformshift{1.000000in}{1.800000in}%
\pgfsys@useobject{currentmarker}{}%
\end{pgfscope}%
\end{pgfscope}%
\begin{pgfscope}%
\pgfsetbuttcap%
\pgfsetroundjoin%
\definecolor{currentfill}{rgb}{0.000000,0.000000,0.000000}%
\pgfsetfillcolor{currentfill}%
\pgfsetlinewidth{0.501875pt}%
\definecolor{currentstroke}{rgb}{0.000000,0.000000,0.000000}%
\pgfsetstrokecolor{currentstroke}%
\pgfsetdash{}{0pt}%
\pgfsys@defobject{currentmarker}{\pgfqpoint{-0.055556in}{0.000000in}}{\pgfqpoint{-0.000000in}{0.000000in}}{%
\pgfpathmoveto{\pgfqpoint{-0.000000in}{0.000000in}}%
\pgfpathlineto{\pgfqpoint{-0.055556in}{0.000000in}}%
\pgfusepath{stroke,fill}%
}%
\begin{pgfscope}%
\pgfsys@transformshift{7.200000in}{1.800000in}%
\pgfsys@useobject{currentmarker}{}%
\end{pgfscope}%
\end{pgfscope}%
\begin{pgfscope}%
\definecolor{textcolor}{rgb}{0.000000,0.000000,0.000000}%
\pgfsetstrokecolor{textcolor}%
\pgfsetfillcolor{textcolor}%
\pgftext[x=0.944444in,y=1.800000in,right,]{\color{textcolor}\rmfamily\fontsize{10.000000}{12.000000}\selectfont \(\displaystyle {10^{-12}}\)}%
\end{pgfscope}%
\begin{pgfscope}%
\pgfpathrectangle{\pgfqpoint{1.000000in}{0.600000in}}{\pgfqpoint{6.200000in}{4.800000in}}%
\pgfusepath{clip}%
\pgfsetbuttcap%
\pgfsetroundjoin%
\pgfsetlinewidth{0.501875pt}%
\definecolor{currentstroke}{rgb}{0.000000,0.000000,0.000000}%
\pgfsetstrokecolor{currentstroke}%
\pgfsetdash{{1.000000pt}{3.000000pt}}{0.000000pt}%
\pgfpathmoveto{\pgfqpoint{1.000000in}{2.400000in}}%
\pgfpathlineto{\pgfqpoint{7.200000in}{2.400000in}}%
\pgfusepath{stroke}%
\end{pgfscope}%
\begin{pgfscope}%
\pgfsetbuttcap%
\pgfsetroundjoin%
\definecolor{currentfill}{rgb}{0.000000,0.000000,0.000000}%
\pgfsetfillcolor{currentfill}%
\pgfsetlinewidth{0.501875pt}%
\definecolor{currentstroke}{rgb}{0.000000,0.000000,0.000000}%
\pgfsetstrokecolor{currentstroke}%
\pgfsetdash{}{0pt}%
\pgfsys@defobject{currentmarker}{\pgfqpoint{0.000000in}{0.000000in}}{\pgfqpoint{0.055556in}{0.000000in}}{%
\pgfpathmoveto{\pgfqpoint{0.000000in}{0.000000in}}%
\pgfpathlineto{\pgfqpoint{0.055556in}{0.000000in}}%
\pgfusepath{stroke,fill}%
}%
\begin{pgfscope}%
\pgfsys@transformshift{1.000000in}{2.400000in}%
\pgfsys@useobject{currentmarker}{}%
\end{pgfscope}%
\end{pgfscope}%
\begin{pgfscope}%
\pgfsetbuttcap%
\pgfsetroundjoin%
\definecolor{currentfill}{rgb}{0.000000,0.000000,0.000000}%
\pgfsetfillcolor{currentfill}%
\pgfsetlinewidth{0.501875pt}%
\definecolor{currentstroke}{rgb}{0.000000,0.000000,0.000000}%
\pgfsetstrokecolor{currentstroke}%
\pgfsetdash{}{0pt}%
\pgfsys@defobject{currentmarker}{\pgfqpoint{-0.055556in}{0.000000in}}{\pgfqpoint{-0.000000in}{0.000000in}}{%
\pgfpathmoveto{\pgfqpoint{-0.000000in}{0.000000in}}%
\pgfpathlineto{\pgfqpoint{-0.055556in}{0.000000in}}%
\pgfusepath{stroke,fill}%
}%
\begin{pgfscope}%
\pgfsys@transformshift{7.200000in}{2.400000in}%
\pgfsys@useobject{currentmarker}{}%
\end{pgfscope}%
\end{pgfscope}%
\begin{pgfscope}%
\definecolor{textcolor}{rgb}{0.000000,0.000000,0.000000}%
\pgfsetstrokecolor{textcolor}%
\pgfsetfillcolor{textcolor}%
\pgftext[x=0.944444in,y=2.400000in,right,]{\color{textcolor}\rmfamily\fontsize{10.000000}{12.000000}\selectfont \(\displaystyle {10^{-10}}\)}%
\end{pgfscope}%
\begin{pgfscope}%
\pgfpathrectangle{\pgfqpoint{1.000000in}{0.600000in}}{\pgfqpoint{6.200000in}{4.800000in}}%
\pgfusepath{clip}%
\pgfsetbuttcap%
\pgfsetroundjoin%
\pgfsetlinewidth{0.501875pt}%
\definecolor{currentstroke}{rgb}{0.000000,0.000000,0.000000}%
\pgfsetstrokecolor{currentstroke}%
\pgfsetdash{{1.000000pt}{3.000000pt}}{0.000000pt}%
\pgfpathmoveto{\pgfqpoint{1.000000in}{3.000000in}}%
\pgfpathlineto{\pgfqpoint{7.200000in}{3.000000in}}%
\pgfusepath{stroke}%
\end{pgfscope}%
\begin{pgfscope}%
\pgfsetbuttcap%
\pgfsetroundjoin%
\definecolor{currentfill}{rgb}{0.000000,0.000000,0.000000}%
\pgfsetfillcolor{currentfill}%
\pgfsetlinewidth{0.501875pt}%
\definecolor{currentstroke}{rgb}{0.000000,0.000000,0.000000}%
\pgfsetstrokecolor{currentstroke}%
\pgfsetdash{}{0pt}%
\pgfsys@defobject{currentmarker}{\pgfqpoint{0.000000in}{0.000000in}}{\pgfqpoint{0.055556in}{0.000000in}}{%
\pgfpathmoveto{\pgfqpoint{0.000000in}{0.000000in}}%
\pgfpathlineto{\pgfqpoint{0.055556in}{0.000000in}}%
\pgfusepath{stroke,fill}%
}%
\begin{pgfscope}%
\pgfsys@transformshift{1.000000in}{3.000000in}%
\pgfsys@useobject{currentmarker}{}%
\end{pgfscope}%
\end{pgfscope}%
\begin{pgfscope}%
\pgfsetbuttcap%
\pgfsetroundjoin%
\definecolor{currentfill}{rgb}{0.000000,0.000000,0.000000}%
\pgfsetfillcolor{currentfill}%
\pgfsetlinewidth{0.501875pt}%
\definecolor{currentstroke}{rgb}{0.000000,0.000000,0.000000}%
\pgfsetstrokecolor{currentstroke}%
\pgfsetdash{}{0pt}%
\pgfsys@defobject{currentmarker}{\pgfqpoint{-0.055556in}{0.000000in}}{\pgfqpoint{-0.000000in}{0.000000in}}{%
\pgfpathmoveto{\pgfqpoint{-0.000000in}{0.000000in}}%
\pgfpathlineto{\pgfqpoint{-0.055556in}{0.000000in}}%
\pgfusepath{stroke,fill}%
}%
\begin{pgfscope}%
\pgfsys@transformshift{7.200000in}{3.000000in}%
\pgfsys@useobject{currentmarker}{}%
\end{pgfscope}%
\end{pgfscope}%
\begin{pgfscope}%
\definecolor{textcolor}{rgb}{0.000000,0.000000,0.000000}%
\pgfsetstrokecolor{textcolor}%
\pgfsetfillcolor{textcolor}%
\pgftext[x=0.944444in,y=3.000000in,right,]{\color{textcolor}\rmfamily\fontsize{10.000000}{12.000000}\selectfont \(\displaystyle {10^{-8}}\)}%
\end{pgfscope}%
\begin{pgfscope}%
\pgfpathrectangle{\pgfqpoint{1.000000in}{0.600000in}}{\pgfqpoint{6.200000in}{4.800000in}}%
\pgfusepath{clip}%
\pgfsetbuttcap%
\pgfsetroundjoin%
\pgfsetlinewidth{0.501875pt}%
\definecolor{currentstroke}{rgb}{0.000000,0.000000,0.000000}%
\pgfsetstrokecolor{currentstroke}%
\pgfsetdash{{1.000000pt}{3.000000pt}}{0.000000pt}%
\pgfpathmoveto{\pgfqpoint{1.000000in}{3.600000in}}%
\pgfpathlineto{\pgfqpoint{7.200000in}{3.600000in}}%
\pgfusepath{stroke}%
\end{pgfscope}%
\begin{pgfscope}%
\pgfsetbuttcap%
\pgfsetroundjoin%
\definecolor{currentfill}{rgb}{0.000000,0.000000,0.000000}%
\pgfsetfillcolor{currentfill}%
\pgfsetlinewidth{0.501875pt}%
\definecolor{currentstroke}{rgb}{0.000000,0.000000,0.000000}%
\pgfsetstrokecolor{currentstroke}%
\pgfsetdash{}{0pt}%
\pgfsys@defobject{currentmarker}{\pgfqpoint{0.000000in}{0.000000in}}{\pgfqpoint{0.055556in}{0.000000in}}{%
\pgfpathmoveto{\pgfqpoint{0.000000in}{0.000000in}}%
\pgfpathlineto{\pgfqpoint{0.055556in}{0.000000in}}%
\pgfusepath{stroke,fill}%
}%
\begin{pgfscope}%
\pgfsys@transformshift{1.000000in}{3.600000in}%
\pgfsys@useobject{currentmarker}{}%
\end{pgfscope}%
\end{pgfscope}%
\begin{pgfscope}%
\pgfsetbuttcap%
\pgfsetroundjoin%
\definecolor{currentfill}{rgb}{0.000000,0.000000,0.000000}%
\pgfsetfillcolor{currentfill}%
\pgfsetlinewidth{0.501875pt}%
\definecolor{currentstroke}{rgb}{0.000000,0.000000,0.000000}%
\pgfsetstrokecolor{currentstroke}%
\pgfsetdash{}{0pt}%
\pgfsys@defobject{currentmarker}{\pgfqpoint{-0.055556in}{0.000000in}}{\pgfqpoint{-0.000000in}{0.000000in}}{%
\pgfpathmoveto{\pgfqpoint{-0.000000in}{0.000000in}}%
\pgfpathlineto{\pgfqpoint{-0.055556in}{0.000000in}}%
\pgfusepath{stroke,fill}%
}%
\begin{pgfscope}%
\pgfsys@transformshift{7.200000in}{3.600000in}%
\pgfsys@useobject{currentmarker}{}%
\end{pgfscope}%
\end{pgfscope}%
\begin{pgfscope}%
\definecolor{textcolor}{rgb}{0.000000,0.000000,0.000000}%
\pgfsetstrokecolor{textcolor}%
\pgfsetfillcolor{textcolor}%
\pgftext[x=0.944444in,y=3.600000in,right,]{\color{textcolor}\rmfamily\fontsize{10.000000}{12.000000}\selectfont \(\displaystyle {10^{-6}}\)}%
\end{pgfscope}%
\begin{pgfscope}%
\pgfpathrectangle{\pgfqpoint{1.000000in}{0.600000in}}{\pgfqpoint{6.200000in}{4.800000in}}%
\pgfusepath{clip}%
\pgfsetbuttcap%
\pgfsetroundjoin%
\pgfsetlinewidth{0.501875pt}%
\definecolor{currentstroke}{rgb}{0.000000,0.000000,0.000000}%
\pgfsetstrokecolor{currentstroke}%
\pgfsetdash{{1.000000pt}{3.000000pt}}{0.000000pt}%
\pgfpathmoveto{\pgfqpoint{1.000000in}{4.200000in}}%
\pgfpathlineto{\pgfqpoint{7.200000in}{4.200000in}}%
\pgfusepath{stroke}%
\end{pgfscope}%
\begin{pgfscope}%
\pgfsetbuttcap%
\pgfsetroundjoin%
\definecolor{currentfill}{rgb}{0.000000,0.000000,0.000000}%
\pgfsetfillcolor{currentfill}%
\pgfsetlinewidth{0.501875pt}%
\definecolor{currentstroke}{rgb}{0.000000,0.000000,0.000000}%
\pgfsetstrokecolor{currentstroke}%
\pgfsetdash{}{0pt}%
\pgfsys@defobject{currentmarker}{\pgfqpoint{0.000000in}{0.000000in}}{\pgfqpoint{0.055556in}{0.000000in}}{%
\pgfpathmoveto{\pgfqpoint{0.000000in}{0.000000in}}%
\pgfpathlineto{\pgfqpoint{0.055556in}{0.000000in}}%
\pgfusepath{stroke,fill}%
}%
\begin{pgfscope}%
\pgfsys@transformshift{1.000000in}{4.200000in}%
\pgfsys@useobject{currentmarker}{}%
\end{pgfscope}%
\end{pgfscope}%
\begin{pgfscope}%
\pgfsetbuttcap%
\pgfsetroundjoin%
\definecolor{currentfill}{rgb}{0.000000,0.000000,0.000000}%
\pgfsetfillcolor{currentfill}%
\pgfsetlinewidth{0.501875pt}%
\definecolor{currentstroke}{rgb}{0.000000,0.000000,0.000000}%
\pgfsetstrokecolor{currentstroke}%
\pgfsetdash{}{0pt}%
\pgfsys@defobject{currentmarker}{\pgfqpoint{-0.055556in}{0.000000in}}{\pgfqpoint{-0.000000in}{0.000000in}}{%
\pgfpathmoveto{\pgfqpoint{-0.000000in}{0.000000in}}%
\pgfpathlineto{\pgfqpoint{-0.055556in}{0.000000in}}%
\pgfusepath{stroke,fill}%
}%
\begin{pgfscope}%
\pgfsys@transformshift{7.200000in}{4.200000in}%
\pgfsys@useobject{currentmarker}{}%
\end{pgfscope}%
\end{pgfscope}%
\begin{pgfscope}%
\definecolor{textcolor}{rgb}{0.000000,0.000000,0.000000}%
\pgfsetstrokecolor{textcolor}%
\pgfsetfillcolor{textcolor}%
\pgftext[x=0.944444in,y=4.200000in,right,]{\color{textcolor}\rmfamily\fontsize{10.000000}{12.000000}\selectfont \(\displaystyle {10^{-4}}\)}%
\end{pgfscope}%
\begin{pgfscope}%
\pgfpathrectangle{\pgfqpoint{1.000000in}{0.600000in}}{\pgfqpoint{6.200000in}{4.800000in}}%
\pgfusepath{clip}%
\pgfsetbuttcap%
\pgfsetroundjoin%
\pgfsetlinewidth{0.501875pt}%
\definecolor{currentstroke}{rgb}{0.000000,0.000000,0.000000}%
\pgfsetstrokecolor{currentstroke}%
\pgfsetdash{{1.000000pt}{3.000000pt}}{0.000000pt}%
\pgfpathmoveto{\pgfqpoint{1.000000in}{4.800000in}}%
\pgfpathlineto{\pgfqpoint{7.200000in}{4.800000in}}%
\pgfusepath{stroke}%
\end{pgfscope}%
\begin{pgfscope}%
\pgfsetbuttcap%
\pgfsetroundjoin%
\definecolor{currentfill}{rgb}{0.000000,0.000000,0.000000}%
\pgfsetfillcolor{currentfill}%
\pgfsetlinewidth{0.501875pt}%
\definecolor{currentstroke}{rgb}{0.000000,0.000000,0.000000}%
\pgfsetstrokecolor{currentstroke}%
\pgfsetdash{}{0pt}%
\pgfsys@defobject{currentmarker}{\pgfqpoint{0.000000in}{0.000000in}}{\pgfqpoint{0.055556in}{0.000000in}}{%
\pgfpathmoveto{\pgfqpoint{0.000000in}{0.000000in}}%
\pgfpathlineto{\pgfqpoint{0.055556in}{0.000000in}}%
\pgfusepath{stroke,fill}%
}%
\begin{pgfscope}%
\pgfsys@transformshift{1.000000in}{4.800000in}%
\pgfsys@useobject{currentmarker}{}%
\end{pgfscope}%
\end{pgfscope}%
\begin{pgfscope}%
\pgfsetbuttcap%
\pgfsetroundjoin%
\definecolor{currentfill}{rgb}{0.000000,0.000000,0.000000}%
\pgfsetfillcolor{currentfill}%
\pgfsetlinewidth{0.501875pt}%
\definecolor{currentstroke}{rgb}{0.000000,0.000000,0.000000}%
\pgfsetstrokecolor{currentstroke}%
\pgfsetdash{}{0pt}%
\pgfsys@defobject{currentmarker}{\pgfqpoint{-0.055556in}{0.000000in}}{\pgfqpoint{-0.000000in}{0.000000in}}{%
\pgfpathmoveto{\pgfqpoint{-0.000000in}{0.000000in}}%
\pgfpathlineto{\pgfqpoint{-0.055556in}{0.000000in}}%
\pgfusepath{stroke,fill}%
}%
\begin{pgfscope}%
\pgfsys@transformshift{7.200000in}{4.800000in}%
\pgfsys@useobject{currentmarker}{}%
\end{pgfscope}%
\end{pgfscope}%
\begin{pgfscope}%
\definecolor{textcolor}{rgb}{0.000000,0.000000,0.000000}%
\pgfsetstrokecolor{textcolor}%
\pgfsetfillcolor{textcolor}%
\pgftext[x=0.944444in,y=4.800000in,right,]{\color{textcolor}\rmfamily\fontsize{10.000000}{12.000000}\selectfont \(\displaystyle {10^{-2}}\)}%
\end{pgfscope}%
\begin{pgfscope}%
\pgfpathrectangle{\pgfqpoint{1.000000in}{0.600000in}}{\pgfqpoint{6.200000in}{4.800000in}}%
\pgfusepath{clip}%
\pgfsetbuttcap%
\pgfsetroundjoin%
\pgfsetlinewidth{0.501875pt}%
\definecolor{currentstroke}{rgb}{0.000000,0.000000,0.000000}%
\pgfsetstrokecolor{currentstroke}%
\pgfsetdash{{1.000000pt}{3.000000pt}}{0.000000pt}%
\pgfpathmoveto{\pgfqpoint{1.000000in}{5.400000in}}%
\pgfpathlineto{\pgfqpoint{7.200000in}{5.400000in}}%
\pgfusepath{stroke}%
\end{pgfscope}%
\begin{pgfscope}%
\pgfsetbuttcap%
\pgfsetroundjoin%
\definecolor{currentfill}{rgb}{0.000000,0.000000,0.000000}%
\pgfsetfillcolor{currentfill}%
\pgfsetlinewidth{0.501875pt}%
\definecolor{currentstroke}{rgb}{0.000000,0.000000,0.000000}%
\pgfsetstrokecolor{currentstroke}%
\pgfsetdash{}{0pt}%
\pgfsys@defobject{currentmarker}{\pgfqpoint{0.000000in}{0.000000in}}{\pgfqpoint{0.055556in}{0.000000in}}{%
\pgfpathmoveto{\pgfqpoint{0.000000in}{0.000000in}}%
\pgfpathlineto{\pgfqpoint{0.055556in}{0.000000in}}%
\pgfusepath{stroke,fill}%
}%
\begin{pgfscope}%
\pgfsys@transformshift{1.000000in}{5.400000in}%
\pgfsys@useobject{currentmarker}{}%
\end{pgfscope}%
\end{pgfscope}%
\begin{pgfscope}%
\pgfsetbuttcap%
\pgfsetroundjoin%
\definecolor{currentfill}{rgb}{0.000000,0.000000,0.000000}%
\pgfsetfillcolor{currentfill}%
\pgfsetlinewidth{0.501875pt}%
\definecolor{currentstroke}{rgb}{0.000000,0.000000,0.000000}%
\pgfsetstrokecolor{currentstroke}%
\pgfsetdash{}{0pt}%
\pgfsys@defobject{currentmarker}{\pgfqpoint{-0.055556in}{0.000000in}}{\pgfqpoint{-0.000000in}{0.000000in}}{%
\pgfpathmoveto{\pgfqpoint{-0.000000in}{0.000000in}}%
\pgfpathlineto{\pgfqpoint{-0.055556in}{0.000000in}}%
\pgfusepath{stroke,fill}%
}%
\begin{pgfscope}%
\pgfsys@transformshift{7.200000in}{5.400000in}%
\pgfsys@useobject{currentmarker}{}%
\end{pgfscope}%
\end{pgfscope}%
\begin{pgfscope}%
\definecolor{textcolor}{rgb}{0.000000,0.000000,0.000000}%
\pgfsetstrokecolor{textcolor}%
\pgfsetfillcolor{textcolor}%
\pgftext[x=0.944444in,y=5.400000in,right,]{\color{textcolor}\rmfamily\fontsize{10.000000}{12.000000}\selectfont \(\displaystyle {10^{0}}\)}%
\end{pgfscope}%
\begin{pgfscope}%
\definecolor{textcolor}{rgb}{0.000000,0.000000,0.000000}%
\pgfsetstrokecolor{textcolor}%
\pgfsetfillcolor{textcolor}%
\pgftext[x=0.531635in,y=3.000000in,,bottom,rotate=90.000000]{\color{textcolor}\rmfamily\fontsize{12.000000}{14.400000}\selectfont \(\displaystyle quadrature\ error\)}%
\end{pgfscope}%
\begin{pgfscope}%
\definecolor{textcolor}{rgb}{0.000000,0.000000,0.000000}%
\pgfsetstrokecolor{textcolor}%
\pgfsetfillcolor{textcolor}%
\pgftext[x=4.100000in,y=5.469444in,,base]{\color{textcolor}\rmfamily\fontsize{12.000000}{14.400000}\selectfont \(\displaystyle Quadrature\ convergence\)}%
\end{pgfscope}%
\begin{pgfscope}%
\pgfsetbuttcap%
\pgfsetmiterjoin%
\definecolor{currentfill}{rgb}{1.000000,1.000000,1.000000}%
\pgfsetfillcolor{currentfill}%
\pgfsetlinewidth{1.003750pt}%
\definecolor{currentstroke}{rgb}{0.000000,0.000000,0.000000}%
\pgfsetstrokecolor{currentstroke}%
\pgfsetdash{}{0pt}%
\pgfpathmoveto{\pgfqpoint{1.083333in}{0.683333in}}%
\pgfpathlineto{\pgfqpoint{4.145656in}{0.683333in}}%
\pgfpathlineto{\pgfqpoint{4.145656in}{1.662962in}}%
\pgfpathlineto{\pgfqpoint{1.083333in}{1.662962in}}%
\pgfpathlineto{\pgfqpoint{1.083333in}{0.683333in}}%
\pgfpathclose%
\pgfusepath{stroke,fill}%
\end{pgfscope}%
\begin{pgfscope}%
\pgfsetbuttcap%
\pgfsetroundjoin%
\pgfsetlinewidth{1.003750pt}%
\definecolor{currentstroke}{rgb}{0.000000,0.000000,1.000000}%
\pgfsetstrokecolor{currentstroke}%
\pgfsetdash{{6.000000pt}{6.000000pt}}{0.000000pt}%
\pgfpathmoveto{\pgfqpoint{1.200000in}{1.537962in}}%
\pgfpathlineto{\pgfqpoint{1.433333in}{1.537962in}}%
\pgfusepath{stroke}%
\end{pgfscope}%
\begin{pgfscope}%
\pgfsetbuttcap%
\pgfsetroundjoin%
\definecolor{currentfill}{rgb}{0.000000,0.000000,1.000000}%
\pgfsetfillcolor{currentfill}%
\pgfsetlinewidth{0.501875pt}%
\definecolor{currentstroke}{rgb}{0.000000,0.000000,1.000000}%
\pgfsetstrokecolor{currentstroke}%
\pgfsetdash{}{0pt}%
\pgfsys@defobject{currentmarker}{\pgfqpoint{-0.020833in}{-0.020833in}}{\pgfqpoint{0.020833in}{0.020833in}}{%
\pgfpathmoveto{\pgfqpoint{0.000000in}{-0.020833in}}%
\pgfpathcurveto{\pgfqpoint{0.005525in}{-0.020833in}}{\pgfqpoint{0.010825in}{-0.018638in}}{\pgfqpoint{0.014731in}{-0.014731in}}%
\pgfpathcurveto{\pgfqpoint{0.018638in}{-0.010825in}}{\pgfqpoint{0.020833in}{-0.005525in}}{\pgfqpoint{0.020833in}{0.000000in}}%
\pgfpathcurveto{\pgfqpoint{0.020833in}{0.005525in}}{\pgfqpoint{0.018638in}{0.010825in}}{\pgfqpoint{0.014731in}{0.014731in}}%
\pgfpathcurveto{\pgfqpoint{0.010825in}{0.018638in}}{\pgfqpoint{0.005525in}{0.020833in}}{\pgfqpoint{0.000000in}{0.020833in}}%
\pgfpathcurveto{\pgfqpoint{-0.005525in}{0.020833in}}{\pgfqpoint{-0.010825in}{0.018638in}}{\pgfqpoint{-0.014731in}{0.014731in}}%
\pgfpathcurveto{\pgfqpoint{-0.018638in}{0.010825in}}{\pgfqpoint{-0.020833in}{0.005525in}}{\pgfqpoint{-0.020833in}{0.000000in}}%
\pgfpathcurveto{\pgfqpoint{-0.020833in}{-0.005525in}}{\pgfqpoint{-0.018638in}{-0.010825in}}{\pgfqpoint{-0.014731in}{-0.014731in}}%
\pgfpathcurveto{\pgfqpoint{-0.010825in}{-0.018638in}}{\pgfqpoint{-0.005525in}{-0.020833in}}{\pgfqpoint{0.000000in}{-0.020833in}}%
\pgfpathlineto{\pgfqpoint{0.000000in}{-0.020833in}}%
\pgfpathclose%
\pgfusepath{stroke,fill}%
}%
\begin{pgfscope}%
\pgfsys@transformshift{1.200000in}{1.537962in}%
\pgfsys@useobject{currentmarker}{}%
\end{pgfscope}%
\begin{pgfscope}%
\pgfsys@transformshift{1.433333in}{1.537962in}%
\pgfsys@useobject{currentmarker}{}%
\end{pgfscope}%
\end{pgfscope}%
\begin{pgfscope}%
\definecolor{textcolor}{rgb}{0.000000,0.000000,0.000000}%
\pgfsetstrokecolor{textcolor}%
\pgfsetfillcolor{textcolor}%
\pgftext[x=1.616667in,y=1.479628in,left,base]{\color{textcolor}\rmfamily\fontsize{12.000000}{14.400000}\selectfont \(\displaystyle trapezoidal\ rule\)}%
\end{pgfscope}%
\begin{pgfscope}%
\pgfsetbuttcap%
\pgfsetroundjoin%
\pgfsetlinewidth{1.003750pt}%
\definecolor{currentstroke}{rgb}{1.000000,0.000000,0.000000}%
\pgfsetstrokecolor{currentstroke}%
\pgfsetdash{{6.000000pt}{6.000000pt}}{0.000000pt}%
\pgfpathmoveto{\pgfqpoint{1.200000in}{1.305555in}}%
\pgfpathlineto{\pgfqpoint{1.433333in}{1.305555in}}%
\pgfusepath{stroke}%
\end{pgfscope}%
\begin{pgfscope}%
\pgfsetbuttcap%
\pgfsetroundjoin%
\definecolor{currentfill}{rgb}{1.000000,0.000000,0.000000}%
\pgfsetfillcolor{currentfill}%
\pgfsetlinewidth{0.501875pt}%
\definecolor{currentstroke}{rgb}{1.000000,0.000000,0.000000}%
\pgfsetstrokecolor{currentstroke}%
\pgfsetdash{}{0pt}%
\pgfsys@defobject{currentmarker}{\pgfqpoint{-0.020833in}{-0.020833in}}{\pgfqpoint{0.020833in}{0.020833in}}{%
\pgfpathmoveto{\pgfqpoint{0.000000in}{-0.020833in}}%
\pgfpathcurveto{\pgfqpoint{0.005525in}{-0.020833in}}{\pgfqpoint{0.010825in}{-0.018638in}}{\pgfqpoint{0.014731in}{-0.014731in}}%
\pgfpathcurveto{\pgfqpoint{0.018638in}{-0.010825in}}{\pgfqpoint{0.020833in}{-0.005525in}}{\pgfqpoint{0.020833in}{0.000000in}}%
\pgfpathcurveto{\pgfqpoint{0.020833in}{0.005525in}}{\pgfqpoint{0.018638in}{0.010825in}}{\pgfqpoint{0.014731in}{0.014731in}}%
\pgfpathcurveto{\pgfqpoint{0.010825in}{0.018638in}}{\pgfqpoint{0.005525in}{0.020833in}}{\pgfqpoint{0.000000in}{0.020833in}}%
\pgfpathcurveto{\pgfqpoint{-0.005525in}{0.020833in}}{\pgfqpoint{-0.010825in}{0.018638in}}{\pgfqpoint{-0.014731in}{0.014731in}}%
\pgfpathcurveto{\pgfqpoint{-0.018638in}{0.010825in}}{\pgfqpoint{-0.020833in}{0.005525in}}{\pgfqpoint{-0.020833in}{0.000000in}}%
\pgfpathcurveto{\pgfqpoint{-0.020833in}{-0.005525in}}{\pgfqpoint{-0.018638in}{-0.010825in}}{\pgfqpoint{-0.014731in}{-0.014731in}}%
\pgfpathcurveto{\pgfqpoint{-0.010825in}{-0.018638in}}{\pgfqpoint{-0.005525in}{-0.020833in}}{\pgfqpoint{0.000000in}{-0.020833in}}%
\pgfpathlineto{\pgfqpoint{0.000000in}{-0.020833in}}%
\pgfpathclose%
\pgfusepath{stroke,fill}%
}%
\begin{pgfscope}%
\pgfsys@transformshift{1.200000in}{1.305555in}%
\pgfsys@useobject{currentmarker}{}%
\end{pgfscope}%
\begin{pgfscope}%
\pgfsys@transformshift{1.433333in}{1.305555in}%
\pgfsys@useobject{currentmarker}{}%
\end{pgfscope}%
\end{pgfscope}%
\begin{pgfscope}%
\definecolor{textcolor}{rgb}{0.000000,0.000000,0.000000}%
\pgfsetstrokecolor{textcolor}%
\pgfsetfillcolor{textcolor}%
\pgftext[x=1.616667in,y=1.247221in,left,base]{\color{textcolor}\rmfamily\fontsize{12.000000}{14.400000}\selectfont \(\displaystyle trapezoidal\ rule\ 1st\ derivative\)}%
\end{pgfscope}%
\begin{pgfscope}%
\pgfsetbuttcap%
\pgfsetroundjoin%
\pgfsetlinewidth{1.003750pt}%
\definecolor{currentstroke}{rgb}{0.750000,0.000000,0.750000}%
\pgfsetstrokecolor{currentstroke}%
\pgfsetdash{{6.000000pt}{6.000000pt}}{0.000000pt}%
\pgfpathmoveto{\pgfqpoint{1.200000in}{1.073147in}}%
\pgfpathlineto{\pgfqpoint{1.433333in}{1.073147in}}%
\pgfusepath{stroke}%
\end{pgfscope}%
\begin{pgfscope}%
\pgfsetbuttcap%
\pgfsetroundjoin%
\definecolor{currentfill}{rgb}{0.750000,0.000000,0.750000}%
\pgfsetfillcolor{currentfill}%
\pgfsetlinewidth{0.501875pt}%
\definecolor{currentstroke}{rgb}{0.750000,0.000000,0.750000}%
\pgfsetstrokecolor{currentstroke}%
\pgfsetdash{}{0pt}%
\pgfsys@defobject{currentmarker}{\pgfqpoint{-0.020833in}{-0.020833in}}{\pgfqpoint{0.020833in}{0.020833in}}{%
\pgfpathmoveto{\pgfqpoint{0.000000in}{-0.020833in}}%
\pgfpathcurveto{\pgfqpoint{0.005525in}{-0.020833in}}{\pgfqpoint{0.010825in}{-0.018638in}}{\pgfqpoint{0.014731in}{-0.014731in}}%
\pgfpathcurveto{\pgfqpoint{0.018638in}{-0.010825in}}{\pgfqpoint{0.020833in}{-0.005525in}}{\pgfqpoint{0.020833in}{0.000000in}}%
\pgfpathcurveto{\pgfqpoint{0.020833in}{0.005525in}}{\pgfqpoint{0.018638in}{0.010825in}}{\pgfqpoint{0.014731in}{0.014731in}}%
\pgfpathcurveto{\pgfqpoint{0.010825in}{0.018638in}}{\pgfqpoint{0.005525in}{0.020833in}}{\pgfqpoint{0.000000in}{0.020833in}}%
\pgfpathcurveto{\pgfqpoint{-0.005525in}{0.020833in}}{\pgfqpoint{-0.010825in}{0.018638in}}{\pgfqpoint{-0.014731in}{0.014731in}}%
\pgfpathcurveto{\pgfqpoint{-0.018638in}{0.010825in}}{\pgfqpoint{-0.020833in}{0.005525in}}{\pgfqpoint{-0.020833in}{0.000000in}}%
\pgfpathcurveto{\pgfqpoint{-0.020833in}{-0.005525in}}{\pgfqpoint{-0.018638in}{-0.010825in}}{\pgfqpoint{-0.014731in}{-0.014731in}}%
\pgfpathcurveto{\pgfqpoint{-0.010825in}{-0.018638in}}{\pgfqpoint{-0.005525in}{-0.020833in}}{\pgfqpoint{0.000000in}{-0.020833in}}%
\pgfpathlineto{\pgfqpoint{0.000000in}{-0.020833in}}%
\pgfpathclose%
\pgfusepath{stroke,fill}%
}%
\begin{pgfscope}%
\pgfsys@transformshift{1.200000in}{1.073147in}%
\pgfsys@useobject{currentmarker}{}%
\end{pgfscope}%
\begin{pgfscope}%
\pgfsys@transformshift{1.433333in}{1.073147in}%
\pgfsys@useobject{currentmarker}{}%
\end{pgfscope}%
\end{pgfscope}%
\begin{pgfscope}%
\definecolor{textcolor}{rgb}{0.000000,0.000000,0.000000}%
\pgfsetstrokecolor{textcolor}%
\pgfsetfillcolor{textcolor}%
\pgftext[x=1.616667in,y=1.014814in,left,base]{\color{textcolor}\rmfamily\fontsize{12.000000}{14.400000}\selectfont \(\displaystyle trapezoidal\ 1st\ \&\ 3rd\ derivative\)}%
\end{pgfscope}%
\begin{pgfscope}%
\pgfsetbuttcap%
\pgfsetroundjoin%
\pgfsetlinewidth{1.003750pt}%
\definecolor{currentstroke}{rgb}{0.000000,0.500000,0.000000}%
\pgfsetstrokecolor{currentstroke}%
\pgfsetdash{{6.000000pt}{6.000000pt}}{0.000000pt}%
\pgfpathmoveto{\pgfqpoint{1.200000in}{0.840740in}}%
\pgfpathlineto{\pgfqpoint{1.433333in}{0.840740in}}%
\pgfusepath{stroke}%
\end{pgfscope}%
\begin{pgfscope}%
\pgfsetbuttcap%
\pgfsetroundjoin%
\definecolor{currentfill}{rgb}{0.000000,0.500000,0.000000}%
\pgfsetfillcolor{currentfill}%
\pgfsetlinewidth{0.501875pt}%
\definecolor{currentstroke}{rgb}{0.000000,0.500000,0.000000}%
\pgfsetstrokecolor{currentstroke}%
\pgfsetdash{}{0pt}%
\pgfsys@defobject{currentmarker}{\pgfqpoint{-0.020833in}{-0.020833in}}{\pgfqpoint{0.020833in}{0.020833in}}{%
\pgfpathmoveto{\pgfqpoint{0.000000in}{-0.020833in}}%
\pgfpathcurveto{\pgfqpoint{0.005525in}{-0.020833in}}{\pgfqpoint{0.010825in}{-0.018638in}}{\pgfqpoint{0.014731in}{-0.014731in}}%
\pgfpathcurveto{\pgfqpoint{0.018638in}{-0.010825in}}{\pgfqpoint{0.020833in}{-0.005525in}}{\pgfqpoint{0.020833in}{0.000000in}}%
\pgfpathcurveto{\pgfqpoint{0.020833in}{0.005525in}}{\pgfqpoint{0.018638in}{0.010825in}}{\pgfqpoint{0.014731in}{0.014731in}}%
\pgfpathcurveto{\pgfqpoint{0.010825in}{0.018638in}}{\pgfqpoint{0.005525in}{0.020833in}}{\pgfqpoint{0.000000in}{0.020833in}}%
\pgfpathcurveto{\pgfqpoint{-0.005525in}{0.020833in}}{\pgfqpoint{-0.010825in}{0.018638in}}{\pgfqpoint{-0.014731in}{0.014731in}}%
\pgfpathcurveto{\pgfqpoint{-0.018638in}{0.010825in}}{\pgfqpoint{-0.020833in}{0.005525in}}{\pgfqpoint{-0.020833in}{0.000000in}}%
\pgfpathcurveto{\pgfqpoint{-0.020833in}{-0.005525in}}{\pgfqpoint{-0.018638in}{-0.010825in}}{\pgfqpoint{-0.014731in}{-0.014731in}}%
\pgfpathcurveto{\pgfqpoint{-0.010825in}{-0.018638in}}{\pgfqpoint{-0.005525in}{-0.020833in}}{\pgfqpoint{0.000000in}{-0.020833in}}%
\pgfpathlineto{\pgfqpoint{0.000000in}{-0.020833in}}%
\pgfpathclose%
\pgfusepath{stroke,fill}%
}%
\begin{pgfscope}%
\pgfsys@transformshift{1.200000in}{0.840740in}%
\pgfsys@useobject{currentmarker}{}%
\end{pgfscope}%
\begin{pgfscope}%
\pgfsys@transformshift{1.433333in}{0.840740in}%
\pgfsys@useobject{currentmarker}{}%
\end{pgfscope}%
\end{pgfscope}%
\begin{pgfscope}%
\definecolor{textcolor}{rgb}{0.000000,0.000000,0.000000}%
\pgfsetstrokecolor{textcolor}%
\pgfsetfillcolor{textcolor}%
\pgftext[x=1.616667in,y=0.782407in,left,base]{\color{textcolor}\rmfamily\fontsize{12.000000}{14.400000}\selectfont \(\displaystyle gauss-legendre\ quadrature\ rule\)}%
\end{pgfscope}%
\end{pgfpicture}%
\makeatother%
\endgroup%
}
    \end{figure}

    \item Evaluate $I = \displaystyle \int_{0}^{1} \frac{e^{-x}}{\sqrt{x}} dx$ by
    subdividing the domain into $N \in \{5, 10, 20, 50, 100, 200, 500, 1000\}$ panels.

    \begin{enumerate}
    \item Using a rectangular rule
    \item Make a change of variables $x = t^{2}$ and use rectangular rule on new variable.
    \end{enumerate}

    \begin{itemize}
    \item Plot the decay of the absolute error using the above two methods.
    \item You may obtain the exact value of the integral up to $20$ digits using wolfram alpha
    \item Compare the two methods above in terms of accuracy and cost.
    \item Explain the difference in solution, if any.
    \item Make sure the figure has a legend and the axes are clearly marked.
    \item Ensure that the font size for title, axes, legend are readable.
    \item Submit the plots obtained, entire code and the write-up.
    \end{itemize}

    \textbf{Program:}
    \lstinputlisting[language=Python]{Scripts/program7.py}

    \begin{figure}[ht!]
    \centering
    \resizebox{0.9\linewidth}{!}{%% Creator: Matplotlib, PGF backend
%%
%% To include the figure in your LaTeX document, write
%%   \input{<filename>.pgf}
%%
%% Make sure the required packages are loaded in your preamble
%%   \usepackage{pgf}
%%
%% Also ensure that all the required font packages are loaded; for instance,
%% the lmodern package is sometimes necessary when using math font.
%%   \usepackage{lmodern}
%%
%% Figures using additional raster images can only be included by \input if
%% they are in the same directory as the main LaTeX file. For loading figures
%% from other directories you can use the `import` package
%%   \usepackage{import}
%%
%% and then include the figures with
%%   \import{<path to file>}{<filename>.pgf}
%%
%% Matplotlib used the following preamble
%%
\begingroup%
\makeatletter%
\begin{pgfpicture}%
\pgfpathrectangle{\pgfpointorigin}{\pgfqpoint{8.000000in}{6.000000in}}%
\pgfusepath{use as bounding box, clip}%
\begin{pgfscope}%
\pgfsetbuttcap%
\pgfsetmiterjoin%
\definecolor{currentfill}{rgb}{1.000000,1.000000,1.000000}%
\pgfsetfillcolor{currentfill}%
\pgfsetlinewidth{0.000000pt}%
\definecolor{currentstroke}{rgb}{1.000000,1.000000,1.000000}%
\pgfsetstrokecolor{currentstroke}%
\pgfsetdash{}{0pt}%
\pgfpathmoveto{\pgfqpoint{0.000000in}{0.000000in}}%
\pgfpathlineto{\pgfqpoint{8.000000in}{0.000000in}}%
\pgfpathlineto{\pgfqpoint{8.000000in}{6.000000in}}%
\pgfpathlineto{\pgfqpoint{0.000000in}{6.000000in}}%
\pgfpathlineto{\pgfqpoint{0.000000in}{0.000000in}}%
\pgfpathclose%
\pgfusepath{fill}%
\end{pgfscope}%
\begin{pgfscope}%
\pgfsetbuttcap%
\pgfsetmiterjoin%
\definecolor{currentfill}{rgb}{1.000000,1.000000,1.000000}%
\pgfsetfillcolor{currentfill}%
\pgfsetlinewidth{0.000000pt}%
\definecolor{currentstroke}{rgb}{0.000000,0.000000,0.000000}%
\pgfsetstrokecolor{currentstroke}%
\pgfsetstrokeopacity{0.000000}%
\pgfsetdash{}{0pt}%
\pgfpathmoveto{\pgfqpoint{1.000000in}{0.600000in}}%
\pgfpathlineto{\pgfqpoint{7.200000in}{0.600000in}}%
\pgfpathlineto{\pgfqpoint{7.200000in}{5.400000in}}%
\pgfpathlineto{\pgfqpoint{1.000000in}{5.400000in}}%
\pgfpathlineto{\pgfqpoint{1.000000in}{0.600000in}}%
\pgfpathclose%
\pgfusepath{fill}%
\end{pgfscope}%
\begin{pgfscope}%
\pgfpathrectangle{\pgfqpoint{1.000000in}{0.600000in}}{\pgfqpoint{6.200000in}{4.800000in}}%
\pgfusepath{clip}%
\pgfsetbuttcap%
\pgfsetroundjoin%
\pgfsetlinewidth{1.003750pt}%
\definecolor{currentstroke}{rgb}{0.750000,0.000000,0.750000}%
\pgfsetstrokecolor{currentstroke}%
\pgfsetdash{{6.000000pt}{6.000000pt}}{0.000000pt}%
\pgfpathmoveto{\pgfqpoint{1.249494in}{5.093442in}}%
\pgfpathlineto{\pgfqpoint{2.024494in}{4.985228in}}%
\pgfpathlineto{\pgfqpoint{2.799494in}{4.886632in}}%
\pgfpathlineto{\pgfqpoint{3.823989in}{4.762455in}}%
\pgfpathlineto{\pgfqpoint{4.598989in}{4.670573in}}%
\pgfpathlineto{\pgfqpoint{5.373989in}{4.579482in}}%
\pgfpathlineto{\pgfqpoint{6.398483in}{4.459631in}}%
\pgfpathlineto{\pgfqpoint{7.173483in}{4.369166in}}%
\pgfusepath{stroke}%
\end{pgfscope}%
\begin{pgfscope}%
\pgfpathrectangle{\pgfqpoint{1.000000in}{0.600000in}}{\pgfqpoint{6.200000in}{4.800000in}}%
\pgfusepath{clip}%
\pgfsetbuttcap%
\pgfsetroundjoin%
\definecolor{currentfill}{rgb}{0.750000,0.000000,0.750000}%
\pgfsetfillcolor{currentfill}%
\pgfsetlinewidth{0.501875pt}%
\definecolor{currentstroke}{rgb}{0.750000,0.000000,0.750000}%
\pgfsetstrokecolor{currentstroke}%
\pgfsetdash{}{0pt}%
\pgfsys@defobject{currentmarker}{\pgfqpoint{-0.020833in}{-0.020833in}}{\pgfqpoint{0.020833in}{0.020833in}}{%
\pgfpathmoveto{\pgfqpoint{0.000000in}{-0.020833in}}%
\pgfpathcurveto{\pgfqpoint{0.005525in}{-0.020833in}}{\pgfqpoint{0.010825in}{-0.018638in}}{\pgfqpoint{0.014731in}{-0.014731in}}%
\pgfpathcurveto{\pgfqpoint{0.018638in}{-0.010825in}}{\pgfqpoint{0.020833in}{-0.005525in}}{\pgfqpoint{0.020833in}{0.000000in}}%
\pgfpathcurveto{\pgfqpoint{0.020833in}{0.005525in}}{\pgfqpoint{0.018638in}{0.010825in}}{\pgfqpoint{0.014731in}{0.014731in}}%
\pgfpathcurveto{\pgfqpoint{0.010825in}{0.018638in}}{\pgfqpoint{0.005525in}{0.020833in}}{\pgfqpoint{0.000000in}{0.020833in}}%
\pgfpathcurveto{\pgfqpoint{-0.005525in}{0.020833in}}{\pgfqpoint{-0.010825in}{0.018638in}}{\pgfqpoint{-0.014731in}{0.014731in}}%
\pgfpathcurveto{\pgfqpoint{-0.018638in}{0.010825in}}{\pgfqpoint{-0.020833in}{0.005525in}}{\pgfqpoint{-0.020833in}{0.000000in}}%
\pgfpathcurveto{\pgfqpoint{-0.020833in}{-0.005525in}}{\pgfqpoint{-0.018638in}{-0.010825in}}{\pgfqpoint{-0.014731in}{-0.014731in}}%
\pgfpathcurveto{\pgfqpoint{-0.010825in}{-0.018638in}}{\pgfqpoint{-0.005525in}{-0.020833in}}{\pgfqpoint{0.000000in}{-0.020833in}}%
\pgfpathlineto{\pgfqpoint{0.000000in}{-0.020833in}}%
\pgfpathclose%
\pgfusepath{stroke,fill}%
}%
\begin{pgfscope}%
\pgfsys@transformshift{1.249494in}{5.093442in}%
\pgfsys@useobject{currentmarker}{}%
\end{pgfscope}%
\begin{pgfscope}%
\pgfsys@transformshift{2.024494in}{4.985228in}%
\pgfsys@useobject{currentmarker}{}%
\end{pgfscope}%
\begin{pgfscope}%
\pgfsys@transformshift{2.799494in}{4.886632in}%
\pgfsys@useobject{currentmarker}{}%
\end{pgfscope}%
\begin{pgfscope}%
\pgfsys@transformshift{3.823989in}{4.762455in}%
\pgfsys@useobject{currentmarker}{}%
\end{pgfscope}%
\begin{pgfscope}%
\pgfsys@transformshift{4.598989in}{4.670573in}%
\pgfsys@useobject{currentmarker}{}%
\end{pgfscope}%
\begin{pgfscope}%
\pgfsys@transformshift{5.373989in}{4.579482in}%
\pgfsys@useobject{currentmarker}{}%
\end{pgfscope}%
\begin{pgfscope}%
\pgfsys@transformshift{6.398483in}{4.459631in}%
\pgfsys@useobject{currentmarker}{}%
\end{pgfscope}%
\begin{pgfscope}%
\pgfsys@transformshift{7.173483in}{4.369166in}%
\pgfsys@useobject{currentmarker}{}%
\end{pgfscope}%
\end{pgfscope}%
\begin{pgfscope}%
\pgfpathrectangle{\pgfqpoint{1.000000in}{0.600000in}}{\pgfqpoint{6.200000in}{4.800000in}}%
\pgfusepath{clip}%
\pgfsetbuttcap%
\pgfsetroundjoin%
\pgfsetlinewidth{1.003750pt}%
\definecolor{currentstroke}{rgb}{0.000000,0.000000,1.000000}%
\pgfsetstrokecolor{currentstroke}%
\pgfsetdash{{6.000000pt}{6.000000pt}}{0.000000pt}%
\pgfpathmoveto{\pgfqpoint{1.249494in}{3.951005in}}%
\pgfpathlineto{\pgfqpoint{2.024494in}{3.527629in}}%
\pgfpathlineto{\pgfqpoint{2.799494in}{3.138070in}}%
\pgfpathlineto{\pgfqpoint{3.823989in}{2.644304in}}%
\pgfpathlineto{\pgfqpoint{4.598989in}{2.277772in}}%
\pgfpathlineto{\pgfqpoint{5.373989in}{1.913909in}}%
\pgfpathlineto{\pgfqpoint{6.398483in}{1.434812in}}%
\pgfpathlineto{\pgfqpoint{7.173483in}{1.073054in}}%
\pgfusepath{stroke}%
\end{pgfscope}%
\begin{pgfscope}%
\pgfpathrectangle{\pgfqpoint{1.000000in}{0.600000in}}{\pgfqpoint{6.200000in}{4.800000in}}%
\pgfusepath{clip}%
\pgfsetbuttcap%
\pgfsetroundjoin%
\definecolor{currentfill}{rgb}{0.000000,0.000000,1.000000}%
\pgfsetfillcolor{currentfill}%
\pgfsetlinewidth{0.501875pt}%
\definecolor{currentstroke}{rgb}{0.000000,0.000000,1.000000}%
\pgfsetstrokecolor{currentstroke}%
\pgfsetdash{}{0pt}%
\pgfsys@defobject{currentmarker}{\pgfqpoint{-0.020833in}{-0.020833in}}{\pgfqpoint{0.020833in}{0.020833in}}{%
\pgfpathmoveto{\pgfqpoint{0.000000in}{-0.020833in}}%
\pgfpathcurveto{\pgfqpoint{0.005525in}{-0.020833in}}{\pgfqpoint{0.010825in}{-0.018638in}}{\pgfqpoint{0.014731in}{-0.014731in}}%
\pgfpathcurveto{\pgfqpoint{0.018638in}{-0.010825in}}{\pgfqpoint{0.020833in}{-0.005525in}}{\pgfqpoint{0.020833in}{0.000000in}}%
\pgfpathcurveto{\pgfqpoint{0.020833in}{0.005525in}}{\pgfqpoint{0.018638in}{0.010825in}}{\pgfqpoint{0.014731in}{0.014731in}}%
\pgfpathcurveto{\pgfqpoint{0.010825in}{0.018638in}}{\pgfqpoint{0.005525in}{0.020833in}}{\pgfqpoint{0.000000in}{0.020833in}}%
\pgfpathcurveto{\pgfqpoint{-0.005525in}{0.020833in}}{\pgfqpoint{-0.010825in}{0.018638in}}{\pgfqpoint{-0.014731in}{0.014731in}}%
\pgfpathcurveto{\pgfqpoint{-0.018638in}{0.010825in}}{\pgfqpoint{-0.020833in}{0.005525in}}{\pgfqpoint{-0.020833in}{0.000000in}}%
\pgfpathcurveto{\pgfqpoint{-0.020833in}{-0.005525in}}{\pgfqpoint{-0.018638in}{-0.010825in}}{\pgfqpoint{-0.014731in}{-0.014731in}}%
\pgfpathcurveto{\pgfqpoint{-0.010825in}{-0.018638in}}{\pgfqpoint{-0.005525in}{-0.020833in}}{\pgfqpoint{0.000000in}{-0.020833in}}%
\pgfpathlineto{\pgfqpoint{0.000000in}{-0.020833in}}%
\pgfpathclose%
\pgfusepath{stroke,fill}%
}%
\begin{pgfscope}%
\pgfsys@transformshift{1.249494in}{3.951005in}%
\pgfsys@useobject{currentmarker}{}%
\end{pgfscope}%
\begin{pgfscope}%
\pgfsys@transformshift{2.024494in}{3.527629in}%
\pgfsys@useobject{currentmarker}{}%
\end{pgfscope}%
\begin{pgfscope}%
\pgfsys@transformshift{2.799494in}{3.138070in}%
\pgfsys@useobject{currentmarker}{}%
\end{pgfscope}%
\begin{pgfscope}%
\pgfsys@transformshift{3.823989in}{2.644304in}%
\pgfsys@useobject{currentmarker}{}%
\end{pgfscope}%
\begin{pgfscope}%
\pgfsys@transformshift{4.598989in}{2.277772in}%
\pgfsys@useobject{currentmarker}{}%
\end{pgfscope}%
\begin{pgfscope}%
\pgfsys@transformshift{5.373989in}{1.913909in}%
\pgfsys@useobject{currentmarker}{}%
\end{pgfscope}%
\begin{pgfscope}%
\pgfsys@transformshift{6.398483in}{1.434812in}%
\pgfsys@useobject{currentmarker}{}%
\end{pgfscope}%
\begin{pgfscope}%
\pgfsys@transformshift{7.173483in}{1.073054in}%
\pgfsys@useobject{currentmarker}{}%
\end{pgfscope}%
\end{pgfscope}%
\begin{pgfscope}%
\pgfsetrectcap%
\pgfsetmiterjoin%
\pgfsetlinewidth{1.003750pt}%
\definecolor{currentstroke}{rgb}{0.000000,0.000000,0.000000}%
\pgfsetstrokecolor{currentstroke}%
\pgfsetdash{}{0pt}%
\pgfpathmoveto{\pgfqpoint{1.000000in}{0.600000in}}%
\pgfpathlineto{\pgfqpoint{1.000000in}{5.400000in}}%
\pgfusepath{stroke}%
\end{pgfscope}%
\begin{pgfscope}%
\pgfsetrectcap%
\pgfsetmiterjoin%
\pgfsetlinewidth{1.003750pt}%
\definecolor{currentstroke}{rgb}{0.000000,0.000000,0.000000}%
\pgfsetstrokecolor{currentstroke}%
\pgfsetdash{}{0pt}%
\pgfpathmoveto{\pgfqpoint{7.200000in}{0.600000in}}%
\pgfpathlineto{\pgfqpoint{7.200000in}{5.400000in}}%
\pgfusepath{stroke}%
\end{pgfscope}%
\begin{pgfscope}%
\pgfsetrectcap%
\pgfsetmiterjoin%
\pgfsetlinewidth{1.003750pt}%
\definecolor{currentstroke}{rgb}{0.000000,0.000000,0.000000}%
\pgfsetstrokecolor{currentstroke}%
\pgfsetdash{}{0pt}%
\pgfpathmoveto{\pgfqpoint{1.000000in}{0.600000in}}%
\pgfpathlineto{\pgfqpoint{7.200000in}{0.600000in}}%
\pgfusepath{stroke}%
\end{pgfscope}%
\begin{pgfscope}%
\pgfsetrectcap%
\pgfsetmiterjoin%
\pgfsetlinewidth{1.003750pt}%
\definecolor{currentstroke}{rgb}{0.000000,0.000000,0.000000}%
\pgfsetstrokecolor{currentstroke}%
\pgfsetdash{}{0pt}%
\pgfpathmoveto{\pgfqpoint{1.000000in}{5.400000in}}%
\pgfpathlineto{\pgfqpoint{7.200000in}{5.400000in}}%
\pgfusepath{stroke}%
\end{pgfscope}%
\begin{pgfscope}%
\pgfpathrectangle{\pgfqpoint{1.000000in}{0.600000in}}{\pgfqpoint{6.200000in}{4.800000in}}%
\pgfusepath{clip}%
\pgfsetbuttcap%
\pgfsetroundjoin%
\pgfsetlinewidth{0.501875pt}%
\definecolor{currentstroke}{rgb}{0.000000,0.000000,0.000000}%
\pgfsetstrokecolor{currentstroke}%
\pgfsetdash{{1.000000pt}{3.000000pt}}{0.000000pt}%
\pgfpathmoveto{\pgfqpoint{1.000000in}{0.600000in}}%
\pgfpathlineto{\pgfqpoint{1.000000in}{5.400000in}}%
\pgfusepath{stroke}%
\end{pgfscope}%
\begin{pgfscope}%
\pgfsetbuttcap%
\pgfsetroundjoin%
\definecolor{currentfill}{rgb}{0.000000,0.000000,0.000000}%
\pgfsetfillcolor{currentfill}%
\pgfsetlinewidth{0.501875pt}%
\definecolor{currentstroke}{rgb}{0.000000,0.000000,0.000000}%
\pgfsetstrokecolor{currentstroke}%
\pgfsetdash{}{0pt}%
\pgfsys@defobject{currentmarker}{\pgfqpoint{0.000000in}{0.000000in}}{\pgfqpoint{0.000000in}{0.055556in}}{%
\pgfpathmoveto{\pgfqpoint{0.000000in}{0.000000in}}%
\pgfpathlineto{\pgfqpoint{0.000000in}{0.055556in}}%
\pgfusepath{stroke,fill}%
}%
\begin{pgfscope}%
\pgfsys@transformshift{1.000000in}{0.600000in}%
\pgfsys@useobject{currentmarker}{}%
\end{pgfscope}%
\end{pgfscope}%
\begin{pgfscope}%
\pgfsetbuttcap%
\pgfsetroundjoin%
\definecolor{currentfill}{rgb}{0.000000,0.000000,0.000000}%
\pgfsetfillcolor{currentfill}%
\pgfsetlinewidth{0.501875pt}%
\definecolor{currentstroke}{rgb}{0.000000,0.000000,0.000000}%
\pgfsetstrokecolor{currentstroke}%
\pgfsetdash{}{0pt}%
\pgfsys@defobject{currentmarker}{\pgfqpoint{0.000000in}{-0.055556in}}{\pgfqpoint{0.000000in}{0.000000in}}{%
\pgfpathmoveto{\pgfqpoint{0.000000in}{0.000000in}}%
\pgfpathlineto{\pgfqpoint{0.000000in}{-0.055556in}}%
\pgfusepath{stroke,fill}%
}%
\begin{pgfscope}%
\pgfsys@transformshift{1.000000in}{5.400000in}%
\pgfsys@useobject{currentmarker}{}%
\end{pgfscope}%
\end{pgfscope}%
\begin{pgfscope}%
\definecolor{textcolor}{rgb}{0.000000,0.000000,0.000000}%
\pgfsetstrokecolor{textcolor}%
\pgfsetfillcolor{textcolor}%
\pgftext[x=1.000000in,y=0.544444in,,top]{\color{textcolor}\rmfamily\fontsize{10.000000}{12.000000}\selectfont \(\displaystyle {4}\)}%
\end{pgfscope}%
\begin{pgfscope}%
\pgfpathrectangle{\pgfqpoint{1.000000in}{0.600000in}}{\pgfqpoint{6.200000in}{4.800000in}}%
\pgfusepath{clip}%
\pgfsetbuttcap%
\pgfsetroundjoin%
\pgfsetlinewidth{0.501875pt}%
\definecolor{currentstroke}{rgb}{0.000000,0.000000,0.000000}%
\pgfsetstrokecolor{currentstroke}%
\pgfsetdash{{1.000000pt}{3.000000pt}}{0.000000pt}%
\pgfpathmoveto{\pgfqpoint{1.775000in}{0.600000in}}%
\pgfpathlineto{\pgfqpoint{1.775000in}{5.400000in}}%
\pgfusepath{stroke}%
\end{pgfscope}%
\begin{pgfscope}%
\pgfsetbuttcap%
\pgfsetroundjoin%
\definecolor{currentfill}{rgb}{0.000000,0.000000,0.000000}%
\pgfsetfillcolor{currentfill}%
\pgfsetlinewidth{0.501875pt}%
\definecolor{currentstroke}{rgb}{0.000000,0.000000,0.000000}%
\pgfsetstrokecolor{currentstroke}%
\pgfsetdash{}{0pt}%
\pgfsys@defobject{currentmarker}{\pgfqpoint{0.000000in}{0.000000in}}{\pgfqpoint{0.000000in}{0.055556in}}{%
\pgfpathmoveto{\pgfqpoint{0.000000in}{0.000000in}}%
\pgfpathlineto{\pgfqpoint{0.000000in}{0.055556in}}%
\pgfusepath{stroke,fill}%
}%
\begin{pgfscope}%
\pgfsys@transformshift{1.775000in}{0.600000in}%
\pgfsys@useobject{currentmarker}{}%
\end{pgfscope}%
\end{pgfscope}%
\begin{pgfscope}%
\pgfsetbuttcap%
\pgfsetroundjoin%
\definecolor{currentfill}{rgb}{0.000000,0.000000,0.000000}%
\pgfsetfillcolor{currentfill}%
\pgfsetlinewidth{0.501875pt}%
\definecolor{currentstroke}{rgb}{0.000000,0.000000,0.000000}%
\pgfsetstrokecolor{currentstroke}%
\pgfsetdash{}{0pt}%
\pgfsys@defobject{currentmarker}{\pgfqpoint{0.000000in}{-0.055556in}}{\pgfqpoint{0.000000in}{0.000000in}}{%
\pgfpathmoveto{\pgfqpoint{0.000000in}{0.000000in}}%
\pgfpathlineto{\pgfqpoint{0.000000in}{-0.055556in}}%
\pgfusepath{stroke,fill}%
}%
\begin{pgfscope}%
\pgfsys@transformshift{1.775000in}{5.400000in}%
\pgfsys@useobject{currentmarker}{}%
\end{pgfscope}%
\end{pgfscope}%
\begin{pgfscope}%
\definecolor{textcolor}{rgb}{0.000000,0.000000,0.000000}%
\pgfsetstrokecolor{textcolor}%
\pgfsetfillcolor{textcolor}%
\pgftext[x=1.775000in,y=0.544444in,,top]{\color{textcolor}\rmfamily\fontsize{10.000000}{12.000000}\selectfont \(\displaystyle {8}\)}%
\end{pgfscope}%
\begin{pgfscope}%
\pgfpathrectangle{\pgfqpoint{1.000000in}{0.600000in}}{\pgfqpoint{6.200000in}{4.800000in}}%
\pgfusepath{clip}%
\pgfsetbuttcap%
\pgfsetroundjoin%
\pgfsetlinewidth{0.501875pt}%
\definecolor{currentstroke}{rgb}{0.000000,0.000000,0.000000}%
\pgfsetstrokecolor{currentstroke}%
\pgfsetdash{{1.000000pt}{3.000000pt}}{0.000000pt}%
\pgfpathmoveto{\pgfqpoint{2.550000in}{0.600000in}}%
\pgfpathlineto{\pgfqpoint{2.550000in}{5.400000in}}%
\pgfusepath{stroke}%
\end{pgfscope}%
\begin{pgfscope}%
\pgfsetbuttcap%
\pgfsetroundjoin%
\definecolor{currentfill}{rgb}{0.000000,0.000000,0.000000}%
\pgfsetfillcolor{currentfill}%
\pgfsetlinewidth{0.501875pt}%
\definecolor{currentstroke}{rgb}{0.000000,0.000000,0.000000}%
\pgfsetstrokecolor{currentstroke}%
\pgfsetdash{}{0pt}%
\pgfsys@defobject{currentmarker}{\pgfqpoint{0.000000in}{0.000000in}}{\pgfqpoint{0.000000in}{0.055556in}}{%
\pgfpathmoveto{\pgfqpoint{0.000000in}{0.000000in}}%
\pgfpathlineto{\pgfqpoint{0.000000in}{0.055556in}}%
\pgfusepath{stroke,fill}%
}%
\begin{pgfscope}%
\pgfsys@transformshift{2.550000in}{0.600000in}%
\pgfsys@useobject{currentmarker}{}%
\end{pgfscope}%
\end{pgfscope}%
\begin{pgfscope}%
\pgfsetbuttcap%
\pgfsetroundjoin%
\definecolor{currentfill}{rgb}{0.000000,0.000000,0.000000}%
\pgfsetfillcolor{currentfill}%
\pgfsetlinewidth{0.501875pt}%
\definecolor{currentstroke}{rgb}{0.000000,0.000000,0.000000}%
\pgfsetstrokecolor{currentstroke}%
\pgfsetdash{}{0pt}%
\pgfsys@defobject{currentmarker}{\pgfqpoint{0.000000in}{-0.055556in}}{\pgfqpoint{0.000000in}{0.000000in}}{%
\pgfpathmoveto{\pgfqpoint{0.000000in}{0.000000in}}%
\pgfpathlineto{\pgfqpoint{0.000000in}{-0.055556in}}%
\pgfusepath{stroke,fill}%
}%
\begin{pgfscope}%
\pgfsys@transformshift{2.550000in}{5.400000in}%
\pgfsys@useobject{currentmarker}{}%
\end{pgfscope}%
\end{pgfscope}%
\begin{pgfscope}%
\definecolor{textcolor}{rgb}{0.000000,0.000000,0.000000}%
\pgfsetstrokecolor{textcolor}%
\pgfsetfillcolor{textcolor}%
\pgftext[x=2.550000in,y=0.544444in,,top]{\color{textcolor}\rmfamily\fontsize{10.000000}{12.000000}\selectfont \(\displaystyle {16}\)}%
\end{pgfscope}%
\begin{pgfscope}%
\pgfpathrectangle{\pgfqpoint{1.000000in}{0.600000in}}{\pgfqpoint{6.200000in}{4.800000in}}%
\pgfusepath{clip}%
\pgfsetbuttcap%
\pgfsetroundjoin%
\pgfsetlinewidth{0.501875pt}%
\definecolor{currentstroke}{rgb}{0.000000,0.000000,0.000000}%
\pgfsetstrokecolor{currentstroke}%
\pgfsetdash{{1.000000pt}{3.000000pt}}{0.000000pt}%
\pgfpathmoveto{\pgfqpoint{3.325000in}{0.600000in}}%
\pgfpathlineto{\pgfqpoint{3.325000in}{5.400000in}}%
\pgfusepath{stroke}%
\end{pgfscope}%
\begin{pgfscope}%
\pgfsetbuttcap%
\pgfsetroundjoin%
\definecolor{currentfill}{rgb}{0.000000,0.000000,0.000000}%
\pgfsetfillcolor{currentfill}%
\pgfsetlinewidth{0.501875pt}%
\definecolor{currentstroke}{rgb}{0.000000,0.000000,0.000000}%
\pgfsetstrokecolor{currentstroke}%
\pgfsetdash{}{0pt}%
\pgfsys@defobject{currentmarker}{\pgfqpoint{0.000000in}{0.000000in}}{\pgfqpoint{0.000000in}{0.055556in}}{%
\pgfpathmoveto{\pgfqpoint{0.000000in}{0.000000in}}%
\pgfpathlineto{\pgfqpoint{0.000000in}{0.055556in}}%
\pgfusepath{stroke,fill}%
}%
\begin{pgfscope}%
\pgfsys@transformshift{3.325000in}{0.600000in}%
\pgfsys@useobject{currentmarker}{}%
\end{pgfscope}%
\end{pgfscope}%
\begin{pgfscope}%
\pgfsetbuttcap%
\pgfsetroundjoin%
\definecolor{currentfill}{rgb}{0.000000,0.000000,0.000000}%
\pgfsetfillcolor{currentfill}%
\pgfsetlinewidth{0.501875pt}%
\definecolor{currentstroke}{rgb}{0.000000,0.000000,0.000000}%
\pgfsetstrokecolor{currentstroke}%
\pgfsetdash{}{0pt}%
\pgfsys@defobject{currentmarker}{\pgfqpoint{0.000000in}{-0.055556in}}{\pgfqpoint{0.000000in}{0.000000in}}{%
\pgfpathmoveto{\pgfqpoint{0.000000in}{0.000000in}}%
\pgfpathlineto{\pgfqpoint{0.000000in}{-0.055556in}}%
\pgfusepath{stroke,fill}%
}%
\begin{pgfscope}%
\pgfsys@transformshift{3.325000in}{5.400000in}%
\pgfsys@useobject{currentmarker}{}%
\end{pgfscope}%
\end{pgfscope}%
\begin{pgfscope}%
\definecolor{textcolor}{rgb}{0.000000,0.000000,0.000000}%
\pgfsetstrokecolor{textcolor}%
\pgfsetfillcolor{textcolor}%
\pgftext[x=3.325000in,y=0.544444in,,top]{\color{textcolor}\rmfamily\fontsize{10.000000}{12.000000}\selectfont \(\displaystyle {32}\)}%
\end{pgfscope}%
\begin{pgfscope}%
\pgfpathrectangle{\pgfqpoint{1.000000in}{0.600000in}}{\pgfqpoint{6.200000in}{4.800000in}}%
\pgfusepath{clip}%
\pgfsetbuttcap%
\pgfsetroundjoin%
\pgfsetlinewidth{0.501875pt}%
\definecolor{currentstroke}{rgb}{0.000000,0.000000,0.000000}%
\pgfsetstrokecolor{currentstroke}%
\pgfsetdash{{1.000000pt}{3.000000pt}}{0.000000pt}%
\pgfpathmoveto{\pgfqpoint{4.100000in}{0.600000in}}%
\pgfpathlineto{\pgfqpoint{4.100000in}{5.400000in}}%
\pgfusepath{stroke}%
\end{pgfscope}%
\begin{pgfscope}%
\pgfsetbuttcap%
\pgfsetroundjoin%
\definecolor{currentfill}{rgb}{0.000000,0.000000,0.000000}%
\pgfsetfillcolor{currentfill}%
\pgfsetlinewidth{0.501875pt}%
\definecolor{currentstroke}{rgb}{0.000000,0.000000,0.000000}%
\pgfsetstrokecolor{currentstroke}%
\pgfsetdash{}{0pt}%
\pgfsys@defobject{currentmarker}{\pgfqpoint{0.000000in}{0.000000in}}{\pgfqpoint{0.000000in}{0.055556in}}{%
\pgfpathmoveto{\pgfqpoint{0.000000in}{0.000000in}}%
\pgfpathlineto{\pgfqpoint{0.000000in}{0.055556in}}%
\pgfusepath{stroke,fill}%
}%
\begin{pgfscope}%
\pgfsys@transformshift{4.100000in}{0.600000in}%
\pgfsys@useobject{currentmarker}{}%
\end{pgfscope}%
\end{pgfscope}%
\begin{pgfscope}%
\pgfsetbuttcap%
\pgfsetroundjoin%
\definecolor{currentfill}{rgb}{0.000000,0.000000,0.000000}%
\pgfsetfillcolor{currentfill}%
\pgfsetlinewidth{0.501875pt}%
\definecolor{currentstroke}{rgb}{0.000000,0.000000,0.000000}%
\pgfsetstrokecolor{currentstroke}%
\pgfsetdash{}{0pt}%
\pgfsys@defobject{currentmarker}{\pgfqpoint{0.000000in}{-0.055556in}}{\pgfqpoint{0.000000in}{0.000000in}}{%
\pgfpathmoveto{\pgfqpoint{0.000000in}{0.000000in}}%
\pgfpathlineto{\pgfqpoint{0.000000in}{-0.055556in}}%
\pgfusepath{stroke,fill}%
}%
\begin{pgfscope}%
\pgfsys@transformshift{4.100000in}{5.400000in}%
\pgfsys@useobject{currentmarker}{}%
\end{pgfscope}%
\end{pgfscope}%
\begin{pgfscope}%
\definecolor{textcolor}{rgb}{0.000000,0.000000,0.000000}%
\pgfsetstrokecolor{textcolor}%
\pgfsetfillcolor{textcolor}%
\pgftext[x=4.100000in,y=0.544444in,,top]{\color{textcolor}\rmfamily\fontsize{10.000000}{12.000000}\selectfont \(\displaystyle {64}\)}%
\end{pgfscope}%
\begin{pgfscope}%
\pgfpathrectangle{\pgfqpoint{1.000000in}{0.600000in}}{\pgfqpoint{6.200000in}{4.800000in}}%
\pgfusepath{clip}%
\pgfsetbuttcap%
\pgfsetroundjoin%
\pgfsetlinewidth{0.501875pt}%
\definecolor{currentstroke}{rgb}{0.000000,0.000000,0.000000}%
\pgfsetstrokecolor{currentstroke}%
\pgfsetdash{{1.000000pt}{3.000000pt}}{0.000000pt}%
\pgfpathmoveto{\pgfqpoint{4.875000in}{0.600000in}}%
\pgfpathlineto{\pgfqpoint{4.875000in}{5.400000in}}%
\pgfusepath{stroke}%
\end{pgfscope}%
\begin{pgfscope}%
\pgfsetbuttcap%
\pgfsetroundjoin%
\definecolor{currentfill}{rgb}{0.000000,0.000000,0.000000}%
\pgfsetfillcolor{currentfill}%
\pgfsetlinewidth{0.501875pt}%
\definecolor{currentstroke}{rgb}{0.000000,0.000000,0.000000}%
\pgfsetstrokecolor{currentstroke}%
\pgfsetdash{}{0pt}%
\pgfsys@defobject{currentmarker}{\pgfqpoint{0.000000in}{0.000000in}}{\pgfqpoint{0.000000in}{0.055556in}}{%
\pgfpathmoveto{\pgfqpoint{0.000000in}{0.000000in}}%
\pgfpathlineto{\pgfqpoint{0.000000in}{0.055556in}}%
\pgfusepath{stroke,fill}%
}%
\begin{pgfscope}%
\pgfsys@transformshift{4.875000in}{0.600000in}%
\pgfsys@useobject{currentmarker}{}%
\end{pgfscope}%
\end{pgfscope}%
\begin{pgfscope}%
\pgfsetbuttcap%
\pgfsetroundjoin%
\definecolor{currentfill}{rgb}{0.000000,0.000000,0.000000}%
\pgfsetfillcolor{currentfill}%
\pgfsetlinewidth{0.501875pt}%
\definecolor{currentstroke}{rgb}{0.000000,0.000000,0.000000}%
\pgfsetstrokecolor{currentstroke}%
\pgfsetdash{}{0pt}%
\pgfsys@defobject{currentmarker}{\pgfqpoint{0.000000in}{-0.055556in}}{\pgfqpoint{0.000000in}{0.000000in}}{%
\pgfpathmoveto{\pgfqpoint{0.000000in}{0.000000in}}%
\pgfpathlineto{\pgfqpoint{0.000000in}{-0.055556in}}%
\pgfusepath{stroke,fill}%
}%
\begin{pgfscope}%
\pgfsys@transformshift{4.875000in}{5.400000in}%
\pgfsys@useobject{currentmarker}{}%
\end{pgfscope}%
\end{pgfscope}%
\begin{pgfscope}%
\definecolor{textcolor}{rgb}{0.000000,0.000000,0.000000}%
\pgfsetstrokecolor{textcolor}%
\pgfsetfillcolor{textcolor}%
\pgftext[x=4.875000in,y=0.544444in,,top]{\color{textcolor}\rmfamily\fontsize{10.000000}{12.000000}\selectfont \(\displaystyle {128}\)}%
\end{pgfscope}%
\begin{pgfscope}%
\pgfpathrectangle{\pgfqpoint{1.000000in}{0.600000in}}{\pgfqpoint{6.200000in}{4.800000in}}%
\pgfusepath{clip}%
\pgfsetbuttcap%
\pgfsetroundjoin%
\pgfsetlinewidth{0.501875pt}%
\definecolor{currentstroke}{rgb}{0.000000,0.000000,0.000000}%
\pgfsetstrokecolor{currentstroke}%
\pgfsetdash{{1.000000pt}{3.000000pt}}{0.000000pt}%
\pgfpathmoveto{\pgfqpoint{5.650000in}{0.600000in}}%
\pgfpathlineto{\pgfqpoint{5.650000in}{5.400000in}}%
\pgfusepath{stroke}%
\end{pgfscope}%
\begin{pgfscope}%
\pgfsetbuttcap%
\pgfsetroundjoin%
\definecolor{currentfill}{rgb}{0.000000,0.000000,0.000000}%
\pgfsetfillcolor{currentfill}%
\pgfsetlinewidth{0.501875pt}%
\definecolor{currentstroke}{rgb}{0.000000,0.000000,0.000000}%
\pgfsetstrokecolor{currentstroke}%
\pgfsetdash{}{0pt}%
\pgfsys@defobject{currentmarker}{\pgfqpoint{0.000000in}{0.000000in}}{\pgfqpoint{0.000000in}{0.055556in}}{%
\pgfpathmoveto{\pgfqpoint{0.000000in}{0.000000in}}%
\pgfpathlineto{\pgfqpoint{0.000000in}{0.055556in}}%
\pgfusepath{stroke,fill}%
}%
\begin{pgfscope}%
\pgfsys@transformshift{5.650000in}{0.600000in}%
\pgfsys@useobject{currentmarker}{}%
\end{pgfscope}%
\end{pgfscope}%
\begin{pgfscope}%
\pgfsetbuttcap%
\pgfsetroundjoin%
\definecolor{currentfill}{rgb}{0.000000,0.000000,0.000000}%
\pgfsetfillcolor{currentfill}%
\pgfsetlinewidth{0.501875pt}%
\definecolor{currentstroke}{rgb}{0.000000,0.000000,0.000000}%
\pgfsetstrokecolor{currentstroke}%
\pgfsetdash{}{0pt}%
\pgfsys@defobject{currentmarker}{\pgfqpoint{0.000000in}{-0.055556in}}{\pgfqpoint{0.000000in}{0.000000in}}{%
\pgfpathmoveto{\pgfqpoint{0.000000in}{0.000000in}}%
\pgfpathlineto{\pgfqpoint{0.000000in}{-0.055556in}}%
\pgfusepath{stroke,fill}%
}%
\begin{pgfscope}%
\pgfsys@transformshift{5.650000in}{5.400000in}%
\pgfsys@useobject{currentmarker}{}%
\end{pgfscope}%
\end{pgfscope}%
\begin{pgfscope}%
\definecolor{textcolor}{rgb}{0.000000,0.000000,0.000000}%
\pgfsetstrokecolor{textcolor}%
\pgfsetfillcolor{textcolor}%
\pgftext[x=5.650000in,y=0.544444in,,top]{\color{textcolor}\rmfamily\fontsize{10.000000}{12.000000}\selectfont \(\displaystyle {256}\)}%
\end{pgfscope}%
\begin{pgfscope}%
\pgfpathrectangle{\pgfqpoint{1.000000in}{0.600000in}}{\pgfqpoint{6.200000in}{4.800000in}}%
\pgfusepath{clip}%
\pgfsetbuttcap%
\pgfsetroundjoin%
\pgfsetlinewidth{0.501875pt}%
\definecolor{currentstroke}{rgb}{0.000000,0.000000,0.000000}%
\pgfsetstrokecolor{currentstroke}%
\pgfsetdash{{1.000000pt}{3.000000pt}}{0.000000pt}%
\pgfpathmoveto{\pgfqpoint{6.425000in}{0.600000in}}%
\pgfpathlineto{\pgfqpoint{6.425000in}{5.400000in}}%
\pgfusepath{stroke}%
\end{pgfscope}%
\begin{pgfscope}%
\pgfsetbuttcap%
\pgfsetroundjoin%
\definecolor{currentfill}{rgb}{0.000000,0.000000,0.000000}%
\pgfsetfillcolor{currentfill}%
\pgfsetlinewidth{0.501875pt}%
\definecolor{currentstroke}{rgb}{0.000000,0.000000,0.000000}%
\pgfsetstrokecolor{currentstroke}%
\pgfsetdash{}{0pt}%
\pgfsys@defobject{currentmarker}{\pgfqpoint{0.000000in}{0.000000in}}{\pgfqpoint{0.000000in}{0.055556in}}{%
\pgfpathmoveto{\pgfqpoint{0.000000in}{0.000000in}}%
\pgfpathlineto{\pgfqpoint{0.000000in}{0.055556in}}%
\pgfusepath{stroke,fill}%
}%
\begin{pgfscope}%
\pgfsys@transformshift{6.425000in}{0.600000in}%
\pgfsys@useobject{currentmarker}{}%
\end{pgfscope}%
\end{pgfscope}%
\begin{pgfscope}%
\pgfsetbuttcap%
\pgfsetroundjoin%
\definecolor{currentfill}{rgb}{0.000000,0.000000,0.000000}%
\pgfsetfillcolor{currentfill}%
\pgfsetlinewidth{0.501875pt}%
\definecolor{currentstroke}{rgb}{0.000000,0.000000,0.000000}%
\pgfsetstrokecolor{currentstroke}%
\pgfsetdash{}{0pt}%
\pgfsys@defobject{currentmarker}{\pgfqpoint{0.000000in}{-0.055556in}}{\pgfqpoint{0.000000in}{0.000000in}}{%
\pgfpathmoveto{\pgfqpoint{0.000000in}{0.000000in}}%
\pgfpathlineto{\pgfqpoint{0.000000in}{-0.055556in}}%
\pgfusepath{stroke,fill}%
}%
\begin{pgfscope}%
\pgfsys@transformshift{6.425000in}{5.400000in}%
\pgfsys@useobject{currentmarker}{}%
\end{pgfscope}%
\end{pgfscope}%
\begin{pgfscope}%
\definecolor{textcolor}{rgb}{0.000000,0.000000,0.000000}%
\pgfsetstrokecolor{textcolor}%
\pgfsetfillcolor{textcolor}%
\pgftext[x=6.425000in,y=0.544444in,,top]{\color{textcolor}\rmfamily\fontsize{10.000000}{12.000000}\selectfont \(\displaystyle {512}\)}%
\end{pgfscope}%
\begin{pgfscope}%
\pgfpathrectangle{\pgfqpoint{1.000000in}{0.600000in}}{\pgfqpoint{6.200000in}{4.800000in}}%
\pgfusepath{clip}%
\pgfsetbuttcap%
\pgfsetroundjoin%
\pgfsetlinewidth{0.501875pt}%
\definecolor{currentstroke}{rgb}{0.000000,0.000000,0.000000}%
\pgfsetstrokecolor{currentstroke}%
\pgfsetdash{{1.000000pt}{3.000000pt}}{0.000000pt}%
\pgfpathmoveto{\pgfqpoint{7.200000in}{0.600000in}}%
\pgfpathlineto{\pgfqpoint{7.200000in}{5.400000in}}%
\pgfusepath{stroke}%
\end{pgfscope}%
\begin{pgfscope}%
\pgfsetbuttcap%
\pgfsetroundjoin%
\definecolor{currentfill}{rgb}{0.000000,0.000000,0.000000}%
\pgfsetfillcolor{currentfill}%
\pgfsetlinewidth{0.501875pt}%
\definecolor{currentstroke}{rgb}{0.000000,0.000000,0.000000}%
\pgfsetstrokecolor{currentstroke}%
\pgfsetdash{}{0pt}%
\pgfsys@defobject{currentmarker}{\pgfqpoint{0.000000in}{0.000000in}}{\pgfqpoint{0.000000in}{0.055556in}}{%
\pgfpathmoveto{\pgfqpoint{0.000000in}{0.000000in}}%
\pgfpathlineto{\pgfqpoint{0.000000in}{0.055556in}}%
\pgfusepath{stroke,fill}%
}%
\begin{pgfscope}%
\pgfsys@transformshift{7.200000in}{0.600000in}%
\pgfsys@useobject{currentmarker}{}%
\end{pgfscope}%
\end{pgfscope}%
\begin{pgfscope}%
\pgfsetbuttcap%
\pgfsetroundjoin%
\definecolor{currentfill}{rgb}{0.000000,0.000000,0.000000}%
\pgfsetfillcolor{currentfill}%
\pgfsetlinewidth{0.501875pt}%
\definecolor{currentstroke}{rgb}{0.000000,0.000000,0.000000}%
\pgfsetstrokecolor{currentstroke}%
\pgfsetdash{}{0pt}%
\pgfsys@defobject{currentmarker}{\pgfqpoint{0.000000in}{-0.055556in}}{\pgfqpoint{0.000000in}{0.000000in}}{%
\pgfpathmoveto{\pgfqpoint{0.000000in}{0.000000in}}%
\pgfpathlineto{\pgfqpoint{0.000000in}{-0.055556in}}%
\pgfusepath{stroke,fill}%
}%
\begin{pgfscope}%
\pgfsys@transformshift{7.200000in}{5.400000in}%
\pgfsys@useobject{currentmarker}{}%
\end{pgfscope}%
\end{pgfscope}%
\begin{pgfscope}%
\definecolor{textcolor}{rgb}{0.000000,0.000000,0.000000}%
\pgfsetstrokecolor{textcolor}%
\pgfsetfillcolor{textcolor}%
\pgftext[x=7.200000in,y=0.544444in,,top]{\color{textcolor}\rmfamily\fontsize{10.000000}{12.000000}\selectfont \(\displaystyle {1024}\)}%
\end{pgfscope}%
\begin{pgfscope}%
\definecolor{textcolor}{rgb}{0.000000,0.000000,0.000000}%
\pgfsetstrokecolor{textcolor}%
\pgfsetfillcolor{textcolor}%
\pgftext[x=4.100000in,y=0.351543in,,top]{\color{textcolor}\rmfamily\fontsize{12.000000}{14.400000}\selectfont \(\displaystyle number\ of\ grid\ points\)}%
\end{pgfscope}%
\begin{pgfscope}%
\pgfpathrectangle{\pgfqpoint{1.000000in}{0.600000in}}{\pgfqpoint{6.200000in}{4.800000in}}%
\pgfusepath{clip}%
\pgfsetbuttcap%
\pgfsetroundjoin%
\pgfsetlinewidth{0.501875pt}%
\definecolor{currentstroke}{rgb}{0.000000,0.000000,0.000000}%
\pgfsetstrokecolor{currentstroke}%
\pgfsetdash{{1.000000pt}{3.000000pt}}{0.000000pt}%
\pgfpathmoveto{\pgfqpoint{1.000000in}{0.600000in}}%
\pgfpathlineto{\pgfqpoint{7.200000in}{0.600000in}}%
\pgfusepath{stroke}%
\end{pgfscope}%
\begin{pgfscope}%
\pgfsetbuttcap%
\pgfsetroundjoin%
\definecolor{currentfill}{rgb}{0.000000,0.000000,0.000000}%
\pgfsetfillcolor{currentfill}%
\pgfsetlinewidth{0.501875pt}%
\definecolor{currentstroke}{rgb}{0.000000,0.000000,0.000000}%
\pgfsetstrokecolor{currentstroke}%
\pgfsetdash{}{0pt}%
\pgfsys@defobject{currentmarker}{\pgfqpoint{0.000000in}{0.000000in}}{\pgfqpoint{0.055556in}{0.000000in}}{%
\pgfpathmoveto{\pgfqpoint{0.000000in}{0.000000in}}%
\pgfpathlineto{\pgfqpoint{0.055556in}{0.000000in}}%
\pgfusepath{stroke,fill}%
}%
\begin{pgfscope}%
\pgfsys@transformshift{1.000000in}{0.600000in}%
\pgfsys@useobject{currentmarker}{}%
\end{pgfscope}%
\end{pgfscope}%
\begin{pgfscope}%
\pgfsetbuttcap%
\pgfsetroundjoin%
\definecolor{currentfill}{rgb}{0.000000,0.000000,0.000000}%
\pgfsetfillcolor{currentfill}%
\pgfsetlinewidth{0.501875pt}%
\definecolor{currentstroke}{rgb}{0.000000,0.000000,0.000000}%
\pgfsetstrokecolor{currentstroke}%
\pgfsetdash{}{0pt}%
\pgfsys@defobject{currentmarker}{\pgfqpoint{-0.055556in}{0.000000in}}{\pgfqpoint{-0.000000in}{0.000000in}}{%
\pgfpathmoveto{\pgfqpoint{-0.000000in}{0.000000in}}%
\pgfpathlineto{\pgfqpoint{-0.055556in}{0.000000in}}%
\pgfusepath{stroke,fill}%
}%
\begin{pgfscope}%
\pgfsys@transformshift{7.200000in}{0.600000in}%
\pgfsys@useobject{currentmarker}{}%
\end{pgfscope}%
\end{pgfscope}%
\begin{pgfscope}%
\definecolor{textcolor}{rgb}{0.000000,0.000000,0.000000}%
\pgfsetstrokecolor{textcolor}%
\pgfsetfillcolor{textcolor}%
\pgftext[x=0.944444in,y=0.600000in,right,]{\color{textcolor}\rmfamily\fontsize{10.000000}{12.000000}\selectfont \(\displaystyle {10^{-8}}\)}%
\end{pgfscope}%
\begin{pgfscope}%
\pgfpathrectangle{\pgfqpoint{1.000000in}{0.600000in}}{\pgfqpoint{6.200000in}{4.800000in}}%
\pgfusepath{clip}%
\pgfsetbuttcap%
\pgfsetroundjoin%
\pgfsetlinewidth{0.501875pt}%
\definecolor{currentstroke}{rgb}{0.000000,0.000000,0.000000}%
\pgfsetstrokecolor{currentstroke}%
\pgfsetdash{{1.000000pt}{3.000000pt}}{0.000000pt}%
\pgfpathmoveto{\pgfqpoint{1.000000in}{1.200000in}}%
\pgfpathlineto{\pgfqpoint{7.200000in}{1.200000in}}%
\pgfusepath{stroke}%
\end{pgfscope}%
\begin{pgfscope}%
\pgfsetbuttcap%
\pgfsetroundjoin%
\definecolor{currentfill}{rgb}{0.000000,0.000000,0.000000}%
\pgfsetfillcolor{currentfill}%
\pgfsetlinewidth{0.501875pt}%
\definecolor{currentstroke}{rgb}{0.000000,0.000000,0.000000}%
\pgfsetstrokecolor{currentstroke}%
\pgfsetdash{}{0pt}%
\pgfsys@defobject{currentmarker}{\pgfqpoint{0.000000in}{0.000000in}}{\pgfqpoint{0.055556in}{0.000000in}}{%
\pgfpathmoveto{\pgfqpoint{0.000000in}{0.000000in}}%
\pgfpathlineto{\pgfqpoint{0.055556in}{0.000000in}}%
\pgfusepath{stroke,fill}%
}%
\begin{pgfscope}%
\pgfsys@transformshift{1.000000in}{1.200000in}%
\pgfsys@useobject{currentmarker}{}%
\end{pgfscope}%
\end{pgfscope}%
\begin{pgfscope}%
\pgfsetbuttcap%
\pgfsetroundjoin%
\definecolor{currentfill}{rgb}{0.000000,0.000000,0.000000}%
\pgfsetfillcolor{currentfill}%
\pgfsetlinewidth{0.501875pt}%
\definecolor{currentstroke}{rgb}{0.000000,0.000000,0.000000}%
\pgfsetstrokecolor{currentstroke}%
\pgfsetdash{}{0pt}%
\pgfsys@defobject{currentmarker}{\pgfqpoint{-0.055556in}{0.000000in}}{\pgfqpoint{-0.000000in}{0.000000in}}{%
\pgfpathmoveto{\pgfqpoint{-0.000000in}{0.000000in}}%
\pgfpathlineto{\pgfqpoint{-0.055556in}{0.000000in}}%
\pgfusepath{stroke,fill}%
}%
\begin{pgfscope}%
\pgfsys@transformshift{7.200000in}{1.200000in}%
\pgfsys@useobject{currentmarker}{}%
\end{pgfscope}%
\end{pgfscope}%
\begin{pgfscope}%
\definecolor{textcolor}{rgb}{0.000000,0.000000,0.000000}%
\pgfsetstrokecolor{textcolor}%
\pgfsetfillcolor{textcolor}%
\pgftext[x=0.944444in,y=1.200000in,right,]{\color{textcolor}\rmfamily\fontsize{10.000000}{12.000000}\selectfont \(\displaystyle {10^{-7}}\)}%
\end{pgfscope}%
\begin{pgfscope}%
\pgfpathrectangle{\pgfqpoint{1.000000in}{0.600000in}}{\pgfqpoint{6.200000in}{4.800000in}}%
\pgfusepath{clip}%
\pgfsetbuttcap%
\pgfsetroundjoin%
\pgfsetlinewidth{0.501875pt}%
\definecolor{currentstroke}{rgb}{0.000000,0.000000,0.000000}%
\pgfsetstrokecolor{currentstroke}%
\pgfsetdash{{1.000000pt}{3.000000pt}}{0.000000pt}%
\pgfpathmoveto{\pgfqpoint{1.000000in}{1.800000in}}%
\pgfpathlineto{\pgfqpoint{7.200000in}{1.800000in}}%
\pgfusepath{stroke}%
\end{pgfscope}%
\begin{pgfscope}%
\pgfsetbuttcap%
\pgfsetroundjoin%
\definecolor{currentfill}{rgb}{0.000000,0.000000,0.000000}%
\pgfsetfillcolor{currentfill}%
\pgfsetlinewidth{0.501875pt}%
\definecolor{currentstroke}{rgb}{0.000000,0.000000,0.000000}%
\pgfsetstrokecolor{currentstroke}%
\pgfsetdash{}{0pt}%
\pgfsys@defobject{currentmarker}{\pgfqpoint{0.000000in}{0.000000in}}{\pgfqpoint{0.055556in}{0.000000in}}{%
\pgfpathmoveto{\pgfqpoint{0.000000in}{0.000000in}}%
\pgfpathlineto{\pgfqpoint{0.055556in}{0.000000in}}%
\pgfusepath{stroke,fill}%
}%
\begin{pgfscope}%
\pgfsys@transformshift{1.000000in}{1.800000in}%
\pgfsys@useobject{currentmarker}{}%
\end{pgfscope}%
\end{pgfscope}%
\begin{pgfscope}%
\pgfsetbuttcap%
\pgfsetroundjoin%
\definecolor{currentfill}{rgb}{0.000000,0.000000,0.000000}%
\pgfsetfillcolor{currentfill}%
\pgfsetlinewidth{0.501875pt}%
\definecolor{currentstroke}{rgb}{0.000000,0.000000,0.000000}%
\pgfsetstrokecolor{currentstroke}%
\pgfsetdash{}{0pt}%
\pgfsys@defobject{currentmarker}{\pgfqpoint{-0.055556in}{0.000000in}}{\pgfqpoint{-0.000000in}{0.000000in}}{%
\pgfpathmoveto{\pgfqpoint{-0.000000in}{0.000000in}}%
\pgfpathlineto{\pgfqpoint{-0.055556in}{0.000000in}}%
\pgfusepath{stroke,fill}%
}%
\begin{pgfscope}%
\pgfsys@transformshift{7.200000in}{1.800000in}%
\pgfsys@useobject{currentmarker}{}%
\end{pgfscope}%
\end{pgfscope}%
\begin{pgfscope}%
\definecolor{textcolor}{rgb}{0.000000,0.000000,0.000000}%
\pgfsetstrokecolor{textcolor}%
\pgfsetfillcolor{textcolor}%
\pgftext[x=0.944444in,y=1.800000in,right,]{\color{textcolor}\rmfamily\fontsize{10.000000}{12.000000}\selectfont \(\displaystyle {10^{-6}}\)}%
\end{pgfscope}%
\begin{pgfscope}%
\pgfpathrectangle{\pgfqpoint{1.000000in}{0.600000in}}{\pgfqpoint{6.200000in}{4.800000in}}%
\pgfusepath{clip}%
\pgfsetbuttcap%
\pgfsetroundjoin%
\pgfsetlinewidth{0.501875pt}%
\definecolor{currentstroke}{rgb}{0.000000,0.000000,0.000000}%
\pgfsetstrokecolor{currentstroke}%
\pgfsetdash{{1.000000pt}{3.000000pt}}{0.000000pt}%
\pgfpathmoveto{\pgfqpoint{1.000000in}{2.400000in}}%
\pgfpathlineto{\pgfqpoint{7.200000in}{2.400000in}}%
\pgfusepath{stroke}%
\end{pgfscope}%
\begin{pgfscope}%
\pgfsetbuttcap%
\pgfsetroundjoin%
\definecolor{currentfill}{rgb}{0.000000,0.000000,0.000000}%
\pgfsetfillcolor{currentfill}%
\pgfsetlinewidth{0.501875pt}%
\definecolor{currentstroke}{rgb}{0.000000,0.000000,0.000000}%
\pgfsetstrokecolor{currentstroke}%
\pgfsetdash{}{0pt}%
\pgfsys@defobject{currentmarker}{\pgfqpoint{0.000000in}{0.000000in}}{\pgfqpoint{0.055556in}{0.000000in}}{%
\pgfpathmoveto{\pgfqpoint{0.000000in}{0.000000in}}%
\pgfpathlineto{\pgfqpoint{0.055556in}{0.000000in}}%
\pgfusepath{stroke,fill}%
}%
\begin{pgfscope}%
\pgfsys@transformshift{1.000000in}{2.400000in}%
\pgfsys@useobject{currentmarker}{}%
\end{pgfscope}%
\end{pgfscope}%
\begin{pgfscope}%
\pgfsetbuttcap%
\pgfsetroundjoin%
\definecolor{currentfill}{rgb}{0.000000,0.000000,0.000000}%
\pgfsetfillcolor{currentfill}%
\pgfsetlinewidth{0.501875pt}%
\definecolor{currentstroke}{rgb}{0.000000,0.000000,0.000000}%
\pgfsetstrokecolor{currentstroke}%
\pgfsetdash{}{0pt}%
\pgfsys@defobject{currentmarker}{\pgfqpoint{-0.055556in}{0.000000in}}{\pgfqpoint{-0.000000in}{0.000000in}}{%
\pgfpathmoveto{\pgfqpoint{-0.000000in}{0.000000in}}%
\pgfpathlineto{\pgfqpoint{-0.055556in}{0.000000in}}%
\pgfusepath{stroke,fill}%
}%
\begin{pgfscope}%
\pgfsys@transformshift{7.200000in}{2.400000in}%
\pgfsys@useobject{currentmarker}{}%
\end{pgfscope}%
\end{pgfscope}%
\begin{pgfscope}%
\definecolor{textcolor}{rgb}{0.000000,0.000000,0.000000}%
\pgfsetstrokecolor{textcolor}%
\pgfsetfillcolor{textcolor}%
\pgftext[x=0.944444in,y=2.400000in,right,]{\color{textcolor}\rmfamily\fontsize{10.000000}{12.000000}\selectfont \(\displaystyle {10^{-5}}\)}%
\end{pgfscope}%
\begin{pgfscope}%
\pgfpathrectangle{\pgfqpoint{1.000000in}{0.600000in}}{\pgfqpoint{6.200000in}{4.800000in}}%
\pgfusepath{clip}%
\pgfsetbuttcap%
\pgfsetroundjoin%
\pgfsetlinewidth{0.501875pt}%
\definecolor{currentstroke}{rgb}{0.000000,0.000000,0.000000}%
\pgfsetstrokecolor{currentstroke}%
\pgfsetdash{{1.000000pt}{3.000000pt}}{0.000000pt}%
\pgfpathmoveto{\pgfqpoint{1.000000in}{3.000000in}}%
\pgfpathlineto{\pgfqpoint{7.200000in}{3.000000in}}%
\pgfusepath{stroke}%
\end{pgfscope}%
\begin{pgfscope}%
\pgfsetbuttcap%
\pgfsetroundjoin%
\definecolor{currentfill}{rgb}{0.000000,0.000000,0.000000}%
\pgfsetfillcolor{currentfill}%
\pgfsetlinewidth{0.501875pt}%
\definecolor{currentstroke}{rgb}{0.000000,0.000000,0.000000}%
\pgfsetstrokecolor{currentstroke}%
\pgfsetdash{}{0pt}%
\pgfsys@defobject{currentmarker}{\pgfqpoint{0.000000in}{0.000000in}}{\pgfqpoint{0.055556in}{0.000000in}}{%
\pgfpathmoveto{\pgfqpoint{0.000000in}{0.000000in}}%
\pgfpathlineto{\pgfqpoint{0.055556in}{0.000000in}}%
\pgfusepath{stroke,fill}%
}%
\begin{pgfscope}%
\pgfsys@transformshift{1.000000in}{3.000000in}%
\pgfsys@useobject{currentmarker}{}%
\end{pgfscope}%
\end{pgfscope}%
\begin{pgfscope}%
\pgfsetbuttcap%
\pgfsetroundjoin%
\definecolor{currentfill}{rgb}{0.000000,0.000000,0.000000}%
\pgfsetfillcolor{currentfill}%
\pgfsetlinewidth{0.501875pt}%
\definecolor{currentstroke}{rgb}{0.000000,0.000000,0.000000}%
\pgfsetstrokecolor{currentstroke}%
\pgfsetdash{}{0pt}%
\pgfsys@defobject{currentmarker}{\pgfqpoint{-0.055556in}{0.000000in}}{\pgfqpoint{-0.000000in}{0.000000in}}{%
\pgfpathmoveto{\pgfqpoint{-0.000000in}{0.000000in}}%
\pgfpathlineto{\pgfqpoint{-0.055556in}{0.000000in}}%
\pgfusepath{stroke,fill}%
}%
\begin{pgfscope}%
\pgfsys@transformshift{7.200000in}{3.000000in}%
\pgfsys@useobject{currentmarker}{}%
\end{pgfscope}%
\end{pgfscope}%
\begin{pgfscope}%
\definecolor{textcolor}{rgb}{0.000000,0.000000,0.000000}%
\pgfsetstrokecolor{textcolor}%
\pgfsetfillcolor{textcolor}%
\pgftext[x=0.944444in,y=3.000000in,right,]{\color{textcolor}\rmfamily\fontsize{10.000000}{12.000000}\selectfont \(\displaystyle {10^{-4}}\)}%
\end{pgfscope}%
\begin{pgfscope}%
\pgfpathrectangle{\pgfqpoint{1.000000in}{0.600000in}}{\pgfqpoint{6.200000in}{4.800000in}}%
\pgfusepath{clip}%
\pgfsetbuttcap%
\pgfsetroundjoin%
\pgfsetlinewidth{0.501875pt}%
\definecolor{currentstroke}{rgb}{0.000000,0.000000,0.000000}%
\pgfsetstrokecolor{currentstroke}%
\pgfsetdash{{1.000000pt}{3.000000pt}}{0.000000pt}%
\pgfpathmoveto{\pgfqpoint{1.000000in}{3.600000in}}%
\pgfpathlineto{\pgfqpoint{7.200000in}{3.600000in}}%
\pgfusepath{stroke}%
\end{pgfscope}%
\begin{pgfscope}%
\pgfsetbuttcap%
\pgfsetroundjoin%
\definecolor{currentfill}{rgb}{0.000000,0.000000,0.000000}%
\pgfsetfillcolor{currentfill}%
\pgfsetlinewidth{0.501875pt}%
\definecolor{currentstroke}{rgb}{0.000000,0.000000,0.000000}%
\pgfsetstrokecolor{currentstroke}%
\pgfsetdash{}{0pt}%
\pgfsys@defobject{currentmarker}{\pgfqpoint{0.000000in}{0.000000in}}{\pgfqpoint{0.055556in}{0.000000in}}{%
\pgfpathmoveto{\pgfqpoint{0.000000in}{0.000000in}}%
\pgfpathlineto{\pgfqpoint{0.055556in}{0.000000in}}%
\pgfusepath{stroke,fill}%
}%
\begin{pgfscope}%
\pgfsys@transformshift{1.000000in}{3.600000in}%
\pgfsys@useobject{currentmarker}{}%
\end{pgfscope}%
\end{pgfscope}%
\begin{pgfscope}%
\pgfsetbuttcap%
\pgfsetroundjoin%
\definecolor{currentfill}{rgb}{0.000000,0.000000,0.000000}%
\pgfsetfillcolor{currentfill}%
\pgfsetlinewidth{0.501875pt}%
\definecolor{currentstroke}{rgb}{0.000000,0.000000,0.000000}%
\pgfsetstrokecolor{currentstroke}%
\pgfsetdash{}{0pt}%
\pgfsys@defobject{currentmarker}{\pgfqpoint{-0.055556in}{0.000000in}}{\pgfqpoint{-0.000000in}{0.000000in}}{%
\pgfpathmoveto{\pgfqpoint{-0.000000in}{0.000000in}}%
\pgfpathlineto{\pgfqpoint{-0.055556in}{0.000000in}}%
\pgfusepath{stroke,fill}%
}%
\begin{pgfscope}%
\pgfsys@transformshift{7.200000in}{3.600000in}%
\pgfsys@useobject{currentmarker}{}%
\end{pgfscope}%
\end{pgfscope}%
\begin{pgfscope}%
\definecolor{textcolor}{rgb}{0.000000,0.000000,0.000000}%
\pgfsetstrokecolor{textcolor}%
\pgfsetfillcolor{textcolor}%
\pgftext[x=0.944444in,y=3.600000in,right,]{\color{textcolor}\rmfamily\fontsize{10.000000}{12.000000}\selectfont \(\displaystyle {10^{-3}}\)}%
\end{pgfscope}%
\begin{pgfscope}%
\pgfpathrectangle{\pgfqpoint{1.000000in}{0.600000in}}{\pgfqpoint{6.200000in}{4.800000in}}%
\pgfusepath{clip}%
\pgfsetbuttcap%
\pgfsetroundjoin%
\pgfsetlinewidth{0.501875pt}%
\definecolor{currentstroke}{rgb}{0.000000,0.000000,0.000000}%
\pgfsetstrokecolor{currentstroke}%
\pgfsetdash{{1.000000pt}{3.000000pt}}{0.000000pt}%
\pgfpathmoveto{\pgfqpoint{1.000000in}{4.200000in}}%
\pgfpathlineto{\pgfqpoint{7.200000in}{4.200000in}}%
\pgfusepath{stroke}%
\end{pgfscope}%
\begin{pgfscope}%
\pgfsetbuttcap%
\pgfsetroundjoin%
\definecolor{currentfill}{rgb}{0.000000,0.000000,0.000000}%
\pgfsetfillcolor{currentfill}%
\pgfsetlinewidth{0.501875pt}%
\definecolor{currentstroke}{rgb}{0.000000,0.000000,0.000000}%
\pgfsetstrokecolor{currentstroke}%
\pgfsetdash{}{0pt}%
\pgfsys@defobject{currentmarker}{\pgfqpoint{0.000000in}{0.000000in}}{\pgfqpoint{0.055556in}{0.000000in}}{%
\pgfpathmoveto{\pgfqpoint{0.000000in}{0.000000in}}%
\pgfpathlineto{\pgfqpoint{0.055556in}{0.000000in}}%
\pgfusepath{stroke,fill}%
}%
\begin{pgfscope}%
\pgfsys@transformshift{1.000000in}{4.200000in}%
\pgfsys@useobject{currentmarker}{}%
\end{pgfscope}%
\end{pgfscope}%
\begin{pgfscope}%
\pgfsetbuttcap%
\pgfsetroundjoin%
\definecolor{currentfill}{rgb}{0.000000,0.000000,0.000000}%
\pgfsetfillcolor{currentfill}%
\pgfsetlinewidth{0.501875pt}%
\definecolor{currentstroke}{rgb}{0.000000,0.000000,0.000000}%
\pgfsetstrokecolor{currentstroke}%
\pgfsetdash{}{0pt}%
\pgfsys@defobject{currentmarker}{\pgfqpoint{-0.055556in}{0.000000in}}{\pgfqpoint{-0.000000in}{0.000000in}}{%
\pgfpathmoveto{\pgfqpoint{-0.000000in}{0.000000in}}%
\pgfpathlineto{\pgfqpoint{-0.055556in}{0.000000in}}%
\pgfusepath{stroke,fill}%
}%
\begin{pgfscope}%
\pgfsys@transformshift{7.200000in}{4.200000in}%
\pgfsys@useobject{currentmarker}{}%
\end{pgfscope}%
\end{pgfscope}%
\begin{pgfscope}%
\definecolor{textcolor}{rgb}{0.000000,0.000000,0.000000}%
\pgfsetstrokecolor{textcolor}%
\pgfsetfillcolor{textcolor}%
\pgftext[x=0.944444in,y=4.200000in,right,]{\color{textcolor}\rmfamily\fontsize{10.000000}{12.000000}\selectfont \(\displaystyle {10^{-2}}\)}%
\end{pgfscope}%
\begin{pgfscope}%
\pgfpathrectangle{\pgfqpoint{1.000000in}{0.600000in}}{\pgfqpoint{6.200000in}{4.800000in}}%
\pgfusepath{clip}%
\pgfsetbuttcap%
\pgfsetroundjoin%
\pgfsetlinewidth{0.501875pt}%
\definecolor{currentstroke}{rgb}{0.000000,0.000000,0.000000}%
\pgfsetstrokecolor{currentstroke}%
\pgfsetdash{{1.000000pt}{3.000000pt}}{0.000000pt}%
\pgfpathmoveto{\pgfqpoint{1.000000in}{4.800000in}}%
\pgfpathlineto{\pgfqpoint{7.200000in}{4.800000in}}%
\pgfusepath{stroke}%
\end{pgfscope}%
\begin{pgfscope}%
\pgfsetbuttcap%
\pgfsetroundjoin%
\definecolor{currentfill}{rgb}{0.000000,0.000000,0.000000}%
\pgfsetfillcolor{currentfill}%
\pgfsetlinewidth{0.501875pt}%
\definecolor{currentstroke}{rgb}{0.000000,0.000000,0.000000}%
\pgfsetstrokecolor{currentstroke}%
\pgfsetdash{}{0pt}%
\pgfsys@defobject{currentmarker}{\pgfqpoint{0.000000in}{0.000000in}}{\pgfqpoint{0.055556in}{0.000000in}}{%
\pgfpathmoveto{\pgfqpoint{0.000000in}{0.000000in}}%
\pgfpathlineto{\pgfqpoint{0.055556in}{0.000000in}}%
\pgfusepath{stroke,fill}%
}%
\begin{pgfscope}%
\pgfsys@transformshift{1.000000in}{4.800000in}%
\pgfsys@useobject{currentmarker}{}%
\end{pgfscope}%
\end{pgfscope}%
\begin{pgfscope}%
\pgfsetbuttcap%
\pgfsetroundjoin%
\definecolor{currentfill}{rgb}{0.000000,0.000000,0.000000}%
\pgfsetfillcolor{currentfill}%
\pgfsetlinewidth{0.501875pt}%
\definecolor{currentstroke}{rgb}{0.000000,0.000000,0.000000}%
\pgfsetstrokecolor{currentstroke}%
\pgfsetdash{}{0pt}%
\pgfsys@defobject{currentmarker}{\pgfqpoint{-0.055556in}{0.000000in}}{\pgfqpoint{-0.000000in}{0.000000in}}{%
\pgfpathmoveto{\pgfqpoint{-0.000000in}{0.000000in}}%
\pgfpathlineto{\pgfqpoint{-0.055556in}{0.000000in}}%
\pgfusepath{stroke,fill}%
}%
\begin{pgfscope}%
\pgfsys@transformshift{7.200000in}{4.800000in}%
\pgfsys@useobject{currentmarker}{}%
\end{pgfscope}%
\end{pgfscope}%
\begin{pgfscope}%
\definecolor{textcolor}{rgb}{0.000000,0.000000,0.000000}%
\pgfsetstrokecolor{textcolor}%
\pgfsetfillcolor{textcolor}%
\pgftext[x=0.944444in,y=4.800000in,right,]{\color{textcolor}\rmfamily\fontsize{10.000000}{12.000000}\selectfont \(\displaystyle {10^{-1}}\)}%
\end{pgfscope}%
\begin{pgfscope}%
\pgfpathrectangle{\pgfqpoint{1.000000in}{0.600000in}}{\pgfqpoint{6.200000in}{4.800000in}}%
\pgfusepath{clip}%
\pgfsetbuttcap%
\pgfsetroundjoin%
\pgfsetlinewidth{0.501875pt}%
\definecolor{currentstroke}{rgb}{0.000000,0.000000,0.000000}%
\pgfsetstrokecolor{currentstroke}%
\pgfsetdash{{1.000000pt}{3.000000pt}}{0.000000pt}%
\pgfpathmoveto{\pgfqpoint{1.000000in}{5.400000in}}%
\pgfpathlineto{\pgfqpoint{7.200000in}{5.400000in}}%
\pgfusepath{stroke}%
\end{pgfscope}%
\begin{pgfscope}%
\pgfsetbuttcap%
\pgfsetroundjoin%
\definecolor{currentfill}{rgb}{0.000000,0.000000,0.000000}%
\pgfsetfillcolor{currentfill}%
\pgfsetlinewidth{0.501875pt}%
\definecolor{currentstroke}{rgb}{0.000000,0.000000,0.000000}%
\pgfsetstrokecolor{currentstroke}%
\pgfsetdash{}{0pt}%
\pgfsys@defobject{currentmarker}{\pgfqpoint{0.000000in}{0.000000in}}{\pgfqpoint{0.055556in}{0.000000in}}{%
\pgfpathmoveto{\pgfqpoint{0.000000in}{0.000000in}}%
\pgfpathlineto{\pgfqpoint{0.055556in}{0.000000in}}%
\pgfusepath{stroke,fill}%
}%
\begin{pgfscope}%
\pgfsys@transformshift{1.000000in}{5.400000in}%
\pgfsys@useobject{currentmarker}{}%
\end{pgfscope}%
\end{pgfscope}%
\begin{pgfscope}%
\pgfsetbuttcap%
\pgfsetroundjoin%
\definecolor{currentfill}{rgb}{0.000000,0.000000,0.000000}%
\pgfsetfillcolor{currentfill}%
\pgfsetlinewidth{0.501875pt}%
\definecolor{currentstroke}{rgb}{0.000000,0.000000,0.000000}%
\pgfsetstrokecolor{currentstroke}%
\pgfsetdash{}{0pt}%
\pgfsys@defobject{currentmarker}{\pgfqpoint{-0.055556in}{0.000000in}}{\pgfqpoint{-0.000000in}{0.000000in}}{%
\pgfpathmoveto{\pgfqpoint{-0.000000in}{0.000000in}}%
\pgfpathlineto{\pgfqpoint{-0.055556in}{0.000000in}}%
\pgfusepath{stroke,fill}%
}%
\begin{pgfscope}%
\pgfsys@transformshift{7.200000in}{5.400000in}%
\pgfsys@useobject{currentmarker}{}%
\end{pgfscope}%
\end{pgfscope}%
\begin{pgfscope}%
\definecolor{textcolor}{rgb}{0.000000,0.000000,0.000000}%
\pgfsetstrokecolor{textcolor}%
\pgfsetfillcolor{textcolor}%
\pgftext[x=0.944444in,y=5.400000in,right,]{\color{textcolor}\rmfamily\fontsize{10.000000}{12.000000}\selectfont \(\displaystyle {10^{0}}\)}%
\end{pgfscope}%
\begin{pgfscope}%
\pgfsetbuttcap%
\pgfsetroundjoin%
\definecolor{currentfill}{rgb}{0.000000,0.000000,0.000000}%
\pgfsetfillcolor{currentfill}%
\pgfsetlinewidth{0.501875pt}%
\definecolor{currentstroke}{rgb}{0.000000,0.000000,0.000000}%
\pgfsetstrokecolor{currentstroke}%
\pgfsetdash{}{0pt}%
\pgfsys@defobject{currentmarker}{\pgfqpoint{0.000000in}{0.000000in}}{\pgfqpoint{0.027778in}{0.000000in}}{%
\pgfpathmoveto{\pgfqpoint{0.000000in}{0.000000in}}%
\pgfpathlineto{\pgfqpoint{0.027778in}{0.000000in}}%
\pgfusepath{stroke,fill}%
}%
\begin{pgfscope}%
\pgfsys@transformshift{1.000000in}{0.780618in}%
\pgfsys@useobject{currentmarker}{}%
\end{pgfscope}%
\end{pgfscope}%
\begin{pgfscope}%
\pgfsetbuttcap%
\pgfsetroundjoin%
\definecolor{currentfill}{rgb}{0.000000,0.000000,0.000000}%
\pgfsetfillcolor{currentfill}%
\pgfsetlinewidth{0.501875pt}%
\definecolor{currentstroke}{rgb}{0.000000,0.000000,0.000000}%
\pgfsetstrokecolor{currentstroke}%
\pgfsetdash{}{0pt}%
\pgfsys@defobject{currentmarker}{\pgfqpoint{-0.027778in}{0.000000in}}{\pgfqpoint{-0.000000in}{0.000000in}}{%
\pgfpathmoveto{\pgfqpoint{-0.000000in}{0.000000in}}%
\pgfpathlineto{\pgfqpoint{-0.027778in}{0.000000in}}%
\pgfusepath{stroke,fill}%
}%
\begin{pgfscope}%
\pgfsys@transformshift{7.200000in}{0.780618in}%
\pgfsys@useobject{currentmarker}{}%
\end{pgfscope}%
\end{pgfscope}%
\begin{pgfscope}%
\pgfsetbuttcap%
\pgfsetroundjoin%
\definecolor{currentfill}{rgb}{0.000000,0.000000,0.000000}%
\pgfsetfillcolor{currentfill}%
\pgfsetlinewidth{0.501875pt}%
\definecolor{currentstroke}{rgb}{0.000000,0.000000,0.000000}%
\pgfsetstrokecolor{currentstroke}%
\pgfsetdash{}{0pt}%
\pgfsys@defobject{currentmarker}{\pgfqpoint{0.000000in}{0.000000in}}{\pgfqpoint{0.027778in}{0.000000in}}{%
\pgfpathmoveto{\pgfqpoint{0.000000in}{0.000000in}}%
\pgfpathlineto{\pgfqpoint{0.027778in}{0.000000in}}%
\pgfusepath{stroke,fill}%
}%
\begin{pgfscope}%
\pgfsys@transformshift{1.000000in}{0.886273in}%
\pgfsys@useobject{currentmarker}{}%
\end{pgfscope}%
\end{pgfscope}%
\begin{pgfscope}%
\pgfsetbuttcap%
\pgfsetroundjoin%
\definecolor{currentfill}{rgb}{0.000000,0.000000,0.000000}%
\pgfsetfillcolor{currentfill}%
\pgfsetlinewidth{0.501875pt}%
\definecolor{currentstroke}{rgb}{0.000000,0.000000,0.000000}%
\pgfsetstrokecolor{currentstroke}%
\pgfsetdash{}{0pt}%
\pgfsys@defobject{currentmarker}{\pgfqpoint{-0.027778in}{0.000000in}}{\pgfqpoint{-0.000000in}{0.000000in}}{%
\pgfpathmoveto{\pgfqpoint{-0.000000in}{0.000000in}}%
\pgfpathlineto{\pgfqpoint{-0.027778in}{0.000000in}}%
\pgfusepath{stroke,fill}%
}%
\begin{pgfscope}%
\pgfsys@transformshift{7.200000in}{0.886273in}%
\pgfsys@useobject{currentmarker}{}%
\end{pgfscope}%
\end{pgfscope}%
\begin{pgfscope}%
\pgfsetbuttcap%
\pgfsetroundjoin%
\definecolor{currentfill}{rgb}{0.000000,0.000000,0.000000}%
\pgfsetfillcolor{currentfill}%
\pgfsetlinewidth{0.501875pt}%
\definecolor{currentstroke}{rgb}{0.000000,0.000000,0.000000}%
\pgfsetstrokecolor{currentstroke}%
\pgfsetdash{}{0pt}%
\pgfsys@defobject{currentmarker}{\pgfqpoint{0.000000in}{0.000000in}}{\pgfqpoint{0.027778in}{0.000000in}}{%
\pgfpathmoveto{\pgfqpoint{0.000000in}{0.000000in}}%
\pgfpathlineto{\pgfqpoint{0.027778in}{0.000000in}}%
\pgfusepath{stroke,fill}%
}%
\begin{pgfscope}%
\pgfsys@transformshift{1.000000in}{0.961236in}%
\pgfsys@useobject{currentmarker}{}%
\end{pgfscope}%
\end{pgfscope}%
\begin{pgfscope}%
\pgfsetbuttcap%
\pgfsetroundjoin%
\definecolor{currentfill}{rgb}{0.000000,0.000000,0.000000}%
\pgfsetfillcolor{currentfill}%
\pgfsetlinewidth{0.501875pt}%
\definecolor{currentstroke}{rgb}{0.000000,0.000000,0.000000}%
\pgfsetstrokecolor{currentstroke}%
\pgfsetdash{}{0pt}%
\pgfsys@defobject{currentmarker}{\pgfqpoint{-0.027778in}{0.000000in}}{\pgfqpoint{-0.000000in}{0.000000in}}{%
\pgfpathmoveto{\pgfqpoint{-0.000000in}{0.000000in}}%
\pgfpathlineto{\pgfqpoint{-0.027778in}{0.000000in}}%
\pgfusepath{stroke,fill}%
}%
\begin{pgfscope}%
\pgfsys@transformshift{7.200000in}{0.961236in}%
\pgfsys@useobject{currentmarker}{}%
\end{pgfscope}%
\end{pgfscope}%
\begin{pgfscope}%
\pgfsetbuttcap%
\pgfsetroundjoin%
\definecolor{currentfill}{rgb}{0.000000,0.000000,0.000000}%
\pgfsetfillcolor{currentfill}%
\pgfsetlinewidth{0.501875pt}%
\definecolor{currentstroke}{rgb}{0.000000,0.000000,0.000000}%
\pgfsetstrokecolor{currentstroke}%
\pgfsetdash{}{0pt}%
\pgfsys@defobject{currentmarker}{\pgfqpoint{0.000000in}{0.000000in}}{\pgfqpoint{0.027778in}{0.000000in}}{%
\pgfpathmoveto{\pgfqpoint{0.000000in}{0.000000in}}%
\pgfpathlineto{\pgfqpoint{0.027778in}{0.000000in}}%
\pgfusepath{stroke,fill}%
}%
\begin{pgfscope}%
\pgfsys@transformshift{1.000000in}{1.019382in}%
\pgfsys@useobject{currentmarker}{}%
\end{pgfscope}%
\end{pgfscope}%
\begin{pgfscope}%
\pgfsetbuttcap%
\pgfsetroundjoin%
\definecolor{currentfill}{rgb}{0.000000,0.000000,0.000000}%
\pgfsetfillcolor{currentfill}%
\pgfsetlinewidth{0.501875pt}%
\definecolor{currentstroke}{rgb}{0.000000,0.000000,0.000000}%
\pgfsetstrokecolor{currentstroke}%
\pgfsetdash{}{0pt}%
\pgfsys@defobject{currentmarker}{\pgfqpoint{-0.027778in}{0.000000in}}{\pgfqpoint{-0.000000in}{0.000000in}}{%
\pgfpathmoveto{\pgfqpoint{-0.000000in}{0.000000in}}%
\pgfpathlineto{\pgfqpoint{-0.027778in}{0.000000in}}%
\pgfusepath{stroke,fill}%
}%
\begin{pgfscope}%
\pgfsys@transformshift{7.200000in}{1.019382in}%
\pgfsys@useobject{currentmarker}{}%
\end{pgfscope}%
\end{pgfscope}%
\begin{pgfscope}%
\pgfsetbuttcap%
\pgfsetroundjoin%
\definecolor{currentfill}{rgb}{0.000000,0.000000,0.000000}%
\pgfsetfillcolor{currentfill}%
\pgfsetlinewidth{0.501875pt}%
\definecolor{currentstroke}{rgb}{0.000000,0.000000,0.000000}%
\pgfsetstrokecolor{currentstroke}%
\pgfsetdash{}{0pt}%
\pgfsys@defobject{currentmarker}{\pgfqpoint{0.000000in}{0.000000in}}{\pgfqpoint{0.027778in}{0.000000in}}{%
\pgfpathmoveto{\pgfqpoint{0.000000in}{0.000000in}}%
\pgfpathlineto{\pgfqpoint{0.027778in}{0.000000in}}%
\pgfusepath{stroke,fill}%
}%
\begin{pgfscope}%
\pgfsys@transformshift{1.000000in}{1.066891in}%
\pgfsys@useobject{currentmarker}{}%
\end{pgfscope}%
\end{pgfscope}%
\begin{pgfscope}%
\pgfsetbuttcap%
\pgfsetroundjoin%
\definecolor{currentfill}{rgb}{0.000000,0.000000,0.000000}%
\pgfsetfillcolor{currentfill}%
\pgfsetlinewidth{0.501875pt}%
\definecolor{currentstroke}{rgb}{0.000000,0.000000,0.000000}%
\pgfsetstrokecolor{currentstroke}%
\pgfsetdash{}{0pt}%
\pgfsys@defobject{currentmarker}{\pgfqpoint{-0.027778in}{0.000000in}}{\pgfqpoint{-0.000000in}{0.000000in}}{%
\pgfpathmoveto{\pgfqpoint{-0.000000in}{0.000000in}}%
\pgfpathlineto{\pgfqpoint{-0.027778in}{0.000000in}}%
\pgfusepath{stroke,fill}%
}%
\begin{pgfscope}%
\pgfsys@transformshift{7.200000in}{1.066891in}%
\pgfsys@useobject{currentmarker}{}%
\end{pgfscope}%
\end{pgfscope}%
\begin{pgfscope}%
\pgfsetbuttcap%
\pgfsetroundjoin%
\definecolor{currentfill}{rgb}{0.000000,0.000000,0.000000}%
\pgfsetfillcolor{currentfill}%
\pgfsetlinewidth{0.501875pt}%
\definecolor{currentstroke}{rgb}{0.000000,0.000000,0.000000}%
\pgfsetstrokecolor{currentstroke}%
\pgfsetdash{}{0pt}%
\pgfsys@defobject{currentmarker}{\pgfqpoint{0.000000in}{0.000000in}}{\pgfqpoint{0.027778in}{0.000000in}}{%
\pgfpathmoveto{\pgfqpoint{0.000000in}{0.000000in}}%
\pgfpathlineto{\pgfqpoint{0.027778in}{0.000000in}}%
\pgfusepath{stroke,fill}%
}%
\begin{pgfscope}%
\pgfsys@transformshift{1.000000in}{1.107059in}%
\pgfsys@useobject{currentmarker}{}%
\end{pgfscope}%
\end{pgfscope}%
\begin{pgfscope}%
\pgfsetbuttcap%
\pgfsetroundjoin%
\definecolor{currentfill}{rgb}{0.000000,0.000000,0.000000}%
\pgfsetfillcolor{currentfill}%
\pgfsetlinewidth{0.501875pt}%
\definecolor{currentstroke}{rgb}{0.000000,0.000000,0.000000}%
\pgfsetstrokecolor{currentstroke}%
\pgfsetdash{}{0pt}%
\pgfsys@defobject{currentmarker}{\pgfqpoint{-0.027778in}{0.000000in}}{\pgfqpoint{-0.000000in}{0.000000in}}{%
\pgfpathmoveto{\pgfqpoint{-0.000000in}{0.000000in}}%
\pgfpathlineto{\pgfqpoint{-0.027778in}{0.000000in}}%
\pgfusepath{stroke,fill}%
}%
\begin{pgfscope}%
\pgfsys@transformshift{7.200000in}{1.107059in}%
\pgfsys@useobject{currentmarker}{}%
\end{pgfscope}%
\end{pgfscope}%
\begin{pgfscope}%
\pgfsetbuttcap%
\pgfsetroundjoin%
\definecolor{currentfill}{rgb}{0.000000,0.000000,0.000000}%
\pgfsetfillcolor{currentfill}%
\pgfsetlinewidth{0.501875pt}%
\definecolor{currentstroke}{rgb}{0.000000,0.000000,0.000000}%
\pgfsetstrokecolor{currentstroke}%
\pgfsetdash{}{0pt}%
\pgfsys@defobject{currentmarker}{\pgfqpoint{0.000000in}{0.000000in}}{\pgfqpoint{0.027778in}{0.000000in}}{%
\pgfpathmoveto{\pgfqpoint{0.000000in}{0.000000in}}%
\pgfpathlineto{\pgfqpoint{0.027778in}{0.000000in}}%
\pgfusepath{stroke,fill}%
}%
\begin{pgfscope}%
\pgfsys@transformshift{1.000000in}{1.141854in}%
\pgfsys@useobject{currentmarker}{}%
\end{pgfscope}%
\end{pgfscope}%
\begin{pgfscope}%
\pgfsetbuttcap%
\pgfsetroundjoin%
\definecolor{currentfill}{rgb}{0.000000,0.000000,0.000000}%
\pgfsetfillcolor{currentfill}%
\pgfsetlinewidth{0.501875pt}%
\definecolor{currentstroke}{rgb}{0.000000,0.000000,0.000000}%
\pgfsetstrokecolor{currentstroke}%
\pgfsetdash{}{0pt}%
\pgfsys@defobject{currentmarker}{\pgfqpoint{-0.027778in}{0.000000in}}{\pgfqpoint{-0.000000in}{0.000000in}}{%
\pgfpathmoveto{\pgfqpoint{-0.000000in}{0.000000in}}%
\pgfpathlineto{\pgfqpoint{-0.027778in}{0.000000in}}%
\pgfusepath{stroke,fill}%
}%
\begin{pgfscope}%
\pgfsys@transformshift{7.200000in}{1.141854in}%
\pgfsys@useobject{currentmarker}{}%
\end{pgfscope}%
\end{pgfscope}%
\begin{pgfscope}%
\pgfsetbuttcap%
\pgfsetroundjoin%
\definecolor{currentfill}{rgb}{0.000000,0.000000,0.000000}%
\pgfsetfillcolor{currentfill}%
\pgfsetlinewidth{0.501875pt}%
\definecolor{currentstroke}{rgb}{0.000000,0.000000,0.000000}%
\pgfsetstrokecolor{currentstroke}%
\pgfsetdash{}{0pt}%
\pgfsys@defobject{currentmarker}{\pgfqpoint{0.000000in}{0.000000in}}{\pgfqpoint{0.027778in}{0.000000in}}{%
\pgfpathmoveto{\pgfqpoint{0.000000in}{0.000000in}}%
\pgfpathlineto{\pgfqpoint{0.027778in}{0.000000in}}%
\pgfusepath{stroke,fill}%
}%
\begin{pgfscope}%
\pgfsys@transformshift{1.000000in}{1.172546in}%
\pgfsys@useobject{currentmarker}{}%
\end{pgfscope}%
\end{pgfscope}%
\begin{pgfscope}%
\pgfsetbuttcap%
\pgfsetroundjoin%
\definecolor{currentfill}{rgb}{0.000000,0.000000,0.000000}%
\pgfsetfillcolor{currentfill}%
\pgfsetlinewidth{0.501875pt}%
\definecolor{currentstroke}{rgb}{0.000000,0.000000,0.000000}%
\pgfsetstrokecolor{currentstroke}%
\pgfsetdash{}{0pt}%
\pgfsys@defobject{currentmarker}{\pgfqpoint{-0.027778in}{0.000000in}}{\pgfqpoint{-0.000000in}{0.000000in}}{%
\pgfpathmoveto{\pgfqpoint{-0.000000in}{0.000000in}}%
\pgfpathlineto{\pgfqpoint{-0.027778in}{0.000000in}}%
\pgfusepath{stroke,fill}%
}%
\begin{pgfscope}%
\pgfsys@transformshift{7.200000in}{1.172546in}%
\pgfsys@useobject{currentmarker}{}%
\end{pgfscope}%
\end{pgfscope}%
\begin{pgfscope}%
\pgfsetbuttcap%
\pgfsetroundjoin%
\definecolor{currentfill}{rgb}{0.000000,0.000000,0.000000}%
\pgfsetfillcolor{currentfill}%
\pgfsetlinewidth{0.501875pt}%
\definecolor{currentstroke}{rgb}{0.000000,0.000000,0.000000}%
\pgfsetstrokecolor{currentstroke}%
\pgfsetdash{}{0pt}%
\pgfsys@defobject{currentmarker}{\pgfqpoint{0.000000in}{0.000000in}}{\pgfqpoint{0.027778in}{0.000000in}}{%
\pgfpathmoveto{\pgfqpoint{0.000000in}{0.000000in}}%
\pgfpathlineto{\pgfqpoint{0.027778in}{0.000000in}}%
\pgfusepath{stroke,fill}%
}%
\begin{pgfscope}%
\pgfsys@transformshift{1.000000in}{1.380618in}%
\pgfsys@useobject{currentmarker}{}%
\end{pgfscope}%
\end{pgfscope}%
\begin{pgfscope}%
\pgfsetbuttcap%
\pgfsetroundjoin%
\definecolor{currentfill}{rgb}{0.000000,0.000000,0.000000}%
\pgfsetfillcolor{currentfill}%
\pgfsetlinewidth{0.501875pt}%
\definecolor{currentstroke}{rgb}{0.000000,0.000000,0.000000}%
\pgfsetstrokecolor{currentstroke}%
\pgfsetdash{}{0pt}%
\pgfsys@defobject{currentmarker}{\pgfqpoint{-0.027778in}{0.000000in}}{\pgfqpoint{-0.000000in}{0.000000in}}{%
\pgfpathmoveto{\pgfqpoint{-0.000000in}{0.000000in}}%
\pgfpathlineto{\pgfqpoint{-0.027778in}{0.000000in}}%
\pgfusepath{stroke,fill}%
}%
\begin{pgfscope}%
\pgfsys@transformshift{7.200000in}{1.380618in}%
\pgfsys@useobject{currentmarker}{}%
\end{pgfscope}%
\end{pgfscope}%
\begin{pgfscope}%
\pgfsetbuttcap%
\pgfsetroundjoin%
\definecolor{currentfill}{rgb}{0.000000,0.000000,0.000000}%
\pgfsetfillcolor{currentfill}%
\pgfsetlinewidth{0.501875pt}%
\definecolor{currentstroke}{rgb}{0.000000,0.000000,0.000000}%
\pgfsetstrokecolor{currentstroke}%
\pgfsetdash{}{0pt}%
\pgfsys@defobject{currentmarker}{\pgfqpoint{0.000000in}{0.000000in}}{\pgfqpoint{0.027778in}{0.000000in}}{%
\pgfpathmoveto{\pgfqpoint{0.000000in}{0.000000in}}%
\pgfpathlineto{\pgfqpoint{0.027778in}{0.000000in}}%
\pgfusepath{stroke,fill}%
}%
\begin{pgfscope}%
\pgfsys@transformshift{1.000000in}{1.486273in}%
\pgfsys@useobject{currentmarker}{}%
\end{pgfscope}%
\end{pgfscope}%
\begin{pgfscope}%
\pgfsetbuttcap%
\pgfsetroundjoin%
\definecolor{currentfill}{rgb}{0.000000,0.000000,0.000000}%
\pgfsetfillcolor{currentfill}%
\pgfsetlinewidth{0.501875pt}%
\definecolor{currentstroke}{rgb}{0.000000,0.000000,0.000000}%
\pgfsetstrokecolor{currentstroke}%
\pgfsetdash{}{0pt}%
\pgfsys@defobject{currentmarker}{\pgfqpoint{-0.027778in}{0.000000in}}{\pgfqpoint{-0.000000in}{0.000000in}}{%
\pgfpathmoveto{\pgfqpoint{-0.000000in}{0.000000in}}%
\pgfpathlineto{\pgfqpoint{-0.027778in}{0.000000in}}%
\pgfusepath{stroke,fill}%
}%
\begin{pgfscope}%
\pgfsys@transformshift{7.200000in}{1.486273in}%
\pgfsys@useobject{currentmarker}{}%
\end{pgfscope}%
\end{pgfscope}%
\begin{pgfscope}%
\pgfsetbuttcap%
\pgfsetroundjoin%
\definecolor{currentfill}{rgb}{0.000000,0.000000,0.000000}%
\pgfsetfillcolor{currentfill}%
\pgfsetlinewidth{0.501875pt}%
\definecolor{currentstroke}{rgb}{0.000000,0.000000,0.000000}%
\pgfsetstrokecolor{currentstroke}%
\pgfsetdash{}{0pt}%
\pgfsys@defobject{currentmarker}{\pgfqpoint{0.000000in}{0.000000in}}{\pgfqpoint{0.027778in}{0.000000in}}{%
\pgfpathmoveto{\pgfqpoint{0.000000in}{0.000000in}}%
\pgfpathlineto{\pgfqpoint{0.027778in}{0.000000in}}%
\pgfusepath{stroke,fill}%
}%
\begin{pgfscope}%
\pgfsys@transformshift{1.000000in}{1.561236in}%
\pgfsys@useobject{currentmarker}{}%
\end{pgfscope}%
\end{pgfscope}%
\begin{pgfscope}%
\pgfsetbuttcap%
\pgfsetroundjoin%
\definecolor{currentfill}{rgb}{0.000000,0.000000,0.000000}%
\pgfsetfillcolor{currentfill}%
\pgfsetlinewidth{0.501875pt}%
\definecolor{currentstroke}{rgb}{0.000000,0.000000,0.000000}%
\pgfsetstrokecolor{currentstroke}%
\pgfsetdash{}{0pt}%
\pgfsys@defobject{currentmarker}{\pgfqpoint{-0.027778in}{0.000000in}}{\pgfqpoint{-0.000000in}{0.000000in}}{%
\pgfpathmoveto{\pgfqpoint{-0.000000in}{0.000000in}}%
\pgfpathlineto{\pgfqpoint{-0.027778in}{0.000000in}}%
\pgfusepath{stroke,fill}%
}%
\begin{pgfscope}%
\pgfsys@transformshift{7.200000in}{1.561236in}%
\pgfsys@useobject{currentmarker}{}%
\end{pgfscope}%
\end{pgfscope}%
\begin{pgfscope}%
\pgfsetbuttcap%
\pgfsetroundjoin%
\definecolor{currentfill}{rgb}{0.000000,0.000000,0.000000}%
\pgfsetfillcolor{currentfill}%
\pgfsetlinewidth{0.501875pt}%
\definecolor{currentstroke}{rgb}{0.000000,0.000000,0.000000}%
\pgfsetstrokecolor{currentstroke}%
\pgfsetdash{}{0pt}%
\pgfsys@defobject{currentmarker}{\pgfqpoint{0.000000in}{0.000000in}}{\pgfqpoint{0.027778in}{0.000000in}}{%
\pgfpathmoveto{\pgfqpoint{0.000000in}{0.000000in}}%
\pgfpathlineto{\pgfqpoint{0.027778in}{0.000000in}}%
\pgfusepath{stroke,fill}%
}%
\begin{pgfscope}%
\pgfsys@transformshift{1.000000in}{1.619382in}%
\pgfsys@useobject{currentmarker}{}%
\end{pgfscope}%
\end{pgfscope}%
\begin{pgfscope}%
\pgfsetbuttcap%
\pgfsetroundjoin%
\definecolor{currentfill}{rgb}{0.000000,0.000000,0.000000}%
\pgfsetfillcolor{currentfill}%
\pgfsetlinewidth{0.501875pt}%
\definecolor{currentstroke}{rgb}{0.000000,0.000000,0.000000}%
\pgfsetstrokecolor{currentstroke}%
\pgfsetdash{}{0pt}%
\pgfsys@defobject{currentmarker}{\pgfqpoint{-0.027778in}{0.000000in}}{\pgfqpoint{-0.000000in}{0.000000in}}{%
\pgfpathmoveto{\pgfqpoint{-0.000000in}{0.000000in}}%
\pgfpathlineto{\pgfqpoint{-0.027778in}{0.000000in}}%
\pgfusepath{stroke,fill}%
}%
\begin{pgfscope}%
\pgfsys@transformshift{7.200000in}{1.619382in}%
\pgfsys@useobject{currentmarker}{}%
\end{pgfscope}%
\end{pgfscope}%
\begin{pgfscope}%
\pgfsetbuttcap%
\pgfsetroundjoin%
\definecolor{currentfill}{rgb}{0.000000,0.000000,0.000000}%
\pgfsetfillcolor{currentfill}%
\pgfsetlinewidth{0.501875pt}%
\definecolor{currentstroke}{rgb}{0.000000,0.000000,0.000000}%
\pgfsetstrokecolor{currentstroke}%
\pgfsetdash{}{0pt}%
\pgfsys@defobject{currentmarker}{\pgfqpoint{0.000000in}{0.000000in}}{\pgfqpoint{0.027778in}{0.000000in}}{%
\pgfpathmoveto{\pgfqpoint{0.000000in}{0.000000in}}%
\pgfpathlineto{\pgfqpoint{0.027778in}{0.000000in}}%
\pgfusepath{stroke,fill}%
}%
\begin{pgfscope}%
\pgfsys@transformshift{1.000000in}{1.666891in}%
\pgfsys@useobject{currentmarker}{}%
\end{pgfscope}%
\end{pgfscope}%
\begin{pgfscope}%
\pgfsetbuttcap%
\pgfsetroundjoin%
\definecolor{currentfill}{rgb}{0.000000,0.000000,0.000000}%
\pgfsetfillcolor{currentfill}%
\pgfsetlinewidth{0.501875pt}%
\definecolor{currentstroke}{rgb}{0.000000,0.000000,0.000000}%
\pgfsetstrokecolor{currentstroke}%
\pgfsetdash{}{0pt}%
\pgfsys@defobject{currentmarker}{\pgfqpoint{-0.027778in}{0.000000in}}{\pgfqpoint{-0.000000in}{0.000000in}}{%
\pgfpathmoveto{\pgfqpoint{-0.000000in}{0.000000in}}%
\pgfpathlineto{\pgfqpoint{-0.027778in}{0.000000in}}%
\pgfusepath{stroke,fill}%
}%
\begin{pgfscope}%
\pgfsys@transformshift{7.200000in}{1.666891in}%
\pgfsys@useobject{currentmarker}{}%
\end{pgfscope}%
\end{pgfscope}%
\begin{pgfscope}%
\pgfsetbuttcap%
\pgfsetroundjoin%
\definecolor{currentfill}{rgb}{0.000000,0.000000,0.000000}%
\pgfsetfillcolor{currentfill}%
\pgfsetlinewidth{0.501875pt}%
\definecolor{currentstroke}{rgb}{0.000000,0.000000,0.000000}%
\pgfsetstrokecolor{currentstroke}%
\pgfsetdash{}{0pt}%
\pgfsys@defobject{currentmarker}{\pgfqpoint{0.000000in}{0.000000in}}{\pgfqpoint{0.027778in}{0.000000in}}{%
\pgfpathmoveto{\pgfqpoint{0.000000in}{0.000000in}}%
\pgfpathlineto{\pgfqpoint{0.027778in}{0.000000in}}%
\pgfusepath{stroke,fill}%
}%
\begin{pgfscope}%
\pgfsys@transformshift{1.000000in}{1.707059in}%
\pgfsys@useobject{currentmarker}{}%
\end{pgfscope}%
\end{pgfscope}%
\begin{pgfscope}%
\pgfsetbuttcap%
\pgfsetroundjoin%
\definecolor{currentfill}{rgb}{0.000000,0.000000,0.000000}%
\pgfsetfillcolor{currentfill}%
\pgfsetlinewidth{0.501875pt}%
\definecolor{currentstroke}{rgb}{0.000000,0.000000,0.000000}%
\pgfsetstrokecolor{currentstroke}%
\pgfsetdash{}{0pt}%
\pgfsys@defobject{currentmarker}{\pgfqpoint{-0.027778in}{0.000000in}}{\pgfqpoint{-0.000000in}{0.000000in}}{%
\pgfpathmoveto{\pgfqpoint{-0.000000in}{0.000000in}}%
\pgfpathlineto{\pgfqpoint{-0.027778in}{0.000000in}}%
\pgfusepath{stroke,fill}%
}%
\begin{pgfscope}%
\pgfsys@transformshift{7.200000in}{1.707059in}%
\pgfsys@useobject{currentmarker}{}%
\end{pgfscope}%
\end{pgfscope}%
\begin{pgfscope}%
\pgfsetbuttcap%
\pgfsetroundjoin%
\definecolor{currentfill}{rgb}{0.000000,0.000000,0.000000}%
\pgfsetfillcolor{currentfill}%
\pgfsetlinewidth{0.501875pt}%
\definecolor{currentstroke}{rgb}{0.000000,0.000000,0.000000}%
\pgfsetstrokecolor{currentstroke}%
\pgfsetdash{}{0pt}%
\pgfsys@defobject{currentmarker}{\pgfqpoint{0.000000in}{0.000000in}}{\pgfqpoint{0.027778in}{0.000000in}}{%
\pgfpathmoveto{\pgfqpoint{0.000000in}{0.000000in}}%
\pgfpathlineto{\pgfqpoint{0.027778in}{0.000000in}}%
\pgfusepath{stroke,fill}%
}%
\begin{pgfscope}%
\pgfsys@transformshift{1.000000in}{1.741854in}%
\pgfsys@useobject{currentmarker}{}%
\end{pgfscope}%
\end{pgfscope}%
\begin{pgfscope}%
\pgfsetbuttcap%
\pgfsetroundjoin%
\definecolor{currentfill}{rgb}{0.000000,0.000000,0.000000}%
\pgfsetfillcolor{currentfill}%
\pgfsetlinewidth{0.501875pt}%
\definecolor{currentstroke}{rgb}{0.000000,0.000000,0.000000}%
\pgfsetstrokecolor{currentstroke}%
\pgfsetdash{}{0pt}%
\pgfsys@defobject{currentmarker}{\pgfqpoint{-0.027778in}{0.000000in}}{\pgfqpoint{-0.000000in}{0.000000in}}{%
\pgfpathmoveto{\pgfqpoint{-0.000000in}{0.000000in}}%
\pgfpathlineto{\pgfqpoint{-0.027778in}{0.000000in}}%
\pgfusepath{stroke,fill}%
}%
\begin{pgfscope}%
\pgfsys@transformshift{7.200000in}{1.741854in}%
\pgfsys@useobject{currentmarker}{}%
\end{pgfscope}%
\end{pgfscope}%
\begin{pgfscope}%
\pgfsetbuttcap%
\pgfsetroundjoin%
\definecolor{currentfill}{rgb}{0.000000,0.000000,0.000000}%
\pgfsetfillcolor{currentfill}%
\pgfsetlinewidth{0.501875pt}%
\definecolor{currentstroke}{rgb}{0.000000,0.000000,0.000000}%
\pgfsetstrokecolor{currentstroke}%
\pgfsetdash{}{0pt}%
\pgfsys@defobject{currentmarker}{\pgfqpoint{0.000000in}{0.000000in}}{\pgfqpoint{0.027778in}{0.000000in}}{%
\pgfpathmoveto{\pgfqpoint{0.000000in}{0.000000in}}%
\pgfpathlineto{\pgfqpoint{0.027778in}{0.000000in}}%
\pgfusepath{stroke,fill}%
}%
\begin{pgfscope}%
\pgfsys@transformshift{1.000000in}{1.772546in}%
\pgfsys@useobject{currentmarker}{}%
\end{pgfscope}%
\end{pgfscope}%
\begin{pgfscope}%
\pgfsetbuttcap%
\pgfsetroundjoin%
\definecolor{currentfill}{rgb}{0.000000,0.000000,0.000000}%
\pgfsetfillcolor{currentfill}%
\pgfsetlinewidth{0.501875pt}%
\definecolor{currentstroke}{rgb}{0.000000,0.000000,0.000000}%
\pgfsetstrokecolor{currentstroke}%
\pgfsetdash{}{0pt}%
\pgfsys@defobject{currentmarker}{\pgfqpoint{-0.027778in}{0.000000in}}{\pgfqpoint{-0.000000in}{0.000000in}}{%
\pgfpathmoveto{\pgfqpoint{-0.000000in}{0.000000in}}%
\pgfpathlineto{\pgfqpoint{-0.027778in}{0.000000in}}%
\pgfusepath{stroke,fill}%
}%
\begin{pgfscope}%
\pgfsys@transformshift{7.200000in}{1.772546in}%
\pgfsys@useobject{currentmarker}{}%
\end{pgfscope}%
\end{pgfscope}%
\begin{pgfscope}%
\pgfsetbuttcap%
\pgfsetroundjoin%
\definecolor{currentfill}{rgb}{0.000000,0.000000,0.000000}%
\pgfsetfillcolor{currentfill}%
\pgfsetlinewidth{0.501875pt}%
\definecolor{currentstroke}{rgb}{0.000000,0.000000,0.000000}%
\pgfsetstrokecolor{currentstroke}%
\pgfsetdash{}{0pt}%
\pgfsys@defobject{currentmarker}{\pgfqpoint{0.000000in}{0.000000in}}{\pgfqpoint{0.027778in}{0.000000in}}{%
\pgfpathmoveto{\pgfqpoint{0.000000in}{0.000000in}}%
\pgfpathlineto{\pgfqpoint{0.027778in}{0.000000in}}%
\pgfusepath{stroke,fill}%
}%
\begin{pgfscope}%
\pgfsys@transformshift{1.000000in}{1.980618in}%
\pgfsys@useobject{currentmarker}{}%
\end{pgfscope}%
\end{pgfscope}%
\begin{pgfscope}%
\pgfsetbuttcap%
\pgfsetroundjoin%
\definecolor{currentfill}{rgb}{0.000000,0.000000,0.000000}%
\pgfsetfillcolor{currentfill}%
\pgfsetlinewidth{0.501875pt}%
\definecolor{currentstroke}{rgb}{0.000000,0.000000,0.000000}%
\pgfsetstrokecolor{currentstroke}%
\pgfsetdash{}{0pt}%
\pgfsys@defobject{currentmarker}{\pgfqpoint{-0.027778in}{0.000000in}}{\pgfqpoint{-0.000000in}{0.000000in}}{%
\pgfpathmoveto{\pgfqpoint{-0.000000in}{0.000000in}}%
\pgfpathlineto{\pgfqpoint{-0.027778in}{0.000000in}}%
\pgfusepath{stroke,fill}%
}%
\begin{pgfscope}%
\pgfsys@transformshift{7.200000in}{1.980618in}%
\pgfsys@useobject{currentmarker}{}%
\end{pgfscope}%
\end{pgfscope}%
\begin{pgfscope}%
\pgfsetbuttcap%
\pgfsetroundjoin%
\definecolor{currentfill}{rgb}{0.000000,0.000000,0.000000}%
\pgfsetfillcolor{currentfill}%
\pgfsetlinewidth{0.501875pt}%
\definecolor{currentstroke}{rgb}{0.000000,0.000000,0.000000}%
\pgfsetstrokecolor{currentstroke}%
\pgfsetdash{}{0pt}%
\pgfsys@defobject{currentmarker}{\pgfqpoint{0.000000in}{0.000000in}}{\pgfqpoint{0.027778in}{0.000000in}}{%
\pgfpathmoveto{\pgfqpoint{0.000000in}{0.000000in}}%
\pgfpathlineto{\pgfqpoint{0.027778in}{0.000000in}}%
\pgfusepath{stroke,fill}%
}%
\begin{pgfscope}%
\pgfsys@transformshift{1.000000in}{2.086273in}%
\pgfsys@useobject{currentmarker}{}%
\end{pgfscope}%
\end{pgfscope}%
\begin{pgfscope}%
\pgfsetbuttcap%
\pgfsetroundjoin%
\definecolor{currentfill}{rgb}{0.000000,0.000000,0.000000}%
\pgfsetfillcolor{currentfill}%
\pgfsetlinewidth{0.501875pt}%
\definecolor{currentstroke}{rgb}{0.000000,0.000000,0.000000}%
\pgfsetstrokecolor{currentstroke}%
\pgfsetdash{}{0pt}%
\pgfsys@defobject{currentmarker}{\pgfqpoint{-0.027778in}{0.000000in}}{\pgfqpoint{-0.000000in}{0.000000in}}{%
\pgfpathmoveto{\pgfqpoint{-0.000000in}{0.000000in}}%
\pgfpathlineto{\pgfqpoint{-0.027778in}{0.000000in}}%
\pgfusepath{stroke,fill}%
}%
\begin{pgfscope}%
\pgfsys@transformshift{7.200000in}{2.086273in}%
\pgfsys@useobject{currentmarker}{}%
\end{pgfscope}%
\end{pgfscope}%
\begin{pgfscope}%
\pgfsetbuttcap%
\pgfsetroundjoin%
\definecolor{currentfill}{rgb}{0.000000,0.000000,0.000000}%
\pgfsetfillcolor{currentfill}%
\pgfsetlinewidth{0.501875pt}%
\definecolor{currentstroke}{rgb}{0.000000,0.000000,0.000000}%
\pgfsetstrokecolor{currentstroke}%
\pgfsetdash{}{0pt}%
\pgfsys@defobject{currentmarker}{\pgfqpoint{0.000000in}{0.000000in}}{\pgfqpoint{0.027778in}{0.000000in}}{%
\pgfpathmoveto{\pgfqpoint{0.000000in}{0.000000in}}%
\pgfpathlineto{\pgfqpoint{0.027778in}{0.000000in}}%
\pgfusepath{stroke,fill}%
}%
\begin{pgfscope}%
\pgfsys@transformshift{1.000000in}{2.161236in}%
\pgfsys@useobject{currentmarker}{}%
\end{pgfscope}%
\end{pgfscope}%
\begin{pgfscope}%
\pgfsetbuttcap%
\pgfsetroundjoin%
\definecolor{currentfill}{rgb}{0.000000,0.000000,0.000000}%
\pgfsetfillcolor{currentfill}%
\pgfsetlinewidth{0.501875pt}%
\definecolor{currentstroke}{rgb}{0.000000,0.000000,0.000000}%
\pgfsetstrokecolor{currentstroke}%
\pgfsetdash{}{0pt}%
\pgfsys@defobject{currentmarker}{\pgfqpoint{-0.027778in}{0.000000in}}{\pgfqpoint{-0.000000in}{0.000000in}}{%
\pgfpathmoveto{\pgfqpoint{-0.000000in}{0.000000in}}%
\pgfpathlineto{\pgfqpoint{-0.027778in}{0.000000in}}%
\pgfusepath{stroke,fill}%
}%
\begin{pgfscope}%
\pgfsys@transformshift{7.200000in}{2.161236in}%
\pgfsys@useobject{currentmarker}{}%
\end{pgfscope}%
\end{pgfscope}%
\begin{pgfscope}%
\pgfsetbuttcap%
\pgfsetroundjoin%
\definecolor{currentfill}{rgb}{0.000000,0.000000,0.000000}%
\pgfsetfillcolor{currentfill}%
\pgfsetlinewidth{0.501875pt}%
\definecolor{currentstroke}{rgb}{0.000000,0.000000,0.000000}%
\pgfsetstrokecolor{currentstroke}%
\pgfsetdash{}{0pt}%
\pgfsys@defobject{currentmarker}{\pgfqpoint{0.000000in}{0.000000in}}{\pgfqpoint{0.027778in}{0.000000in}}{%
\pgfpathmoveto{\pgfqpoint{0.000000in}{0.000000in}}%
\pgfpathlineto{\pgfqpoint{0.027778in}{0.000000in}}%
\pgfusepath{stroke,fill}%
}%
\begin{pgfscope}%
\pgfsys@transformshift{1.000000in}{2.219382in}%
\pgfsys@useobject{currentmarker}{}%
\end{pgfscope}%
\end{pgfscope}%
\begin{pgfscope}%
\pgfsetbuttcap%
\pgfsetroundjoin%
\definecolor{currentfill}{rgb}{0.000000,0.000000,0.000000}%
\pgfsetfillcolor{currentfill}%
\pgfsetlinewidth{0.501875pt}%
\definecolor{currentstroke}{rgb}{0.000000,0.000000,0.000000}%
\pgfsetstrokecolor{currentstroke}%
\pgfsetdash{}{0pt}%
\pgfsys@defobject{currentmarker}{\pgfqpoint{-0.027778in}{0.000000in}}{\pgfqpoint{-0.000000in}{0.000000in}}{%
\pgfpathmoveto{\pgfqpoint{-0.000000in}{0.000000in}}%
\pgfpathlineto{\pgfqpoint{-0.027778in}{0.000000in}}%
\pgfusepath{stroke,fill}%
}%
\begin{pgfscope}%
\pgfsys@transformshift{7.200000in}{2.219382in}%
\pgfsys@useobject{currentmarker}{}%
\end{pgfscope}%
\end{pgfscope}%
\begin{pgfscope}%
\pgfsetbuttcap%
\pgfsetroundjoin%
\definecolor{currentfill}{rgb}{0.000000,0.000000,0.000000}%
\pgfsetfillcolor{currentfill}%
\pgfsetlinewidth{0.501875pt}%
\definecolor{currentstroke}{rgb}{0.000000,0.000000,0.000000}%
\pgfsetstrokecolor{currentstroke}%
\pgfsetdash{}{0pt}%
\pgfsys@defobject{currentmarker}{\pgfqpoint{0.000000in}{0.000000in}}{\pgfqpoint{0.027778in}{0.000000in}}{%
\pgfpathmoveto{\pgfqpoint{0.000000in}{0.000000in}}%
\pgfpathlineto{\pgfqpoint{0.027778in}{0.000000in}}%
\pgfusepath{stroke,fill}%
}%
\begin{pgfscope}%
\pgfsys@transformshift{1.000000in}{2.266891in}%
\pgfsys@useobject{currentmarker}{}%
\end{pgfscope}%
\end{pgfscope}%
\begin{pgfscope}%
\pgfsetbuttcap%
\pgfsetroundjoin%
\definecolor{currentfill}{rgb}{0.000000,0.000000,0.000000}%
\pgfsetfillcolor{currentfill}%
\pgfsetlinewidth{0.501875pt}%
\definecolor{currentstroke}{rgb}{0.000000,0.000000,0.000000}%
\pgfsetstrokecolor{currentstroke}%
\pgfsetdash{}{0pt}%
\pgfsys@defobject{currentmarker}{\pgfqpoint{-0.027778in}{0.000000in}}{\pgfqpoint{-0.000000in}{0.000000in}}{%
\pgfpathmoveto{\pgfqpoint{-0.000000in}{0.000000in}}%
\pgfpathlineto{\pgfqpoint{-0.027778in}{0.000000in}}%
\pgfusepath{stroke,fill}%
}%
\begin{pgfscope}%
\pgfsys@transformshift{7.200000in}{2.266891in}%
\pgfsys@useobject{currentmarker}{}%
\end{pgfscope}%
\end{pgfscope}%
\begin{pgfscope}%
\pgfsetbuttcap%
\pgfsetroundjoin%
\definecolor{currentfill}{rgb}{0.000000,0.000000,0.000000}%
\pgfsetfillcolor{currentfill}%
\pgfsetlinewidth{0.501875pt}%
\definecolor{currentstroke}{rgb}{0.000000,0.000000,0.000000}%
\pgfsetstrokecolor{currentstroke}%
\pgfsetdash{}{0pt}%
\pgfsys@defobject{currentmarker}{\pgfqpoint{0.000000in}{0.000000in}}{\pgfqpoint{0.027778in}{0.000000in}}{%
\pgfpathmoveto{\pgfqpoint{0.000000in}{0.000000in}}%
\pgfpathlineto{\pgfqpoint{0.027778in}{0.000000in}}%
\pgfusepath{stroke,fill}%
}%
\begin{pgfscope}%
\pgfsys@transformshift{1.000000in}{2.307059in}%
\pgfsys@useobject{currentmarker}{}%
\end{pgfscope}%
\end{pgfscope}%
\begin{pgfscope}%
\pgfsetbuttcap%
\pgfsetroundjoin%
\definecolor{currentfill}{rgb}{0.000000,0.000000,0.000000}%
\pgfsetfillcolor{currentfill}%
\pgfsetlinewidth{0.501875pt}%
\definecolor{currentstroke}{rgb}{0.000000,0.000000,0.000000}%
\pgfsetstrokecolor{currentstroke}%
\pgfsetdash{}{0pt}%
\pgfsys@defobject{currentmarker}{\pgfqpoint{-0.027778in}{0.000000in}}{\pgfqpoint{-0.000000in}{0.000000in}}{%
\pgfpathmoveto{\pgfqpoint{-0.000000in}{0.000000in}}%
\pgfpathlineto{\pgfqpoint{-0.027778in}{0.000000in}}%
\pgfusepath{stroke,fill}%
}%
\begin{pgfscope}%
\pgfsys@transformshift{7.200000in}{2.307059in}%
\pgfsys@useobject{currentmarker}{}%
\end{pgfscope}%
\end{pgfscope}%
\begin{pgfscope}%
\pgfsetbuttcap%
\pgfsetroundjoin%
\definecolor{currentfill}{rgb}{0.000000,0.000000,0.000000}%
\pgfsetfillcolor{currentfill}%
\pgfsetlinewidth{0.501875pt}%
\definecolor{currentstroke}{rgb}{0.000000,0.000000,0.000000}%
\pgfsetstrokecolor{currentstroke}%
\pgfsetdash{}{0pt}%
\pgfsys@defobject{currentmarker}{\pgfqpoint{0.000000in}{0.000000in}}{\pgfqpoint{0.027778in}{0.000000in}}{%
\pgfpathmoveto{\pgfqpoint{0.000000in}{0.000000in}}%
\pgfpathlineto{\pgfqpoint{0.027778in}{0.000000in}}%
\pgfusepath{stroke,fill}%
}%
\begin{pgfscope}%
\pgfsys@transformshift{1.000000in}{2.341854in}%
\pgfsys@useobject{currentmarker}{}%
\end{pgfscope}%
\end{pgfscope}%
\begin{pgfscope}%
\pgfsetbuttcap%
\pgfsetroundjoin%
\definecolor{currentfill}{rgb}{0.000000,0.000000,0.000000}%
\pgfsetfillcolor{currentfill}%
\pgfsetlinewidth{0.501875pt}%
\definecolor{currentstroke}{rgb}{0.000000,0.000000,0.000000}%
\pgfsetstrokecolor{currentstroke}%
\pgfsetdash{}{0pt}%
\pgfsys@defobject{currentmarker}{\pgfqpoint{-0.027778in}{0.000000in}}{\pgfqpoint{-0.000000in}{0.000000in}}{%
\pgfpathmoveto{\pgfqpoint{-0.000000in}{0.000000in}}%
\pgfpathlineto{\pgfqpoint{-0.027778in}{0.000000in}}%
\pgfusepath{stroke,fill}%
}%
\begin{pgfscope}%
\pgfsys@transformshift{7.200000in}{2.341854in}%
\pgfsys@useobject{currentmarker}{}%
\end{pgfscope}%
\end{pgfscope}%
\begin{pgfscope}%
\pgfsetbuttcap%
\pgfsetroundjoin%
\definecolor{currentfill}{rgb}{0.000000,0.000000,0.000000}%
\pgfsetfillcolor{currentfill}%
\pgfsetlinewidth{0.501875pt}%
\definecolor{currentstroke}{rgb}{0.000000,0.000000,0.000000}%
\pgfsetstrokecolor{currentstroke}%
\pgfsetdash{}{0pt}%
\pgfsys@defobject{currentmarker}{\pgfqpoint{0.000000in}{0.000000in}}{\pgfqpoint{0.027778in}{0.000000in}}{%
\pgfpathmoveto{\pgfqpoint{0.000000in}{0.000000in}}%
\pgfpathlineto{\pgfqpoint{0.027778in}{0.000000in}}%
\pgfusepath{stroke,fill}%
}%
\begin{pgfscope}%
\pgfsys@transformshift{1.000000in}{2.372546in}%
\pgfsys@useobject{currentmarker}{}%
\end{pgfscope}%
\end{pgfscope}%
\begin{pgfscope}%
\pgfsetbuttcap%
\pgfsetroundjoin%
\definecolor{currentfill}{rgb}{0.000000,0.000000,0.000000}%
\pgfsetfillcolor{currentfill}%
\pgfsetlinewidth{0.501875pt}%
\definecolor{currentstroke}{rgb}{0.000000,0.000000,0.000000}%
\pgfsetstrokecolor{currentstroke}%
\pgfsetdash{}{0pt}%
\pgfsys@defobject{currentmarker}{\pgfqpoint{-0.027778in}{0.000000in}}{\pgfqpoint{-0.000000in}{0.000000in}}{%
\pgfpathmoveto{\pgfqpoint{-0.000000in}{0.000000in}}%
\pgfpathlineto{\pgfqpoint{-0.027778in}{0.000000in}}%
\pgfusepath{stroke,fill}%
}%
\begin{pgfscope}%
\pgfsys@transformshift{7.200000in}{2.372546in}%
\pgfsys@useobject{currentmarker}{}%
\end{pgfscope}%
\end{pgfscope}%
\begin{pgfscope}%
\pgfsetbuttcap%
\pgfsetroundjoin%
\definecolor{currentfill}{rgb}{0.000000,0.000000,0.000000}%
\pgfsetfillcolor{currentfill}%
\pgfsetlinewidth{0.501875pt}%
\definecolor{currentstroke}{rgb}{0.000000,0.000000,0.000000}%
\pgfsetstrokecolor{currentstroke}%
\pgfsetdash{}{0pt}%
\pgfsys@defobject{currentmarker}{\pgfqpoint{0.000000in}{0.000000in}}{\pgfqpoint{0.027778in}{0.000000in}}{%
\pgfpathmoveto{\pgfqpoint{0.000000in}{0.000000in}}%
\pgfpathlineto{\pgfqpoint{0.027778in}{0.000000in}}%
\pgfusepath{stroke,fill}%
}%
\begin{pgfscope}%
\pgfsys@transformshift{1.000000in}{2.580618in}%
\pgfsys@useobject{currentmarker}{}%
\end{pgfscope}%
\end{pgfscope}%
\begin{pgfscope}%
\pgfsetbuttcap%
\pgfsetroundjoin%
\definecolor{currentfill}{rgb}{0.000000,0.000000,0.000000}%
\pgfsetfillcolor{currentfill}%
\pgfsetlinewidth{0.501875pt}%
\definecolor{currentstroke}{rgb}{0.000000,0.000000,0.000000}%
\pgfsetstrokecolor{currentstroke}%
\pgfsetdash{}{0pt}%
\pgfsys@defobject{currentmarker}{\pgfqpoint{-0.027778in}{0.000000in}}{\pgfqpoint{-0.000000in}{0.000000in}}{%
\pgfpathmoveto{\pgfqpoint{-0.000000in}{0.000000in}}%
\pgfpathlineto{\pgfqpoint{-0.027778in}{0.000000in}}%
\pgfusepath{stroke,fill}%
}%
\begin{pgfscope}%
\pgfsys@transformshift{7.200000in}{2.580618in}%
\pgfsys@useobject{currentmarker}{}%
\end{pgfscope}%
\end{pgfscope}%
\begin{pgfscope}%
\pgfsetbuttcap%
\pgfsetroundjoin%
\definecolor{currentfill}{rgb}{0.000000,0.000000,0.000000}%
\pgfsetfillcolor{currentfill}%
\pgfsetlinewidth{0.501875pt}%
\definecolor{currentstroke}{rgb}{0.000000,0.000000,0.000000}%
\pgfsetstrokecolor{currentstroke}%
\pgfsetdash{}{0pt}%
\pgfsys@defobject{currentmarker}{\pgfqpoint{0.000000in}{0.000000in}}{\pgfqpoint{0.027778in}{0.000000in}}{%
\pgfpathmoveto{\pgfqpoint{0.000000in}{0.000000in}}%
\pgfpathlineto{\pgfqpoint{0.027778in}{0.000000in}}%
\pgfusepath{stroke,fill}%
}%
\begin{pgfscope}%
\pgfsys@transformshift{1.000000in}{2.686273in}%
\pgfsys@useobject{currentmarker}{}%
\end{pgfscope}%
\end{pgfscope}%
\begin{pgfscope}%
\pgfsetbuttcap%
\pgfsetroundjoin%
\definecolor{currentfill}{rgb}{0.000000,0.000000,0.000000}%
\pgfsetfillcolor{currentfill}%
\pgfsetlinewidth{0.501875pt}%
\definecolor{currentstroke}{rgb}{0.000000,0.000000,0.000000}%
\pgfsetstrokecolor{currentstroke}%
\pgfsetdash{}{0pt}%
\pgfsys@defobject{currentmarker}{\pgfqpoint{-0.027778in}{0.000000in}}{\pgfqpoint{-0.000000in}{0.000000in}}{%
\pgfpathmoveto{\pgfqpoint{-0.000000in}{0.000000in}}%
\pgfpathlineto{\pgfqpoint{-0.027778in}{0.000000in}}%
\pgfusepath{stroke,fill}%
}%
\begin{pgfscope}%
\pgfsys@transformshift{7.200000in}{2.686273in}%
\pgfsys@useobject{currentmarker}{}%
\end{pgfscope}%
\end{pgfscope}%
\begin{pgfscope}%
\pgfsetbuttcap%
\pgfsetroundjoin%
\definecolor{currentfill}{rgb}{0.000000,0.000000,0.000000}%
\pgfsetfillcolor{currentfill}%
\pgfsetlinewidth{0.501875pt}%
\definecolor{currentstroke}{rgb}{0.000000,0.000000,0.000000}%
\pgfsetstrokecolor{currentstroke}%
\pgfsetdash{}{0pt}%
\pgfsys@defobject{currentmarker}{\pgfqpoint{0.000000in}{0.000000in}}{\pgfqpoint{0.027778in}{0.000000in}}{%
\pgfpathmoveto{\pgfqpoint{0.000000in}{0.000000in}}%
\pgfpathlineto{\pgfqpoint{0.027778in}{0.000000in}}%
\pgfusepath{stroke,fill}%
}%
\begin{pgfscope}%
\pgfsys@transformshift{1.000000in}{2.761236in}%
\pgfsys@useobject{currentmarker}{}%
\end{pgfscope}%
\end{pgfscope}%
\begin{pgfscope}%
\pgfsetbuttcap%
\pgfsetroundjoin%
\definecolor{currentfill}{rgb}{0.000000,0.000000,0.000000}%
\pgfsetfillcolor{currentfill}%
\pgfsetlinewidth{0.501875pt}%
\definecolor{currentstroke}{rgb}{0.000000,0.000000,0.000000}%
\pgfsetstrokecolor{currentstroke}%
\pgfsetdash{}{0pt}%
\pgfsys@defobject{currentmarker}{\pgfqpoint{-0.027778in}{0.000000in}}{\pgfqpoint{-0.000000in}{0.000000in}}{%
\pgfpathmoveto{\pgfqpoint{-0.000000in}{0.000000in}}%
\pgfpathlineto{\pgfqpoint{-0.027778in}{0.000000in}}%
\pgfusepath{stroke,fill}%
}%
\begin{pgfscope}%
\pgfsys@transformshift{7.200000in}{2.761236in}%
\pgfsys@useobject{currentmarker}{}%
\end{pgfscope}%
\end{pgfscope}%
\begin{pgfscope}%
\pgfsetbuttcap%
\pgfsetroundjoin%
\definecolor{currentfill}{rgb}{0.000000,0.000000,0.000000}%
\pgfsetfillcolor{currentfill}%
\pgfsetlinewidth{0.501875pt}%
\definecolor{currentstroke}{rgb}{0.000000,0.000000,0.000000}%
\pgfsetstrokecolor{currentstroke}%
\pgfsetdash{}{0pt}%
\pgfsys@defobject{currentmarker}{\pgfqpoint{0.000000in}{0.000000in}}{\pgfqpoint{0.027778in}{0.000000in}}{%
\pgfpathmoveto{\pgfqpoint{0.000000in}{0.000000in}}%
\pgfpathlineto{\pgfqpoint{0.027778in}{0.000000in}}%
\pgfusepath{stroke,fill}%
}%
\begin{pgfscope}%
\pgfsys@transformshift{1.000000in}{2.819382in}%
\pgfsys@useobject{currentmarker}{}%
\end{pgfscope}%
\end{pgfscope}%
\begin{pgfscope}%
\pgfsetbuttcap%
\pgfsetroundjoin%
\definecolor{currentfill}{rgb}{0.000000,0.000000,0.000000}%
\pgfsetfillcolor{currentfill}%
\pgfsetlinewidth{0.501875pt}%
\definecolor{currentstroke}{rgb}{0.000000,0.000000,0.000000}%
\pgfsetstrokecolor{currentstroke}%
\pgfsetdash{}{0pt}%
\pgfsys@defobject{currentmarker}{\pgfqpoint{-0.027778in}{0.000000in}}{\pgfqpoint{-0.000000in}{0.000000in}}{%
\pgfpathmoveto{\pgfqpoint{-0.000000in}{0.000000in}}%
\pgfpathlineto{\pgfqpoint{-0.027778in}{0.000000in}}%
\pgfusepath{stroke,fill}%
}%
\begin{pgfscope}%
\pgfsys@transformshift{7.200000in}{2.819382in}%
\pgfsys@useobject{currentmarker}{}%
\end{pgfscope}%
\end{pgfscope}%
\begin{pgfscope}%
\pgfsetbuttcap%
\pgfsetroundjoin%
\definecolor{currentfill}{rgb}{0.000000,0.000000,0.000000}%
\pgfsetfillcolor{currentfill}%
\pgfsetlinewidth{0.501875pt}%
\definecolor{currentstroke}{rgb}{0.000000,0.000000,0.000000}%
\pgfsetstrokecolor{currentstroke}%
\pgfsetdash{}{0pt}%
\pgfsys@defobject{currentmarker}{\pgfqpoint{0.000000in}{0.000000in}}{\pgfqpoint{0.027778in}{0.000000in}}{%
\pgfpathmoveto{\pgfqpoint{0.000000in}{0.000000in}}%
\pgfpathlineto{\pgfqpoint{0.027778in}{0.000000in}}%
\pgfusepath{stroke,fill}%
}%
\begin{pgfscope}%
\pgfsys@transformshift{1.000000in}{2.866891in}%
\pgfsys@useobject{currentmarker}{}%
\end{pgfscope}%
\end{pgfscope}%
\begin{pgfscope}%
\pgfsetbuttcap%
\pgfsetroundjoin%
\definecolor{currentfill}{rgb}{0.000000,0.000000,0.000000}%
\pgfsetfillcolor{currentfill}%
\pgfsetlinewidth{0.501875pt}%
\definecolor{currentstroke}{rgb}{0.000000,0.000000,0.000000}%
\pgfsetstrokecolor{currentstroke}%
\pgfsetdash{}{0pt}%
\pgfsys@defobject{currentmarker}{\pgfqpoint{-0.027778in}{0.000000in}}{\pgfqpoint{-0.000000in}{0.000000in}}{%
\pgfpathmoveto{\pgfqpoint{-0.000000in}{0.000000in}}%
\pgfpathlineto{\pgfqpoint{-0.027778in}{0.000000in}}%
\pgfusepath{stroke,fill}%
}%
\begin{pgfscope}%
\pgfsys@transformshift{7.200000in}{2.866891in}%
\pgfsys@useobject{currentmarker}{}%
\end{pgfscope}%
\end{pgfscope}%
\begin{pgfscope}%
\pgfsetbuttcap%
\pgfsetroundjoin%
\definecolor{currentfill}{rgb}{0.000000,0.000000,0.000000}%
\pgfsetfillcolor{currentfill}%
\pgfsetlinewidth{0.501875pt}%
\definecolor{currentstroke}{rgb}{0.000000,0.000000,0.000000}%
\pgfsetstrokecolor{currentstroke}%
\pgfsetdash{}{0pt}%
\pgfsys@defobject{currentmarker}{\pgfqpoint{0.000000in}{0.000000in}}{\pgfqpoint{0.027778in}{0.000000in}}{%
\pgfpathmoveto{\pgfqpoint{0.000000in}{0.000000in}}%
\pgfpathlineto{\pgfqpoint{0.027778in}{0.000000in}}%
\pgfusepath{stroke,fill}%
}%
\begin{pgfscope}%
\pgfsys@transformshift{1.000000in}{2.907059in}%
\pgfsys@useobject{currentmarker}{}%
\end{pgfscope}%
\end{pgfscope}%
\begin{pgfscope}%
\pgfsetbuttcap%
\pgfsetroundjoin%
\definecolor{currentfill}{rgb}{0.000000,0.000000,0.000000}%
\pgfsetfillcolor{currentfill}%
\pgfsetlinewidth{0.501875pt}%
\definecolor{currentstroke}{rgb}{0.000000,0.000000,0.000000}%
\pgfsetstrokecolor{currentstroke}%
\pgfsetdash{}{0pt}%
\pgfsys@defobject{currentmarker}{\pgfqpoint{-0.027778in}{0.000000in}}{\pgfqpoint{-0.000000in}{0.000000in}}{%
\pgfpathmoveto{\pgfqpoint{-0.000000in}{0.000000in}}%
\pgfpathlineto{\pgfqpoint{-0.027778in}{0.000000in}}%
\pgfusepath{stroke,fill}%
}%
\begin{pgfscope}%
\pgfsys@transformshift{7.200000in}{2.907059in}%
\pgfsys@useobject{currentmarker}{}%
\end{pgfscope}%
\end{pgfscope}%
\begin{pgfscope}%
\pgfsetbuttcap%
\pgfsetroundjoin%
\definecolor{currentfill}{rgb}{0.000000,0.000000,0.000000}%
\pgfsetfillcolor{currentfill}%
\pgfsetlinewidth{0.501875pt}%
\definecolor{currentstroke}{rgb}{0.000000,0.000000,0.000000}%
\pgfsetstrokecolor{currentstroke}%
\pgfsetdash{}{0pt}%
\pgfsys@defobject{currentmarker}{\pgfqpoint{0.000000in}{0.000000in}}{\pgfqpoint{0.027778in}{0.000000in}}{%
\pgfpathmoveto{\pgfqpoint{0.000000in}{0.000000in}}%
\pgfpathlineto{\pgfqpoint{0.027778in}{0.000000in}}%
\pgfusepath{stroke,fill}%
}%
\begin{pgfscope}%
\pgfsys@transformshift{1.000000in}{2.941854in}%
\pgfsys@useobject{currentmarker}{}%
\end{pgfscope}%
\end{pgfscope}%
\begin{pgfscope}%
\pgfsetbuttcap%
\pgfsetroundjoin%
\definecolor{currentfill}{rgb}{0.000000,0.000000,0.000000}%
\pgfsetfillcolor{currentfill}%
\pgfsetlinewidth{0.501875pt}%
\definecolor{currentstroke}{rgb}{0.000000,0.000000,0.000000}%
\pgfsetstrokecolor{currentstroke}%
\pgfsetdash{}{0pt}%
\pgfsys@defobject{currentmarker}{\pgfqpoint{-0.027778in}{0.000000in}}{\pgfqpoint{-0.000000in}{0.000000in}}{%
\pgfpathmoveto{\pgfqpoint{-0.000000in}{0.000000in}}%
\pgfpathlineto{\pgfqpoint{-0.027778in}{0.000000in}}%
\pgfusepath{stroke,fill}%
}%
\begin{pgfscope}%
\pgfsys@transformshift{7.200000in}{2.941854in}%
\pgfsys@useobject{currentmarker}{}%
\end{pgfscope}%
\end{pgfscope}%
\begin{pgfscope}%
\pgfsetbuttcap%
\pgfsetroundjoin%
\definecolor{currentfill}{rgb}{0.000000,0.000000,0.000000}%
\pgfsetfillcolor{currentfill}%
\pgfsetlinewidth{0.501875pt}%
\definecolor{currentstroke}{rgb}{0.000000,0.000000,0.000000}%
\pgfsetstrokecolor{currentstroke}%
\pgfsetdash{}{0pt}%
\pgfsys@defobject{currentmarker}{\pgfqpoint{0.000000in}{0.000000in}}{\pgfqpoint{0.027778in}{0.000000in}}{%
\pgfpathmoveto{\pgfqpoint{0.000000in}{0.000000in}}%
\pgfpathlineto{\pgfqpoint{0.027778in}{0.000000in}}%
\pgfusepath{stroke,fill}%
}%
\begin{pgfscope}%
\pgfsys@transformshift{1.000000in}{2.972546in}%
\pgfsys@useobject{currentmarker}{}%
\end{pgfscope}%
\end{pgfscope}%
\begin{pgfscope}%
\pgfsetbuttcap%
\pgfsetroundjoin%
\definecolor{currentfill}{rgb}{0.000000,0.000000,0.000000}%
\pgfsetfillcolor{currentfill}%
\pgfsetlinewidth{0.501875pt}%
\definecolor{currentstroke}{rgb}{0.000000,0.000000,0.000000}%
\pgfsetstrokecolor{currentstroke}%
\pgfsetdash{}{0pt}%
\pgfsys@defobject{currentmarker}{\pgfqpoint{-0.027778in}{0.000000in}}{\pgfqpoint{-0.000000in}{0.000000in}}{%
\pgfpathmoveto{\pgfqpoint{-0.000000in}{0.000000in}}%
\pgfpathlineto{\pgfqpoint{-0.027778in}{0.000000in}}%
\pgfusepath{stroke,fill}%
}%
\begin{pgfscope}%
\pgfsys@transformshift{7.200000in}{2.972546in}%
\pgfsys@useobject{currentmarker}{}%
\end{pgfscope}%
\end{pgfscope}%
\begin{pgfscope}%
\pgfsetbuttcap%
\pgfsetroundjoin%
\definecolor{currentfill}{rgb}{0.000000,0.000000,0.000000}%
\pgfsetfillcolor{currentfill}%
\pgfsetlinewidth{0.501875pt}%
\definecolor{currentstroke}{rgb}{0.000000,0.000000,0.000000}%
\pgfsetstrokecolor{currentstroke}%
\pgfsetdash{}{0pt}%
\pgfsys@defobject{currentmarker}{\pgfqpoint{0.000000in}{0.000000in}}{\pgfqpoint{0.027778in}{0.000000in}}{%
\pgfpathmoveto{\pgfqpoint{0.000000in}{0.000000in}}%
\pgfpathlineto{\pgfqpoint{0.027778in}{0.000000in}}%
\pgfusepath{stroke,fill}%
}%
\begin{pgfscope}%
\pgfsys@transformshift{1.000000in}{3.180618in}%
\pgfsys@useobject{currentmarker}{}%
\end{pgfscope}%
\end{pgfscope}%
\begin{pgfscope}%
\pgfsetbuttcap%
\pgfsetroundjoin%
\definecolor{currentfill}{rgb}{0.000000,0.000000,0.000000}%
\pgfsetfillcolor{currentfill}%
\pgfsetlinewidth{0.501875pt}%
\definecolor{currentstroke}{rgb}{0.000000,0.000000,0.000000}%
\pgfsetstrokecolor{currentstroke}%
\pgfsetdash{}{0pt}%
\pgfsys@defobject{currentmarker}{\pgfqpoint{-0.027778in}{0.000000in}}{\pgfqpoint{-0.000000in}{0.000000in}}{%
\pgfpathmoveto{\pgfqpoint{-0.000000in}{0.000000in}}%
\pgfpathlineto{\pgfqpoint{-0.027778in}{0.000000in}}%
\pgfusepath{stroke,fill}%
}%
\begin{pgfscope}%
\pgfsys@transformshift{7.200000in}{3.180618in}%
\pgfsys@useobject{currentmarker}{}%
\end{pgfscope}%
\end{pgfscope}%
\begin{pgfscope}%
\pgfsetbuttcap%
\pgfsetroundjoin%
\definecolor{currentfill}{rgb}{0.000000,0.000000,0.000000}%
\pgfsetfillcolor{currentfill}%
\pgfsetlinewidth{0.501875pt}%
\definecolor{currentstroke}{rgb}{0.000000,0.000000,0.000000}%
\pgfsetstrokecolor{currentstroke}%
\pgfsetdash{}{0pt}%
\pgfsys@defobject{currentmarker}{\pgfqpoint{0.000000in}{0.000000in}}{\pgfqpoint{0.027778in}{0.000000in}}{%
\pgfpathmoveto{\pgfqpoint{0.000000in}{0.000000in}}%
\pgfpathlineto{\pgfqpoint{0.027778in}{0.000000in}}%
\pgfusepath{stroke,fill}%
}%
\begin{pgfscope}%
\pgfsys@transformshift{1.000000in}{3.286273in}%
\pgfsys@useobject{currentmarker}{}%
\end{pgfscope}%
\end{pgfscope}%
\begin{pgfscope}%
\pgfsetbuttcap%
\pgfsetroundjoin%
\definecolor{currentfill}{rgb}{0.000000,0.000000,0.000000}%
\pgfsetfillcolor{currentfill}%
\pgfsetlinewidth{0.501875pt}%
\definecolor{currentstroke}{rgb}{0.000000,0.000000,0.000000}%
\pgfsetstrokecolor{currentstroke}%
\pgfsetdash{}{0pt}%
\pgfsys@defobject{currentmarker}{\pgfqpoint{-0.027778in}{0.000000in}}{\pgfqpoint{-0.000000in}{0.000000in}}{%
\pgfpathmoveto{\pgfqpoint{-0.000000in}{0.000000in}}%
\pgfpathlineto{\pgfqpoint{-0.027778in}{0.000000in}}%
\pgfusepath{stroke,fill}%
}%
\begin{pgfscope}%
\pgfsys@transformshift{7.200000in}{3.286273in}%
\pgfsys@useobject{currentmarker}{}%
\end{pgfscope}%
\end{pgfscope}%
\begin{pgfscope}%
\pgfsetbuttcap%
\pgfsetroundjoin%
\definecolor{currentfill}{rgb}{0.000000,0.000000,0.000000}%
\pgfsetfillcolor{currentfill}%
\pgfsetlinewidth{0.501875pt}%
\definecolor{currentstroke}{rgb}{0.000000,0.000000,0.000000}%
\pgfsetstrokecolor{currentstroke}%
\pgfsetdash{}{0pt}%
\pgfsys@defobject{currentmarker}{\pgfqpoint{0.000000in}{0.000000in}}{\pgfqpoint{0.027778in}{0.000000in}}{%
\pgfpathmoveto{\pgfqpoint{0.000000in}{0.000000in}}%
\pgfpathlineto{\pgfqpoint{0.027778in}{0.000000in}}%
\pgfusepath{stroke,fill}%
}%
\begin{pgfscope}%
\pgfsys@transformshift{1.000000in}{3.361236in}%
\pgfsys@useobject{currentmarker}{}%
\end{pgfscope}%
\end{pgfscope}%
\begin{pgfscope}%
\pgfsetbuttcap%
\pgfsetroundjoin%
\definecolor{currentfill}{rgb}{0.000000,0.000000,0.000000}%
\pgfsetfillcolor{currentfill}%
\pgfsetlinewidth{0.501875pt}%
\definecolor{currentstroke}{rgb}{0.000000,0.000000,0.000000}%
\pgfsetstrokecolor{currentstroke}%
\pgfsetdash{}{0pt}%
\pgfsys@defobject{currentmarker}{\pgfqpoint{-0.027778in}{0.000000in}}{\pgfqpoint{-0.000000in}{0.000000in}}{%
\pgfpathmoveto{\pgfqpoint{-0.000000in}{0.000000in}}%
\pgfpathlineto{\pgfqpoint{-0.027778in}{0.000000in}}%
\pgfusepath{stroke,fill}%
}%
\begin{pgfscope}%
\pgfsys@transformshift{7.200000in}{3.361236in}%
\pgfsys@useobject{currentmarker}{}%
\end{pgfscope}%
\end{pgfscope}%
\begin{pgfscope}%
\pgfsetbuttcap%
\pgfsetroundjoin%
\definecolor{currentfill}{rgb}{0.000000,0.000000,0.000000}%
\pgfsetfillcolor{currentfill}%
\pgfsetlinewidth{0.501875pt}%
\definecolor{currentstroke}{rgb}{0.000000,0.000000,0.000000}%
\pgfsetstrokecolor{currentstroke}%
\pgfsetdash{}{0pt}%
\pgfsys@defobject{currentmarker}{\pgfqpoint{0.000000in}{0.000000in}}{\pgfqpoint{0.027778in}{0.000000in}}{%
\pgfpathmoveto{\pgfqpoint{0.000000in}{0.000000in}}%
\pgfpathlineto{\pgfqpoint{0.027778in}{0.000000in}}%
\pgfusepath{stroke,fill}%
}%
\begin{pgfscope}%
\pgfsys@transformshift{1.000000in}{3.419382in}%
\pgfsys@useobject{currentmarker}{}%
\end{pgfscope}%
\end{pgfscope}%
\begin{pgfscope}%
\pgfsetbuttcap%
\pgfsetroundjoin%
\definecolor{currentfill}{rgb}{0.000000,0.000000,0.000000}%
\pgfsetfillcolor{currentfill}%
\pgfsetlinewidth{0.501875pt}%
\definecolor{currentstroke}{rgb}{0.000000,0.000000,0.000000}%
\pgfsetstrokecolor{currentstroke}%
\pgfsetdash{}{0pt}%
\pgfsys@defobject{currentmarker}{\pgfqpoint{-0.027778in}{0.000000in}}{\pgfqpoint{-0.000000in}{0.000000in}}{%
\pgfpathmoveto{\pgfqpoint{-0.000000in}{0.000000in}}%
\pgfpathlineto{\pgfqpoint{-0.027778in}{0.000000in}}%
\pgfusepath{stroke,fill}%
}%
\begin{pgfscope}%
\pgfsys@transformshift{7.200000in}{3.419382in}%
\pgfsys@useobject{currentmarker}{}%
\end{pgfscope}%
\end{pgfscope}%
\begin{pgfscope}%
\pgfsetbuttcap%
\pgfsetroundjoin%
\definecolor{currentfill}{rgb}{0.000000,0.000000,0.000000}%
\pgfsetfillcolor{currentfill}%
\pgfsetlinewidth{0.501875pt}%
\definecolor{currentstroke}{rgb}{0.000000,0.000000,0.000000}%
\pgfsetstrokecolor{currentstroke}%
\pgfsetdash{}{0pt}%
\pgfsys@defobject{currentmarker}{\pgfqpoint{0.000000in}{0.000000in}}{\pgfqpoint{0.027778in}{0.000000in}}{%
\pgfpathmoveto{\pgfqpoint{0.000000in}{0.000000in}}%
\pgfpathlineto{\pgfqpoint{0.027778in}{0.000000in}}%
\pgfusepath{stroke,fill}%
}%
\begin{pgfscope}%
\pgfsys@transformshift{1.000000in}{3.466891in}%
\pgfsys@useobject{currentmarker}{}%
\end{pgfscope}%
\end{pgfscope}%
\begin{pgfscope}%
\pgfsetbuttcap%
\pgfsetroundjoin%
\definecolor{currentfill}{rgb}{0.000000,0.000000,0.000000}%
\pgfsetfillcolor{currentfill}%
\pgfsetlinewidth{0.501875pt}%
\definecolor{currentstroke}{rgb}{0.000000,0.000000,0.000000}%
\pgfsetstrokecolor{currentstroke}%
\pgfsetdash{}{0pt}%
\pgfsys@defobject{currentmarker}{\pgfqpoint{-0.027778in}{0.000000in}}{\pgfqpoint{-0.000000in}{0.000000in}}{%
\pgfpathmoveto{\pgfqpoint{-0.000000in}{0.000000in}}%
\pgfpathlineto{\pgfqpoint{-0.027778in}{0.000000in}}%
\pgfusepath{stroke,fill}%
}%
\begin{pgfscope}%
\pgfsys@transformshift{7.200000in}{3.466891in}%
\pgfsys@useobject{currentmarker}{}%
\end{pgfscope}%
\end{pgfscope}%
\begin{pgfscope}%
\pgfsetbuttcap%
\pgfsetroundjoin%
\definecolor{currentfill}{rgb}{0.000000,0.000000,0.000000}%
\pgfsetfillcolor{currentfill}%
\pgfsetlinewidth{0.501875pt}%
\definecolor{currentstroke}{rgb}{0.000000,0.000000,0.000000}%
\pgfsetstrokecolor{currentstroke}%
\pgfsetdash{}{0pt}%
\pgfsys@defobject{currentmarker}{\pgfqpoint{0.000000in}{0.000000in}}{\pgfqpoint{0.027778in}{0.000000in}}{%
\pgfpathmoveto{\pgfqpoint{0.000000in}{0.000000in}}%
\pgfpathlineto{\pgfqpoint{0.027778in}{0.000000in}}%
\pgfusepath{stroke,fill}%
}%
\begin{pgfscope}%
\pgfsys@transformshift{1.000000in}{3.507059in}%
\pgfsys@useobject{currentmarker}{}%
\end{pgfscope}%
\end{pgfscope}%
\begin{pgfscope}%
\pgfsetbuttcap%
\pgfsetroundjoin%
\definecolor{currentfill}{rgb}{0.000000,0.000000,0.000000}%
\pgfsetfillcolor{currentfill}%
\pgfsetlinewidth{0.501875pt}%
\definecolor{currentstroke}{rgb}{0.000000,0.000000,0.000000}%
\pgfsetstrokecolor{currentstroke}%
\pgfsetdash{}{0pt}%
\pgfsys@defobject{currentmarker}{\pgfqpoint{-0.027778in}{0.000000in}}{\pgfqpoint{-0.000000in}{0.000000in}}{%
\pgfpathmoveto{\pgfqpoint{-0.000000in}{0.000000in}}%
\pgfpathlineto{\pgfqpoint{-0.027778in}{0.000000in}}%
\pgfusepath{stroke,fill}%
}%
\begin{pgfscope}%
\pgfsys@transformshift{7.200000in}{3.507059in}%
\pgfsys@useobject{currentmarker}{}%
\end{pgfscope}%
\end{pgfscope}%
\begin{pgfscope}%
\pgfsetbuttcap%
\pgfsetroundjoin%
\definecolor{currentfill}{rgb}{0.000000,0.000000,0.000000}%
\pgfsetfillcolor{currentfill}%
\pgfsetlinewidth{0.501875pt}%
\definecolor{currentstroke}{rgb}{0.000000,0.000000,0.000000}%
\pgfsetstrokecolor{currentstroke}%
\pgfsetdash{}{0pt}%
\pgfsys@defobject{currentmarker}{\pgfqpoint{0.000000in}{0.000000in}}{\pgfqpoint{0.027778in}{0.000000in}}{%
\pgfpathmoveto{\pgfqpoint{0.000000in}{0.000000in}}%
\pgfpathlineto{\pgfqpoint{0.027778in}{0.000000in}}%
\pgfusepath{stroke,fill}%
}%
\begin{pgfscope}%
\pgfsys@transformshift{1.000000in}{3.541854in}%
\pgfsys@useobject{currentmarker}{}%
\end{pgfscope}%
\end{pgfscope}%
\begin{pgfscope}%
\pgfsetbuttcap%
\pgfsetroundjoin%
\definecolor{currentfill}{rgb}{0.000000,0.000000,0.000000}%
\pgfsetfillcolor{currentfill}%
\pgfsetlinewidth{0.501875pt}%
\definecolor{currentstroke}{rgb}{0.000000,0.000000,0.000000}%
\pgfsetstrokecolor{currentstroke}%
\pgfsetdash{}{0pt}%
\pgfsys@defobject{currentmarker}{\pgfqpoint{-0.027778in}{0.000000in}}{\pgfqpoint{-0.000000in}{0.000000in}}{%
\pgfpathmoveto{\pgfqpoint{-0.000000in}{0.000000in}}%
\pgfpathlineto{\pgfqpoint{-0.027778in}{0.000000in}}%
\pgfusepath{stroke,fill}%
}%
\begin{pgfscope}%
\pgfsys@transformshift{7.200000in}{3.541854in}%
\pgfsys@useobject{currentmarker}{}%
\end{pgfscope}%
\end{pgfscope}%
\begin{pgfscope}%
\pgfsetbuttcap%
\pgfsetroundjoin%
\definecolor{currentfill}{rgb}{0.000000,0.000000,0.000000}%
\pgfsetfillcolor{currentfill}%
\pgfsetlinewidth{0.501875pt}%
\definecolor{currentstroke}{rgb}{0.000000,0.000000,0.000000}%
\pgfsetstrokecolor{currentstroke}%
\pgfsetdash{}{0pt}%
\pgfsys@defobject{currentmarker}{\pgfqpoint{0.000000in}{0.000000in}}{\pgfqpoint{0.027778in}{0.000000in}}{%
\pgfpathmoveto{\pgfqpoint{0.000000in}{0.000000in}}%
\pgfpathlineto{\pgfqpoint{0.027778in}{0.000000in}}%
\pgfusepath{stroke,fill}%
}%
\begin{pgfscope}%
\pgfsys@transformshift{1.000000in}{3.572546in}%
\pgfsys@useobject{currentmarker}{}%
\end{pgfscope}%
\end{pgfscope}%
\begin{pgfscope}%
\pgfsetbuttcap%
\pgfsetroundjoin%
\definecolor{currentfill}{rgb}{0.000000,0.000000,0.000000}%
\pgfsetfillcolor{currentfill}%
\pgfsetlinewidth{0.501875pt}%
\definecolor{currentstroke}{rgb}{0.000000,0.000000,0.000000}%
\pgfsetstrokecolor{currentstroke}%
\pgfsetdash{}{0pt}%
\pgfsys@defobject{currentmarker}{\pgfqpoint{-0.027778in}{0.000000in}}{\pgfqpoint{-0.000000in}{0.000000in}}{%
\pgfpathmoveto{\pgfqpoint{-0.000000in}{0.000000in}}%
\pgfpathlineto{\pgfqpoint{-0.027778in}{0.000000in}}%
\pgfusepath{stroke,fill}%
}%
\begin{pgfscope}%
\pgfsys@transformshift{7.200000in}{3.572546in}%
\pgfsys@useobject{currentmarker}{}%
\end{pgfscope}%
\end{pgfscope}%
\begin{pgfscope}%
\pgfsetbuttcap%
\pgfsetroundjoin%
\definecolor{currentfill}{rgb}{0.000000,0.000000,0.000000}%
\pgfsetfillcolor{currentfill}%
\pgfsetlinewidth{0.501875pt}%
\definecolor{currentstroke}{rgb}{0.000000,0.000000,0.000000}%
\pgfsetstrokecolor{currentstroke}%
\pgfsetdash{}{0pt}%
\pgfsys@defobject{currentmarker}{\pgfqpoint{0.000000in}{0.000000in}}{\pgfqpoint{0.027778in}{0.000000in}}{%
\pgfpathmoveto{\pgfqpoint{0.000000in}{0.000000in}}%
\pgfpathlineto{\pgfqpoint{0.027778in}{0.000000in}}%
\pgfusepath{stroke,fill}%
}%
\begin{pgfscope}%
\pgfsys@transformshift{1.000000in}{3.780618in}%
\pgfsys@useobject{currentmarker}{}%
\end{pgfscope}%
\end{pgfscope}%
\begin{pgfscope}%
\pgfsetbuttcap%
\pgfsetroundjoin%
\definecolor{currentfill}{rgb}{0.000000,0.000000,0.000000}%
\pgfsetfillcolor{currentfill}%
\pgfsetlinewidth{0.501875pt}%
\definecolor{currentstroke}{rgb}{0.000000,0.000000,0.000000}%
\pgfsetstrokecolor{currentstroke}%
\pgfsetdash{}{0pt}%
\pgfsys@defobject{currentmarker}{\pgfqpoint{-0.027778in}{0.000000in}}{\pgfqpoint{-0.000000in}{0.000000in}}{%
\pgfpathmoveto{\pgfqpoint{-0.000000in}{0.000000in}}%
\pgfpathlineto{\pgfqpoint{-0.027778in}{0.000000in}}%
\pgfusepath{stroke,fill}%
}%
\begin{pgfscope}%
\pgfsys@transformshift{7.200000in}{3.780618in}%
\pgfsys@useobject{currentmarker}{}%
\end{pgfscope}%
\end{pgfscope}%
\begin{pgfscope}%
\pgfsetbuttcap%
\pgfsetroundjoin%
\definecolor{currentfill}{rgb}{0.000000,0.000000,0.000000}%
\pgfsetfillcolor{currentfill}%
\pgfsetlinewidth{0.501875pt}%
\definecolor{currentstroke}{rgb}{0.000000,0.000000,0.000000}%
\pgfsetstrokecolor{currentstroke}%
\pgfsetdash{}{0pt}%
\pgfsys@defobject{currentmarker}{\pgfqpoint{0.000000in}{0.000000in}}{\pgfqpoint{0.027778in}{0.000000in}}{%
\pgfpathmoveto{\pgfqpoint{0.000000in}{0.000000in}}%
\pgfpathlineto{\pgfqpoint{0.027778in}{0.000000in}}%
\pgfusepath{stroke,fill}%
}%
\begin{pgfscope}%
\pgfsys@transformshift{1.000000in}{3.886273in}%
\pgfsys@useobject{currentmarker}{}%
\end{pgfscope}%
\end{pgfscope}%
\begin{pgfscope}%
\pgfsetbuttcap%
\pgfsetroundjoin%
\definecolor{currentfill}{rgb}{0.000000,0.000000,0.000000}%
\pgfsetfillcolor{currentfill}%
\pgfsetlinewidth{0.501875pt}%
\definecolor{currentstroke}{rgb}{0.000000,0.000000,0.000000}%
\pgfsetstrokecolor{currentstroke}%
\pgfsetdash{}{0pt}%
\pgfsys@defobject{currentmarker}{\pgfqpoint{-0.027778in}{0.000000in}}{\pgfqpoint{-0.000000in}{0.000000in}}{%
\pgfpathmoveto{\pgfqpoint{-0.000000in}{0.000000in}}%
\pgfpathlineto{\pgfqpoint{-0.027778in}{0.000000in}}%
\pgfusepath{stroke,fill}%
}%
\begin{pgfscope}%
\pgfsys@transformshift{7.200000in}{3.886273in}%
\pgfsys@useobject{currentmarker}{}%
\end{pgfscope}%
\end{pgfscope}%
\begin{pgfscope}%
\pgfsetbuttcap%
\pgfsetroundjoin%
\definecolor{currentfill}{rgb}{0.000000,0.000000,0.000000}%
\pgfsetfillcolor{currentfill}%
\pgfsetlinewidth{0.501875pt}%
\definecolor{currentstroke}{rgb}{0.000000,0.000000,0.000000}%
\pgfsetstrokecolor{currentstroke}%
\pgfsetdash{}{0pt}%
\pgfsys@defobject{currentmarker}{\pgfqpoint{0.000000in}{0.000000in}}{\pgfqpoint{0.027778in}{0.000000in}}{%
\pgfpathmoveto{\pgfqpoint{0.000000in}{0.000000in}}%
\pgfpathlineto{\pgfqpoint{0.027778in}{0.000000in}}%
\pgfusepath{stroke,fill}%
}%
\begin{pgfscope}%
\pgfsys@transformshift{1.000000in}{3.961236in}%
\pgfsys@useobject{currentmarker}{}%
\end{pgfscope}%
\end{pgfscope}%
\begin{pgfscope}%
\pgfsetbuttcap%
\pgfsetroundjoin%
\definecolor{currentfill}{rgb}{0.000000,0.000000,0.000000}%
\pgfsetfillcolor{currentfill}%
\pgfsetlinewidth{0.501875pt}%
\definecolor{currentstroke}{rgb}{0.000000,0.000000,0.000000}%
\pgfsetstrokecolor{currentstroke}%
\pgfsetdash{}{0pt}%
\pgfsys@defobject{currentmarker}{\pgfqpoint{-0.027778in}{0.000000in}}{\pgfqpoint{-0.000000in}{0.000000in}}{%
\pgfpathmoveto{\pgfqpoint{-0.000000in}{0.000000in}}%
\pgfpathlineto{\pgfqpoint{-0.027778in}{0.000000in}}%
\pgfusepath{stroke,fill}%
}%
\begin{pgfscope}%
\pgfsys@transformshift{7.200000in}{3.961236in}%
\pgfsys@useobject{currentmarker}{}%
\end{pgfscope}%
\end{pgfscope}%
\begin{pgfscope}%
\pgfsetbuttcap%
\pgfsetroundjoin%
\definecolor{currentfill}{rgb}{0.000000,0.000000,0.000000}%
\pgfsetfillcolor{currentfill}%
\pgfsetlinewidth{0.501875pt}%
\definecolor{currentstroke}{rgb}{0.000000,0.000000,0.000000}%
\pgfsetstrokecolor{currentstroke}%
\pgfsetdash{}{0pt}%
\pgfsys@defobject{currentmarker}{\pgfqpoint{0.000000in}{0.000000in}}{\pgfqpoint{0.027778in}{0.000000in}}{%
\pgfpathmoveto{\pgfqpoint{0.000000in}{0.000000in}}%
\pgfpathlineto{\pgfqpoint{0.027778in}{0.000000in}}%
\pgfusepath{stroke,fill}%
}%
\begin{pgfscope}%
\pgfsys@transformshift{1.000000in}{4.019382in}%
\pgfsys@useobject{currentmarker}{}%
\end{pgfscope}%
\end{pgfscope}%
\begin{pgfscope}%
\pgfsetbuttcap%
\pgfsetroundjoin%
\definecolor{currentfill}{rgb}{0.000000,0.000000,0.000000}%
\pgfsetfillcolor{currentfill}%
\pgfsetlinewidth{0.501875pt}%
\definecolor{currentstroke}{rgb}{0.000000,0.000000,0.000000}%
\pgfsetstrokecolor{currentstroke}%
\pgfsetdash{}{0pt}%
\pgfsys@defobject{currentmarker}{\pgfqpoint{-0.027778in}{0.000000in}}{\pgfqpoint{-0.000000in}{0.000000in}}{%
\pgfpathmoveto{\pgfqpoint{-0.000000in}{0.000000in}}%
\pgfpathlineto{\pgfqpoint{-0.027778in}{0.000000in}}%
\pgfusepath{stroke,fill}%
}%
\begin{pgfscope}%
\pgfsys@transformshift{7.200000in}{4.019382in}%
\pgfsys@useobject{currentmarker}{}%
\end{pgfscope}%
\end{pgfscope}%
\begin{pgfscope}%
\pgfsetbuttcap%
\pgfsetroundjoin%
\definecolor{currentfill}{rgb}{0.000000,0.000000,0.000000}%
\pgfsetfillcolor{currentfill}%
\pgfsetlinewidth{0.501875pt}%
\definecolor{currentstroke}{rgb}{0.000000,0.000000,0.000000}%
\pgfsetstrokecolor{currentstroke}%
\pgfsetdash{}{0pt}%
\pgfsys@defobject{currentmarker}{\pgfqpoint{0.000000in}{0.000000in}}{\pgfqpoint{0.027778in}{0.000000in}}{%
\pgfpathmoveto{\pgfqpoint{0.000000in}{0.000000in}}%
\pgfpathlineto{\pgfqpoint{0.027778in}{0.000000in}}%
\pgfusepath{stroke,fill}%
}%
\begin{pgfscope}%
\pgfsys@transformshift{1.000000in}{4.066891in}%
\pgfsys@useobject{currentmarker}{}%
\end{pgfscope}%
\end{pgfscope}%
\begin{pgfscope}%
\pgfsetbuttcap%
\pgfsetroundjoin%
\definecolor{currentfill}{rgb}{0.000000,0.000000,0.000000}%
\pgfsetfillcolor{currentfill}%
\pgfsetlinewidth{0.501875pt}%
\definecolor{currentstroke}{rgb}{0.000000,0.000000,0.000000}%
\pgfsetstrokecolor{currentstroke}%
\pgfsetdash{}{0pt}%
\pgfsys@defobject{currentmarker}{\pgfqpoint{-0.027778in}{0.000000in}}{\pgfqpoint{-0.000000in}{0.000000in}}{%
\pgfpathmoveto{\pgfqpoint{-0.000000in}{0.000000in}}%
\pgfpathlineto{\pgfqpoint{-0.027778in}{0.000000in}}%
\pgfusepath{stroke,fill}%
}%
\begin{pgfscope}%
\pgfsys@transformshift{7.200000in}{4.066891in}%
\pgfsys@useobject{currentmarker}{}%
\end{pgfscope}%
\end{pgfscope}%
\begin{pgfscope}%
\pgfsetbuttcap%
\pgfsetroundjoin%
\definecolor{currentfill}{rgb}{0.000000,0.000000,0.000000}%
\pgfsetfillcolor{currentfill}%
\pgfsetlinewidth{0.501875pt}%
\definecolor{currentstroke}{rgb}{0.000000,0.000000,0.000000}%
\pgfsetstrokecolor{currentstroke}%
\pgfsetdash{}{0pt}%
\pgfsys@defobject{currentmarker}{\pgfqpoint{0.000000in}{0.000000in}}{\pgfqpoint{0.027778in}{0.000000in}}{%
\pgfpathmoveto{\pgfqpoint{0.000000in}{0.000000in}}%
\pgfpathlineto{\pgfqpoint{0.027778in}{0.000000in}}%
\pgfusepath{stroke,fill}%
}%
\begin{pgfscope}%
\pgfsys@transformshift{1.000000in}{4.107059in}%
\pgfsys@useobject{currentmarker}{}%
\end{pgfscope}%
\end{pgfscope}%
\begin{pgfscope}%
\pgfsetbuttcap%
\pgfsetroundjoin%
\definecolor{currentfill}{rgb}{0.000000,0.000000,0.000000}%
\pgfsetfillcolor{currentfill}%
\pgfsetlinewidth{0.501875pt}%
\definecolor{currentstroke}{rgb}{0.000000,0.000000,0.000000}%
\pgfsetstrokecolor{currentstroke}%
\pgfsetdash{}{0pt}%
\pgfsys@defobject{currentmarker}{\pgfqpoint{-0.027778in}{0.000000in}}{\pgfqpoint{-0.000000in}{0.000000in}}{%
\pgfpathmoveto{\pgfqpoint{-0.000000in}{0.000000in}}%
\pgfpathlineto{\pgfqpoint{-0.027778in}{0.000000in}}%
\pgfusepath{stroke,fill}%
}%
\begin{pgfscope}%
\pgfsys@transformshift{7.200000in}{4.107059in}%
\pgfsys@useobject{currentmarker}{}%
\end{pgfscope}%
\end{pgfscope}%
\begin{pgfscope}%
\pgfsetbuttcap%
\pgfsetroundjoin%
\definecolor{currentfill}{rgb}{0.000000,0.000000,0.000000}%
\pgfsetfillcolor{currentfill}%
\pgfsetlinewidth{0.501875pt}%
\definecolor{currentstroke}{rgb}{0.000000,0.000000,0.000000}%
\pgfsetstrokecolor{currentstroke}%
\pgfsetdash{}{0pt}%
\pgfsys@defobject{currentmarker}{\pgfqpoint{0.000000in}{0.000000in}}{\pgfqpoint{0.027778in}{0.000000in}}{%
\pgfpathmoveto{\pgfqpoint{0.000000in}{0.000000in}}%
\pgfpathlineto{\pgfqpoint{0.027778in}{0.000000in}}%
\pgfusepath{stroke,fill}%
}%
\begin{pgfscope}%
\pgfsys@transformshift{1.000000in}{4.141854in}%
\pgfsys@useobject{currentmarker}{}%
\end{pgfscope}%
\end{pgfscope}%
\begin{pgfscope}%
\pgfsetbuttcap%
\pgfsetroundjoin%
\definecolor{currentfill}{rgb}{0.000000,0.000000,0.000000}%
\pgfsetfillcolor{currentfill}%
\pgfsetlinewidth{0.501875pt}%
\definecolor{currentstroke}{rgb}{0.000000,0.000000,0.000000}%
\pgfsetstrokecolor{currentstroke}%
\pgfsetdash{}{0pt}%
\pgfsys@defobject{currentmarker}{\pgfqpoint{-0.027778in}{0.000000in}}{\pgfqpoint{-0.000000in}{0.000000in}}{%
\pgfpathmoveto{\pgfqpoint{-0.000000in}{0.000000in}}%
\pgfpathlineto{\pgfqpoint{-0.027778in}{0.000000in}}%
\pgfusepath{stroke,fill}%
}%
\begin{pgfscope}%
\pgfsys@transformshift{7.200000in}{4.141854in}%
\pgfsys@useobject{currentmarker}{}%
\end{pgfscope}%
\end{pgfscope}%
\begin{pgfscope}%
\pgfsetbuttcap%
\pgfsetroundjoin%
\definecolor{currentfill}{rgb}{0.000000,0.000000,0.000000}%
\pgfsetfillcolor{currentfill}%
\pgfsetlinewidth{0.501875pt}%
\definecolor{currentstroke}{rgb}{0.000000,0.000000,0.000000}%
\pgfsetstrokecolor{currentstroke}%
\pgfsetdash{}{0pt}%
\pgfsys@defobject{currentmarker}{\pgfqpoint{0.000000in}{0.000000in}}{\pgfqpoint{0.027778in}{0.000000in}}{%
\pgfpathmoveto{\pgfqpoint{0.000000in}{0.000000in}}%
\pgfpathlineto{\pgfqpoint{0.027778in}{0.000000in}}%
\pgfusepath{stroke,fill}%
}%
\begin{pgfscope}%
\pgfsys@transformshift{1.000000in}{4.172546in}%
\pgfsys@useobject{currentmarker}{}%
\end{pgfscope}%
\end{pgfscope}%
\begin{pgfscope}%
\pgfsetbuttcap%
\pgfsetroundjoin%
\definecolor{currentfill}{rgb}{0.000000,0.000000,0.000000}%
\pgfsetfillcolor{currentfill}%
\pgfsetlinewidth{0.501875pt}%
\definecolor{currentstroke}{rgb}{0.000000,0.000000,0.000000}%
\pgfsetstrokecolor{currentstroke}%
\pgfsetdash{}{0pt}%
\pgfsys@defobject{currentmarker}{\pgfqpoint{-0.027778in}{0.000000in}}{\pgfqpoint{-0.000000in}{0.000000in}}{%
\pgfpathmoveto{\pgfqpoint{-0.000000in}{0.000000in}}%
\pgfpathlineto{\pgfqpoint{-0.027778in}{0.000000in}}%
\pgfusepath{stroke,fill}%
}%
\begin{pgfscope}%
\pgfsys@transformshift{7.200000in}{4.172546in}%
\pgfsys@useobject{currentmarker}{}%
\end{pgfscope}%
\end{pgfscope}%
\begin{pgfscope}%
\pgfsetbuttcap%
\pgfsetroundjoin%
\definecolor{currentfill}{rgb}{0.000000,0.000000,0.000000}%
\pgfsetfillcolor{currentfill}%
\pgfsetlinewidth{0.501875pt}%
\definecolor{currentstroke}{rgb}{0.000000,0.000000,0.000000}%
\pgfsetstrokecolor{currentstroke}%
\pgfsetdash{}{0pt}%
\pgfsys@defobject{currentmarker}{\pgfqpoint{0.000000in}{0.000000in}}{\pgfqpoint{0.027778in}{0.000000in}}{%
\pgfpathmoveto{\pgfqpoint{0.000000in}{0.000000in}}%
\pgfpathlineto{\pgfqpoint{0.027778in}{0.000000in}}%
\pgfusepath{stroke,fill}%
}%
\begin{pgfscope}%
\pgfsys@transformshift{1.000000in}{4.380618in}%
\pgfsys@useobject{currentmarker}{}%
\end{pgfscope}%
\end{pgfscope}%
\begin{pgfscope}%
\pgfsetbuttcap%
\pgfsetroundjoin%
\definecolor{currentfill}{rgb}{0.000000,0.000000,0.000000}%
\pgfsetfillcolor{currentfill}%
\pgfsetlinewidth{0.501875pt}%
\definecolor{currentstroke}{rgb}{0.000000,0.000000,0.000000}%
\pgfsetstrokecolor{currentstroke}%
\pgfsetdash{}{0pt}%
\pgfsys@defobject{currentmarker}{\pgfqpoint{-0.027778in}{0.000000in}}{\pgfqpoint{-0.000000in}{0.000000in}}{%
\pgfpathmoveto{\pgfqpoint{-0.000000in}{0.000000in}}%
\pgfpathlineto{\pgfqpoint{-0.027778in}{0.000000in}}%
\pgfusepath{stroke,fill}%
}%
\begin{pgfscope}%
\pgfsys@transformshift{7.200000in}{4.380618in}%
\pgfsys@useobject{currentmarker}{}%
\end{pgfscope}%
\end{pgfscope}%
\begin{pgfscope}%
\pgfsetbuttcap%
\pgfsetroundjoin%
\definecolor{currentfill}{rgb}{0.000000,0.000000,0.000000}%
\pgfsetfillcolor{currentfill}%
\pgfsetlinewidth{0.501875pt}%
\definecolor{currentstroke}{rgb}{0.000000,0.000000,0.000000}%
\pgfsetstrokecolor{currentstroke}%
\pgfsetdash{}{0pt}%
\pgfsys@defobject{currentmarker}{\pgfqpoint{0.000000in}{0.000000in}}{\pgfqpoint{0.027778in}{0.000000in}}{%
\pgfpathmoveto{\pgfqpoint{0.000000in}{0.000000in}}%
\pgfpathlineto{\pgfqpoint{0.027778in}{0.000000in}}%
\pgfusepath{stroke,fill}%
}%
\begin{pgfscope}%
\pgfsys@transformshift{1.000000in}{4.486273in}%
\pgfsys@useobject{currentmarker}{}%
\end{pgfscope}%
\end{pgfscope}%
\begin{pgfscope}%
\pgfsetbuttcap%
\pgfsetroundjoin%
\definecolor{currentfill}{rgb}{0.000000,0.000000,0.000000}%
\pgfsetfillcolor{currentfill}%
\pgfsetlinewidth{0.501875pt}%
\definecolor{currentstroke}{rgb}{0.000000,0.000000,0.000000}%
\pgfsetstrokecolor{currentstroke}%
\pgfsetdash{}{0pt}%
\pgfsys@defobject{currentmarker}{\pgfqpoint{-0.027778in}{0.000000in}}{\pgfqpoint{-0.000000in}{0.000000in}}{%
\pgfpathmoveto{\pgfqpoint{-0.000000in}{0.000000in}}%
\pgfpathlineto{\pgfqpoint{-0.027778in}{0.000000in}}%
\pgfusepath{stroke,fill}%
}%
\begin{pgfscope}%
\pgfsys@transformshift{7.200000in}{4.486273in}%
\pgfsys@useobject{currentmarker}{}%
\end{pgfscope}%
\end{pgfscope}%
\begin{pgfscope}%
\pgfsetbuttcap%
\pgfsetroundjoin%
\definecolor{currentfill}{rgb}{0.000000,0.000000,0.000000}%
\pgfsetfillcolor{currentfill}%
\pgfsetlinewidth{0.501875pt}%
\definecolor{currentstroke}{rgb}{0.000000,0.000000,0.000000}%
\pgfsetstrokecolor{currentstroke}%
\pgfsetdash{}{0pt}%
\pgfsys@defobject{currentmarker}{\pgfqpoint{0.000000in}{0.000000in}}{\pgfqpoint{0.027778in}{0.000000in}}{%
\pgfpathmoveto{\pgfqpoint{0.000000in}{0.000000in}}%
\pgfpathlineto{\pgfqpoint{0.027778in}{0.000000in}}%
\pgfusepath{stroke,fill}%
}%
\begin{pgfscope}%
\pgfsys@transformshift{1.000000in}{4.561236in}%
\pgfsys@useobject{currentmarker}{}%
\end{pgfscope}%
\end{pgfscope}%
\begin{pgfscope}%
\pgfsetbuttcap%
\pgfsetroundjoin%
\definecolor{currentfill}{rgb}{0.000000,0.000000,0.000000}%
\pgfsetfillcolor{currentfill}%
\pgfsetlinewidth{0.501875pt}%
\definecolor{currentstroke}{rgb}{0.000000,0.000000,0.000000}%
\pgfsetstrokecolor{currentstroke}%
\pgfsetdash{}{0pt}%
\pgfsys@defobject{currentmarker}{\pgfqpoint{-0.027778in}{0.000000in}}{\pgfqpoint{-0.000000in}{0.000000in}}{%
\pgfpathmoveto{\pgfqpoint{-0.000000in}{0.000000in}}%
\pgfpathlineto{\pgfqpoint{-0.027778in}{0.000000in}}%
\pgfusepath{stroke,fill}%
}%
\begin{pgfscope}%
\pgfsys@transformshift{7.200000in}{4.561236in}%
\pgfsys@useobject{currentmarker}{}%
\end{pgfscope}%
\end{pgfscope}%
\begin{pgfscope}%
\pgfsetbuttcap%
\pgfsetroundjoin%
\definecolor{currentfill}{rgb}{0.000000,0.000000,0.000000}%
\pgfsetfillcolor{currentfill}%
\pgfsetlinewidth{0.501875pt}%
\definecolor{currentstroke}{rgb}{0.000000,0.000000,0.000000}%
\pgfsetstrokecolor{currentstroke}%
\pgfsetdash{}{0pt}%
\pgfsys@defobject{currentmarker}{\pgfqpoint{0.000000in}{0.000000in}}{\pgfqpoint{0.027778in}{0.000000in}}{%
\pgfpathmoveto{\pgfqpoint{0.000000in}{0.000000in}}%
\pgfpathlineto{\pgfqpoint{0.027778in}{0.000000in}}%
\pgfusepath{stroke,fill}%
}%
\begin{pgfscope}%
\pgfsys@transformshift{1.000000in}{4.619382in}%
\pgfsys@useobject{currentmarker}{}%
\end{pgfscope}%
\end{pgfscope}%
\begin{pgfscope}%
\pgfsetbuttcap%
\pgfsetroundjoin%
\definecolor{currentfill}{rgb}{0.000000,0.000000,0.000000}%
\pgfsetfillcolor{currentfill}%
\pgfsetlinewidth{0.501875pt}%
\definecolor{currentstroke}{rgb}{0.000000,0.000000,0.000000}%
\pgfsetstrokecolor{currentstroke}%
\pgfsetdash{}{0pt}%
\pgfsys@defobject{currentmarker}{\pgfqpoint{-0.027778in}{0.000000in}}{\pgfqpoint{-0.000000in}{0.000000in}}{%
\pgfpathmoveto{\pgfqpoint{-0.000000in}{0.000000in}}%
\pgfpathlineto{\pgfqpoint{-0.027778in}{0.000000in}}%
\pgfusepath{stroke,fill}%
}%
\begin{pgfscope}%
\pgfsys@transformshift{7.200000in}{4.619382in}%
\pgfsys@useobject{currentmarker}{}%
\end{pgfscope}%
\end{pgfscope}%
\begin{pgfscope}%
\pgfsetbuttcap%
\pgfsetroundjoin%
\definecolor{currentfill}{rgb}{0.000000,0.000000,0.000000}%
\pgfsetfillcolor{currentfill}%
\pgfsetlinewidth{0.501875pt}%
\definecolor{currentstroke}{rgb}{0.000000,0.000000,0.000000}%
\pgfsetstrokecolor{currentstroke}%
\pgfsetdash{}{0pt}%
\pgfsys@defobject{currentmarker}{\pgfqpoint{0.000000in}{0.000000in}}{\pgfqpoint{0.027778in}{0.000000in}}{%
\pgfpathmoveto{\pgfqpoint{0.000000in}{0.000000in}}%
\pgfpathlineto{\pgfqpoint{0.027778in}{0.000000in}}%
\pgfusepath{stroke,fill}%
}%
\begin{pgfscope}%
\pgfsys@transformshift{1.000000in}{4.666891in}%
\pgfsys@useobject{currentmarker}{}%
\end{pgfscope}%
\end{pgfscope}%
\begin{pgfscope}%
\pgfsetbuttcap%
\pgfsetroundjoin%
\definecolor{currentfill}{rgb}{0.000000,0.000000,0.000000}%
\pgfsetfillcolor{currentfill}%
\pgfsetlinewidth{0.501875pt}%
\definecolor{currentstroke}{rgb}{0.000000,0.000000,0.000000}%
\pgfsetstrokecolor{currentstroke}%
\pgfsetdash{}{0pt}%
\pgfsys@defobject{currentmarker}{\pgfqpoint{-0.027778in}{0.000000in}}{\pgfqpoint{-0.000000in}{0.000000in}}{%
\pgfpathmoveto{\pgfqpoint{-0.000000in}{0.000000in}}%
\pgfpathlineto{\pgfqpoint{-0.027778in}{0.000000in}}%
\pgfusepath{stroke,fill}%
}%
\begin{pgfscope}%
\pgfsys@transformshift{7.200000in}{4.666891in}%
\pgfsys@useobject{currentmarker}{}%
\end{pgfscope}%
\end{pgfscope}%
\begin{pgfscope}%
\pgfsetbuttcap%
\pgfsetroundjoin%
\definecolor{currentfill}{rgb}{0.000000,0.000000,0.000000}%
\pgfsetfillcolor{currentfill}%
\pgfsetlinewidth{0.501875pt}%
\definecolor{currentstroke}{rgb}{0.000000,0.000000,0.000000}%
\pgfsetstrokecolor{currentstroke}%
\pgfsetdash{}{0pt}%
\pgfsys@defobject{currentmarker}{\pgfqpoint{0.000000in}{0.000000in}}{\pgfqpoint{0.027778in}{0.000000in}}{%
\pgfpathmoveto{\pgfqpoint{0.000000in}{0.000000in}}%
\pgfpathlineto{\pgfqpoint{0.027778in}{0.000000in}}%
\pgfusepath{stroke,fill}%
}%
\begin{pgfscope}%
\pgfsys@transformshift{1.000000in}{4.707059in}%
\pgfsys@useobject{currentmarker}{}%
\end{pgfscope}%
\end{pgfscope}%
\begin{pgfscope}%
\pgfsetbuttcap%
\pgfsetroundjoin%
\definecolor{currentfill}{rgb}{0.000000,0.000000,0.000000}%
\pgfsetfillcolor{currentfill}%
\pgfsetlinewidth{0.501875pt}%
\definecolor{currentstroke}{rgb}{0.000000,0.000000,0.000000}%
\pgfsetstrokecolor{currentstroke}%
\pgfsetdash{}{0pt}%
\pgfsys@defobject{currentmarker}{\pgfqpoint{-0.027778in}{0.000000in}}{\pgfqpoint{-0.000000in}{0.000000in}}{%
\pgfpathmoveto{\pgfqpoint{-0.000000in}{0.000000in}}%
\pgfpathlineto{\pgfqpoint{-0.027778in}{0.000000in}}%
\pgfusepath{stroke,fill}%
}%
\begin{pgfscope}%
\pgfsys@transformshift{7.200000in}{4.707059in}%
\pgfsys@useobject{currentmarker}{}%
\end{pgfscope}%
\end{pgfscope}%
\begin{pgfscope}%
\pgfsetbuttcap%
\pgfsetroundjoin%
\definecolor{currentfill}{rgb}{0.000000,0.000000,0.000000}%
\pgfsetfillcolor{currentfill}%
\pgfsetlinewidth{0.501875pt}%
\definecolor{currentstroke}{rgb}{0.000000,0.000000,0.000000}%
\pgfsetstrokecolor{currentstroke}%
\pgfsetdash{}{0pt}%
\pgfsys@defobject{currentmarker}{\pgfqpoint{0.000000in}{0.000000in}}{\pgfqpoint{0.027778in}{0.000000in}}{%
\pgfpathmoveto{\pgfqpoint{0.000000in}{0.000000in}}%
\pgfpathlineto{\pgfqpoint{0.027778in}{0.000000in}}%
\pgfusepath{stroke,fill}%
}%
\begin{pgfscope}%
\pgfsys@transformshift{1.000000in}{4.741854in}%
\pgfsys@useobject{currentmarker}{}%
\end{pgfscope}%
\end{pgfscope}%
\begin{pgfscope}%
\pgfsetbuttcap%
\pgfsetroundjoin%
\definecolor{currentfill}{rgb}{0.000000,0.000000,0.000000}%
\pgfsetfillcolor{currentfill}%
\pgfsetlinewidth{0.501875pt}%
\definecolor{currentstroke}{rgb}{0.000000,0.000000,0.000000}%
\pgfsetstrokecolor{currentstroke}%
\pgfsetdash{}{0pt}%
\pgfsys@defobject{currentmarker}{\pgfqpoint{-0.027778in}{0.000000in}}{\pgfqpoint{-0.000000in}{0.000000in}}{%
\pgfpathmoveto{\pgfqpoint{-0.000000in}{0.000000in}}%
\pgfpathlineto{\pgfqpoint{-0.027778in}{0.000000in}}%
\pgfusepath{stroke,fill}%
}%
\begin{pgfscope}%
\pgfsys@transformshift{7.200000in}{4.741854in}%
\pgfsys@useobject{currentmarker}{}%
\end{pgfscope}%
\end{pgfscope}%
\begin{pgfscope}%
\pgfsetbuttcap%
\pgfsetroundjoin%
\definecolor{currentfill}{rgb}{0.000000,0.000000,0.000000}%
\pgfsetfillcolor{currentfill}%
\pgfsetlinewidth{0.501875pt}%
\definecolor{currentstroke}{rgb}{0.000000,0.000000,0.000000}%
\pgfsetstrokecolor{currentstroke}%
\pgfsetdash{}{0pt}%
\pgfsys@defobject{currentmarker}{\pgfqpoint{0.000000in}{0.000000in}}{\pgfqpoint{0.027778in}{0.000000in}}{%
\pgfpathmoveto{\pgfqpoint{0.000000in}{0.000000in}}%
\pgfpathlineto{\pgfqpoint{0.027778in}{0.000000in}}%
\pgfusepath{stroke,fill}%
}%
\begin{pgfscope}%
\pgfsys@transformshift{1.000000in}{4.772546in}%
\pgfsys@useobject{currentmarker}{}%
\end{pgfscope}%
\end{pgfscope}%
\begin{pgfscope}%
\pgfsetbuttcap%
\pgfsetroundjoin%
\definecolor{currentfill}{rgb}{0.000000,0.000000,0.000000}%
\pgfsetfillcolor{currentfill}%
\pgfsetlinewidth{0.501875pt}%
\definecolor{currentstroke}{rgb}{0.000000,0.000000,0.000000}%
\pgfsetstrokecolor{currentstroke}%
\pgfsetdash{}{0pt}%
\pgfsys@defobject{currentmarker}{\pgfqpoint{-0.027778in}{0.000000in}}{\pgfqpoint{-0.000000in}{0.000000in}}{%
\pgfpathmoveto{\pgfqpoint{-0.000000in}{0.000000in}}%
\pgfpathlineto{\pgfqpoint{-0.027778in}{0.000000in}}%
\pgfusepath{stroke,fill}%
}%
\begin{pgfscope}%
\pgfsys@transformshift{7.200000in}{4.772546in}%
\pgfsys@useobject{currentmarker}{}%
\end{pgfscope}%
\end{pgfscope}%
\begin{pgfscope}%
\pgfsetbuttcap%
\pgfsetroundjoin%
\definecolor{currentfill}{rgb}{0.000000,0.000000,0.000000}%
\pgfsetfillcolor{currentfill}%
\pgfsetlinewidth{0.501875pt}%
\definecolor{currentstroke}{rgb}{0.000000,0.000000,0.000000}%
\pgfsetstrokecolor{currentstroke}%
\pgfsetdash{}{0pt}%
\pgfsys@defobject{currentmarker}{\pgfqpoint{0.000000in}{0.000000in}}{\pgfqpoint{0.027778in}{0.000000in}}{%
\pgfpathmoveto{\pgfqpoint{0.000000in}{0.000000in}}%
\pgfpathlineto{\pgfqpoint{0.027778in}{0.000000in}}%
\pgfusepath{stroke,fill}%
}%
\begin{pgfscope}%
\pgfsys@transformshift{1.000000in}{4.980618in}%
\pgfsys@useobject{currentmarker}{}%
\end{pgfscope}%
\end{pgfscope}%
\begin{pgfscope}%
\pgfsetbuttcap%
\pgfsetroundjoin%
\definecolor{currentfill}{rgb}{0.000000,0.000000,0.000000}%
\pgfsetfillcolor{currentfill}%
\pgfsetlinewidth{0.501875pt}%
\definecolor{currentstroke}{rgb}{0.000000,0.000000,0.000000}%
\pgfsetstrokecolor{currentstroke}%
\pgfsetdash{}{0pt}%
\pgfsys@defobject{currentmarker}{\pgfqpoint{-0.027778in}{0.000000in}}{\pgfqpoint{-0.000000in}{0.000000in}}{%
\pgfpathmoveto{\pgfqpoint{-0.000000in}{0.000000in}}%
\pgfpathlineto{\pgfqpoint{-0.027778in}{0.000000in}}%
\pgfusepath{stroke,fill}%
}%
\begin{pgfscope}%
\pgfsys@transformshift{7.200000in}{4.980618in}%
\pgfsys@useobject{currentmarker}{}%
\end{pgfscope}%
\end{pgfscope}%
\begin{pgfscope}%
\pgfsetbuttcap%
\pgfsetroundjoin%
\definecolor{currentfill}{rgb}{0.000000,0.000000,0.000000}%
\pgfsetfillcolor{currentfill}%
\pgfsetlinewidth{0.501875pt}%
\definecolor{currentstroke}{rgb}{0.000000,0.000000,0.000000}%
\pgfsetstrokecolor{currentstroke}%
\pgfsetdash{}{0pt}%
\pgfsys@defobject{currentmarker}{\pgfqpoint{0.000000in}{0.000000in}}{\pgfqpoint{0.027778in}{0.000000in}}{%
\pgfpathmoveto{\pgfqpoint{0.000000in}{0.000000in}}%
\pgfpathlineto{\pgfqpoint{0.027778in}{0.000000in}}%
\pgfusepath{stroke,fill}%
}%
\begin{pgfscope}%
\pgfsys@transformshift{1.000000in}{5.086273in}%
\pgfsys@useobject{currentmarker}{}%
\end{pgfscope}%
\end{pgfscope}%
\begin{pgfscope}%
\pgfsetbuttcap%
\pgfsetroundjoin%
\definecolor{currentfill}{rgb}{0.000000,0.000000,0.000000}%
\pgfsetfillcolor{currentfill}%
\pgfsetlinewidth{0.501875pt}%
\definecolor{currentstroke}{rgb}{0.000000,0.000000,0.000000}%
\pgfsetstrokecolor{currentstroke}%
\pgfsetdash{}{0pt}%
\pgfsys@defobject{currentmarker}{\pgfqpoint{-0.027778in}{0.000000in}}{\pgfqpoint{-0.000000in}{0.000000in}}{%
\pgfpathmoveto{\pgfqpoint{-0.000000in}{0.000000in}}%
\pgfpathlineto{\pgfqpoint{-0.027778in}{0.000000in}}%
\pgfusepath{stroke,fill}%
}%
\begin{pgfscope}%
\pgfsys@transformshift{7.200000in}{5.086273in}%
\pgfsys@useobject{currentmarker}{}%
\end{pgfscope}%
\end{pgfscope}%
\begin{pgfscope}%
\pgfsetbuttcap%
\pgfsetroundjoin%
\definecolor{currentfill}{rgb}{0.000000,0.000000,0.000000}%
\pgfsetfillcolor{currentfill}%
\pgfsetlinewidth{0.501875pt}%
\definecolor{currentstroke}{rgb}{0.000000,0.000000,0.000000}%
\pgfsetstrokecolor{currentstroke}%
\pgfsetdash{}{0pt}%
\pgfsys@defobject{currentmarker}{\pgfqpoint{0.000000in}{0.000000in}}{\pgfqpoint{0.027778in}{0.000000in}}{%
\pgfpathmoveto{\pgfqpoint{0.000000in}{0.000000in}}%
\pgfpathlineto{\pgfqpoint{0.027778in}{0.000000in}}%
\pgfusepath{stroke,fill}%
}%
\begin{pgfscope}%
\pgfsys@transformshift{1.000000in}{5.161236in}%
\pgfsys@useobject{currentmarker}{}%
\end{pgfscope}%
\end{pgfscope}%
\begin{pgfscope}%
\pgfsetbuttcap%
\pgfsetroundjoin%
\definecolor{currentfill}{rgb}{0.000000,0.000000,0.000000}%
\pgfsetfillcolor{currentfill}%
\pgfsetlinewidth{0.501875pt}%
\definecolor{currentstroke}{rgb}{0.000000,0.000000,0.000000}%
\pgfsetstrokecolor{currentstroke}%
\pgfsetdash{}{0pt}%
\pgfsys@defobject{currentmarker}{\pgfqpoint{-0.027778in}{0.000000in}}{\pgfqpoint{-0.000000in}{0.000000in}}{%
\pgfpathmoveto{\pgfqpoint{-0.000000in}{0.000000in}}%
\pgfpathlineto{\pgfqpoint{-0.027778in}{0.000000in}}%
\pgfusepath{stroke,fill}%
}%
\begin{pgfscope}%
\pgfsys@transformshift{7.200000in}{5.161236in}%
\pgfsys@useobject{currentmarker}{}%
\end{pgfscope}%
\end{pgfscope}%
\begin{pgfscope}%
\pgfsetbuttcap%
\pgfsetroundjoin%
\definecolor{currentfill}{rgb}{0.000000,0.000000,0.000000}%
\pgfsetfillcolor{currentfill}%
\pgfsetlinewidth{0.501875pt}%
\definecolor{currentstroke}{rgb}{0.000000,0.000000,0.000000}%
\pgfsetstrokecolor{currentstroke}%
\pgfsetdash{}{0pt}%
\pgfsys@defobject{currentmarker}{\pgfqpoint{0.000000in}{0.000000in}}{\pgfqpoint{0.027778in}{0.000000in}}{%
\pgfpathmoveto{\pgfqpoint{0.000000in}{0.000000in}}%
\pgfpathlineto{\pgfqpoint{0.027778in}{0.000000in}}%
\pgfusepath{stroke,fill}%
}%
\begin{pgfscope}%
\pgfsys@transformshift{1.000000in}{5.219382in}%
\pgfsys@useobject{currentmarker}{}%
\end{pgfscope}%
\end{pgfscope}%
\begin{pgfscope}%
\pgfsetbuttcap%
\pgfsetroundjoin%
\definecolor{currentfill}{rgb}{0.000000,0.000000,0.000000}%
\pgfsetfillcolor{currentfill}%
\pgfsetlinewidth{0.501875pt}%
\definecolor{currentstroke}{rgb}{0.000000,0.000000,0.000000}%
\pgfsetstrokecolor{currentstroke}%
\pgfsetdash{}{0pt}%
\pgfsys@defobject{currentmarker}{\pgfqpoint{-0.027778in}{0.000000in}}{\pgfqpoint{-0.000000in}{0.000000in}}{%
\pgfpathmoveto{\pgfqpoint{-0.000000in}{0.000000in}}%
\pgfpathlineto{\pgfqpoint{-0.027778in}{0.000000in}}%
\pgfusepath{stroke,fill}%
}%
\begin{pgfscope}%
\pgfsys@transformshift{7.200000in}{5.219382in}%
\pgfsys@useobject{currentmarker}{}%
\end{pgfscope}%
\end{pgfscope}%
\begin{pgfscope}%
\pgfsetbuttcap%
\pgfsetroundjoin%
\definecolor{currentfill}{rgb}{0.000000,0.000000,0.000000}%
\pgfsetfillcolor{currentfill}%
\pgfsetlinewidth{0.501875pt}%
\definecolor{currentstroke}{rgb}{0.000000,0.000000,0.000000}%
\pgfsetstrokecolor{currentstroke}%
\pgfsetdash{}{0pt}%
\pgfsys@defobject{currentmarker}{\pgfqpoint{0.000000in}{0.000000in}}{\pgfqpoint{0.027778in}{0.000000in}}{%
\pgfpathmoveto{\pgfqpoint{0.000000in}{0.000000in}}%
\pgfpathlineto{\pgfqpoint{0.027778in}{0.000000in}}%
\pgfusepath{stroke,fill}%
}%
\begin{pgfscope}%
\pgfsys@transformshift{1.000000in}{5.266891in}%
\pgfsys@useobject{currentmarker}{}%
\end{pgfscope}%
\end{pgfscope}%
\begin{pgfscope}%
\pgfsetbuttcap%
\pgfsetroundjoin%
\definecolor{currentfill}{rgb}{0.000000,0.000000,0.000000}%
\pgfsetfillcolor{currentfill}%
\pgfsetlinewidth{0.501875pt}%
\definecolor{currentstroke}{rgb}{0.000000,0.000000,0.000000}%
\pgfsetstrokecolor{currentstroke}%
\pgfsetdash{}{0pt}%
\pgfsys@defobject{currentmarker}{\pgfqpoint{-0.027778in}{0.000000in}}{\pgfqpoint{-0.000000in}{0.000000in}}{%
\pgfpathmoveto{\pgfqpoint{-0.000000in}{0.000000in}}%
\pgfpathlineto{\pgfqpoint{-0.027778in}{0.000000in}}%
\pgfusepath{stroke,fill}%
}%
\begin{pgfscope}%
\pgfsys@transformshift{7.200000in}{5.266891in}%
\pgfsys@useobject{currentmarker}{}%
\end{pgfscope}%
\end{pgfscope}%
\begin{pgfscope}%
\pgfsetbuttcap%
\pgfsetroundjoin%
\definecolor{currentfill}{rgb}{0.000000,0.000000,0.000000}%
\pgfsetfillcolor{currentfill}%
\pgfsetlinewidth{0.501875pt}%
\definecolor{currentstroke}{rgb}{0.000000,0.000000,0.000000}%
\pgfsetstrokecolor{currentstroke}%
\pgfsetdash{}{0pt}%
\pgfsys@defobject{currentmarker}{\pgfqpoint{0.000000in}{0.000000in}}{\pgfqpoint{0.027778in}{0.000000in}}{%
\pgfpathmoveto{\pgfqpoint{0.000000in}{0.000000in}}%
\pgfpathlineto{\pgfqpoint{0.027778in}{0.000000in}}%
\pgfusepath{stroke,fill}%
}%
\begin{pgfscope}%
\pgfsys@transformshift{1.000000in}{5.307059in}%
\pgfsys@useobject{currentmarker}{}%
\end{pgfscope}%
\end{pgfscope}%
\begin{pgfscope}%
\pgfsetbuttcap%
\pgfsetroundjoin%
\definecolor{currentfill}{rgb}{0.000000,0.000000,0.000000}%
\pgfsetfillcolor{currentfill}%
\pgfsetlinewidth{0.501875pt}%
\definecolor{currentstroke}{rgb}{0.000000,0.000000,0.000000}%
\pgfsetstrokecolor{currentstroke}%
\pgfsetdash{}{0pt}%
\pgfsys@defobject{currentmarker}{\pgfqpoint{-0.027778in}{0.000000in}}{\pgfqpoint{-0.000000in}{0.000000in}}{%
\pgfpathmoveto{\pgfqpoint{-0.000000in}{0.000000in}}%
\pgfpathlineto{\pgfqpoint{-0.027778in}{0.000000in}}%
\pgfusepath{stroke,fill}%
}%
\begin{pgfscope}%
\pgfsys@transformshift{7.200000in}{5.307059in}%
\pgfsys@useobject{currentmarker}{}%
\end{pgfscope}%
\end{pgfscope}%
\begin{pgfscope}%
\pgfsetbuttcap%
\pgfsetroundjoin%
\definecolor{currentfill}{rgb}{0.000000,0.000000,0.000000}%
\pgfsetfillcolor{currentfill}%
\pgfsetlinewidth{0.501875pt}%
\definecolor{currentstroke}{rgb}{0.000000,0.000000,0.000000}%
\pgfsetstrokecolor{currentstroke}%
\pgfsetdash{}{0pt}%
\pgfsys@defobject{currentmarker}{\pgfqpoint{0.000000in}{0.000000in}}{\pgfqpoint{0.027778in}{0.000000in}}{%
\pgfpathmoveto{\pgfqpoint{0.000000in}{0.000000in}}%
\pgfpathlineto{\pgfqpoint{0.027778in}{0.000000in}}%
\pgfusepath{stroke,fill}%
}%
\begin{pgfscope}%
\pgfsys@transformshift{1.000000in}{5.341854in}%
\pgfsys@useobject{currentmarker}{}%
\end{pgfscope}%
\end{pgfscope}%
\begin{pgfscope}%
\pgfsetbuttcap%
\pgfsetroundjoin%
\definecolor{currentfill}{rgb}{0.000000,0.000000,0.000000}%
\pgfsetfillcolor{currentfill}%
\pgfsetlinewidth{0.501875pt}%
\definecolor{currentstroke}{rgb}{0.000000,0.000000,0.000000}%
\pgfsetstrokecolor{currentstroke}%
\pgfsetdash{}{0pt}%
\pgfsys@defobject{currentmarker}{\pgfqpoint{-0.027778in}{0.000000in}}{\pgfqpoint{-0.000000in}{0.000000in}}{%
\pgfpathmoveto{\pgfqpoint{-0.000000in}{0.000000in}}%
\pgfpathlineto{\pgfqpoint{-0.027778in}{0.000000in}}%
\pgfusepath{stroke,fill}%
}%
\begin{pgfscope}%
\pgfsys@transformshift{7.200000in}{5.341854in}%
\pgfsys@useobject{currentmarker}{}%
\end{pgfscope}%
\end{pgfscope}%
\begin{pgfscope}%
\pgfsetbuttcap%
\pgfsetroundjoin%
\definecolor{currentfill}{rgb}{0.000000,0.000000,0.000000}%
\pgfsetfillcolor{currentfill}%
\pgfsetlinewidth{0.501875pt}%
\definecolor{currentstroke}{rgb}{0.000000,0.000000,0.000000}%
\pgfsetstrokecolor{currentstroke}%
\pgfsetdash{}{0pt}%
\pgfsys@defobject{currentmarker}{\pgfqpoint{0.000000in}{0.000000in}}{\pgfqpoint{0.027778in}{0.000000in}}{%
\pgfpathmoveto{\pgfqpoint{0.000000in}{0.000000in}}%
\pgfpathlineto{\pgfqpoint{0.027778in}{0.000000in}}%
\pgfusepath{stroke,fill}%
}%
\begin{pgfscope}%
\pgfsys@transformshift{1.000000in}{5.372546in}%
\pgfsys@useobject{currentmarker}{}%
\end{pgfscope}%
\end{pgfscope}%
\begin{pgfscope}%
\pgfsetbuttcap%
\pgfsetroundjoin%
\definecolor{currentfill}{rgb}{0.000000,0.000000,0.000000}%
\pgfsetfillcolor{currentfill}%
\pgfsetlinewidth{0.501875pt}%
\definecolor{currentstroke}{rgb}{0.000000,0.000000,0.000000}%
\pgfsetstrokecolor{currentstroke}%
\pgfsetdash{}{0pt}%
\pgfsys@defobject{currentmarker}{\pgfqpoint{-0.027778in}{0.000000in}}{\pgfqpoint{-0.000000in}{0.000000in}}{%
\pgfpathmoveto{\pgfqpoint{-0.000000in}{0.000000in}}%
\pgfpathlineto{\pgfqpoint{-0.027778in}{0.000000in}}%
\pgfusepath{stroke,fill}%
}%
\begin{pgfscope}%
\pgfsys@transformshift{7.200000in}{5.372546in}%
\pgfsys@useobject{currentmarker}{}%
\end{pgfscope}%
\end{pgfscope}%
\begin{pgfscope}%
\definecolor{textcolor}{rgb}{0.000000,0.000000,0.000000}%
\pgfsetstrokecolor{textcolor}%
\pgfsetfillcolor{textcolor}%
\pgftext[x=0.586997in,y=3.000000in,,bottom,rotate=90.000000]{\color{textcolor}\rmfamily\fontsize{12.000000}{14.400000}\selectfont \(\displaystyle error\ in\ quadrature\)}%
\end{pgfscope}%
\begin{pgfscope}%
\definecolor{textcolor}{rgb}{0.000000,0.000000,0.000000}%
\pgfsetstrokecolor{textcolor}%
\pgfsetfillcolor{textcolor}%
\pgftext[x=4.100000in,y=5.469444in,,base]{\color{textcolor}\rmfamily\fontsize{12.000000}{14.400000}\selectfont \(\displaystyle Quadrature\ convergence\)}%
\end{pgfscope}%
\begin{pgfscope}%
\pgfsetbuttcap%
\pgfsetmiterjoin%
\definecolor{currentfill}{rgb}{1.000000,1.000000,1.000000}%
\pgfsetfillcolor{currentfill}%
\pgfsetlinewidth{1.003750pt}%
\definecolor{currentstroke}{rgb}{0.000000,0.000000,0.000000}%
\pgfsetstrokecolor{currentstroke}%
\pgfsetdash{}{0pt}%
\pgfpathmoveto{\pgfqpoint{1.083333in}{0.683333in}}%
\pgfpathlineto{\pgfqpoint{3.514064in}{0.683333in}}%
\pgfpathlineto{\pgfqpoint{3.514064in}{1.231161in}}%
\pgfpathlineto{\pgfqpoint{1.083333in}{1.231161in}}%
\pgfpathlineto{\pgfqpoint{1.083333in}{0.683333in}}%
\pgfpathclose%
\pgfusepath{stroke,fill}%
\end{pgfscope}%
\begin{pgfscope}%
\pgfsetbuttcap%
\pgfsetroundjoin%
\pgfsetlinewidth{1.003750pt}%
\definecolor{currentstroke}{rgb}{0.750000,0.000000,0.750000}%
\pgfsetstrokecolor{currentstroke}%
\pgfsetdash{{6.000000pt}{6.000000pt}}{0.000000pt}%
\pgfpathmoveto{\pgfqpoint{1.200000in}{1.106161in}}%
\pgfpathlineto{\pgfqpoint{1.433333in}{1.106161in}}%
\pgfusepath{stroke}%
\end{pgfscope}%
\begin{pgfscope}%
\pgfsetbuttcap%
\pgfsetroundjoin%
\definecolor{currentfill}{rgb}{0.750000,0.000000,0.750000}%
\pgfsetfillcolor{currentfill}%
\pgfsetlinewidth{0.501875pt}%
\definecolor{currentstroke}{rgb}{0.750000,0.000000,0.750000}%
\pgfsetstrokecolor{currentstroke}%
\pgfsetdash{}{0pt}%
\pgfsys@defobject{currentmarker}{\pgfqpoint{-0.020833in}{-0.020833in}}{\pgfqpoint{0.020833in}{0.020833in}}{%
\pgfpathmoveto{\pgfqpoint{0.000000in}{-0.020833in}}%
\pgfpathcurveto{\pgfqpoint{0.005525in}{-0.020833in}}{\pgfqpoint{0.010825in}{-0.018638in}}{\pgfqpoint{0.014731in}{-0.014731in}}%
\pgfpathcurveto{\pgfqpoint{0.018638in}{-0.010825in}}{\pgfqpoint{0.020833in}{-0.005525in}}{\pgfqpoint{0.020833in}{0.000000in}}%
\pgfpathcurveto{\pgfqpoint{0.020833in}{0.005525in}}{\pgfqpoint{0.018638in}{0.010825in}}{\pgfqpoint{0.014731in}{0.014731in}}%
\pgfpathcurveto{\pgfqpoint{0.010825in}{0.018638in}}{\pgfqpoint{0.005525in}{0.020833in}}{\pgfqpoint{0.000000in}{0.020833in}}%
\pgfpathcurveto{\pgfqpoint{-0.005525in}{0.020833in}}{\pgfqpoint{-0.010825in}{0.018638in}}{\pgfqpoint{-0.014731in}{0.014731in}}%
\pgfpathcurveto{\pgfqpoint{-0.018638in}{0.010825in}}{\pgfqpoint{-0.020833in}{0.005525in}}{\pgfqpoint{-0.020833in}{0.000000in}}%
\pgfpathcurveto{\pgfqpoint{-0.020833in}{-0.005525in}}{\pgfqpoint{-0.018638in}{-0.010825in}}{\pgfqpoint{-0.014731in}{-0.014731in}}%
\pgfpathcurveto{\pgfqpoint{-0.010825in}{-0.018638in}}{\pgfqpoint{-0.005525in}{-0.020833in}}{\pgfqpoint{0.000000in}{-0.020833in}}%
\pgfpathlineto{\pgfqpoint{0.000000in}{-0.020833in}}%
\pgfpathclose%
\pgfusepath{stroke,fill}%
}%
\begin{pgfscope}%
\pgfsys@transformshift{1.200000in}{1.106161in}%
\pgfsys@useobject{currentmarker}{}%
\end{pgfscope}%
\begin{pgfscope}%
\pgfsys@transformshift{1.433333in}{1.106161in}%
\pgfsys@useobject{currentmarker}{}%
\end{pgfscope}%
\end{pgfscope}%
\begin{pgfscope}%
\definecolor{textcolor}{rgb}{0.000000,0.000000,0.000000}%
\pgfsetstrokecolor{textcolor}%
\pgfsetfillcolor{textcolor}%
\pgftext[x=1.616667in,y=1.047827in,left,base]{\color{textcolor}\rmfamily\fontsize{12.000000}{14.400000}\selectfont \(\displaystyle rectangular\ rule\)}%
\end{pgfscope}%
\begin{pgfscope}%
\pgfsetbuttcap%
\pgfsetroundjoin%
\pgfsetlinewidth{1.003750pt}%
\definecolor{currentstroke}{rgb}{0.000000,0.000000,1.000000}%
\pgfsetstrokecolor{currentstroke}%
\pgfsetdash{{6.000000pt}{6.000000pt}}{0.000000pt}%
\pgfpathmoveto{\pgfqpoint{1.200000in}{0.850000in}}%
\pgfpathlineto{\pgfqpoint{1.433333in}{0.850000in}}%
\pgfusepath{stroke}%
\end{pgfscope}%
\begin{pgfscope}%
\pgfsetbuttcap%
\pgfsetroundjoin%
\definecolor{currentfill}{rgb}{0.000000,0.000000,1.000000}%
\pgfsetfillcolor{currentfill}%
\pgfsetlinewidth{0.501875pt}%
\definecolor{currentstroke}{rgb}{0.000000,0.000000,1.000000}%
\pgfsetstrokecolor{currentstroke}%
\pgfsetdash{}{0pt}%
\pgfsys@defobject{currentmarker}{\pgfqpoint{-0.020833in}{-0.020833in}}{\pgfqpoint{0.020833in}{0.020833in}}{%
\pgfpathmoveto{\pgfqpoint{0.000000in}{-0.020833in}}%
\pgfpathcurveto{\pgfqpoint{0.005525in}{-0.020833in}}{\pgfqpoint{0.010825in}{-0.018638in}}{\pgfqpoint{0.014731in}{-0.014731in}}%
\pgfpathcurveto{\pgfqpoint{0.018638in}{-0.010825in}}{\pgfqpoint{0.020833in}{-0.005525in}}{\pgfqpoint{0.020833in}{0.000000in}}%
\pgfpathcurveto{\pgfqpoint{0.020833in}{0.005525in}}{\pgfqpoint{0.018638in}{0.010825in}}{\pgfqpoint{0.014731in}{0.014731in}}%
\pgfpathcurveto{\pgfqpoint{0.010825in}{0.018638in}}{\pgfqpoint{0.005525in}{0.020833in}}{\pgfqpoint{0.000000in}{0.020833in}}%
\pgfpathcurveto{\pgfqpoint{-0.005525in}{0.020833in}}{\pgfqpoint{-0.010825in}{0.018638in}}{\pgfqpoint{-0.014731in}{0.014731in}}%
\pgfpathcurveto{\pgfqpoint{-0.018638in}{0.010825in}}{\pgfqpoint{-0.020833in}{0.005525in}}{\pgfqpoint{-0.020833in}{0.000000in}}%
\pgfpathcurveto{\pgfqpoint{-0.020833in}{-0.005525in}}{\pgfqpoint{-0.018638in}{-0.010825in}}{\pgfqpoint{-0.014731in}{-0.014731in}}%
\pgfpathcurveto{\pgfqpoint{-0.010825in}{-0.018638in}}{\pgfqpoint{-0.005525in}{-0.020833in}}{\pgfqpoint{0.000000in}{-0.020833in}}%
\pgfpathlineto{\pgfqpoint{0.000000in}{-0.020833in}}%
\pgfpathclose%
\pgfusepath{stroke,fill}%
}%
\begin{pgfscope}%
\pgfsys@transformshift{1.200000in}{0.850000in}%
\pgfsys@useobject{currentmarker}{}%
\end{pgfscope}%
\begin{pgfscope}%
\pgfsys@transformshift{1.433333in}{0.850000in}%
\pgfsys@useobject{currentmarker}{}%
\end{pgfscope}%
\end{pgfscope}%
\begin{pgfscope}%
\definecolor{textcolor}{rgb}{0.000000,0.000000,0.000000}%
\pgfsetstrokecolor{textcolor}%
\pgfsetfillcolor{textcolor}%
\pgftext[x=1.616667in,y=0.791667in,left,base]{\color{textcolor}\rmfamily\fontsize{12.000000}{14.400000}\selectfont \(\displaystyle rectangular\ rule\ (x = t^2)\)}%
\end{pgfscope}%
\end{pgfpicture}%
\makeatother%
\endgroup%
}
    \end{figure}

    \item Consider the motion of a simple pendulum. The restoring force is $mg \sin \theta$
    and hence the governing equation is

    \begin{equation*}
    mL \frac{d^{2} \theta}{dt^{2}} + mg \sin(\theta) = 0
    \end{equation*}

    Let the length of the string be $g$. Hence, the governing equation simplifies to
                    
    \begin{equation*}
    \frac{d^{2} \theta}{dt^{2}} + \sin(\theta) = 0
    \end{equation*}

    At the initial time, the pendulum is pulled to an angle of $\theta = 30^{\circ} =
    \displaystyle \frac{\pi}{6}$
    before being let loose without any velocity imparted. Write a code to solve for the
    motion of the pendulum till $t = 100$ seconds using

    \begin{enumerate}
    \item Forward Euler
    \item Backward Euler
    \item Trapezoidal Rule
    \end{enumerate}

    \begin{itemize}
    \item Recall that you need to reformulate the second order differential equation as a
    system of first order differential equation.
    \item Vary your time step $\Delta t$ in $\{0.01, 0.02, 0.05, 0.1, 0.2, 0.5, 1, 2, 5,
    10, 20\}$.
    \item For each $\Delta t$ plot the solution obtained by the three methods on a separate
    figure till the final time of $100$.
    \item Discuss the stability of the schemes. From your plots, at what $\Delta t$ do these
    schemes become unstable (if at all they become unstable)?
    \item Analyse the stability of the three numerical methods to solve the differential
    equation by approximating $\sin(\theta)$ to be $\theta$.
    \item Make sure each figure has a legend and the axes are clearly marked.
    \item Ensure that the font size for title, axes, legend are readable.
    \item Submit the plots obtained, entire code and the write-up.
    \end{itemize}


    \textbf{Solution:}

    We first reduce the second order differential equation

    \begin{equation*}
    \frac{d^{2} \theta}{dt^{2}} = - \sin(\theta)
    \end{equation*}

    to a system of first order equations.

    \begin{equation*}
    \begin{aligned}
    \frac{d \theta}{dt} &= \omega \\ \\
    \frac{d \omega}{dt} &= -\sin(\theta)
    \end{aligned}
    \end{equation*}

    Notice that $(\theta^\prime, \omega^\prime)$ is given as a function of $(\theta,
    \omega)$. The entire motion of the pendulum is determined if we know $(\theta,
    \omega)$ at some instant. So we call $(\theta, \omega)$ the phase of the system. We
    are given the initial phase of the system, \textit{i.e.,}\ we know from which initial
    angle we have released the pendulum, and with what angular velocity. Our aim is to 
    know the phase at all time points during the swing.

    Thus, at $t = t_0$, we know
    
    \begin{equation*}
    \begin{aligned}
    \theta &= \theta_0 = \frac{\pi}{6} \\
    \omega &= \omega_0 = 0
    \end{aligned}
    \end{equation*}

    We want to know the values $\theta(t)$ and $\omega(t)$ at any given $t > t_0$. We 
    also know the rate at which they are increasing at $t = t_0$:

    \begin{equation*}
    \begin{aligned}
    \theta^{\prime}(t_0) &= \omega_0 \\
    \omega^{\prime}(t_0) &= - \sin(\theta_0)
    \end{aligned}
    \end{equation*}

    Now advance time a little to $t_1 = t_0 + \delta t$, say. By this time $\theta$ and
    $\omega$ will roughly change to

    \begin{equation*}
    \begin{aligned}
    \theta_1 &= \theta_0 + \theta^{\prime}(t_0) \delta t = \theta_0 + \omega_0 \delta t \\
    \omega_1 &= \omega_0 + \omega^{\prime}(t_0) \delta t = \omega_0 - \sin(\theta_0) \delta t
    \end{aligned}
    \end{equation*}

    So we get the phase (approximately) at $t_1 = t_0 + \delta t$. Now we keep on
    advancing time by $\delta t$ increments. The same logic may be used repeatedly to 
    give, at $t_k = t_0 + k \cdot \delta t$,

    \begin{equation*}
    \begin{aligned}
    \theta_k &= \theta_{k-1} + \omega_{k-1} \delta t \\
    \omega_k &= \omega_{k-1} - \sin(\theta_{k-1}) \delta t
    \end{aligned}
    \end{equation*}

    Admittedly, this is a rather crude approximation. However, if $\delta t$ is pretty
    small, the accuracy increases.

    Physical stability for the simple pendulum:
    
    The stability analysis of the simple pendulum problem will be done by 
    approximating $sin(\theta)$ to be $\theta$.

    \begin{equation*}
    \frac{d^2 \theta}{dt^2} = - \theta
    \end{equation*}

    converting into a system of first order ordinary differential equations,

    \begin{equation*}
    \begin{aligned}
    \theta^{\prime} &= \frac{d \theta}{dt} = \omega \\
    \theta^{\prime\prime} &= \frac{d \omega}{dt} = - \theta
    \end{aligned}
    \end{equation*}

    rewriting in matrix form,
        
    \begin{equation*}
    \frac{d}{dt} \begin{bmatrix} \theta \\ \omega \end{bmatrix} = \begin{bmatrix} 0 & 1 \\ -1 & 0 \end{bmatrix} \begin{bmatrix} \theta \\ \omega \end{bmatrix}
    \end{equation*}

    then,

    \begin{equation*}
    det \Bigg( \begin{bmatrix} 0 & 1 \\ -1 & 0 \end{bmatrix} - \lambda \begin{bmatrix} 1 & 0 \\ 0 & 1 \end{bmatrix} \Bigg) = 0
    \end{equation*}

    solving, we obtain $\lambda = \pm i$.

    For the exact solution to be stable and bounded, the $Re(\lambda) \le 0$. 
    Since, $Re(\lambda) = 0$, the exact solution for the simple pendulum problem
    is stable and bounded.


    \begin{enumerate}

    \item Forward Euler:
    
    \begin{equation*}
    \begin{aligned}
    \theta_{k+1} &= \theta_{k} + \omega_{k} \delta t \\
    \omega_{k+1} &= \omega_{k} - \sin(\theta_{k}) \delta t
    \end{aligned}
    \end{equation*}

    \item Backward Euler:

    \begin{equation*}
    \begin{aligned}
    \theta_{k+1} &= \theta_{k} + \omega_{k+1} \delta t \\
                 &= \theta_{k} + \Big( \omega_{k} - \sin(\theta_{k}) \delta t \Big) \delta t \\
    \omega_{k+1} &= \omega_{k} - \sin(\theta_{k+1}) \delta t \\
                 &= \omega_{k} - \sin(\theta_{k} + \omega_{k} \delta t) \delta t
    \end{aligned}
    \end{equation*}

    \item Trapezoidal rule:

    \begin{equation*}
    \begin{aligned}
    \theta_{k+1} &= \theta_{k} + \frac{\delta t}{2} \Big( \omega_{k} + \omega_{k+1} \Big) \\
                 &= \theta_{k} + \frac{\delta t}{2} \Big( \omega_{k} + \omega_{k} - \sin(\theta_{k}) \delta t \Big) \\ \\
    \omega_{k+1} &= \omega_{k} - \frac{\delta t}{2} \Big( \sin(\theta_{k}) + \sin(\theta_{k+1}) \Big) \\
                 &= \omega_{k} - \frac{\delta t}{2} \Big( \sin(\theta_{k}) + \sin(\theta_{k} + \omega_{k} \delta t) \Big)
    \end{aligned}
    \end{equation*}

    \end{enumerate}


    \textbf{Program:}
    \lstinputlisting[language=Python]{Scripts/Program8/program8.py}

    \begin{figure}[ht!]
    \centering
    \resizebox{0.9\linewidth}{!}{%% Creator: Matplotlib, PGF backend
%%
%% To include the figure in your LaTeX document, write
%%   \input{<filename>.pgf}
%%
%% Make sure the required packages are loaded in your preamble
%%   \usepackage{pgf}
%%
%% Also ensure that all the required font packages are loaded; for instance,
%% the lmodern package is sometimes necessary when using math font.
%%   \usepackage{lmodern}
%%
%% Figures using additional raster images can only be included by \input if
%% they are in the same directory as the main LaTeX file. For loading figures
%% from other directories you can use the `import` package
%%   \usepackage{import}
%%
%% and then include the figures with
%%   \import{<path to file>}{<filename>.pgf}
%%
%% Matplotlib used the following preamble
%%
\begingroup%
\makeatletter%
\begin{pgfpicture}%
\pgfpathrectangle{\pgfpointorigin}{\pgfqpoint{8.000000in}{6.000000in}}%
\pgfusepath{use as bounding box, clip}%
\begin{pgfscope}%
\pgfsetbuttcap%
\pgfsetmiterjoin%
\definecolor{currentfill}{rgb}{1.000000,1.000000,1.000000}%
\pgfsetfillcolor{currentfill}%
\pgfsetlinewidth{0.000000pt}%
\definecolor{currentstroke}{rgb}{1.000000,1.000000,1.000000}%
\pgfsetstrokecolor{currentstroke}%
\pgfsetdash{}{0pt}%
\pgfpathmoveto{\pgfqpoint{0.000000in}{0.000000in}}%
\pgfpathlineto{\pgfqpoint{8.000000in}{0.000000in}}%
\pgfpathlineto{\pgfqpoint{8.000000in}{6.000000in}}%
\pgfpathlineto{\pgfqpoint{0.000000in}{6.000000in}}%
\pgfpathlineto{\pgfqpoint{0.000000in}{0.000000in}}%
\pgfpathclose%
\pgfusepath{fill}%
\end{pgfscope}%
\begin{pgfscope}%
\pgfsetbuttcap%
\pgfsetmiterjoin%
\definecolor{currentfill}{rgb}{1.000000,1.000000,1.000000}%
\pgfsetfillcolor{currentfill}%
\pgfsetlinewidth{0.000000pt}%
\definecolor{currentstroke}{rgb}{0.000000,0.000000,0.000000}%
\pgfsetstrokecolor{currentstroke}%
\pgfsetstrokeopacity{0.000000}%
\pgfsetdash{}{0pt}%
\pgfpathmoveto{\pgfqpoint{1.000000in}{0.600000in}}%
\pgfpathlineto{\pgfqpoint{7.200000in}{0.600000in}}%
\pgfpathlineto{\pgfqpoint{7.200000in}{5.400000in}}%
\pgfpathlineto{\pgfqpoint{1.000000in}{5.400000in}}%
\pgfpathlineto{\pgfqpoint{1.000000in}{0.600000in}}%
\pgfpathclose%
\pgfusepath{fill}%
\end{pgfscope}%
\begin{pgfscope}%
\pgfpathrectangle{\pgfqpoint{1.000000in}{0.600000in}}{\pgfqpoint{6.200000in}{4.800000in}}%
\pgfusepath{clip}%
\pgfsetrectcap%
\pgfsetroundjoin%
\pgfsetlinewidth{1.003750pt}%
\definecolor{currentstroke}{rgb}{1.000000,0.000000,0.000000}%
\pgfsetstrokecolor{currentstroke}%
\pgfsetdash{}{0pt}%
\pgfpathmoveto{\pgfqpoint{1.000000in}{4.256637in}}%
\pgfpathlineto{\pgfqpoint{1.002480in}{4.255917in}}%
\pgfpathlineto{\pgfqpoint{1.005580in}{4.252318in}}%
\pgfpathlineto{\pgfqpoint{1.009300in}{4.244051in}}%
\pgfpathlineto{\pgfqpoint{1.014260in}{4.226369in}}%
\pgfpathlineto{\pgfqpoint{1.019840in}{4.197491in}}%
\pgfpathlineto{\pgfqpoint{1.026660in}{4.149560in}}%
\pgfpathlineto{\pgfqpoint{1.034720in}{4.075656in}}%
\pgfpathlineto{\pgfqpoint{1.044020in}{3.968537in}}%
\pgfpathlineto{\pgfqpoint{1.055180in}{3.812068in}}%
\pgfpathlineto{\pgfqpoint{1.068820in}{3.586034in}}%
\pgfpathlineto{\pgfqpoint{1.086800in}{3.246502in}}%
\pgfpathlineto{\pgfqpoint{1.135780in}{2.299628in}}%
\pgfpathlineto{\pgfqpoint{1.149420in}{2.087442in}}%
\pgfpathlineto{\pgfqpoint{1.160580in}{1.945099in}}%
\pgfpathlineto{\pgfqpoint{1.169880in}{1.851380in}}%
\pgfpathlineto{\pgfqpoint{1.177940in}{1.790093in}}%
\pgfpathlineto{\pgfqpoint{1.184760in}{1.753443in}}%
\pgfpathlineto{\pgfqpoint{1.190340in}{1.734134in}}%
\pgfpathlineto{\pgfqpoint{1.194680in}{1.725864in}}%
\pgfpathlineto{\pgfqpoint{1.197780in}{1.723597in}}%
\pgfpathlineto{\pgfqpoint{1.200260in}{1.723974in}}%
\pgfpathlineto{\pgfqpoint{1.203360in}{1.727182in}}%
\pgfpathlineto{\pgfqpoint{1.207080in}{1.735040in}}%
\pgfpathlineto{\pgfqpoint{1.211420in}{1.749702in}}%
\pgfpathlineto{\pgfqpoint{1.217000in}{1.777155in}}%
\pgfpathlineto{\pgfqpoint{1.223820in}{1.823587in}}%
\pgfpathlineto{\pgfqpoint{1.231880in}{1.896090in}}%
\pgfpathlineto{\pgfqpoint{1.241180in}{2.002125in}}%
\pgfpathlineto{\pgfqpoint{1.251720in}{2.148671in}}%
\pgfpathlineto{\pgfqpoint{1.264740in}{2.362733in}}%
\pgfpathlineto{\pgfqpoint{1.281480in}{2.677852in}}%
\pgfpathlineto{\pgfqpoint{1.338520in}{3.785720in}}%
\pgfpathlineto{\pgfqpoint{1.351540in}{3.979824in}}%
\pgfpathlineto{\pgfqpoint{1.362080in}{4.106858in}}%
\pgfpathlineto{\pgfqpoint{1.371380in}{4.193606in}}%
\pgfpathlineto{\pgfqpoint{1.378820in}{4.244626in}}%
\pgfpathlineto{\pgfqpoint{1.385020in}{4.274126in}}%
\pgfpathlineto{\pgfqpoint{1.389980in}{4.289008in}}%
\pgfpathlineto{\pgfqpoint{1.394320in}{4.295595in}}%
\pgfpathlineto{\pgfqpoint{1.397420in}{4.296602in}}%
\pgfpathlineto{\pgfqpoint{1.399900in}{4.295186in}}%
\pgfpathlineto{\pgfqpoint{1.403000in}{4.290641in}}%
\pgfpathlineto{\pgfqpoint{1.406720in}{4.281129in}}%
\pgfpathlineto{\pgfqpoint{1.411680in}{4.261608in}}%
\pgfpathlineto{\pgfqpoint{1.417880in}{4.226366in}}%
\pgfpathlineto{\pgfqpoint{1.424700in}{4.174027in}}%
\pgfpathlineto{\pgfqpoint{1.432760in}{4.094574in}}%
\pgfpathlineto{\pgfqpoint{1.442680in}{3.972306in}}%
\pgfpathlineto{\pgfqpoint{1.453840in}{3.805802in}}%
\pgfpathlineto{\pgfqpoint{1.467480in}{3.567774in}}%
\pgfpathlineto{\pgfqpoint{1.486080in}{3.201072in}}%
\pgfpathlineto{\pgfqpoint{1.529480in}{2.330640in}}%
\pgfpathlineto{\pgfqpoint{1.543740in}{2.094105in}}%
\pgfpathlineto{\pgfqpoint{1.555520in}{1.933043in}}%
\pgfpathlineto{\pgfqpoint{1.565440in}{1.825852in}}%
\pgfpathlineto{\pgfqpoint{1.573500in}{1.759787in}}%
\pgfpathlineto{\pgfqpoint{1.580320in}{1.719436in}}%
\pgfpathlineto{\pgfqpoint{1.585900in}{1.697360in}}%
\pgfpathlineto{\pgfqpoint{1.590240in}{1.687112in}}%
\pgfpathlineto{\pgfqpoint{1.593960in}{1.683191in}}%
\pgfpathlineto{\pgfqpoint{1.596440in}{1.683078in}}%
\pgfpathlineto{\pgfqpoint{1.598920in}{1.684969in}}%
\pgfpathlineto{\pgfqpoint{1.602020in}{1.690145in}}%
\pgfpathlineto{\pgfqpoint{1.606360in}{1.702625in}}%
\pgfpathlineto{\pgfqpoint{1.611320in}{1.724303in}}%
\pgfpathlineto{\pgfqpoint{1.617520in}{1.762350in}}%
\pgfpathlineto{\pgfqpoint{1.624960in}{1.823631in}}%
\pgfpathlineto{\pgfqpoint{1.633640in}{1.915693in}}%
\pgfpathlineto{\pgfqpoint{1.643560in}{2.046059in}}%
\pgfpathlineto{\pgfqpoint{1.655340in}{2.231675in}}%
\pgfpathlineto{\pgfqpoint{1.670220in}{2.504592in}}%
\pgfpathlineto{\pgfqpoint{1.691920in}{2.949675in}}%
\pgfpathlineto{\pgfqpoint{1.724780in}{3.620217in}}%
\pgfpathlineto{\pgfqpoint{1.739660in}{3.879290in}}%
\pgfpathlineto{\pgfqpoint{1.752060in}{4.058475in}}%
\pgfpathlineto{\pgfqpoint{1.761980in}{4.172892in}}%
\pgfpathlineto{\pgfqpoint{1.770660in}{4.249583in}}%
\pgfpathlineto{\pgfqpoint{1.778100in}{4.296825in}}%
\pgfpathlineto{\pgfqpoint{1.784300in}{4.322676in}}%
\pgfpathlineto{\pgfqpoint{1.789260in}{4.334335in}}%
\pgfpathlineto{\pgfqpoint{1.792980in}{4.337767in}}%
\pgfpathlineto{\pgfqpoint{1.795460in}{4.337517in}}%
\pgfpathlineto{\pgfqpoint{1.798560in}{4.334349in}}%
\pgfpathlineto{\pgfqpoint{1.802280in}{4.326367in}}%
\pgfpathlineto{\pgfqpoint{1.806620in}{4.311319in}}%
\pgfpathlineto{\pgfqpoint{1.812200in}{4.282992in}}%
\pgfpathlineto{\pgfqpoint{1.819020in}{4.234917in}}%
\pgfpathlineto{\pgfqpoint{1.827080in}{4.159674in}}%
\pgfpathlineto{\pgfqpoint{1.836380in}{4.049435in}}%
\pgfpathlineto{\pgfqpoint{1.846920in}{3.896847in}}%
\pgfpathlineto{\pgfqpoint{1.859940in}{3.673636in}}%
\pgfpathlineto{\pgfqpoint{1.876680in}{3.344527in}}%
\pgfpathlineto{\pgfqpoint{1.934340in}{2.172850in}}%
\pgfpathlineto{\pgfqpoint{1.947360in}{1.970376in}}%
\pgfpathlineto{\pgfqpoint{1.957900in}{1.838013in}}%
\pgfpathlineto{\pgfqpoint{1.967200in}{1.747731in}}%
\pgfpathlineto{\pgfqpoint{1.974640in}{1.694710in}}%
\pgfpathlineto{\pgfqpoint{1.980840in}{1.664115in}}%
\pgfpathlineto{\pgfqpoint{1.985800in}{1.648737in}}%
\pgfpathlineto{\pgfqpoint{1.989520in}{1.642572in}}%
\pgfpathlineto{\pgfqpoint{1.992620in}{1.640970in}}%
\pgfpathlineto{\pgfqpoint{1.995100in}{1.642005in}}%
\pgfpathlineto{\pgfqpoint{1.998200in}{1.646194in}}%
\pgfpathlineto{\pgfqpoint{2.001920in}{1.655456in}}%
\pgfpathlineto{\pgfqpoint{2.006880in}{1.674948in}}%
\pgfpathlineto{\pgfqpoint{2.012460in}{1.706514in}}%
\pgfpathlineto{\pgfqpoint{2.019280in}{1.758646in}}%
\pgfpathlineto{\pgfqpoint{2.027340in}{1.838774in}}%
\pgfpathlineto{\pgfqpoint{2.036640in}{1.954680in}}%
\pgfpathlineto{\pgfqpoint{2.047800in}{2.123771in}}%
\pgfpathlineto{\pgfqpoint{2.061440in}{2.367860in}}%
\pgfpathlineto{\pgfqpoint{2.079420in}{2.734386in}}%
\pgfpathlineto{\pgfqpoint{2.128400in}{3.756342in}}%
\pgfpathlineto{\pgfqpoint{2.142040in}{3.985409in}}%
\pgfpathlineto{\pgfqpoint{2.153200in}{4.139200in}}%
\pgfpathlineto{\pgfqpoint{2.163120in}{4.246447in}}%
\pgfpathlineto{\pgfqpoint{2.171180in}{4.311362in}}%
\pgfpathlineto{\pgfqpoint{2.178000in}{4.349926in}}%
\pgfpathlineto{\pgfqpoint{2.183580in}{4.370005in}}%
\pgfpathlineto{\pgfqpoint{2.187920in}{4.378378in}}%
\pgfpathlineto{\pgfqpoint{2.191020in}{4.380451in}}%
\pgfpathlineto{\pgfqpoint{2.193500in}{4.379760in}}%
\pgfpathlineto{\pgfqpoint{2.196600in}{4.375960in}}%
\pgfpathlineto{\pgfqpoint{2.200320in}{4.367103in}}%
\pgfpathlineto{\pgfqpoint{2.205280in}{4.348045in}}%
\pgfpathlineto{\pgfqpoint{2.210860in}{4.316818in}}%
\pgfpathlineto{\pgfqpoint{2.217680in}{4.264881in}}%
\pgfpathlineto{\pgfqpoint{2.225740in}{4.184666in}}%
\pgfpathlineto{\pgfqpoint{2.235040in}{4.068225in}}%
\pgfpathlineto{\pgfqpoint{2.246200in}{3.897879in}}%
\pgfpathlineto{\pgfqpoint{2.259220in}{3.663352in}}%
\pgfpathlineto{\pgfqpoint{2.276580in}{3.307047in}}%
\pgfpathlineto{\pgfqpoint{2.329280in}{2.195615in}}%
\pgfpathlineto{\pgfqpoint{2.342920in}{1.969173in}}%
\pgfpathlineto{\pgfqpoint{2.354080in}{1.819037in}}%
\pgfpathlineto{\pgfqpoint{2.363380in}{1.721600in}}%
\pgfpathlineto{\pgfqpoint{2.371440in}{1.659154in}}%
\pgfpathlineto{\pgfqpoint{2.378260in}{1.623014in}}%
\pgfpathlineto{\pgfqpoint{2.383220in}{1.606589in}}%
\pgfpathlineto{\pgfqpoint{2.387560in}{1.599117in}}%
\pgfpathlineto{\pgfqpoint{2.390660in}{1.597744in}}%
\pgfpathlineto{\pgfqpoint{2.393140in}{1.599028in}}%
\pgfpathlineto{\pgfqpoint{2.396240in}{1.603610in}}%
\pgfpathlineto{\pgfqpoint{2.399960in}{1.613461in}}%
\pgfpathlineto{\pgfqpoint{2.404920in}{1.633937in}}%
\pgfpathlineto{\pgfqpoint{2.410500in}{1.666876in}}%
\pgfpathlineto{\pgfqpoint{2.417320in}{1.721058in}}%
\pgfpathlineto{\pgfqpoint{2.425380in}{1.804117in}}%
\pgfpathlineto{\pgfqpoint{2.434680in}{1.924048in}}%
\pgfpathlineto{\pgfqpoint{2.445840in}{2.098787in}}%
\pgfpathlineto{\pgfqpoint{2.459480in}{2.350796in}}%
\pgfpathlineto{\pgfqpoint{2.477460in}{2.728934in}}%
\pgfpathlineto{\pgfqpoint{2.525820in}{3.770409in}}%
\pgfpathlineto{\pgfqpoint{2.539460in}{4.008126in}}%
\pgfpathlineto{\pgfqpoint{2.550620in}{4.168275in}}%
\pgfpathlineto{\pgfqpoint{2.560540in}{4.280486in}}%
\pgfpathlineto{\pgfqpoint{2.568600in}{4.348888in}}%
\pgfpathlineto{\pgfqpoint{2.575420in}{4.389986in}}%
\pgfpathlineto{\pgfqpoint{2.581000in}{4.411838in}}%
\pgfpathlineto{\pgfqpoint{2.585340in}{4.421393in}}%
\pgfpathlineto{\pgfqpoint{2.588440in}{4.424205in}}%
\pgfpathlineto{\pgfqpoint{2.590920in}{4.424039in}}%
\pgfpathlineto{\pgfqpoint{2.593400in}{4.421726in}}%
\pgfpathlineto{\pgfqpoint{2.596500in}{4.415820in}}%
\pgfpathlineto{\pgfqpoint{2.600840in}{4.401943in}}%
\pgfpathlineto{\pgfqpoint{2.605800in}{4.378137in}}%
\pgfpathlineto{\pgfqpoint{2.612000in}{4.336645in}}%
\pgfpathlineto{\pgfqpoint{2.619440in}{4.270106in}}%
\pgfpathlineto{\pgfqpoint{2.628120in}{4.170421in}}%
\pgfpathlineto{\pgfqpoint{2.638040in}{4.029500in}}%
\pgfpathlineto{\pgfqpoint{2.649820in}{3.829057in}}%
\pgfpathlineto{\pgfqpoint{2.664700in}{3.534491in}}%
\pgfpathlineto{\pgfqpoint{2.686400in}{3.054188in}}%
\pgfpathlineto{\pgfqpoint{2.719260in}{2.330606in}}%
\pgfpathlineto{\pgfqpoint{2.734140in}{2.050972in}}%
\pgfpathlineto{\pgfqpoint{2.746540in}{1.857421in}}%
\pgfpathlineto{\pgfqpoint{2.756460in}{1.733647in}}%
\pgfpathlineto{\pgfqpoint{2.765140in}{1.650467in}}%
\pgfpathlineto{\pgfqpoint{2.772580in}{1.598992in}}%
\pgfpathlineto{\pgfqpoint{2.778780in}{1.570580in}}%
\pgfpathlineto{\pgfqpoint{2.783740in}{1.557515in}}%
\pgfpathlineto{\pgfqpoint{2.787460in}{1.553408in}}%
\pgfpathlineto{\pgfqpoint{2.789940in}{1.553390in}}%
\pgfpathlineto{\pgfqpoint{2.792420in}{1.555549in}}%
\pgfpathlineto{\pgfqpoint{2.795520in}{1.561305in}}%
\pgfpathlineto{\pgfqpoint{2.799860in}{1.575051in}}%
\pgfpathlineto{\pgfqpoint{2.804820in}{1.598822in}}%
\pgfpathlineto{\pgfqpoint{2.811020in}{1.640443in}}%
\pgfpathlineto{\pgfqpoint{2.818460in}{1.707393in}}%
\pgfpathlineto{\pgfqpoint{2.827140in}{1.807910in}}%
\pgfpathlineto{\pgfqpoint{2.837060in}{1.950239in}}%
\pgfpathlineto{\pgfqpoint{2.848840in}{2.152965in}}%
\pgfpathlineto{\pgfqpoint{2.863720in}{2.451279in}}%
\pgfpathlineto{\pgfqpoint{2.885420in}{2.938420in}}%
\pgfpathlineto{\pgfqpoint{2.918280in}{3.673800in}}%
\pgfpathlineto{\pgfqpoint{2.933780in}{3.969418in}}%
\pgfpathlineto{\pgfqpoint{2.946180in}{4.164866in}}%
\pgfpathlineto{\pgfqpoint{2.956100in}{4.289591in}}%
\pgfpathlineto{\pgfqpoint{2.964780in}{4.373184in}}%
\pgfpathlineto{\pgfqpoint{2.972220in}{4.424701in}}%
\pgfpathlineto{\pgfqpoint{2.978420in}{4.452932in}}%
\pgfpathlineto{\pgfqpoint{2.983380in}{4.465711in}}%
\pgfpathlineto{\pgfqpoint{2.987100in}{4.469524in}}%
\pgfpathlineto{\pgfqpoint{2.989580in}{4.469308in}}%
\pgfpathlineto{\pgfqpoint{2.992060in}{4.466885in}}%
\pgfpathlineto{\pgfqpoint{2.995160in}{4.460757in}}%
\pgfpathlineto{\pgfqpoint{2.999500in}{4.446415in}}%
\pgfpathlineto{\pgfqpoint{3.004460in}{4.421856in}}%
\pgfpathlineto{\pgfqpoint{3.010660in}{4.379093in}}%
\pgfpathlineto{\pgfqpoint{3.018100in}{4.310557in}}%
\pgfpathlineto{\pgfqpoint{3.026780in}{4.207910in}}%
\pgfpathlineto{\pgfqpoint{3.036700in}{4.062813in}}%
\pgfpathlineto{\pgfqpoint{3.048480in}{3.856409in}}%
\pgfpathlineto{\pgfqpoint{3.063360in}{3.553005in}}%
\pgfpathlineto{\pgfqpoint{3.085060in}{3.058074in}}%
\pgfpathlineto{\pgfqpoint{3.117920in}{2.311948in}}%
\pgfpathlineto{\pgfqpoint{3.133420in}{2.012412in}}%
\pgfpathlineto{\pgfqpoint{3.145820in}{1.814566in}}%
\pgfpathlineto{\pgfqpoint{3.155740in}{1.688439in}}%
\pgfpathlineto{\pgfqpoint{3.164420in}{1.604014in}}%
\pgfpathlineto{\pgfqpoint{3.171860in}{1.552081in}}%
\pgfpathlineto{\pgfqpoint{3.178060in}{1.523713in}}%
\pgfpathlineto{\pgfqpoint{3.183020in}{1.510963in}}%
\pgfpathlineto{\pgfqpoint{3.186120in}{1.507522in}}%
\pgfpathlineto{\pgfqpoint{3.188600in}{1.507284in}}%
\pgfpathlineto{\pgfqpoint{3.191080in}{1.509284in}}%
\pgfpathlineto{\pgfqpoint{3.194180in}{1.514926in}}%
\pgfpathlineto{\pgfqpoint{3.198520in}{1.528673in}}%
\pgfpathlineto{\pgfqpoint{3.203480in}{1.552674in}}%
\pgfpathlineto{\pgfqpoint{3.209680in}{1.594927in}}%
\pgfpathlineto{\pgfqpoint{3.217120in}{1.663138in}}%
\pgfpathlineto{\pgfqpoint{3.225800in}{1.765813in}}%
\pgfpathlineto{\pgfqpoint{3.235720in}{1.911493in}}%
\pgfpathlineto{\pgfqpoint{3.247500in}{2.119356in}}%
\pgfpathlineto{\pgfqpoint{3.261760in}{2.412209in}}%
\pgfpathlineto{\pgfqpoint{3.282220in}{2.882878in}}%
\pgfpathlineto{\pgfqpoint{3.318180in}{3.712450in}}%
\pgfpathlineto{\pgfqpoint{3.333060in}{4.003355in}}%
\pgfpathlineto{\pgfqpoint{3.345460in}{4.204144in}}%
\pgfpathlineto{\pgfqpoint{3.355380in}{4.332160in}}%
\pgfpathlineto{\pgfqpoint{3.364060in}{4.417871in}}%
\pgfpathlineto{\pgfqpoint{3.371500in}{4.470624in}}%
\pgfpathlineto{\pgfqpoint{3.377700in}{4.499469in}}%
\pgfpathlineto{\pgfqpoint{3.382660in}{4.512466in}}%
\pgfpathlineto{\pgfqpoint{3.385760in}{4.516000in}}%
\pgfpathlineto{\pgfqpoint{3.388240in}{4.516278in}}%
\pgfpathlineto{\pgfqpoint{3.390720in}{4.514288in}}%
\pgfpathlineto{\pgfqpoint{3.393820in}{4.508614in}}%
\pgfpathlineto{\pgfqpoint{3.398160in}{4.494744in}}%
\pgfpathlineto{\pgfqpoint{3.403120in}{4.470487in}}%
\pgfpathlineto{\pgfqpoint{3.409320in}{4.427743in}}%
\pgfpathlineto{\pgfqpoint{3.416760in}{4.358694in}}%
\pgfpathlineto{\pgfqpoint{3.425440in}{4.254705in}}%
\pgfpathlineto{\pgfqpoint{3.435360in}{4.107096in}}%
\pgfpathlineto{\pgfqpoint{3.447140in}{3.896389in}}%
\pgfpathlineto{\pgfqpoint{3.461400in}{3.599391in}}%
\pgfpathlineto{\pgfqpoint{3.481860in}{3.121785in}}%
\pgfpathlineto{\pgfqpoint{3.517820in}{2.279238in}}%
\pgfpathlineto{\pgfqpoint{3.532700in}{1.983488in}}%
\pgfpathlineto{\pgfqpoint{3.545100in}{1.779170in}}%
\pgfpathlineto{\pgfqpoint{3.555020in}{1.648744in}}%
\pgfpathlineto{\pgfqpoint{3.563700in}{1.561260in}}%
\pgfpathlineto{\pgfqpoint{3.571140in}{1.507254in}}%
\pgfpathlineto{\pgfqpoint{3.577340in}{1.477562in}}%
\pgfpathlineto{\pgfqpoint{3.582300in}{1.464021in}}%
\pgfpathlineto{\pgfqpoint{3.586020in}{1.459876in}}%
\pgfpathlineto{\pgfqpoint{3.588500in}{1.459984in}}%
\pgfpathlineto{\pgfqpoint{3.590980in}{1.462391in}}%
\pgfpathlineto{\pgfqpoint{3.594080in}{1.468628in}}%
\pgfpathlineto{\pgfqpoint{3.598420in}{1.483364in}}%
\pgfpathlineto{\pgfqpoint{3.603380in}{1.508716in}}%
\pgfpathlineto{\pgfqpoint{3.609580in}{1.552983in}}%
\pgfpathlineto{\pgfqpoint{3.617020in}{1.624070in}}%
\pgfpathlineto{\pgfqpoint{3.625700in}{1.730700in}}%
\pgfpathlineto{\pgfqpoint{3.635620in}{1.881632in}}%
\pgfpathlineto{\pgfqpoint{3.647400in}{2.096618in}}%
\pgfpathlineto{\pgfqpoint{3.662280in}{2.413100in}}%
\pgfpathlineto{\pgfqpoint{3.683360in}{2.915088in}}%
\pgfpathlineto{\pgfqpoint{3.717460in}{3.725618in}}%
\pgfpathlineto{\pgfqpoint{3.732340in}{4.026820in}}%
\pgfpathlineto{\pgfqpoint{3.744740in}{4.235290in}}%
\pgfpathlineto{\pgfqpoint{3.755280in}{4.375955in}}%
\pgfpathlineto{\pgfqpoint{3.763960in}{4.463869in}}%
\pgfpathlineto{\pgfqpoint{3.771400in}{4.517919in}}%
\pgfpathlineto{\pgfqpoint{3.777600in}{4.547424in}}%
\pgfpathlineto{\pgfqpoint{3.781940in}{4.559522in}}%
\pgfpathlineto{\pgfqpoint{3.785660in}{4.564236in}}%
\pgfpathlineto{\pgfqpoint{3.788140in}{4.564468in}}%
\pgfpathlineto{\pgfqpoint{3.790620in}{4.562371in}}%
\pgfpathlineto{\pgfqpoint{3.793720in}{4.556477in}}%
\pgfpathlineto{\pgfqpoint{3.798060in}{4.542136in}}%
\pgfpathlineto{\pgfqpoint{3.803020in}{4.517113in}}%
\pgfpathlineto{\pgfqpoint{3.809220in}{4.473070in}}%
\pgfpathlineto{\pgfqpoint{3.816660in}{4.401969in}}%
\pgfpathlineto{\pgfqpoint{3.825340in}{4.294925in}}%
\pgfpathlineto{\pgfqpoint{3.835260in}{4.142994in}}%
\pgfpathlineto{\pgfqpoint{3.847040in}{3.926093in}}%
\pgfpathlineto{\pgfqpoint{3.861300in}{3.620276in}}%
\pgfpathlineto{\pgfqpoint{3.881760in}{3.128249in}}%
\pgfpathlineto{\pgfqpoint{3.917720in}{2.259556in}}%
\pgfpathlineto{\pgfqpoint{3.932600in}{1.954322in}}%
\pgfpathlineto{\pgfqpoint{3.945000in}{1.743233in}}%
\pgfpathlineto{\pgfqpoint{3.954920in}{1.608273in}}%
\pgfpathlineto{\pgfqpoint{3.963600in}{1.517530in}}%
\pgfpathlineto{\pgfqpoint{3.971040in}{1.461287in}}%
\pgfpathlineto{\pgfqpoint{3.977240in}{1.430137in}}%
\pgfpathlineto{\pgfqpoint{3.982200in}{1.415702in}}%
\pgfpathlineto{\pgfqpoint{3.985920in}{1.411047in}}%
\pgfpathlineto{\pgfqpoint{3.988400in}{1.410893in}}%
\pgfpathlineto{\pgfqpoint{3.990880in}{1.413100in}}%
\pgfpathlineto{\pgfqpoint{3.993980in}{1.419175in}}%
\pgfpathlineto{\pgfqpoint{3.998320in}{1.433849in}}%
\pgfpathlineto{\pgfqpoint{4.003280in}{1.459370in}}%
\pgfpathlineto{\pgfqpoint{4.009480in}{1.504204in}}%
\pgfpathlineto{\pgfqpoint{4.016920in}{1.576499in}}%
\pgfpathlineto{\pgfqpoint{4.025600in}{1.685262in}}%
\pgfpathlineto{\pgfqpoint{4.035520in}{1.839565in}}%
\pgfpathlineto{\pgfqpoint{4.047300in}{2.059795in}}%
\pgfpathlineto{\pgfqpoint{4.061560in}{2.370267in}}%
\pgfpathlineto{\pgfqpoint{4.082020in}{2.869764in}}%
\pgfpathlineto{\pgfqpoint{4.117980in}{3.751657in}}%
\pgfpathlineto{\pgfqpoint{4.132860in}{4.061556in}}%
\pgfpathlineto{\pgfqpoint{4.145260in}{4.275918in}}%
\pgfpathlineto{\pgfqpoint{4.155800in}{4.420499in}}%
\pgfpathlineto{\pgfqpoint{4.164480in}{4.510834in}}%
\pgfpathlineto{\pgfqpoint{4.171920in}{4.566366in}}%
\pgfpathlineto{\pgfqpoint{4.178120in}{4.596680in}}%
\pgfpathlineto{\pgfqpoint{4.182460in}{4.609112in}}%
\pgfpathlineto{\pgfqpoint{4.186180in}{4.613959in}}%
\pgfpathlineto{\pgfqpoint{4.188660in}{4.614203in}}%
\pgfpathlineto{\pgfqpoint{4.191140in}{4.612054in}}%
\pgfpathlineto{\pgfqpoint{4.194240in}{4.606006in}}%
\pgfpathlineto{\pgfqpoint{4.198580in}{4.591287in}}%
\pgfpathlineto{\pgfqpoint{4.203540in}{4.565596in}}%
\pgfpathlineto{\pgfqpoint{4.209740in}{4.520369in}}%
\pgfpathlineto{\pgfqpoint{4.217180in}{4.447341in}}%
\pgfpathlineto{\pgfqpoint{4.225860in}{4.337365in}}%
\pgfpathlineto{\pgfqpoint{4.235780in}{4.181213in}}%
\pgfpathlineto{\pgfqpoint{4.247560in}{3.958179in}}%
\pgfpathlineto{\pgfqpoint{4.261820in}{3.643518in}}%
\pgfpathlineto{\pgfqpoint{4.282280in}{3.136832in}}%
\pgfpathlineto{\pgfqpoint{4.318860in}{2.226947in}}%
\pgfpathlineto{\pgfqpoint{4.333740in}{1.913842in}}%
\pgfpathlineto{\pgfqpoint{4.346140in}{1.697776in}}%
\pgfpathlineto{\pgfqpoint{4.356060in}{1.559952in}}%
\pgfpathlineto{\pgfqpoint{4.364740in}{1.467544in}}%
\pgfpathlineto{\pgfqpoint{4.372180in}{1.410505in}}%
\pgfpathlineto{\pgfqpoint{4.378380in}{1.379135in}}%
\pgfpathlineto{\pgfqpoint{4.383340in}{1.364813in}}%
\pgfpathlineto{\pgfqpoint{4.387060in}{1.360411in}}%
\pgfpathlineto{\pgfqpoint{4.389540in}{1.360504in}}%
\pgfpathlineto{\pgfqpoint{4.392020in}{1.363022in}}%
\pgfpathlineto{\pgfqpoint{4.395120in}{1.369574in}}%
\pgfpathlineto{\pgfqpoint{4.399460in}{1.385080in}}%
\pgfpathlineto{\pgfqpoint{4.404420in}{1.411781in}}%
\pgfpathlineto{\pgfqpoint{4.410620in}{1.458433in}}%
\pgfpathlineto{\pgfqpoint{4.418060in}{1.533396in}}%
\pgfpathlineto{\pgfqpoint{4.426740in}{1.645917in}}%
\pgfpathlineto{\pgfqpoint{4.436660in}{1.805318in}}%
\pgfpathlineto{\pgfqpoint{4.448440in}{2.032603in}}%
\pgfpathlineto{\pgfqpoint{4.462700in}{2.352813in}}%
\pgfpathlineto{\pgfqpoint{4.483160in}{2.867729in}}%
\pgfpathlineto{\pgfqpoint{4.519120in}{3.776346in}}%
\pgfpathlineto{\pgfqpoint{4.534000in}{4.095502in}}%
\pgfpathlineto{\pgfqpoint{4.546400in}{4.316266in}}%
\pgfpathlineto{\pgfqpoint{4.556940in}{4.465214in}}%
\pgfpathlineto{\pgfqpoint{4.565620in}{4.558352in}}%
\pgfpathlineto{\pgfqpoint{4.573060in}{4.615694in}}%
\pgfpathlineto{\pgfqpoint{4.579260in}{4.647090in}}%
\pgfpathlineto{\pgfqpoint{4.584220in}{4.661285in}}%
\pgfpathlineto{\pgfqpoint{4.587320in}{4.665187in}}%
\pgfpathlineto{\pgfqpoint{4.589800in}{4.665548in}}%
\pgfpathlineto{\pgfqpoint{4.592280in}{4.663452in}}%
\pgfpathlineto{\pgfqpoint{4.595380in}{4.657382in}}%
\pgfpathlineto{\pgfqpoint{4.599100in}{4.645051in}}%
\pgfpathlineto{\pgfqpoint{4.604060in}{4.620101in}}%
\pgfpathlineto{\pgfqpoint{4.610260in}{4.575414in}}%
\pgfpathlineto{\pgfqpoint{4.617080in}{4.509322in}}%
\pgfpathlineto{\pgfqpoint{4.625140in}{4.409185in}}%
\pgfpathlineto{\pgfqpoint{4.635060in}{4.255145in}}%
\pgfpathlineto{\pgfqpoint{4.646220in}{4.045170in}}%
\pgfpathlineto{\pgfqpoint{4.659860in}{3.744331in}}%
\pgfpathlineto{\pgfqpoint{4.678460in}{3.279199in}}%
\pgfpathlineto{\pgfqpoint{4.723720in}{2.124636in}}%
\pgfpathlineto{\pgfqpoint{4.737980in}{1.826326in}}%
\pgfpathlineto{\pgfqpoint{4.749760in}{1.623717in}}%
\pgfpathlineto{\pgfqpoint{4.759680in}{1.489012in}}%
\pgfpathlineto{\pgfqpoint{4.767740in}{1.405931in}}%
\pgfpathlineto{\pgfqpoint{4.774560in}{1.355045in}}%
\pgfpathlineto{\pgfqpoint{4.780140in}{1.327028in}}%
\pgfpathlineto{\pgfqpoint{4.785100in}{1.312579in}}%
\pgfpathlineto{\pgfqpoint{4.788200in}{1.308583in}}%
\pgfpathlineto{\pgfqpoint{4.790680in}{1.308184in}}%
\pgfpathlineto{\pgfqpoint{4.793160in}{1.310273in}}%
\pgfpathlineto{\pgfqpoint{4.796260in}{1.316381in}}%
\pgfpathlineto{\pgfqpoint{4.799980in}{1.328823in}}%
\pgfpathlineto{\pgfqpoint{4.804940in}{1.354033in}}%
\pgfpathlineto{\pgfqpoint{4.811140in}{1.399227in}}%
\pgfpathlineto{\pgfqpoint{4.817960in}{1.466108in}}%
\pgfpathlineto{\pgfqpoint{4.826020in}{1.567485in}}%
\pgfpathlineto{\pgfqpoint{4.835320in}{1.712794in}}%
\pgfpathlineto{\pgfqpoint{4.846480in}{1.923499in}}%
\pgfpathlineto{\pgfqpoint{4.860120in}{2.226473in}}%
\pgfpathlineto{\pgfqpoint{4.878100in}{2.680283in}}%
\pgfpathlineto{\pgfqpoint{4.926460in}{3.927980in}}%
\pgfpathlineto{\pgfqpoint{4.940100in}{4.212623in}}%
\pgfpathlineto{\pgfqpoint{4.951880in}{4.414190in}}%
\pgfpathlineto{\pgfqpoint{4.961800in}{4.547122in}}%
\pgfpathlineto{\pgfqpoint{4.969860in}{4.628209in}}%
\pgfpathlineto{\pgfqpoint{4.976680in}{4.677052in}}%
\pgfpathlineto{\pgfqpoint{4.982260in}{4.703171in}}%
\pgfpathlineto{\pgfqpoint{4.986600in}{4.714750in}}%
\pgfpathlineto{\pgfqpoint{4.989700in}{4.718308in}}%
\pgfpathlineto{\pgfqpoint{4.992180in}{4.718320in}}%
\pgfpathlineto{\pgfqpoint{4.994660in}{4.715812in}}%
\pgfpathlineto{\pgfqpoint{4.997760in}{4.709136in}}%
\pgfpathlineto{\pgfqpoint{5.002100in}{4.693203in}}%
\pgfpathlineto{\pgfqpoint{5.007060in}{4.665651in}}%
\pgfpathlineto{\pgfqpoint{5.013260in}{4.617393in}}%
\pgfpathlineto{\pgfqpoint{5.020700in}{4.539712in}}%
\pgfpathlineto{\pgfqpoint{5.029380in}{4.422941in}}%
\pgfpathlineto{\pgfqpoint{5.039300in}{4.257301in}}%
\pgfpathlineto{\pgfqpoint{5.051080in}{4.020797in}}%
\pgfpathlineto{\pgfqpoint{5.065340in}{3.687084in}}%
\pgfpathlineto{\pgfqpoint{5.085800in}{3.149410in}}%
\pgfpathlineto{\pgfqpoint{5.122380in}{2.182777in}}%
\pgfpathlineto{\pgfqpoint{5.137260in}{1.849633in}}%
\pgfpathlineto{\pgfqpoint{5.149660in}{1.619319in}}%
\pgfpathlineto{\pgfqpoint{5.160200in}{1.463948in}}%
\pgfpathlineto{\pgfqpoint{5.168880in}{1.366759in}}%
\pgfpathlineto{\pgfqpoint{5.176320in}{1.306862in}}%
\pgfpathlineto{\pgfqpoint{5.182520in}{1.273993in}}%
\pgfpathlineto{\pgfqpoint{5.187480in}{1.259051in}}%
\pgfpathlineto{\pgfqpoint{5.190580in}{1.254879in}}%
\pgfpathlineto{\pgfqpoint{5.193060in}{1.254412in}}%
\pgfpathlineto{\pgfqpoint{5.195540in}{1.256498in}}%
\pgfpathlineto{\pgfqpoint{5.198640in}{1.262693in}}%
\pgfpathlineto{\pgfqpoint{5.202360in}{1.275373in}}%
\pgfpathlineto{\pgfqpoint{5.207320in}{1.301130in}}%
\pgfpathlineto{\pgfqpoint{5.213520in}{1.347371in}}%
\pgfpathlineto{\pgfqpoint{5.220340in}{1.415869in}}%
\pgfpathlineto{\pgfqpoint{5.228400in}{1.519784in}}%
\pgfpathlineto{\pgfqpoint{5.237700in}{1.668841in}}%
\pgfpathlineto{\pgfqpoint{5.248860in}{1.885159in}}%
\pgfpathlineto{\pgfqpoint{5.262500in}{2.196496in}}%
\pgfpathlineto{\pgfqpoint{5.280480in}{2.663374in}}%
\pgfpathlineto{\pgfqpoint{5.328840in}{3.949990in}}%
\pgfpathlineto{\pgfqpoint{5.343100in}{4.256474in}}%
\pgfpathlineto{\pgfqpoint{5.354880in}{4.462961in}}%
\pgfpathlineto{\pgfqpoint{5.364800in}{4.598918in}}%
\pgfpathlineto{\pgfqpoint{5.372860in}{4.681689in}}%
\pgfpathlineto{\pgfqpoint{5.379680in}{4.731413in}}%
\pgfpathlineto{\pgfqpoint{5.385260in}{4.757883in}}%
\pgfpathlineto{\pgfqpoint{5.389600in}{4.769503in}}%
\pgfpathlineto{\pgfqpoint{5.392700in}{4.772968in}}%
\pgfpathlineto{\pgfqpoint{5.395180in}{4.772832in}}%
\pgfpathlineto{\pgfqpoint{5.397660in}{4.770111in}}%
\pgfpathlineto{\pgfqpoint{5.400760in}{4.763077in}}%
\pgfpathlineto{\pgfqpoint{5.405100in}{4.746473in}}%
\pgfpathlineto{\pgfqpoint{5.410060in}{4.717914in}}%
\pgfpathlineto{\pgfqpoint{5.416260in}{4.668038in}}%
\pgfpathlineto{\pgfqpoint{5.423700in}{4.587897in}}%
\pgfpathlineto{\pgfqpoint{5.432380in}{4.467560in}}%
\pgfpathlineto{\pgfqpoint{5.442300in}{4.296962in}}%
\pgfpathlineto{\pgfqpoint{5.454080in}{4.053437in}}%
\pgfpathlineto{\pgfqpoint{5.468340in}{3.709807in}}%
\pgfpathlineto{\pgfqpoint{5.488800in}{3.156007in}}%
\pgfpathlineto{\pgfqpoint{5.525380in}{2.159831in}}%
\pgfpathlineto{\pgfqpoint{5.540260in}{1.816242in}}%
\pgfpathlineto{\pgfqpoint{5.552660in}{1.578485in}}%
\pgfpathlineto{\pgfqpoint{5.563200in}{1.417845in}}%
\pgfpathlineto{\pgfqpoint{5.571880in}{1.317106in}}%
\pgfpathlineto{\pgfqpoint{5.579320in}{1.254760in}}%
\pgfpathlineto{\pgfqpoint{5.585520in}{1.220281in}}%
\pgfpathlineto{\pgfqpoint{5.590480in}{1.204338in}}%
\pgfpathlineto{\pgfqpoint{5.594200in}{1.199230in}}%
\pgfpathlineto{\pgfqpoint{5.596680in}{1.199095in}}%
\pgfpathlineto{\pgfqpoint{5.599160in}{1.201579in}}%
\pgfpathlineto{\pgfqpoint{5.602260in}{1.208363in}}%
\pgfpathlineto{\pgfqpoint{5.606600in}{1.224705in}}%
\pgfpathlineto{\pgfqpoint{5.611560in}{1.253095in}}%
\pgfpathlineto{\pgfqpoint{5.617760in}{1.302953in}}%
\pgfpathlineto{\pgfqpoint{5.625200in}{1.383366in}}%
\pgfpathlineto{\pgfqpoint{5.633880in}{1.504430in}}%
\pgfpathlineto{\pgfqpoint{5.643800in}{1.676406in}}%
\pgfpathlineto{\pgfqpoint{5.655580in}{1.922323in}}%
\pgfpathlineto{\pgfqpoint{5.669840in}{2.269898in}}%
\pgfpathlineto{\pgfqpoint{5.689680in}{2.813484in}}%
\pgfpathlineto{\pgfqpoint{5.728120in}{3.874636in}}%
\pgfpathlineto{\pgfqpoint{5.743000in}{4.220068in}}%
\pgfpathlineto{\pgfqpoint{5.755400in}{4.457959in}}%
\pgfpathlineto{\pgfqpoint{5.765320in}{4.609521in}}%
\pgfpathlineto{\pgfqpoint{5.774000in}{4.711100in}}%
\pgfpathlineto{\pgfqpoint{5.781440in}{4.773832in}}%
\pgfpathlineto{\pgfqpoint{5.787640in}{4.808399in}}%
\pgfpathlineto{\pgfqpoint{5.792600in}{4.824259in}}%
\pgfpathlineto{\pgfqpoint{5.796320in}{4.829219in}}%
\pgfpathlineto{\pgfqpoint{5.798800in}{4.829212in}}%
\pgfpathlineto{\pgfqpoint{5.801280in}{4.826555in}}%
\pgfpathlineto{\pgfqpoint{5.804380in}{4.819507in}}%
\pgfpathlineto{\pgfqpoint{5.808720in}{4.802710in}}%
\pgfpathlineto{\pgfqpoint{5.813680in}{4.773681in}}%
\pgfpathlineto{\pgfqpoint{5.819880in}{4.722843in}}%
\pgfpathlineto{\pgfqpoint{5.827320in}{4.640997in}}%
\pgfpathlineto{\pgfqpoint{5.836000in}{4.517916in}}%
\pgfpathlineto{\pgfqpoint{5.845920in}{4.343201in}}%
\pgfpathlineto{\pgfqpoint{5.857700in}{4.093479in}}%
\pgfpathlineto{\pgfqpoint{5.871960in}{3.740619in}}%
\pgfpathlineto{\pgfqpoint{5.891800in}{3.188848in}}%
\pgfpathlineto{\pgfqpoint{5.930240in}{2.111848in}}%
\pgfpathlineto{\pgfqpoint{5.945120in}{1.761264in}}%
\pgfpathlineto{\pgfqpoint{5.957520in}{1.519778in}}%
\pgfpathlineto{\pgfqpoint{5.967440in}{1.365858in}}%
\pgfpathlineto{\pgfqpoint{5.976120in}{1.262617in}}%
\pgfpathlineto{\pgfqpoint{5.983560in}{1.198768in}}%
\pgfpathlineto{\pgfqpoint{5.989760in}{1.163492in}}%
\pgfpathlineto{\pgfqpoint{5.994720in}{1.147210in}}%
\pgfpathlineto{\pgfqpoint{5.998440in}{1.142021in}}%
\pgfpathlineto{\pgfqpoint{6.000920in}{1.141915in}}%
\pgfpathlineto{\pgfqpoint{6.003400in}{1.144494in}}%
\pgfpathlineto{\pgfqpoint{6.006500in}{1.151490in}}%
\pgfpathlineto{\pgfqpoint{6.010840in}{1.168303in}}%
\pgfpathlineto{\pgfqpoint{6.015800in}{1.197477in}}%
\pgfpathlineto{\pgfqpoint{6.022000in}{1.248686in}}%
\pgfpathlineto{\pgfqpoint{6.029440in}{1.331258in}}%
\pgfpathlineto{\pgfqpoint{6.038120in}{1.455577in}}%
\pgfpathlineto{\pgfqpoint{6.048040in}{1.632217in}}%
\pgfpathlineto{\pgfqpoint{6.059820in}{1.884912in}}%
\pgfpathlineto{\pgfqpoint{6.074080in}{2.242301in}}%
\pgfpathlineto{\pgfqpoint{6.093920in}{2.801756in}}%
\pgfpathlineto{\pgfqpoint{6.132360in}{3.895633in}}%
\pgfpathlineto{\pgfqpoint{6.147240in}{4.252399in}}%
\pgfpathlineto{\pgfqpoint{6.159640in}{4.498537in}}%
\pgfpathlineto{\pgfqpoint{6.169560in}{4.655760in}}%
\pgfpathlineto{\pgfqpoint{6.178240in}{4.761544in}}%
\pgfpathlineto{\pgfqpoint{6.185680in}{4.827294in}}%
\pgfpathlineto{\pgfqpoint{6.191880in}{4.863947in}}%
\pgfpathlineto{\pgfqpoint{6.196840in}{4.881189in}}%
\pgfpathlineto{\pgfqpoint{6.200560in}{4.887015in}}%
\pgfpathlineto{\pgfqpoint{6.203040in}{4.887504in}}%
\pgfpathlineto{\pgfqpoint{6.205520in}{4.885274in}}%
\pgfpathlineto{\pgfqpoint{6.208620in}{4.878668in}}%
\pgfpathlineto{\pgfqpoint{6.212340in}{4.865153in}}%
\pgfpathlineto{\pgfqpoint{6.217300in}{4.837708in}}%
\pgfpathlineto{\pgfqpoint{6.223500in}{4.788430in}}%
\pgfpathlineto{\pgfqpoint{6.230320in}{4.715409in}}%
\pgfpathlineto{\pgfqpoint{6.238380in}{4.604568in}}%
\pgfpathlineto{\pgfqpoint{6.247680in}{4.445429in}}%
\pgfpathlineto{\pgfqpoint{6.258840in}{4.214179in}}%
\pgfpathlineto{\pgfqpoint{6.272480in}{3.880753in}}%
\pgfpathlineto{\pgfqpoint{6.290460in}{3.379532in}}%
\pgfpathlineto{\pgfqpoint{6.340060in}{1.960052in}}%
\pgfpathlineto{\pgfqpoint{6.353700in}{1.645359in}}%
\pgfpathlineto{\pgfqpoint{6.365480in}{1.422521in}}%
\pgfpathlineto{\pgfqpoint{6.375400in}{1.275354in}}%
\pgfpathlineto{\pgfqpoint{6.383460in}{1.185287in}}%
\pgfpathlineto{\pgfqpoint{6.390280in}{1.130696in}}%
\pgfpathlineto{\pgfqpoint{6.395860in}{1.101150in}}%
\pgfpathlineto{\pgfqpoint{6.400200in}{1.087707in}}%
\pgfpathlineto{\pgfqpoint{6.403300in}{1.083248in}}%
\pgfpathlineto{\pgfqpoint{6.405780in}{1.082774in}}%
\pgfpathlineto{\pgfqpoint{6.408260in}{1.085051in}}%
\pgfpathlineto{\pgfqpoint{6.411360in}{1.091764in}}%
\pgfpathlineto{\pgfqpoint{6.415080in}{1.105476in}}%
\pgfpathlineto{\pgfqpoint{6.420040in}{1.133300in}}%
\pgfpathlineto{\pgfqpoint{6.426240in}{1.183239in}}%
\pgfpathlineto{\pgfqpoint{6.433060in}{1.257227in}}%
\pgfpathlineto{\pgfqpoint{6.441120in}{1.369531in}}%
\pgfpathlineto{\pgfqpoint{6.450420in}{1.530783in}}%
\pgfpathlineto{\pgfqpoint{6.461580in}{1.765147in}}%
\pgfpathlineto{\pgfqpoint{6.475220in}{2.103167in}}%
\pgfpathlineto{\pgfqpoint{6.492580in}{2.593181in}}%
\pgfpathlineto{\pgfqpoint{6.544040in}{4.083944in}}%
\pgfpathlineto{\pgfqpoint{6.557680in}{4.398716in}}%
\pgfpathlineto{\pgfqpoint{6.568840in}{4.609805in}}%
\pgfpathlineto{\pgfqpoint{6.578760in}{4.757335in}}%
\pgfpathlineto{\pgfqpoint{6.586820in}{4.847232in}}%
\pgfpathlineto{\pgfqpoint{6.593640in}{4.901369in}}%
\pgfpathlineto{\pgfqpoint{6.599220in}{4.930337in}}%
\pgfpathlineto{\pgfqpoint{6.603560in}{4.943208in}}%
\pgfpathlineto{\pgfqpoint{6.606660in}{4.947193in}}%
\pgfpathlineto{\pgfqpoint{6.609140in}{4.947250in}}%
\pgfpathlineto{\pgfqpoint{6.611620in}{4.944521in}}%
\pgfpathlineto{\pgfqpoint{6.614720in}{4.937199in}}%
\pgfpathlineto{\pgfqpoint{6.619060in}{4.919665in}}%
\pgfpathlineto{\pgfqpoint{6.624020in}{4.889294in}}%
\pgfpathlineto{\pgfqpoint{6.630220in}{4.836026in}}%
\pgfpathlineto{\pgfqpoint{6.637660in}{4.750163in}}%
\pgfpathlineto{\pgfqpoint{6.646340in}{4.620884in}}%
\pgfpathlineto{\pgfqpoint{6.656260in}{4.437127in}}%
\pgfpathlineto{\pgfqpoint{6.668040in}{4.174063in}}%
\pgfpathlineto{\pgfqpoint{6.682300in}{3.801613in}}%
\pgfpathlineto{\pgfqpoint{6.702140in}{3.217691in}}%
\pgfpathlineto{\pgfqpoint{6.741200in}{2.056214in}}%
\pgfpathlineto{\pgfqpoint{6.756080in}{1.684283in}}%
\pgfpathlineto{\pgfqpoint{6.768480in}{1.427857in}}%
\pgfpathlineto{\pgfqpoint{6.778400in}{1.264087in}}%
\pgfpathlineto{\pgfqpoint{6.787080in}{1.153851in}}%
\pgfpathlineto{\pgfqpoint{6.794520in}{1.085248in}}%
\pgfpathlineto{\pgfqpoint{6.800720in}{1.046900in}}%
\pgfpathlineto{\pgfqpoint{6.805680in}{1.028751in}}%
\pgfpathlineto{\pgfqpoint{6.809400in}{1.022508in}}%
\pgfpathlineto{\pgfqpoint{6.811880in}{1.021866in}}%
\pgfpathlineto{\pgfqpoint{6.814360in}{1.024044in}}%
\pgfpathlineto{\pgfqpoint{6.817460in}{1.030726in}}%
\pgfpathlineto{\pgfqpoint{6.821180in}{1.044539in}}%
\pgfpathlineto{\pgfqpoint{6.826140in}{1.072737in}}%
\pgfpathlineto{\pgfqpoint{6.832340in}{1.123526in}}%
\pgfpathlineto{\pgfqpoint{6.839160in}{1.198943in}}%
\pgfpathlineto{\pgfqpoint{6.847220in}{1.313610in}}%
\pgfpathlineto{\pgfqpoint{6.856520in}{1.478491in}}%
\pgfpathlineto{\pgfqpoint{6.867680in}{1.718469in}}%
\pgfpathlineto{\pgfqpoint{6.880700in}{2.048268in}}%
\pgfpathlineto{\pgfqpoint{6.898060in}{2.549723in}}%
\pgfpathlineto{\pgfqpoint{6.952000in}{4.153601in}}%
\pgfpathlineto{\pgfqpoint{6.965640in}{4.471638in}}%
\pgfpathlineto{\pgfqpoint{6.976800in}{4.683223in}}%
\pgfpathlineto{\pgfqpoint{6.986100in}{4.821700in}}%
\pgfpathlineto{\pgfqpoint{6.994160in}{4.911873in}}%
\pgfpathlineto{\pgfqpoint{7.000980in}{4.965636in}}%
\pgfpathlineto{\pgfqpoint{7.006560in}{4.993896in}}%
\pgfpathlineto{\pgfqpoint{7.010900in}{5.005971in}}%
\pgfpathlineto{\pgfqpoint{7.014000in}{5.009259in}}%
\pgfpathlineto{\pgfqpoint{7.016480in}{5.008682in}}%
\pgfpathlineto{\pgfqpoint{7.018960in}{5.005253in}}%
\pgfpathlineto{\pgfqpoint{7.022060in}{4.996962in}}%
\pgfpathlineto{\pgfqpoint{7.026400in}{4.977901in}}%
\pgfpathlineto{\pgfqpoint{7.031360in}{4.945542in}}%
\pgfpathlineto{\pgfqpoint{7.037560in}{4.889438in}}%
\pgfpathlineto{\pgfqpoint{7.045000in}{4.799667in}}%
\pgfpathlineto{\pgfqpoint{7.053680in}{4.665164in}}%
\pgfpathlineto{\pgfqpoint{7.063600in}{4.474612in}}%
\pgfpathlineto{\pgfqpoint{7.075380in}{4.202460in}}%
\pgfpathlineto{\pgfqpoint{7.089640in}{3.817818in}}%
\pgfpathlineto{\pgfqpoint{7.109480in}{3.215698in}}%
\pgfpathlineto{\pgfqpoint{7.148540in}{2.020518in}}%
\pgfpathlineto{\pgfqpoint{7.163420in}{1.638535in}}%
\pgfpathlineto{\pgfqpoint{7.175820in}{1.375412in}}%
\pgfpathlineto{\pgfqpoint{7.185740in}{1.207464in}}%
\pgfpathlineto{\pgfqpoint{7.194420in}{1.094451in}}%
\pgfpathlineto{\pgfqpoint{7.200000in}{1.039360in}}%
\pgfpathlineto{\pgfqpoint{7.200000in}{1.039360in}}%
\pgfusepath{stroke}%
\end{pgfscope}%
\begin{pgfscope}%
\pgfpathrectangle{\pgfqpoint{1.000000in}{0.600000in}}{\pgfqpoint{6.200000in}{4.800000in}}%
\pgfusepath{clip}%
\pgfsetrectcap%
\pgfsetroundjoin%
\pgfsetlinewidth{1.003750pt}%
\definecolor{currentstroke}{rgb}{0.000000,0.000000,1.000000}%
\pgfsetstrokecolor{currentstroke}%
\pgfsetdash{}{0pt}%
\pgfpathmoveto{\pgfqpoint{1.000000in}{4.256637in}}%
\pgfpathlineto{\pgfqpoint{1.002480in}{4.255437in}}%
\pgfpathlineto{\pgfqpoint{1.005580in}{4.251241in}}%
\pgfpathlineto{\pgfqpoint{1.009300in}{4.242267in}}%
\pgfpathlineto{\pgfqpoint{1.014260in}{4.223667in}}%
\pgfpathlineto{\pgfqpoint{1.020460in}{4.189918in}}%
\pgfpathlineto{\pgfqpoint{1.027280in}{4.139669in}}%
\pgfpathlineto{\pgfqpoint{1.035340in}{4.063304in}}%
\pgfpathlineto{\pgfqpoint{1.045260in}{3.945762in}}%
\pgfpathlineto{\pgfqpoint{1.056420in}{3.785780in}}%
\pgfpathlineto{\pgfqpoint{1.070680in}{3.546256in}}%
\pgfpathlineto{\pgfqpoint{1.089900in}{3.181737in}}%
\pgfpathlineto{\pgfqpoint{1.130200in}{2.407954in}}%
\pgfpathlineto{\pgfqpoint{1.145080in}{2.169212in}}%
\pgfpathlineto{\pgfqpoint{1.156860in}{2.013106in}}%
\pgfpathlineto{\pgfqpoint{1.166780in}{1.908445in}}%
\pgfpathlineto{\pgfqpoint{1.175460in}{1.839008in}}%
\pgfpathlineto{\pgfqpoint{1.182900in}{1.796903in}}%
\pgfpathlineto{\pgfqpoint{1.188480in}{1.776217in}}%
\pgfpathlineto{\pgfqpoint{1.192820in}{1.766688in}}%
\pgfpathlineto{\pgfqpoint{1.196540in}{1.763123in}}%
\pgfpathlineto{\pgfqpoint{1.199020in}{1.763113in}}%
\pgfpathlineto{\pgfqpoint{1.202120in}{1.765760in}}%
\pgfpathlineto{\pgfqpoint{1.205840in}{1.772829in}}%
\pgfpathlineto{\pgfqpoint{1.210180in}{1.786410in}}%
\pgfpathlineto{\pgfqpoint{1.215760in}{1.812214in}}%
\pgfpathlineto{\pgfqpoint{1.222580in}{1.856228in}}%
\pgfpathlineto{\pgfqpoint{1.230640in}{1.925295in}}%
\pgfpathlineto{\pgfqpoint{1.239940in}{2.026587in}}%
\pgfpathlineto{\pgfqpoint{1.250480in}{2.166777in}}%
\pgfpathlineto{\pgfqpoint{1.263500in}{2.371643in}}%
\pgfpathlineto{\pgfqpoint{1.280240in}{2.673109in}}%
\pgfpathlineto{\pgfqpoint{1.336660in}{3.720917in}}%
\pgfpathlineto{\pgfqpoint{1.349680in}{3.907545in}}%
\pgfpathlineto{\pgfqpoint{1.360220in}{4.030170in}}%
\pgfpathlineto{\pgfqpoint{1.369520in}{4.114395in}}%
\pgfpathlineto{\pgfqpoint{1.377580in}{4.167830in}}%
\pgfpathlineto{\pgfqpoint{1.383780in}{4.196038in}}%
\pgfpathlineto{\pgfqpoint{1.388740in}{4.210344in}}%
\pgfpathlineto{\pgfqpoint{1.393080in}{4.216771in}}%
\pgfpathlineto{\pgfqpoint{1.396180in}{4.217865in}}%
\pgfpathlineto{\pgfqpoint{1.398660in}{4.216641in}}%
\pgfpathlineto{\pgfqpoint{1.401760in}{4.212491in}}%
\pgfpathlineto{\pgfqpoint{1.405480in}{4.203683in}}%
\pgfpathlineto{\pgfqpoint{1.410440in}{4.185494in}}%
\pgfpathlineto{\pgfqpoint{1.416640in}{4.152558in}}%
\pgfpathlineto{\pgfqpoint{1.423460in}{4.103585in}}%
\pgfpathlineto{\pgfqpoint{1.431520in}{4.029227in}}%
\pgfpathlineto{\pgfqpoint{1.441440in}{3.914865in}}%
\pgfpathlineto{\pgfqpoint{1.453220in}{3.749936in}}%
\pgfpathlineto{\pgfqpoint{1.467480in}{3.515804in}}%
\pgfpathlineto{\pgfqpoint{1.487320in}{3.149075in}}%
\pgfpathlineto{\pgfqpoint{1.525760in}{2.433241in}}%
\pgfpathlineto{\pgfqpoint{1.540640in}{2.200558in}}%
\pgfpathlineto{\pgfqpoint{1.552420in}{2.048052in}}%
\pgfpathlineto{\pgfqpoint{1.562340in}{1.945525in}}%
\pgfpathlineto{\pgfqpoint{1.571020in}{1.877253in}}%
\pgfpathlineto{\pgfqpoint{1.578460in}{1.835617in}}%
\pgfpathlineto{\pgfqpoint{1.584040in}{1.814960in}}%
\pgfpathlineto{\pgfqpoint{1.589000in}{1.804337in}}%
\pgfpathlineto{\pgfqpoint{1.592720in}{1.801187in}}%
\pgfpathlineto{\pgfqpoint{1.595200in}{1.801386in}}%
\pgfpathlineto{\pgfqpoint{1.598300in}{1.804220in}}%
\pgfpathlineto{\pgfqpoint{1.602020in}{1.811403in}}%
\pgfpathlineto{\pgfqpoint{1.606360in}{1.824962in}}%
\pgfpathlineto{\pgfqpoint{1.611940in}{1.850494in}}%
\pgfpathlineto{\pgfqpoint{1.618760in}{1.893804in}}%
\pgfpathlineto{\pgfqpoint{1.626820in}{1.961517in}}%
\pgfpathlineto{\pgfqpoint{1.636120in}{2.060563in}}%
\pgfpathlineto{\pgfqpoint{1.647280in}{2.206150in}}%
\pgfpathlineto{\pgfqpoint{1.660300in}{2.407129in}}%
\pgfpathlineto{\pgfqpoint{1.677660in}{2.712855in}}%
\pgfpathlineto{\pgfqpoint{1.730360in}{3.666896in}}%
\pgfpathlineto{\pgfqpoint{1.744000in}{3.861341in}}%
\pgfpathlineto{\pgfqpoint{1.755160in}{3.990275in}}%
\pgfpathlineto{\pgfqpoint{1.764460in}{4.073949in}}%
\pgfpathlineto{\pgfqpoint{1.772520in}{4.127567in}}%
\pgfpathlineto{\pgfqpoint{1.779340in}{4.158592in}}%
\pgfpathlineto{\pgfqpoint{1.784300in}{4.172693in}}%
\pgfpathlineto{\pgfqpoint{1.788640in}{4.179115in}}%
\pgfpathlineto{\pgfqpoint{1.791740in}{4.180305in}}%
\pgfpathlineto{\pgfqpoint{1.794220in}{4.179216in}}%
\pgfpathlineto{\pgfqpoint{1.797320in}{4.175310in}}%
\pgfpathlineto{\pgfqpoint{1.801040in}{4.166902in}}%
\pgfpathlineto{\pgfqpoint{1.806000in}{4.149426in}}%
\pgfpathlineto{\pgfqpoint{1.812200in}{4.117669in}}%
\pgfpathlineto{\pgfqpoint{1.819020in}{4.070348in}}%
\pgfpathlineto{\pgfqpoint{1.827080in}{3.998406in}}%
\pgfpathlineto{\pgfqpoint{1.837000in}{3.887661in}}%
\pgfpathlineto{\pgfqpoint{1.848160in}{3.736953in}}%
\pgfpathlineto{\pgfqpoint{1.862420in}{3.511396in}}%
\pgfpathlineto{\pgfqpoint{1.881640in}{3.168340in}}%
\pgfpathlineto{\pgfqpoint{1.921940in}{2.440916in}}%
\pgfpathlineto{\pgfqpoint{1.936820in}{2.216792in}}%
\pgfpathlineto{\pgfqpoint{1.948600in}{2.070447in}}%
\pgfpathlineto{\pgfqpoint{1.958520in}{1.972529in}}%
\pgfpathlineto{\pgfqpoint{1.967200in}{1.907776in}}%
\pgfpathlineto{\pgfqpoint{1.974020in}{1.871388in}}%
\pgfpathlineto{\pgfqpoint{1.979600in}{1.851405in}}%
\pgfpathlineto{\pgfqpoint{1.984560in}{1.841162in}}%
\pgfpathlineto{\pgfqpoint{1.988280in}{1.838160in}}%
\pgfpathlineto{\pgfqpoint{1.991380in}{1.838731in}}%
\pgfpathlineto{\pgfqpoint{1.994480in}{1.842091in}}%
\pgfpathlineto{\pgfqpoint{1.998200in}{1.849794in}}%
\pgfpathlineto{\pgfqpoint{2.003160in}{1.866247in}}%
\pgfpathlineto{\pgfqpoint{2.008740in}{1.893085in}}%
\pgfpathlineto{\pgfqpoint{2.015560in}{1.937570in}}%
\pgfpathlineto{\pgfqpoint{2.023620in}{2.006056in}}%
\pgfpathlineto{\pgfqpoint{2.032920in}{2.105139in}}%
\pgfpathlineto{\pgfqpoint{2.044080in}{2.249543in}}%
\pgfpathlineto{\pgfqpoint{2.057720in}{2.457552in}}%
\pgfpathlineto{\pgfqpoint{2.075700in}{2.768845in}}%
\pgfpathlineto{\pgfqpoint{2.122820in}{3.601484in}}%
\pgfpathlineto{\pgfqpoint{2.137080in}{3.805627in}}%
\pgfpathlineto{\pgfqpoint{2.148240in}{3.936508in}}%
\pgfpathlineto{\pgfqpoint{2.158160in}{4.028016in}}%
\pgfpathlineto{\pgfqpoint{2.166220in}{4.083604in}}%
\pgfpathlineto{\pgfqpoint{2.173040in}{4.116815in}}%
\pgfpathlineto{\pgfqpoint{2.178620in}{4.134295in}}%
\pgfpathlineto{\pgfqpoint{2.182960in}{4.141771in}}%
\pgfpathlineto{\pgfqpoint{2.186060in}{4.143814in}}%
\pgfpathlineto{\pgfqpoint{2.189160in}{4.143104in}}%
\pgfpathlineto{\pgfqpoint{2.192260in}{4.139643in}}%
\pgfpathlineto{\pgfqpoint{2.195980in}{4.131874in}}%
\pgfpathlineto{\pgfqpoint{2.200940in}{4.115425in}}%
\pgfpathlineto{\pgfqpoint{2.206520in}{4.088717in}}%
\pgfpathlineto{\pgfqpoint{2.213340in}{4.044569in}}%
\pgfpathlineto{\pgfqpoint{2.221400in}{3.976733in}}%
\pgfpathlineto{\pgfqpoint{2.230700in}{3.878729in}}%
\pgfpathlineto{\pgfqpoint{2.241860in}{3.736063in}}%
\pgfpathlineto{\pgfqpoint{2.255500in}{3.530769in}}%
\pgfpathlineto{\pgfqpoint{2.274100in}{3.212808in}}%
\pgfpathlineto{\pgfqpoint{2.319360in}{2.423538in}}%
\pgfpathlineto{\pgfqpoint{2.333620in}{2.219900in}}%
\pgfpathlineto{\pgfqpoint{2.345400in}{2.082044in}}%
\pgfpathlineto{\pgfqpoint{2.355320in}{1.990976in}}%
\pgfpathlineto{\pgfqpoint{2.363380in}{1.935420in}}%
\pgfpathlineto{\pgfqpoint{2.370200in}{1.902009in}}%
\pgfpathlineto{\pgfqpoint{2.375780in}{1.884214in}}%
\pgfpathlineto{\pgfqpoint{2.380120in}{1.876401in}}%
\pgfpathlineto{\pgfqpoint{2.383840in}{1.873928in}}%
\pgfpathlineto{\pgfqpoint{2.386940in}{1.874852in}}%
\pgfpathlineto{\pgfqpoint{2.390040in}{1.878485in}}%
\pgfpathlineto{\pgfqpoint{2.393760in}{1.886409in}}%
\pgfpathlineto{\pgfqpoint{2.398720in}{1.902973in}}%
\pgfpathlineto{\pgfqpoint{2.404300in}{1.929684in}}%
\pgfpathlineto{\pgfqpoint{2.411120in}{1.973654in}}%
\pgfpathlineto{\pgfqpoint{2.419180in}{2.041021in}}%
\pgfpathlineto{\pgfqpoint{2.428480in}{2.138142in}}%
\pgfpathlineto{\pgfqpoint{2.439640in}{2.279285in}}%
\pgfpathlineto{\pgfqpoint{2.453280in}{2.482082in}}%
\pgfpathlineto{\pgfqpoint{2.471880in}{2.795701in}}%
\pgfpathlineto{\pgfqpoint{2.516520in}{3.562760in}}%
\pgfpathlineto{\pgfqpoint{2.530780in}{3.764072in}}%
\pgfpathlineto{\pgfqpoint{2.542560in}{3.900611in}}%
\pgfpathlineto{\pgfqpoint{2.552480in}{3.991020in}}%
\pgfpathlineto{\pgfqpoint{2.560540in}{4.046350in}}%
\pgfpathlineto{\pgfqpoint{2.567360in}{4.079788in}}%
\pgfpathlineto{\pgfqpoint{2.572940in}{4.097751in}}%
\pgfpathlineto{\pgfqpoint{2.577280in}{4.105787in}}%
\pgfpathlineto{\pgfqpoint{2.581000in}{4.108511in}}%
\pgfpathlineto{\pgfqpoint{2.584100in}{4.107841in}}%
\pgfpathlineto{\pgfqpoint{2.587200in}{4.104499in}}%
\pgfpathlineto{\pgfqpoint{2.590920in}{4.096975in}}%
\pgfpathlineto{\pgfqpoint{2.595880in}{4.081027in}}%
\pgfpathlineto{\pgfqpoint{2.601460in}{4.055118in}}%
\pgfpathlineto{\pgfqpoint{2.608280in}{4.012279in}}%
\pgfpathlineto{\pgfqpoint{2.616340in}{3.946445in}}%
\pgfpathlineto{\pgfqpoint{2.625640in}{3.851332in}}%
\pgfpathlineto{\pgfqpoint{2.636800in}{3.712884in}}%
\pgfpathlineto{\pgfqpoint{2.650440in}{3.513692in}}%
\pgfpathlineto{\pgfqpoint{2.669040in}{3.205264in}}%
\pgfpathlineto{\pgfqpoint{2.714300in}{2.440067in}}%
\pgfpathlineto{\pgfqpoint{2.728560in}{2.242780in}}%
\pgfpathlineto{\pgfqpoint{2.740340in}{2.109310in}}%
\pgfpathlineto{\pgfqpoint{2.750260in}{2.021227in}}%
\pgfpathlineto{\pgfqpoint{2.758320in}{1.967575in}}%
\pgfpathlineto{\pgfqpoint{2.765140in}{1.935391in}}%
\pgfpathlineto{\pgfqpoint{2.770720in}{1.918329in}}%
\pgfpathlineto{\pgfqpoint{2.775060in}{1.910915in}}%
\pgfpathlineto{\pgfqpoint{2.778780in}{1.908663in}}%
\pgfpathlineto{\pgfqpoint{2.781880in}{1.909686in}}%
\pgfpathlineto{\pgfqpoint{2.784980in}{1.913340in}}%
\pgfpathlineto{\pgfqpoint{2.788700in}{1.921185in}}%
\pgfpathlineto{\pgfqpoint{2.793660in}{1.937471in}}%
\pgfpathlineto{\pgfqpoint{2.799860in}{1.967042in}}%
\pgfpathlineto{\pgfqpoint{2.806680in}{2.011080in}}%
\pgfpathlineto{\pgfqpoint{2.814740in}{2.077994in}}%
\pgfpathlineto{\pgfqpoint{2.824660in}{2.180936in}}%
\pgfpathlineto{\pgfqpoint{2.836440in}{2.329374in}}%
\pgfpathlineto{\pgfqpoint{2.850700in}{2.539990in}}%
\pgfpathlineto{\pgfqpoint{2.870540in}{2.869595in}}%
\pgfpathlineto{\pgfqpoint{2.908360in}{3.502557in}}%
\pgfpathlineto{\pgfqpoint{2.923240in}{3.712426in}}%
\pgfpathlineto{\pgfqpoint{2.935020in}{3.850216in}}%
\pgfpathlineto{\pgfqpoint{2.944940in}{3.942995in}}%
\pgfpathlineto{\pgfqpoint{2.953620in}{4.004877in}}%
\pgfpathlineto{\pgfqpoint{2.961060in}{4.042697in}}%
\pgfpathlineto{\pgfqpoint{2.967260in}{4.063101in}}%
\pgfpathlineto{\pgfqpoint{2.971600in}{4.071266in}}%
\pgfpathlineto{\pgfqpoint{2.975320in}{4.074222in}}%
\pgfpathlineto{\pgfqpoint{2.978420in}{4.073829in}}%
\pgfpathlineto{\pgfqpoint{2.981520in}{4.070841in}}%
\pgfpathlineto{\pgfqpoint{2.985240in}{4.063841in}}%
\pgfpathlineto{\pgfqpoint{2.990200in}{4.048758in}}%
\pgfpathlineto{\pgfqpoint{2.995780in}{4.024045in}}%
\pgfpathlineto{\pgfqpoint{3.002600in}{3.982974in}}%
\pgfpathlineto{\pgfqpoint{3.010660in}{3.919645in}}%
\pgfpathlineto{\pgfqpoint{3.019960in}{3.827936in}}%
\pgfpathlineto{\pgfqpoint{3.031120in}{3.694211in}}%
\pgfpathlineto{\pgfqpoint{3.044760in}{3.501550in}}%
\pgfpathlineto{\pgfqpoint{3.062740in}{3.213249in}}%
\pgfpathlineto{\pgfqpoint{3.109860in}{2.442417in}}%
\pgfpathlineto{\pgfqpoint{3.124120in}{2.253572in}}%
\pgfpathlineto{\pgfqpoint{3.135280in}{2.132612in}}%
\pgfpathlineto{\pgfqpoint{3.145200in}{2.048175in}}%
\pgfpathlineto{\pgfqpoint{3.153260in}{1.997018in}}%
\pgfpathlineto{\pgfqpoint{3.160080in}{1.966591in}}%
\pgfpathlineto{\pgfqpoint{3.165660in}{1.950712in}}%
\pgfpathlineto{\pgfqpoint{3.170000in}{1.944053in}}%
\pgfpathlineto{\pgfqpoint{3.173100in}{1.942364in}}%
\pgfpathlineto{\pgfqpoint{3.176200in}{1.943234in}}%
\pgfpathlineto{\pgfqpoint{3.179300in}{1.946660in}}%
\pgfpathlineto{\pgfqpoint{3.183020in}{1.954133in}}%
\pgfpathlineto{\pgfqpoint{3.187980in}{1.969755in}}%
\pgfpathlineto{\pgfqpoint{3.194180in}{1.998235in}}%
\pgfpathlineto{\pgfqpoint{3.201000in}{2.040746in}}%
\pgfpathlineto{\pgfqpoint{3.209060in}{2.105434in}}%
\pgfpathlineto{\pgfqpoint{3.218980in}{2.205052in}}%
\pgfpathlineto{\pgfqpoint{3.230140in}{2.340618in}}%
\pgfpathlineto{\pgfqpoint{3.244400in}{2.543438in}}%
\pgfpathlineto{\pgfqpoint{3.264240in}{2.862022in}}%
\pgfpathlineto{\pgfqpoint{3.303300in}{3.495371in}}%
\pgfpathlineto{\pgfqpoint{3.318180in}{3.697583in}}%
\pgfpathlineto{\pgfqpoint{3.329960in}{3.829880in}}%
\pgfpathlineto{\pgfqpoint{3.339880in}{3.918567in}}%
\pgfpathlineto{\pgfqpoint{3.348560in}{3.977344in}}%
\pgfpathlineto{\pgfqpoint{3.355380in}{4.010470in}}%
\pgfpathlineto{\pgfqpoint{3.361580in}{4.030276in}}%
\pgfpathlineto{\pgfqpoint{3.365920in}{4.038199in}}%
\pgfpathlineto{\pgfqpoint{3.369640in}{4.041065in}}%
\pgfpathlineto{\pgfqpoint{3.372740in}{4.040680in}}%
\pgfpathlineto{\pgfqpoint{3.375840in}{4.037774in}}%
\pgfpathlineto{\pgfqpoint{3.379560in}{4.030972in}}%
\pgfpathlineto{\pgfqpoint{3.384520in}{4.016319in}}%
\pgfpathlineto{\pgfqpoint{3.390100in}{3.992313in}}%
\pgfpathlineto{\pgfqpoint{3.396920in}{3.952426in}}%
\pgfpathlineto{\pgfqpoint{3.404980in}{3.890931in}}%
\pgfpathlineto{\pgfqpoint{3.414280in}{3.801892in}}%
\pgfpathlineto{\pgfqpoint{3.425440in}{3.672091in}}%
\pgfpathlineto{\pgfqpoint{3.439080in}{3.485135in}}%
\pgfpathlineto{\pgfqpoint{3.457060in}{3.205469in}}%
\pgfpathlineto{\pgfqpoint{3.503560in}{2.466921in}}%
\pgfpathlineto{\pgfqpoint{3.517820in}{2.282569in}}%
\pgfpathlineto{\pgfqpoint{3.529600in}{2.158283in}}%
\pgfpathlineto{\pgfqpoint{3.539520in}{2.076655in}}%
\pgfpathlineto{\pgfqpoint{3.547580in}{2.027286in}}%
\pgfpathlineto{\pgfqpoint{3.554400in}{1.998004in}}%
\pgfpathlineto{\pgfqpoint{3.559980in}{1.982805in}}%
\pgfpathlineto{\pgfqpoint{3.564320in}{1.976511in}}%
\pgfpathlineto{\pgfqpoint{3.567420in}{1.974994in}}%
\pgfpathlineto{\pgfqpoint{3.570520in}{1.975963in}}%
\pgfpathlineto{\pgfqpoint{3.573620in}{1.979413in}}%
\pgfpathlineto{\pgfqpoint{3.577960in}{1.988395in}}%
\pgfpathlineto{\pgfqpoint{3.582920in}{2.004538in}}%
\pgfpathlineto{\pgfqpoint{3.589120in}{2.033383in}}%
\pgfpathlineto{\pgfqpoint{3.596560in}{2.080336in}}%
\pgfpathlineto{\pgfqpoint{3.605240in}{2.151299in}}%
\pgfpathlineto{\pgfqpoint{3.615160in}{2.252098in}}%
\pgfpathlineto{\pgfqpoint{3.626940in}{2.395764in}}%
\pgfpathlineto{\pgfqpoint{3.641820in}{2.606870in}}%
\pgfpathlineto{\pgfqpoint{3.664140in}{2.960464in}}%
\pgfpathlineto{\pgfqpoint{3.695140in}{3.447907in}}%
\pgfpathlineto{\pgfqpoint{3.710640in}{3.656556in}}%
\pgfpathlineto{\pgfqpoint{3.723040in}{3.794643in}}%
\pgfpathlineto{\pgfqpoint{3.732960in}{3.882756in}}%
\pgfpathlineto{\pgfqpoint{3.741640in}{3.941730in}}%
\pgfpathlineto{\pgfqpoint{3.749080in}{3.977962in}}%
\pgfpathlineto{\pgfqpoint{3.755280in}{3.997692in}}%
\pgfpathlineto{\pgfqpoint{3.760240in}{4.006498in}}%
\pgfpathlineto{\pgfqpoint{3.763960in}{4.008998in}}%
\pgfpathlineto{\pgfqpoint{3.767060in}{4.008388in}}%
\pgfpathlineto{\pgfqpoint{3.770160in}{4.005331in}}%
\pgfpathlineto{\pgfqpoint{3.773880in}{3.998445in}}%
\pgfpathlineto{\pgfqpoint{3.778840in}{3.983846in}}%
\pgfpathlineto{\pgfqpoint{3.784420in}{3.960129in}}%
\pgfpathlineto{\pgfqpoint{3.791240in}{3.920918in}}%
\pgfpathlineto{\pgfqpoint{3.799300in}{3.860675in}}%
\pgfpathlineto{\pgfqpoint{3.808600in}{3.773672in}}%
\pgfpathlineto{\pgfqpoint{3.819760in}{3.647097in}}%
\pgfpathlineto{\pgfqpoint{3.833400in}{3.465120in}}%
\pgfpathlineto{\pgfqpoint{3.852000in}{3.183624in}}%
\pgfpathlineto{\pgfqpoint{3.896640in}{2.495040in}}%
\pgfpathlineto{\pgfqpoint{3.910900in}{2.314361in}}%
\pgfpathlineto{\pgfqpoint{3.922680in}{2.191901in}}%
\pgfpathlineto{\pgfqpoint{3.932600in}{2.110925in}}%
\pgfpathlineto{\pgfqpoint{3.940660in}{2.061486in}}%
\pgfpathlineto{\pgfqpoint{3.947480in}{2.031730in}}%
\pgfpathlineto{\pgfqpoint{3.953060in}{2.015865in}}%
\pgfpathlineto{\pgfqpoint{3.957400in}{2.008884in}}%
\pgfpathlineto{\pgfqpoint{3.961120in}{2.006657in}}%
\pgfpathlineto{\pgfqpoint{3.964220in}{2.007457in}}%
\pgfpathlineto{\pgfqpoint{3.967320in}{2.010666in}}%
\pgfpathlineto{\pgfqpoint{3.971040in}{2.017686in}}%
\pgfpathlineto{\pgfqpoint{3.976000in}{2.032382in}}%
\pgfpathlineto{\pgfqpoint{3.982200in}{2.059190in}}%
\pgfpathlineto{\pgfqpoint{3.989020in}{2.099216in}}%
\pgfpathlineto{\pgfqpoint{3.997080in}{2.160130in}}%
\pgfpathlineto{\pgfqpoint{4.007000in}{2.253930in}}%
\pgfpathlineto{\pgfqpoint{4.018780in}{2.389255in}}%
\pgfpathlineto{\pgfqpoint{4.033040in}{2.581294in}}%
\pgfpathlineto{\pgfqpoint{4.052880in}{2.881800in}}%
\pgfpathlineto{\pgfqpoint{4.090700in}{3.458692in}}%
\pgfpathlineto{\pgfqpoint{4.105580in}{3.649862in}}%
\pgfpathlineto{\pgfqpoint{4.117360in}{3.775274in}}%
\pgfpathlineto{\pgfqpoint{4.127280in}{3.859603in}}%
\pgfpathlineto{\pgfqpoint{4.135960in}{3.915722in}}%
\pgfpathlineto{\pgfqpoint{4.143400in}{3.949882in}}%
\pgfpathlineto{\pgfqpoint{4.148980in}{3.966764in}}%
\pgfpathlineto{\pgfqpoint{4.153940in}{3.975369in}}%
\pgfpathlineto{\pgfqpoint{4.157660in}{3.977838in}}%
\pgfpathlineto{\pgfqpoint{4.160760in}{3.977281in}}%
\pgfpathlineto{\pgfqpoint{4.163860in}{3.974347in}}%
\pgfpathlineto{\pgfqpoint{4.167580in}{3.967704in}}%
\pgfpathlineto{\pgfqpoint{4.172540in}{3.953585in}}%
\pgfpathlineto{\pgfqpoint{4.178120in}{3.930621in}}%
\pgfpathlineto{\pgfqpoint{4.184940in}{3.892629in}}%
\pgfpathlineto{\pgfqpoint{4.193000in}{3.834233in}}%
\pgfpathlineto{\pgfqpoint{4.202300in}{3.749876in}}%
\pgfpathlineto{\pgfqpoint{4.213460in}{3.627137in}}%
\pgfpathlineto{\pgfqpoint{4.227100in}{3.450672in}}%
\pgfpathlineto{\pgfqpoint{4.245700in}{3.177717in}}%
\pgfpathlineto{\pgfqpoint{4.290340in}{2.510143in}}%
\pgfpathlineto{\pgfqpoint{4.304600in}{2.335025in}}%
\pgfpathlineto{\pgfqpoint{4.316380in}{2.216372in}}%
\pgfpathlineto{\pgfqpoint{4.326300in}{2.137956in}}%
\pgfpathlineto{\pgfqpoint{4.334360in}{2.090122in}}%
\pgfpathlineto{\pgfqpoint{4.341180in}{2.061373in}}%
\pgfpathlineto{\pgfqpoint{4.346760in}{2.046087in}}%
\pgfpathlineto{\pgfqpoint{4.351100in}{2.039400in}}%
\pgfpathlineto{\pgfqpoint{4.354820in}{2.037316in}}%
\pgfpathlineto{\pgfqpoint{4.357920in}{2.038157in}}%
\pgfpathlineto{\pgfqpoint{4.361020in}{2.041337in}}%
\pgfpathlineto{\pgfqpoint{4.365360in}{2.049704in}}%
\pgfpathlineto{\pgfqpoint{4.370320in}{2.064806in}}%
\pgfpathlineto{\pgfqpoint{4.376520in}{2.091853in}}%
\pgfpathlineto{\pgfqpoint{4.383960in}{2.135939in}}%
\pgfpathlineto{\pgfqpoint{4.392640in}{2.202622in}}%
\pgfpathlineto{\pgfqpoint{4.402560in}{2.297383in}}%
\pgfpathlineto{\pgfqpoint{4.414340in}{2.432469in}}%
\pgfpathlineto{\pgfqpoint{4.429220in}{2.630970in}}%
\pgfpathlineto{\pgfqpoint{4.451540in}{2.963402in}}%
\pgfpathlineto{\pgfqpoint{4.482540in}{3.421552in}}%
\pgfpathlineto{\pgfqpoint{4.498040in}{3.617587in}}%
\pgfpathlineto{\pgfqpoint{4.510440in}{3.747255in}}%
\pgfpathlineto{\pgfqpoint{4.520360in}{3.829921in}}%
\pgfpathlineto{\pgfqpoint{4.529040in}{3.885169in}}%
\pgfpathlineto{\pgfqpoint{4.536480in}{3.919026in}}%
\pgfpathlineto{\pgfqpoint{4.542680in}{3.937376in}}%
\pgfpathlineto{\pgfqpoint{4.547640in}{3.945477in}}%
\pgfpathlineto{\pgfqpoint{4.551360in}{3.947683in}}%
\pgfpathlineto{\pgfqpoint{4.554460in}{3.946983in}}%
\pgfpathlineto{\pgfqpoint{4.557560in}{3.943978in}}%
\pgfpathlineto{\pgfqpoint{4.561900in}{3.935913in}}%
\pgfpathlineto{\pgfqpoint{4.566860in}{3.921234in}}%
\pgfpathlineto{\pgfqpoint{4.573060in}{3.894832in}}%
\pgfpathlineto{\pgfqpoint{4.580500in}{3.851682in}}%
\pgfpathlineto{\pgfqpoint{4.589180in}{3.786301in}}%
\pgfpathlineto{\pgfqpoint{4.599100in}{3.693274in}}%
\pgfpathlineto{\pgfqpoint{4.610880in}{3.560537in}}%
\pgfpathlineto{\pgfqpoint{4.625760in}{3.365337in}}%
\pgfpathlineto{\pgfqpoint{4.648080in}{3.038169in}}%
\pgfpathlineto{\pgfqpoint{4.679700in}{2.578510in}}%
\pgfpathlineto{\pgfqpoint{4.695200in}{2.386495in}}%
\pgfpathlineto{\pgfqpoint{4.707600in}{2.259872in}}%
\pgfpathlineto{\pgfqpoint{4.717520in}{2.179448in}}%
\pgfpathlineto{\pgfqpoint{4.726200in}{2.125976in}}%
\pgfpathlineto{\pgfqpoint{4.733640in}{2.093479in}}%
\pgfpathlineto{\pgfqpoint{4.739220in}{2.077464in}}%
\pgfpathlineto{\pgfqpoint{4.744180in}{2.069352in}}%
\pgfpathlineto{\pgfqpoint{4.747900in}{2.067080in}}%
\pgfpathlineto{\pgfqpoint{4.751000in}{2.067688in}}%
\pgfpathlineto{\pgfqpoint{4.754100in}{2.070568in}}%
\pgfpathlineto{\pgfqpoint{4.757820in}{2.077011in}}%
\pgfpathlineto{\pgfqpoint{4.762780in}{2.090630in}}%
\pgfpathlineto{\pgfqpoint{4.768360in}{2.112721in}}%
\pgfpathlineto{\pgfqpoint{4.775180in}{2.149205in}}%
\pgfpathlineto{\pgfqpoint{4.783240in}{2.205212in}}%
\pgfpathlineto{\pgfqpoint{4.792540in}{2.286038in}}%
\pgfpathlineto{\pgfqpoint{4.803700in}{2.403538in}}%
\pgfpathlineto{\pgfqpoint{4.817340in}{2.572329in}}%
\pgfpathlineto{\pgfqpoint{4.835940in}{2.833169in}}%
\pgfpathlineto{\pgfqpoint{4.880580in}{3.470078in}}%
\pgfpathlineto{\pgfqpoint{4.894840in}{3.636831in}}%
\pgfpathlineto{\pgfqpoint{4.906620in}{3.749647in}}%
\pgfpathlineto{\pgfqpoint{4.916540in}{3.824048in}}%
\pgfpathlineto{\pgfqpoint{4.924600in}{3.869292in}}%
\pgfpathlineto{\pgfqpoint{4.931420in}{3.896349in}}%
\pgfpathlineto{\pgfqpoint{4.937000in}{3.910605in}}%
\pgfpathlineto{\pgfqpoint{4.941340in}{3.916714in}}%
\pgfpathlineto{\pgfqpoint{4.945060in}{3.918461in}}%
\pgfpathlineto{\pgfqpoint{4.948160in}{3.917452in}}%
\pgfpathlineto{\pgfqpoint{4.951880in}{3.913289in}}%
\pgfpathlineto{\pgfqpoint{4.956220in}{3.904383in}}%
\pgfpathlineto{\pgfqpoint{4.961180in}{3.888919in}}%
\pgfpathlineto{\pgfqpoint{4.967380in}{3.861807in}}%
\pgfpathlineto{\pgfqpoint{4.974820in}{3.818214in}}%
\pgfpathlineto{\pgfqpoint{4.983500in}{3.752888in}}%
\pgfpathlineto{\pgfqpoint{4.993420in}{3.660684in}}%
\pgfpathlineto{\pgfqpoint{5.005200in}{3.529960in}}%
\pgfpathlineto{\pgfqpoint{5.020080in}{3.338817in}}%
\pgfpathlineto{\pgfqpoint{5.043020in}{3.011382in}}%
\pgfpathlineto{\pgfqpoint{5.072780in}{2.592745in}}%
\pgfpathlineto{\pgfqpoint{5.088280in}{2.406380in}}%
\pgfpathlineto{\pgfqpoint{5.100680in}{2.283420in}}%
\pgfpathlineto{\pgfqpoint{5.110600in}{2.205282in}}%
\pgfpathlineto{\pgfqpoint{5.119280in}{2.153299in}}%
\pgfpathlineto{\pgfqpoint{5.126720in}{2.121678in}}%
\pgfpathlineto{\pgfqpoint{5.132300in}{2.106073in}}%
\pgfpathlineto{\pgfqpoint{5.137260in}{2.098144in}}%
\pgfpathlineto{\pgfqpoint{5.140980in}{2.095897in}}%
\pgfpathlineto{\pgfqpoint{5.144080in}{2.096453in}}%
\pgfpathlineto{\pgfqpoint{5.147180in}{2.099213in}}%
\pgfpathlineto{\pgfqpoint{5.150900in}{2.105425in}}%
\pgfpathlineto{\pgfqpoint{5.155860in}{2.118590in}}%
\pgfpathlineto{\pgfqpoint{5.161440in}{2.139970in}}%
\pgfpathlineto{\pgfqpoint{5.168260in}{2.175307in}}%
\pgfpathlineto{\pgfqpoint{5.176320in}{2.229579in}}%
\pgfpathlineto{\pgfqpoint{5.185620in}{2.307923in}}%
\pgfpathlineto{\pgfqpoint{5.196780in}{2.421833in}}%
\pgfpathlineto{\pgfqpoint{5.210420in}{2.585479in}}%
\pgfpathlineto{\pgfqpoint{5.229020in}{2.838369in}}%
\pgfpathlineto{\pgfqpoint{5.273040in}{3.448183in}}%
\pgfpathlineto{\pgfqpoint{5.287300in}{3.611056in}}%
\pgfpathlineto{\pgfqpoint{5.299080in}{3.721656in}}%
\pgfpathlineto{\pgfqpoint{5.309000in}{3.794937in}}%
\pgfpathlineto{\pgfqpoint{5.317060in}{3.839787in}}%
\pgfpathlineto{\pgfqpoint{5.323880in}{3.866876in}}%
\pgfpathlineto{\pgfqpoint{5.329460in}{3.881405in}}%
\pgfpathlineto{\pgfqpoint{5.333800in}{3.887879in}}%
\pgfpathlineto{\pgfqpoint{5.337520in}{3.890043in}}%
\pgfpathlineto{\pgfqpoint{5.340620in}{3.889454in}}%
\pgfpathlineto{\pgfqpoint{5.344340in}{3.885881in}}%
\pgfpathlineto{\pgfqpoint{5.348680in}{3.877779in}}%
\pgfpathlineto{\pgfqpoint{5.353640in}{3.863381in}}%
\pgfpathlineto{\pgfqpoint{5.359840in}{3.837814in}}%
\pgfpathlineto{\pgfqpoint{5.367280in}{3.796369in}}%
\pgfpathlineto{\pgfqpoint{5.375960in}{3.733919in}}%
\pgfpathlineto{\pgfqpoint{5.385880in}{3.645426in}}%
\pgfpathlineto{\pgfqpoint{5.397660in}{3.519582in}}%
\pgfpathlineto{\pgfqpoint{5.412540in}{3.335088in}}%
\pgfpathlineto{\pgfqpoint{5.435480in}{3.018147in}}%
\pgfpathlineto{\pgfqpoint{5.465240in}{2.611511in}}%
\pgfpathlineto{\pgfqpoint{5.480740in}{2.429856in}}%
\pgfpathlineto{\pgfqpoint{5.493140in}{2.309647in}}%
\pgfpathlineto{\pgfqpoint{5.503060in}{2.232990in}}%
\pgfpathlineto{\pgfqpoint{5.511740in}{2.181750in}}%
\pgfpathlineto{\pgfqpoint{5.519180in}{2.150351in}}%
\pgfpathlineto{\pgfqpoint{5.525380in}{2.133336in}}%
\pgfpathlineto{\pgfqpoint{5.530340in}{2.125831in}}%
\pgfpathlineto{\pgfqpoint{5.534060in}{2.123792in}}%
\pgfpathlineto{\pgfqpoint{5.537160in}{2.124450in}}%
\pgfpathlineto{\pgfqpoint{5.540880in}{2.128063in}}%
\pgfpathlineto{\pgfqpoint{5.545220in}{2.136153in}}%
\pgfpathlineto{\pgfqpoint{5.550180in}{2.150458in}}%
\pgfpathlineto{\pgfqpoint{5.556380in}{2.175793in}}%
\pgfpathlineto{\pgfqpoint{5.563820in}{2.216791in}}%
\pgfpathlineto{\pgfqpoint{5.572500in}{2.278495in}}%
\pgfpathlineto{\pgfqpoint{5.582420in}{2.365852in}}%
\pgfpathlineto{\pgfqpoint{5.594200in}{2.489992in}}%
\pgfpathlineto{\pgfqpoint{5.609080in}{2.671866in}}%
\pgfpathlineto{\pgfqpoint{5.632020in}{2.984066in}}%
\pgfpathlineto{\pgfqpoint{5.661780in}{3.384233in}}%
\pgfpathlineto{\pgfqpoint{5.677280in}{3.562820in}}%
\pgfpathlineto{\pgfqpoint{5.689680in}{3.680890in}}%
\pgfpathlineto{\pgfqpoint{5.699600in}{3.756097in}}%
\pgfpathlineto{\pgfqpoint{5.708280in}{3.806284in}}%
\pgfpathlineto{\pgfqpoint{5.715720in}{3.836959in}}%
\pgfpathlineto{\pgfqpoint{5.721920in}{3.853500in}}%
\pgfpathlineto{\pgfqpoint{5.726880in}{3.860716in}}%
\pgfpathlineto{\pgfqpoint{5.730600in}{3.862590in}}%
\pgfpathlineto{\pgfqpoint{5.733700in}{3.861830in}}%
\pgfpathlineto{\pgfqpoint{5.737420in}{3.858137in}}%
\pgfpathlineto{\pgfqpoint{5.741760in}{3.850013in}}%
\pgfpathlineto{\pgfqpoint{5.746720in}{3.835746in}}%
\pgfpathlineto{\pgfqpoint{5.752920in}{3.810575in}}%
\pgfpathlineto{\pgfqpoint{5.760360in}{3.769944in}}%
\pgfpathlineto{\pgfqpoint{5.769040in}{3.708895in}}%
\pgfpathlineto{\pgfqpoint{5.778960in}{3.622574in}}%
\pgfpathlineto{\pgfqpoint{5.790740in}{3.500030in}}%
\pgfpathlineto{\pgfqpoint{5.805620in}{3.320662in}}%
\pgfpathlineto{\pgfqpoint{5.828560in}{3.013087in}}%
\pgfpathlineto{\pgfqpoint{5.858320in}{2.619366in}}%
\pgfpathlineto{\pgfqpoint{5.873820in}{2.443895in}}%
\pgfpathlineto{\pgfqpoint{5.886220in}{2.328028in}}%
\pgfpathlineto{\pgfqpoint{5.896140in}{2.254339in}}%
\pgfpathlineto{\pgfqpoint{5.904820in}{2.205270in}}%
\pgfpathlineto{\pgfqpoint{5.912260in}{2.175385in}}%
\pgfpathlineto{\pgfqpoint{5.918460in}{2.159373in}}%
\pgfpathlineto{\pgfqpoint{5.923420in}{2.152494in}}%
\pgfpathlineto{\pgfqpoint{5.927140in}{2.150820in}}%
\pgfpathlineto{\pgfqpoint{5.930240in}{2.151712in}}%
\pgfpathlineto{\pgfqpoint{5.933960in}{2.155522in}}%
\pgfpathlineto{\pgfqpoint{5.938300in}{2.163725in}}%
\pgfpathlineto{\pgfqpoint{5.943260in}{2.178003in}}%
\pgfpathlineto{\pgfqpoint{5.949460in}{2.203072in}}%
\pgfpathlineto{\pgfqpoint{5.956900in}{2.243411in}}%
\pgfpathlineto{\pgfqpoint{5.965580in}{2.303890in}}%
\pgfpathlineto{\pgfqpoint{5.975500in}{2.389268in}}%
\pgfpathlineto{\pgfqpoint{5.987900in}{2.517205in}}%
\pgfpathlineto{\pgfqpoint{6.003400in}{2.702966in}}%
\pgfpathlineto{\pgfqpoint{6.028200in}{3.032244in}}%
\pgfpathlineto{\pgfqpoint{6.054860in}{3.377641in}}%
\pgfpathlineto{\pgfqpoint{6.070360in}{3.549954in}}%
\pgfpathlineto{\pgfqpoint{6.082760in}{3.663559in}}%
\pgfpathlineto{\pgfqpoint{6.092680in}{3.735670in}}%
\pgfpathlineto{\pgfqpoint{6.101360in}{3.783558in}}%
\pgfpathlineto{\pgfqpoint{6.108800in}{3.812597in}}%
\pgfpathlineto{\pgfqpoint{6.114380in}{3.826848in}}%
\pgfpathlineto{\pgfqpoint{6.119340in}{3.834001in}}%
\pgfpathlineto{\pgfqpoint{6.123060in}{3.835933in}}%
\pgfpathlineto{\pgfqpoint{6.126160in}{3.835290in}}%
\pgfpathlineto{\pgfqpoint{6.129880in}{3.831819in}}%
\pgfpathlineto{\pgfqpoint{6.134220in}{3.824066in}}%
\pgfpathlineto{\pgfqpoint{6.139180in}{3.810369in}}%
\pgfpathlineto{\pgfqpoint{6.145380in}{3.786126in}}%
\pgfpathlineto{\pgfqpoint{6.152820in}{3.746909in}}%
\pgfpathlineto{\pgfqpoint{6.161500in}{3.687905in}}%
\pgfpathlineto{\pgfqpoint{6.171420in}{3.604397in}}%
\pgfpathlineto{\pgfqpoint{6.183200in}{3.485766in}}%
\pgfpathlineto{\pgfqpoint{6.198080in}{3.312030in}}%
\pgfpathlineto{\pgfqpoint{6.221020in}{3.013946in}}%
\pgfpathlineto{\pgfqpoint{6.250780in}{2.632126in}}%
\pgfpathlineto{\pgfqpoint{6.266280in}{2.461852in}}%
\pgfpathlineto{\pgfqpoint{6.278680in}{2.349364in}}%
\pgfpathlineto{\pgfqpoint{6.288600in}{2.277790in}}%
\pgfpathlineto{\pgfqpoint{6.297280in}{2.230103in}}%
\pgfpathlineto{\pgfqpoint{6.304720in}{2.201034in}}%
\pgfpathlineto{\pgfqpoint{6.310920in}{2.185437in}}%
\pgfpathlineto{\pgfqpoint{6.315880in}{2.178712in}}%
\pgfpathlineto{\pgfqpoint{6.319600in}{2.177051in}}%
\pgfpathlineto{\pgfqpoint{6.322700in}{2.177886in}}%
\pgfpathlineto{\pgfqpoint{6.326420in}{2.181546in}}%
\pgfpathlineto{\pgfqpoint{6.330760in}{2.189463in}}%
\pgfpathlineto{\pgfqpoint{6.335720in}{2.203270in}}%
\pgfpathlineto{\pgfqpoint{6.341920in}{2.227535in}}%
\pgfpathlineto{\pgfqpoint{6.349360in}{2.266607in}}%
\pgfpathlineto{\pgfqpoint{6.358040in}{2.325208in}}%
\pgfpathlineto{\pgfqpoint{6.367960in}{2.407955in}}%
\pgfpathlineto{\pgfqpoint{6.379740in}{2.525289in}}%
\pgfpathlineto{\pgfqpoint{6.394620in}{2.696841in}}%
\pgfpathlineto{\pgfqpoint{6.418180in}{2.998758in}}%
\pgfpathlineto{\pgfqpoint{6.447320in}{3.366083in}}%
\pgfpathlineto{\pgfqpoint{6.462820in}{3.533119in}}%
\pgfpathlineto{\pgfqpoint{6.474600in}{3.638303in}}%
\pgfpathlineto{\pgfqpoint{6.484520in}{3.709258in}}%
\pgfpathlineto{\pgfqpoint{6.493200in}{3.756663in}}%
\pgfpathlineto{\pgfqpoint{6.500640in}{3.785687in}}%
\pgfpathlineto{\pgfqpoint{6.506840in}{3.801384in}}%
\pgfpathlineto{\pgfqpoint{6.511800in}{3.808278in}}%
\pgfpathlineto{\pgfqpoint{6.515520in}{3.810117in}}%
\pgfpathlineto{\pgfqpoint{6.518620in}{3.809464in}}%
\pgfpathlineto{\pgfqpoint{6.522340in}{3.806062in}}%
\pgfpathlineto{\pgfqpoint{6.526680in}{3.798499in}}%
\pgfpathlineto{\pgfqpoint{6.531640in}{3.785164in}}%
\pgfpathlineto{\pgfqpoint{6.537840in}{3.761586in}}%
\pgfpathlineto{\pgfqpoint{6.545280in}{3.723474in}}%
\pgfpathlineto{\pgfqpoint{6.553960in}{3.666162in}}%
\pgfpathlineto{\pgfqpoint{6.563880in}{3.585083in}}%
\pgfpathlineto{\pgfqpoint{6.575660in}{3.469947in}}%
\pgfpathlineto{\pgfqpoint{6.590540in}{3.301397in}}%
\pgfpathlineto{\pgfqpoint{6.613480in}{3.012352in}}%
\pgfpathlineto{\pgfqpoint{6.643240in}{2.642343in}}%
\pgfpathlineto{\pgfqpoint{6.658740in}{2.477448in}}%
\pgfpathlineto{\pgfqpoint{6.671140in}{2.368587in}}%
\pgfpathlineto{\pgfqpoint{6.681060in}{2.299384in}}%
\pgfpathlineto{\pgfqpoint{6.689740in}{2.253337in}}%
\pgfpathlineto{\pgfqpoint{6.697180in}{2.225330in}}%
\pgfpathlineto{\pgfqpoint{6.703380in}{2.210364in}}%
\pgfpathlineto{\pgfqpoint{6.708340in}{2.203975in}}%
\pgfpathlineto{\pgfqpoint{6.712060in}{2.202466in}}%
\pgfpathlineto{\pgfqpoint{6.715160in}{2.203361in}}%
\pgfpathlineto{\pgfqpoint{6.718880in}{2.207015in}}%
\pgfpathlineto{\pgfqpoint{6.723220in}{2.214814in}}%
\pgfpathlineto{\pgfqpoint{6.728800in}{2.230376in}}%
\pgfpathlineto{\pgfqpoint{6.735000in}{2.254826in}}%
\pgfpathlineto{\pgfqpoint{6.742440in}{2.293776in}}%
\pgfpathlineto{\pgfqpoint{6.751120in}{2.351764in}}%
\pgfpathlineto{\pgfqpoint{6.761660in}{2.438784in}}%
\pgfpathlineto{\pgfqpoint{6.774060in}{2.561276in}}%
\pgfpathlineto{\pgfqpoint{6.789560in}{2.737850in}}%
\pgfpathlineto{\pgfqpoint{6.816840in}{3.079666in}}%
\pgfpathlineto{\pgfqpoint{6.841020in}{3.370791in}}%
\pgfpathlineto{\pgfqpoint{6.855900in}{3.524341in}}%
\pgfpathlineto{\pgfqpoint{6.867680in}{3.624888in}}%
\pgfpathlineto{\pgfqpoint{6.877600in}{3.692301in}}%
\pgfpathlineto{\pgfqpoint{6.886280in}{3.736948in}}%
\pgfpathlineto{\pgfqpoint{6.893720in}{3.763901in}}%
\pgfpathlineto{\pgfqpoint{6.899300in}{3.777023in}}%
\pgfpathlineto{\pgfqpoint{6.904260in}{3.783492in}}%
\pgfpathlineto{\pgfqpoint{6.907980in}{3.785111in}}%
\pgfpathlineto{\pgfqpoint{6.911080in}{3.784340in}}%
\pgfpathlineto{\pgfqpoint{6.914800in}{3.780874in}}%
\pgfpathlineto{\pgfqpoint{6.919140in}{3.773344in}}%
\pgfpathlineto{\pgfqpoint{6.924100in}{3.760191in}}%
\pgfpathlineto{\pgfqpoint{6.930300in}{3.737053in}}%
\pgfpathlineto{\pgfqpoint{6.937740in}{3.699776in}}%
\pgfpathlineto{\pgfqpoint{6.946420in}{3.643848in}}%
\pgfpathlineto{\pgfqpoint{6.956340in}{3.564862in}}%
\pgfpathlineto{\pgfqpoint{6.968740in}{3.446478in}}%
\pgfpathlineto{\pgfqpoint{6.984240in}{3.274580in}}%
\pgfpathlineto{\pgfqpoint{7.009040in}{2.969901in}}%
\pgfpathlineto{\pgfqpoint{7.035700in}{2.650356in}}%
\pgfpathlineto{\pgfqpoint{7.051200in}{2.490978in}}%
\pgfpathlineto{\pgfqpoint{7.062980in}{2.390647in}}%
\pgfpathlineto{\pgfqpoint{7.072900in}{2.322996in}}%
\pgfpathlineto{\pgfqpoint{7.081580in}{2.277833in}}%
\pgfpathlineto{\pgfqpoint{7.089020in}{2.250219in}}%
\pgfpathlineto{\pgfqpoint{7.095220in}{2.235321in}}%
\pgfpathlineto{\pgfqpoint{7.100180in}{2.228816in}}%
\pgfpathlineto{\pgfqpoint{7.103900in}{2.227121in}}%
\pgfpathlineto{\pgfqpoint{7.107000in}{2.227797in}}%
\pgfpathlineto{\pgfqpoint{7.110720in}{2.231111in}}%
\pgfpathlineto{\pgfqpoint{7.115060in}{2.238412in}}%
\pgfpathlineto{\pgfqpoint{7.120020in}{2.251237in}}%
\pgfpathlineto{\pgfqpoint{7.126220in}{2.273867in}}%
\pgfpathlineto{\pgfqpoint{7.133660in}{2.310399in}}%
\pgfpathlineto{\pgfqpoint{7.142340in}{2.365282in}}%
\pgfpathlineto{\pgfqpoint{7.152260in}{2.442867in}}%
\pgfpathlineto{\pgfqpoint{7.164040in}{2.552966in}}%
\pgfpathlineto{\pgfqpoint{7.178920in}{2.714037in}}%
\pgfpathlineto{\pgfqpoint{7.200000in}{2.967128in}}%
\pgfpathlineto{\pgfqpoint{7.200000in}{2.967128in}}%
\pgfusepath{stroke}%
\end{pgfscope}%
\begin{pgfscope}%
\pgfpathrectangle{\pgfqpoint{1.000000in}{0.600000in}}{\pgfqpoint{6.200000in}{4.800000in}}%
\pgfusepath{clip}%
\pgfsetrectcap%
\pgfsetroundjoin%
\pgfsetlinewidth{1.003750pt}%
\definecolor{currentstroke}{rgb}{0.000000,0.000000,0.000000}%
\pgfsetstrokecolor{currentstroke}%
\pgfsetdash{}{0pt}%
\pgfpathmoveto{\pgfqpoint{1.000000in}{4.256637in}}%
\pgfpathlineto{\pgfqpoint{1.002480in}{4.255677in}}%
\pgfpathlineto{\pgfqpoint{1.005580in}{4.251780in}}%
\pgfpathlineto{\pgfqpoint{1.009300in}{4.243159in}}%
\pgfpathlineto{\pgfqpoint{1.014260in}{4.225017in}}%
\pgfpathlineto{\pgfqpoint{1.019840in}{4.195649in}}%
\pgfpathlineto{\pgfqpoint{1.026660in}{4.147174in}}%
\pgfpathlineto{\pgfqpoint{1.034720in}{4.072729in}}%
\pgfpathlineto{\pgfqpoint{1.044020in}{3.965160in}}%
\pgfpathlineto{\pgfqpoint{1.055180in}{3.808461in}}%
\pgfpathlineto{\pgfqpoint{1.068820in}{3.582690in}}%
\pgfpathlineto{\pgfqpoint{1.086800in}{3.244526in}}%
\pgfpathlineto{\pgfqpoint{1.134540in}{2.327465in}}%
\pgfpathlineto{\pgfqpoint{1.148180in}{2.115119in}}%
\pgfpathlineto{\pgfqpoint{1.159960in}{1.964976in}}%
\pgfpathlineto{\pgfqpoint{1.169880in}{1.866381in}}%
\pgfpathlineto{\pgfqpoint{1.177940in}{1.806732in}}%
\pgfpathlineto{\pgfqpoint{1.184760in}{1.771323in}}%
\pgfpathlineto{\pgfqpoint{1.190340in}{1.752913in}}%
\pgfpathlineto{\pgfqpoint{1.194680in}{1.745260in}}%
\pgfpathlineto{\pgfqpoint{1.197780in}{1.743386in}}%
\pgfpathlineto{\pgfqpoint{1.200260in}{1.744046in}}%
\pgfpathlineto{\pgfqpoint{1.203360in}{1.747570in}}%
\pgfpathlineto{\pgfqpoint{1.207080in}{1.755745in}}%
\pgfpathlineto{\pgfqpoint{1.212040in}{1.773297in}}%
\pgfpathlineto{\pgfqpoint{1.217620in}{1.802013in}}%
\pgfpathlineto{\pgfqpoint{1.224440in}{1.849715in}}%
\pgfpathlineto{\pgfqpoint{1.232500in}{1.923291in}}%
\pgfpathlineto{\pgfqpoint{1.241800in}{2.029931in}}%
\pgfpathlineto{\pgfqpoint{1.252960in}{2.185652in}}%
\pgfpathlineto{\pgfqpoint{1.266600in}{2.410475in}}%
\pgfpathlineto{\pgfqpoint{1.284580in}{2.747884in}}%
\pgfpathlineto{\pgfqpoint{1.332940in}{3.676468in}}%
\pgfpathlineto{\pgfqpoint{1.346580in}{3.888177in}}%
\pgfpathlineto{\pgfqpoint{1.357740in}{4.030646in}}%
\pgfpathlineto{\pgfqpoint{1.367660in}{4.130287in}}%
\pgfpathlineto{\pgfqpoint{1.375720in}{4.190843in}}%
\pgfpathlineto{\pgfqpoint{1.382540in}{4.227047in}}%
\pgfpathlineto{\pgfqpoint{1.388120in}{4.246122in}}%
\pgfpathlineto{\pgfqpoint{1.392460in}{4.254297in}}%
\pgfpathlineto{\pgfqpoint{1.395560in}{4.256545in}}%
\pgfpathlineto{\pgfqpoint{1.398040in}{4.256183in}}%
\pgfpathlineto{\pgfqpoint{1.401140in}{4.253033in}}%
\pgfpathlineto{\pgfqpoint{1.404860in}{4.245305in}}%
\pgfpathlineto{\pgfqpoint{1.409200in}{4.230877in}}%
\pgfpathlineto{\pgfqpoint{1.414780in}{4.203862in}}%
\pgfpathlineto{\pgfqpoint{1.421600in}{4.158181in}}%
\pgfpathlineto{\pgfqpoint{1.429660in}{4.086884in}}%
\pgfpathlineto{\pgfqpoint{1.438960in}{3.982678in}}%
\pgfpathlineto{\pgfqpoint{1.450120in}{3.829530in}}%
\pgfpathlineto{\pgfqpoint{1.463140in}{3.618032in}}%
\pgfpathlineto{\pgfqpoint{1.480500in}{3.296022in}}%
\pgfpathlineto{\pgfqpoint{1.533820in}{2.278518in}}%
\pgfpathlineto{\pgfqpoint{1.547460in}{2.074424in}}%
\pgfpathlineto{\pgfqpoint{1.558620in}{1.939453in}}%
\pgfpathlineto{\pgfqpoint{1.567920in}{1.852176in}}%
\pgfpathlineto{\pgfqpoint{1.575980in}{1.796554in}}%
\pgfpathlineto{\pgfqpoint{1.582180in}{1.766985in}}%
\pgfpathlineto{\pgfqpoint{1.587760in}{1.750446in}}%
\pgfpathlineto{\pgfqpoint{1.591480in}{1.744785in}}%
\pgfpathlineto{\pgfqpoint{1.594580in}{1.743363in}}%
\pgfpathlineto{\pgfqpoint{1.597060in}{1.744385in}}%
\pgfpathlineto{\pgfqpoint{1.600160in}{1.748359in}}%
\pgfpathlineto{\pgfqpoint{1.603880in}{1.757073in}}%
\pgfpathlineto{\pgfqpoint{1.608840in}{1.775337in}}%
\pgfpathlineto{\pgfqpoint{1.614420in}{1.804840in}}%
\pgfpathlineto{\pgfqpoint{1.621240in}{1.853475in}}%
\pgfpathlineto{\pgfqpoint{1.629300in}{1.928101in}}%
\pgfpathlineto{\pgfqpoint{1.638600in}{2.035862in}}%
\pgfpathlineto{\pgfqpoint{1.649760in}{2.192763in}}%
\pgfpathlineto{\pgfqpoint{1.663400in}{2.418731in}}%
\pgfpathlineto{\pgfqpoint{1.681380in}{2.757051in}}%
\pgfpathlineto{\pgfqpoint{1.728500in}{3.663404in}}%
\pgfpathlineto{\pgfqpoint{1.742760in}{3.886016in}}%
\pgfpathlineto{\pgfqpoint{1.753920in}{4.028902in}}%
\pgfpathlineto{\pgfqpoint{1.763840in}{4.128956in}}%
\pgfpathlineto{\pgfqpoint{1.771900in}{4.189870in}}%
\pgfpathlineto{\pgfqpoint{1.778720in}{4.226390in}}%
\pgfpathlineto{\pgfqpoint{1.784300in}{4.245727in}}%
\pgfpathlineto{\pgfqpoint{1.788640in}{4.254109in}}%
\pgfpathlineto{\pgfqpoint{1.791740in}{4.256505in}}%
\pgfpathlineto{\pgfqpoint{1.794220in}{4.256262in}}%
\pgfpathlineto{\pgfqpoint{1.797320in}{4.253260in}}%
\pgfpathlineto{\pgfqpoint{1.801040in}{4.245709in}}%
\pgfpathlineto{\pgfqpoint{1.805380in}{4.231485in}}%
\pgfpathlineto{\pgfqpoint{1.810960in}{4.204730in}}%
\pgfpathlineto{\pgfqpoint{1.817780in}{4.159358in}}%
\pgfpathlineto{\pgfqpoint{1.825840in}{4.088409in}}%
\pgfpathlineto{\pgfqpoint{1.835140in}{3.984576in}}%
\pgfpathlineto{\pgfqpoint{1.845680in}{3.841046in}}%
\pgfpathlineto{\pgfqpoint{1.858700in}{3.631437in}}%
\pgfpathlineto{\pgfqpoint{1.875440in}{3.323059in}}%
\pgfpathlineto{\pgfqpoint{1.931860in}{2.250894in}}%
\pgfpathlineto{\pgfqpoint{1.944880in}{2.059925in}}%
\pgfpathlineto{\pgfqpoint{1.955420in}{1.934505in}}%
\pgfpathlineto{\pgfqpoint{1.964720in}{1.848435in}}%
\pgfpathlineto{\pgfqpoint{1.972780in}{1.793921in}}%
\pgfpathlineto{\pgfqpoint{1.978980in}{1.765233in}}%
\pgfpathlineto{\pgfqpoint{1.983940in}{1.750770in}}%
\pgfpathlineto{\pgfqpoint{1.988280in}{1.744378in}}%
\pgfpathlineto{\pgfqpoint{1.991380in}{1.743407in}}%
\pgfpathlineto{\pgfqpoint{1.993860in}{1.744791in}}%
\pgfpathlineto{\pgfqpoint{1.996960in}{1.749216in}}%
\pgfpathlineto{\pgfqpoint{2.000680in}{1.758468in}}%
\pgfpathlineto{\pgfqpoint{2.005640in}{1.777443in}}%
\pgfpathlineto{\pgfqpoint{2.011840in}{1.811673in}}%
\pgfpathlineto{\pgfqpoint{2.018660in}{1.862473in}}%
\pgfpathlineto{\pgfqpoint{2.026720in}{1.939530in}}%
\pgfpathlineto{\pgfqpoint{2.036640in}{2.058000in}}%
\pgfpathlineto{\pgfqpoint{2.047800in}{2.219146in}}%
\pgfpathlineto{\pgfqpoint{2.062060in}{2.460361in}}%
\pgfpathlineto{\pgfqpoint{2.081280in}{2.827506in}}%
\pgfpathlineto{\pgfqpoint{2.122200in}{3.618159in}}%
\pgfpathlineto{\pgfqpoint{2.137080in}{3.857122in}}%
\pgfpathlineto{\pgfqpoint{2.148860in}{4.012748in}}%
\pgfpathlineto{\pgfqpoint{2.158780in}{4.116537in}}%
\pgfpathlineto{\pgfqpoint{2.167460in}{4.184851in}}%
\pgfpathlineto{\pgfqpoint{2.174280in}{4.222958in}}%
\pgfpathlineto{\pgfqpoint{2.179860in}{4.243622in}}%
\pgfpathlineto{\pgfqpoint{2.184200in}{4.253046in}}%
\pgfpathlineto{\pgfqpoint{2.187920in}{4.256458in}}%
\pgfpathlineto{\pgfqpoint{2.190400in}{4.256334in}}%
\pgfpathlineto{\pgfqpoint{2.193500in}{4.253480in}}%
\pgfpathlineto{\pgfqpoint{2.197220in}{4.246105in}}%
\pgfpathlineto{\pgfqpoint{2.201560in}{4.232087in}}%
\pgfpathlineto{\pgfqpoint{2.207140in}{4.205591in}}%
\pgfpathlineto{\pgfqpoint{2.213960in}{4.160528in}}%
\pgfpathlineto{\pgfqpoint{2.222020in}{4.089928in}}%
\pgfpathlineto{\pgfqpoint{2.231320in}{3.986467in}}%
\pgfpathlineto{\pgfqpoint{2.241860in}{3.843313in}}%
\pgfpathlineto{\pgfqpoint{2.254880in}{3.634081in}}%
\pgfpathlineto{\pgfqpoint{2.271620in}{3.326023in}}%
\pgfpathlineto{\pgfqpoint{2.328040in}{2.253350in}}%
\pgfpathlineto{\pgfqpoint{2.341060in}{2.061951in}}%
\pgfpathlineto{\pgfqpoint{2.351600in}{1.936122in}}%
\pgfpathlineto{\pgfqpoint{2.360900in}{1.849655in}}%
\pgfpathlineto{\pgfqpoint{2.368960in}{1.794777in}}%
\pgfpathlineto{\pgfqpoint{2.375160in}{1.765799in}}%
\pgfpathlineto{\pgfqpoint{2.380120in}{1.751102in}}%
\pgfpathlineto{\pgfqpoint{2.384460in}{1.744503in}}%
\pgfpathlineto{\pgfqpoint{2.387560in}{1.743384in}}%
\pgfpathlineto{\pgfqpoint{2.390040in}{1.744649in}}%
\pgfpathlineto{\pgfqpoint{2.393140in}{1.748926in}}%
\pgfpathlineto{\pgfqpoint{2.396860in}{1.758002in}}%
\pgfpathlineto{\pgfqpoint{2.401820in}{1.776743in}}%
\pgfpathlineto{\pgfqpoint{2.408020in}{1.810687in}}%
\pgfpathlineto{\pgfqpoint{2.414840in}{1.861183in}}%
\pgfpathlineto{\pgfqpoint{2.422900in}{1.937899in}}%
\pgfpathlineto{\pgfqpoint{2.432820in}{2.055984in}}%
\pgfpathlineto{\pgfqpoint{2.443980in}{2.216753in}}%
\pgfpathlineto{\pgfqpoint{2.457620in}{2.446440in}}%
\pgfpathlineto{\pgfqpoint{2.476840in}{2.812176in}}%
\pgfpathlineto{\pgfqpoint{2.519000in}{3.626263in}}%
\pgfpathlineto{\pgfqpoint{2.533260in}{3.854887in}}%
\pgfpathlineto{\pgfqpoint{2.545040in}{4.010942in}}%
\pgfpathlineto{\pgfqpoint{2.554960in}{4.115138in}}%
\pgfpathlineto{\pgfqpoint{2.563640in}{4.183837in}}%
\pgfpathlineto{\pgfqpoint{2.570460in}{4.222257in}}%
\pgfpathlineto{\pgfqpoint{2.576040in}{4.243183in}}%
\pgfpathlineto{\pgfqpoint{2.580380in}{4.252814in}}%
\pgfpathlineto{\pgfqpoint{2.584100in}{4.256404in}}%
\pgfpathlineto{\pgfqpoint{2.586580in}{4.256398in}}%
\pgfpathlineto{\pgfqpoint{2.589680in}{4.253692in}}%
\pgfpathlineto{\pgfqpoint{2.593400in}{4.246495in}}%
\pgfpathlineto{\pgfqpoint{2.597740in}{4.232682in}}%
\pgfpathlineto{\pgfqpoint{2.603320in}{4.206445in}}%
\pgfpathlineto{\pgfqpoint{2.610140in}{4.161691in}}%
\pgfpathlineto{\pgfqpoint{2.618200in}{4.091440in}}%
\pgfpathlineto{\pgfqpoint{2.627500in}{3.988353in}}%
\pgfpathlineto{\pgfqpoint{2.638040in}{3.845575in}}%
\pgfpathlineto{\pgfqpoint{2.651060in}{3.636723in}}%
\pgfpathlineto{\pgfqpoint{2.667800in}{3.328985in}}%
\pgfpathlineto{\pgfqpoint{2.724840in}{2.245878in}}%
\pgfpathlineto{\pgfqpoint{2.737860in}{2.055793in}}%
\pgfpathlineto{\pgfqpoint{2.748400in}{1.931213in}}%
\pgfpathlineto{\pgfqpoint{2.757700in}{1.845957in}}%
\pgfpathlineto{\pgfqpoint{2.765140in}{1.795639in}}%
\pgfpathlineto{\pgfqpoint{2.771340in}{1.766373in}}%
\pgfpathlineto{\pgfqpoint{2.776300in}{1.751441in}}%
\pgfpathlineto{\pgfqpoint{2.780640in}{1.744635in}}%
\pgfpathlineto{\pgfqpoint{2.783740in}{1.743368in}}%
\pgfpathlineto{\pgfqpoint{2.786220in}{1.744514in}}%
\pgfpathlineto{\pgfqpoint{2.789320in}{1.748644in}}%
\pgfpathlineto{\pgfqpoint{2.793040in}{1.757542in}}%
\pgfpathlineto{\pgfqpoint{2.798000in}{1.776050in}}%
\pgfpathlineto{\pgfqpoint{2.803580in}{1.805824in}}%
\pgfpathlineto{\pgfqpoint{2.810400in}{1.854780in}}%
\pgfpathlineto{\pgfqpoint{2.818460in}{1.929766in}}%
\pgfpathlineto{\pgfqpoint{2.827760in}{2.037910in}}%
\pgfpathlineto{\pgfqpoint{2.838920in}{2.195216in}}%
\pgfpathlineto{\pgfqpoint{2.852560in}{2.421574in}}%
\pgfpathlineto{\pgfqpoint{2.870540in}{2.760203in}}%
\pgfpathlineto{\pgfqpoint{2.917660in}{3.666124in}}%
\pgfpathlineto{\pgfqpoint{2.931920in}{3.888281in}}%
\pgfpathlineto{\pgfqpoint{2.943080in}{4.030731in}}%
\pgfpathlineto{\pgfqpoint{2.953000in}{4.130353in}}%
\pgfpathlineto{\pgfqpoint{2.961060in}{4.190892in}}%
\pgfpathlineto{\pgfqpoint{2.967880in}{4.227082in}}%
\pgfpathlineto{\pgfqpoint{2.973460in}{4.246144in}}%
\pgfpathlineto{\pgfqpoint{2.977800in}{4.254310in}}%
\pgfpathlineto{\pgfqpoint{2.980900in}{4.256550in}}%
\pgfpathlineto{\pgfqpoint{2.983380in}{4.256183in}}%
\pgfpathlineto{\pgfqpoint{2.986480in}{4.253026in}}%
\pgfpathlineto{\pgfqpoint{2.990200in}{4.245289in}}%
\pgfpathlineto{\pgfqpoint{2.994540in}{4.230851in}}%
\pgfpathlineto{\pgfqpoint{3.000120in}{4.203824in}}%
\pgfpathlineto{\pgfqpoint{3.006940in}{4.158129in}}%
\pgfpathlineto{\pgfqpoint{3.015000in}{4.086816in}}%
\pgfpathlineto{\pgfqpoint{3.024300in}{3.982592in}}%
\pgfpathlineto{\pgfqpoint{3.035460in}{3.829424in}}%
\pgfpathlineto{\pgfqpoint{3.048480in}{3.617909in}}%
\pgfpathlineto{\pgfqpoint{3.065840in}{3.295883in}}%
\pgfpathlineto{\pgfqpoint{3.119160in}{2.278398in}}%
\pgfpathlineto{\pgfqpoint{3.132800in}{2.074325in}}%
\pgfpathlineto{\pgfqpoint{3.143960in}{1.939374in}}%
\pgfpathlineto{\pgfqpoint{3.153260in}{1.852114in}}%
\pgfpathlineto{\pgfqpoint{3.161320in}{1.796509in}}%
\pgfpathlineto{\pgfqpoint{3.167520in}{1.766954in}}%
\pgfpathlineto{\pgfqpoint{3.173100in}{1.750427in}}%
\pgfpathlineto{\pgfqpoint{3.176820in}{1.744774in}}%
\pgfpathlineto{\pgfqpoint{3.179920in}{1.743359in}}%
\pgfpathlineto{\pgfqpoint{3.182400in}{1.744386in}}%
\pgfpathlineto{\pgfqpoint{3.185500in}{1.748368in}}%
\pgfpathlineto{\pgfqpoint{3.189220in}{1.757090in}}%
\pgfpathlineto{\pgfqpoint{3.194180in}{1.775365in}}%
\pgfpathlineto{\pgfqpoint{3.199760in}{1.804880in}}%
\pgfpathlineto{\pgfqpoint{3.206580in}{1.853530in}}%
\pgfpathlineto{\pgfqpoint{3.214640in}{1.928173in}}%
\pgfpathlineto{\pgfqpoint{3.223940in}{2.035950in}}%
\pgfpathlineto{\pgfqpoint{3.235100in}{2.192871in}}%
\pgfpathlineto{\pgfqpoint{3.248740in}{2.418856in}}%
\pgfpathlineto{\pgfqpoint{3.266720in}{2.757191in}}%
\pgfpathlineto{\pgfqpoint{3.313840in}{3.663528in}}%
\pgfpathlineto{\pgfqpoint{3.328100in}{3.886120in}}%
\pgfpathlineto{\pgfqpoint{3.339260in}{4.028987in}}%
\pgfpathlineto{\pgfqpoint{3.349180in}{4.129022in}}%
\pgfpathlineto{\pgfqpoint{3.357240in}{4.189920in}}%
\pgfpathlineto{\pgfqpoint{3.364060in}{4.226424in}}%
\pgfpathlineto{\pgfqpoint{3.369640in}{4.245750in}}%
\pgfpathlineto{\pgfqpoint{3.373980in}{4.254122in}}%
\pgfpathlineto{\pgfqpoint{3.377080in}{4.256511in}}%
\pgfpathlineto{\pgfqpoint{3.379560in}{4.256263in}}%
\pgfpathlineto{\pgfqpoint{3.382660in}{4.253253in}}%
\pgfpathlineto{\pgfqpoint{3.386380in}{4.245694in}}%
\pgfpathlineto{\pgfqpoint{3.390720in}{4.231461in}}%
\pgfpathlineto{\pgfqpoint{3.396300in}{4.204693in}}%
\pgfpathlineto{\pgfqpoint{3.403120in}{4.159307in}}%
\pgfpathlineto{\pgfqpoint{3.411180in}{4.088341in}}%
\pgfpathlineto{\pgfqpoint{3.420480in}{3.984490in}}%
\pgfpathlineto{\pgfqpoint{3.431020in}{3.840942in}}%
\pgfpathlineto{\pgfqpoint{3.444040in}{3.631315in}}%
\pgfpathlineto{\pgfqpoint{3.460780in}{3.322922in}}%
\pgfpathlineto{\pgfqpoint{3.517200in}{2.250777in}}%
\pgfpathlineto{\pgfqpoint{3.530220in}{2.059828in}}%
\pgfpathlineto{\pgfqpoint{3.540760in}{1.934427in}}%
\pgfpathlineto{\pgfqpoint{3.550060in}{1.848375in}}%
\pgfpathlineto{\pgfqpoint{3.558120in}{1.793878in}}%
\pgfpathlineto{\pgfqpoint{3.564320in}{1.765202in}}%
\pgfpathlineto{\pgfqpoint{3.569280in}{1.750751in}}%
\pgfpathlineto{\pgfqpoint{3.573620in}{1.744368in}}%
\pgfpathlineto{\pgfqpoint{3.576720in}{1.743405in}}%
\pgfpathlineto{\pgfqpoint{3.579200in}{1.744794in}}%
\pgfpathlineto{\pgfqpoint{3.582300in}{1.749226in}}%
\pgfpathlineto{\pgfqpoint{3.586020in}{1.758486in}}%
\pgfpathlineto{\pgfqpoint{3.590980in}{1.777472in}}%
\pgfpathlineto{\pgfqpoint{3.597180in}{1.811715in}}%
\pgfpathlineto{\pgfqpoint{3.604000in}{1.862530in}}%
\pgfpathlineto{\pgfqpoint{3.612060in}{1.939603in}}%
\pgfpathlineto{\pgfqpoint{3.621980in}{2.058091in}}%
\pgfpathlineto{\pgfqpoint{3.633140in}{2.219255in}}%
\pgfpathlineto{\pgfqpoint{3.647400in}{2.460488in}}%
\pgfpathlineto{\pgfqpoint{3.666620in}{2.827647in}}%
\pgfpathlineto{\pgfqpoint{3.707540in}{3.618285in}}%
\pgfpathlineto{\pgfqpoint{3.722420in}{3.857229in}}%
\pgfpathlineto{\pgfqpoint{3.734200in}{4.012835in}}%
\pgfpathlineto{\pgfqpoint{3.744120in}{4.116605in}}%
\pgfpathlineto{\pgfqpoint{3.752800in}{4.184902in}}%
\pgfpathlineto{\pgfqpoint{3.759620in}{4.222994in}}%
\pgfpathlineto{\pgfqpoint{3.765200in}{4.243646in}}%
\pgfpathlineto{\pgfqpoint{3.769540in}{4.253060in}}%
\pgfpathlineto{\pgfqpoint{3.773260in}{4.256464in}}%
\pgfpathlineto{\pgfqpoint{3.775740in}{4.256334in}}%
\pgfpathlineto{\pgfqpoint{3.778840in}{4.253473in}}%
\pgfpathlineto{\pgfqpoint{3.782560in}{4.246091in}}%
\pgfpathlineto{\pgfqpoint{3.786900in}{4.232063in}}%
\pgfpathlineto{\pgfqpoint{3.792480in}{4.205555in}}%
\pgfpathlineto{\pgfqpoint{3.799300in}{4.160477in}}%
\pgfpathlineto{\pgfqpoint{3.807360in}{4.089861in}}%
\pgfpathlineto{\pgfqpoint{3.816660in}{3.986383in}}%
\pgfpathlineto{\pgfqpoint{3.827200in}{3.843210in}}%
\pgfpathlineto{\pgfqpoint{3.840220in}{3.633961in}}%
\pgfpathlineto{\pgfqpoint{3.856960in}{3.325886in}}%
\pgfpathlineto{\pgfqpoint{3.913380in}{2.253234in}}%
\pgfpathlineto{\pgfqpoint{3.926400in}{2.061854in}}%
\pgfpathlineto{\pgfqpoint{3.936940in}{1.936043in}}%
\pgfpathlineto{\pgfqpoint{3.946240in}{1.849594in}}%
\pgfpathlineto{\pgfqpoint{3.954300in}{1.794733in}}%
\pgfpathlineto{\pgfqpoint{3.960500in}{1.765769in}}%
\pgfpathlineto{\pgfqpoint{3.965460in}{1.751083in}}%
\pgfpathlineto{\pgfqpoint{3.969800in}{1.744493in}}%
\pgfpathlineto{\pgfqpoint{3.972900in}{1.743381in}}%
\pgfpathlineto{\pgfqpoint{3.975380in}{1.744651in}}%
\pgfpathlineto{\pgfqpoint{3.978480in}{1.748936in}}%
\pgfpathlineto{\pgfqpoint{3.982200in}{1.758019in}}%
\pgfpathlineto{\pgfqpoint{3.987160in}{1.776771in}}%
\pgfpathlineto{\pgfqpoint{3.993360in}{1.810729in}}%
\pgfpathlineto{\pgfqpoint{4.000180in}{1.861239in}}%
\pgfpathlineto{\pgfqpoint{4.008240in}{1.937971in}}%
\pgfpathlineto{\pgfqpoint{4.018160in}{2.056074in}}%
\pgfpathlineto{\pgfqpoint{4.029320in}{2.216861in}}%
\pgfpathlineto{\pgfqpoint{4.042960in}{2.446566in}}%
\pgfpathlineto{\pgfqpoint{4.062180in}{2.812316in}}%
\pgfpathlineto{\pgfqpoint{4.104340in}{3.626388in}}%
\pgfpathlineto{\pgfqpoint{4.118600in}{3.854994in}}%
\pgfpathlineto{\pgfqpoint{4.130380in}{4.011029in}}%
\pgfpathlineto{\pgfqpoint{4.140300in}{4.115207in}}%
\pgfpathlineto{\pgfqpoint{4.148980in}{4.183887in}}%
\pgfpathlineto{\pgfqpoint{4.155800in}{4.222293in}}%
\pgfpathlineto{\pgfqpoint{4.161380in}{4.243208in}}%
\pgfpathlineto{\pgfqpoint{4.165720in}{4.252828in}}%
\pgfpathlineto{\pgfqpoint{4.169440in}{4.256410in}}%
\pgfpathlineto{\pgfqpoint{4.171920in}{4.256399in}}%
\pgfpathlineto{\pgfqpoint{4.175020in}{4.253686in}}%
\pgfpathlineto{\pgfqpoint{4.178740in}{4.246481in}}%
\pgfpathlineto{\pgfqpoint{4.183080in}{4.232658in}}%
\pgfpathlineto{\pgfqpoint{4.188660in}{4.206410in}}%
\pgfpathlineto{\pgfqpoint{4.195480in}{4.161641in}}%
\pgfpathlineto{\pgfqpoint{4.203540in}{4.091374in}}%
\pgfpathlineto{\pgfqpoint{4.212840in}{3.988270in}}%
\pgfpathlineto{\pgfqpoint{4.223380in}{3.845473in}}%
\pgfpathlineto{\pgfqpoint{4.236400in}{3.636603in}}%
\pgfpathlineto{\pgfqpoint{4.253140in}{3.328849in}}%
\pgfpathlineto{\pgfqpoint{4.310180in}{2.245763in}}%
\pgfpathlineto{\pgfqpoint{4.323200in}{2.055697in}}%
\pgfpathlineto{\pgfqpoint{4.333740in}{1.931136in}}%
\pgfpathlineto{\pgfqpoint{4.343040in}{1.845898in}}%
\pgfpathlineto{\pgfqpoint{4.350480in}{1.795596in}}%
\pgfpathlineto{\pgfqpoint{4.356680in}{1.766342in}}%
\pgfpathlineto{\pgfqpoint{4.361640in}{1.751422in}}%
\pgfpathlineto{\pgfqpoint{4.365980in}{1.744625in}}%
\pgfpathlineto{\pgfqpoint{4.369080in}{1.743364in}}%
\pgfpathlineto{\pgfqpoint{4.371560in}{1.744516in}}%
\pgfpathlineto{\pgfqpoint{4.374660in}{1.748653in}}%
\pgfpathlineto{\pgfqpoint{4.378380in}{1.757560in}}%
\pgfpathlineto{\pgfqpoint{4.383340in}{1.776078in}}%
\pgfpathlineto{\pgfqpoint{4.388920in}{1.805864in}}%
\pgfpathlineto{\pgfqpoint{4.395740in}{1.854834in}}%
\pgfpathlineto{\pgfqpoint{4.403800in}{1.929836in}}%
\pgfpathlineto{\pgfqpoint{4.413100in}{2.037998in}}%
\pgfpathlineto{\pgfqpoint{4.424260in}{2.195321in}}%
\pgfpathlineto{\pgfqpoint{4.437900in}{2.421698in}}%
\pgfpathlineto{\pgfqpoint{4.455880in}{2.760341in}}%
\pgfpathlineto{\pgfqpoint{4.503000in}{3.666246in}}%
\pgfpathlineto{\pgfqpoint{4.517260in}{3.888383in}}%
\pgfpathlineto{\pgfqpoint{4.528420in}{4.030815in}}%
\pgfpathlineto{\pgfqpoint{4.538340in}{4.130418in}}%
\pgfpathlineto{\pgfqpoint{4.546400in}{4.190941in}}%
\pgfpathlineto{\pgfqpoint{4.553220in}{4.227116in}}%
\pgfpathlineto{\pgfqpoint{4.558800in}{4.246166in}}%
\pgfpathlineto{\pgfqpoint{4.563140in}{4.254322in}}%
\pgfpathlineto{\pgfqpoint{4.566240in}{4.256556in}}%
\pgfpathlineto{\pgfqpoint{4.568720in}{4.256183in}}%
\pgfpathlineto{\pgfqpoint{4.571820in}{4.253019in}}%
\pgfpathlineto{\pgfqpoint{4.575540in}{4.245274in}}%
\pgfpathlineto{\pgfqpoint{4.579880in}{4.230827in}}%
\pgfpathlineto{\pgfqpoint{4.585460in}{4.203788in}}%
\pgfpathlineto{\pgfqpoint{4.592280in}{4.158079in}}%
\pgfpathlineto{\pgfqpoint{4.600340in}{4.086749in}}%
\pgfpathlineto{\pgfqpoint{4.609640in}{3.982508in}}%
\pgfpathlineto{\pgfqpoint{4.620800in}{3.829322in}}%
\pgfpathlineto{\pgfqpoint{4.633820in}{3.617788in}}%
\pgfpathlineto{\pgfqpoint{4.651180in}{3.295747in}}%
\pgfpathlineto{\pgfqpoint{4.704500in}{2.278281in}}%
\pgfpathlineto{\pgfqpoint{4.718140in}{2.074227in}}%
\pgfpathlineto{\pgfqpoint{4.729300in}{1.939295in}}%
\pgfpathlineto{\pgfqpoint{4.738600in}{1.852054in}}%
\pgfpathlineto{\pgfqpoint{4.746660in}{1.796465in}}%
\pgfpathlineto{\pgfqpoint{4.752860in}{1.766923in}}%
\pgfpathlineto{\pgfqpoint{4.758440in}{1.750408in}}%
\pgfpathlineto{\pgfqpoint{4.762160in}{1.744764in}}%
\pgfpathlineto{\pgfqpoint{4.765260in}{1.743355in}}%
\pgfpathlineto{\pgfqpoint{4.767740in}{1.744388in}}%
\pgfpathlineto{\pgfqpoint{4.770840in}{1.748377in}}%
\pgfpathlineto{\pgfqpoint{4.774560in}{1.757107in}}%
\pgfpathlineto{\pgfqpoint{4.779520in}{1.775392in}}%
\pgfpathlineto{\pgfqpoint{4.785100in}{1.804920in}}%
\pgfpathlineto{\pgfqpoint{4.791920in}{1.853584in}}%
\pgfpathlineto{\pgfqpoint{4.799980in}{1.928242in}}%
\pgfpathlineto{\pgfqpoint{4.809280in}{2.036037in}}%
\pgfpathlineto{\pgfqpoint{4.820440in}{2.192975in}}%
\pgfpathlineto{\pgfqpoint{4.834080in}{2.418979in}}%
\pgfpathlineto{\pgfqpoint{4.852060in}{2.757328in}}%
\pgfpathlineto{\pgfqpoint{4.899180in}{3.663649in}}%
\pgfpathlineto{\pgfqpoint{4.913440in}{3.886221in}}%
\pgfpathlineto{\pgfqpoint{4.924600in}{4.029070in}}%
\pgfpathlineto{\pgfqpoint{4.934520in}{4.129086in}}%
\pgfpathlineto{\pgfqpoint{4.942580in}{4.189968in}}%
\pgfpathlineto{\pgfqpoint{4.949400in}{4.226458in}}%
\pgfpathlineto{\pgfqpoint{4.954980in}{4.245772in}}%
\pgfpathlineto{\pgfqpoint{4.959320in}{4.254135in}}%
\pgfpathlineto{\pgfqpoint{4.962420in}{4.256517in}}%
\pgfpathlineto{\pgfqpoint{4.964900in}{4.256263in}}%
\pgfpathlineto{\pgfqpoint{4.968000in}{4.253247in}}%
\pgfpathlineto{\pgfqpoint{4.971720in}{4.245679in}}%
\pgfpathlineto{\pgfqpoint{4.976060in}{4.231437in}}%
\pgfpathlineto{\pgfqpoint{4.981640in}{4.204657in}}%
\pgfpathlineto{\pgfqpoint{4.988460in}{4.159257in}}%
\pgfpathlineto{\pgfqpoint{4.996520in}{4.088275in}}%
\pgfpathlineto{\pgfqpoint{5.005820in}{3.984407in}}%
\pgfpathlineto{\pgfqpoint{5.016360in}{3.840841in}}%
\pgfpathlineto{\pgfqpoint{5.029380in}{3.631196in}}%
\pgfpathlineto{\pgfqpoint{5.046120in}{3.322788in}}%
\pgfpathlineto{\pgfqpoint{5.102540in}{2.250663in}}%
\pgfpathlineto{\pgfqpoint{5.115560in}{2.059733in}}%
\pgfpathlineto{\pgfqpoint{5.126100in}{1.934350in}}%
\pgfpathlineto{\pgfqpoint{5.135400in}{1.848316in}}%
\pgfpathlineto{\pgfqpoint{5.143460in}{1.793835in}}%
\pgfpathlineto{\pgfqpoint{5.149660in}{1.765173in}}%
\pgfpathlineto{\pgfqpoint{5.154620in}{1.750732in}}%
\pgfpathlineto{\pgfqpoint{5.158960in}{1.744359in}}%
\pgfpathlineto{\pgfqpoint{5.162060in}{1.743402in}}%
\pgfpathlineto{\pgfqpoint{5.164540in}{1.744796in}}%
\pgfpathlineto{\pgfqpoint{5.167640in}{1.749236in}}%
\pgfpathlineto{\pgfqpoint{5.171360in}{1.758504in}}%
\pgfpathlineto{\pgfqpoint{5.176320in}{1.777500in}}%
\pgfpathlineto{\pgfqpoint{5.182520in}{1.811757in}}%
\pgfpathlineto{\pgfqpoint{5.189340in}{1.862585in}}%
\pgfpathlineto{\pgfqpoint{5.197400in}{1.939674in}}%
\pgfpathlineto{\pgfqpoint{5.207320in}{2.058180in}}%
\pgfpathlineto{\pgfqpoint{5.218480in}{2.219362in}}%
\pgfpathlineto{\pgfqpoint{5.232740in}{2.460612in}}%
\pgfpathlineto{\pgfqpoint{5.251960in}{2.827785in}}%
\pgfpathlineto{\pgfqpoint{5.292880in}{3.618408in}}%
\pgfpathlineto{\pgfqpoint{5.307760in}{3.857333in}}%
\pgfpathlineto{\pgfqpoint{5.319540in}{4.012920in}}%
\pgfpathlineto{\pgfqpoint{5.329460in}{4.116672in}}%
\pgfpathlineto{\pgfqpoint{5.338140in}{4.184952in}}%
\pgfpathlineto{\pgfqpoint{5.344960in}{4.223029in}}%
\pgfpathlineto{\pgfqpoint{5.350540in}{4.243669in}}%
\pgfpathlineto{\pgfqpoint{5.354880in}{4.253074in}}%
\pgfpathlineto{\pgfqpoint{5.358600in}{4.256470in}}%
\pgfpathlineto{\pgfqpoint{5.361080in}{4.256335in}}%
\pgfpathlineto{\pgfqpoint{5.364180in}{4.253467in}}%
\pgfpathlineto{\pgfqpoint{5.367900in}{4.246077in}}%
\pgfpathlineto{\pgfqpoint{5.372240in}{4.232040in}}%
\pgfpathlineto{\pgfqpoint{5.377820in}{4.205520in}}%
\pgfpathlineto{\pgfqpoint{5.384640in}{4.160428in}}%
\pgfpathlineto{\pgfqpoint{5.392700in}{4.089795in}}%
\pgfpathlineto{\pgfqpoint{5.402000in}{3.986300in}}%
\pgfpathlineto{\pgfqpoint{5.412540in}{3.843110in}}%
\pgfpathlineto{\pgfqpoint{5.425560in}{3.633843in}}%
\pgfpathlineto{\pgfqpoint{5.442300in}{3.325753in}}%
\pgfpathlineto{\pgfqpoint{5.498720in}{2.253120in}}%
\pgfpathlineto{\pgfqpoint{5.511740in}{2.061760in}}%
\pgfpathlineto{\pgfqpoint{5.522280in}{1.935967in}}%
\pgfpathlineto{\pgfqpoint{5.531580in}{1.849536in}}%
\pgfpathlineto{\pgfqpoint{5.539640in}{1.794691in}}%
\pgfpathlineto{\pgfqpoint{5.545840in}{1.765739in}}%
\pgfpathlineto{\pgfqpoint{5.550800in}{1.751064in}}%
\pgfpathlineto{\pgfqpoint{5.555140in}{1.744483in}}%
\pgfpathlineto{\pgfqpoint{5.558240in}{1.743378in}}%
\pgfpathlineto{\pgfqpoint{5.560720in}{1.744654in}}%
\pgfpathlineto{\pgfqpoint{5.563820in}{1.748945in}}%
\pgfpathlineto{\pgfqpoint{5.567540in}{1.758037in}}%
\pgfpathlineto{\pgfqpoint{5.572500in}{1.776799in}}%
\pgfpathlineto{\pgfqpoint{5.578700in}{1.810770in}}%
\pgfpathlineto{\pgfqpoint{5.585520in}{1.861294in}}%
\pgfpathlineto{\pgfqpoint{5.593580in}{1.938042in}}%
\pgfpathlineto{\pgfqpoint{5.603500in}{2.056162in}}%
\pgfpathlineto{\pgfqpoint{5.614660in}{2.216967in}}%
\pgfpathlineto{\pgfqpoint{5.628300in}{2.446689in}}%
\pgfpathlineto{\pgfqpoint{5.647520in}{2.812453in}}%
\pgfpathlineto{\pgfqpoint{5.689680in}{3.626510in}}%
\pgfpathlineto{\pgfqpoint{5.703940in}{3.855097in}}%
\pgfpathlineto{\pgfqpoint{5.715720in}{4.011114in}}%
\pgfpathlineto{\pgfqpoint{5.725640in}{4.115273in}}%
\pgfpathlineto{\pgfqpoint{5.734320in}{4.183937in}}%
\pgfpathlineto{\pgfqpoint{5.741140in}{4.222329in}}%
\pgfpathlineto{\pgfqpoint{5.746720in}{4.243231in}}%
\pgfpathlineto{\pgfqpoint{5.751060in}{4.252843in}}%
\pgfpathlineto{\pgfqpoint{5.754780in}{4.256417in}}%
\pgfpathlineto{\pgfqpoint{5.757260in}{4.256400in}}%
\pgfpathlineto{\pgfqpoint{5.760360in}{4.253681in}}%
\pgfpathlineto{\pgfqpoint{5.764080in}{4.246467in}}%
\pgfpathlineto{\pgfqpoint{5.768420in}{4.232635in}}%
\pgfpathlineto{\pgfqpoint{5.774000in}{4.206375in}}%
\pgfpathlineto{\pgfqpoint{5.780820in}{4.161593in}}%
\pgfpathlineto{\pgfqpoint{5.788880in}{4.091309in}}%
\pgfpathlineto{\pgfqpoint{5.798180in}{3.988188in}}%
\pgfpathlineto{\pgfqpoint{5.808720in}{3.845374in}}%
\pgfpathlineto{\pgfqpoint{5.821740in}{3.636486in}}%
\pgfpathlineto{\pgfqpoint{5.838480in}{3.328716in}}%
\pgfpathlineto{\pgfqpoint{5.895520in}{2.245650in}}%
\pgfpathlineto{\pgfqpoint{5.908540in}{2.055603in}}%
\pgfpathlineto{\pgfqpoint{5.919080in}{1.931061in}}%
\pgfpathlineto{\pgfqpoint{5.928380in}{1.845840in}}%
\pgfpathlineto{\pgfqpoint{5.935820in}{1.795553in}}%
\pgfpathlineto{\pgfqpoint{5.942020in}{1.766313in}}%
\pgfpathlineto{\pgfqpoint{5.946980in}{1.751402in}}%
\pgfpathlineto{\pgfqpoint{5.951320in}{1.744615in}}%
\pgfpathlineto{\pgfqpoint{5.954420in}{1.743361in}}%
\pgfpathlineto{\pgfqpoint{5.956900in}{1.744518in}}%
\pgfpathlineto{\pgfqpoint{5.960000in}{1.748661in}}%
\pgfpathlineto{\pgfqpoint{5.963720in}{1.757576in}}%
\pgfpathlineto{\pgfqpoint{5.968680in}{1.776106in}}%
\pgfpathlineto{\pgfqpoint{5.974260in}{1.805903in}}%
\pgfpathlineto{\pgfqpoint{5.981080in}{1.854887in}}%
\pgfpathlineto{\pgfqpoint{5.989140in}{1.929905in}}%
\pgfpathlineto{\pgfqpoint{5.998440in}{2.038083in}}%
\pgfpathlineto{\pgfqpoint{6.009600in}{2.195424in}}%
\pgfpathlineto{\pgfqpoint{6.023240in}{2.421818in}}%
\pgfpathlineto{\pgfqpoint{6.041220in}{2.760475in}}%
\pgfpathlineto{\pgfqpoint{6.088340in}{3.666364in}}%
\pgfpathlineto{\pgfqpoint{6.102600in}{3.888483in}}%
\pgfpathlineto{\pgfqpoint{6.113760in}{4.030896in}}%
\pgfpathlineto{\pgfqpoint{6.123680in}{4.130481in}}%
\pgfpathlineto{\pgfqpoint{6.131740in}{4.190988in}}%
\pgfpathlineto{\pgfqpoint{6.138560in}{4.227149in}}%
\pgfpathlineto{\pgfqpoint{6.144140in}{4.246187in}}%
\pgfpathlineto{\pgfqpoint{6.148480in}{4.254334in}}%
\pgfpathlineto{\pgfqpoint{6.151580in}{4.256561in}}%
\pgfpathlineto{\pgfqpoint{6.154060in}{4.256183in}}%
\pgfpathlineto{\pgfqpoint{6.157160in}{4.253013in}}%
\pgfpathlineto{\pgfqpoint{6.160880in}{4.245260in}}%
\pgfpathlineto{\pgfqpoint{6.165220in}{4.230803in}}%
\pgfpathlineto{\pgfqpoint{6.170800in}{4.203752in}}%
\pgfpathlineto{\pgfqpoint{6.177620in}{4.158029in}}%
\pgfpathlineto{\pgfqpoint{6.185680in}{4.086683in}}%
\pgfpathlineto{\pgfqpoint{6.194980in}{3.982425in}}%
\pgfpathlineto{\pgfqpoint{6.206140in}{3.829221in}}%
\pgfpathlineto{\pgfqpoint{6.219160in}{3.617671in}}%
\pgfpathlineto{\pgfqpoint{6.236520in}{3.295615in}}%
\pgfpathlineto{\pgfqpoint{6.289840in}{2.278167in}}%
\pgfpathlineto{\pgfqpoint{6.303480in}{2.074132in}}%
\pgfpathlineto{\pgfqpoint{6.314640in}{1.939219in}}%
\pgfpathlineto{\pgfqpoint{6.323940in}{1.851995in}}%
\pgfpathlineto{\pgfqpoint{6.332000in}{1.796422in}}%
\pgfpathlineto{\pgfqpoint{6.338200in}{1.766893in}}%
\pgfpathlineto{\pgfqpoint{6.343780in}{1.750390in}}%
\pgfpathlineto{\pgfqpoint{6.347500in}{1.744753in}}%
\pgfpathlineto{\pgfqpoint{6.350600in}{1.743352in}}%
\pgfpathlineto{\pgfqpoint{6.353080in}{1.744390in}}%
\pgfpathlineto{\pgfqpoint{6.356180in}{1.748385in}}%
\pgfpathlineto{\pgfqpoint{6.359900in}{1.757123in}}%
\pgfpathlineto{\pgfqpoint{6.364860in}{1.775419in}}%
\pgfpathlineto{\pgfqpoint{6.370440in}{1.804958in}}%
\pgfpathlineto{\pgfqpoint{6.377260in}{1.853636in}}%
\pgfpathlineto{\pgfqpoint{6.385320in}{1.928310in}}%
\pgfpathlineto{\pgfqpoint{6.394620in}{2.036122in}}%
\pgfpathlineto{\pgfqpoint{6.405780in}{2.193077in}}%
\pgfpathlineto{\pgfqpoint{6.419420in}{2.419098in}}%
\pgfpathlineto{\pgfqpoint{6.437400in}{2.757462in}}%
\pgfpathlineto{\pgfqpoint{6.484520in}{3.663767in}}%
\pgfpathlineto{\pgfqpoint{6.498780in}{3.886321in}}%
\pgfpathlineto{\pgfqpoint{6.509940in}{4.029151in}}%
\pgfpathlineto{\pgfqpoint{6.519860in}{4.129149in}}%
\pgfpathlineto{\pgfqpoint{6.527920in}{4.190015in}}%
\pgfpathlineto{\pgfqpoint{6.534740in}{4.226492in}}%
\pgfpathlineto{\pgfqpoint{6.540320in}{4.245793in}}%
\pgfpathlineto{\pgfqpoint{6.544660in}{4.254147in}}%
\pgfpathlineto{\pgfqpoint{6.547760in}{4.256522in}}%
\pgfpathlineto{\pgfqpoint{6.550240in}{4.256263in}}%
\pgfpathlineto{\pgfqpoint{6.553340in}{4.253241in}}%
\pgfpathlineto{\pgfqpoint{6.557060in}{4.245665in}}%
\pgfpathlineto{\pgfqpoint{6.561400in}{4.231414in}}%
\pgfpathlineto{\pgfqpoint{6.566980in}{4.204622in}}%
\pgfpathlineto{\pgfqpoint{6.573800in}{4.159208in}}%
\pgfpathlineto{\pgfqpoint{6.581860in}{4.088211in}}%
\pgfpathlineto{\pgfqpoint{6.591160in}{3.984325in}}%
\pgfpathlineto{\pgfqpoint{6.601700in}{3.840743in}}%
\pgfpathlineto{\pgfqpoint{6.614720in}{3.631080in}}%
\pgfpathlineto{\pgfqpoint{6.631460in}{3.322656in}}%
\pgfpathlineto{\pgfqpoint{6.687880in}{2.250552in}}%
\pgfpathlineto{\pgfqpoint{6.700900in}{2.059640in}}%
\pgfpathlineto{\pgfqpoint{6.711440in}{1.934275in}}%
\pgfpathlineto{\pgfqpoint{6.720740in}{1.848259in}}%
\pgfpathlineto{\pgfqpoint{6.728800in}{1.793794in}}%
\pgfpathlineto{\pgfqpoint{6.735000in}{1.765144in}}%
\pgfpathlineto{\pgfqpoint{6.739960in}{1.750714in}}%
\pgfpathlineto{\pgfqpoint{6.744300in}{1.744350in}}%
\pgfpathlineto{\pgfqpoint{6.747400in}{1.743399in}}%
\pgfpathlineto{\pgfqpoint{6.749880in}{1.744799in}}%
\pgfpathlineto{\pgfqpoint{6.752980in}{1.749245in}}%
\pgfpathlineto{\pgfqpoint{6.756700in}{1.758521in}}%
\pgfpathlineto{\pgfqpoint{6.761660in}{1.777528in}}%
\pgfpathlineto{\pgfqpoint{6.767860in}{1.811797in}}%
\pgfpathlineto{\pgfqpoint{6.774680in}{1.862639in}}%
\pgfpathlineto{\pgfqpoint{6.782740in}{1.939744in}}%
\pgfpathlineto{\pgfqpoint{6.792660in}{2.058267in}}%
\pgfpathlineto{\pgfqpoint{6.803820in}{2.219466in}}%
\pgfpathlineto{\pgfqpoint{6.818080in}{2.460734in}}%
\pgfpathlineto{\pgfqpoint{6.837300in}{2.827919in}}%
\pgfpathlineto{\pgfqpoint{6.878220in}{3.618528in}}%
\pgfpathlineto{\pgfqpoint{6.893100in}{3.857434in}}%
\pgfpathlineto{\pgfqpoint{6.904880in}{4.013003in}}%
\pgfpathlineto{\pgfqpoint{6.914800in}{4.116737in}}%
\pgfpathlineto{\pgfqpoint{6.923480in}{4.185000in}}%
\pgfpathlineto{\pgfqpoint{6.930300in}{4.223064in}}%
\pgfpathlineto{\pgfqpoint{6.935880in}{4.243692in}}%
\pgfpathlineto{\pgfqpoint{6.940220in}{4.253088in}}%
\pgfpathlineto{\pgfqpoint{6.943940in}{4.256476in}}%
\pgfpathlineto{\pgfqpoint{6.946420in}{4.256336in}}%
\pgfpathlineto{\pgfqpoint{6.949520in}{4.253462in}}%
\pgfpathlineto{\pgfqpoint{6.953240in}{4.246063in}}%
\pgfpathlineto{\pgfqpoint{6.957580in}{4.232017in}}%
\pgfpathlineto{\pgfqpoint{6.963160in}{4.205485in}}%
\pgfpathlineto{\pgfqpoint{6.969980in}{4.160380in}}%
\pgfpathlineto{\pgfqpoint{6.978040in}{4.089731in}}%
\pgfpathlineto{\pgfqpoint{6.987340in}{3.986219in}}%
\pgfpathlineto{\pgfqpoint{6.997880in}{3.843013in}}%
\pgfpathlineto{\pgfqpoint{7.010900in}{3.633728in}}%
\pgfpathlineto{\pgfqpoint{7.027640in}{3.325623in}}%
\pgfpathlineto{\pgfqpoint{7.084060in}{2.253009in}}%
\pgfpathlineto{\pgfqpoint{7.097080in}{2.061667in}}%
\pgfpathlineto{\pgfqpoint{7.107620in}{1.935892in}}%
\pgfpathlineto{\pgfqpoint{7.116920in}{1.849478in}}%
\pgfpathlineto{\pgfqpoint{7.124980in}{1.794649in}}%
\pgfpathlineto{\pgfqpoint{7.131180in}{1.765710in}}%
\pgfpathlineto{\pgfqpoint{7.136140in}{1.751045in}}%
\pgfpathlineto{\pgfqpoint{7.140480in}{1.744474in}}%
\pgfpathlineto{\pgfqpoint{7.143580in}{1.743375in}}%
\pgfpathlineto{\pgfqpoint{7.146060in}{1.744656in}}%
\pgfpathlineto{\pgfqpoint{7.149160in}{1.748954in}}%
\pgfpathlineto{\pgfqpoint{7.152880in}{1.758053in}}%
\pgfpathlineto{\pgfqpoint{7.157840in}{1.776826in}}%
\pgfpathlineto{\pgfqpoint{7.164040in}{1.810810in}}%
\pgfpathlineto{\pgfqpoint{7.170860in}{1.861347in}}%
\pgfpathlineto{\pgfqpoint{7.178920in}{1.938110in}}%
\pgfpathlineto{\pgfqpoint{7.188840in}{2.056248in}}%
\pgfpathlineto{\pgfqpoint{7.200000in}{2.217070in}}%
\pgfpathlineto{\pgfqpoint{7.200000in}{2.217070in}}%
\pgfusepath{stroke}%
\end{pgfscope}%
\begin{pgfscope}%
\pgfsetrectcap%
\pgfsetmiterjoin%
\pgfsetlinewidth{1.003750pt}%
\definecolor{currentstroke}{rgb}{0.000000,0.000000,0.000000}%
\pgfsetstrokecolor{currentstroke}%
\pgfsetdash{}{0pt}%
\pgfpathmoveto{\pgfqpoint{1.000000in}{0.600000in}}%
\pgfpathlineto{\pgfqpoint{1.000000in}{5.400000in}}%
\pgfusepath{stroke}%
\end{pgfscope}%
\begin{pgfscope}%
\pgfsetrectcap%
\pgfsetmiterjoin%
\pgfsetlinewidth{1.003750pt}%
\definecolor{currentstroke}{rgb}{0.000000,0.000000,0.000000}%
\pgfsetstrokecolor{currentstroke}%
\pgfsetdash{}{0pt}%
\pgfpathmoveto{\pgfqpoint{7.200000in}{0.600000in}}%
\pgfpathlineto{\pgfqpoint{7.200000in}{5.400000in}}%
\pgfusepath{stroke}%
\end{pgfscope}%
\begin{pgfscope}%
\pgfsetrectcap%
\pgfsetmiterjoin%
\pgfsetlinewidth{1.003750pt}%
\definecolor{currentstroke}{rgb}{0.000000,0.000000,0.000000}%
\pgfsetstrokecolor{currentstroke}%
\pgfsetdash{}{0pt}%
\pgfpathmoveto{\pgfqpoint{1.000000in}{0.600000in}}%
\pgfpathlineto{\pgfqpoint{7.200000in}{0.600000in}}%
\pgfusepath{stroke}%
\end{pgfscope}%
\begin{pgfscope}%
\pgfsetrectcap%
\pgfsetmiterjoin%
\pgfsetlinewidth{1.003750pt}%
\definecolor{currentstroke}{rgb}{0.000000,0.000000,0.000000}%
\pgfsetstrokecolor{currentstroke}%
\pgfsetdash{}{0pt}%
\pgfpathmoveto{\pgfqpoint{1.000000in}{5.400000in}}%
\pgfpathlineto{\pgfqpoint{7.200000in}{5.400000in}}%
\pgfusepath{stroke}%
\end{pgfscope}%
\begin{pgfscope}%
\pgfpathrectangle{\pgfqpoint{1.000000in}{0.600000in}}{\pgfqpoint{6.200000in}{4.800000in}}%
\pgfusepath{clip}%
\pgfsetbuttcap%
\pgfsetroundjoin%
\pgfsetlinewidth{0.501875pt}%
\definecolor{currentstroke}{rgb}{0.000000,0.000000,0.000000}%
\pgfsetstrokecolor{currentstroke}%
\pgfsetdash{{1.000000pt}{3.000000pt}}{0.000000pt}%
\pgfpathmoveto{\pgfqpoint{1.000000in}{0.600000in}}%
\pgfpathlineto{\pgfqpoint{1.000000in}{5.400000in}}%
\pgfusepath{stroke}%
\end{pgfscope}%
\begin{pgfscope}%
\pgfsetbuttcap%
\pgfsetroundjoin%
\definecolor{currentfill}{rgb}{0.000000,0.000000,0.000000}%
\pgfsetfillcolor{currentfill}%
\pgfsetlinewidth{0.501875pt}%
\definecolor{currentstroke}{rgb}{0.000000,0.000000,0.000000}%
\pgfsetstrokecolor{currentstroke}%
\pgfsetdash{}{0pt}%
\pgfsys@defobject{currentmarker}{\pgfqpoint{0.000000in}{0.000000in}}{\pgfqpoint{0.000000in}{0.055556in}}{%
\pgfpathmoveto{\pgfqpoint{0.000000in}{0.000000in}}%
\pgfpathlineto{\pgfqpoint{0.000000in}{0.055556in}}%
\pgfusepath{stroke,fill}%
}%
\begin{pgfscope}%
\pgfsys@transformshift{1.000000in}{0.600000in}%
\pgfsys@useobject{currentmarker}{}%
\end{pgfscope}%
\end{pgfscope}%
\begin{pgfscope}%
\pgfsetbuttcap%
\pgfsetroundjoin%
\definecolor{currentfill}{rgb}{0.000000,0.000000,0.000000}%
\pgfsetfillcolor{currentfill}%
\pgfsetlinewidth{0.501875pt}%
\definecolor{currentstroke}{rgb}{0.000000,0.000000,0.000000}%
\pgfsetstrokecolor{currentstroke}%
\pgfsetdash{}{0pt}%
\pgfsys@defobject{currentmarker}{\pgfqpoint{0.000000in}{-0.055556in}}{\pgfqpoint{0.000000in}{0.000000in}}{%
\pgfpathmoveto{\pgfqpoint{0.000000in}{0.000000in}}%
\pgfpathlineto{\pgfqpoint{0.000000in}{-0.055556in}}%
\pgfusepath{stroke,fill}%
}%
\begin{pgfscope}%
\pgfsys@transformshift{1.000000in}{5.400000in}%
\pgfsys@useobject{currentmarker}{}%
\end{pgfscope}%
\end{pgfscope}%
\begin{pgfscope}%
\definecolor{textcolor}{rgb}{0.000000,0.000000,0.000000}%
\pgfsetstrokecolor{textcolor}%
\pgfsetfillcolor{textcolor}%
\pgftext[x=1.000000in,y=0.544444in,,top]{\color{textcolor}\rmfamily\fontsize{10.000000}{12.000000}\selectfont \(\displaystyle {0}\)}%
\end{pgfscope}%
\begin{pgfscope}%
\pgfpathrectangle{\pgfqpoint{1.000000in}{0.600000in}}{\pgfqpoint{6.200000in}{4.800000in}}%
\pgfusepath{clip}%
\pgfsetbuttcap%
\pgfsetroundjoin%
\pgfsetlinewidth{0.501875pt}%
\definecolor{currentstroke}{rgb}{0.000000,0.000000,0.000000}%
\pgfsetstrokecolor{currentstroke}%
\pgfsetdash{{1.000000pt}{3.000000pt}}{0.000000pt}%
\pgfpathmoveto{\pgfqpoint{2.240000in}{0.600000in}}%
\pgfpathlineto{\pgfqpoint{2.240000in}{5.400000in}}%
\pgfusepath{stroke}%
\end{pgfscope}%
\begin{pgfscope}%
\pgfsetbuttcap%
\pgfsetroundjoin%
\definecolor{currentfill}{rgb}{0.000000,0.000000,0.000000}%
\pgfsetfillcolor{currentfill}%
\pgfsetlinewidth{0.501875pt}%
\definecolor{currentstroke}{rgb}{0.000000,0.000000,0.000000}%
\pgfsetstrokecolor{currentstroke}%
\pgfsetdash{}{0pt}%
\pgfsys@defobject{currentmarker}{\pgfqpoint{0.000000in}{0.000000in}}{\pgfqpoint{0.000000in}{0.055556in}}{%
\pgfpathmoveto{\pgfqpoint{0.000000in}{0.000000in}}%
\pgfpathlineto{\pgfqpoint{0.000000in}{0.055556in}}%
\pgfusepath{stroke,fill}%
}%
\begin{pgfscope}%
\pgfsys@transformshift{2.240000in}{0.600000in}%
\pgfsys@useobject{currentmarker}{}%
\end{pgfscope}%
\end{pgfscope}%
\begin{pgfscope}%
\pgfsetbuttcap%
\pgfsetroundjoin%
\definecolor{currentfill}{rgb}{0.000000,0.000000,0.000000}%
\pgfsetfillcolor{currentfill}%
\pgfsetlinewidth{0.501875pt}%
\definecolor{currentstroke}{rgb}{0.000000,0.000000,0.000000}%
\pgfsetstrokecolor{currentstroke}%
\pgfsetdash{}{0pt}%
\pgfsys@defobject{currentmarker}{\pgfqpoint{0.000000in}{-0.055556in}}{\pgfqpoint{0.000000in}{0.000000in}}{%
\pgfpathmoveto{\pgfqpoint{0.000000in}{0.000000in}}%
\pgfpathlineto{\pgfqpoint{0.000000in}{-0.055556in}}%
\pgfusepath{stroke,fill}%
}%
\begin{pgfscope}%
\pgfsys@transformshift{2.240000in}{5.400000in}%
\pgfsys@useobject{currentmarker}{}%
\end{pgfscope}%
\end{pgfscope}%
\begin{pgfscope}%
\definecolor{textcolor}{rgb}{0.000000,0.000000,0.000000}%
\pgfsetstrokecolor{textcolor}%
\pgfsetfillcolor{textcolor}%
\pgftext[x=2.240000in,y=0.544444in,,top]{\color{textcolor}\rmfamily\fontsize{10.000000}{12.000000}\selectfont \(\displaystyle {20}\)}%
\end{pgfscope}%
\begin{pgfscope}%
\pgfpathrectangle{\pgfqpoint{1.000000in}{0.600000in}}{\pgfqpoint{6.200000in}{4.800000in}}%
\pgfusepath{clip}%
\pgfsetbuttcap%
\pgfsetroundjoin%
\pgfsetlinewidth{0.501875pt}%
\definecolor{currentstroke}{rgb}{0.000000,0.000000,0.000000}%
\pgfsetstrokecolor{currentstroke}%
\pgfsetdash{{1.000000pt}{3.000000pt}}{0.000000pt}%
\pgfpathmoveto{\pgfqpoint{3.480000in}{0.600000in}}%
\pgfpathlineto{\pgfqpoint{3.480000in}{5.400000in}}%
\pgfusepath{stroke}%
\end{pgfscope}%
\begin{pgfscope}%
\pgfsetbuttcap%
\pgfsetroundjoin%
\definecolor{currentfill}{rgb}{0.000000,0.000000,0.000000}%
\pgfsetfillcolor{currentfill}%
\pgfsetlinewidth{0.501875pt}%
\definecolor{currentstroke}{rgb}{0.000000,0.000000,0.000000}%
\pgfsetstrokecolor{currentstroke}%
\pgfsetdash{}{0pt}%
\pgfsys@defobject{currentmarker}{\pgfqpoint{0.000000in}{0.000000in}}{\pgfqpoint{0.000000in}{0.055556in}}{%
\pgfpathmoveto{\pgfqpoint{0.000000in}{0.000000in}}%
\pgfpathlineto{\pgfqpoint{0.000000in}{0.055556in}}%
\pgfusepath{stroke,fill}%
}%
\begin{pgfscope}%
\pgfsys@transformshift{3.480000in}{0.600000in}%
\pgfsys@useobject{currentmarker}{}%
\end{pgfscope}%
\end{pgfscope}%
\begin{pgfscope}%
\pgfsetbuttcap%
\pgfsetroundjoin%
\definecolor{currentfill}{rgb}{0.000000,0.000000,0.000000}%
\pgfsetfillcolor{currentfill}%
\pgfsetlinewidth{0.501875pt}%
\definecolor{currentstroke}{rgb}{0.000000,0.000000,0.000000}%
\pgfsetstrokecolor{currentstroke}%
\pgfsetdash{}{0pt}%
\pgfsys@defobject{currentmarker}{\pgfqpoint{0.000000in}{-0.055556in}}{\pgfqpoint{0.000000in}{0.000000in}}{%
\pgfpathmoveto{\pgfqpoint{0.000000in}{0.000000in}}%
\pgfpathlineto{\pgfqpoint{0.000000in}{-0.055556in}}%
\pgfusepath{stroke,fill}%
}%
\begin{pgfscope}%
\pgfsys@transformshift{3.480000in}{5.400000in}%
\pgfsys@useobject{currentmarker}{}%
\end{pgfscope}%
\end{pgfscope}%
\begin{pgfscope}%
\definecolor{textcolor}{rgb}{0.000000,0.000000,0.000000}%
\pgfsetstrokecolor{textcolor}%
\pgfsetfillcolor{textcolor}%
\pgftext[x=3.480000in,y=0.544444in,,top]{\color{textcolor}\rmfamily\fontsize{10.000000}{12.000000}\selectfont \(\displaystyle {40}\)}%
\end{pgfscope}%
\begin{pgfscope}%
\pgfpathrectangle{\pgfqpoint{1.000000in}{0.600000in}}{\pgfqpoint{6.200000in}{4.800000in}}%
\pgfusepath{clip}%
\pgfsetbuttcap%
\pgfsetroundjoin%
\pgfsetlinewidth{0.501875pt}%
\definecolor{currentstroke}{rgb}{0.000000,0.000000,0.000000}%
\pgfsetstrokecolor{currentstroke}%
\pgfsetdash{{1.000000pt}{3.000000pt}}{0.000000pt}%
\pgfpathmoveto{\pgfqpoint{4.720000in}{0.600000in}}%
\pgfpathlineto{\pgfqpoint{4.720000in}{5.400000in}}%
\pgfusepath{stroke}%
\end{pgfscope}%
\begin{pgfscope}%
\pgfsetbuttcap%
\pgfsetroundjoin%
\definecolor{currentfill}{rgb}{0.000000,0.000000,0.000000}%
\pgfsetfillcolor{currentfill}%
\pgfsetlinewidth{0.501875pt}%
\definecolor{currentstroke}{rgb}{0.000000,0.000000,0.000000}%
\pgfsetstrokecolor{currentstroke}%
\pgfsetdash{}{0pt}%
\pgfsys@defobject{currentmarker}{\pgfqpoint{0.000000in}{0.000000in}}{\pgfqpoint{0.000000in}{0.055556in}}{%
\pgfpathmoveto{\pgfqpoint{0.000000in}{0.000000in}}%
\pgfpathlineto{\pgfqpoint{0.000000in}{0.055556in}}%
\pgfusepath{stroke,fill}%
}%
\begin{pgfscope}%
\pgfsys@transformshift{4.720000in}{0.600000in}%
\pgfsys@useobject{currentmarker}{}%
\end{pgfscope}%
\end{pgfscope}%
\begin{pgfscope}%
\pgfsetbuttcap%
\pgfsetroundjoin%
\definecolor{currentfill}{rgb}{0.000000,0.000000,0.000000}%
\pgfsetfillcolor{currentfill}%
\pgfsetlinewidth{0.501875pt}%
\definecolor{currentstroke}{rgb}{0.000000,0.000000,0.000000}%
\pgfsetstrokecolor{currentstroke}%
\pgfsetdash{}{0pt}%
\pgfsys@defobject{currentmarker}{\pgfqpoint{0.000000in}{-0.055556in}}{\pgfqpoint{0.000000in}{0.000000in}}{%
\pgfpathmoveto{\pgfqpoint{0.000000in}{0.000000in}}%
\pgfpathlineto{\pgfqpoint{0.000000in}{-0.055556in}}%
\pgfusepath{stroke,fill}%
}%
\begin{pgfscope}%
\pgfsys@transformshift{4.720000in}{5.400000in}%
\pgfsys@useobject{currentmarker}{}%
\end{pgfscope}%
\end{pgfscope}%
\begin{pgfscope}%
\definecolor{textcolor}{rgb}{0.000000,0.000000,0.000000}%
\pgfsetstrokecolor{textcolor}%
\pgfsetfillcolor{textcolor}%
\pgftext[x=4.720000in,y=0.544444in,,top]{\color{textcolor}\rmfamily\fontsize{10.000000}{12.000000}\selectfont \(\displaystyle {60}\)}%
\end{pgfscope}%
\begin{pgfscope}%
\pgfpathrectangle{\pgfqpoint{1.000000in}{0.600000in}}{\pgfqpoint{6.200000in}{4.800000in}}%
\pgfusepath{clip}%
\pgfsetbuttcap%
\pgfsetroundjoin%
\pgfsetlinewidth{0.501875pt}%
\definecolor{currentstroke}{rgb}{0.000000,0.000000,0.000000}%
\pgfsetstrokecolor{currentstroke}%
\pgfsetdash{{1.000000pt}{3.000000pt}}{0.000000pt}%
\pgfpathmoveto{\pgfqpoint{5.960000in}{0.600000in}}%
\pgfpathlineto{\pgfqpoint{5.960000in}{5.400000in}}%
\pgfusepath{stroke}%
\end{pgfscope}%
\begin{pgfscope}%
\pgfsetbuttcap%
\pgfsetroundjoin%
\definecolor{currentfill}{rgb}{0.000000,0.000000,0.000000}%
\pgfsetfillcolor{currentfill}%
\pgfsetlinewidth{0.501875pt}%
\definecolor{currentstroke}{rgb}{0.000000,0.000000,0.000000}%
\pgfsetstrokecolor{currentstroke}%
\pgfsetdash{}{0pt}%
\pgfsys@defobject{currentmarker}{\pgfqpoint{0.000000in}{0.000000in}}{\pgfqpoint{0.000000in}{0.055556in}}{%
\pgfpathmoveto{\pgfqpoint{0.000000in}{0.000000in}}%
\pgfpathlineto{\pgfqpoint{0.000000in}{0.055556in}}%
\pgfusepath{stroke,fill}%
}%
\begin{pgfscope}%
\pgfsys@transformshift{5.960000in}{0.600000in}%
\pgfsys@useobject{currentmarker}{}%
\end{pgfscope}%
\end{pgfscope}%
\begin{pgfscope}%
\pgfsetbuttcap%
\pgfsetroundjoin%
\definecolor{currentfill}{rgb}{0.000000,0.000000,0.000000}%
\pgfsetfillcolor{currentfill}%
\pgfsetlinewidth{0.501875pt}%
\definecolor{currentstroke}{rgb}{0.000000,0.000000,0.000000}%
\pgfsetstrokecolor{currentstroke}%
\pgfsetdash{}{0pt}%
\pgfsys@defobject{currentmarker}{\pgfqpoint{0.000000in}{-0.055556in}}{\pgfqpoint{0.000000in}{0.000000in}}{%
\pgfpathmoveto{\pgfqpoint{0.000000in}{0.000000in}}%
\pgfpathlineto{\pgfqpoint{0.000000in}{-0.055556in}}%
\pgfusepath{stroke,fill}%
}%
\begin{pgfscope}%
\pgfsys@transformshift{5.960000in}{5.400000in}%
\pgfsys@useobject{currentmarker}{}%
\end{pgfscope}%
\end{pgfscope}%
\begin{pgfscope}%
\definecolor{textcolor}{rgb}{0.000000,0.000000,0.000000}%
\pgfsetstrokecolor{textcolor}%
\pgfsetfillcolor{textcolor}%
\pgftext[x=5.960000in,y=0.544444in,,top]{\color{textcolor}\rmfamily\fontsize{10.000000}{12.000000}\selectfont \(\displaystyle {80}\)}%
\end{pgfscope}%
\begin{pgfscope}%
\pgfpathrectangle{\pgfqpoint{1.000000in}{0.600000in}}{\pgfqpoint{6.200000in}{4.800000in}}%
\pgfusepath{clip}%
\pgfsetbuttcap%
\pgfsetroundjoin%
\pgfsetlinewidth{0.501875pt}%
\definecolor{currentstroke}{rgb}{0.000000,0.000000,0.000000}%
\pgfsetstrokecolor{currentstroke}%
\pgfsetdash{{1.000000pt}{3.000000pt}}{0.000000pt}%
\pgfpathmoveto{\pgfqpoint{7.200000in}{0.600000in}}%
\pgfpathlineto{\pgfqpoint{7.200000in}{5.400000in}}%
\pgfusepath{stroke}%
\end{pgfscope}%
\begin{pgfscope}%
\pgfsetbuttcap%
\pgfsetroundjoin%
\definecolor{currentfill}{rgb}{0.000000,0.000000,0.000000}%
\pgfsetfillcolor{currentfill}%
\pgfsetlinewidth{0.501875pt}%
\definecolor{currentstroke}{rgb}{0.000000,0.000000,0.000000}%
\pgfsetstrokecolor{currentstroke}%
\pgfsetdash{}{0pt}%
\pgfsys@defobject{currentmarker}{\pgfqpoint{0.000000in}{0.000000in}}{\pgfqpoint{0.000000in}{0.055556in}}{%
\pgfpathmoveto{\pgfqpoint{0.000000in}{0.000000in}}%
\pgfpathlineto{\pgfqpoint{0.000000in}{0.055556in}}%
\pgfusepath{stroke,fill}%
}%
\begin{pgfscope}%
\pgfsys@transformshift{7.200000in}{0.600000in}%
\pgfsys@useobject{currentmarker}{}%
\end{pgfscope}%
\end{pgfscope}%
\begin{pgfscope}%
\pgfsetbuttcap%
\pgfsetroundjoin%
\definecolor{currentfill}{rgb}{0.000000,0.000000,0.000000}%
\pgfsetfillcolor{currentfill}%
\pgfsetlinewidth{0.501875pt}%
\definecolor{currentstroke}{rgb}{0.000000,0.000000,0.000000}%
\pgfsetstrokecolor{currentstroke}%
\pgfsetdash{}{0pt}%
\pgfsys@defobject{currentmarker}{\pgfqpoint{0.000000in}{-0.055556in}}{\pgfqpoint{0.000000in}{0.000000in}}{%
\pgfpathmoveto{\pgfqpoint{0.000000in}{0.000000in}}%
\pgfpathlineto{\pgfqpoint{0.000000in}{-0.055556in}}%
\pgfusepath{stroke,fill}%
}%
\begin{pgfscope}%
\pgfsys@transformshift{7.200000in}{5.400000in}%
\pgfsys@useobject{currentmarker}{}%
\end{pgfscope}%
\end{pgfscope}%
\begin{pgfscope}%
\definecolor{textcolor}{rgb}{0.000000,0.000000,0.000000}%
\pgfsetstrokecolor{textcolor}%
\pgfsetfillcolor{textcolor}%
\pgftext[x=7.200000in,y=0.544444in,,top]{\color{textcolor}\rmfamily\fontsize{10.000000}{12.000000}\selectfont \(\displaystyle {100}\)}%
\end{pgfscope}%
\begin{pgfscope}%
\definecolor{textcolor}{rgb}{0.000000,0.000000,0.000000}%
\pgfsetstrokecolor{textcolor}%
\pgfsetfillcolor{textcolor}%
\pgftext[x=4.100000in,y=0.351543in,,top]{\color{textcolor}\rmfamily\fontsize{12.000000}{14.400000}\selectfont \(\displaystyle time\ (s)\)}%
\end{pgfscope}%
\begin{pgfscope}%
\pgfpathrectangle{\pgfqpoint{1.000000in}{0.600000in}}{\pgfqpoint{6.200000in}{4.800000in}}%
\pgfusepath{clip}%
\pgfsetbuttcap%
\pgfsetroundjoin%
\pgfsetlinewidth{0.501875pt}%
\definecolor{currentstroke}{rgb}{0.000000,0.000000,0.000000}%
\pgfsetstrokecolor{currentstroke}%
\pgfsetdash{{1.000000pt}{3.000000pt}}{0.000000pt}%
\pgfpathmoveto{\pgfqpoint{1.000000in}{0.600000in}}%
\pgfpathlineto{\pgfqpoint{7.200000in}{0.600000in}}%
\pgfusepath{stroke}%
\end{pgfscope}%
\begin{pgfscope}%
\pgfsetbuttcap%
\pgfsetroundjoin%
\definecolor{currentfill}{rgb}{0.000000,0.000000,0.000000}%
\pgfsetfillcolor{currentfill}%
\pgfsetlinewidth{0.501875pt}%
\definecolor{currentstroke}{rgb}{0.000000,0.000000,0.000000}%
\pgfsetstrokecolor{currentstroke}%
\pgfsetdash{}{0pt}%
\pgfsys@defobject{currentmarker}{\pgfqpoint{0.000000in}{0.000000in}}{\pgfqpoint{0.055556in}{0.000000in}}{%
\pgfpathmoveto{\pgfqpoint{0.000000in}{0.000000in}}%
\pgfpathlineto{\pgfqpoint{0.055556in}{0.000000in}}%
\pgfusepath{stroke,fill}%
}%
\begin{pgfscope}%
\pgfsys@transformshift{1.000000in}{0.600000in}%
\pgfsys@useobject{currentmarker}{}%
\end{pgfscope}%
\end{pgfscope}%
\begin{pgfscope}%
\pgfsetbuttcap%
\pgfsetroundjoin%
\definecolor{currentfill}{rgb}{0.000000,0.000000,0.000000}%
\pgfsetfillcolor{currentfill}%
\pgfsetlinewidth{0.501875pt}%
\definecolor{currentstroke}{rgb}{0.000000,0.000000,0.000000}%
\pgfsetstrokecolor{currentstroke}%
\pgfsetdash{}{0pt}%
\pgfsys@defobject{currentmarker}{\pgfqpoint{-0.055556in}{0.000000in}}{\pgfqpoint{-0.000000in}{0.000000in}}{%
\pgfpathmoveto{\pgfqpoint{-0.000000in}{0.000000in}}%
\pgfpathlineto{\pgfqpoint{-0.055556in}{0.000000in}}%
\pgfusepath{stroke,fill}%
}%
\begin{pgfscope}%
\pgfsys@transformshift{7.200000in}{0.600000in}%
\pgfsys@useobject{currentmarker}{}%
\end{pgfscope}%
\end{pgfscope}%
\begin{pgfscope}%
\definecolor{textcolor}{rgb}{0.000000,0.000000,0.000000}%
\pgfsetstrokecolor{textcolor}%
\pgfsetfillcolor{textcolor}%
\pgftext[x=0.944444in,y=0.600000in,right,]{\color{textcolor}\rmfamily\fontsize{10.000000}{12.000000}\selectfont \(\displaystyle {\ensuremath{-}1.0}\)}%
\end{pgfscope}%
\begin{pgfscope}%
\pgfpathrectangle{\pgfqpoint{1.000000in}{0.600000in}}{\pgfqpoint{6.200000in}{4.800000in}}%
\pgfusepath{clip}%
\pgfsetbuttcap%
\pgfsetroundjoin%
\pgfsetlinewidth{0.501875pt}%
\definecolor{currentstroke}{rgb}{0.000000,0.000000,0.000000}%
\pgfsetstrokecolor{currentstroke}%
\pgfsetdash{{1.000000pt}{3.000000pt}}{0.000000pt}%
\pgfpathmoveto{\pgfqpoint{1.000000in}{1.800000in}}%
\pgfpathlineto{\pgfqpoint{7.200000in}{1.800000in}}%
\pgfusepath{stroke}%
\end{pgfscope}%
\begin{pgfscope}%
\pgfsetbuttcap%
\pgfsetroundjoin%
\definecolor{currentfill}{rgb}{0.000000,0.000000,0.000000}%
\pgfsetfillcolor{currentfill}%
\pgfsetlinewidth{0.501875pt}%
\definecolor{currentstroke}{rgb}{0.000000,0.000000,0.000000}%
\pgfsetstrokecolor{currentstroke}%
\pgfsetdash{}{0pt}%
\pgfsys@defobject{currentmarker}{\pgfqpoint{0.000000in}{0.000000in}}{\pgfqpoint{0.055556in}{0.000000in}}{%
\pgfpathmoveto{\pgfqpoint{0.000000in}{0.000000in}}%
\pgfpathlineto{\pgfqpoint{0.055556in}{0.000000in}}%
\pgfusepath{stroke,fill}%
}%
\begin{pgfscope}%
\pgfsys@transformshift{1.000000in}{1.800000in}%
\pgfsys@useobject{currentmarker}{}%
\end{pgfscope}%
\end{pgfscope}%
\begin{pgfscope}%
\pgfsetbuttcap%
\pgfsetroundjoin%
\definecolor{currentfill}{rgb}{0.000000,0.000000,0.000000}%
\pgfsetfillcolor{currentfill}%
\pgfsetlinewidth{0.501875pt}%
\definecolor{currentstroke}{rgb}{0.000000,0.000000,0.000000}%
\pgfsetstrokecolor{currentstroke}%
\pgfsetdash{}{0pt}%
\pgfsys@defobject{currentmarker}{\pgfqpoint{-0.055556in}{0.000000in}}{\pgfqpoint{-0.000000in}{0.000000in}}{%
\pgfpathmoveto{\pgfqpoint{-0.000000in}{0.000000in}}%
\pgfpathlineto{\pgfqpoint{-0.055556in}{0.000000in}}%
\pgfusepath{stroke,fill}%
}%
\begin{pgfscope}%
\pgfsys@transformshift{7.200000in}{1.800000in}%
\pgfsys@useobject{currentmarker}{}%
\end{pgfscope}%
\end{pgfscope}%
\begin{pgfscope}%
\definecolor{textcolor}{rgb}{0.000000,0.000000,0.000000}%
\pgfsetstrokecolor{textcolor}%
\pgfsetfillcolor{textcolor}%
\pgftext[x=0.944444in,y=1.800000in,right,]{\color{textcolor}\rmfamily\fontsize{10.000000}{12.000000}\selectfont \(\displaystyle {\ensuremath{-}0.5}\)}%
\end{pgfscope}%
\begin{pgfscope}%
\pgfpathrectangle{\pgfqpoint{1.000000in}{0.600000in}}{\pgfqpoint{6.200000in}{4.800000in}}%
\pgfusepath{clip}%
\pgfsetbuttcap%
\pgfsetroundjoin%
\pgfsetlinewidth{0.501875pt}%
\definecolor{currentstroke}{rgb}{0.000000,0.000000,0.000000}%
\pgfsetstrokecolor{currentstroke}%
\pgfsetdash{{1.000000pt}{3.000000pt}}{0.000000pt}%
\pgfpathmoveto{\pgfqpoint{1.000000in}{3.000000in}}%
\pgfpathlineto{\pgfqpoint{7.200000in}{3.000000in}}%
\pgfusepath{stroke}%
\end{pgfscope}%
\begin{pgfscope}%
\pgfsetbuttcap%
\pgfsetroundjoin%
\definecolor{currentfill}{rgb}{0.000000,0.000000,0.000000}%
\pgfsetfillcolor{currentfill}%
\pgfsetlinewidth{0.501875pt}%
\definecolor{currentstroke}{rgb}{0.000000,0.000000,0.000000}%
\pgfsetstrokecolor{currentstroke}%
\pgfsetdash{}{0pt}%
\pgfsys@defobject{currentmarker}{\pgfqpoint{0.000000in}{0.000000in}}{\pgfqpoint{0.055556in}{0.000000in}}{%
\pgfpathmoveto{\pgfqpoint{0.000000in}{0.000000in}}%
\pgfpathlineto{\pgfqpoint{0.055556in}{0.000000in}}%
\pgfusepath{stroke,fill}%
}%
\begin{pgfscope}%
\pgfsys@transformshift{1.000000in}{3.000000in}%
\pgfsys@useobject{currentmarker}{}%
\end{pgfscope}%
\end{pgfscope}%
\begin{pgfscope}%
\pgfsetbuttcap%
\pgfsetroundjoin%
\definecolor{currentfill}{rgb}{0.000000,0.000000,0.000000}%
\pgfsetfillcolor{currentfill}%
\pgfsetlinewidth{0.501875pt}%
\definecolor{currentstroke}{rgb}{0.000000,0.000000,0.000000}%
\pgfsetstrokecolor{currentstroke}%
\pgfsetdash{}{0pt}%
\pgfsys@defobject{currentmarker}{\pgfqpoint{-0.055556in}{0.000000in}}{\pgfqpoint{-0.000000in}{0.000000in}}{%
\pgfpathmoveto{\pgfqpoint{-0.000000in}{0.000000in}}%
\pgfpathlineto{\pgfqpoint{-0.055556in}{0.000000in}}%
\pgfusepath{stroke,fill}%
}%
\begin{pgfscope}%
\pgfsys@transformshift{7.200000in}{3.000000in}%
\pgfsys@useobject{currentmarker}{}%
\end{pgfscope}%
\end{pgfscope}%
\begin{pgfscope}%
\definecolor{textcolor}{rgb}{0.000000,0.000000,0.000000}%
\pgfsetstrokecolor{textcolor}%
\pgfsetfillcolor{textcolor}%
\pgftext[x=0.944444in,y=3.000000in,right,]{\color{textcolor}\rmfamily\fontsize{10.000000}{12.000000}\selectfont \(\displaystyle {0.0}\)}%
\end{pgfscope}%
\begin{pgfscope}%
\pgfpathrectangle{\pgfqpoint{1.000000in}{0.600000in}}{\pgfqpoint{6.200000in}{4.800000in}}%
\pgfusepath{clip}%
\pgfsetbuttcap%
\pgfsetroundjoin%
\pgfsetlinewidth{0.501875pt}%
\definecolor{currentstroke}{rgb}{0.000000,0.000000,0.000000}%
\pgfsetstrokecolor{currentstroke}%
\pgfsetdash{{1.000000pt}{3.000000pt}}{0.000000pt}%
\pgfpathmoveto{\pgfqpoint{1.000000in}{4.200000in}}%
\pgfpathlineto{\pgfqpoint{7.200000in}{4.200000in}}%
\pgfusepath{stroke}%
\end{pgfscope}%
\begin{pgfscope}%
\pgfsetbuttcap%
\pgfsetroundjoin%
\definecolor{currentfill}{rgb}{0.000000,0.000000,0.000000}%
\pgfsetfillcolor{currentfill}%
\pgfsetlinewidth{0.501875pt}%
\definecolor{currentstroke}{rgb}{0.000000,0.000000,0.000000}%
\pgfsetstrokecolor{currentstroke}%
\pgfsetdash{}{0pt}%
\pgfsys@defobject{currentmarker}{\pgfqpoint{0.000000in}{0.000000in}}{\pgfqpoint{0.055556in}{0.000000in}}{%
\pgfpathmoveto{\pgfqpoint{0.000000in}{0.000000in}}%
\pgfpathlineto{\pgfqpoint{0.055556in}{0.000000in}}%
\pgfusepath{stroke,fill}%
}%
\begin{pgfscope}%
\pgfsys@transformshift{1.000000in}{4.200000in}%
\pgfsys@useobject{currentmarker}{}%
\end{pgfscope}%
\end{pgfscope}%
\begin{pgfscope}%
\pgfsetbuttcap%
\pgfsetroundjoin%
\definecolor{currentfill}{rgb}{0.000000,0.000000,0.000000}%
\pgfsetfillcolor{currentfill}%
\pgfsetlinewidth{0.501875pt}%
\definecolor{currentstroke}{rgb}{0.000000,0.000000,0.000000}%
\pgfsetstrokecolor{currentstroke}%
\pgfsetdash{}{0pt}%
\pgfsys@defobject{currentmarker}{\pgfqpoint{-0.055556in}{0.000000in}}{\pgfqpoint{-0.000000in}{0.000000in}}{%
\pgfpathmoveto{\pgfqpoint{-0.000000in}{0.000000in}}%
\pgfpathlineto{\pgfqpoint{-0.055556in}{0.000000in}}%
\pgfusepath{stroke,fill}%
}%
\begin{pgfscope}%
\pgfsys@transformshift{7.200000in}{4.200000in}%
\pgfsys@useobject{currentmarker}{}%
\end{pgfscope}%
\end{pgfscope}%
\begin{pgfscope}%
\definecolor{textcolor}{rgb}{0.000000,0.000000,0.000000}%
\pgfsetstrokecolor{textcolor}%
\pgfsetfillcolor{textcolor}%
\pgftext[x=0.944444in,y=4.200000in,right,]{\color{textcolor}\rmfamily\fontsize{10.000000}{12.000000}\selectfont \(\displaystyle {0.5}\)}%
\end{pgfscope}%
\begin{pgfscope}%
\pgfpathrectangle{\pgfqpoint{1.000000in}{0.600000in}}{\pgfqpoint{6.200000in}{4.800000in}}%
\pgfusepath{clip}%
\pgfsetbuttcap%
\pgfsetroundjoin%
\pgfsetlinewidth{0.501875pt}%
\definecolor{currentstroke}{rgb}{0.000000,0.000000,0.000000}%
\pgfsetstrokecolor{currentstroke}%
\pgfsetdash{{1.000000pt}{3.000000pt}}{0.000000pt}%
\pgfpathmoveto{\pgfqpoint{1.000000in}{5.400000in}}%
\pgfpathlineto{\pgfqpoint{7.200000in}{5.400000in}}%
\pgfusepath{stroke}%
\end{pgfscope}%
\begin{pgfscope}%
\pgfsetbuttcap%
\pgfsetroundjoin%
\definecolor{currentfill}{rgb}{0.000000,0.000000,0.000000}%
\pgfsetfillcolor{currentfill}%
\pgfsetlinewidth{0.501875pt}%
\definecolor{currentstroke}{rgb}{0.000000,0.000000,0.000000}%
\pgfsetstrokecolor{currentstroke}%
\pgfsetdash{}{0pt}%
\pgfsys@defobject{currentmarker}{\pgfqpoint{0.000000in}{0.000000in}}{\pgfqpoint{0.055556in}{0.000000in}}{%
\pgfpathmoveto{\pgfqpoint{0.000000in}{0.000000in}}%
\pgfpathlineto{\pgfqpoint{0.055556in}{0.000000in}}%
\pgfusepath{stroke,fill}%
}%
\begin{pgfscope}%
\pgfsys@transformshift{1.000000in}{5.400000in}%
\pgfsys@useobject{currentmarker}{}%
\end{pgfscope}%
\end{pgfscope}%
\begin{pgfscope}%
\pgfsetbuttcap%
\pgfsetroundjoin%
\definecolor{currentfill}{rgb}{0.000000,0.000000,0.000000}%
\pgfsetfillcolor{currentfill}%
\pgfsetlinewidth{0.501875pt}%
\definecolor{currentstroke}{rgb}{0.000000,0.000000,0.000000}%
\pgfsetstrokecolor{currentstroke}%
\pgfsetdash{}{0pt}%
\pgfsys@defobject{currentmarker}{\pgfqpoint{-0.055556in}{0.000000in}}{\pgfqpoint{-0.000000in}{0.000000in}}{%
\pgfpathmoveto{\pgfqpoint{-0.000000in}{0.000000in}}%
\pgfpathlineto{\pgfqpoint{-0.055556in}{0.000000in}}%
\pgfusepath{stroke,fill}%
}%
\begin{pgfscope}%
\pgfsys@transformshift{7.200000in}{5.400000in}%
\pgfsys@useobject{currentmarker}{}%
\end{pgfscope}%
\end{pgfscope}%
\begin{pgfscope}%
\definecolor{textcolor}{rgb}{0.000000,0.000000,0.000000}%
\pgfsetstrokecolor{textcolor}%
\pgfsetfillcolor{textcolor}%
\pgftext[x=0.944444in,y=5.400000in,right,]{\color{textcolor}\rmfamily\fontsize{10.000000}{12.000000}\selectfont \(\displaystyle {1.0}\)}%
\end{pgfscope}%
\begin{pgfscope}%
\definecolor{textcolor}{rgb}{0.000000,0.000000,0.000000}%
\pgfsetstrokecolor{textcolor}%
\pgfsetfillcolor{textcolor}%
\pgftext[x=0.589505in,y=3.000000in,,bottom,rotate=90.000000]{\color{textcolor}\rmfamily\fontsize{12.000000}{14.400000}\selectfont \(\displaystyle \theta\ (rad)\)}%
\end{pgfscope}%
\begin{pgfscope}%
\definecolor{textcolor}{rgb}{0.000000,0.000000,0.000000}%
\pgfsetstrokecolor{textcolor}%
\pgfsetfillcolor{textcolor}%
\pgftext[x=4.100000in,y=5.469444in,,base]{\color{textcolor}\rmfamily\fontsize{12.000000}{14.400000}\selectfont \(\displaystyle Simple\ pendulum\ using\ Euler's\ methods\ (time\ step = 0.01\ (s))\)}%
\end{pgfscope}%
\begin{pgfscope}%
\pgfsetbuttcap%
\pgfsetmiterjoin%
\definecolor{currentfill}{rgb}{1.000000,1.000000,1.000000}%
\pgfsetfillcolor{currentfill}%
\pgfsetlinewidth{1.003750pt}%
\definecolor{currentstroke}{rgb}{0.000000,0.000000,0.000000}%
\pgfsetstrokecolor{currentstroke}%
\pgfsetdash{}{0pt}%
\pgfpathmoveto{\pgfqpoint{1.083333in}{4.569445in}}%
\pgfpathlineto{\pgfqpoint{3.093110in}{4.569445in}}%
\pgfpathlineto{\pgfqpoint{3.093110in}{5.316667in}}%
\pgfpathlineto{\pgfqpoint{1.083333in}{5.316667in}}%
\pgfpathlineto{\pgfqpoint{1.083333in}{4.569445in}}%
\pgfpathclose%
\pgfusepath{stroke,fill}%
\end{pgfscope}%
\begin{pgfscope}%
\pgfsetrectcap%
\pgfsetroundjoin%
\pgfsetlinewidth{1.003750pt}%
\definecolor{currentstroke}{rgb}{1.000000,0.000000,0.000000}%
\pgfsetstrokecolor{currentstroke}%
\pgfsetdash{}{0pt}%
\pgfpathmoveto{\pgfqpoint{1.200000in}{5.191667in}}%
\pgfpathlineto{\pgfqpoint{1.433333in}{5.191667in}}%
\pgfusepath{stroke}%
\end{pgfscope}%
\begin{pgfscope}%
\definecolor{textcolor}{rgb}{0.000000,0.000000,0.000000}%
\pgfsetstrokecolor{textcolor}%
\pgfsetfillcolor{textcolor}%
\pgftext[x=1.616667in,y=5.133333in,left,base]{\color{textcolor}\rmfamily\fontsize{12.000000}{14.400000}\selectfont \(\displaystyle euler\ explicit\)}%
\end{pgfscope}%
\begin{pgfscope}%
\pgfsetrectcap%
\pgfsetroundjoin%
\pgfsetlinewidth{1.003750pt}%
\definecolor{currentstroke}{rgb}{0.000000,0.000000,1.000000}%
\pgfsetstrokecolor{currentstroke}%
\pgfsetdash{}{0pt}%
\pgfpathmoveto{\pgfqpoint{1.200000in}{4.959260in}}%
\pgfpathlineto{\pgfqpoint{1.433333in}{4.959260in}}%
\pgfusepath{stroke}%
\end{pgfscope}%
\begin{pgfscope}%
\definecolor{textcolor}{rgb}{0.000000,0.000000,0.000000}%
\pgfsetstrokecolor{textcolor}%
\pgfsetfillcolor{textcolor}%
\pgftext[x=1.616667in,y=4.900926in,left,base]{\color{textcolor}\rmfamily\fontsize{12.000000}{14.400000}\selectfont \(\displaystyle euler\ implicit\)}%
\end{pgfscope}%
\begin{pgfscope}%
\pgfsetrectcap%
\pgfsetroundjoin%
\pgfsetlinewidth{1.003750pt}%
\definecolor{currentstroke}{rgb}{0.000000,0.000000,0.000000}%
\pgfsetstrokecolor{currentstroke}%
\pgfsetdash{}{0pt}%
\pgfpathmoveto{\pgfqpoint{1.200000in}{4.726852in}}%
\pgfpathlineto{\pgfqpoint{1.433333in}{4.726852in}}%
\pgfusepath{stroke}%
\end{pgfscope}%
\begin{pgfscope}%
\definecolor{textcolor}{rgb}{0.000000,0.000000,0.000000}%
\pgfsetstrokecolor{textcolor}%
\pgfsetfillcolor{textcolor}%
\pgftext[x=1.616667in,y=4.668519in,left,base]{\color{textcolor}\rmfamily\fontsize{12.000000}{14.400000}\selectfont \(\displaystyle trapezoidal\ scheme\)}%
\end{pgfscope}%
\end{pgfpicture}%
\makeatother%
\endgroup%
}
    \end{figure}

    \begin{figure}[ht!]
    \centering
    \resizebox{0.9\linewidth}{!}{%% Creator: Matplotlib, PGF backend
%%
%% To include the figure in your LaTeX document, write
%%   \input{<filename>.pgf}
%%
%% Make sure the required packages are loaded in your preamble
%%   \usepackage{pgf}
%%
%% Also ensure that all the required font packages are loaded; for instance,
%% the lmodern package is sometimes necessary when using math font.
%%   \usepackage{lmodern}
%%
%% Figures using additional raster images can only be included by \input if
%% they are in the same directory as the main LaTeX file. For loading figures
%% from other directories you can use the `import` package
%%   \usepackage{import}
%%
%% and then include the figures with
%%   \import{<path to file>}{<filename>.pgf}
%%
%% Matplotlib used the following preamble
%%
\begingroup%
\makeatletter%
\begin{pgfpicture}%
\pgfpathrectangle{\pgfpointorigin}{\pgfqpoint{8.000000in}{6.000000in}}%
\pgfusepath{use as bounding box, clip}%
\begin{pgfscope}%
\pgfsetbuttcap%
\pgfsetmiterjoin%
\definecolor{currentfill}{rgb}{1.000000,1.000000,1.000000}%
\pgfsetfillcolor{currentfill}%
\pgfsetlinewidth{0.000000pt}%
\definecolor{currentstroke}{rgb}{1.000000,1.000000,1.000000}%
\pgfsetstrokecolor{currentstroke}%
\pgfsetdash{}{0pt}%
\pgfpathmoveto{\pgfqpoint{0.000000in}{0.000000in}}%
\pgfpathlineto{\pgfqpoint{8.000000in}{0.000000in}}%
\pgfpathlineto{\pgfqpoint{8.000000in}{6.000000in}}%
\pgfpathlineto{\pgfqpoint{0.000000in}{6.000000in}}%
\pgfpathlineto{\pgfqpoint{0.000000in}{0.000000in}}%
\pgfpathclose%
\pgfusepath{fill}%
\end{pgfscope}%
\begin{pgfscope}%
\pgfsetbuttcap%
\pgfsetmiterjoin%
\definecolor{currentfill}{rgb}{1.000000,1.000000,1.000000}%
\pgfsetfillcolor{currentfill}%
\pgfsetlinewidth{0.000000pt}%
\definecolor{currentstroke}{rgb}{0.000000,0.000000,0.000000}%
\pgfsetstrokecolor{currentstroke}%
\pgfsetstrokeopacity{0.000000}%
\pgfsetdash{}{0pt}%
\pgfpathmoveto{\pgfqpoint{1.000000in}{0.600000in}}%
\pgfpathlineto{\pgfqpoint{7.200000in}{0.600000in}}%
\pgfpathlineto{\pgfqpoint{7.200000in}{5.400000in}}%
\pgfpathlineto{\pgfqpoint{1.000000in}{5.400000in}}%
\pgfpathlineto{\pgfqpoint{1.000000in}{0.600000in}}%
\pgfpathclose%
\pgfusepath{fill}%
\end{pgfscope}%
\begin{pgfscope}%
\pgfpathrectangle{\pgfqpoint{1.000000in}{0.600000in}}{\pgfqpoint{6.200000in}{4.800000in}}%
\pgfusepath{clip}%
\pgfsetrectcap%
\pgfsetroundjoin%
\pgfsetlinewidth{1.003750pt}%
\definecolor{currentstroke}{rgb}{1.000000,0.000000,0.000000}%
\pgfsetstrokecolor{currentstroke}%
\pgfsetdash{}{0pt}%
\pgfpathmoveto{\pgfqpoint{1.000000in}{3.837758in}}%
\pgfpathlineto{\pgfqpoint{1.003720in}{3.836798in}}%
\pgfpathlineto{\pgfqpoint{1.007440in}{3.832960in}}%
\pgfpathlineto{\pgfqpoint{1.012400in}{3.823381in}}%
\pgfpathlineto{\pgfqpoint{1.017360in}{3.808749in}}%
\pgfpathlineto{\pgfqpoint{1.023560in}{3.783468in}}%
\pgfpathlineto{\pgfqpoint{1.031000in}{3.743162in}}%
\pgfpathlineto{\pgfqpoint{1.039680in}{3.683030in}}%
\pgfpathlineto{\pgfqpoint{1.049600in}{3.598300in}}%
\pgfpathlineto{\pgfqpoint{1.062000in}{3.471279in}}%
\pgfpathlineto{\pgfqpoint{1.076880in}{3.294213in}}%
\pgfpathlineto{\pgfqpoint{1.100440in}{2.982148in}}%
\pgfpathlineto{\pgfqpoint{1.128960in}{2.608865in}}%
\pgfpathlineto{\pgfqpoint{1.143840in}{2.440438in}}%
\pgfpathlineto{\pgfqpoint{1.156240in}{2.323202in}}%
\pgfpathlineto{\pgfqpoint{1.166160in}{2.247808in}}%
\pgfpathlineto{\pgfqpoint{1.174840in}{2.196821in}}%
\pgfpathlineto{\pgfqpoint{1.182280in}{2.164994in}}%
\pgfpathlineto{\pgfqpoint{1.188480in}{2.147180in}}%
\pgfpathlineto{\pgfqpoint{1.193440in}{2.138755in}}%
\pgfpathlineto{\pgfqpoint{1.197160in}{2.135871in}}%
\pgfpathlineto{\pgfqpoint{1.200880in}{2.135947in}}%
\pgfpathlineto{\pgfqpoint{1.204600in}{2.138985in}}%
\pgfpathlineto{\pgfqpoint{1.208320in}{2.144979in}}%
\pgfpathlineto{\pgfqpoint{1.213280in}{2.157540in}}%
\pgfpathlineto{\pgfqpoint{1.219480in}{2.180493in}}%
\pgfpathlineto{\pgfqpoint{1.226920in}{2.218419in}}%
\pgfpathlineto{\pgfqpoint{1.235600in}{2.276390in}}%
\pgfpathlineto{\pgfqpoint{1.245520in}{2.359525in}}%
\pgfpathlineto{\pgfqpoint{1.256680in}{2.472296in}}%
\pgfpathlineto{\pgfqpoint{1.270320in}{2.633028in}}%
\pgfpathlineto{\pgfqpoint{1.290160in}{2.897142in}}%
\pgfpathlineto{\pgfqpoint{1.329840in}{3.433810in}}%
\pgfpathlineto{\pgfqpoint{1.344720in}{3.602865in}}%
\pgfpathlineto{\pgfqpoint{1.357120in}{3.718839in}}%
\pgfpathlineto{\pgfqpoint{1.367040in}{3.792114in}}%
\pgfpathlineto{\pgfqpoint{1.375720in}{3.840496in}}%
\pgfpathlineto{\pgfqpoint{1.383160in}{3.869573in}}%
\pgfpathlineto{\pgfqpoint{1.389360in}{3.884762in}}%
\pgfpathlineto{\pgfqpoint{1.394320in}{3.890885in}}%
\pgfpathlineto{\pgfqpoint{1.398040in}{3.891933in}}%
\pgfpathlineto{\pgfqpoint{1.401760in}{3.889934in}}%
\pgfpathlineto{\pgfqpoint{1.405480in}{3.884892in}}%
\pgfpathlineto{\pgfqpoint{1.410440in}{3.873457in}}%
\pgfpathlineto{\pgfqpoint{1.416640in}{3.851674in}}%
\pgfpathlineto{\pgfqpoint{1.424080in}{3.814786in}}%
\pgfpathlineto{\pgfqpoint{1.432760in}{3.757498in}}%
\pgfpathlineto{\pgfqpoint{1.442680in}{3.674420in}}%
\pgfpathlineto{\pgfqpoint{1.453840in}{3.560763in}}%
\pgfpathlineto{\pgfqpoint{1.467480in}{3.397581in}}%
\pgfpathlineto{\pgfqpoint{1.486080in}{3.144980in}}%
\pgfpathlineto{\pgfqpoint{1.530720in}{2.524884in}}%
\pgfpathlineto{\pgfqpoint{1.545600in}{2.355092in}}%
\pgfpathlineto{\pgfqpoint{1.556760in}{2.250470in}}%
\pgfpathlineto{\pgfqpoint{1.566680in}{2.176734in}}%
\pgfpathlineto{\pgfqpoint{1.575360in}{2.128522in}}%
\pgfpathlineto{\pgfqpoint{1.582800in}{2.100009in}}%
\pgfpathlineto{\pgfqpoint{1.587760in}{2.087775in}}%
\pgfpathlineto{\pgfqpoint{1.592720in}{2.081053in}}%
\pgfpathlineto{\pgfqpoint{1.596440in}{2.079656in}}%
\pgfpathlineto{\pgfqpoint{1.600160in}{2.081390in}}%
\pgfpathlineto{\pgfqpoint{1.603880in}{2.086255in}}%
\pgfpathlineto{\pgfqpoint{1.608840in}{2.097590in}}%
\pgfpathlineto{\pgfqpoint{1.615040in}{2.119470in}}%
\pgfpathlineto{\pgfqpoint{1.621240in}{2.149755in}}%
\pgfpathlineto{\pgfqpoint{1.628680in}{2.196840in}}%
\pgfpathlineto{\pgfqpoint{1.637360in}{2.265858in}}%
\pgfpathlineto{\pgfqpoint{1.648520in}{2.375054in}}%
\pgfpathlineto{\pgfqpoint{1.660920in}{2.519764in}}%
\pgfpathlineto{\pgfqpoint{1.677040in}{2.736025in}}%
\pgfpathlineto{\pgfqpoint{1.709280in}{3.209291in}}%
\pgfpathlineto{\pgfqpoint{1.730360in}{3.499116in}}%
\pgfpathlineto{\pgfqpoint{1.744000in}{3.659441in}}%
\pgfpathlineto{\pgfqpoint{1.755160in}{3.768367in}}%
\pgfpathlineto{\pgfqpoint{1.765080in}{3.845488in}}%
\pgfpathlineto{\pgfqpoint{1.773760in}{3.896265in}}%
\pgfpathlineto{\pgfqpoint{1.781200in}{3.926655in}}%
\pgfpathlineto{\pgfqpoint{1.787400in}{3.942406in}}%
\pgfpathlineto{\pgfqpoint{1.792360in}{3.948632in}}%
\pgfpathlineto{\pgfqpoint{1.796080in}{3.949552in}}%
\pgfpathlineto{\pgfqpoint{1.799800in}{3.947252in}}%
\pgfpathlineto{\pgfqpoint{1.803520in}{3.941734in}}%
\pgfpathlineto{\pgfqpoint{1.808480in}{3.929398in}}%
\pgfpathlineto{\pgfqpoint{1.814680in}{3.906061in}}%
\pgfpathlineto{\pgfqpoint{1.822120in}{3.866695in}}%
\pgfpathlineto{\pgfqpoint{1.830800in}{3.805697in}}%
\pgfpathlineto{\pgfqpoint{1.840720in}{3.717352in}}%
\pgfpathlineto{\pgfqpoint{1.851880in}{3.596572in}}%
\pgfpathlineto{\pgfqpoint{1.865520in}{3.423207in}}%
\pgfpathlineto{\pgfqpoint{1.884120in}{3.154826in}}%
\pgfpathlineto{\pgfqpoint{1.928760in}{2.495727in}}%
\pgfpathlineto{\pgfqpoint{1.943640in}{2.315109in}}%
\pgfpathlineto{\pgfqpoint{1.954800in}{2.203697in}}%
\pgfpathlineto{\pgfqpoint{1.964720in}{2.125031in}}%
\pgfpathlineto{\pgfqpoint{1.973400in}{2.073431in}}%
\pgfpathlineto{\pgfqpoint{1.980840in}{2.042735in}}%
\pgfpathlineto{\pgfqpoint{1.987040in}{2.027007in}}%
\pgfpathlineto{\pgfqpoint{1.990760in}{2.021940in}}%
\pgfpathlineto{\pgfqpoint{1.994480in}{2.020173in}}%
\pgfpathlineto{\pgfqpoint{1.998200in}{2.021717in}}%
\pgfpathlineto{\pgfqpoint{2.001920in}{2.026568in}}%
\pgfpathlineto{\pgfqpoint{2.006880in}{2.038163in}}%
\pgfpathlineto{\pgfqpoint{2.011840in}{2.055565in}}%
\pgfpathlineto{\pgfqpoint{2.018040in}{2.085352in}}%
\pgfpathlineto{\pgfqpoint{2.025480in}{2.132564in}}%
\pgfpathlineto{\pgfqpoint{2.034160in}{2.202740in}}%
\pgfpathlineto{\pgfqpoint{2.044080in}{2.301415in}}%
\pgfpathlineto{\pgfqpoint{2.056480in}{2.449191in}}%
\pgfpathlineto{\pgfqpoint{2.071360in}{2.655158in}}%
\pgfpathlineto{\pgfqpoint{2.094920in}{3.018370in}}%
\pgfpathlineto{\pgfqpoint{2.123440in}{3.453298in}}%
\pgfpathlineto{\pgfqpoint{2.138320in}{3.649836in}}%
\pgfpathlineto{\pgfqpoint{2.150720in}{3.786957in}}%
\pgfpathlineto{\pgfqpoint{2.160640in}{3.875483in}}%
\pgfpathlineto{\pgfqpoint{2.169320in}{3.935732in}}%
\pgfpathlineto{\pgfqpoint{2.176760in}{3.973744in}}%
\pgfpathlineto{\pgfqpoint{2.182960in}{3.995433in}}%
\pgfpathlineto{\pgfqpoint{2.187920in}{4.006103in}}%
\pgfpathlineto{\pgfqpoint{2.191640in}{4.010164in}}%
\pgfpathlineto{\pgfqpoint{2.195360in}{4.010829in}}%
\pgfpathlineto{\pgfqpoint{2.199080in}{4.008092in}}%
\pgfpathlineto{\pgfqpoint{2.202800in}{4.001958in}}%
\pgfpathlineto{\pgfqpoint{2.207760in}{3.988523in}}%
\pgfpathlineto{\pgfqpoint{2.213960in}{3.963373in}}%
\pgfpathlineto{\pgfqpoint{2.221400in}{3.921198in}}%
\pgfpathlineto{\pgfqpoint{2.230080in}{3.856078in}}%
\pgfpathlineto{\pgfqpoint{2.240000in}{3.761969in}}%
\pgfpathlineto{\pgfqpoint{2.251160in}{3.633474in}}%
\pgfpathlineto{\pgfqpoint{2.264800in}{3.449172in}}%
\pgfpathlineto{\pgfqpoint{2.283400in}{3.163971in}}%
\pgfpathlineto{\pgfqpoint{2.328040in}{2.463762in}}%
\pgfpathlineto{\pgfqpoint{2.342920in}{2.271865in}}%
\pgfpathlineto{\pgfqpoint{2.354080in}{2.153416in}}%
\pgfpathlineto{\pgfqpoint{2.364000in}{2.069668in}}%
\pgfpathlineto{\pgfqpoint{2.372680in}{2.014592in}}%
\pgfpathlineto{\pgfqpoint{2.380120in}{1.981670in}}%
\pgfpathlineto{\pgfqpoint{2.386320in}{1.964632in}}%
\pgfpathlineto{\pgfqpoint{2.390040in}{1.959020in}}%
\pgfpathlineto{\pgfqpoint{2.393760in}{1.956892in}}%
\pgfpathlineto{\pgfqpoint{2.396240in}{1.957414in}}%
\pgfpathlineto{\pgfqpoint{2.399960in}{1.961108in}}%
\pgfpathlineto{\pgfqpoint{2.403680in}{1.968288in}}%
\pgfpathlineto{\pgfqpoint{2.408640in}{1.983254in}}%
\pgfpathlineto{\pgfqpoint{2.414840in}{2.010526in}}%
\pgfpathlineto{\pgfqpoint{2.422280in}{2.055531in}}%
\pgfpathlineto{\pgfqpoint{2.430960in}{2.124310in}}%
\pgfpathlineto{\pgfqpoint{2.440880in}{2.223010in}}%
\pgfpathlineto{\pgfqpoint{2.452040in}{2.357084in}}%
\pgfpathlineto{\pgfqpoint{2.465680in}{2.548595in}}%
\pgfpathlineto{\pgfqpoint{2.484280in}{2.843770in}}%
\pgfpathlineto{\pgfqpoint{2.527680in}{3.545917in}}%
\pgfpathlineto{\pgfqpoint{2.542560in}{3.744855in}}%
\pgfpathlineto{\pgfqpoint{2.553720in}{3.868054in}}%
\pgfpathlineto{\pgfqpoint{2.563640in}{3.955537in}}%
\pgfpathlineto{\pgfqpoint{2.572320in}{4.013449in}}%
\pgfpathlineto{\pgfqpoint{2.579760in}{4.048453in}}%
\pgfpathlineto{\pgfqpoint{2.585960in}{4.066964in}}%
\pgfpathlineto{\pgfqpoint{2.590920in}{4.074674in}}%
\pgfpathlineto{\pgfqpoint{2.594640in}{4.076282in}}%
\pgfpathlineto{\pgfqpoint{2.597120in}{4.075360in}}%
\pgfpathlineto{\pgfqpoint{2.600840in}{4.070989in}}%
\pgfpathlineto{\pgfqpoint{2.604560in}{4.063039in}}%
\pgfpathlineto{\pgfqpoint{2.609520in}{4.046907in}}%
\pgfpathlineto{\pgfqpoint{2.615720in}{4.017961in}}%
\pgfpathlineto{\pgfqpoint{2.623160in}{3.970640in}}%
\pgfpathlineto{\pgfqpoint{2.631840in}{3.898762in}}%
\pgfpathlineto{\pgfqpoint{2.641760in}{3.796046in}}%
\pgfpathlineto{\pgfqpoint{2.652920in}{3.656933in}}%
\pgfpathlineto{\pgfqpoint{2.666560in}{3.458695in}}%
\pgfpathlineto{\pgfqpoint{2.685160in}{3.153824in}}%
\pgfpathlineto{\pgfqpoint{2.728560in}{2.431111in}}%
\pgfpathlineto{\pgfqpoint{2.743440in}{2.227085in}}%
\pgfpathlineto{\pgfqpoint{2.754600in}{2.100992in}}%
\pgfpathlineto{\pgfqpoint{2.764520in}{2.011648in}}%
\pgfpathlineto{\pgfqpoint{2.773200in}{1.952671in}}%
\pgfpathlineto{\pgfqpoint{2.780640in}{1.917177in}}%
\pgfpathlineto{\pgfqpoint{2.786840in}{1.898555in}}%
\pgfpathlineto{\pgfqpoint{2.791800in}{1.890950in}}%
\pgfpathlineto{\pgfqpoint{2.795520in}{1.889533in}}%
\pgfpathlineto{\pgfqpoint{2.798000in}{1.890635in}}%
\pgfpathlineto{\pgfqpoint{2.801720in}{1.895357in}}%
\pgfpathlineto{\pgfqpoint{2.805440in}{1.903751in}}%
\pgfpathlineto{\pgfqpoint{2.810400in}{1.920621in}}%
\pgfpathlineto{\pgfqpoint{2.816600in}{1.950721in}}%
\pgfpathlineto{\pgfqpoint{2.824040in}{1.999757in}}%
\pgfpathlineto{\pgfqpoint{2.832720in}{2.074077in}}%
\pgfpathlineto{\pgfqpoint{2.842640in}{2.180135in}}%
\pgfpathlineto{\pgfqpoint{2.853800in}{2.323650in}}%
\pgfpathlineto{\pgfqpoint{2.867440in}{2.528050in}}%
\pgfpathlineto{\pgfqpoint{2.886040in}{2.842291in}}%
\pgfpathlineto{\pgfqpoint{2.928200in}{3.567810in}}%
\pgfpathlineto{\pgfqpoint{2.943080in}{3.781104in}}%
\pgfpathlineto{\pgfqpoint{2.955480in}{3.927041in}}%
\pgfpathlineto{\pgfqpoint{2.965400in}{4.019162in}}%
\pgfpathlineto{\pgfqpoint{2.974080in}{4.080049in}}%
\pgfpathlineto{\pgfqpoint{2.981520in}{4.116781in}}%
\pgfpathlineto{\pgfqpoint{2.987720in}{4.136147in}}%
\pgfpathlineto{\pgfqpoint{2.992680in}{4.144157in}}%
\pgfpathlineto{\pgfqpoint{2.996400in}{4.145765in}}%
\pgfpathlineto{\pgfqpoint{2.998880in}{4.144737in}}%
\pgfpathlineto{\pgfqpoint{3.002600in}{4.140046in}}%
\pgfpathlineto{\pgfqpoint{3.006320in}{4.131584in}}%
\pgfpathlineto{\pgfqpoint{3.011280in}{4.114473in}}%
\pgfpathlineto{\pgfqpoint{3.017480in}{4.083827in}}%
\pgfpathlineto{\pgfqpoint{3.024920in}{4.033778in}}%
\pgfpathlineto{\pgfqpoint{3.033600in}{3.957785in}}%
\pgfpathlineto{\pgfqpoint{3.043520in}{3.849177in}}%
\pgfpathlineto{\pgfqpoint{3.054680in}{3.702016in}}%
\pgfpathlineto{\pgfqpoint{3.068320in}{3.492133in}}%
\pgfpathlineto{\pgfqpoint{3.086920in}{3.168939in}}%
\pgfpathlineto{\pgfqpoint{3.130320in}{2.400850in}}%
\pgfpathlineto{\pgfqpoint{3.145200in}{2.183304in}}%
\pgfpathlineto{\pgfqpoint{3.156360in}{2.048462in}}%
\pgfpathlineto{\pgfqpoint{3.166280in}{1.952514in}}%
\pgfpathlineto{\pgfqpoint{3.174960in}{1.888747in}}%
\pgfpathlineto{\pgfqpoint{3.182400in}{1.849917in}}%
\pgfpathlineto{\pgfqpoint{3.188600in}{1.829079in}}%
\pgfpathlineto{\pgfqpoint{3.193560in}{1.820079in}}%
\pgfpathlineto{\pgfqpoint{3.197280in}{1.817839in}}%
\pgfpathlineto{\pgfqpoint{3.199760in}{1.818499in}}%
\pgfpathlineto{\pgfqpoint{3.203480in}{1.822722in}}%
\pgfpathlineto{\pgfqpoint{3.207200in}{1.830815in}}%
\pgfpathlineto{\pgfqpoint{3.212160in}{1.847592in}}%
\pgfpathlineto{\pgfqpoint{3.218360in}{1.878079in}}%
\pgfpathlineto{\pgfqpoint{3.225800in}{1.928325in}}%
\pgfpathlineto{\pgfqpoint{3.234480in}{2.005091in}}%
\pgfpathlineto{\pgfqpoint{3.244400in}{2.115311in}}%
\pgfpathlineto{\pgfqpoint{3.255560in}{2.265217in}}%
\pgfpathlineto{\pgfqpoint{3.269200in}{2.479749in}}%
\pgfpathlineto{\pgfqpoint{3.287800in}{2.811339in}}%
\pgfpathlineto{\pgfqpoint{3.332440in}{3.624737in}}%
\pgfpathlineto{\pgfqpoint{3.347320in}{3.847708in}}%
\pgfpathlineto{\pgfqpoint{3.358480in}{3.985620in}}%
\pgfpathlineto{\pgfqpoint{3.368400in}{4.083539in}}%
\pgfpathlineto{\pgfqpoint{3.377080in}{4.148438in}}%
\pgfpathlineto{\pgfqpoint{3.384520in}{4.187795in}}%
\pgfpathlineto{\pgfqpoint{3.390720in}{4.208763in}}%
\pgfpathlineto{\pgfqpoint{3.395680in}{4.217667in}}%
\pgfpathlineto{\pgfqpoint{3.399400in}{4.219718in}}%
\pgfpathlineto{\pgfqpoint{3.401880in}{4.218876in}}%
\pgfpathlineto{\pgfqpoint{3.405600in}{4.214300in}}%
\pgfpathlineto{\pgfqpoint{3.409320in}{4.205756in}}%
\pgfpathlineto{\pgfqpoint{3.414280in}{4.188225in}}%
\pgfpathlineto{\pgfqpoint{3.420480in}{4.156556in}}%
\pgfpathlineto{\pgfqpoint{3.427920in}{4.104545in}}%
\pgfpathlineto{\pgfqpoint{3.436600in}{4.025252in}}%
\pgfpathlineto{\pgfqpoint{3.446520in}{3.911556in}}%
\pgfpathlineto{\pgfqpoint{3.457680in}{3.757040in}}%
\pgfpathlineto{\pgfqpoint{3.471320in}{3.535998in}}%
\pgfpathlineto{\pgfqpoint{3.489920in}{3.194397in}}%
\pgfpathlineto{\pgfqpoint{3.534560in}{2.356431in}}%
\pgfpathlineto{\pgfqpoint{3.549440in}{2.126657in}}%
\pgfpathlineto{\pgfqpoint{3.560600in}{1.984446in}}%
\pgfpathlineto{\pgfqpoint{3.570520in}{1.883358in}}%
\pgfpathlineto{\pgfqpoint{3.579200in}{1.816218in}}%
\pgfpathlineto{\pgfqpoint{3.586640in}{1.775349in}}%
\pgfpathlineto{\pgfqpoint{3.592840in}{1.753416in}}%
\pgfpathlineto{\pgfqpoint{3.597800in}{1.743935in}}%
\pgfpathlineto{\pgfqpoint{3.601520in}{1.741566in}}%
\pgfpathlineto{\pgfqpoint{3.604000in}{1.742251in}}%
\pgfpathlineto{\pgfqpoint{3.607720in}{1.746675in}}%
\pgfpathlineto{\pgfqpoint{3.611440in}{1.755169in}}%
\pgfpathlineto{\pgfqpoint{3.616400in}{1.772789in}}%
\pgfpathlineto{\pgfqpoint{3.622600in}{1.804825in}}%
\pgfpathlineto{\pgfqpoint{3.630040in}{1.857654in}}%
\pgfpathlineto{\pgfqpoint{3.638720in}{1.938425in}}%
\pgfpathlineto{\pgfqpoint{3.648640in}{2.054504in}}%
\pgfpathlineto{\pgfqpoint{3.659800in}{2.212568in}}%
\pgfpathlineto{\pgfqpoint{3.673440in}{2.439130in}}%
\pgfpathlineto{\pgfqpoint{3.692040in}{2.790055in}}%
\pgfpathlineto{\pgfqpoint{3.737920in}{3.675810in}}%
\pgfpathlineto{\pgfqpoint{3.752800in}{3.910420in}}%
\pgfpathlineto{\pgfqpoint{3.763960in}{4.055080in}}%
\pgfpathlineto{\pgfqpoint{3.773880in}{4.157489in}}%
\pgfpathlineto{\pgfqpoint{3.782560in}{4.225131in}}%
\pgfpathlineto{\pgfqpoint{3.790000in}{4.265958in}}%
\pgfpathlineto{\pgfqpoint{3.796200in}{4.287531in}}%
\pgfpathlineto{\pgfqpoint{3.801160in}{4.296515in}}%
\pgfpathlineto{\pgfqpoint{3.804880in}{4.298392in}}%
\pgfpathlineto{\pgfqpoint{3.807360in}{4.297323in}}%
\pgfpathlineto{\pgfqpoint{3.811080in}{4.292239in}}%
\pgfpathlineto{\pgfqpoint{3.814800in}{4.282989in}}%
\pgfpathlineto{\pgfqpoint{3.819760in}{4.264207in}}%
\pgfpathlineto{\pgfqpoint{3.825960in}{4.230482in}}%
\pgfpathlineto{\pgfqpoint{3.833400in}{4.175287in}}%
\pgfpathlineto{\pgfqpoint{3.842080in}{4.091303in}}%
\pgfpathlineto{\pgfqpoint{3.852000in}{3.970993in}}%
\pgfpathlineto{\pgfqpoint{3.863160in}{3.807520in}}%
\pgfpathlineto{\pgfqpoint{3.876800in}{3.573570in}}%
\pgfpathlineto{\pgfqpoint{3.895400in}{3.211670in}}%
\pgfpathlineto{\pgfqpoint{3.941280in}{2.299965in}}%
\pgfpathlineto{\pgfqpoint{3.954920in}{2.077238in}}%
\pgfpathlineto{\pgfqpoint{3.967320in}{1.910365in}}%
\pgfpathlineto{\pgfqpoint{3.977240in}{1.805234in}}%
\pgfpathlineto{\pgfqpoint{3.985920in}{1.735788in}}%
\pgfpathlineto{\pgfqpoint{3.993360in}{1.693850in}}%
\pgfpathlineto{\pgfqpoint{3.999560in}{1.671656in}}%
\pgfpathlineto{\pgfqpoint{4.004520in}{1.662377in}}%
\pgfpathlineto{\pgfqpoint{4.008240in}{1.660397in}}%
\pgfpathlineto{\pgfqpoint{4.010720in}{1.661452in}}%
\pgfpathlineto{\pgfqpoint{4.014440in}{1.666600in}}%
\pgfpathlineto{\pgfqpoint{4.018160in}{1.676016in}}%
\pgfpathlineto{\pgfqpoint{4.023120in}{1.695174in}}%
\pgfpathlineto{\pgfqpoint{4.029320in}{1.729625in}}%
\pgfpathlineto{\pgfqpoint{4.036760in}{1.786065in}}%
\pgfpathlineto{\pgfqpoint{4.045440in}{1.872016in}}%
\pgfpathlineto{\pgfqpoint{4.055360in}{1.995252in}}%
\pgfpathlineto{\pgfqpoint{4.066520in}{2.162858in}}%
\pgfpathlineto{\pgfqpoint{4.080160in}{2.402994in}}%
\pgfpathlineto{\pgfqpoint{4.098760in}{2.775004in}}%
\pgfpathlineto{\pgfqpoint{4.144640in}{3.714844in}}%
\pgfpathlineto{\pgfqpoint{4.159520in}{3.964207in}}%
\pgfpathlineto{\pgfqpoint{4.170680in}{4.118295in}}%
\pgfpathlineto{\pgfqpoint{4.180600in}{4.227764in}}%
\pgfpathlineto{\pgfqpoint{4.189280in}{4.300509in}}%
\pgfpathlineto{\pgfqpoint{4.196720in}{4.344885in}}%
\pgfpathlineto{\pgfqpoint{4.202920in}{4.368823in}}%
\pgfpathlineto{\pgfqpoint{4.207880in}{4.379306in}}%
\pgfpathlineto{\pgfqpoint{4.211600in}{4.382075in}}%
\pgfpathlineto{\pgfqpoint{4.214080in}{4.381489in}}%
\pgfpathlineto{\pgfqpoint{4.216560in}{4.378956in}}%
\pgfpathlineto{\pgfqpoint{4.220280in}{4.371513in}}%
\pgfpathlineto{\pgfqpoint{4.225240in}{4.354809in}}%
\pgfpathlineto{\pgfqpoint{4.231440in}{4.323126in}}%
\pgfpathlineto{\pgfqpoint{4.238880in}{4.269533in}}%
\pgfpathlineto{\pgfqpoint{4.247560in}{4.186187in}}%
\pgfpathlineto{\pgfqpoint{4.257480in}{4.064891in}}%
\pgfpathlineto{\pgfqpoint{4.268640in}{3.898010in}}%
\pgfpathlineto{\pgfqpoint{4.282280in}{3.656480in}}%
\pgfpathlineto{\pgfqpoint{4.299640in}{3.304745in}}%
\pgfpathlineto{\pgfqpoint{4.352960in}{2.189916in}}%
\pgfpathlineto{\pgfqpoint{4.366600in}{1.964436in}}%
\pgfpathlineto{\pgfqpoint{4.377760in}{1.813646in}}%
\pgfpathlineto{\pgfqpoint{4.387680in}{1.708528in}}%
\pgfpathlineto{\pgfqpoint{4.396360in}{1.640556in}}%
\pgfpathlineto{\pgfqpoint{4.402560in}{1.606300in}}%
\pgfpathlineto{\pgfqpoint{4.408760in}{1.584240in}}%
\pgfpathlineto{\pgfqpoint{4.412480in}{1.576926in}}%
\pgfpathlineto{\pgfqpoint{4.416200in}{1.574082in}}%
\pgfpathlineto{\pgfqpoint{4.418680in}{1.574673in}}%
\pgfpathlineto{\pgfqpoint{4.421160in}{1.577256in}}%
\pgfpathlineto{\pgfqpoint{4.424880in}{1.584858in}}%
\pgfpathlineto{\pgfqpoint{4.429840in}{1.601932in}}%
\pgfpathlineto{\pgfqpoint{4.436040in}{1.634331in}}%
\pgfpathlineto{\pgfqpoint{4.443480in}{1.689159in}}%
\pgfpathlineto{\pgfqpoint{4.452160in}{1.774463in}}%
\pgfpathlineto{\pgfqpoint{4.462080in}{1.898687in}}%
\pgfpathlineto{\pgfqpoint{4.473240in}{2.069729in}}%
\pgfpathlineto{\pgfqpoint{4.486880in}{2.317536in}}%
\pgfpathlineto{\pgfqpoint{4.504240in}{2.678903in}}%
\pgfpathlineto{\pgfqpoint{4.558800in}{3.850748in}}%
\pgfpathlineto{\pgfqpoint{4.572440in}{4.080179in}}%
\pgfpathlineto{\pgfqpoint{4.583600in}{4.232933in}}%
\pgfpathlineto{\pgfqpoint{4.593520in}{4.338870in}}%
\pgfpathlineto{\pgfqpoint{4.600960in}{4.398613in}}%
\pgfpathlineto{\pgfqpoint{4.608400in}{4.440780in}}%
\pgfpathlineto{\pgfqpoint{4.613360in}{4.458916in}}%
\pgfpathlineto{\pgfqpoint{4.618320in}{4.468974in}}%
\pgfpathlineto{\pgfqpoint{4.622040in}{4.471186in}}%
\pgfpathlineto{\pgfqpoint{4.624520in}{4.470116in}}%
\pgfpathlineto{\pgfqpoint{4.628240in}{4.464696in}}%
\pgfpathlineto{\pgfqpoint{4.631960in}{4.454705in}}%
\pgfpathlineto{\pgfqpoint{4.636920in}{4.434306in}}%
\pgfpathlineto{\pgfqpoint{4.643120in}{4.397542in}}%
\pgfpathlineto{\pgfqpoint{4.650560in}{4.337200in}}%
\pgfpathlineto{\pgfqpoint{4.659240in}{4.245130in}}%
\pgfpathlineto{\pgfqpoint{4.669160in}{4.112832in}}%
\pgfpathlineto{\pgfqpoint{4.680320in}{3.932421in}}%
\pgfpathlineto{\pgfqpoint{4.693960in}{3.673079in}}%
\pgfpathlineto{\pgfqpoint{4.711320in}{3.297637in}}%
\pgfpathlineto{\pgfqpoint{4.762160in}{2.167451in}}%
\pgfpathlineto{\pgfqpoint{4.775800in}{1.923903in}}%
\pgfpathlineto{\pgfqpoint{4.786960in}{1.759415in}}%
\pgfpathlineto{\pgfqpoint{4.796880in}{1.643161in}}%
\pgfpathlineto{\pgfqpoint{4.805560in}{1.566357in}}%
\pgfpathlineto{\pgfqpoint{4.813000in}{1.519871in}}%
\pgfpathlineto{\pgfqpoint{4.819200in}{1.495123in}}%
\pgfpathlineto{\pgfqpoint{4.824160in}{1.484604in}}%
\pgfpathlineto{\pgfqpoint{4.827880in}{1.482162in}}%
\pgfpathlineto{\pgfqpoint{4.830360in}{1.483134in}}%
\pgfpathlineto{\pgfqpoint{4.834080in}{1.488491in}}%
\pgfpathlineto{\pgfqpoint{4.837800in}{1.498520in}}%
\pgfpathlineto{\pgfqpoint{4.842760in}{1.519126in}}%
\pgfpathlineto{\pgfqpoint{4.848960in}{1.556405in}}%
\pgfpathlineto{\pgfqpoint{4.856400in}{1.617749in}}%
\pgfpathlineto{\pgfqpoint{4.865080in}{1.711526in}}%
\pgfpathlineto{\pgfqpoint{4.875000in}{1.846506in}}%
\pgfpathlineto{\pgfqpoint{4.886160in}{2.030874in}}%
\pgfpathlineto{\pgfqpoint{4.899800in}{2.296387in}}%
\pgfpathlineto{\pgfqpoint{4.917160in}{2.681605in}}%
\pgfpathlineto{\pgfqpoint{4.969240in}{3.871291in}}%
\pgfpathlineto{\pgfqpoint{4.982880in}{4.119767in}}%
\pgfpathlineto{\pgfqpoint{4.994040in}{4.287073in}}%
\pgfpathlineto{\pgfqpoint{5.003960in}{4.404938in}}%
\pgfpathlineto{\pgfqpoint{5.012640in}{4.482489in}}%
\pgfpathlineto{\pgfqpoint{5.020080in}{4.529145in}}%
\pgfpathlineto{\pgfqpoint{5.026280in}{4.553715in}}%
\pgfpathlineto{\pgfqpoint{5.031240in}{4.563886in}}%
\pgfpathlineto{\pgfqpoint{5.033720in}{4.565793in}}%
\pgfpathlineto{\pgfqpoint{5.036200in}{4.565577in}}%
\pgfpathlineto{\pgfqpoint{5.038680in}{4.563236in}}%
\pgfpathlineto{\pgfqpoint{5.042400in}{4.555746in}}%
\pgfpathlineto{\pgfqpoint{5.047360in}{4.538350in}}%
\pgfpathlineto{\pgfqpoint{5.052320in}{4.512542in}}%
\pgfpathlineto{\pgfqpoint{5.058520in}{4.468586in}}%
\pgfpathlineto{\pgfqpoint{5.065960in}{4.399022in}}%
\pgfpathlineto{\pgfqpoint{5.074640in}{4.295449in}}%
\pgfpathlineto{\pgfqpoint{5.084560in}{4.149146in}}%
\pgfpathlineto{\pgfqpoint{5.096960in}{3.928301in}}%
\pgfpathlineto{\pgfqpoint{5.111840in}{3.616887in}}%
\pgfpathlineto{\pgfqpoint{5.131680in}{3.149108in}}%
\pgfpathlineto{\pgfqpoint{5.170120in}{2.234939in}}%
\pgfpathlineto{\pgfqpoint{5.185000in}{1.936380in}}%
\pgfpathlineto{\pgfqpoint{5.197400in}{1.729443in}}%
\pgfpathlineto{\pgfqpoint{5.207320in}{1.596024in}}%
\pgfpathlineto{\pgfqpoint{5.216000in}{1.504812in}}%
\pgfpathlineto{\pgfqpoint{5.223440in}{1.446542in}}%
\pgfpathlineto{\pgfqpoint{5.229640in}{1.412430in}}%
\pgfpathlineto{\pgfqpoint{5.234600in}{1.394751in}}%
\pgfpathlineto{\pgfqpoint{5.238320in}{1.387146in}}%
\pgfpathlineto{\pgfqpoint{5.242040in}{1.384408in}}%
\pgfpathlineto{\pgfqpoint{5.244520in}{1.385291in}}%
\pgfpathlineto{\pgfqpoint{5.247000in}{1.388342in}}%
\pgfpathlineto{\pgfqpoint{5.250720in}{1.396977in}}%
\pgfpathlineto{\pgfqpoint{5.255680in}{1.416045in}}%
\pgfpathlineto{\pgfqpoint{5.261880in}{1.451933in}}%
\pgfpathlineto{\pgfqpoint{5.269320in}{1.512421in}}%
\pgfpathlineto{\pgfqpoint{5.278000in}{1.606389in}}%
\pgfpathlineto{\pgfqpoint{5.287920in}{1.743263in}}%
\pgfpathlineto{\pgfqpoint{5.299080in}{1.932052in}}%
\pgfpathlineto{\pgfqpoint{5.312720in}{2.206446in}}%
\pgfpathlineto{\pgfqpoint{5.330080in}{2.608533in}}%
\pgfpathlineto{\pgfqpoint{5.387120in}{3.979312in}}%
\pgfpathlineto{\pgfqpoint{5.400760in}{4.233167in}}%
\pgfpathlineto{\pgfqpoint{5.411920in}{4.401812in}}%
\pgfpathlineto{\pgfqpoint{5.421840in}{4.518754in}}%
\pgfpathlineto{\pgfqpoint{5.430520in}{4.594012in}}%
\pgfpathlineto{\pgfqpoint{5.436720in}{4.631766in}}%
\pgfpathlineto{\pgfqpoint{5.441680in}{4.652190in}}%
\pgfpathlineto{\pgfqpoint{5.446640in}{4.663837in}}%
\pgfpathlineto{\pgfqpoint{5.450360in}{4.666786in}}%
\pgfpathlineto{\pgfqpoint{5.452840in}{4.665990in}}%
\pgfpathlineto{\pgfqpoint{5.455320in}{4.662984in}}%
\pgfpathlineto{\pgfqpoint{5.459040in}{4.654336in}}%
\pgfpathlineto{\pgfqpoint{5.464000in}{4.635100in}}%
\pgfpathlineto{\pgfqpoint{5.470200in}{4.598756in}}%
\pgfpathlineto{\pgfqpoint{5.477640in}{4.537352in}}%
\pgfpathlineto{\pgfqpoint{5.485080in}{4.456983in}}%
\pgfpathlineto{\pgfqpoint{5.493760in}{4.340134in}}%
\pgfpathlineto{\pgfqpoint{5.504920in}{4.155616in}}%
\pgfpathlineto{\pgfqpoint{5.517320in}{3.910128in}}%
\pgfpathlineto{\pgfqpoint{5.532200in}{3.569928in}}%
\pgfpathlineto{\pgfqpoint{5.555760in}{2.970917in}}%
\pgfpathlineto{\pgfqpoint{5.585520in}{2.224437in}}%
\pgfpathlineto{\pgfqpoint{5.600400in}{1.903124in}}%
\pgfpathlineto{\pgfqpoint{5.612800in}{1.678264in}}%
\pgfpathlineto{\pgfqpoint{5.623960in}{1.515352in}}%
\pgfpathlineto{\pgfqpoint{5.632640in}{1.416907in}}%
\pgfpathlineto{\pgfqpoint{5.640080in}{1.353219in}}%
\pgfpathlineto{\pgfqpoint{5.646280in}{1.315147in}}%
\pgfpathlineto{\pgfqpoint{5.651240in}{1.294665in}}%
\pgfpathlineto{\pgfqpoint{5.656200in}{1.283129in}}%
\pgfpathlineto{\pgfqpoint{5.659920in}{1.280374in}}%
\pgfpathlineto{\pgfqpoint{5.662400in}{1.281352in}}%
\pgfpathlineto{\pgfqpoint{5.664880in}{1.284580in}}%
\pgfpathlineto{\pgfqpoint{5.668600in}{1.293640in}}%
\pgfpathlineto{\pgfqpoint{5.673560in}{1.313570in}}%
\pgfpathlineto{\pgfqpoint{5.679760in}{1.351014in}}%
\pgfpathlineto{\pgfqpoint{5.687200in}{1.414090in}}%
\pgfpathlineto{\pgfqpoint{5.695880in}{1.512098in}}%
\pgfpathlineto{\pgfqpoint{5.705800in}{1.654989in}}%
\pgfpathlineto{\pgfqpoint{5.716960in}{1.852397in}}%
\pgfpathlineto{\pgfqpoint{5.729360in}{2.111999in}}%
\pgfpathlineto{\pgfqpoint{5.745480in}{2.499542in}}%
\pgfpathlineto{\pgfqpoint{5.774000in}{3.253519in}}%
\pgfpathlineto{\pgfqpoint{5.797560in}{3.848971in}}%
\pgfpathlineto{\pgfqpoint{5.812440in}{4.171970in}}%
\pgfpathlineto{\pgfqpoint{5.824840in}{4.395683in}}%
\pgfpathlineto{\pgfqpoint{5.834760in}{4.540104in}}%
\pgfpathlineto{\pgfqpoint{5.843440in}{4.639205in}}%
\pgfpathlineto{\pgfqpoint{5.850880in}{4.702980in}}%
\pgfpathlineto{\pgfqpoint{5.857080in}{4.740810in}}%
\pgfpathlineto{\pgfqpoint{5.862040in}{4.760904in}}%
\pgfpathlineto{\pgfqpoint{5.865760in}{4.769996in}}%
\pgfpathlineto{\pgfqpoint{5.869480in}{4.773942in}}%
\pgfpathlineto{\pgfqpoint{5.871960in}{4.773710in}}%
\pgfpathlineto{\pgfqpoint{5.874440in}{4.771186in}}%
\pgfpathlineto{\pgfqpoint{5.878160in}{4.763107in}}%
\pgfpathlineto{\pgfqpoint{5.883120in}{4.744337in}}%
\pgfpathlineto{\pgfqpoint{5.888080in}{4.716473in}}%
\pgfpathlineto{\pgfqpoint{5.894280in}{4.668973in}}%
\pgfpathlineto{\pgfqpoint{5.901720in}{4.593690in}}%
\pgfpathlineto{\pgfqpoint{5.910400in}{4.481351in}}%
\pgfpathlineto{\pgfqpoint{5.920320in}{4.322160in}}%
\pgfpathlineto{\pgfqpoint{5.931480in}{4.106860in}}%
\pgfpathlineto{\pgfqpoint{5.945120in}{3.798728in}}%
\pgfpathlineto{\pgfqpoint{5.962480in}{3.353433in}}%
\pgfpathlineto{\pgfqpoint{6.013320in}{2.012870in}}%
\pgfpathlineto{\pgfqpoint{6.026960in}{1.722931in}}%
\pgfpathlineto{\pgfqpoint{6.038120in}{1.525747in}}%
\pgfpathlineto{\pgfqpoint{6.048040in}{1.384607in}}%
\pgfpathlineto{\pgfqpoint{6.056720in}{1.289275in}}%
\pgfpathlineto{\pgfqpoint{6.064160in}{1.229306in}}%
\pgfpathlineto{\pgfqpoint{6.070360in}{1.195005in}}%
\pgfpathlineto{\pgfqpoint{6.075320in}{1.177947in}}%
\pgfpathlineto{\pgfqpoint{6.079040in}{1.171246in}}%
\pgfpathlineto{\pgfqpoint{6.081520in}{1.169689in}}%
\pgfpathlineto{\pgfqpoint{6.084000in}{1.170461in}}%
\pgfpathlineto{\pgfqpoint{6.086480in}{1.173564in}}%
\pgfpathlineto{\pgfqpoint{6.090200in}{1.182583in}}%
\pgfpathlineto{\pgfqpoint{6.095160in}{1.202740in}}%
\pgfpathlineto{\pgfqpoint{6.101360in}{1.240929in}}%
\pgfpathlineto{\pgfqpoint{6.107560in}{1.293407in}}%
\pgfpathlineto{\pgfqpoint{6.115000in}{1.374911in}}%
\pgfpathlineto{\pgfqpoint{6.123680in}{1.494817in}}%
\pgfpathlineto{\pgfqpoint{6.133600in}{1.663002in}}%
\pgfpathlineto{\pgfqpoint{6.146000in}{1.916004in}}%
\pgfpathlineto{\pgfqpoint{6.160880in}{2.272614in}}%
\pgfpathlineto{\pgfqpoint{6.180720in}{2.809427in}}%
\pgfpathlineto{\pgfqpoint{6.220400in}{3.894767in}}%
\pgfpathlineto{\pgfqpoint{6.235280in}{4.237007in}}%
\pgfpathlineto{\pgfqpoint{6.247680in}{4.474718in}}%
\pgfpathlineto{\pgfqpoint{6.258840in}{4.645811in}}%
\pgfpathlineto{\pgfqpoint{6.267520in}{4.748569in}}%
\pgfpathlineto{\pgfqpoint{6.274960in}{4.814630in}}%
\pgfpathlineto{\pgfqpoint{6.281160in}{4.853802in}}%
\pgfpathlineto{\pgfqpoint{6.286120in}{4.874613in}}%
\pgfpathlineto{\pgfqpoint{6.289840in}{4.884040in}}%
\pgfpathlineto{\pgfqpoint{6.293560in}{4.888152in}}%
\pgfpathlineto{\pgfqpoint{6.296040in}{4.887935in}}%
\pgfpathlineto{\pgfqpoint{6.298520in}{4.885350in}}%
\pgfpathlineto{\pgfqpoint{6.302240in}{4.877038in}}%
\pgfpathlineto{\pgfqpoint{6.307200in}{4.857691in}}%
\pgfpathlineto{\pgfqpoint{6.312160in}{4.828938in}}%
\pgfpathlineto{\pgfqpoint{6.318360in}{4.779878in}}%
\pgfpathlineto{\pgfqpoint{6.325800in}{4.702034in}}%
\pgfpathlineto{\pgfqpoint{6.334480in}{4.585699in}}%
\pgfpathlineto{\pgfqpoint{6.344400in}{4.420510in}}%
\pgfpathlineto{\pgfqpoint{6.355560in}{4.196508in}}%
\pgfpathlineto{\pgfqpoint{6.369200in}{3.874794in}}%
\pgfpathlineto{\pgfqpoint{6.386560in}{3.407659in}}%
\pgfpathlineto{\pgfqpoint{6.439880in}{1.925945in}}%
\pgfpathlineto{\pgfqpoint{6.453520in}{1.623971in}}%
\pgfpathlineto{\pgfqpoint{6.464680in}{1.419360in}}%
\pgfpathlineto{\pgfqpoint{6.474600in}{1.273313in}}%
\pgfpathlineto{\pgfqpoint{6.483280in}{1.174881in}}%
\pgfpathlineto{\pgfqpoint{6.490720in}{1.113071in}}%
\pgfpathlineto{\pgfqpoint{6.496920in}{1.077772in}}%
\pgfpathlineto{\pgfqpoint{6.501880in}{1.060253in}}%
\pgfpathlineto{\pgfqpoint{6.505600in}{1.053399in}}%
\pgfpathlineto{\pgfqpoint{6.508080in}{1.051828in}}%
\pgfpathlineto{\pgfqpoint{6.510560in}{1.052660in}}%
\pgfpathlineto{\pgfqpoint{6.513040in}{1.055893in}}%
\pgfpathlineto{\pgfqpoint{6.516760in}{1.065244in}}%
\pgfpathlineto{\pgfqpoint{6.521720in}{1.086096in}}%
\pgfpathlineto{\pgfqpoint{6.527920in}{1.125572in}}%
\pgfpathlineto{\pgfqpoint{6.534120in}{1.179820in}}%
\pgfpathlineto{\pgfqpoint{6.541560in}{1.264120in}}%
\pgfpathlineto{\pgfqpoint{6.550240in}{1.388283in}}%
\pgfpathlineto{\pgfqpoint{6.560160in}{1.562759in}}%
\pgfpathlineto{\pgfqpoint{6.571320in}{1.797537in}}%
\pgfpathlineto{\pgfqpoint{6.584960in}{2.132613in}}%
\pgfpathlineto{\pgfqpoint{6.603560in}{2.652517in}}%
\pgfpathlineto{\pgfqpoint{6.651920in}{4.041086in}}%
\pgfpathlineto{\pgfqpoint{6.666800in}{4.387848in}}%
\pgfpathlineto{\pgfqpoint{6.679200in}{4.624695in}}%
\pgfpathlineto{\pgfqpoint{6.689120in}{4.775676in}}%
\pgfpathlineto{\pgfqpoint{6.697800in}{4.877949in}}%
\pgfpathlineto{\pgfqpoint{6.705240in}{4.942717in}}%
\pgfpathlineto{\pgfqpoint{6.711440in}{4.980254in}}%
\pgfpathlineto{\pgfqpoint{6.716400in}{4.999420in}}%
\pgfpathlineto{\pgfqpoint{6.720120in}{5.007425in}}%
\pgfpathlineto{\pgfqpoint{6.722600in}{5.009723in}}%
\pgfpathlineto{\pgfqpoint{6.725080in}{5.009587in}}%
\pgfpathlineto{\pgfqpoint{6.727560in}{5.007016in}}%
\pgfpathlineto{\pgfqpoint{6.731280in}{4.998598in}}%
\pgfpathlineto{\pgfqpoint{6.736240in}{4.978871in}}%
\pgfpathlineto{\pgfqpoint{6.741200in}{4.949458in}}%
\pgfpathlineto{\pgfqpoint{6.747400in}{4.899164in}}%
\pgfpathlineto{\pgfqpoint{6.754840in}{4.819200in}}%
\pgfpathlineto{\pgfqpoint{6.763520in}{4.699433in}}%
\pgfpathlineto{\pgfqpoint{6.773440in}{4.528924in}}%
\pgfpathlineto{\pgfqpoint{6.784600in}{4.296950in}}%
\pgfpathlineto{\pgfqpoint{6.798240in}{3.962372in}}%
\pgfpathlineto{\pgfqpoint{6.815600in}{3.473784in}}%
\pgfpathlineto{\pgfqpoint{6.872640in}{1.809251in}}%
\pgfpathlineto{\pgfqpoint{6.886280in}{1.499136in}}%
\pgfpathlineto{\pgfqpoint{6.897440in}{1.290724in}}%
\pgfpathlineto{\pgfqpoint{6.907360in}{1.143149in}}%
\pgfpathlineto{\pgfqpoint{6.916040in}{1.044607in}}%
\pgfpathlineto{\pgfqpoint{6.923480in}{0.983494in}}%
\pgfpathlineto{\pgfqpoint{6.929680in}{0.949280in}}%
\pgfpathlineto{\pgfqpoint{6.934640in}{0.932931in}}%
\pgfpathlineto{\pgfqpoint{6.938360in}{0.927125in}}%
\pgfpathlineto{\pgfqpoint{6.940840in}{0.926332in}}%
\pgfpathlineto{\pgfqpoint{6.943320in}{0.928004in}}%
\pgfpathlineto{\pgfqpoint{6.947040in}{0.935130in}}%
\pgfpathlineto{\pgfqpoint{6.950760in}{0.947793in}}%
\pgfpathlineto{\pgfqpoint{6.955720in}{0.973273in}}%
\pgfpathlineto{\pgfqpoint{6.961920in}{1.018872in}}%
\pgfpathlineto{\pgfqpoint{6.969360in}{1.093575in}}%
\pgfpathlineto{\pgfqpoint{6.978040in}{1.207821in}}%
\pgfpathlineto{\pgfqpoint{6.987960in}{1.373031in}}%
\pgfpathlineto{\pgfqpoint{6.999120in}{1.600672in}}%
\pgfpathlineto{\pgfqpoint{7.011520in}{1.900507in}}%
\pgfpathlineto{\pgfqpoint{7.027640in}{2.350512in}}%
\pgfpathlineto{\pgfqpoint{7.052440in}{3.118648in}}%
\pgfpathlineto{\pgfqpoint{7.079720in}{3.944712in}}%
\pgfpathlineto{\pgfqpoint{7.095840in}{4.364288in}}%
\pgfpathlineto{\pgfqpoint{7.108240in}{4.633832in}}%
\pgfpathlineto{\pgfqpoint{7.119400in}{4.830518in}}%
\pgfpathlineto{\pgfqpoint{7.129320in}{4.966110in}}%
\pgfpathlineto{\pgfqpoint{7.138000in}{5.053275in}}%
\pgfpathlineto{\pgfqpoint{7.144200in}{5.097196in}}%
\pgfpathlineto{\pgfqpoint{7.150400in}{5.125678in}}%
\pgfpathlineto{\pgfqpoint{7.154120in}{5.135321in}}%
\pgfpathlineto{\pgfqpoint{7.157840in}{5.139368in}}%
\pgfpathlineto{\pgfqpoint{7.160320in}{5.138953in}}%
\pgfpathlineto{\pgfqpoint{7.162800in}{5.136047in}}%
\pgfpathlineto{\pgfqpoint{7.166520in}{5.127021in}}%
\pgfpathlineto{\pgfqpoint{7.171480in}{5.106284in}}%
\pgfpathlineto{\pgfqpoint{7.176440in}{5.075628in}}%
\pgfpathlineto{\pgfqpoint{7.182640in}{5.023432in}}%
\pgfpathlineto{\pgfqpoint{7.190080in}{4.940637in}}%
\pgfpathlineto{\pgfqpoint{7.198760in}{4.816725in}}%
\pgfpathlineto{\pgfqpoint{7.200000in}{4.796666in}}%
\pgfpathlineto{\pgfqpoint{7.200000in}{4.796666in}}%
\pgfusepath{stroke}%
\end{pgfscope}%
\begin{pgfscope}%
\pgfpathrectangle{\pgfqpoint{1.000000in}{0.600000in}}{\pgfqpoint{6.200000in}{4.800000in}}%
\pgfusepath{clip}%
\pgfsetrectcap%
\pgfsetroundjoin%
\pgfsetlinewidth{1.003750pt}%
\definecolor{currentstroke}{rgb}{0.000000,0.000000,1.000000}%
\pgfsetstrokecolor{currentstroke}%
\pgfsetdash{}{0pt}%
\pgfpathmoveto{\pgfqpoint{1.000000in}{3.837758in}}%
\pgfpathlineto{\pgfqpoint{1.003720in}{3.835839in}}%
\pgfpathlineto{\pgfqpoint{1.007440in}{3.831048in}}%
\pgfpathlineto{\pgfqpoint{1.012400in}{3.820226in}}%
\pgfpathlineto{\pgfqpoint{1.018600in}{3.799672in}}%
\pgfpathlineto{\pgfqpoint{1.026040in}{3.764966in}}%
\pgfpathlineto{\pgfqpoint{1.034720in}{3.711233in}}%
\pgfpathlineto{\pgfqpoint{1.044640in}{3.633583in}}%
\pgfpathlineto{\pgfqpoint{1.055800in}{3.527783in}}%
\pgfpathlineto{\pgfqpoint{1.069440in}{3.376625in}}%
\pgfpathlineto{\pgfqpoint{1.089280in}{3.128025in}}%
\pgfpathlineto{\pgfqpoint{1.128960in}{2.623098in}}%
\pgfpathlineto{\pgfqpoint{1.143840in}{2.463971in}}%
\pgfpathlineto{\pgfqpoint{1.156240in}{2.354594in}}%
\pgfpathlineto{\pgfqpoint{1.166160in}{2.285230in}}%
\pgfpathlineto{\pgfqpoint{1.174840in}{2.239131in}}%
\pgfpathlineto{\pgfqpoint{1.182280in}{2.211090in}}%
\pgfpathlineto{\pgfqpoint{1.188480in}{2.196075in}}%
\pgfpathlineto{\pgfqpoint{1.193440in}{2.189620in}}%
\pgfpathlineto{\pgfqpoint{1.197160in}{2.188043in}}%
\pgfpathlineto{\pgfqpoint{1.200880in}{2.189265in}}%
\pgfpathlineto{\pgfqpoint{1.204600in}{2.193281in}}%
\pgfpathlineto{\pgfqpoint{1.209560in}{2.202953in}}%
\pgfpathlineto{\pgfqpoint{1.215760in}{2.221890in}}%
\pgfpathlineto{\pgfqpoint{1.221960in}{2.248264in}}%
\pgfpathlineto{\pgfqpoint{1.229400in}{2.289375in}}%
\pgfpathlineto{\pgfqpoint{1.238080in}{2.349670in}}%
\pgfpathlineto{\pgfqpoint{1.249240in}{2.444958in}}%
\pgfpathlineto{\pgfqpoint{1.261640in}{2.570896in}}%
\pgfpathlineto{\pgfqpoint{1.277760in}{2.758296in}}%
\pgfpathlineto{\pgfqpoint{1.337280in}{3.476253in}}%
\pgfpathlineto{\pgfqpoint{1.349680in}{3.590290in}}%
\pgfpathlineto{\pgfqpoint{1.360840in}{3.673020in}}%
\pgfpathlineto{\pgfqpoint{1.369520in}{3.722696in}}%
\pgfpathlineto{\pgfqpoint{1.376960in}{3.754330in}}%
\pgfpathlineto{\pgfqpoint{1.383160in}{3.772672in}}%
\pgfpathlineto{\pgfqpoint{1.388120in}{3.781983in}}%
\pgfpathlineto{\pgfqpoint{1.391840in}{3.785806in}}%
\pgfpathlineto{\pgfqpoint{1.395560in}{3.786909in}}%
\pgfpathlineto{\pgfqpoint{1.399280in}{3.785291in}}%
\pgfpathlineto{\pgfqpoint{1.403000in}{3.780961in}}%
\pgfpathlineto{\pgfqpoint{1.407960in}{3.770998in}}%
\pgfpathlineto{\pgfqpoint{1.414160in}{3.751904in}}%
\pgfpathlineto{\pgfqpoint{1.421600in}{3.719505in}}%
\pgfpathlineto{\pgfqpoint{1.430280in}{3.669194in}}%
\pgfpathlineto{\pgfqpoint{1.440200in}{3.596358in}}%
\pgfpathlineto{\pgfqpoint{1.451360in}{3.497006in}}%
\pgfpathlineto{\pgfqpoint{1.465000in}{3.354962in}}%
\pgfpathlineto{\pgfqpoint{1.484840in}{3.121235in}}%
\pgfpathlineto{\pgfqpoint{1.524520in}{2.646262in}}%
\pgfpathlineto{\pgfqpoint{1.539400in}{2.496520in}}%
\pgfpathlineto{\pgfqpoint{1.551800in}{2.393603in}}%
\pgfpathlineto{\pgfqpoint{1.561720in}{2.328367in}}%
\pgfpathlineto{\pgfqpoint{1.570400in}{2.285054in}}%
\pgfpathlineto{\pgfqpoint{1.577840in}{2.258757in}}%
\pgfpathlineto{\pgfqpoint{1.584040in}{2.244729in}}%
\pgfpathlineto{\pgfqpoint{1.589000in}{2.238753in}}%
\pgfpathlineto{\pgfqpoint{1.592720in}{2.237353in}}%
\pgfpathlineto{\pgfqpoint{1.596440in}{2.238597in}}%
\pgfpathlineto{\pgfqpoint{1.600160in}{2.242477in}}%
\pgfpathlineto{\pgfqpoint{1.605120in}{2.251724in}}%
\pgfpathlineto{\pgfqpoint{1.611320in}{2.269744in}}%
\pgfpathlineto{\pgfqpoint{1.617520in}{2.294775in}}%
\pgfpathlineto{\pgfqpoint{1.624960in}{2.333727in}}%
\pgfpathlineto{\pgfqpoint{1.633640in}{2.390778in}}%
\pgfpathlineto{\pgfqpoint{1.644800in}{2.480822in}}%
\pgfpathlineto{\pgfqpoint{1.657200in}{2.599677in}}%
\pgfpathlineto{\pgfqpoint{1.673320in}{2.776296in}}%
\pgfpathlineto{\pgfqpoint{1.731600in}{3.439057in}}%
\pgfpathlineto{\pgfqpoint{1.744000in}{3.547751in}}%
\pgfpathlineto{\pgfqpoint{1.755160in}{3.627040in}}%
\pgfpathlineto{\pgfqpoint{1.763840in}{3.674986in}}%
\pgfpathlineto{\pgfqpoint{1.771280in}{3.705812in}}%
\pgfpathlineto{\pgfqpoint{1.777480in}{3.723959in}}%
\pgfpathlineto{\pgfqpoint{1.782440in}{3.733428in}}%
\pgfpathlineto{\pgfqpoint{1.787400in}{3.738354in}}%
\pgfpathlineto{\pgfqpoint{1.791120in}{3.739055in}}%
\pgfpathlineto{\pgfqpoint{1.794840in}{3.737188in}}%
\pgfpathlineto{\pgfqpoint{1.798560in}{3.732762in}}%
\pgfpathlineto{\pgfqpoint{1.803520in}{3.722913in}}%
\pgfpathlineto{\pgfqpoint{1.809720in}{3.704348in}}%
\pgfpathlineto{\pgfqpoint{1.817160in}{3.673144in}}%
\pgfpathlineto{\pgfqpoint{1.825840in}{3.624990in}}%
\pgfpathlineto{\pgfqpoint{1.835760in}{3.555584in}}%
\pgfpathlineto{\pgfqpoint{1.846920in}{3.461241in}}%
\pgfpathlineto{\pgfqpoint{1.860560in}{3.326779in}}%
\pgfpathlineto{\pgfqpoint{1.880400in}{3.106283in}}%
\pgfpathlineto{\pgfqpoint{1.918840in}{2.673218in}}%
\pgfpathlineto{\pgfqpoint{1.933720in}{2.531530in}}%
\pgfpathlineto{\pgfqpoint{1.946120in}{2.433854in}}%
\pgfpathlineto{\pgfqpoint{1.956040in}{2.371726in}}%
\pgfpathlineto{\pgfqpoint{1.964720in}{2.330295in}}%
\pgfpathlineto{\pgfqpoint{1.972160in}{2.304969in}}%
\pgfpathlineto{\pgfqpoint{1.978360in}{2.291292in}}%
\pgfpathlineto{\pgfqpoint{1.983320in}{2.285297in}}%
\pgfpathlineto{\pgfqpoint{1.987040in}{2.283706in}}%
\pgfpathlineto{\pgfqpoint{1.990760in}{2.284611in}}%
\pgfpathlineto{\pgfqpoint{1.994480in}{2.288004in}}%
\pgfpathlineto{\pgfqpoint{1.999440in}{2.296375in}}%
\pgfpathlineto{\pgfqpoint{2.005640in}{2.312939in}}%
\pgfpathlineto{\pgfqpoint{2.011840in}{2.336128in}}%
\pgfpathlineto{\pgfqpoint{2.019280in}{2.372378in}}%
\pgfpathlineto{\pgfqpoint{2.027960in}{2.425640in}}%
\pgfpathlineto{\pgfqpoint{2.039120in}{2.509901in}}%
\pgfpathlineto{\pgfqpoint{2.051520in}{2.621315in}}%
\pgfpathlineto{\pgfqpoint{2.067640in}{2.787097in}}%
\pgfpathlineto{\pgfqpoint{2.127160in}{3.421759in}}%
\pgfpathlineto{\pgfqpoint{2.139560in}{3.522392in}}%
\pgfpathlineto{\pgfqpoint{2.150720in}{3.595258in}}%
\pgfpathlineto{\pgfqpoint{2.159400in}{3.638871in}}%
\pgfpathlineto{\pgfqpoint{2.166840in}{3.666506in}}%
\pgfpathlineto{\pgfqpoint{2.173040in}{3.682389in}}%
\pgfpathlineto{\pgfqpoint{2.178000in}{3.690319in}}%
\pgfpathlineto{\pgfqpoint{2.182960in}{3.693958in}}%
\pgfpathlineto{\pgfqpoint{2.186680in}{3.693861in}}%
\pgfpathlineto{\pgfqpoint{2.190400in}{3.691345in}}%
\pgfpathlineto{\pgfqpoint{2.195360in}{3.684246in}}%
\pgfpathlineto{\pgfqpoint{2.200320in}{3.672913in}}%
\pgfpathlineto{\pgfqpoint{2.206520in}{3.652905in}}%
\pgfpathlineto{\pgfqpoint{2.213960in}{3.620594in}}%
\pgfpathlineto{\pgfqpoint{2.222640in}{3.572031in}}%
\pgfpathlineto{\pgfqpoint{2.232560in}{3.503343in}}%
\pgfpathlineto{\pgfqpoint{2.244960in}{3.400239in}}%
\pgfpathlineto{\pgfqpoint{2.259840in}{3.256633in}}%
\pgfpathlineto{\pgfqpoint{2.283400in}{3.004328in}}%
\pgfpathlineto{\pgfqpoint{2.311920in}{2.704046in}}%
\pgfpathlineto{\pgfqpoint{2.326800in}{2.569148in}}%
\pgfpathlineto{\pgfqpoint{2.339200in}{2.475541in}}%
\pgfpathlineto{\pgfqpoint{2.349120in}{2.415537in}}%
\pgfpathlineto{\pgfqpoint{2.357800in}{2.375109in}}%
\pgfpathlineto{\pgfqpoint{2.365240in}{2.349999in}}%
\pgfpathlineto{\pgfqpoint{2.371440in}{2.336051in}}%
\pgfpathlineto{\pgfqpoint{2.376400in}{2.329547in}}%
\pgfpathlineto{\pgfqpoint{2.380120in}{2.327406in}}%
\pgfpathlineto{\pgfqpoint{2.383840in}{2.327618in}}%
\pgfpathlineto{\pgfqpoint{2.387560in}{2.330179in}}%
\pgfpathlineto{\pgfqpoint{2.392520in}{2.337228in}}%
\pgfpathlineto{\pgfqpoint{2.397480in}{2.348385in}}%
\pgfpathlineto{\pgfqpoint{2.403680in}{2.367999in}}%
\pgfpathlineto{\pgfqpoint{2.411120in}{2.399584in}}%
\pgfpathlineto{\pgfqpoint{2.419800in}{2.446961in}}%
\pgfpathlineto{\pgfqpoint{2.429720in}{2.513873in}}%
\pgfpathlineto{\pgfqpoint{2.442120in}{2.614179in}}%
\pgfpathlineto{\pgfqpoint{2.457000in}{2.753717in}}%
\pgfpathlineto{\pgfqpoint{2.481800in}{3.011679in}}%
\pgfpathlineto{\pgfqpoint{2.509080in}{3.289376in}}%
\pgfpathlineto{\pgfqpoint{2.523960in}{3.419802in}}%
\pgfpathlineto{\pgfqpoint{2.536360in}{3.510154in}}%
\pgfpathlineto{\pgfqpoint{2.546280in}{3.567947in}}%
\pgfpathlineto{\pgfqpoint{2.554960in}{3.606768in}}%
\pgfpathlineto{\pgfqpoint{2.562400in}{3.630763in}}%
\pgfpathlineto{\pgfqpoint{2.568600in}{3.643979in}}%
\pgfpathlineto{\pgfqpoint{2.573560in}{3.650030in}}%
\pgfpathlineto{\pgfqpoint{2.577280in}{3.651909in}}%
\pgfpathlineto{\pgfqpoint{2.581000in}{3.651504in}}%
\pgfpathlineto{\pgfqpoint{2.584720in}{3.648819in}}%
\pgfpathlineto{\pgfqpoint{2.589680in}{3.641712in}}%
\pgfpathlineto{\pgfqpoint{2.594640in}{3.630621in}}%
\pgfpathlineto{\pgfqpoint{2.600840in}{3.611265in}}%
\pgfpathlineto{\pgfqpoint{2.608280in}{3.580241in}}%
\pgfpathlineto{\pgfqpoint{2.616960in}{3.533858in}}%
\pgfpathlineto{\pgfqpoint{2.626880in}{3.468512in}}%
\pgfpathlineto{\pgfqpoint{2.639280in}{3.370757in}}%
\pgfpathlineto{\pgfqpoint{2.655400in}{3.223028in}}%
\pgfpathlineto{\pgfqpoint{2.682680in}{2.946504in}}%
\pgfpathlineto{\pgfqpoint{2.706240in}{2.715873in}}%
\pgfpathlineto{\pgfqpoint{2.721120in}{2.589961in}}%
\pgfpathlineto{\pgfqpoint{2.733520in}{2.502952in}}%
\pgfpathlineto{\pgfqpoint{2.743440in}{2.447473in}}%
\pgfpathlineto{\pgfqpoint{2.752120in}{2.410370in}}%
\pgfpathlineto{\pgfqpoint{2.759560in}{2.387596in}}%
\pgfpathlineto{\pgfqpoint{2.765760in}{2.375211in}}%
\pgfpathlineto{\pgfqpoint{2.770720in}{2.369696in}}%
\pgfpathlineto{\pgfqpoint{2.774440in}{2.368142in}}%
\pgfpathlineto{\pgfqpoint{2.778160in}{2.368806in}}%
\pgfpathlineto{\pgfqpoint{2.781880in}{2.371682in}}%
\pgfpathlineto{\pgfqpoint{2.786840in}{2.378936in}}%
\pgfpathlineto{\pgfqpoint{2.793040in}{2.393428in}}%
\pgfpathlineto{\pgfqpoint{2.799240in}{2.413808in}}%
\pgfpathlineto{\pgfqpoint{2.806680in}{2.445750in}}%
\pgfpathlineto{\pgfqpoint{2.815360in}{2.492756in}}%
\pgfpathlineto{\pgfqpoint{2.826520in}{2.567193in}}%
\pgfpathlineto{\pgfqpoint{2.838920in}{2.665664in}}%
\pgfpathlineto{\pgfqpoint{2.855040in}{2.812202in}}%
\pgfpathlineto{\pgfqpoint{2.914560in}{3.372987in}}%
\pgfpathlineto{\pgfqpoint{2.926960in}{3.461809in}}%
\pgfpathlineto{\pgfqpoint{2.938120in}{3.526035in}}%
\pgfpathlineto{\pgfqpoint{2.946800in}{3.564389in}}%
\pgfpathlineto{\pgfqpoint{2.954240in}{3.588601in}}%
\pgfpathlineto{\pgfqpoint{2.960440in}{3.602426in}}%
\pgfpathlineto{\pgfqpoint{2.965400in}{3.609241in}}%
\pgfpathlineto{\pgfqpoint{2.970360in}{3.612242in}}%
\pgfpathlineto{\pgfqpoint{2.974080in}{3.611982in}}%
\pgfpathlineto{\pgfqpoint{2.977800in}{3.609574in}}%
\pgfpathlineto{\pgfqpoint{2.982760in}{3.603039in}}%
\pgfpathlineto{\pgfqpoint{2.987720in}{3.592747in}}%
\pgfpathlineto{\pgfqpoint{2.993920in}{3.574704in}}%
\pgfpathlineto{\pgfqpoint{3.001360in}{3.545700in}}%
\pgfpathlineto{\pgfqpoint{3.010040in}{3.502257in}}%
\pgfpathlineto{\pgfqpoint{3.019960in}{3.440976in}}%
\pgfpathlineto{\pgfqpoint{3.032360in}{3.349222in}}%
\pgfpathlineto{\pgfqpoint{3.047240in}{3.221745in}}%
\pgfpathlineto{\pgfqpoint{3.072040in}{2.986469in}}%
\pgfpathlineto{\pgfqpoint{3.099320in}{2.733717in}}%
\pgfpathlineto{\pgfqpoint{3.114200in}{2.615260in}}%
\pgfpathlineto{\pgfqpoint{3.126600in}{2.533380in}}%
\pgfpathlineto{\pgfqpoint{3.136520in}{2.481166in}}%
\pgfpathlineto{\pgfqpoint{3.145200in}{2.446248in}}%
\pgfpathlineto{\pgfqpoint{3.152640in}{2.424821in}}%
\pgfpathlineto{\pgfqpoint{3.158840in}{2.413175in}}%
\pgfpathlineto{\pgfqpoint{3.163800in}{2.407998in}}%
\pgfpathlineto{\pgfqpoint{3.167520in}{2.406548in}}%
\pgfpathlineto{\pgfqpoint{3.171240in}{2.407186in}}%
\pgfpathlineto{\pgfqpoint{3.174960in}{2.409910in}}%
\pgfpathlineto{\pgfqpoint{3.179920in}{2.416762in}}%
\pgfpathlineto{\pgfqpoint{3.186120in}{2.430436in}}%
\pgfpathlineto{\pgfqpoint{3.192320in}{2.449654in}}%
\pgfpathlineto{\pgfqpoint{3.199760in}{2.479762in}}%
\pgfpathlineto{\pgfqpoint{3.208440in}{2.524051in}}%
\pgfpathlineto{\pgfqpoint{3.219600in}{2.594156in}}%
\pgfpathlineto{\pgfqpoint{3.232000in}{2.686850in}}%
\pgfpathlineto{\pgfqpoint{3.248120in}{2.824713in}}%
\pgfpathlineto{\pgfqpoint{3.307640in}{3.351566in}}%
\pgfpathlineto{\pgfqpoint{3.320040in}{3.434852in}}%
\pgfpathlineto{\pgfqpoint{3.331200in}{3.494989in}}%
\pgfpathlineto{\pgfqpoint{3.339880in}{3.530825in}}%
\pgfpathlineto{\pgfqpoint{3.347320in}{3.553374in}}%
\pgfpathlineto{\pgfqpoint{3.353520in}{3.566178in}}%
\pgfpathlineto{\pgfqpoint{3.358480in}{3.572420in}}%
\pgfpathlineto{\pgfqpoint{3.363440in}{3.575069in}}%
\pgfpathlineto{\pgfqpoint{3.367160in}{3.574691in}}%
\pgfpathlineto{\pgfqpoint{3.370880in}{3.572289in}}%
\pgfpathlineto{\pgfqpoint{3.375840in}{3.565958in}}%
\pgfpathlineto{\pgfqpoint{3.382040in}{3.553079in}}%
\pgfpathlineto{\pgfqpoint{3.388240in}{3.534810in}}%
\pgfpathlineto{\pgfqpoint{3.395680in}{3.506030in}}%
\pgfpathlineto{\pgfqpoint{3.404360in}{3.463530in}}%
\pgfpathlineto{\pgfqpoint{3.415520in}{3.396057in}}%
\pgfpathlineto{\pgfqpoint{3.427920in}{3.306631in}}%
\pgfpathlineto{\pgfqpoint{3.444040in}{3.173358in}}%
\pgfpathlineto{\pgfqpoint{3.503560in}{2.661981in}}%
\pgfpathlineto{\pgfqpoint{3.515960in}{2.580761in}}%
\pgfpathlineto{\pgfqpoint{3.527120in}{2.521965in}}%
\pgfpathlineto{\pgfqpoint{3.535800in}{2.486811in}}%
\pgfpathlineto{\pgfqpoint{3.543240in}{2.464585in}}%
\pgfpathlineto{\pgfqpoint{3.549440in}{2.451863in}}%
\pgfpathlineto{\pgfqpoint{3.554400in}{2.445565in}}%
\pgfpathlineto{\pgfqpoint{3.559360in}{2.442752in}}%
\pgfpathlineto{\pgfqpoint{3.563080in}{2.442937in}}%
\pgfpathlineto{\pgfqpoint{3.566800in}{2.445086in}}%
\pgfpathlineto{\pgfqpoint{3.571760in}{2.450991in}}%
\pgfpathlineto{\pgfqpoint{3.576720in}{2.460329in}}%
\pgfpathlineto{\pgfqpoint{3.582920in}{2.476735in}}%
\pgfpathlineto{\pgfqpoint{3.590360in}{2.503141in}}%
\pgfpathlineto{\pgfqpoint{3.599040in}{2.542722in}}%
\pgfpathlineto{\pgfqpoint{3.608960in}{2.598576in}}%
\pgfpathlineto{\pgfqpoint{3.621360in}{2.682215in}}%
\pgfpathlineto{\pgfqpoint{3.637480in}{2.808684in}}%
\pgfpathlineto{\pgfqpoint{3.664760in}{3.045489in}}%
\pgfpathlineto{\pgfqpoint{3.688320in}{3.243039in}}%
\pgfpathlineto{\pgfqpoint{3.703200in}{3.350891in}}%
\pgfpathlineto{\pgfqpoint{3.715600in}{3.425387in}}%
\pgfpathlineto{\pgfqpoint{3.725520in}{3.472839in}}%
\pgfpathlineto{\pgfqpoint{3.734200in}{3.504516in}}%
\pgfpathlineto{\pgfqpoint{3.741640in}{3.523894in}}%
\pgfpathlineto{\pgfqpoint{3.747840in}{3.534365in}}%
\pgfpathlineto{\pgfqpoint{3.752800in}{3.538959in}}%
\pgfpathlineto{\pgfqpoint{3.756520in}{3.540180in}}%
\pgfpathlineto{\pgfqpoint{3.760240in}{3.539492in}}%
\pgfpathlineto{\pgfqpoint{3.763960in}{3.536901in}}%
\pgfpathlineto{\pgfqpoint{3.768920in}{3.530503in}}%
\pgfpathlineto{\pgfqpoint{3.775120in}{3.517841in}}%
\pgfpathlineto{\pgfqpoint{3.782560in}{3.495985in}}%
\pgfpathlineto{\pgfqpoint{3.791240in}{3.461708in}}%
\pgfpathlineto{\pgfqpoint{3.801160in}{3.411792in}}%
\pgfpathlineto{\pgfqpoint{3.812320in}{3.343471in}}%
\pgfpathlineto{\pgfqpoint{3.825960in}{3.245605in}}%
\pgfpathlineto{\pgfqpoint{3.844560in}{3.094882in}}%
\pgfpathlineto{\pgfqpoint{3.886720in}{2.747348in}}%
\pgfpathlineto{\pgfqpoint{3.901600in}{2.645479in}}%
\pgfpathlineto{\pgfqpoint{3.912760in}{2.582328in}}%
\pgfpathlineto{\pgfqpoint{3.922680in}{2.537516in}}%
\pgfpathlineto{\pgfqpoint{3.931360in}{2.507938in}}%
\pgfpathlineto{\pgfqpoint{3.938800in}{2.490171in}}%
\pgfpathlineto{\pgfqpoint{3.945000in}{2.480894in}}%
\pgfpathlineto{\pgfqpoint{3.949960in}{2.477152in}}%
\pgfpathlineto{\pgfqpoint{3.953680in}{2.476504in}}%
\pgfpathlineto{\pgfqpoint{3.957400in}{2.477707in}}%
\pgfpathlineto{\pgfqpoint{3.962360in}{2.482180in}}%
\pgfpathlineto{\pgfqpoint{3.967320in}{2.489900in}}%
\pgfpathlineto{\pgfqpoint{3.973520in}{2.504037in}}%
\pgfpathlineto{\pgfqpoint{3.980960in}{2.527385in}}%
\pgfpathlineto{\pgfqpoint{3.989640in}{2.562990in}}%
\pgfpathlineto{\pgfqpoint{3.999560in}{2.613849in}}%
\pgfpathlineto{\pgfqpoint{4.011960in}{2.690752in}}%
\pgfpathlineto{\pgfqpoint{4.026840in}{2.798481in}}%
\pgfpathlineto{\pgfqpoint{4.050400in}{2.988822in}}%
\pgfpathlineto{\pgfqpoint{4.080160in}{3.226011in}}%
\pgfpathlineto{\pgfqpoint{4.095040in}{3.327713in}}%
\pgfpathlineto{\pgfqpoint{4.107440in}{3.398086in}}%
\pgfpathlineto{\pgfqpoint{4.117360in}{3.443000in}}%
\pgfpathlineto{\pgfqpoint{4.126040in}{3.473061in}}%
\pgfpathlineto{\pgfqpoint{4.133480in}{3.491523in}}%
\pgfpathlineto{\pgfqpoint{4.139680in}{3.501569in}}%
\pgfpathlineto{\pgfqpoint{4.144640in}{3.506047in}}%
\pgfpathlineto{\pgfqpoint{4.149600in}{3.507335in}}%
\pgfpathlineto{\pgfqpoint{4.153320in}{3.506206in}}%
\pgfpathlineto{\pgfqpoint{4.158280in}{3.501916in}}%
\pgfpathlineto{\pgfqpoint{4.163240in}{3.494476in}}%
\pgfpathlineto{\pgfqpoint{4.169440in}{3.480822in}}%
\pgfpathlineto{\pgfqpoint{4.176880in}{3.458243in}}%
\pgfpathlineto{\pgfqpoint{4.185560in}{3.423781in}}%
\pgfpathlineto{\pgfqpoint{4.195480in}{3.374529in}}%
\pgfpathlineto{\pgfqpoint{4.207880in}{3.300029in}}%
\pgfpathlineto{\pgfqpoint{4.222760in}{3.195638in}}%
\pgfpathlineto{\pgfqpoint{4.246320in}{3.011149in}}%
\pgfpathlineto{\pgfqpoint{4.276080in}{2.781191in}}%
\pgfpathlineto{\pgfqpoint{4.290960in}{2.682565in}}%
\pgfpathlineto{\pgfqpoint{4.303360in}{2.614312in}}%
\pgfpathlineto{\pgfqpoint{4.313280in}{2.570745in}}%
\pgfpathlineto{\pgfqpoint{4.321960in}{2.541585in}}%
\pgfpathlineto{\pgfqpoint{4.329400in}{2.523675in}}%
\pgfpathlineto{\pgfqpoint{4.335600in}{2.513928in}}%
\pgfpathlineto{\pgfqpoint{4.340560in}{2.509583in}}%
\pgfpathlineto{\pgfqpoint{4.345520in}{2.508333in}}%
\pgfpathlineto{\pgfqpoint{4.349240in}{2.509429in}}%
\pgfpathlineto{\pgfqpoint{4.354200in}{2.513589in}}%
\pgfpathlineto{\pgfqpoint{4.359160in}{2.520807in}}%
\pgfpathlineto{\pgfqpoint{4.365360in}{2.534051in}}%
\pgfpathlineto{\pgfqpoint{4.372800in}{2.555953in}}%
\pgfpathlineto{\pgfqpoint{4.381480in}{2.589379in}}%
\pgfpathlineto{\pgfqpoint{4.391400in}{2.637148in}}%
\pgfpathlineto{\pgfqpoint{4.403800in}{2.709396in}}%
\pgfpathlineto{\pgfqpoint{4.418680in}{2.810619in}}%
\pgfpathlineto{\pgfqpoint{4.442240in}{2.989475in}}%
\pgfpathlineto{\pgfqpoint{4.472000in}{3.212354in}}%
\pgfpathlineto{\pgfqpoint{4.486880in}{3.307917in}}%
\pgfpathlineto{\pgfqpoint{4.499280in}{3.374030in}}%
\pgfpathlineto{\pgfqpoint{4.509200in}{3.416212in}}%
\pgfpathlineto{\pgfqpoint{4.517880in}{3.444426in}}%
\pgfpathlineto{\pgfqpoint{4.525320in}{3.461736in}}%
\pgfpathlineto{\pgfqpoint{4.531520in}{3.471137in}}%
\pgfpathlineto{\pgfqpoint{4.536480in}{3.475308in}}%
\pgfpathlineto{\pgfqpoint{4.541440in}{3.476477in}}%
\pgfpathlineto{\pgfqpoint{4.545160in}{3.475382in}}%
\pgfpathlineto{\pgfqpoint{4.550120in}{3.471302in}}%
\pgfpathlineto{\pgfqpoint{4.555080in}{3.464258in}}%
\pgfpathlineto{\pgfqpoint{4.561280in}{3.451359in}}%
\pgfpathlineto{\pgfqpoint{4.568720in}{3.430054in}}%
\pgfpathlineto{\pgfqpoint{4.577400in}{3.397567in}}%
\pgfpathlineto{\pgfqpoint{4.587320in}{3.351170in}}%
\pgfpathlineto{\pgfqpoint{4.599720in}{3.281036in}}%
\pgfpathlineto{\pgfqpoint{4.614600in}{3.182825in}}%
\pgfpathlineto{\pgfqpoint{4.638160in}{3.009396in}}%
\pgfpathlineto{\pgfqpoint{4.666680in}{2.801783in}}%
\pgfpathlineto{\pgfqpoint{4.681560in}{2.708028in}}%
\pgfpathlineto{\pgfqpoint{4.693960in}{2.642733in}}%
\pgfpathlineto{\pgfqpoint{4.703880in}{2.600735in}}%
\pgfpathlineto{\pgfqpoint{4.712560in}{2.572338in}}%
\pgfpathlineto{\pgfqpoint{4.720000in}{2.554621in}}%
\pgfpathlineto{\pgfqpoint{4.726200in}{2.544708in}}%
\pgfpathlineto{\pgfqpoint{4.731160in}{2.540020in}}%
\pgfpathlineto{\pgfqpoint{4.736120in}{2.538240in}}%
\pgfpathlineto{\pgfqpoint{4.741080in}{2.539375in}}%
\pgfpathlineto{\pgfqpoint{4.746040in}{2.543414in}}%
\pgfpathlineto{\pgfqpoint{4.751000in}{2.550327in}}%
\pgfpathlineto{\pgfqpoint{4.757200in}{2.562938in}}%
\pgfpathlineto{\pgfqpoint{4.764640in}{2.583716in}}%
\pgfpathlineto{\pgfqpoint{4.773320in}{2.615349in}}%
\pgfpathlineto{\pgfqpoint{4.783240in}{2.660473in}}%
\pgfpathlineto{\pgfqpoint{4.795640in}{2.728619in}}%
\pgfpathlineto{\pgfqpoint{4.810520in}{2.823961in}}%
\pgfpathlineto{\pgfqpoint{4.834080in}{2.992160in}}%
\pgfpathlineto{\pgfqpoint{4.862600in}{3.193271in}}%
\pgfpathlineto{\pgfqpoint{4.877480in}{3.283983in}}%
\pgfpathlineto{\pgfqpoint{4.889880in}{3.347090in}}%
\pgfpathlineto{\pgfqpoint{4.899800in}{3.387624in}}%
\pgfpathlineto{\pgfqpoint{4.908480in}{3.414979in}}%
\pgfpathlineto{\pgfqpoint{4.915920in}{3.431996in}}%
\pgfpathlineto{\pgfqpoint{4.922120in}{3.441467in}}%
\pgfpathlineto{\pgfqpoint{4.927080in}{3.445898in}}%
\pgfpathlineto{\pgfqpoint{4.932040in}{3.447508in}}%
\pgfpathlineto{\pgfqpoint{4.937000in}{3.446290in}}%
\pgfpathlineto{\pgfqpoint{4.941960in}{3.442256in}}%
\pgfpathlineto{\pgfqpoint{4.946920in}{3.435437in}}%
\pgfpathlineto{\pgfqpoint{4.953120in}{3.423066in}}%
\pgfpathlineto{\pgfqpoint{4.960560in}{3.402753in}}%
\pgfpathlineto{\pgfqpoint{4.969240in}{3.371899in}}%
\pgfpathlineto{\pgfqpoint{4.979160in}{3.327957in}}%
\pgfpathlineto{\pgfqpoint{4.991560in}{3.261688in}}%
\pgfpathlineto{\pgfqpoint{5.006440in}{3.169082in}}%
\pgfpathlineto{\pgfqpoint{5.030000in}{3.005929in}}%
\pgfpathlineto{\pgfqpoint{5.058520in}{2.811167in}}%
\pgfpathlineto{\pgfqpoint{5.073400in}{2.723460in}}%
\pgfpathlineto{\pgfqpoint{5.085800in}{2.662533in}}%
\pgfpathlineto{\pgfqpoint{5.095720in}{2.623471in}}%
\pgfpathlineto{\pgfqpoint{5.104400in}{2.597175in}}%
\pgfpathlineto{\pgfqpoint{5.111840in}{2.580882in}}%
\pgfpathlineto{\pgfqpoint{5.118040in}{2.571876in}}%
\pgfpathlineto{\pgfqpoint{5.123000in}{2.567724in}}%
\pgfpathlineto{\pgfqpoint{5.127960in}{2.566310in}}%
\pgfpathlineto{\pgfqpoint{5.132920in}{2.567637in}}%
\pgfpathlineto{\pgfqpoint{5.137880in}{2.571695in}}%
\pgfpathlineto{\pgfqpoint{5.144080in}{2.580559in}}%
\pgfpathlineto{\pgfqpoint{5.150280in}{2.593548in}}%
\pgfpathlineto{\pgfqpoint{5.157720in}{2.614388in}}%
\pgfpathlineto{\pgfqpoint{5.166400in}{2.645553in}}%
\pgfpathlineto{\pgfqpoint{5.176320in}{2.689442in}}%
\pgfpathlineto{\pgfqpoint{5.188720in}{2.755034in}}%
\pgfpathlineto{\pgfqpoint{5.204840in}{2.853997in}}%
\pgfpathlineto{\pgfqpoint{5.233360in}{3.047215in}}%
\pgfpathlineto{\pgfqpoint{5.256920in}{3.199865in}}%
\pgfpathlineto{\pgfqpoint{5.271800in}{3.282292in}}%
\pgfpathlineto{\pgfqpoint{5.284200in}{3.338577in}}%
\pgfpathlineto{\pgfqpoint{5.294120in}{3.373899in}}%
\pgfpathlineto{\pgfqpoint{5.302800in}{3.396981in}}%
\pgfpathlineto{\pgfqpoint{5.310240in}{3.410614in}}%
\pgfpathlineto{\pgfqpoint{5.316440in}{3.417495in}}%
\pgfpathlineto{\pgfqpoint{5.321400in}{3.420023in}}%
\pgfpathlineto{\pgfqpoint{5.326360in}{3.419892in}}%
\pgfpathlineto{\pgfqpoint{5.331320in}{3.417106in}}%
\pgfpathlineto{\pgfqpoint{5.336280in}{3.411687in}}%
\pgfpathlineto{\pgfqpoint{5.342480in}{3.401269in}}%
\pgfpathlineto{\pgfqpoint{5.349920in}{3.383572in}}%
\pgfpathlineto{\pgfqpoint{5.358600in}{3.356096in}}%
\pgfpathlineto{\pgfqpoint{5.368520in}{3.316369in}}%
\pgfpathlineto{\pgfqpoint{5.379680in}{3.262296in}}%
\pgfpathlineto{\pgfqpoint{5.394560in}{3.177730in}}%
\pgfpathlineto{\pgfqpoint{5.415640in}{3.042526in}}%
\pgfpathlineto{\pgfqpoint{5.450360in}{2.818771in}}%
\pgfpathlineto{\pgfqpoint{5.465240in}{2.736940in}}%
\pgfpathlineto{\pgfqpoint{5.477640in}{2.680320in}}%
\pgfpathlineto{\pgfqpoint{5.487560in}{2.644199in}}%
\pgfpathlineto{\pgfqpoint{5.496240in}{2.620049in}}%
\pgfpathlineto{\pgfqpoint{5.503680in}{2.605246in}}%
\pgfpathlineto{\pgfqpoint{5.509880in}{2.597222in}}%
\pgfpathlineto{\pgfqpoint{5.514840in}{2.593678in}}%
\pgfpathlineto{\pgfqpoint{5.519800in}{2.592711in}}%
\pgfpathlineto{\pgfqpoint{5.524760in}{2.594322in}}%
\pgfpathlineto{\pgfqpoint{5.529720in}{2.598497in}}%
\pgfpathlineto{\pgfqpoint{5.535920in}{2.607275in}}%
\pgfpathlineto{\pgfqpoint{5.542120in}{2.619919in}}%
\pgfpathlineto{\pgfqpoint{5.549560in}{2.640010in}}%
\pgfpathlineto{\pgfqpoint{5.558240in}{2.669851in}}%
\pgfpathlineto{\pgfqpoint{5.569400in}{2.717420in}}%
\pgfpathlineto{\pgfqpoint{5.581800in}{2.780641in}}%
\pgfpathlineto{\pgfqpoint{5.597920in}{2.875046in}}%
\pgfpathlineto{\pgfqpoint{5.658680in}{3.244694in}}%
\pgfpathlineto{\pgfqpoint{5.671080in}{3.301467in}}%
\pgfpathlineto{\pgfqpoint{5.682240in}{3.342224in}}%
\pgfpathlineto{\pgfqpoint{5.690920in}{3.366298in}}%
\pgfpathlineto{\pgfqpoint{5.698360in}{3.381243in}}%
\pgfpathlineto{\pgfqpoint{5.704560in}{3.389529in}}%
\pgfpathlineto{\pgfqpoint{5.709520in}{3.393375in}}%
\pgfpathlineto{\pgfqpoint{5.714480in}{3.394724in}}%
\pgfpathlineto{\pgfqpoint{5.719440in}{3.393572in}}%
\pgfpathlineto{\pgfqpoint{5.724400in}{3.389930in}}%
\pgfpathlineto{\pgfqpoint{5.730600in}{3.381918in}}%
\pgfpathlineto{\pgfqpoint{5.736800in}{3.370143in}}%
\pgfpathlineto{\pgfqpoint{5.744240in}{3.351220in}}%
\pgfpathlineto{\pgfqpoint{5.752920in}{3.322892in}}%
\pgfpathlineto{\pgfqpoint{5.762840in}{3.282971in}}%
\pgfpathlineto{\pgfqpoint{5.775240in}{3.223283in}}%
\pgfpathlineto{\pgfqpoint{5.791360in}{3.133197in}}%
\pgfpathlineto{\pgfqpoint{5.819880in}{2.957261in}}%
\pgfpathlineto{\pgfqpoint{5.843440in}{2.818226in}}%
\pgfpathlineto{\pgfqpoint{5.858320in}{2.743137in}}%
\pgfpathlineto{\pgfqpoint{5.870720in}{2.691860in}}%
\pgfpathlineto{\pgfqpoint{5.880640in}{2.659684in}}%
\pgfpathlineto{\pgfqpoint{5.889320in}{2.638662in}}%
\pgfpathlineto{\pgfqpoint{5.896760in}{2.626254in}}%
\pgfpathlineto{\pgfqpoint{5.902960in}{2.619998in}}%
\pgfpathlineto{\pgfqpoint{5.907920in}{2.617708in}}%
\pgfpathlineto{\pgfqpoint{5.912880in}{2.617844in}}%
\pgfpathlineto{\pgfqpoint{5.917840in}{2.620400in}}%
\pgfpathlineto{\pgfqpoint{5.922800in}{2.625357in}}%
\pgfpathlineto{\pgfqpoint{5.929000in}{2.634877in}}%
\pgfpathlineto{\pgfqpoint{5.936440in}{2.651036in}}%
\pgfpathlineto{\pgfqpoint{5.945120in}{2.676111in}}%
\pgfpathlineto{\pgfqpoint{5.955040in}{2.712350in}}%
\pgfpathlineto{\pgfqpoint{5.966200in}{2.761658in}}%
\pgfpathlineto{\pgfqpoint{5.981080in}{2.838737in}}%
\pgfpathlineto{\pgfqpoint{6.002160in}{2.961905in}}%
\pgfpathlineto{\pgfqpoint{6.036880in}{3.165584in}}%
\pgfpathlineto{\pgfqpoint{6.051760in}{3.240010in}}%
\pgfpathlineto{\pgfqpoint{6.064160in}{3.291464in}}%
\pgfpathlineto{\pgfqpoint{6.074080in}{3.324253in}}%
\pgfpathlineto{\pgfqpoint{6.082760in}{3.346139in}}%
\pgfpathlineto{\pgfqpoint{6.090200in}{3.359517in}}%
\pgfpathlineto{\pgfqpoint{6.096400in}{3.366732in}}%
\pgfpathlineto{\pgfqpoint{6.101360in}{3.369882in}}%
\pgfpathlineto{\pgfqpoint{6.106320in}{3.370682in}}%
\pgfpathlineto{\pgfqpoint{6.111280in}{3.369131in}}%
\pgfpathlineto{\pgfqpoint{6.116240in}{3.365243in}}%
\pgfpathlineto{\pgfqpoint{6.122440in}{3.357139in}}%
\pgfpathlineto{\pgfqpoint{6.129880in}{3.342776in}}%
\pgfpathlineto{\pgfqpoint{6.138560in}{3.319900in}}%
\pgfpathlineto{\pgfqpoint{6.148480in}{3.286256in}}%
\pgfpathlineto{\pgfqpoint{6.159640in}{3.239898in}}%
\pgfpathlineto{\pgfqpoint{6.173280in}{3.173159in}}%
\pgfpathlineto{\pgfqpoint{6.191880in}{3.069917in}}%
\pgfpathlineto{\pgfqpoint{6.235280in}{2.823970in}}%
\pgfpathlineto{\pgfqpoint{6.250160in}{2.754290in}}%
\pgfpathlineto{\pgfqpoint{6.261320in}{2.711254in}}%
\pgfpathlineto{\pgfqpoint{6.271240in}{2.680873in}}%
\pgfpathlineto{\pgfqpoint{6.279920in}{2.660985in}}%
\pgfpathlineto{\pgfqpoint{6.287360in}{2.649208in}}%
\pgfpathlineto{\pgfqpoint{6.293560in}{2.643232in}}%
\pgfpathlineto{\pgfqpoint{6.298520in}{2.641002in}}%
\pgfpathlineto{\pgfqpoint{6.303480in}{2.641053in}}%
\pgfpathlineto{\pgfqpoint{6.308440in}{2.643380in}}%
\pgfpathlineto{\pgfqpoint{6.313400in}{2.647966in}}%
\pgfpathlineto{\pgfqpoint{6.319600in}{2.656823in}}%
\pgfpathlineto{\pgfqpoint{6.327040in}{2.671909in}}%
\pgfpathlineto{\pgfqpoint{6.335720in}{2.695364in}}%
\pgfpathlineto{\pgfqpoint{6.345640in}{2.729308in}}%
\pgfpathlineto{\pgfqpoint{6.356800in}{2.775535in}}%
\pgfpathlineto{\pgfqpoint{6.371680in}{2.847853in}}%
\pgfpathlineto{\pgfqpoint{6.392760in}{2.963499in}}%
\pgfpathlineto{\pgfqpoint{6.427480in}{3.154915in}}%
\pgfpathlineto{\pgfqpoint{6.442360in}{3.224925in}}%
\pgfpathlineto{\pgfqpoint{6.454760in}{3.273358in}}%
\pgfpathlineto{\pgfqpoint{6.464680in}{3.304245in}}%
\pgfpathlineto{\pgfqpoint{6.473360in}{3.324881in}}%
\pgfpathlineto{\pgfqpoint{6.480800in}{3.337514in}}%
\pgfpathlineto{\pgfqpoint{6.487000in}{3.344345in}}%
\pgfpathlineto{\pgfqpoint{6.493200in}{3.347749in}}%
\pgfpathlineto{\pgfqpoint{6.498160in}{3.347987in}}%
\pgfpathlineto{\pgfqpoint{6.503120in}{3.346017in}}%
\pgfpathlineto{\pgfqpoint{6.508080in}{3.341853in}}%
\pgfpathlineto{\pgfqpoint{6.514280in}{3.333612in}}%
\pgfpathlineto{\pgfqpoint{6.521720in}{3.319386in}}%
\pgfpathlineto{\pgfqpoint{6.530400in}{3.297087in}}%
\pgfpathlineto{\pgfqpoint{6.540320in}{3.264635in}}%
\pgfpathlineto{\pgfqpoint{6.551480in}{3.220261in}}%
\pgfpathlineto{\pgfqpoint{6.565120in}{3.156782in}}%
\pgfpathlineto{\pgfqpoint{6.584960in}{3.052415in}}%
\pgfpathlineto{\pgfqpoint{6.624640in}{2.840824in}}%
\pgfpathlineto{\pgfqpoint{6.639520in}{2.774366in}}%
\pgfpathlineto{\pgfqpoint{6.651920in}{2.728919in}}%
\pgfpathlineto{\pgfqpoint{6.661840in}{2.700354in}}%
\pgfpathlineto{\pgfqpoint{6.670520in}{2.681650in}}%
\pgfpathlineto{\pgfqpoint{6.677960in}{2.670572in}}%
\pgfpathlineto{\pgfqpoint{6.684160in}{2.664948in}}%
\pgfpathlineto{\pgfqpoint{6.689120in}{2.662847in}}%
\pgfpathlineto{\pgfqpoint{6.694080in}{2.662890in}}%
\pgfpathlineto{\pgfqpoint{6.699040in}{2.665073in}}%
\pgfpathlineto{\pgfqpoint{6.705240in}{2.670784in}}%
\pgfpathlineto{\pgfqpoint{6.711440in}{2.679749in}}%
\pgfpathlineto{\pgfqpoint{6.718880in}{2.694667in}}%
\pgfpathlineto{\pgfqpoint{6.727560in}{2.717523in}}%
\pgfpathlineto{\pgfqpoint{6.737480in}{2.750263in}}%
\pgfpathlineto{\pgfqpoint{6.749880in}{2.799854in}}%
\pgfpathlineto{\pgfqpoint{6.764760in}{2.869397in}}%
\pgfpathlineto{\pgfqpoint{6.788320in}{2.992346in}}%
\pgfpathlineto{\pgfqpoint{6.818080in}{3.145641in}}%
\pgfpathlineto{\pgfqpoint{6.832960in}{3.211387in}}%
\pgfpathlineto{\pgfqpoint{6.845360in}{3.256861in}}%
\pgfpathlineto{\pgfqpoint{6.855280in}{3.285849in}}%
\pgfpathlineto{\pgfqpoint{6.863960in}{3.305207in}}%
\pgfpathlineto{\pgfqpoint{6.871400in}{3.317046in}}%
\pgfpathlineto{\pgfqpoint{6.877600in}{3.323436in}}%
\pgfpathlineto{\pgfqpoint{6.883800in}{3.326606in}}%
\pgfpathlineto{\pgfqpoint{6.888760in}{3.326804in}}%
\pgfpathlineto{\pgfqpoint{6.893720in}{3.324927in}}%
\pgfpathlineto{\pgfqpoint{6.899920in}{3.319685in}}%
\pgfpathlineto{\pgfqpoint{6.906120in}{3.311282in}}%
\pgfpathlineto{\pgfqpoint{6.913560in}{3.297151in}}%
\pgfpathlineto{\pgfqpoint{6.922240in}{3.275358in}}%
\pgfpathlineto{\pgfqpoint{6.932160in}{3.243994in}}%
\pgfpathlineto{\pgfqpoint{6.944560in}{3.196317in}}%
\pgfpathlineto{\pgfqpoint{6.959440in}{3.129254in}}%
\pgfpathlineto{\pgfqpoint{6.981760in}{3.016733in}}%
\pgfpathlineto{\pgfqpoint{7.012760in}{2.861416in}}%
\pgfpathlineto{\pgfqpoint{7.027640in}{2.797312in}}%
\pgfpathlineto{\pgfqpoint{7.040040in}{2.752828in}}%
\pgfpathlineto{\pgfqpoint{7.049960in}{2.724354in}}%
\pgfpathlineto{\pgfqpoint{7.058640in}{2.705236in}}%
\pgfpathlineto{\pgfqpoint{7.066080in}{2.693443in}}%
\pgfpathlineto{\pgfqpoint{7.072280in}{2.686977in}}%
\pgfpathlineto{\pgfqpoint{7.078480in}{2.683630in}}%
\pgfpathlineto{\pgfqpoint{7.083440in}{2.683217in}}%
\pgfpathlineto{\pgfqpoint{7.088400in}{2.684818in}}%
\pgfpathlineto{\pgfqpoint{7.094600in}{2.689630in}}%
\pgfpathlineto{\pgfqpoint{7.100800in}{2.697512in}}%
\pgfpathlineto{\pgfqpoint{7.108240in}{2.710906in}}%
\pgfpathlineto{\pgfqpoint{7.116920in}{2.731698in}}%
\pgfpathlineto{\pgfqpoint{7.126840in}{2.761756in}}%
\pgfpathlineto{\pgfqpoint{7.139240in}{2.807607in}}%
\pgfpathlineto{\pgfqpoint{7.154120in}{2.872291in}}%
\pgfpathlineto{\pgfqpoint{7.176440in}{2.981142in}}%
\pgfpathlineto{\pgfqpoint{7.200000in}{3.097519in}}%
\pgfpathlineto{\pgfqpoint{7.200000in}{3.097519in}}%
\pgfusepath{stroke}%
\end{pgfscope}%
\begin{pgfscope}%
\pgfpathrectangle{\pgfqpoint{1.000000in}{0.600000in}}{\pgfqpoint{6.200000in}{4.800000in}}%
\pgfusepath{clip}%
\pgfsetrectcap%
\pgfsetroundjoin%
\pgfsetlinewidth{1.003750pt}%
\definecolor{currentstroke}{rgb}{0.000000,0.000000,0.000000}%
\pgfsetstrokecolor{currentstroke}%
\pgfsetdash{}{0pt}%
\pgfpathmoveto{\pgfqpoint{1.000000in}{3.837758in}}%
\pgfpathlineto{\pgfqpoint{1.003720in}{3.836318in}}%
\pgfpathlineto{\pgfqpoint{1.007440in}{3.832003in}}%
\pgfpathlineto{\pgfqpoint{1.012400in}{3.821803in}}%
\pgfpathlineto{\pgfqpoint{1.018600in}{3.801988in}}%
\pgfpathlineto{\pgfqpoint{1.024800in}{3.774490in}}%
\pgfpathlineto{\pgfqpoint{1.032240in}{3.731696in}}%
\pgfpathlineto{\pgfqpoint{1.040920in}{3.668972in}}%
\pgfpathlineto{\pgfqpoint{1.052080in}{3.569831in}}%
\pgfpathlineto{\pgfqpoint{1.064480in}{3.438684in}}%
\pgfpathlineto{\pgfqpoint{1.080600in}{3.243228in}}%
\pgfpathlineto{\pgfqpoint{1.141360in}{2.477692in}}%
\pgfpathlineto{\pgfqpoint{1.153760in}{2.359794in}}%
\pgfpathlineto{\pgfqpoint{1.164920in}{2.274826in}}%
\pgfpathlineto{\pgfqpoint{1.173600in}{2.224291in}}%
\pgfpathlineto{\pgfqpoint{1.181040in}{2.192566in}}%
\pgfpathlineto{\pgfqpoint{1.187240in}{2.174616in}}%
\pgfpathlineto{\pgfqpoint{1.192200in}{2.165931in}}%
\pgfpathlineto{\pgfqpoint{1.195920in}{2.162760in}}%
\pgfpathlineto{\pgfqpoint{1.199640in}{2.162468in}}%
\pgfpathlineto{\pgfqpoint{1.203360in}{2.165055in}}%
\pgfpathlineto{\pgfqpoint{1.207080in}{2.170512in}}%
\pgfpathlineto{\pgfqpoint{1.212040in}{2.182224in}}%
\pgfpathlineto{\pgfqpoint{1.218240in}{2.203894in}}%
\pgfpathlineto{\pgfqpoint{1.225680in}{2.239950in}}%
\pgfpathlineto{\pgfqpoint{1.234360in}{2.295281in}}%
\pgfpathlineto{\pgfqpoint{1.244280in}{2.374796in}}%
\pgfpathlineto{\pgfqpoint{1.255440in}{2.482750in}}%
\pgfpathlineto{\pgfqpoint{1.269080in}{2.636616in}}%
\pgfpathlineto{\pgfqpoint{1.288920in}{2.889233in}}%
\pgfpathlineto{\pgfqpoint{1.328600in}{3.401410in}}%
\pgfpathlineto{\pgfqpoint{1.343480in}{3.562437in}}%
\pgfpathlineto{\pgfqpoint{1.355880in}{3.672827in}}%
\pgfpathlineto{\pgfqpoint{1.365800in}{3.742549in}}%
\pgfpathlineto{\pgfqpoint{1.374480in}{3.788589in}}%
\pgfpathlineto{\pgfqpoint{1.381920in}{3.816282in}}%
\pgfpathlineto{\pgfqpoint{1.388120in}{3.830783in}}%
\pgfpathlineto{\pgfqpoint{1.393080in}{3.836674in}}%
\pgfpathlineto{\pgfqpoint{1.396800in}{3.837739in}}%
\pgfpathlineto{\pgfqpoint{1.400520in}{3.835924in}}%
\pgfpathlineto{\pgfqpoint{1.404240in}{3.831235in}}%
\pgfpathlineto{\pgfqpoint{1.409200in}{3.820539in}}%
\pgfpathlineto{\pgfqpoint{1.415400in}{3.800116in}}%
\pgfpathlineto{\pgfqpoint{1.422840in}{3.765509in}}%
\pgfpathlineto{\pgfqpoint{1.431520in}{3.711779in}}%
\pgfpathlineto{\pgfqpoint{1.441440in}{3.633940in}}%
\pgfpathlineto{\pgfqpoint{1.452600in}{3.527623in}}%
\pgfpathlineto{\pgfqpoint{1.466240in}{3.375320in}}%
\pgfpathlineto{\pgfqpoint{1.484840in}{3.140313in}}%
\pgfpathlineto{\pgfqpoint{1.528240in}{2.581367in}}%
\pgfpathlineto{\pgfqpoint{1.543120in}{2.423118in}}%
\pgfpathlineto{\pgfqpoint{1.554280in}{2.325261in}}%
\pgfpathlineto{\pgfqpoint{1.564200in}{2.255968in}}%
\pgfpathlineto{\pgfqpoint{1.572880in}{2.210332in}}%
\pgfpathlineto{\pgfqpoint{1.580320in}{2.183001in}}%
\pgfpathlineto{\pgfqpoint{1.586520in}{2.168809in}}%
\pgfpathlineto{\pgfqpoint{1.591480in}{2.163166in}}%
\pgfpathlineto{\pgfqpoint{1.595200in}{2.162289in}}%
\pgfpathlineto{\pgfqpoint{1.598920in}{2.164292in}}%
\pgfpathlineto{\pgfqpoint{1.602640in}{2.169168in}}%
\pgfpathlineto{\pgfqpoint{1.607600in}{2.180111in}}%
\pgfpathlineto{\pgfqpoint{1.613800in}{2.200837in}}%
\pgfpathlineto{\pgfqpoint{1.621240in}{2.235797in}}%
\pgfpathlineto{\pgfqpoint{1.629920in}{2.289917in}}%
\pgfpathlineto{\pgfqpoint{1.639840in}{2.368166in}}%
\pgfpathlineto{\pgfqpoint{1.651000in}{2.474884in}}%
\pgfpathlineto{\pgfqpoint{1.664640in}{2.627571in}}%
\pgfpathlineto{\pgfqpoint{1.683240in}{2.862880in}}%
\pgfpathlineto{\pgfqpoint{1.726640in}{3.421428in}}%
\pgfpathlineto{\pgfqpoint{1.741520in}{3.579215in}}%
\pgfpathlineto{\pgfqpoint{1.752680in}{3.676641in}}%
\pgfpathlineto{\pgfqpoint{1.762600in}{3.745504in}}%
\pgfpathlineto{\pgfqpoint{1.771280in}{3.790735in}}%
\pgfpathlineto{\pgfqpoint{1.778720in}{3.817704in}}%
\pgfpathlineto{\pgfqpoint{1.784920in}{3.831588in}}%
\pgfpathlineto{\pgfqpoint{1.789880in}{3.836981in}}%
\pgfpathlineto{\pgfqpoint{1.793600in}{3.837670in}}%
\pgfpathlineto{\pgfqpoint{1.797320in}{3.835480in}}%
\pgfpathlineto{\pgfqpoint{1.801040in}{3.830418in}}%
\pgfpathlineto{\pgfqpoint{1.806000in}{3.819228in}}%
\pgfpathlineto{\pgfqpoint{1.812200in}{3.798198in}}%
\pgfpathlineto{\pgfqpoint{1.819640in}{3.762885in}}%
\pgfpathlineto{\pgfqpoint{1.828320in}{3.708375in}}%
\pgfpathlineto{\pgfqpoint{1.838240in}{3.629719in}}%
\pgfpathlineto{\pgfqpoint{1.849400in}{3.522602in}}%
\pgfpathlineto{\pgfqpoint{1.863040in}{3.369533in}}%
\pgfpathlineto{\pgfqpoint{1.881640in}{3.133926in}}%
\pgfpathlineto{\pgfqpoint{1.923800in}{2.590109in}}%
\pgfpathlineto{\pgfqpoint{1.938680in}{2.430432in}}%
\pgfpathlineto{\pgfqpoint{1.951080in}{2.321466in}}%
\pgfpathlineto{\pgfqpoint{1.961000in}{2.253036in}}%
\pgfpathlineto{\pgfqpoint{1.969680in}{2.208210in}}%
\pgfpathlineto{\pgfqpoint{1.977120in}{2.181603in}}%
\pgfpathlineto{\pgfqpoint{1.983320in}{2.168027in}}%
\pgfpathlineto{\pgfqpoint{1.988280in}{2.162884in}}%
\pgfpathlineto{\pgfqpoint{1.992000in}{2.162382in}}%
\pgfpathlineto{\pgfqpoint{1.995720in}{2.164760in}}%
\pgfpathlineto{\pgfqpoint{1.999440in}{2.170009in}}%
\pgfpathlineto{\pgfqpoint{2.004400in}{2.181446in}}%
\pgfpathlineto{\pgfqpoint{2.010600in}{2.202779in}}%
\pgfpathlineto{\pgfqpoint{2.018040in}{2.238444in}}%
\pgfpathlineto{\pgfqpoint{2.026720in}{2.293342in}}%
\pgfpathlineto{\pgfqpoint{2.036640in}{2.372406in}}%
\pgfpathlineto{\pgfqpoint{2.047800in}{2.479920in}}%
\pgfpathlineto{\pgfqpoint{2.061440in}{2.633368in}}%
\pgfpathlineto{\pgfqpoint{2.081280in}{2.885654in}}%
\pgfpathlineto{\pgfqpoint{2.120960in}{3.398249in}}%
\pgfpathlineto{\pgfqpoint{2.135840in}{3.559776in}}%
\pgfpathlineto{\pgfqpoint{2.148240in}{3.670693in}}%
\pgfpathlineto{\pgfqpoint{2.158160in}{3.740894in}}%
\pgfpathlineto{\pgfqpoint{2.166840in}{3.787385in}}%
\pgfpathlineto{\pgfqpoint{2.174280in}{3.815480in}}%
\pgfpathlineto{\pgfqpoint{2.180480in}{3.830325in}}%
\pgfpathlineto{\pgfqpoint{2.185440in}{3.836495in}}%
\pgfpathlineto{\pgfqpoint{2.189160in}{3.837768in}}%
\pgfpathlineto{\pgfqpoint{2.192880in}{3.836163in}}%
\pgfpathlineto{\pgfqpoint{2.196600in}{3.831682in}}%
\pgfpathlineto{\pgfqpoint{2.201560in}{3.821262in}}%
\pgfpathlineto{\pgfqpoint{2.207760in}{3.801178in}}%
\pgfpathlineto{\pgfqpoint{2.215200in}{3.766965in}}%
\pgfpathlineto{\pgfqpoint{2.223880in}{3.713671in}}%
\pgfpathlineto{\pgfqpoint{2.233800in}{3.636288in}}%
\pgfpathlineto{\pgfqpoint{2.244960in}{3.530417in}}%
\pgfpathlineto{\pgfqpoint{2.258600in}{3.378543in}}%
\pgfpathlineto{\pgfqpoint{2.277200in}{3.143874in}}%
\pgfpathlineto{\pgfqpoint{2.321840in}{2.570226in}}%
\pgfpathlineto{\pgfqpoint{2.335480in}{2.425722in}}%
\pgfpathlineto{\pgfqpoint{2.346640in}{2.327384in}}%
\pgfpathlineto{\pgfqpoint{2.356560in}{2.257611in}}%
\pgfpathlineto{\pgfqpoint{2.365240in}{2.211524in}}%
\pgfpathlineto{\pgfqpoint{2.372680in}{2.183789in}}%
\pgfpathlineto{\pgfqpoint{2.378880in}{2.169253in}}%
\pgfpathlineto{\pgfqpoint{2.383840in}{2.163332in}}%
\pgfpathlineto{\pgfqpoint{2.387560in}{2.162246in}}%
\pgfpathlineto{\pgfqpoint{2.391280in}{2.164039in}}%
\pgfpathlineto{\pgfqpoint{2.395000in}{2.168707in}}%
\pgfpathlineto{\pgfqpoint{2.399960in}{2.179374in}}%
\pgfpathlineto{\pgfqpoint{2.406160in}{2.199762in}}%
\pgfpathlineto{\pgfqpoint{2.413600in}{2.234329in}}%
\pgfpathlineto{\pgfqpoint{2.422280in}{2.288014in}}%
\pgfpathlineto{\pgfqpoint{2.432200in}{2.365807in}}%
\pgfpathlineto{\pgfqpoint{2.443360in}{2.472079in}}%
\pgfpathlineto{\pgfqpoint{2.457000in}{2.624340in}}%
\pgfpathlineto{\pgfqpoint{2.475600in}{2.859315in}}%
\pgfpathlineto{\pgfqpoint{2.519000in}{3.418315in}}%
\pgfpathlineto{\pgfqpoint{2.533880in}{3.576620in}}%
\pgfpathlineto{\pgfqpoint{2.545040in}{3.674529in}}%
\pgfpathlineto{\pgfqpoint{2.554960in}{3.743872in}}%
\pgfpathlineto{\pgfqpoint{2.563640in}{3.789556in}}%
\pgfpathlineto{\pgfqpoint{2.571080in}{3.816929in}}%
\pgfpathlineto{\pgfqpoint{2.577280in}{3.831158in}}%
\pgfpathlineto{\pgfqpoint{2.582240in}{3.836829in}}%
\pgfpathlineto{\pgfqpoint{2.585960in}{3.837727in}}%
\pgfpathlineto{\pgfqpoint{2.589680in}{3.835747in}}%
\pgfpathlineto{\pgfqpoint{2.593400in}{3.830892in}}%
\pgfpathlineto{\pgfqpoint{2.598360in}{3.819978in}}%
\pgfpathlineto{\pgfqpoint{2.604560in}{3.799287in}}%
\pgfpathlineto{\pgfqpoint{2.612000in}{3.764367in}}%
\pgfpathlineto{\pgfqpoint{2.620680in}{3.710291in}}%
\pgfpathlineto{\pgfqpoint{2.630600in}{3.632089in}}%
\pgfpathlineto{\pgfqpoint{2.641760in}{3.525416in}}%
\pgfpathlineto{\pgfqpoint{2.655400in}{3.372771in}}%
\pgfpathlineto{\pgfqpoint{2.674000in}{3.137494in}}%
\pgfpathlineto{\pgfqpoint{2.717400in}{2.578890in}}%
\pgfpathlineto{\pgfqpoint{2.732280in}{2.421047in}}%
\pgfpathlineto{\pgfqpoint{2.743440in}{2.323569in}}%
\pgfpathlineto{\pgfqpoint{2.753360in}{2.254656in}}%
\pgfpathlineto{\pgfqpoint{2.762040in}{2.209376in}}%
\pgfpathlineto{\pgfqpoint{2.769480in}{2.182365in}}%
\pgfpathlineto{\pgfqpoint{2.775680in}{2.168444in}}%
\pgfpathlineto{\pgfqpoint{2.780640in}{2.163022in}}%
\pgfpathlineto{\pgfqpoint{2.784360in}{2.162311in}}%
\pgfpathlineto{\pgfqpoint{2.788080in}{2.164480in}}%
\pgfpathlineto{\pgfqpoint{2.791800in}{2.169521in}}%
\pgfpathlineto{\pgfqpoint{2.796760in}{2.180682in}}%
\pgfpathlineto{\pgfqpoint{2.802960in}{2.201677in}}%
\pgfpathlineto{\pgfqpoint{2.810400in}{2.236949in}}%
\pgfpathlineto{\pgfqpoint{2.819080in}{2.291414in}}%
\pgfpathlineto{\pgfqpoint{2.829000in}{2.370024in}}%
\pgfpathlineto{\pgfqpoint{2.840160in}{2.477096in}}%
\pgfpathlineto{\pgfqpoint{2.853800in}{2.630123in}}%
\pgfpathlineto{\pgfqpoint{2.872400in}{2.865699in}}%
\pgfpathlineto{\pgfqpoint{2.914560in}{3.409569in}}%
\pgfpathlineto{\pgfqpoint{2.929440in}{3.569300in}}%
\pgfpathlineto{\pgfqpoint{2.941840in}{3.678324in}}%
\pgfpathlineto{\pgfqpoint{2.951760in}{3.746806in}}%
\pgfpathlineto{\pgfqpoint{2.960440in}{3.791680in}}%
\pgfpathlineto{\pgfqpoint{2.967880in}{3.818330in}}%
\pgfpathlineto{\pgfqpoint{2.974080in}{3.831941in}}%
\pgfpathlineto{\pgfqpoint{2.979040in}{3.837114in}}%
\pgfpathlineto{\pgfqpoint{2.982760in}{3.837638in}}%
\pgfpathlineto{\pgfqpoint{2.986480in}{3.835282in}}%
\pgfpathlineto{\pgfqpoint{2.990200in}{3.830054in}}%
\pgfpathlineto{\pgfqpoint{2.995160in}{3.818646in}}%
\pgfpathlineto{\pgfqpoint{3.001360in}{3.797349in}}%
\pgfpathlineto{\pgfqpoint{3.008800in}{3.761724in}}%
\pgfpathlineto{\pgfqpoint{3.017480in}{3.706870in}}%
\pgfpathlineto{\pgfqpoint{3.027400in}{3.627853in}}%
\pgfpathlineto{\pgfqpoint{3.038560in}{3.520384in}}%
\pgfpathlineto{\pgfqpoint{3.052200in}{3.366978in}}%
\pgfpathlineto{\pgfqpoint{3.072040in}{3.114724in}}%
\pgfpathlineto{\pgfqpoint{3.111720in}{2.602078in}}%
\pgfpathlineto{\pgfqpoint{3.126600in}{2.440496in}}%
\pgfpathlineto{\pgfqpoint{3.139000in}{2.329523in}}%
\pgfpathlineto{\pgfqpoint{3.148920in}{2.259270in}}%
\pgfpathlineto{\pgfqpoint{3.157600in}{2.212732in}}%
\pgfpathlineto{\pgfqpoint{3.165040in}{2.184594in}}%
\pgfpathlineto{\pgfqpoint{3.171240in}{2.169712in}}%
\pgfpathlineto{\pgfqpoint{3.176200in}{2.163513in}}%
\pgfpathlineto{\pgfqpoint{3.179920in}{2.162218in}}%
\pgfpathlineto{\pgfqpoint{3.183640in}{2.163801in}}%
\pgfpathlineto{\pgfqpoint{3.187360in}{2.168260in}}%
\pgfpathlineto{\pgfqpoint{3.192320in}{2.178651in}}%
\pgfpathlineto{\pgfqpoint{3.198520in}{2.198700in}}%
\pgfpathlineto{\pgfqpoint{3.205960in}{2.232872in}}%
\pgfpathlineto{\pgfqpoint{3.214640in}{2.286121in}}%
\pgfpathlineto{\pgfqpoint{3.224560in}{2.363456in}}%
\pgfpathlineto{\pgfqpoint{3.235720in}{2.469282in}}%
\pgfpathlineto{\pgfqpoint{3.249360in}{2.621113in}}%
\pgfpathlineto{\pgfqpoint{3.267960in}{2.855749in}}%
\pgfpathlineto{\pgfqpoint{3.312600in}{3.429455in}}%
\pgfpathlineto{\pgfqpoint{3.326240in}{3.574011in}}%
\pgfpathlineto{\pgfqpoint{3.337400in}{3.672400in}}%
\pgfpathlineto{\pgfqpoint{3.347320in}{3.742225in}}%
\pgfpathlineto{\pgfqpoint{3.356000in}{3.788360in}}%
\pgfpathlineto{\pgfqpoint{3.363440in}{3.816138in}}%
\pgfpathlineto{\pgfqpoint{3.369640in}{3.830711in}}%
\pgfpathlineto{\pgfqpoint{3.374600in}{3.836661in}}%
\pgfpathlineto{\pgfqpoint{3.378320in}{3.837769in}}%
\pgfpathlineto{\pgfqpoint{3.382040in}{3.835998in}}%
\pgfpathlineto{\pgfqpoint{3.385760in}{3.831353in}}%
\pgfpathlineto{\pgfqpoint{3.390720in}{3.820714in}}%
\pgfpathlineto{\pgfqpoint{3.396920in}{3.800362in}}%
\pgfpathlineto{\pgfqpoint{3.404360in}{3.765836in}}%
\pgfpathlineto{\pgfqpoint{3.413040in}{3.712196in}}%
\pgfpathlineto{\pgfqpoint{3.422960in}{3.634451in}}%
\pgfpathlineto{\pgfqpoint{3.434120in}{3.528224in}}%
\pgfpathlineto{\pgfqpoint{3.447760in}{3.376006in}}%
\pgfpathlineto{\pgfqpoint{3.466360in}{3.141063in}}%
\pgfpathlineto{\pgfqpoint{3.509760in}{2.582009in}}%
\pgfpathlineto{\pgfqpoint{3.524640in}{2.423648in}}%
\pgfpathlineto{\pgfqpoint{3.535800in}{2.325686in}}%
\pgfpathlineto{\pgfqpoint{3.545720in}{2.256291in}}%
\pgfpathlineto{\pgfqpoint{3.554400in}{2.210559in}}%
\pgfpathlineto{\pgfqpoint{3.561840in}{2.183143in}}%
\pgfpathlineto{\pgfqpoint{3.568040in}{2.168877in}}%
\pgfpathlineto{\pgfqpoint{3.573000in}{2.163177in}}%
\pgfpathlineto{\pgfqpoint{3.576720in}{2.162256in}}%
\pgfpathlineto{\pgfqpoint{3.580440in}{2.164214in}}%
\pgfpathlineto{\pgfqpoint{3.584160in}{2.169047in}}%
\pgfpathlineto{\pgfqpoint{3.589120in}{2.179932in}}%
\pgfpathlineto{\pgfqpoint{3.595320in}{2.200588in}}%
\pgfpathlineto{\pgfqpoint{3.602760in}{2.235466in}}%
\pgfpathlineto{\pgfqpoint{3.611440in}{2.289497in}}%
\pgfpathlineto{\pgfqpoint{3.621360in}{2.367651in}}%
\pgfpathlineto{\pgfqpoint{3.632520in}{2.474279in}}%
\pgfpathlineto{\pgfqpoint{3.646160in}{2.626881in}}%
\pgfpathlineto{\pgfqpoint{3.664760in}{2.862126in}}%
\pgfpathlineto{\pgfqpoint{3.708160in}{3.420786in}}%
\pgfpathlineto{\pgfqpoint{3.723040in}{3.578686in}}%
\pgfpathlineto{\pgfqpoint{3.734200in}{3.676217in}}%
\pgfpathlineto{\pgfqpoint{3.744120in}{3.745182in}}%
\pgfpathlineto{\pgfqpoint{3.752800in}{3.790510in}}%
\pgfpathlineto{\pgfqpoint{3.760240in}{3.817565in}}%
\pgfpathlineto{\pgfqpoint{3.766440in}{3.831522in}}%
\pgfpathlineto{\pgfqpoint{3.771400in}{3.836974in}}%
\pgfpathlineto{\pgfqpoint{3.775120in}{3.837707in}}%
\pgfpathlineto{\pgfqpoint{3.778840in}{3.835561in}}%
\pgfpathlineto{\pgfqpoint{3.782560in}{3.830542in}}%
\pgfpathlineto{\pgfqpoint{3.787520in}{3.819410in}}%
\pgfpathlineto{\pgfqpoint{3.793720in}{3.798451in}}%
\pgfpathlineto{\pgfqpoint{3.801160in}{3.763220in}}%
\pgfpathlineto{\pgfqpoint{3.809840in}{3.708800in}}%
\pgfpathlineto{\pgfqpoint{3.819760in}{3.630237in}}%
\pgfpathlineto{\pgfqpoint{3.830920in}{3.523211in}}%
\pgfpathlineto{\pgfqpoint{3.844560in}{3.370227in}}%
\pgfpathlineto{\pgfqpoint{3.863160in}{3.134683in}}%
\pgfpathlineto{\pgfqpoint{3.905320in}{2.590760in}}%
\pgfpathlineto{\pgfqpoint{3.920200in}{2.430972in}}%
\pgfpathlineto{\pgfqpoint{3.932600in}{2.321891in}}%
\pgfpathlineto{\pgfqpoint{3.942520in}{2.253356in}}%
\pgfpathlineto{\pgfqpoint{3.951200in}{2.208433in}}%
\pgfpathlineto{\pgfqpoint{3.958640in}{2.181740in}}%
\pgfpathlineto{\pgfqpoint{3.964840in}{2.168091in}}%
\pgfpathlineto{\pgfqpoint{3.969800in}{2.162889in}}%
\pgfpathlineto{\pgfqpoint{3.973520in}{2.162343in}}%
\pgfpathlineto{\pgfqpoint{3.977240in}{2.164676in}}%
\pgfpathlineto{\pgfqpoint{3.980960in}{2.169881in}}%
\pgfpathlineto{\pgfqpoint{3.985920in}{2.181260in}}%
\pgfpathlineto{\pgfqpoint{3.992120in}{2.202522in}}%
\pgfpathlineto{\pgfqpoint{3.999560in}{2.238105in}}%
\pgfpathlineto{\pgfqpoint{4.008240in}{2.292914in}}%
\pgfpathlineto{\pgfqpoint{4.018160in}{2.371883in}}%
\pgfpathlineto{\pgfqpoint{4.029320in}{2.479307in}}%
\pgfpathlineto{\pgfqpoint{4.042960in}{2.632670in}}%
\pgfpathlineto{\pgfqpoint{4.062800in}{2.884892in}}%
\pgfpathlineto{\pgfqpoint{4.102480in}{3.397589in}}%
\pgfpathlineto{\pgfqpoint{4.117360in}{3.559226in}}%
\pgfpathlineto{\pgfqpoint{4.129760in}{3.670257in}}%
\pgfpathlineto{\pgfqpoint{4.139680in}{3.740562in}}%
\pgfpathlineto{\pgfqpoint{4.148360in}{3.787149in}}%
\pgfpathlineto{\pgfqpoint{4.155800in}{3.815331in}}%
\pgfpathlineto{\pgfqpoint{4.162000in}{3.830249in}}%
\pgfpathlineto{\pgfqpoint{4.166960in}{3.836478in}}%
\pgfpathlineto{\pgfqpoint{4.170680in}{3.837796in}}%
\pgfpathlineto{\pgfqpoint{4.174400in}{3.836235in}}%
\pgfpathlineto{\pgfqpoint{4.178120in}{3.831798in}}%
\pgfpathlineto{\pgfqpoint{4.183080in}{3.821437in}}%
\pgfpathlineto{\pgfqpoint{4.189280in}{3.801424in}}%
\pgfpathlineto{\pgfqpoint{4.196720in}{3.767294in}}%
\pgfpathlineto{\pgfqpoint{4.205400in}{3.714091in}}%
\pgfpathlineto{\pgfqpoint{4.215320in}{3.636803in}}%
\pgfpathlineto{\pgfqpoint{4.226480in}{3.531025in}}%
\pgfpathlineto{\pgfqpoint{4.240120in}{3.379238in}}%
\pgfpathlineto{\pgfqpoint{4.258720in}{3.144635in}}%
\pgfpathlineto{\pgfqpoint{4.303360in}{2.570871in}}%
\pgfpathlineto{\pgfqpoint{4.317000in}{2.426262in}}%
\pgfpathlineto{\pgfqpoint{4.328160in}{2.327819in}}%
\pgfpathlineto{\pgfqpoint{4.338080in}{2.257942in}}%
\pgfpathlineto{\pgfqpoint{4.346760in}{2.211758in}}%
\pgfpathlineto{\pgfqpoint{4.354200in}{2.183937in}}%
\pgfpathlineto{\pgfqpoint{4.360400in}{2.169326in}}%
\pgfpathlineto{\pgfqpoint{4.365360in}{2.163346in}}%
\pgfpathlineto{\pgfqpoint{4.369080in}{2.162215in}}%
\pgfpathlineto{\pgfqpoint{4.372800in}{2.163964in}}%
\pgfpathlineto{\pgfqpoint{4.376520in}{2.168587in}}%
\pgfpathlineto{\pgfqpoint{4.381480in}{2.179196in}}%
\pgfpathlineto{\pgfqpoint{4.387680in}{2.199512in}}%
\pgfpathlineto{\pgfqpoint{4.395120in}{2.233996in}}%
\pgfpathlineto{\pgfqpoint{4.403800in}{2.287590in}}%
\pgfpathlineto{\pgfqpoint{4.413720in}{2.365288in}}%
\pgfpathlineto{\pgfqpoint{4.424880in}{2.471468in}}%
\pgfpathlineto{\pgfqpoint{4.438520in}{2.623642in}}%
\pgfpathlineto{\pgfqpoint{4.457120in}{2.858551in}}%
\pgfpathlineto{\pgfqpoint{4.500520in}{3.417661in}}%
\pgfpathlineto{\pgfqpoint{4.515400in}{3.576080in}}%
\pgfpathlineto{\pgfqpoint{4.526560in}{3.674094in}}%
\pgfpathlineto{\pgfqpoint{4.536480in}{3.743542in}}%
\pgfpathlineto{\pgfqpoint{4.545160in}{3.789324in}}%
\pgfpathlineto{\pgfqpoint{4.552600in}{3.816784in}}%
\pgfpathlineto{\pgfqpoint{4.558800in}{3.831087in}}%
\pgfpathlineto{\pgfqpoint{4.563760in}{3.836817in}}%
\pgfpathlineto{\pgfqpoint{4.567480in}{3.837761in}}%
\pgfpathlineto{\pgfqpoint{4.571200in}{3.835825in}}%
\pgfpathlineto{\pgfqpoint{4.574920in}{3.831015in}}%
\pgfpathlineto{\pgfqpoint{4.579880in}{3.820159in}}%
\pgfpathlineto{\pgfqpoint{4.586080in}{3.799540in}}%
\pgfpathlineto{\pgfqpoint{4.593520in}{3.764703in}}%
\pgfpathlineto{\pgfqpoint{4.602200in}{3.710719in}}%
\pgfpathlineto{\pgfqpoint{4.612120in}{3.632612in}}%
\pgfpathlineto{\pgfqpoint{4.623280in}{3.526031in}}%
\pgfpathlineto{\pgfqpoint{4.636920in}{3.373474in}}%
\pgfpathlineto{\pgfqpoint{4.655520in}{3.138261in}}%
\pgfpathlineto{\pgfqpoint{4.698920in}{2.579545in}}%
\pgfpathlineto{\pgfqpoint{4.713800in}{2.421587in}}%
\pgfpathlineto{\pgfqpoint{4.724960in}{2.324002in}}%
\pgfpathlineto{\pgfqpoint{4.734880in}{2.254984in}}%
\pgfpathlineto{\pgfqpoint{4.743560in}{2.209607in}}%
\pgfpathlineto{\pgfqpoint{4.751000in}{2.182507in}}%
\pgfpathlineto{\pgfqpoint{4.757200in}{2.168513in}}%
\pgfpathlineto{\pgfqpoint{4.762160in}{2.163031in}}%
\pgfpathlineto{\pgfqpoint{4.765880in}{2.162275in}}%
\pgfpathlineto{\pgfqpoint{4.769600in}{2.164398in}}%
\pgfpathlineto{\pgfqpoint{4.773320in}{2.169395in}}%
\pgfpathlineto{\pgfqpoint{4.778280in}{2.180497in}}%
\pgfpathlineto{\pgfqpoint{4.784480in}{2.201420in}}%
\pgfpathlineto{\pgfqpoint{4.791920in}{2.236609in}}%
\pgfpathlineto{\pgfqpoint{4.800600in}{2.290982in}}%
\pgfpathlineto{\pgfqpoint{4.810520in}{2.369497in}}%
\pgfpathlineto{\pgfqpoint{4.821680in}{2.476477in}}%
\pgfpathlineto{\pgfqpoint{4.835320in}{2.629417in}}%
\pgfpathlineto{\pgfqpoint{4.853920in}{2.864929in}}%
\pgfpathlineto{\pgfqpoint{4.897320in}{3.423243in}}%
\pgfpathlineto{\pgfqpoint{4.912200in}{3.580738in}}%
\pgfpathlineto{\pgfqpoint{4.923360in}{3.677891in}}%
\pgfpathlineto{\pgfqpoint{4.933280in}{3.746479in}}%
\pgfpathlineto{\pgfqpoint{4.941960in}{3.791452in}}%
\pgfpathlineto{\pgfqpoint{4.949400in}{3.818189in}}%
\pgfpathlineto{\pgfqpoint{4.955600in}{3.831876in}}%
\pgfpathlineto{\pgfqpoint{4.960560in}{3.837108in}}%
\pgfpathlineto{\pgfqpoint{4.964280in}{3.837677in}}%
\pgfpathlineto{\pgfqpoint{4.968000in}{3.835366in}}%
\pgfpathlineto{\pgfqpoint{4.971720in}{3.830183in}}%
\pgfpathlineto{\pgfqpoint{4.976680in}{3.818835in}}%
\pgfpathlineto{\pgfqpoint{4.982880in}{3.797609in}}%
\pgfpathlineto{\pgfqpoint{4.990320in}{3.762068in}}%
\pgfpathlineto{\pgfqpoint{4.999000in}{3.707306in}}%
\pgfpathlineto{\pgfqpoint{5.008920in}{3.628384in}}%
\pgfpathlineto{\pgfqpoint{5.020080in}{3.521007in}}%
\pgfpathlineto{\pgfqpoint{5.033720in}{3.367688in}}%
\pgfpathlineto{\pgfqpoint{5.053560in}{3.115499in}}%
\pgfpathlineto{\pgfqpoint{5.093240in}{2.602750in}}%
\pgfpathlineto{\pgfqpoint{5.108120in}{2.441057in}}%
\pgfpathlineto{\pgfqpoint{5.120520in}{2.329967in}}%
\pgfpathlineto{\pgfqpoint{5.130440in}{2.259610in}}%
\pgfpathlineto{\pgfqpoint{5.139120in}{2.212973in}}%
\pgfpathlineto{\pgfqpoint{5.146560in}{2.184747in}}%
\pgfpathlineto{\pgfqpoint{5.152760in}{2.169791in}}%
\pgfpathlineto{\pgfqpoint{5.157720in}{2.163531in}}%
\pgfpathlineto{\pgfqpoint{5.161440in}{2.162190in}}%
\pgfpathlineto{\pgfqpoint{5.165160in}{2.163729in}}%
\pgfpathlineto{\pgfqpoint{5.168880in}{2.168143in}}%
\pgfpathlineto{\pgfqpoint{5.173840in}{2.178474in}}%
\pgfpathlineto{\pgfqpoint{5.180040in}{2.198450in}}%
\pgfpathlineto{\pgfqpoint{5.187480in}{2.232538in}}%
\pgfpathlineto{\pgfqpoint{5.196160in}{2.285694in}}%
\pgfpathlineto{\pgfqpoint{5.206080in}{2.362932in}}%
\pgfpathlineto{\pgfqpoint{5.217240in}{2.468664in}}%
\pgfpathlineto{\pgfqpoint{5.230880in}{2.620406in}}%
\pgfpathlineto{\pgfqpoint{5.249480in}{2.854975in}}%
\pgfpathlineto{\pgfqpoint{5.294120in}{3.428798in}}%
\pgfpathlineto{\pgfqpoint{5.309000in}{3.585360in}}%
\pgfpathlineto{\pgfqpoint{5.320160in}{3.681647in}}%
\pgfpathlineto{\pgfqpoint{5.330080in}{3.749371in}}%
\pgfpathlineto{\pgfqpoint{5.338760in}{3.793533in}}%
\pgfpathlineto{\pgfqpoint{5.346200in}{3.819546in}}%
\pgfpathlineto{\pgfqpoint{5.351160in}{3.830635in}}%
\pgfpathlineto{\pgfqpoint{5.356120in}{3.836646in}}%
\pgfpathlineto{\pgfqpoint{5.359840in}{3.837800in}}%
\pgfpathlineto{\pgfqpoint{5.363560in}{3.836074in}}%
\pgfpathlineto{\pgfqpoint{5.367280in}{3.831473in}}%
\pgfpathlineto{\pgfqpoint{5.372240in}{3.820895in}}%
\pgfpathlineto{\pgfqpoint{5.378440in}{3.800615in}}%
\pgfpathlineto{\pgfqpoint{5.385880in}{3.766174in}}%
\pgfpathlineto{\pgfqpoint{5.394560in}{3.712627in}}%
\pgfpathlineto{\pgfqpoint{5.404480in}{3.634979in}}%
\pgfpathlineto{\pgfqpoint{5.415640in}{3.528846in}}%
\pgfpathlineto{\pgfqpoint{5.429280in}{3.376717in}}%
\pgfpathlineto{\pgfqpoint{5.447880in}{3.141841in}}%
\pgfpathlineto{\pgfqpoint{5.491280in}{2.582675in}}%
\pgfpathlineto{\pgfqpoint{5.506160in}{2.424197in}}%
\pgfpathlineto{\pgfqpoint{5.517320in}{2.326129in}}%
\pgfpathlineto{\pgfqpoint{5.527240in}{2.256627in}}%
\pgfpathlineto{\pgfqpoint{5.535920in}{2.210796in}}%
\pgfpathlineto{\pgfqpoint{5.543360in}{2.183291in}}%
\pgfpathlineto{\pgfqpoint{5.549560in}{2.168950in}}%
\pgfpathlineto{\pgfqpoint{5.554520in}{2.163189in}}%
\pgfpathlineto{\pgfqpoint{5.558240in}{2.162222in}}%
\pgfpathlineto{\pgfqpoint{5.561960in}{2.164136in}}%
\pgfpathlineto{\pgfqpoint{5.565680in}{2.168923in}}%
\pgfpathlineto{\pgfqpoint{5.570640in}{2.179748in}}%
\pgfpathlineto{\pgfqpoint{5.576840in}{2.200331in}}%
\pgfpathlineto{\pgfqpoint{5.584280in}{2.235125in}}%
\pgfpathlineto{\pgfqpoint{5.592960in}{2.289062in}}%
\pgfpathlineto{\pgfqpoint{5.602880in}{2.367119in}}%
\pgfpathlineto{\pgfqpoint{5.614040in}{2.473653in}}%
\pgfpathlineto{\pgfqpoint{5.627680in}{2.626166in}}%
\pgfpathlineto{\pgfqpoint{5.646280in}{2.861345in}}%
\pgfpathlineto{\pgfqpoint{5.689680in}{3.420118in}}%
\pgfpathlineto{\pgfqpoint{5.704560in}{3.578136in}}%
\pgfpathlineto{\pgfqpoint{5.715720in}{3.675775in}}%
\pgfpathlineto{\pgfqpoint{5.725640in}{3.744847in}}%
\pgfpathlineto{\pgfqpoint{5.734320in}{3.790275in}}%
\pgfpathlineto{\pgfqpoint{5.741760in}{3.817419in}}%
\pgfpathlineto{\pgfqpoint{5.747960in}{3.831451in}}%
\pgfpathlineto{\pgfqpoint{5.752920in}{3.836964in}}%
\pgfpathlineto{\pgfqpoint{5.756640in}{3.837743in}}%
\pgfpathlineto{\pgfqpoint{5.760360in}{3.835643in}}%
\pgfpathlineto{\pgfqpoint{5.764080in}{3.830669in}}%
\pgfpathlineto{\pgfqpoint{5.769040in}{3.819597in}}%
\pgfpathlineto{\pgfqpoint{5.775240in}{3.798711in}}%
\pgfpathlineto{\pgfqpoint{5.782680in}{3.763565in}}%
\pgfpathlineto{\pgfqpoint{5.791360in}{3.709239in}}%
\pgfpathlineto{\pgfqpoint{5.801280in}{3.630773in}}%
\pgfpathlineto{\pgfqpoint{5.812440in}{3.523841in}}%
\pgfpathlineto{\pgfqpoint{5.826080in}{3.370946in}}%
\pgfpathlineto{\pgfqpoint{5.844680in}{3.135467in}}%
\pgfpathlineto{\pgfqpoint{5.888080in}{2.577094in}}%
\pgfpathlineto{\pgfqpoint{5.902960in}{2.419540in}}%
\pgfpathlineto{\pgfqpoint{5.914120in}{2.322332in}}%
\pgfpathlineto{\pgfqpoint{5.924040in}{2.253690in}}%
\pgfpathlineto{\pgfqpoint{5.932720in}{2.208666in}}%
\pgfpathlineto{\pgfqpoint{5.940160in}{2.181884in}}%
\pgfpathlineto{\pgfqpoint{5.946360in}{2.168159in}}%
\pgfpathlineto{\pgfqpoint{5.951320in}{2.162895in}}%
\pgfpathlineto{\pgfqpoint{5.955040in}{2.162303in}}%
\pgfpathlineto{\pgfqpoint{5.958760in}{2.164591in}}%
\pgfpathlineto{\pgfqpoint{5.962480in}{2.169751in}}%
\pgfpathlineto{\pgfqpoint{5.967440in}{2.181070in}}%
\pgfpathlineto{\pgfqpoint{5.973640in}{2.202258in}}%
\pgfpathlineto{\pgfqpoint{5.981080in}{2.237756in}}%
\pgfpathlineto{\pgfqpoint{5.989760in}{2.292471in}}%
\pgfpathlineto{\pgfqpoint{5.999680in}{2.371343in}}%
\pgfpathlineto{\pgfqpoint{6.010840in}{2.478673in}}%
\pgfpathlineto{\pgfqpoint{6.024480in}{2.631948in}}%
\pgfpathlineto{\pgfqpoint{6.044320in}{2.884103in}}%
\pgfpathlineto{\pgfqpoint{6.084000in}{3.396904in}}%
\pgfpathlineto{\pgfqpoint{6.098880in}{3.558654in}}%
\pgfpathlineto{\pgfqpoint{6.111280in}{3.669804in}}%
\pgfpathlineto{\pgfqpoint{6.121200in}{3.740216in}}%
\pgfpathlineto{\pgfqpoint{6.129880in}{3.786903in}}%
\pgfpathlineto{\pgfqpoint{6.137320in}{3.815174in}}%
\pgfpathlineto{\pgfqpoint{6.143520in}{3.830169in}}%
\pgfpathlineto{\pgfqpoint{6.148480in}{3.836459in}}%
\pgfpathlineto{\pgfqpoint{6.152200in}{3.837823in}}%
\pgfpathlineto{\pgfqpoint{6.155920in}{3.836308in}}%
\pgfpathlineto{\pgfqpoint{6.159640in}{3.831917in}}%
\pgfpathlineto{\pgfqpoint{6.164600in}{3.821616in}}%
\pgfpathlineto{\pgfqpoint{6.170800in}{3.801677in}}%
\pgfpathlineto{\pgfqpoint{6.177000in}{3.774058in}}%
\pgfpathlineto{\pgfqpoint{6.184440in}{3.731124in}}%
\pgfpathlineto{\pgfqpoint{6.193120in}{3.668246in}}%
\pgfpathlineto{\pgfqpoint{6.204280in}{3.568925in}}%
\pgfpathlineto{\pgfqpoint{6.216680in}{3.437612in}}%
\pgfpathlineto{\pgfqpoint{6.232800in}{3.242003in}}%
\pgfpathlineto{\pgfqpoint{6.293560in}{2.476640in}}%
\pgfpathlineto{\pgfqpoint{6.305960in}{2.358912in}}%
\pgfpathlineto{\pgfqpoint{6.317120in}{2.274125in}}%
\pgfpathlineto{\pgfqpoint{6.325800in}{2.223745in}}%
\pgfpathlineto{\pgfqpoint{6.333240in}{2.192160in}}%
\pgfpathlineto{\pgfqpoint{6.339440in}{2.174332in}}%
\pgfpathlineto{\pgfqpoint{6.344400in}{2.165746in}}%
\pgfpathlineto{\pgfqpoint{6.348120in}{2.162650in}}%
\pgfpathlineto{\pgfqpoint{6.351840in}{2.162433in}}%
\pgfpathlineto{\pgfqpoint{6.355560in}{2.165095in}}%
\pgfpathlineto{\pgfqpoint{6.359280in}{2.170627in}}%
\pgfpathlineto{\pgfqpoint{6.364240in}{2.182438in}}%
\pgfpathlineto{\pgfqpoint{6.370440in}{2.204231in}}%
\pgfpathlineto{\pgfqpoint{6.377880in}{2.240431in}}%
\pgfpathlineto{\pgfqpoint{6.386560in}{2.295921in}}%
\pgfpathlineto{\pgfqpoint{6.396480in}{2.375604in}}%
\pgfpathlineto{\pgfqpoint{6.407640in}{2.483723in}}%
\pgfpathlineto{\pgfqpoint{6.421280in}{2.637751in}}%
\pgfpathlineto{\pgfqpoint{6.441120in}{2.890507in}}%
\pgfpathlineto{\pgfqpoint{6.480800in}{3.402573in}}%
\pgfpathlineto{\pgfqpoint{6.495680in}{3.563434in}}%
\pgfpathlineto{\pgfqpoint{6.508080in}{3.673643in}}%
\pgfpathlineto{\pgfqpoint{6.518000in}{3.743199in}}%
\pgfpathlineto{\pgfqpoint{6.526680in}{3.789082in}}%
\pgfpathlineto{\pgfqpoint{6.534120in}{3.816632in}}%
\pgfpathlineto{\pgfqpoint{6.540320in}{3.831011in}}%
\pgfpathlineto{\pgfqpoint{6.545280in}{3.836804in}}%
\pgfpathlineto{\pgfqpoint{6.549000in}{3.837794in}}%
\pgfpathlineto{\pgfqpoint{6.552720in}{3.835904in}}%
\pgfpathlineto{\pgfqpoint{6.556440in}{3.831140in}}%
\pgfpathlineto{\pgfqpoint{6.561400in}{3.820346in}}%
\pgfpathlineto{\pgfqpoint{6.567600in}{3.799800in}}%
\pgfpathlineto{\pgfqpoint{6.575040in}{3.765050in}}%
\pgfpathlineto{\pgfqpoint{6.583720in}{3.711161in}}%
\pgfpathlineto{\pgfqpoint{6.593640in}{3.633153in}}%
\pgfpathlineto{\pgfqpoint{6.604800in}{3.526668in}}%
\pgfpathlineto{\pgfqpoint{6.618440in}{3.374201in}}%
\pgfpathlineto{\pgfqpoint{6.637040in}{3.139056in}}%
\pgfpathlineto{\pgfqpoint{6.680440in}{2.580224in}}%
\pgfpathlineto{\pgfqpoint{6.695320in}{2.422147in}}%
\pgfpathlineto{\pgfqpoint{6.706480in}{2.324453in}}%
\pgfpathlineto{\pgfqpoint{6.716400in}{2.255325in}}%
\pgfpathlineto{\pgfqpoint{6.725080in}{2.209846in}}%
\pgfpathlineto{\pgfqpoint{6.732520in}{2.182657in}}%
\pgfpathlineto{\pgfqpoint{6.738720in}{2.168585in}}%
\pgfpathlineto{\pgfqpoint{6.743680in}{2.163041in}}%
\pgfpathlineto{\pgfqpoint{6.747400in}{2.162239in}}%
\pgfpathlineto{\pgfqpoint{6.751120in}{2.164316in}}%
\pgfpathlineto{\pgfqpoint{6.754840in}{2.169266in}}%
\pgfpathlineto{\pgfqpoint{6.759800in}{2.180307in}}%
\pgfpathlineto{\pgfqpoint{6.766000in}{2.201156in}}%
\pgfpathlineto{\pgfqpoint{6.773440in}{2.236258in}}%
\pgfpathlineto{\pgfqpoint{6.782120in}{2.290537in}}%
\pgfpathlineto{\pgfqpoint{6.792040in}{2.368952in}}%
\pgfpathlineto{\pgfqpoint{6.803200in}{2.475836in}}%
\pgfpathlineto{\pgfqpoint{6.816840in}{2.628686in}}%
\pgfpathlineto{\pgfqpoint{6.835440in}{2.864131in}}%
\pgfpathlineto{\pgfqpoint{6.878840in}{3.422562in}}%
\pgfpathlineto{\pgfqpoint{6.893720in}{3.580178in}}%
\pgfpathlineto{\pgfqpoint{6.904880in}{3.677442in}}%
\pgfpathlineto{\pgfqpoint{6.914800in}{3.746139in}}%
\pgfpathlineto{\pgfqpoint{6.923480in}{3.791214in}}%
\pgfpathlineto{\pgfqpoint{6.930920in}{3.818042in}}%
\pgfpathlineto{\pgfqpoint{6.937120in}{3.831806in}}%
\pgfpathlineto{\pgfqpoint{6.942080in}{3.837101in}}%
\pgfpathlineto{\pgfqpoint{6.945800in}{3.837716in}}%
\pgfpathlineto{\pgfqpoint{6.949520in}{3.835452in}}%
\pgfpathlineto{\pgfqpoint{6.953240in}{3.830316in}}%
\pgfpathlineto{\pgfqpoint{6.958200in}{3.819028in}}%
\pgfpathlineto{\pgfqpoint{6.964400in}{3.797877in}}%
\pgfpathlineto{\pgfqpoint{6.971840in}{3.762423in}}%
\pgfpathlineto{\pgfqpoint{6.980520in}{3.707756in}}%
\pgfpathlineto{\pgfqpoint{6.990440in}{3.628933in}}%
\pgfpathlineto{\pgfqpoint{7.001600in}{3.521652in}}%
\pgfpathlineto{\pgfqpoint{7.015240in}{3.368422in}}%
\pgfpathlineto{\pgfqpoint{7.033840in}{3.132681in}}%
\pgfpathlineto{\pgfqpoint{7.076000in}{2.588971in}}%
\pgfpathlineto{\pgfqpoint{7.090880in}{2.429461in}}%
\pgfpathlineto{\pgfqpoint{7.103280in}{2.320674in}}%
\pgfpathlineto{\pgfqpoint{7.113200in}{2.252408in}}%
\pgfpathlineto{\pgfqpoint{7.121880in}{2.207738in}}%
\pgfpathlineto{\pgfqpoint{7.129320in}{2.181271in}}%
\pgfpathlineto{\pgfqpoint{7.135520in}{2.167815in}}%
\pgfpathlineto{\pgfqpoint{7.140480in}{2.162769in}}%
\pgfpathlineto{\pgfqpoint{7.144200in}{2.162341in}}%
\pgfpathlineto{\pgfqpoint{7.147920in}{2.164792in}}%
\pgfpathlineto{\pgfqpoint{7.151640in}{2.170115in}}%
\pgfpathlineto{\pgfqpoint{7.156600in}{2.181649in}}%
\pgfpathlineto{\pgfqpoint{7.162800in}{2.203102in}}%
\pgfpathlineto{\pgfqpoint{7.170240in}{2.238907in}}%
\pgfpathlineto{\pgfqpoint{7.178920in}{2.293962in}}%
\pgfpathlineto{\pgfqpoint{7.188840in}{2.373190in}}%
\pgfpathlineto{\pgfqpoint{7.200000in}{2.480867in}}%
\pgfpathlineto{\pgfqpoint{7.200000in}{2.480867in}}%
\pgfusepath{stroke}%
\end{pgfscope}%
\begin{pgfscope}%
\pgfsetrectcap%
\pgfsetmiterjoin%
\pgfsetlinewidth{1.003750pt}%
\definecolor{currentstroke}{rgb}{0.000000,0.000000,0.000000}%
\pgfsetstrokecolor{currentstroke}%
\pgfsetdash{}{0pt}%
\pgfpathmoveto{\pgfqpoint{1.000000in}{0.600000in}}%
\pgfpathlineto{\pgfqpoint{1.000000in}{5.400000in}}%
\pgfusepath{stroke}%
\end{pgfscope}%
\begin{pgfscope}%
\pgfsetrectcap%
\pgfsetmiterjoin%
\pgfsetlinewidth{1.003750pt}%
\definecolor{currentstroke}{rgb}{0.000000,0.000000,0.000000}%
\pgfsetstrokecolor{currentstroke}%
\pgfsetdash{}{0pt}%
\pgfpathmoveto{\pgfqpoint{7.200000in}{0.600000in}}%
\pgfpathlineto{\pgfqpoint{7.200000in}{5.400000in}}%
\pgfusepath{stroke}%
\end{pgfscope}%
\begin{pgfscope}%
\pgfsetrectcap%
\pgfsetmiterjoin%
\pgfsetlinewidth{1.003750pt}%
\definecolor{currentstroke}{rgb}{0.000000,0.000000,0.000000}%
\pgfsetstrokecolor{currentstroke}%
\pgfsetdash{}{0pt}%
\pgfpathmoveto{\pgfqpoint{1.000000in}{0.600000in}}%
\pgfpathlineto{\pgfqpoint{7.200000in}{0.600000in}}%
\pgfusepath{stroke}%
\end{pgfscope}%
\begin{pgfscope}%
\pgfsetrectcap%
\pgfsetmiterjoin%
\pgfsetlinewidth{1.003750pt}%
\definecolor{currentstroke}{rgb}{0.000000,0.000000,0.000000}%
\pgfsetstrokecolor{currentstroke}%
\pgfsetdash{}{0pt}%
\pgfpathmoveto{\pgfqpoint{1.000000in}{5.400000in}}%
\pgfpathlineto{\pgfqpoint{7.200000in}{5.400000in}}%
\pgfusepath{stroke}%
\end{pgfscope}%
\begin{pgfscope}%
\pgfpathrectangle{\pgfqpoint{1.000000in}{0.600000in}}{\pgfqpoint{6.200000in}{4.800000in}}%
\pgfusepath{clip}%
\pgfsetbuttcap%
\pgfsetroundjoin%
\pgfsetlinewidth{0.501875pt}%
\definecolor{currentstroke}{rgb}{0.000000,0.000000,0.000000}%
\pgfsetstrokecolor{currentstroke}%
\pgfsetdash{{1.000000pt}{3.000000pt}}{0.000000pt}%
\pgfpathmoveto{\pgfqpoint{1.000000in}{0.600000in}}%
\pgfpathlineto{\pgfqpoint{1.000000in}{5.400000in}}%
\pgfusepath{stroke}%
\end{pgfscope}%
\begin{pgfscope}%
\pgfsetbuttcap%
\pgfsetroundjoin%
\definecolor{currentfill}{rgb}{0.000000,0.000000,0.000000}%
\pgfsetfillcolor{currentfill}%
\pgfsetlinewidth{0.501875pt}%
\definecolor{currentstroke}{rgb}{0.000000,0.000000,0.000000}%
\pgfsetstrokecolor{currentstroke}%
\pgfsetdash{}{0pt}%
\pgfsys@defobject{currentmarker}{\pgfqpoint{0.000000in}{0.000000in}}{\pgfqpoint{0.000000in}{0.055556in}}{%
\pgfpathmoveto{\pgfqpoint{0.000000in}{0.000000in}}%
\pgfpathlineto{\pgfqpoint{0.000000in}{0.055556in}}%
\pgfusepath{stroke,fill}%
}%
\begin{pgfscope}%
\pgfsys@transformshift{1.000000in}{0.600000in}%
\pgfsys@useobject{currentmarker}{}%
\end{pgfscope}%
\end{pgfscope}%
\begin{pgfscope}%
\pgfsetbuttcap%
\pgfsetroundjoin%
\definecolor{currentfill}{rgb}{0.000000,0.000000,0.000000}%
\pgfsetfillcolor{currentfill}%
\pgfsetlinewidth{0.501875pt}%
\definecolor{currentstroke}{rgb}{0.000000,0.000000,0.000000}%
\pgfsetstrokecolor{currentstroke}%
\pgfsetdash{}{0pt}%
\pgfsys@defobject{currentmarker}{\pgfqpoint{0.000000in}{-0.055556in}}{\pgfqpoint{0.000000in}{0.000000in}}{%
\pgfpathmoveto{\pgfqpoint{0.000000in}{0.000000in}}%
\pgfpathlineto{\pgfqpoint{0.000000in}{-0.055556in}}%
\pgfusepath{stroke,fill}%
}%
\begin{pgfscope}%
\pgfsys@transformshift{1.000000in}{5.400000in}%
\pgfsys@useobject{currentmarker}{}%
\end{pgfscope}%
\end{pgfscope}%
\begin{pgfscope}%
\definecolor{textcolor}{rgb}{0.000000,0.000000,0.000000}%
\pgfsetstrokecolor{textcolor}%
\pgfsetfillcolor{textcolor}%
\pgftext[x=1.000000in,y=0.544444in,,top]{\color{textcolor}\rmfamily\fontsize{10.000000}{12.000000}\selectfont \(\displaystyle {0}\)}%
\end{pgfscope}%
\begin{pgfscope}%
\pgfpathrectangle{\pgfqpoint{1.000000in}{0.600000in}}{\pgfqpoint{6.200000in}{4.800000in}}%
\pgfusepath{clip}%
\pgfsetbuttcap%
\pgfsetroundjoin%
\pgfsetlinewidth{0.501875pt}%
\definecolor{currentstroke}{rgb}{0.000000,0.000000,0.000000}%
\pgfsetstrokecolor{currentstroke}%
\pgfsetdash{{1.000000pt}{3.000000pt}}{0.000000pt}%
\pgfpathmoveto{\pgfqpoint{2.240000in}{0.600000in}}%
\pgfpathlineto{\pgfqpoint{2.240000in}{5.400000in}}%
\pgfusepath{stroke}%
\end{pgfscope}%
\begin{pgfscope}%
\pgfsetbuttcap%
\pgfsetroundjoin%
\definecolor{currentfill}{rgb}{0.000000,0.000000,0.000000}%
\pgfsetfillcolor{currentfill}%
\pgfsetlinewidth{0.501875pt}%
\definecolor{currentstroke}{rgb}{0.000000,0.000000,0.000000}%
\pgfsetstrokecolor{currentstroke}%
\pgfsetdash{}{0pt}%
\pgfsys@defobject{currentmarker}{\pgfqpoint{0.000000in}{0.000000in}}{\pgfqpoint{0.000000in}{0.055556in}}{%
\pgfpathmoveto{\pgfqpoint{0.000000in}{0.000000in}}%
\pgfpathlineto{\pgfqpoint{0.000000in}{0.055556in}}%
\pgfusepath{stroke,fill}%
}%
\begin{pgfscope}%
\pgfsys@transformshift{2.240000in}{0.600000in}%
\pgfsys@useobject{currentmarker}{}%
\end{pgfscope}%
\end{pgfscope}%
\begin{pgfscope}%
\pgfsetbuttcap%
\pgfsetroundjoin%
\definecolor{currentfill}{rgb}{0.000000,0.000000,0.000000}%
\pgfsetfillcolor{currentfill}%
\pgfsetlinewidth{0.501875pt}%
\definecolor{currentstroke}{rgb}{0.000000,0.000000,0.000000}%
\pgfsetstrokecolor{currentstroke}%
\pgfsetdash{}{0pt}%
\pgfsys@defobject{currentmarker}{\pgfqpoint{0.000000in}{-0.055556in}}{\pgfqpoint{0.000000in}{0.000000in}}{%
\pgfpathmoveto{\pgfqpoint{0.000000in}{0.000000in}}%
\pgfpathlineto{\pgfqpoint{0.000000in}{-0.055556in}}%
\pgfusepath{stroke,fill}%
}%
\begin{pgfscope}%
\pgfsys@transformshift{2.240000in}{5.400000in}%
\pgfsys@useobject{currentmarker}{}%
\end{pgfscope}%
\end{pgfscope}%
\begin{pgfscope}%
\definecolor{textcolor}{rgb}{0.000000,0.000000,0.000000}%
\pgfsetstrokecolor{textcolor}%
\pgfsetfillcolor{textcolor}%
\pgftext[x=2.240000in,y=0.544444in,,top]{\color{textcolor}\rmfamily\fontsize{10.000000}{12.000000}\selectfont \(\displaystyle {20}\)}%
\end{pgfscope}%
\begin{pgfscope}%
\pgfpathrectangle{\pgfqpoint{1.000000in}{0.600000in}}{\pgfqpoint{6.200000in}{4.800000in}}%
\pgfusepath{clip}%
\pgfsetbuttcap%
\pgfsetroundjoin%
\pgfsetlinewidth{0.501875pt}%
\definecolor{currentstroke}{rgb}{0.000000,0.000000,0.000000}%
\pgfsetstrokecolor{currentstroke}%
\pgfsetdash{{1.000000pt}{3.000000pt}}{0.000000pt}%
\pgfpathmoveto{\pgfqpoint{3.480000in}{0.600000in}}%
\pgfpathlineto{\pgfqpoint{3.480000in}{5.400000in}}%
\pgfusepath{stroke}%
\end{pgfscope}%
\begin{pgfscope}%
\pgfsetbuttcap%
\pgfsetroundjoin%
\definecolor{currentfill}{rgb}{0.000000,0.000000,0.000000}%
\pgfsetfillcolor{currentfill}%
\pgfsetlinewidth{0.501875pt}%
\definecolor{currentstroke}{rgb}{0.000000,0.000000,0.000000}%
\pgfsetstrokecolor{currentstroke}%
\pgfsetdash{}{0pt}%
\pgfsys@defobject{currentmarker}{\pgfqpoint{0.000000in}{0.000000in}}{\pgfqpoint{0.000000in}{0.055556in}}{%
\pgfpathmoveto{\pgfqpoint{0.000000in}{0.000000in}}%
\pgfpathlineto{\pgfqpoint{0.000000in}{0.055556in}}%
\pgfusepath{stroke,fill}%
}%
\begin{pgfscope}%
\pgfsys@transformshift{3.480000in}{0.600000in}%
\pgfsys@useobject{currentmarker}{}%
\end{pgfscope}%
\end{pgfscope}%
\begin{pgfscope}%
\pgfsetbuttcap%
\pgfsetroundjoin%
\definecolor{currentfill}{rgb}{0.000000,0.000000,0.000000}%
\pgfsetfillcolor{currentfill}%
\pgfsetlinewidth{0.501875pt}%
\definecolor{currentstroke}{rgb}{0.000000,0.000000,0.000000}%
\pgfsetstrokecolor{currentstroke}%
\pgfsetdash{}{0pt}%
\pgfsys@defobject{currentmarker}{\pgfqpoint{0.000000in}{-0.055556in}}{\pgfqpoint{0.000000in}{0.000000in}}{%
\pgfpathmoveto{\pgfqpoint{0.000000in}{0.000000in}}%
\pgfpathlineto{\pgfqpoint{0.000000in}{-0.055556in}}%
\pgfusepath{stroke,fill}%
}%
\begin{pgfscope}%
\pgfsys@transformshift{3.480000in}{5.400000in}%
\pgfsys@useobject{currentmarker}{}%
\end{pgfscope}%
\end{pgfscope}%
\begin{pgfscope}%
\definecolor{textcolor}{rgb}{0.000000,0.000000,0.000000}%
\pgfsetstrokecolor{textcolor}%
\pgfsetfillcolor{textcolor}%
\pgftext[x=3.480000in,y=0.544444in,,top]{\color{textcolor}\rmfamily\fontsize{10.000000}{12.000000}\selectfont \(\displaystyle {40}\)}%
\end{pgfscope}%
\begin{pgfscope}%
\pgfpathrectangle{\pgfqpoint{1.000000in}{0.600000in}}{\pgfqpoint{6.200000in}{4.800000in}}%
\pgfusepath{clip}%
\pgfsetbuttcap%
\pgfsetroundjoin%
\pgfsetlinewidth{0.501875pt}%
\definecolor{currentstroke}{rgb}{0.000000,0.000000,0.000000}%
\pgfsetstrokecolor{currentstroke}%
\pgfsetdash{{1.000000pt}{3.000000pt}}{0.000000pt}%
\pgfpathmoveto{\pgfqpoint{4.720000in}{0.600000in}}%
\pgfpathlineto{\pgfqpoint{4.720000in}{5.400000in}}%
\pgfusepath{stroke}%
\end{pgfscope}%
\begin{pgfscope}%
\pgfsetbuttcap%
\pgfsetroundjoin%
\definecolor{currentfill}{rgb}{0.000000,0.000000,0.000000}%
\pgfsetfillcolor{currentfill}%
\pgfsetlinewidth{0.501875pt}%
\definecolor{currentstroke}{rgb}{0.000000,0.000000,0.000000}%
\pgfsetstrokecolor{currentstroke}%
\pgfsetdash{}{0pt}%
\pgfsys@defobject{currentmarker}{\pgfqpoint{0.000000in}{0.000000in}}{\pgfqpoint{0.000000in}{0.055556in}}{%
\pgfpathmoveto{\pgfqpoint{0.000000in}{0.000000in}}%
\pgfpathlineto{\pgfqpoint{0.000000in}{0.055556in}}%
\pgfusepath{stroke,fill}%
}%
\begin{pgfscope}%
\pgfsys@transformshift{4.720000in}{0.600000in}%
\pgfsys@useobject{currentmarker}{}%
\end{pgfscope}%
\end{pgfscope}%
\begin{pgfscope}%
\pgfsetbuttcap%
\pgfsetroundjoin%
\definecolor{currentfill}{rgb}{0.000000,0.000000,0.000000}%
\pgfsetfillcolor{currentfill}%
\pgfsetlinewidth{0.501875pt}%
\definecolor{currentstroke}{rgb}{0.000000,0.000000,0.000000}%
\pgfsetstrokecolor{currentstroke}%
\pgfsetdash{}{0pt}%
\pgfsys@defobject{currentmarker}{\pgfqpoint{0.000000in}{-0.055556in}}{\pgfqpoint{0.000000in}{0.000000in}}{%
\pgfpathmoveto{\pgfqpoint{0.000000in}{0.000000in}}%
\pgfpathlineto{\pgfqpoint{0.000000in}{-0.055556in}}%
\pgfusepath{stroke,fill}%
}%
\begin{pgfscope}%
\pgfsys@transformshift{4.720000in}{5.400000in}%
\pgfsys@useobject{currentmarker}{}%
\end{pgfscope}%
\end{pgfscope}%
\begin{pgfscope}%
\definecolor{textcolor}{rgb}{0.000000,0.000000,0.000000}%
\pgfsetstrokecolor{textcolor}%
\pgfsetfillcolor{textcolor}%
\pgftext[x=4.720000in,y=0.544444in,,top]{\color{textcolor}\rmfamily\fontsize{10.000000}{12.000000}\selectfont \(\displaystyle {60}\)}%
\end{pgfscope}%
\begin{pgfscope}%
\pgfpathrectangle{\pgfqpoint{1.000000in}{0.600000in}}{\pgfqpoint{6.200000in}{4.800000in}}%
\pgfusepath{clip}%
\pgfsetbuttcap%
\pgfsetroundjoin%
\pgfsetlinewidth{0.501875pt}%
\definecolor{currentstroke}{rgb}{0.000000,0.000000,0.000000}%
\pgfsetstrokecolor{currentstroke}%
\pgfsetdash{{1.000000pt}{3.000000pt}}{0.000000pt}%
\pgfpathmoveto{\pgfqpoint{5.960000in}{0.600000in}}%
\pgfpathlineto{\pgfqpoint{5.960000in}{5.400000in}}%
\pgfusepath{stroke}%
\end{pgfscope}%
\begin{pgfscope}%
\pgfsetbuttcap%
\pgfsetroundjoin%
\definecolor{currentfill}{rgb}{0.000000,0.000000,0.000000}%
\pgfsetfillcolor{currentfill}%
\pgfsetlinewidth{0.501875pt}%
\definecolor{currentstroke}{rgb}{0.000000,0.000000,0.000000}%
\pgfsetstrokecolor{currentstroke}%
\pgfsetdash{}{0pt}%
\pgfsys@defobject{currentmarker}{\pgfqpoint{0.000000in}{0.000000in}}{\pgfqpoint{0.000000in}{0.055556in}}{%
\pgfpathmoveto{\pgfqpoint{0.000000in}{0.000000in}}%
\pgfpathlineto{\pgfqpoint{0.000000in}{0.055556in}}%
\pgfusepath{stroke,fill}%
}%
\begin{pgfscope}%
\pgfsys@transformshift{5.960000in}{0.600000in}%
\pgfsys@useobject{currentmarker}{}%
\end{pgfscope}%
\end{pgfscope}%
\begin{pgfscope}%
\pgfsetbuttcap%
\pgfsetroundjoin%
\definecolor{currentfill}{rgb}{0.000000,0.000000,0.000000}%
\pgfsetfillcolor{currentfill}%
\pgfsetlinewidth{0.501875pt}%
\definecolor{currentstroke}{rgb}{0.000000,0.000000,0.000000}%
\pgfsetstrokecolor{currentstroke}%
\pgfsetdash{}{0pt}%
\pgfsys@defobject{currentmarker}{\pgfqpoint{0.000000in}{-0.055556in}}{\pgfqpoint{0.000000in}{0.000000in}}{%
\pgfpathmoveto{\pgfqpoint{0.000000in}{0.000000in}}%
\pgfpathlineto{\pgfqpoint{0.000000in}{-0.055556in}}%
\pgfusepath{stroke,fill}%
}%
\begin{pgfscope}%
\pgfsys@transformshift{5.960000in}{5.400000in}%
\pgfsys@useobject{currentmarker}{}%
\end{pgfscope}%
\end{pgfscope}%
\begin{pgfscope}%
\definecolor{textcolor}{rgb}{0.000000,0.000000,0.000000}%
\pgfsetstrokecolor{textcolor}%
\pgfsetfillcolor{textcolor}%
\pgftext[x=5.960000in,y=0.544444in,,top]{\color{textcolor}\rmfamily\fontsize{10.000000}{12.000000}\selectfont \(\displaystyle {80}\)}%
\end{pgfscope}%
\begin{pgfscope}%
\pgfpathrectangle{\pgfqpoint{1.000000in}{0.600000in}}{\pgfqpoint{6.200000in}{4.800000in}}%
\pgfusepath{clip}%
\pgfsetbuttcap%
\pgfsetroundjoin%
\pgfsetlinewidth{0.501875pt}%
\definecolor{currentstroke}{rgb}{0.000000,0.000000,0.000000}%
\pgfsetstrokecolor{currentstroke}%
\pgfsetdash{{1.000000pt}{3.000000pt}}{0.000000pt}%
\pgfpathmoveto{\pgfqpoint{7.200000in}{0.600000in}}%
\pgfpathlineto{\pgfqpoint{7.200000in}{5.400000in}}%
\pgfusepath{stroke}%
\end{pgfscope}%
\begin{pgfscope}%
\pgfsetbuttcap%
\pgfsetroundjoin%
\definecolor{currentfill}{rgb}{0.000000,0.000000,0.000000}%
\pgfsetfillcolor{currentfill}%
\pgfsetlinewidth{0.501875pt}%
\definecolor{currentstroke}{rgb}{0.000000,0.000000,0.000000}%
\pgfsetstrokecolor{currentstroke}%
\pgfsetdash{}{0pt}%
\pgfsys@defobject{currentmarker}{\pgfqpoint{0.000000in}{0.000000in}}{\pgfqpoint{0.000000in}{0.055556in}}{%
\pgfpathmoveto{\pgfqpoint{0.000000in}{0.000000in}}%
\pgfpathlineto{\pgfqpoint{0.000000in}{0.055556in}}%
\pgfusepath{stroke,fill}%
}%
\begin{pgfscope}%
\pgfsys@transformshift{7.200000in}{0.600000in}%
\pgfsys@useobject{currentmarker}{}%
\end{pgfscope}%
\end{pgfscope}%
\begin{pgfscope}%
\pgfsetbuttcap%
\pgfsetroundjoin%
\definecolor{currentfill}{rgb}{0.000000,0.000000,0.000000}%
\pgfsetfillcolor{currentfill}%
\pgfsetlinewidth{0.501875pt}%
\definecolor{currentstroke}{rgb}{0.000000,0.000000,0.000000}%
\pgfsetstrokecolor{currentstroke}%
\pgfsetdash{}{0pt}%
\pgfsys@defobject{currentmarker}{\pgfqpoint{0.000000in}{-0.055556in}}{\pgfqpoint{0.000000in}{0.000000in}}{%
\pgfpathmoveto{\pgfqpoint{0.000000in}{0.000000in}}%
\pgfpathlineto{\pgfqpoint{0.000000in}{-0.055556in}}%
\pgfusepath{stroke,fill}%
}%
\begin{pgfscope}%
\pgfsys@transformshift{7.200000in}{5.400000in}%
\pgfsys@useobject{currentmarker}{}%
\end{pgfscope}%
\end{pgfscope}%
\begin{pgfscope}%
\definecolor{textcolor}{rgb}{0.000000,0.000000,0.000000}%
\pgfsetstrokecolor{textcolor}%
\pgfsetfillcolor{textcolor}%
\pgftext[x=7.200000in,y=0.544444in,,top]{\color{textcolor}\rmfamily\fontsize{10.000000}{12.000000}\selectfont \(\displaystyle {100}\)}%
\end{pgfscope}%
\begin{pgfscope}%
\definecolor{textcolor}{rgb}{0.000000,0.000000,0.000000}%
\pgfsetstrokecolor{textcolor}%
\pgfsetfillcolor{textcolor}%
\pgftext[x=4.100000in,y=0.351543in,,top]{\color{textcolor}\rmfamily\fontsize{12.000000}{14.400000}\selectfont \(\displaystyle time\ (s)\)}%
\end{pgfscope}%
\begin{pgfscope}%
\pgfpathrectangle{\pgfqpoint{1.000000in}{0.600000in}}{\pgfqpoint{6.200000in}{4.800000in}}%
\pgfusepath{clip}%
\pgfsetbuttcap%
\pgfsetroundjoin%
\pgfsetlinewidth{0.501875pt}%
\definecolor{currentstroke}{rgb}{0.000000,0.000000,0.000000}%
\pgfsetstrokecolor{currentstroke}%
\pgfsetdash{{1.000000pt}{3.000000pt}}{0.000000pt}%
\pgfpathmoveto{\pgfqpoint{1.000000in}{0.600000in}}%
\pgfpathlineto{\pgfqpoint{7.200000in}{0.600000in}}%
\pgfusepath{stroke}%
\end{pgfscope}%
\begin{pgfscope}%
\pgfsetbuttcap%
\pgfsetroundjoin%
\definecolor{currentfill}{rgb}{0.000000,0.000000,0.000000}%
\pgfsetfillcolor{currentfill}%
\pgfsetlinewidth{0.501875pt}%
\definecolor{currentstroke}{rgb}{0.000000,0.000000,0.000000}%
\pgfsetstrokecolor{currentstroke}%
\pgfsetdash{}{0pt}%
\pgfsys@defobject{currentmarker}{\pgfqpoint{0.000000in}{0.000000in}}{\pgfqpoint{0.055556in}{0.000000in}}{%
\pgfpathmoveto{\pgfqpoint{0.000000in}{0.000000in}}%
\pgfpathlineto{\pgfqpoint{0.055556in}{0.000000in}}%
\pgfusepath{stroke,fill}%
}%
\begin{pgfscope}%
\pgfsys@transformshift{1.000000in}{0.600000in}%
\pgfsys@useobject{currentmarker}{}%
\end{pgfscope}%
\end{pgfscope}%
\begin{pgfscope}%
\pgfsetbuttcap%
\pgfsetroundjoin%
\definecolor{currentfill}{rgb}{0.000000,0.000000,0.000000}%
\pgfsetfillcolor{currentfill}%
\pgfsetlinewidth{0.501875pt}%
\definecolor{currentstroke}{rgb}{0.000000,0.000000,0.000000}%
\pgfsetstrokecolor{currentstroke}%
\pgfsetdash{}{0pt}%
\pgfsys@defobject{currentmarker}{\pgfqpoint{-0.055556in}{0.000000in}}{\pgfqpoint{-0.000000in}{0.000000in}}{%
\pgfpathmoveto{\pgfqpoint{-0.000000in}{0.000000in}}%
\pgfpathlineto{\pgfqpoint{-0.055556in}{0.000000in}}%
\pgfusepath{stroke,fill}%
}%
\begin{pgfscope}%
\pgfsys@transformshift{7.200000in}{0.600000in}%
\pgfsys@useobject{currentmarker}{}%
\end{pgfscope}%
\end{pgfscope}%
\begin{pgfscope}%
\definecolor{textcolor}{rgb}{0.000000,0.000000,0.000000}%
\pgfsetstrokecolor{textcolor}%
\pgfsetfillcolor{textcolor}%
\pgftext[x=0.944444in,y=0.600000in,right,]{\color{textcolor}\rmfamily\fontsize{10.000000}{12.000000}\selectfont \(\displaystyle {\ensuremath{-}1.5}\)}%
\end{pgfscope}%
\begin{pgfscope}%
\pgfpathrectangle{\pgfqpoint{1.000000in}{0.600000in}}{\pgfqpoint{6.200000in}{4.800000in}}%
\pgfusepath{clip}%
\pgfsetbuttcap%
\pgfsetroundjoin%
\pgfsetlinewidth{0.501875pt}%
\definecolor{currentstroke}{rgb}{0.000000,0.000000,0.000000}%
\pgfsetstrokecolor{currentstroke}%
\pgfsetdash{{1.000000pt}{3.000000pt}}{0.000000pt}%
\pgfpathmoveto{\pgfqpoint{1.000000in}{1.400000in}}%
\pgfpathlineto{\pgfqpoint{7.200000in}{1.400000in}}%
\pgfusepath{stroke}%
\end{pgfscope}%
\begin{pgfscope}%
\pgfsetbuttcap%
\pgfsetroundjoin%
\definecolor{currentfill}{rgb}{0.000000,0.000000,0.000000}%
\pgfsetfillcolor{currentfill}%
\pgfsetlinewidth{0.501875pt}%
\definecolor{currentstroke}{rgb}{0.000000,0.000000,0.000000}%
\pgfsetstrokecolor{currentstroke}%
\pgfsetdash{}{0pt}%
\pgfsys@defobject{currentmarker}{\pgfqpoint{0.000000in}{0.000000in}}{\pgfqpoint{0.055556in}{0.000000in}}{%
\pgfpathmoveto{\pgfqpoint{0.000000in}{0.000000in}}%
\pgfpathlineto{\pgfqpoint{0.055556in}{0.000000in}}%
\pgfusepath{stroke,fill}%
}%
\begin{pgfscope}%
\pgfsys@transformshift{1.000000in}{1.400000in}%
\pgfsys@useobject{currentmarker}{}%
\end{pgfscope}%
\end{pgfscope}%
\begin{pgfscope}%
\pgfsetbuttcap%
\pgfsetroundjoin%
\definecolor{currentfill}{rgb}{0.000000,0.000000,0.000000}%
\pgfsetfillcolor{currentfill}%
\pgfsetlinewidth{0.501875pt}%
\definecolor{currentstroke}{rgb}{0.000000,0.000000,0.000000}%
\pgfsetstrokecolor{currentstroke}%
\pgfsetdash{}{0pt}%
\pgfsys@defobject{currentmarker}{\pgfqpoint{-0.055556in}{0.000000in}}{\pgfqpoint{-0.000000in}{0.000000in}}{%
\pgfpathmoveto{\pgfqpoint{-0.000000in}{0.000000in}}%
\pgfpathlineto{\pgfqpoint{-0.055556in}{0.000000in}}%
\pgfusepath{stroke,fill}%
}%
\begin{pgfscope}%
\pgfsys@transformshift{7.200000in}{1.400000in}%
\pgfsys@useobject{currentmarker}{}%
\end{pgfscope}%
\end{pgfscope}%
\begin{pgfscope}%
\definecolor{textcolor}{rgb}{0.000000,0.000000,0.000000}%
\pgfsetstrokecolor{textcolor}%
\pgfsetfillcolor{textcolor}%
\pgftext[x=0.944444in,y=1.400000in,right,]{\color{textcolor}\rmfamily\fontsize{10.000000}{12.000000}\selectfont \(\displaystyle {\ensuremath{-}1.0}\)}%
\end{pgfscope}%
\begin{pgfscope}%
\pgfpathrectangle{\pgfqpoint{1.000000in}{0.600000in}}{\pgfqpoint{6.200000in}{4.800000in}}%
\pgfusepath{clip}%
\pgfsetbuttcap%
\pgfsetroundjoin%
\pgfsetlinewidth{0.501875pt}%
\definecolor{currentstroke}{rgb}{0.000000,0.000000,0.000000}%
\pgfsetstrokecolor{currentstroke}%
\pgfsetdash{{1.000000pt}{3.000000pt}}{0.000000pt}%
\pgfpathmoveto{\pgfqpoint{1.000000in}{2.200000in}}%
\pgfpathlineto{\pgfqpoint{7.200000in}{2.200000in}}%
\pgfusepath{stroke}%
\end{pgfscope}%
\begin{pgfscope}%
\pgfsetbuttcap%
\pgfsetroundjoin%
\definecolor{currentfill}{rgb}{0.000000,0.000000,0.000000}%
\pgfsetfillcolor{currentfill}%
\pgfsetlinewidth{0.501875pt}%
\definecolor{currentstroke}{rgb}{0.000000,0.000000,0.000000}%
\pgfsetstrokecolor{currentstroke}%
\pgfsetdash{}{0pt}%
\pgfsys@defobject{currentmarker}{\pgfqpoint{0.000000in}{0.000000in}}{\pgfqpoint{0.055556in}{0.000000in}}{%
\pgfpathmoveto{\pgfqpoint{0.000000in}{0.000000in}}%
\pgfpathlineto{\pgfqpoint{0.055556in}{0.000000in}}%
\pgfusepath{stroke,fill}%
}%
\begin{pgfscope}%
\pgfsys@transformshift{1.000000in}{2.200000in}%
\pgfsys@useobject{currentmarker}{}%
\end{pgfscope}%
\end{pgfscope}%
\begin{pgfscope}%
\pgfsetbuttcap%
\pgfsetroundjoin%
\definecolor{currentfill}{rgb}{0.000000,0.000000,0.000000}%
\pgfsetfillcolor{currentfill}%
\pgfsetlinewidth{0.501875pt}%
\definecolor{currentstroke}{rgb}{0.000000,0.000000,0.000000}%
\pgfsetstrokecolor{currentstroke}%
\pgfsetdash{}{0pt}%
\pgfsys@defobject{currentmarker}{\pgfqpoint{-0.055556in}{0.000000in}}{\pgfqpoint{-0.000000in}{0.000000in}}{%
\pgfpathmoveto{\pgfqpoint{-0.000000in}{0.000000in}}%
\pgfpathlineto{\pgfqpoint{-0.055556in}{0.000000in}}%
\pgfusepath{stroke,fill}%
}%
\begin{pgfscope}%
\pgfsys@transformshift{7.200000in}{2.200000in}%
\pgfsys@useobject{currentmarker}{}%
\end{pgfscope}%
\end{pgfscope}%
\begin{pgfscope}%
\definecolor{textcolor}{rgb}{0.000000,0.000000,0.000000}%
\pgfsetstrokecolor{textcolor}%
\pgfsetfillcolor{textcolor}%
\pgftext[x=0.944444in,y=2.200000in,right,]{\color{textcolor}\rmfamily\fontsize{10.000000}{12.000000}\selectfont \(\displaystyle {\ensuremath{-}0.5}\)}%
\end{pgfscope}%
\begin{pgfscope}%
\pgfpathrectangle{\pgfqpoint{1.000000in}{0.600000in}}{\pgfqpoint{6.200000in}{4.800000in}}%
\pgfusepath{clip}%
\pgfsetbuttcap%
\pgfsetroundjoin%
\pgfsetlinewidth{0.501875pt}%
\definecolor{currentstroke}{rgb}{0.000000,0.000000,0.000000}%
\pgfsetstrokecolor{currentstroke}%
\pgfsetdash{{1.000000pt}{3.000000pt}}{0.000000pt}%
\pgfpathmoveto{\pgfqpoint{1.000000in}{3.000000in}}%
\pgfpathlineto{\pgfqpoint{7.200000in}{3.000000in}}%
\pgfusepath{stroke}%
\end{pgfscope}%
\begin{pgfscope}%
\pgfsetbuttcap%
\pgfsetroundjoin%
\definecolor{currentfill}{rgb}{0.000000,0.000000,0.000000}%
\pgfsetfillcolor{currentfill}%
\pgfsetlinewidth{0.501875pt}%
\definecolor{currentstroke}{rgb}{0.000000,0.000000,0.000000}%
\pgfsetstrokecolor{currentstroke}%
\pgfsetdash{}{0pt}%
\pgfsys@defobject{currentmarker}{\pgfqpoint{0.000000in}{0.000000in}}{\pgfqpoint{0.055556in}{0.000000in}}{%
\pgfpathmoveto{\pgfqpoint{0.000000in}{0.000000in}}%
\pgfpathlineto{\pgfqpoint{0.055556in}{0.000000in}}%
\pgfusepath{stroke,fill}%
}%
\begin{pgfscope}%
\pgfsys@transformshift{1.000000in}{3.000000in}%
\pgfsys@useobject{currentmarker}{}%
\end{pgfscope}%
\end{pgfscope}%
\begin{pgfscope}%
\pgfsetbuttcap%
\pgfsetroundjoin%
\definecolor{currentfill}{rgb}{0.000000,0.000000,0.000000}%
\pgfsetfillcolor{currentfill}%
\pgfsetlinewidth{0.501875pt}%
\definecolor{currentstroke}{rgb}{0.000000,0.000000,0.000000}%
\pgfsetstrokecolor{currentstroke}%
\pgfsetdash{}{0pt}%
\pgfsys@defobject{currentmarker}{\pgfqpoint{-0.055556in}{0.000000in}}{\pgfqpoint{-0.000000in}{0.000000in}}{%
\pgfpathmoveto{\pgfqpoint{-0.000000in}{0.000000in}}%
\pgfpathlineto{\pgfqpoint{-0.055556in}{0.000000in}}%
\pgfusepath{stroke,fill}%
}%
\begin{pgfscope}%
\pgfsys@transformshift{7.200000in}{3.000000in}%
\pgfsys@useobject{currentmarker}{}%
\end{pgfscope}%
\end{pgfscope}%
\begin{pgfscope}%
\definecolor{textcolor}{rgb}{0.000000,0.000000,0.000000}%
\pgfsetstrokecolor{textcolor}%
\pgfsetfillcolor{textcolor}%
\pgftext[x=0.944444in,y=3.000000in,right,]{\color{textcolor}\rmfamily\fontsize{10.000000}{12.000000}\selectfont \(\displaystyle {0.0}\)}%
\end{pgfscope}%
\begin{pgfscope}%
\pgfpathrectangle{\pgfqpoint{1.000000in}{0.600000in}}{\pgfqpoint{6.200000in}{4.800000in}}%
\pgfusepath{clip}%
\pgfsetbuttcap%
\pgfsetroundjoin%
\pgfsetlinewidth{0.501875pt}%
\definecolor{currentstroke}{rgb}{0.000000,0.000000,0.000000}%
\pgfsetstrokecolor{currentstroke}%
\pgfsetdash{{1.000000pt}{3.000000pt}}{0.000000pt}%
\pgfpathmoveto{\pgfqpoint{1.000000in}{3.800000in}}%
\pgfpathlineto{\pgfqpoint{7.200000in}{3.800000in}}%
\pgfusepath{stroke}%
\end{pgfscope}%
\begin{pgfscope}%
\pgfsetbuttcap%
\pgfsetroundjoin%
\definecolor{currentfill}{rgb}{0.000000,0.000000,0.000000}%
\pgfsetfillcolor{currentfill}%
\pgfsetlinewidth{0.501875pt}%
\definecolor{currentstroke}{rgb}{0.000000,0.000000,0.000000}%
\pgfsetstrokecolor{currentstroke}%
\pgfsetdash{}{0pt}%
\pgfsys@defobject{currentmarker}{\pgfqpoint{0.000000in}{0.000000in}}{\pgfqpoint{0.055556in}{0.000000in}}{%
\pgfpathmoveto{\pgfqpoint{0.000000in}{0.000000in}}%
\pgfpathlineto{\pgfqpoint{0.055556in}{0.000000in}}%
\pgfusepath{stroke,fill}%
}%
\begin{pgfscope}%
\pgfsys@transformshift{1.000000in}{3.800000in}%
\pgfsys@useobject{currentmarker}{}%
\end{pgfscope}%
\end{pgfscope}%
\begin{pgfscope}%
\pgfsetbuttcap%
\pgfsetroundjoin%
\definecolor{currentfill}{rgb}{0.000000,0.000000,0.000000}%
\pgfsetfillcolor{currentfill}%
\pgfsetlinewidth{0.501875pt}%
\definecolor{currentstroke}{rgb}{0.000000,0.000000,0.000000}%
\pgfsetstrokecolor{currentstroke}%
\pgfsetdash{}{0pt}%
\pgfsys@defobject{currentmarker}{\pgfqpoint{-0.055556in}{0.000000in}}{\pgfqpoint{-0.000000in}{0.000000in}}{%
\pgfpathmoveto{\pgfqpoint{-0.000000in}{0.000000in}}%
\pgfpathlineto{\pgfqpoint{-0.055556in}{0.000000in}}%
\pgfusepath{stroke,fill}%
}%
\begin{pgfscope}%
\pgfsys@transformshift{7.200000in}{3.800000in}%
\pgfsys@useobject{currentmarker}{}%
\end{pgfscope}%
\end{pgfscope}%
\begin{pgfscope}%
\definecolor{textcolor}{rgb}{0.000000,0.000000,0.000000}%
\pgfsetstrokecolor{textcolor}%
\pgfsetfillcolor{textcolor}%
\pgftext[x=0.944444in,y=3.800000in,right,]{\color{textcolor}\rmfamily\fontsize{10.000000}{12.000000}\selectfont \(\displaystyle {0.5}\)}%
\end{pgfscope}%
\begin{pgfscope}%
\pgfpathrectangle{\pgfqpoint{1.000000in}{0.600000in}}{\pgfqpoint{6.200000in}{4.800000in}}%
\pgfusepath{clip}%
\pgfsetbuttcap%
\pgfsetroundjoin%
\pgfsetlinewidth{0.501875pt}%
\definecolor{currentstroke}{rgb}{0.000000,0.000000,0.000000}%
\pgfsetstrokecolor{currentstroke}%
\pgfsetdash{{1.000000pt}{3.000000pt}}{0.000000pt}%
\pgfpathmoveto{\pgfqpoint{1.000000in}{4.600000in}}%
\pgfpathlineto{\pgfqpoint{7.200000in}{4.600000in}}%
\pgfusepath{stroke}%
\end{pgfscope}%
\begin{pgfscope}%
\pgfsetbuttcap%
\pgfsetroundjoin%
\definecolor{currentfill}{rgb}{0.000000,0.000000,0.000000}%
\pgfsetfillcolor{currentfill}%
\pgfsetlinewidth{0.501875pt}%
\definecolor{currentstroke}{rgb}{0.000000,0.000000,0.000000}%
\pgfsetstrokecolor{currentstroke}%
\pgfsetdash{}{0pt}%
\pgfsys@defobject{currentmarker}{\pgfqpoint{0.000000in}{0.000000in}}{\pgfqpoint{0.055556in}{0.000000in}}{%
\pgfpathmoveto{\pgfqpoint{0.000000in}{0.000000in}}%
\pgfpathlineto{\pgfqpoint{0.055556in}{0.000000in}}%
\pgfusepath{stroke,fill}%
}%
\begin{pgfscope}%
\pgfsys@transformshift{1.000000in}{4.600000in}%
\pgfsys@useobject{currentmarker}{}%
\end{pgfscope}%
\end{pgfscope}%
\begin{pgfscope}%
\pgfsetbuttcap%
\pgfsetroundjoin%
\definecolor{currentfill}{rgb}{0.000000,0.000000,0.000000}%
\pgfsetfillcolor{currentfill}%
\pgfsetlinewidth{0.501875pt}%
\definecolor{currentstroke}{rgb}{0.000000,0.000000,0.000000}%
\pgfsetstrokecolor{currentstroke}%
\pgfsetdash{}{0pt}%
\pgfsys@defobject{currentmarker}{\pgfqpoint{-0.055556in}{0.000000in}}{\pgfqpoint{-0.000000in}{0.000000in}}{%
\pgfpathmoveto{\pgfqpoint{-0.000000in}{0.000000in}}%
\pgfpathlineto{\pgfqpoint{-0.055556in}{0.000000in}}%
\pgfusepath{stroke,fill}%
}%
\begin{pgfscope}%
\pgfsys@transformshift{7.200000in}{4.600000in}%
\pgfsys@useobject{currentmarker}{}%
\end{pgfscope}%
\end{pgfscope}%
\begin{pgfscope}%
\definecolor{textcolor}{rgb}{0.000000,0.000000,0.000000}%
\pgfsetstrokecolor{textcolor}%
\pgfsetfillcolor{textcolor}%
\pgftext[x=0.944444in,y=4.600000in,right,]{\color{textcolor}\rmfamily\fontsize{10.000000}{12.000000}\selectfont \(\displaystyle {1.0}\)}%
\end{pgfscope}%
\begin{pgfscope}%
\pgfpathrectangle{\pgfqpoint{1.000000in}{0.600000in}}{\pgfqpoint{6.200000in}{4.800000in}}%
\pgfusepath{clip}%
\pgfsetbuttcap%
\pgfsetroundjoin%
\pgfsetlinewidth{0.501875pt}%
\definecolor{currentstroke}{rgb}{0.000000,0.000000,0.000000}%
\pgfsetstrokecolor{currentstroke}%
\pgfsetdash{{1.000000pt}{3.000000pt}}{0.000000pt}%
\pgfpathmoveto{\pgfqpoint{1.000000in}{5.400000in}}%
\pgfpathlineto{\pgfqpoint{7.200000in}{5.400000in}}%
\pgfusepath{stroke}%
\end{pgfscope}%
\begin{pgfscope}%
\pgfsetbuttcap%
\pgfsetroundjoin%
\definecolor{currentfill}{rgb}{0.000000,0.000000,0.000000}%
\pgfsetfillcolor{currentfill}%
\pgfsetlinewidth{0.501875pt}%
\definecolor{currentstroke}{rgb}{0.000000,0.000000,0.000000}%
\pgfsetstrokecolor{currentstroke}%
\pgfsetdash{}{0pt}%
\pgfsys@defobject{currentmarker}{\pgfqpoint{0.000000in}{0.000000in}}{\pgfqpoint{0.055556in}{0.000000in}}{%
\pgfpathmoveto{\pgfqpoint{0.000000in}{0.000000in}}%
\pgfpathlineto{\pgfqpoint{0.055556in}{0.000000in}}%
\pgfusepath{stroke,fill}%
}%
\begin{pgfscope}%
\pgfsys@transformshift{1.000000in}{5.400000in}%
\pgfsys@useobject{currentmarker}{}%
\end{pgfscope}%
\end{pgfscope}%
\begin{pgfscope}%
\pgfsetbuttcap%
\pgfsetroundjoin%
\definecolor{currentfill}{rgb}{0.000000,0.000000,0.000000}%
\pgfsetfillcolor{currentfill}%
\pgfsetlinewidth{0.501875pt}%
\definecolor{currentstroke}{rgb}{0.000000,0.000000,0.000000}%
\pgfsetstrokecolor{currentstroke}%
\pgfsetdash{}{0pt}%
\pgfsys@defobject{currentmarker}{\pgfqpoint{-0.055556in}{0.000000in}}{\pgfqpoint{-0.000000in}{0.000000in}}{%
\pgfpathmoveto{\pgfqpoint{-0.000000in}{0.000000in}}%
\pgfpathlineto{\pgfqpoint{-0.055556in}{0.000000in}}%
\pgfusepath{stroke,fill}%
}%
\begin{pgfscope}%
\pgfsys@transformshift{7.200000in}{5.400000in}%
\pgfsys@useobject{currentmarker}{}%
\end{pgfscope}%
\end{pgfscope}%
\begin{pgfscope}%
\definecolor{textcolor}{rgb}{0.000000,0.000000,0.000000}%
\pgfsetstrokecolor{textcolor}%
\pgfsetfillcolor{textcolor}%
\pgftext[x=0.944444in,y=5.400000in,right,]{\color{textcolor}\rmfamily\fontsize{10.000000}{12.000000}\selectfont \(\displaystyle {1.5}\)}%
\end{pgfscope}%
\begin{pgfscope}%
\definecolor{textcolor}{rgb}{0.000000,0.000000,0.000000}%
\pgfsetstrokecolor{textcolor}%
\pgfsetfillcolor{textcolor}%
\pgftext[x=0.589505in,y=3.000000in,,bottom,rotate=90.000000]{\color{textcolor}\rmfamily\fontsize{12.000000}{14.400000}\selectfont \(\displaystyle \theta\ (rad)\)}%
\end{pgfscope}%
\begin{pgfscope}%
\definecolor{textcolor}{rgb}{0.000000,0.000000,0.000000}%
\pgfsetstrokecolor{textcolor}%
\pgfsetfillcolor{textcolor}%
\pgftext[x=4.100000in,y=5.469444in,,base]{\color{textcolor}\rmfamily\fontsize{12.000000}{14.400000}\selectfont \(\displaystyle Simple\ pendulum\ solution\ (time\ step = 0.02\ s)\)}%
\end{pgfscope}%
\begin{pgfscope}%
\pgfsetbuttcap%
\pgfsetmiterjoin%
\definecolor{currentfill}{rgb}{1.000000,1.000000,1.000000}%
\pgfsetfillcolor{currentfill}%
\pgfsetlinewidth{1.003750pt}%
\definecolor{currentstroke}{rgb}{0.000000,0.000000,0.000000}%
\pgfsetstrokecolor{currentstroke}%
\pgfsetdash{}{0pt}%
\pgfpathmoveto{\pgfqpoint{1.083333in}{4.569445in}}%
\pgfpathlineto{\pgfqpoint{3.093110in}{4.569445in}}%
\pgfpathlineto{\pgfqpoint{3.093110in}{5.316667in}}%
\pgfpathlineto{\pgfqpoint{1.083333in}{5.316667in}}%
\pgfpathlineto{\pgfqpoint{1.083333in}{4.569445in}}%
\pgfpathclose%
\pgfusepath{stroke,fill}%
\end{pgfscope}%
\begin{pgfscope}%
\pgfsetrectcap%
\pgfsetroundjoin%
\pgfsetlinewidth{1.003750pt}%
\definecolor{currentstroke}{rgb}{1.000000,0.000000,0.000000}%
\pgfsetstrokecolor{currentstroke}%
\pgfsetdash{}{0pt}%
\pgfpathmoveto{\pgfqpoint{1.200000in}{5.191667in}}%
\pgfpathlineto{\pgfqpoint{1.433333in}{5.191667in}}%
\pgfusepath{stroke}%
\end{pgfscope}%
\begin{pgfscope}%
\definecolor{textcolor}{rgb}{0.000000,0.000000,0.000000}%
\pgfsetstrokecolor{textcolor}%
\pgfsetfillcolor{textcolor}%
\pgftext[x=1.616667in,y=5.133333in,left,base]{\color{textcolor}\rmfamily\fontsize{12.000000}{14.400000}\selectfont \(\displaystyle euler\ explicit\)}%
\end{pgfscope}%
\begin{pgfscope}%
\pgfsetrectcap%
\pgfsetroundjoin%
\pgfsetlinewidth{1.003750pt}%
\definecolor{currentstroke}{rgb}{0.000000,0.000000,1.000000}%
\pgfsetstrokecolor{currentstroke}%
\pgfsetdash{}{0pt}%
\pgfpathmoveto{\pgfqpoint{1.200000in}{4.959260in}}%
\pgfpathlineto{\pgfqpoint{1.433333in}{4.959260in}}%
\pgfusepath{stroke}%
\end{pgfscope}%
\begin{pgfscope}%
\definecolor{textcolor}{rgb}{0.000000,0.000000,0.000000}%
\pgfsetstrokecolor{textcolor}%
\pgfsetfillcolor{textcolor}%
\pgftext[x=1.616667in,y=4.900926in,left,base]{\color{textcolor}\rmfamily\fontsize{12.000000}{14.400000}\selectfont \(\displaystyle euler\ implicit\)}%
\end{pgfscope}%
\begin{pgfscope}%
\pgfsetrectcap%
\pgfsetroundjoin%
\pgfsetlinewidth{1.003750pt}%
\definecolor{currentstroke}{rgb}{0.000000,0.000000,0.000000}%
\pgfsetstrokecolor{currentstroke}%
\pgfsetdash{}{0pt}%
\pgfpathmoveto{\pgfqpoint{1.200000in}{4.726852in}}%
\pgfpathlineto{\pgfqpoint{1.433333in}{4.726852in}}%
\pgfusepath{stroke}%
\end{pgfscope}%
\begin{pgfscope}%
\definecolor{textcolor}{rgb}{0.000000,0.000000,0.000000}%
\pgfsetstrokecolor{textcolor}%
\pgfsetfillcolor{textcolor}%
\pgftext[x=1.616667in,y=4.668519in,left,base]{\color{textcolor}\rmfamily\fontsize{12.000000}{14.400000}\selectfont \(\displaystyle trapezoidal\ scheme\)}%
\end{pgfscope}%
\end{pgfpicture}%
\makeatother%
\endgroup%
}
    \end{figure}
    
    \begin{figure}[ht!]
    \centering
    \resizebox{0.9\linewidth}{!}{%% Creator: Matplotlib, PGF backend
%%
%% To include the figure in your LaTeX document, write
%%   \input{<filename>.pgf}
%%
%% Make sure the required packages are loaded in your preamble
%%   \usepackage{pgf}
%%
%% Also ensure that all the required font packages are loaded; for instance,
%% the lmodern package is sometimes necessary when using math font.
%%   \usepackage{lmodern}
%%
%% Figures using additional raster images can only be included by \input if
%% they are in the same directory as the main LaTeX file. For loading figures
%% from other directories you can use the `import` package
%%   \usepackage{import}
%%
%% and then include the figures with
%%   \import{<path to file>}{<filename>.pgf}
%%
%% Matplotlib used the following preamble
%%
\begingroup%
\makeatletter%
\begin{pgfpicture}%
\pgfpathrectangle{\pgfpointorigin}{\pgfqpoint{8.000000in}{6.000000in}}%
\pgfusepath{use as bounding box, clip}%
\begin{pgfscope}%
\pgfsetbuttcap%
\pgfsetmiterjoin%
\definecolor{currentfill}{rgb}{1.000000,1.000000,1.000000}%
\pgfsetfillcolor{currentfill}%
\pgfsetlinewidth{0.000000pt}%
\definecolor{currentstroke}{rgb}{1.000000,1.000000,1.000000}%
\pgfsetstrokecolor{currentstroke}%
\pgfsetdash{}{0pt}%
\pgfpathmoveto{\pgfqpoint{0.000000in}{0.000000in}}%
\pgfpathlineto{\pgfqpoint{8.000000in}{0.000000in}}%
\pgfpathlineto{\pgfqpoint{8.000000in}{6.000000in}}%
\pgfpathlineto{\pgfqpoint{0.000000in}{6.000000in}}%
\pgfpathlineto{\pgfqpoint{0.000000in}{0.000000in}}%
\pgfpathclose%
\pgfusepath{fill}%
\end{pgfscope}%
\begin{pgfscope}%
\pgfsetbuttcap%
\pgfsetmiterjoin%
\definecolor{currentfill}{rgb}{1.000000,1.000000,1.000000}%
\pgfsetfillcolor{currentfill}%
\pgfsetlinewidth{0.000000pt}%
\definecolor{currentstroke}{rgb}{0.000000,0.000000,0.000000}%
\pgfsetstrokecolor{currentstroke}%
\pgfsetstrokeopacity{0.000000}%
\pgfsetdash{}{0pt}%
\pgfpathmoveto{\pgfqpoint{1.000000in}{0.600000in}}%
\pgfpathlineto{\pgfqpoint{7.200000in}{0.600000in}}%
\pgfpathlineto{\pgfqpoint{7.200000in}{5.400000in}}%
\pgfpathlineto{\pgfqpoint{1.000000in}{5.400000in}}%
\pgfpathlineto{\pgfqpoint{1.000000in}{0.600000in}}%
\pgfpathclose%
\pgfusepath{fill}%
\end{pgfscope}%
\begin{pgfscope}%
\pgfpathrectangle{\pgfqpoint{1.000000in}{0.600000in}}{\pgfqpoint{6.200000in}{4.800000in}}%
\pgfusepath{clip}%
\pgfsetrectcap%
\pgfsetroundjoin%
\pgfsetlinewidth{1.003750pt}%
\definecolor{currentstroke}{rgb}{1.000000,0.000000,0.000000}%
\pgfsetstrokecolor{currentstroke}%
\pgfsetdash{}{0pt}%
\pgfpathmoveto{\pgfqpoint{1.000000in}{4.786094in}}%
\pgfpathlineto{\pgfqpoint{1.012400in}{4.785065in}}%
\pgfpathlineto{\pgfqpoint{1.024800in}{4.781320in}}%
\pgfpathlineto{\pgfqpoint{1.037200in}{4.774964in}}%
\pgfpathlineto{\pgfqpoint{1.049600in}{4.766201in}}%
\pgfpathlineto{\pgfqpoint{1.065100in}{4.752325in}}%
\pgfpathlineto{\pgfqpoint{1.083700in}{4.732439in}}%
\pgfpathlineto{\pgfqpoint{1.145700in}{4.663095in}}%
\pgfpathlineto{\pgfqpoint{1.161200in}{4.650594in}}%
\pgfpathlineto{\pgfqpoint{1.173600in}{4.643226in}}%
\pgfpathlineto{\pgfqpoint{1.186000in}{4.638511in}}%
\pgfpathlineto{\pgfqpoint{1.198400in}{4.636637in}}%
\pgfpathlineto{\pgfqpoint{1.210800in}{4.637690in}}%
\pgfpathlineto{\pgfqpoint{1.223200in}{4.641661in}}%
\pgfpathlineto{\pgfqpoint{1.235600in}{4.648439in}}%
\pgfpathlineto{\pgfqpoint{1.248000in}{4.657812in}}%
\pgfpathlineto{\pgfqpoint{1.263500in}{4.672690in}}%
\pgfpathlineto{\pgfqpoint{1.282100in}{4.694067in}}%
\pgfpathlineto{\pgfqpoint{1.344100in}{4.769027in}}%
\pgfpathlineto{\pgfqpoint{1.359600in}{4.782651in}}%
\pgfpathlineto{\pgfqpoint{1.372000in}{4.790742in}}%
\pgfpathlineto{\pgfqpoint{1.384400in}{4.795998in}}%
\pgfpathlineto{\pgfqpoint{1.393700in}{4.797955in}}%
\pgfpathlineto{\pgfqpoint{1.403000in}{4.798155in}}%
\pgfpathlineto{\pgfqpoint{1.412300in}{4.796583in}}%
\pgfpathlineto{\pgfqpoint{1.424700in}{4.791768in}}%
\pgfpathlineto{\pgfqpoint{1.437100in}{4.783974in}}%
\pgfpathlineto{\pgfqpoint{1.449500in}{4.773442in}}%
\pgfpathlineto{\pgfqpoint{1.465000in}{4.756968in}}%
\pgfpathlineto{\pgfqpoint{1.483600in}{4.733577in}}%
\pgfpathlineto{\pgfqpoint{1.539400in}{4.659951in}}%
\pgfpathlineto{\pgfqpoint{1.554900in}{4.644205in}}%
\pgfpathlineto{\pgfqpoint{1.567300in}{4.634463in}}%
\pgfpathlineto{\pgfqpoint{1.579700in}{4.627654in}}%
\pgfpathlineto{\pgfqpoint{1.589000in}{4.624629in}}%
\pgfpathlineto{\pgfqpoint{1.598300in}{4.623463in}}%
\pgfpathlineto{\pgfqpoint{1.607600in}{4.624188in}}%
\pgfpathlineto{\pgfqpoint{1.616900in}{4.626804in}}%
\pgfpathlineto{\pgfqpoint{1.626200in}{4.631274in}}%
\pgfpathlineto{\pgfqpoint{1.638600in}{4.639993in}}%
\pgfpathlineto{\pgfqpoint{1.651000in}{4.651616in}}%
\pgfpathlineto{\pgfqpoint{1.666500in}{4.669638in}}%
\pgfpathlineto{\pgfqpoint{1.685100in}{4.695055in}}%
\pgfpathlineto{\pgfqpoint{1.740900in}{4.774192in}}%
\pgfpathlineto{\pgfqpoint{1.756400in}{4.790913in}}%
\pgfpathlineto{\pgfqpoint{1.768800in}{4.801188in}}%
\pgfpathlineto{\pgfqpoint{1.781200in}{4.808299in}}%
\pgfpathlineto{\pgfqpoint{1.790500in}{4.811398in}}%
\pgfpathlineto{\pgfqpoint{1.799800in}{4.812506in}}%
\pgfpathlineto{\pgfqpoint{1.809100in}{4.811594in}}%
\pgfpathlineto{\pgfqpoint{1.818400in}{4.808665in}}%
\pgfpathlineto{\pgfqpoint{1.827700in}{4.803758in}}%
\pgfpathlineto{\pgfqpoint{1.840100in}{4.794272in}}%
\pgfpathlineto{\pgfqpoint{1.852500in}{4.781687in}}%
\pgfpathlineto{\pgfqpoint{1.868000in}{4.762218in}}%
\pgfpathlineto{\pgfqpoint{1.886600in}{4.734798in}}%
\pgfpathlineto{\pgfqpoint{1.939300in}{4.653622in}}%
\pgfpathlineto{\pgfqpoint{1.954800in}{4.634796in}}%
\pgfpathlineto{\pgfqpoint{1.967200in}{4.622925in}}%
\pgfpathlineto{\pgfqpoint{1.979600in}{4.614350in}}%
\pgfpathlineto{\pgfqpoint{1.988900in}{4.610262in}}%
\pgfpathlineto{\pgfqpoint{1.998200in}{4.608276in}}%
\pgfpathlineto{\pgfqpoint{2.007500in}{4.608435in}}%
\pgfpathlineto{\pgfqpoint{2.016800in}{4.610748in}}%
\pgfpathlineto{\pgfqpoint{2.026100in}{4.615189in}}%
\pgfpathlineto{\pgfqpoint{2.035400in}{4.621697in}}%
\pgfpathlineto{\pgfqpoint{2.047800in}{4.633412in}}%
\pgfpathlineto{\pgfqpoint{2.060200in}{4.648275in}}%
\pgfpathlineto{\pgfqpoint{2.075700in}{4.670538in}}%
\pgfpathlineto{\pgfqpoint{2.097400in}{4.706330in}}%
\pgfpathlineto{\pgfqpoint{2.134600in}{4.768889in}}%
\pgfpathlineto{\pgfqpoint{2.150100in}{4.791042in}}%
\pgfpathlineto{\pgfqpoint{2.162500in}{4.805717in}}%
\pgfpathlineto{\pgfqpoint{2.174900in}{4.817118in}}%
\pgfpathlineto{\pgfqpoint{2.184200in}{4.823286in}}%
\pgfpathlineto{\pgfqpoint{2.193500in}{4.827284in}}%
\pgfpathlineto{\pgfqpoint{2.202800in}{4.829036in}}%
\pgfpathlineto{\pgfqpoint{2.212100in}{4.828505in}}%
\pgfpathlineto{\pgfqpoint{2.221400in}{4.825687in}}%
\pgfpathlineto{\pgfqpoint{2.230700in}{4.820614in}}%
\pgfpathlineto{\pgfqpoint{2.240000in}{4.813353in}}%
\pgfpathlineto{\pgfqpoint{2.252400in}{4.800457in}}%
\pgfpathlineto{\pgfqpoint{2.264800in}{4.784235in}}%
\pgfpathlineto{\pgfqpoint{2.280300in}{4.760069in}}%
\pgfpathlineto{\pgfqpoint{2.302000in}{4.721403in}}%
\pgfpathlineto{\pgfqpoint{2.339200in}{4.654228in}}%
\pgfpathlineto{\pgfqpoint{2.354700in}{4.630572in}}%
\pgfpathlineto{\pgfqpoint{2.367100in}{4.614948in}}%
\pgfpathlineto{\pgfqpoint{2.379500in}{4.602839in}}%
\pgfpathlineto{\pgfqpoint{2.388800in}{4.596299in}}%
\pgfpathlineto{\pgfqpoint{2.398100in}{4.592067in}}%
\pgfpathlineto{\pgfqpoint{2.407400in}{4.590216in}}%
\pgfpathlineto{\pgfqpoint{2.416700in}{4.590784in}}%
\pgfpathlineto{\pgfqpoint{2.426000in}{4.593774in}}%
\pgfpathlineto{\pgfqpoint{2.435300in}{4.599156in}}%
\pgfpathlineto{\pgfqpoint{2.444600in}{4.606861in}}%
\pgfpathlineto{\pgfqpoint{2.457000in}{4.620562in}}%
\pgfpathlineto{\pgfqpoint{2.469400in}{4.637831in}}%
\pgfpathlineto{\pgfqpoint{2.484900in}{4.663623in}}%
\pgfpathlineto{\pgfqpoint{2.506600in}{4.705047in}}%
\pgfpathlineto{\pgfqpoint{2.543800in}{4.777455in}}%
\pgfpathlineto{\pgfqpoint{2.559300in}{4.803132in}}%
\pgfpathlineto{\pgfqpoint{2.571700in}{4.820202in}}%
\pgfpathlineto{\pgfqpoint{2.584100in}{4.833570in}}%
\pgfpathlineto{\pgfqpoint{2.593400in}{4.840920in}}%
\pgfpathlineto{\pgfqpoint{2.602700in}{4.845842in}}%
\pgfpathlineto{\pgfqpoint{2.612000in}{4.848257in}}%
\pgfpathlineto{\pgfqpoint{2.621300in}{4.848125in}}%
\pgfpathlineto{\pgfqpoint{2.630600in}{4.845434in}}%
\pgfpathlineto{\pgfqpoint{2.639900in}{4.840210in}}%
\pgfpathlineto{\pgfqpoint{2.649200in}{4.832510in}}%
\pgfpathlineto{\pgfqpoint{2.661600in}{4.818563in}}%
\pgfpathlineto{\pgfqpoint{2.674000in}{4.800737in}}%
\pgfpathlineto{\pgfqpoint{2.689500in}{4.773787in}}%
\pgfpathlineto{\pgfqpoint{2.708100in}{4.736454in}}%
\pgfpathlineto{\pgfqpoint{2.757700in}{4.634051in}}%
\pgfpathlineto{\pgfqpoint{2.773200in}{4.608515in}}%
\pgfpathlineto{\pgfqpoint{2.785600in}{4.592161in}}%
\pgfpathlineto{\pgfqpoint{2.798000in}{4.579963in}}%
\pgfpathlineto{\pgfqpoint{2.807300in}{4.573745in}}%
\pgfpathlineto{\pgfqpoint{2.816600in}{4.570140in}}%
\pgfpathlineto{\pgfqpoint{2.825900in}{4.569201in}}%
\pgfpathlineto{\pgfqpoint{2.835200in}{4.570948in}}%
\pgfpathlineto{\pgfqpoint{2.844500in}{4.575372in}}%
\pgfpathlineto{\pgfqpoint{2.853800in}{4.582433in}}%
\pgfpathlineto{\pgfqpoint{2.863100in}{4.592054in}}%
\pgfpathlineto{\pgfqpoint{2.875500in}{4.608659in}}%
\pgfpathlineto{\pgfqpoint{2.887900in}{4.629198in}}%
\pgfpathlineto{\pgfqpoint{2.903400in}{4.659496in}}%
\pgfpathlineto{\pgfqpoint{2.925100in}{4.707643in}}%
\pgfpathlineto{\pgfqpoint{2.962300in}{4.790684in}}%
\pgfpathlineto{\pgfqpoint{2.977800in}{4.819784in}}%
\pgfpathlineto{\pgfqpoint{2.990200in}{4.839033in}}%
\pgfpathlineto{\pgfqpoint{3.002600in}{4.854079in}}%
\pgfpathlineto{\pgfqpoint{3.011900in}{4.862366in}}%
\pgfpathlineto{\pgfqpoint{3.021200in}{4.867962in}}%
\pgfpathlineto{\pgfqpoint{3.030500in}{4.870797in}}%
\pgfpathlineto{\pgfqpoint{3.039800in}{4.870835in}}%
\pgfpathlineto{\pgfqpoint{3.049100in}{4.868066in}}%
\pgfpathlineto{\pgfqpoint{3.058400in}{4.862508in}}%
\pgfpathlineto{\pgfqpoint{3.067700in}{4.854208in}}%
\pgfpathlineto{\pgfqpoint{3.077000in}{4.843245in}}%
\pgfpathlineto{\pgfqpoint{3.089400in}{4.824706in}}%
\pgfpathlineto{\pgfqpoint{3.101800in}{4.802089in}}%
\pgfpathlineto{\pgfqpoint{3.117300in}{4.769061in}}%
\pgfpathlineto{\pgfqpoint{3.142100in}{4.709406in}}%
\pgfpathlineto{\pgfqpoint{3.173100in}{4.635357in}}%
\pgfpathlineto{\pgfqpoint{3.188600in}{4.603677in}}%
\pgfpathlineto{\pgfqpoint{3.201000in}{4.582509in}}%
\pgfpathlineto{\pgfqpoint{3.213400in}{4.565706in}}%
\pgfpathlineto{\pgfqpoint{3.222700in}{4.556214in}}%
\pgfpathlineto{\pgfqpoint{3.232000in}{4.549513in}}%
\pgfpathlineto{\pgfqpoint{3.241300in}{4.545671in}}%
\pgfpathlineto{\pgfqpoint{3.250600in}{4.544727in}}%
\pgfpathlineto{\pgfqpoint{3.259900in}{4.546697in}}%
\pgfpathlineto{\pgfqpoint{3.269200in}{4.551572in}}%
\pgfpathlineto{\pgfqpoint{3.278500in}{4.559322in}}%
\pgfpathlineto{\pgfqpoint{3.287800in}{4.569888in}}%
\pgfpathlineto{\pgfqpoint{3.300200in}{4.588182in}}%
\pgfpathlineto{\pgfqpoint{3.312600in}{4.610941in}}%
\pgfpathlineto{\pgfqpoint{3.328100in}{4.644801in}}%
\pgfpathlineto{\pgfqpoint{3.346700in}{4.691197in}}%
\pgfpathlineto{\pgfqpoint{3.390100in}{4.802636in}}%
\pgfpathlineto{\pgfqpoint{3.405600in}{4.835986in}}%
\pgfpathlineto{\pgfqpoint{3.418000in}{4.858152in}}%
\pgfpathlineto{\pgfqpoint{3.430400in}{4.875685in}}%
\pgfpathlineto{\pgfqpoint{3.439700in}{4.885572in}}%
\pgfpathlineto{\pgfqpoint{3.449000in}{4.892556in}}%
\pgfpathlineto{\pgfqpoint{3.458300in}{4.896582in}}%
\pgfpathlineto{\pgfqpoint{3.467600in}{4.897618in}}%
\pgfpathlineto{\pgfqpoint{3.476900in}{4.895654in}}%
\pgfpathlineto{\pgfqpoint{3.486200in}{4.890695in}}%
\pgfpathlineto{\pgfqpoint{3.495500in}{4.882763in}}%
\pgfpathlineto{\pgfqpoint{3.504800in}{4.871906in}}%
\pgfpathlineto{\pgfqpoint{3.517200in}{4.853019in}}%
\pgfpathlineto{\pgfqpoint{3.529600in}{4.829391in}}%
\pgfpathlineto{\pgfqpoint{3.545100in}{4.793995in}}%
\pgfpathlineto{\pgfqpoint{3.563700in}{4.745045in}}%
\pgfpathlineto{\pgfqpoint{3.613300in}{4.610115in}}%
\pgfpathlineto{\pgfqpoint{3.628800in}{4.576020in}}%
\pgfpathlineto{\pgfqpoint{3.641200in}{4.553723in}}%
\pgfpathlineto{\pgfqpoint{3.653600in}{4.536379in}}%
\pgfpathlineto{\pgfqpoint{3.662900in}{4.526792in}}%
\pgfpathlineto{\pgfqpoint{3.672200in}{4.520212in}}%
\pgfpathlineto{\pgfqpoint{3.681500in}{4.516672in}}%
\pgfpathlineto{\pgfqpoint{3.690800in}{4.516188in}}%
\pgfpathlineto{\pgfqpoint{3.700100in}{4.518765in}}%
\pgfpathlineto{\pgfqpoint{3.709400in}{4.524399in}}%
\pgfpathlineto{\pgfqpoint{3.718700in}{4.533072in}}%
\pgfpathlineto{\pgfqpoint{3.728000in}{4.544750in}}%
\pgfpathlineto{\pgfqpoint{3.740400in}{4.564880in}}%
\pgfpathlineto{\pgfqpoint{3.752800in}{4.589954in}}%
\pgfpathlineto{\pgfqpoint{3.768300in}{4.627484in}}%
\pgfpathlineto{\pgfqpoint{3.786900in}{4.679472in}}%
\pgfpathlineto{\pgfqpoint{3.836500in}{4.823619in}}%
\pgfpathlineto{\pgfqpoint{3.852000in}{4.860356in}}%
\pgfpathlineto{\pgfqpoint{3.864400in}{4.884591in}}%
\pgfpathlineto{\pgfqpoint{3.876800in}{4.903727in}}%
\pgfpathlineto{\pgfqpoint{3.886100in}{4.914581in}}%
\pgfpathlineto{\pgfqpoint{3.895400in}{4.922380in}}%
\pgfpathlineto{\pgfqpoint{3.904700in}{4.927101in}}%
\pgfpathlineto{\pgfqpoint{3.914000in}{4.928737in}}%
\pgfpathlineto{\pgfqpoint{3.923300in}{4.927289in}}%
\pgfpathlineto{\pgfqpoint{3.932600in}{4.922755in}}%
\pgfpathlineto{\pgfqpoint{3.941900in}{4.915138in}}%
\pgfpathlineto{\pgfqpoint{3.951200in}{4.904447in}}%
\pgfpathlineto{\pgfqpoint{3.960500in}{4.890711in}}%
\pgfpathlineto{\pgfqpoint{3.972900in}{4.867765in}}%
\pgfpathlineto{\pgfqpoint{3.985300in}{4.839800in}}%
\pgfpathlineto{\pgfqpoint{4.000800in}{4.798605in}}%
\pgfpathlineto{\pgfqpoint{4.019400in}{4.742359in}}%
\pgfpathlineto{\pgfqpoint{4.062800in}{4.607261in}}%
\pgfpathlineto{\pgfqpoint{4.078300in}{4.566628in}}%
\pgfpathlineto{\pgfqpoint{4.090700in}{4.539285in}}%
\pgfpathlineto{\pgfqpoint{4.103100in}{4.517091in}}%
\pgfpathlineto{\pgfqpoint{4.112400in}{4.503979in}}%
\pgfpathlineto{\pgfqpoint{4.121700in}{4.493944in}}%
\pgfpathlineto{\pgfqpoint{4.131000in}{4.486993in}}%
\pgfpathlineto{\pgfqpoint{4.140300in}{4.483120in}}%
\pgfpathlineto{\pgfqpoint{4.149600in}{4.482315in}}%
\pgfpathlineto{\pgfqpoint{4.158900in}{4.484573in}}%
\pgfpathlineto{\pgfqpoint{4.168200in}{4.489899in}}%
\pgfpathlineto{\pgfqpoint{4.177500in}{4.498300in}}%
\pgfpathlineto{\pgfqpoint{4.186800in}{4.509785in}}%
\pgfpathlineto{\pgfqpoint{4.196100in}{4.524349in}}%
\pgfpathlineto{\pgfqpoint{4.208500in}{4.548499in}}%
\pgfpathlineto{\pgfqpoint{4.220900in}{4.577847in}}%
\pgfpathlineto{\pgfqpoint{4.236400in}{4.621109in}}%
\pgfpathlineto{\pgfqpoint{4.255000in}{4.680404in}}%
\pgfpathlineto{\pgfqpoint{4.301500in}{4.833455in}}%
\pgfpathlineto{\pgfqpoint{4.317000in}{4.875821in}}%
\pgfpathlineto{\pgfqpoint{4.329400in}{4.904253in}}%
\pgfpathlineto{\pgfqpoint{4.341800in}{4.927369in}}%
\pgfpathlineto{\pgfqpoint{4.354200in}{4.945019in}}%
\pgfpathlineto{\pgfqpoint{4.363500in}{4.954669in}}%
\pgfpathlineto{\pgfqpoint{4.372800in}{4.961273in}}%
\pgfpathlineto{\pgfqpoint{4.382100in}{4.964863in}}%
\pgfpathlineto{\pgfqpoint{4.391400in}{4.965463in}}%
\pgfpathlineto{\pgfqpoint{4.400700in}{4.963080in}}%
\pgfpathlineto{\pgfqpoint{4.410000in}{4.957707in}}%
\pgfpathlineto{\pgfqpoint{4.419300in}{4.949323in}}%
\pgfpathlineto{\pgfqpoint{4.428600in}{4.937894in}}%
\pgfpathlineto{\pgfqpoint{4.437900in}{4.923394in}}%
\pgfpathlineto{\pgfqpoint{4.450300in}{4.899269in}}%
\pgfpathlineto{\pgfqpoint{4.462700in}{4.869758in}}%
\pgfpathlineto{\pgfqpoint{4.478200in}{4.825809in}}%
\pgfpathlineto{\pgfqpoint{4.496800in}{4.764634in}}%
\pgfpathlineto{\pgfqpoint{4.552600in}{4.572402in}}%
\pgfpathlineto{\pgfqpoint{4.568100in}{4.529964in}}%
\pgfpathlineto{\pgfqpoint{4.580500in}{4.501862in}}%
\pgfpathlineto{\pgfqpoint{4.592900in}{4.479215in}}%
\pgfpathlineto{\pgfqpoint{4.605300in}{4.462014in}}%
\pgfpathlineto{\pgfqpoint{4.614600in}{4.452617in}}%
\pgfpathlineto{\pgfqpoint{4.623900in}{4.446153in}}%
\pgfpathlineto{\pgfqpoint{4.633200in}{4.442566in}}%
\pgfpathlineto{\pgfqpoint{4.642500in}{4.441817in}}%
\pgfpathlineto{\pgfqpoint{4.651800in}{4.443893in}}%
\pgfpathlineto{\pgfqpoint{4.661100in}{4.448803in}}%
\pgfpathlineto{\pgfqpoint{4.670400in}{4.456580in}}%
\pgfpathlineto{\pgfqpoint{4.679700in}{4.467278in}}%
\pgfpathlineto{\pgfqpoint{4.689000in}{4.480959in}}%
\pgfpathlineto{\pgfqpoint{4.701400in}{4.503939in}}%
\pgfpathlineto{\pgfqpoint{4.713800in}{4.532390in}}%
\pgfpathlineto{\pgfqpoint{4.726200in}{4.566207in}}%
\pgfpathlineto{\pgfqpoint{4.741700in}{4.615374in}}%
\pgfpathlineto{\pgfqpoint{4.760300in}{4.682122in}}%
\pgfpathlineto{\pgfqpoint{4.803700in}{4.842040in}}%
\pgfpathlineto{\pgfqpoint{4.819200in}{4.890320in}}%
\pgfpathlineto{\pgfqpoint{4.831600in}{4.923254in}}%
\pgfpathlineto{\pgfqpoint{4.844000in}{4.950755in}}%
\pgfpathlineto{\pgfqpoint{4.856400in}{4.972816in}}%
\pgfpathlineto{\pgfqpoint{4.868800in}{4.989607in}}%
\pgfpathlineto{\pgfqpoint{4.878100in}{4.998884in}}%
\pgfpathlineto{\pgfqpoint{4.887400in}{5.005428in}}%
\pgfpathlineto{\pgfqpoint{4.896700in}{5.009321in}}%
\pgfpathlineto{\pgfqpoint{4.906000in}{5.010620in}}%
\pgfpathlineto{\pgfqpoint{4.915300in}{5.009350in}}%
\pgfpathlineto{\pgfqpoint{4.924600in}{5.005507in}}%
\pgfpathlineto{\pgfqpoint{4.933900in}{4.999054in}}%
\pgfpathlineto{\pgfqpoint{4.943200in}{4.989927in}}%
\pgfpathlineto{\pgfqpoint{4.952500in}{4.978038in}}%
\pgfpathlineto{\pgfqpoint{4.964900in}{4.957718in}}%
\pgfpathlineto{\pgfqpoint{4.977300in}{4.932083in}}%
\pgfpathlineto{\pgfqpoint{4.989700in}{4.900989in}}%
\pgfpathlineto{\pgfqpoint{5.002100in}{4.864497in}}%
\pgfpathlineto{\pgfqpoint{5.017600in}{4.811932in}}%
\pgfpathlineto{\pgfqpoint{5.039300in}{4.728879in}}%
\pgfpathlineto{\pgfqpoint{5.076500in}{4.584797in}}%
\pgfpathlineto{\pgfqpoint{5.092000in}{4.533304in}}%
\pgfpathlineto{\pgfqpoint{5.107500in}{4.489847in}}%
\pgfpathlineto{\pgfqpoint{5.119900in}{4.461214in}}%
\pgfpathlineto{\pgfqpoint{5.132300in}{4.437916in}}%
\pgfpathlineto{\pgfqpoint{5.144700in}{4.419654in}}%
\pgfpathlineto{\pgfqpoint{5.157100in}{4.406078in}}%
\pgfpathlineto{\pgfqpoint{5.166400in}{4.398771in}}%
\pgfpathlineto{\pgfqpoint{5.175700in}{4.393797in}}%
\pgfpathlineto{\pgfqpoint{5.185000in}{4.391072in}}%
\pgfpathlineto{\pgfqpoint{5.194300in}{4.390540in}}%
\pgfpathlineto{\pgfqpoint{5.203600in}{4.392182in}}%
\pgfpathlineto{\pgfqpoint{5.212900in}{4.396013in}}%
\pgfpathlineto{\pgfqpoint{5.222200in}{4.402080in}}%
\pgfpathlineto{\pgfqpoint{5.231500in}{4.410464in}}%
\pgfpathlineto{\pgfqpoint{5.240800in}{4.421275in}}%
\pgfpathlineto{\pgfqpoint{5.253200in}{4.439694in}}%
\pgfpathlineto{\pgfqpoint{5.265600in}{4.463007in}}%
\pgfpathlineto{\pgfqpoint{5.278000in}{4.491540in}}%
\pgfpathlineto{\pgfqpoint{5.290400in}{4.525508in}}%
\pgfpathlineto{\pgfqpoint{5.305900in}{4.575538in}}%
\pgfpathlineto{\pgfqpoint{5.324500in}{4.645292in}}%
\pgfpathlineto{\pgfqpoint{5.386500in}{4.891035in}}%
\pgfpathlineto{\pgfqpoint{5.402000in}{4.938944in}}%
\pgfpathlineto{\pgfqpoint{5.414400in}{4.971147in}}%
\pgfpathlineto{\pgfqpoint{5.426800in}{4.998048in}}%
\pgfpathlineto{\pgfqpoint{5.439200in}{5.020022in}}%
\pgfpathlineto{\pgfqpoint{5.451600in}{5.037537in}}%
\pgfpathlineto{\pgfqpoint{5.464000in}{5.051061in}}%
\pgfpathlineto{\pgfqpoint{5.476400in}{5.061011in}}%
\pgfpathlineto{\pgfqpoint{5.488800in}{5.067725in}}%
\pgfpathlineto{\pgfqpoint{5.501200in}{5.071450in}}%
\pgfpathlineto{\pgfqpoint{5.513600in}{5.072337in}}%
\pgfpathlineto{\pgfqpoint{5.526000in}{5.070442in}}%
\pgfpathlineto{\pgfqpoint{5.538400in}{5.065722in}}%
\pgfpathlineto{\pgfqpoint{5.550800in}{5.058041in}}%
\pgfpathlineto{\pgfqpoint{5.563200in}{5.047171in}}%
\pgfpathlineto{\pgfqpoint{5.575600in}{5.032794in}}%
\pgfpathlineto{\pgfqpoint{5.588000in}{5.014514in}}%
\pgfpathlineto{\pgfqpoint{5.600400in}{4.991880in}}%
\pgfpathlineto{\pgfqpoint{5.612800in}{4.964431in}}%
\pgfpathlineto{\pgfqpoint{5.625200in}{4.931765in}}%
\pgfpathlineto{\pgfqpoint{5.640700in}{4.883279in}}%
\pgfpathlineto{\pgfqpoint{5.656200in}{4.826618in}}%
\pgfpathlineto{\pgfqpoint{5.677900in}{4.736782in}}%
\pgfpathlineto{\pgfqpoint{5.718200in}{4.567194in}}%
\pgfpathlineto{\pgfqpoint{5.733700in}{4.511703in}}%
\pgfpathlineto{\pgfqpoint{5.749200in}{4.464552in}}%
\pgfpathlineto{\pgfqpoint{5.761600in}{4.432880in}}%
\pgfpathlineto{\pgfqpoint{5.774000in}{4.406190in}}%
\pgfpathlineto{\pgfqpoint{5.786400in}{4.383927in}}%
\pgfpathlineto{\pgfqpoint{5.798800in}{4.365487in}}%
\pgfpathlineto{\pgfqpoint{5.811200in}{4.350280in}}%
\pgfpathlineto{\pgfqpoint{5.826700in}{4.335011in}}%
\pgfpathlineto{\pgfqpoint{5.842200in}{4.323062in}}%
\pgfpathlineto{\pgfqpoint{5.857700in}{4.313695in}}%
\pgfpathlineto{\pgfqpoint{5.876300in}{4.305037in}}%
\pgfpathlineto{\pgfqpoint{5.898000in}{4.297500in}}%
\pgfpathlineto{\pgfqpoint{5.925900in}{4.290427in}}%
\pgfpathlineto{\pgfqpoint{5.972400in}{4.281576in}}%
\pgfpathlineto{\pgfqpoint{6.015800in}{4.272736in}}%
\pgfpathlineto{\pgfqpoint{6.043700in}{4.264727in}}%
\pgfpathlineto{\pgfqpoint{6.065400in}{4.256093in}}%
\pgfpathlineto{\pgfqpoint{6.084000in}{4.246203in}}%
\pgfpathlineto{\pgfqpoint{6.099500in}{4.235583in}}%
\pgfpathlineto{\pgfqpoint{6.115000in}{4.222157in}}%
\pgfpathlineto{\pgfqpoint{6.130500in}{4.205183in}}%
\pgfpathlineto{\pgfqpoint{6.142900in}{4.188451in}}%
\pgfpathlineto{\pgfqpoint{6.155300in}{4.168353in}}%
\pgfpathlineto{\pgfqpoint{6.167700in}{4.144318in}}%
\pgfpathlineto{\pgfqpoint{6.180100in}{4.115769in}}%
\pgfpathlineto{\pgfqpoint{6.192500in}{4.082188in}}%
\pgfpathlineto{\pgfqpoint{6.208000in}{4.032628in}}%
\pgfpathlineto{\pgfqpoint{6.223500in}{3.974740in}}%
\pgfpathlineto{\pgfqpoint{6.242100in}{3.896278in}}%
\pgfpathlineto{\pgfqpoint{6.288600in}{3.694269in}}%
\pgfpathlineto{\pgfqpoint{6.304100in}{3.637637in}}%
\pgfpathlineto{\pgfqpoint{6.319600in}{3.589439in}}%
\pgfpathlineto{\pgfqpoint{6.335100in}{3.549410in}}%
\pgfpathlineto{\pgfqpoint{6.350600in}{3.516500in}}%
\pgfpathlineto{\pgfqpoint{6.366100in}{3.489337in}}%
\pgfpathlineto{\pgfqpoint{6.381600in}{3.466512in}}%
\pgfpathlineto{\pgfqpoint{6.400200in}{3.442987in}}%
\pgfpathlineto{\pgfqpoint{6.468400in}{3.361346in}}%
\pgfpathlineto{\pgfqpoint{6.483900in}{3.337129in}}%
\pgfpathlineto{\pgfqpoint{6.499400in}{3.308215in}}%
\pgfpathlineto{\pgfqpoint{6.511800in}{3.280799in}}%
\pgfpathlineto{\pgfqpoint{6.524200in}{3.248851in}}%
\pgfpathlineto{\pgfqpoint{6.539700in}{3.201708in}}%
\pgfpathlineto{\pgfqpoint{6.555200in}{3.146089in}}%
\pgfpathlineto{\pgfqpoint{6.573800in}{3.069055in}}%
\pgfpathlineto{\pgfqpoint{6.638900in}{2.785370in}}%
\pgfpathlineto{\pgfqpoint{6.654400in}{2.733125in}}%
\pgfpathlineto{\pgfqpoint{6.669900in}{2.689133in}}%
\pgfpathlineto{\pgfqpoint{6.685400in}{2.652323in}}%
\pgfpathlineto{\pgfqpoint{6.700900in}{2.621177in}}%
\pgfpathlineto{\pgfqpoint{6.719500in}{2.588973in}}%
\pgfpathlineto{\pgfqpoint{6.756700in}{2.531058in}}%
\pgfpathlineto{\pgfqpoint{6.778400in}{2.494596in}}%
\pgfpathlineto{\pgfqpoint{6.793900in}{2.464442in}}%
\pgfpathlineto{\pgfqpoint{6.809400in}{2.429100in}}%
\pgfpathlineto{\pgfqpoint{6.824900in}{2.387065in}}%
\pgfpathlineto{\pgfqpoint{6.840400in}{2.337108in}}%
\pgfpathlineto{\pgfqpoint{6.855900in}{2.278680in}}%
\pgfpathlineto{\pgfqpoint{6.874500in}{2.198466in}}%
\pgfpathlineto{\pgfqpoint{6.933400in}{1.933100in}}%
\pgfpathlineto{\pgfqpoint{6.948900in}{1.877245in}}%
\pgfpathlineto{\pgfqpoint{6.964400in}{1.829714in}}%
\pgfpathlineto{\pgfqpoint{6.979900in}{1.789454in}}%
\pgfpathlineto{\pgfqpoint{6.995400in}{1.754828in}}%
\pgfpathlineto{\pgfqpoint{7.017100in}{1.712327in}}%
\pgfpathlineto{\pgfqpoint{7.060500in}{1.629695in}}%
\pgfpathlineto{\pgfqpoint{7.076000in}{1.595074in}}%
\pgfpathlineto{\pgfqpoint{7.091500in}{1.554943in}}%
\pgfpathlineto{\pgfqpoint{7.107000in}{1.507713in}}%
\pgfpathlineto{\pgfqpoint{7.122500in}{1.452263in}}%
\pgfpathlineto{\pgfqpoint{7.138000in}{1.388450in}}%
\pgfpathlineto{\pgfqpoint{7.159700in}{1.288069in}}%
\pgfpathlineto{\pgfqpoint{7.196900in}{1.113200in}}%
\pgfpathlineto{\pgfqpoint{7.200000in}{1.099905in}}%
\pgfpathlineto{\pgfqpoint{7.200000in}{1.099905in}}%
\pgfusepath{stroke}%
\end{pgfscope}%
\begin{pgfscope}%
\pgfpathrectangle{\pgfqpoint{1.000000in}{0.600000in}}{\pgfqpoint{6.200000in}{4.800000in}}%
\pgfusepath{clip}%
\pgfsetrectcap%
\pgfsetroundjoin%
\pgfsetlinewidth{1.003750pt}%
\definecolor{currentstroke}{rgb}{0.000000,0.000000,1.000000}%
\pgfsetstrokecolor{currentstroke}%
\pgfsetdash{}{0pt}%
\pgfpathmoveto{\pgfqpoint{1.000000in}{4.786094in}}%
\pgfpathlineto{\pgfqpoint{1.012400in}{4.784386in}}%
\pgfpathlineto{\pgfqpoint{1.024800in}{4.780021in}}%
\pgfpathlineto{\pgfqpoint{1.037200in}{4.773174in}}%
\pgfpathlineto{\pgfqpoint{1.052700in}{4.761544in}}%
\pgfpathlineto{\pgfqpoint{1.071300in}{4.744051in}}%
\pgfpathlineto{\pgfqpoint{1.099200in}{4.713918in}}%
\pgfpathlineto{\pgfqpoint{1.127100in}{4.684524in}}%
\pgfpathlineto{\pgfqpoint{1.145700in}{4.668218in}}%
\pgfpathlineto{\pgfqpoint{1.161200in}{4.657833in}}%
\pgfpathlineto{\pgfqpoint{1.173600in}{4.652043in}}%
\pgfpathlineto{\pgfqpoint{1.186000in}{4.648699in}}%
\pgfpathlineto{\pgfqpoint{1.198400in}{4.647900in}}%
\pgfpathlineto{\pgfqpoint{1.210800in}{4.649651in}}%
\pgfpathlineto{\pgfqpoint{1.223200in}{4.653868in}}%
\pgfpathlineto{\pgfqpoint{1.235600in}{4.660376in}}%
\pgfpathlineto{\pgfqpoint{1.251100in}{4.671332in}}%
\pgfpathlineto{\pgfqpoint{1.269700in}{4.687696in}}%
\pgfpathlineto{\pgfqpoint{1.300700in}{4.718841in}}%
\pgfpathlineto{\pgfqpoint{1.325500in}{4.742775in}}%
\pgfpathlineto{\pgfqpoint{1.344100in}{4.757684in}}%
\pgfpathlineto{\pgfqpoint{1.359600in}{4.767087in}}%
\pgfpathlineto{\pgfqpoint{1.372000in}{4.772248in}}%
\pgfpathlineto{\pgfqpoint{1.384400in}{4.775121in}}%
\pgfpathlineto{\pgfqpoint{1.396800in}{4.775623in}}%
\pgfpathlineto{\pgfqpoint{1.409200in}{4.773757in}}%
\pgfpathlineto{\pgfqpoint{1.421600in}{4.769612in}}%
\pgfpathlineto{\pgfqpoint{1.434000in}{4.763361in}}%
\pgfpathlineto{\pgfqpoint{1.449500in}{4.752975in}}%
\pgfpathlineto{\pgfqpoint{1.468100in}{4.737617in}}%
\pgfpathlineto{\pgfqpoint{1.539400in}{4.675183in}}%
\pgfpathlineto{\pgfqpoint{1.554900in}{4.666186in}}%
\pgfpathlineto{\pgfqpoint{1.567300in}{4.661151in}}%
\pgfpathlineto{\pgfqpoint{1.579700in}{4.658227in}}%
\pgfpathlineto{\pgfqpoint{1.592100in}{4.657502in}}%
\pgfpathlineto{\pgfqpoint{1.604500in}{4.658981in}}%
\pgfpathlineto{\pgfqpoint{1.616900in}{4.662590in}}%
\pgfpathlineto{\pgfqpoint{1.632400in}{4.669858in}}%
\pgfpathlineto{\pgfqpoint{1.647900in}{4.679748in}}%
\pgfpathlineto{\pgfqpoint{1.669600in}{4.696791in}}%
\pgfpathlineto{\pgfqpoint{1.728500in}{4.745596in}}%
\pgfpathlineto{\pgfqpoint{1.744000in}{4.755120in}}%
\pgfpathlineto{\pgfqpoint{1.759500in}{4.762034in}}%
\pgfpathlineto{\pgfqpoint{1.771900in}{4.765428in}}%
\pgfpathlineto{\pgfqpoint{1.784300in}{4.766801in}}%
\pgfpathlineto{\pgfqpoint{1.796700in}{4.766119in}}%
\pgfpathlineto{\pgfqpoint{1.809100in}{4.763428in}}%
\pgfpathlineto{\pgfqpoint{1.824600in}{4.757426in}}%
\pgfpathlineto{\pgfqpoint{1.840100in}{4.748862in}}%
\pgfpathlineto{\pgfqpoint{1.858700in}{4.735977in}}%
\pgfpathlineto{\pgfqpoint{1.892800in}{4.708806in}}%
\pgfpathlineto{\pgfqpoint{1.917600in}{4.690061in}}%
\pgfpathlineto{\pgfqpoint{1.936200in}{4.678587in}}%
\pgfpathlineto{\pgfqpoint{1.951700in}{4.671521in}}%
\pgfpathlineto{\pgfqpoint{1.967200in}{4.667148in}}%
\pgfpathlineto{\pgfqpoint{1.979600in}{4.665747in}}%
\pgfpathlineto{\pgfqpoint{1.992000in}{4.666252in}}%
\pgfpathlineto{\pgfqpoint{2.004400in}{4.668628in}}%
\pgfpathlineto{\pgfqpoint{2.019900in}{4.674057in}}%
\pgfpathlineto{\pgfqpoint{2.035400in}{4.681882in}}%
\pgfpathlineto{\pgfqpoint{2.054000in}{4.693723in}}%
\pgfpathlineto{\pgfqpoint{2.088100in}{4.718835in}}%
\pgfpathlineto{\pgfqpoint{2.116000in}{4.738218in}}%
\pgfpathlineto{\pgfqpoint{2.134600in}{4.748484in}}%
\pgfpathlineto{\pgfqpoint{2.150100in}{4.754652in}}%
\pgfpathlineto{\pgfqpoint{2.165600in}{4.758278in}}%
\pgfpathlineto{\pgfqpoint{2.181100in}{4.759178in}}%
\pgfpathlineto{\pgfqpoint{2.196600in}{4.757332in}}%
\pgfpathlineto{\pgfqpoint{2.212100in}{4.752879in}}%
\pgfpathlineto{\pgfqpoint{2.227600in}{4.746117in}}%
\pgfpathlineto{\pgfqpoint{2.246200in}{4.735573in}}%
\pgfpathlineto{\pgfqpoint{2.274100in}{4.716856in}}%
\pgfpathlineto{\pgfqpoint{2.308200in}{4.694239in}}%
\pgfpathlineto{\pgfqpoint{2.326800in}{4.684261in}}%
\pgfpathlineto{\pgfqpoint{2.342300in}{4.678054in}}%
\pgfpathlineto{\pgfqpoint{2.357800in}{4.674144in}}%
\pgfpathlineto{\pgfqpoint{2.373300in}{4.672735in}}%
\pgfpathlineto{\pgfqpoint{2.388800in}{4.673882in}}%
\pgfpathlineto{\pgfqpoint{2.404300in}{4.677487in}}%
\pgfpathlineto{\pgfqpoint{2.419800in}{4.683304in}}%
\pgfpathlineto{\pgfqpoint{2.438400in}{4.692666in}}%
\pgfpathlineto{\pgfqpoint{2.466300in}{4.709730in}}%
\pgfpathlineto{\pgfqpoint{2.506600in}{4.734370in}}%
\pgfpathlineto{\pgfqpoint{2.525200in}{4.743254in}}%
\pgfpathlineto{\pgfqpoint{2.540700in}{4.748627in}}%
\pgfpathlineto{\pgfqpoint{2.556200in}{4.751828in}}%
\pgfpathlineto{\pgfqpoint{2.571700in}{4.752689in}}%
\pgfpathlineto{\pgfqpoint{2.587200in}{4.751189in}}%
\pgfpathlineto{\pgfqpoint{2.602700in}{4.747444in}}%
\pgfpathlineto{\pgfqpoint{2.621300in}{4.740352in}}%
\pgfpathlineto{\pgfqpoint{2.643000in}{4.729351in}}%
\pgfpathlineto{\pgfqpoint{2.717400in}{4.688747in}}%
\pgfpathlineto{\pgfqpoint{2.736000in}{4.682548in}}%
\pgfpathlineto{\pgfqpoint{2.751500in}{4.679577in}}%
\pgfpathlineto{\pgfqpoint{2.767000in}{4.678772in}}%
\pgfpathlineto{\pgfqpoint{2.782500in}{4.680156in}}%
\pgfpathlineto{\pgfqpoint{2.798000in}{4.683618in}}%
\pgfpathlineto{\pgfqpoint{2.816600in}{4.690180in}}%
\pgfpathlineto{\pgfqpoint{2.838300in}{4.700359in}}%
\pgfpathlineto{\pgfqpoint{2.912700in}{4.737919in}}%
\pgfpathlineto{\pgfqpoint{2.931300in}{4.743648in}}%
\pgfpathlineto{\pgfqpoint{2.946800in}{4.746390in}}%
\pgfpathlineto{\pgfqpoint{2.962300in}{4.747125in}}%
\pgfpathlineto{\pgfqpoint{2.977800in}{4.745832in}}%
\pgfpathlineto{\pgfqpoint{2.996400in}{4.741759in}}%
\pgfpathlineto{\pgfqpoint{3.015000in}{4.735315in}}%
\pgfpathlineto{\pgfqpoint{3.039800in}{4.724098in}}%
\pgfpathlineto{\pgfqpoint{3.101800in}{4.694618in}}%
\pgfpathlineto{\pgfqpoint{3.120400in}{4.688597in}}%
\pgfpathlineto{\pgfqpoint{3.139000in}{4.684937in}}%
\pgfpathlineto{\pgfqpoint{3.154500in}{4.683903in}}%
\pgfpathlineto{\pgfqpoint{3.170000in}{4.684751in}}%
\pgfpathlineto{\pgfqpoint{3.188600in}{4.688142in}}%
\pgfpathlineto{\pgfqpoint{3.207200in}{4.693796in}}%
\pgfpathlineto{\pgfqpoint{3.232000in}{4.703917in}}%
\pgfpathlineto{\pgfqpoint{3.300200in}{4.733621in}}%
\pgfpathlineto{\pgfqpoint{3.318800in}{4.738826in}}%
\pgfpathlineto{\pgfqpoint{3.337400in}{4.741782in}}%
\pgfpathlineto{\pgfqpoint{3.356000in}{4.742263in}}%
\pgfpathlineto{\pgfqpoint{3.374600in}{4.740263in}}%
\pgfpathlineto{\pgfqpoint{3.393200in}{4.735994in}}%
\pgfpathlineto{\pgfqpoint{3.414900in}{4.728698in}}%
\pgfpathlineto{\pgfqpoint{3.449000in}{4.714401in}}%
\pgfpathlineto{\pgfqpoint{3.486200in}{4.699355in}}%
\pgfpathlineto{\pgfqpoint{3.507900in}{4.692898in}}%
\pgfpathlineto{\pgfqpoint{3.526500in}{4.689477in}}%
\pgfpathlineto{\pgfqpoint{3.545100in}{4.688304in}}%
\pgfpathlineto{\pgfqpoint{3.563700in}{4.689447in}}%
\pgfpathlineto{\pgfqpoint{3.582300in}{4.692772in}}%
\pgfpathlineto{\pgfqpoint{3.604000in}{4.698969in}}%
\pgfpathlineto{\pgfqpoint{3.635000in}{4.710595in}}%
\pgfpathlineto{\pgfqpoint{3.681500in}{4.728226in}}%
\pgfpathlineto{\pgfqpoint{3.703200in}{4.734157in}}%
\pgfpathlineto{\pgfqpoint{3.721800in}{4.737275in}}%
\pgfpathlineto{\pgfqpoint{3.740400in}{4.738311in}}%
\pgfpathlineto{\pgfqpoint{3.759000in}{4.737203in}}%
\pgfpathlineto{\pgfqpoint{3.780700in}{4.733384in}}%
\pgfpathlineto{\pgfqpoint{3.805500in}{4.726329in}}%
\pgfpathlineto{\pgfqpoint{3.845800in}{4.711760in}}%
\pgfpathlineto{\pgfqpoint{3.879900in}{4.700355in}}%
\pgfpathlineto{\pgfqpoint{3.901600in}{4.695214in}}%
\pgfpathlineto{\pgfqpoint{3.923300in}{4.692454in}}%
\pgfpathlineto{\pgfqpoint{3.941900in}{4.692213in}}%
\pgfpathlineto{\pgfqpoint{3.960500in}{4.693932in}}%
\pgfpathlineto{\pgfqpoint{3.982200in}{4.698165in}}%
\pgfpathlineto{\pgfqpoint{4.010100in}{4.706249in}}%
\pgfpathlineto{\pgfqpoint{4.084500in}{4.729566in}}%
\pgfpathlineto{\pgfqpoint{4.106200in}{4.733374in}}%
\pgfpathlineto{\pgfqpoint{4.127900in}{4.734821in}}%
\pgfpathlineto{\pgfqpoint{4.149600in}{4.733774in}}%
\pgfpathlineto{\pgfqpoint{4.171300in}{4.730401in}}%
\pgfpathlineto{\pgfqpoint{4.196100in}{4.724277in}}%
\pgfpathlineto{\pgfqpoint{4.248800in}{4.707980in}}%
\pgfpathlineto{\pgfqpoint{4.276700in}{4.700692in}}%
\pgfpathlineto{\pgfqpoint{4.301500in}{4.696569in}}%
\pgfpathlineto{\pgfqpoint{4.323200in}{4.695293in}}%
\pgfpathlineto{\pgfqpoint{4.344900in}{4.696323in}}%
\pgfpathlineto{\pgfqpoint{4.366600in}{4.699498in}}%
\pgfpathlineto{\pgfqpoint{4.394500in}{4.706034in}}%
\pgfpathlineto{\pgfqpoint{4.481300in}{4.728671in}}%
\pgfpathlineto{\pgfqpoint{4.503000in}{4.731268in}}%
\pgfpathlineto{\pgfqpoint{4.524700in}{4.731776in}}%
\pgfpathlineto{\pgfqpoint{4.546400in}{4.730171in}}%
\pgfpathlineto{\pgfqpoint{4.571200in}{4.726051in}}%
\pgfpathlineto{\pgfqpoint{4.605300in}{4.717699in}}%
\pgfpathlineto{\pgfqpoint{4.661100in}{4.703616in}}%
\pgfpathlineto{\pgfqpoint{4.685900in}{4.699653in}}%
\pgfpathlineto{\pgfqpoint{4.710700in}{4.698061in}}%
\pgfpathlineto{\pgfqpoint{4.732400in}{4.698780in}}%
\pgfpathlineto{\pgfqpoint{4.757200in}{4.701856in}}%
\pgfpathlineto{\pgfqpoint{4.788200in}{4.708264in}}%
\pgfpathlineto{\pgfqpoint{4.865700in}{4.725863in}}%
\pgfpathlineto{\pgfqpoint{4.890500in}{4.728691in}}%
\pgfpathlineto{\pgfqpoint{4.915300in}{4.729207in}}%
\pgfpathlineto{\pgfqpoint{4.940100in}{4.727376in}}%
\pgfpathlineto{\pgfqpoint{4.968000in}{4.722937in}}%
\pgfpathlineto{\pgfqpoint{5.011400in}{4.713278in}}%
\pgfpathlineto{\pgfqpoint{5.051700in}{4.704928in}}%
\pgfpathlineto{\pgfqpoint{5.079600in}{4.701368in}}%
\pgfpathlineto{\pgfqpoint{5.104400in}{4.700402in}}%
\pgfpathlineto{\pgfqpoint{5.129200in}{4.701630in}}%
\pgfpathlineto{\pgfqpoint{5.157100in}{4.705329in}}%
\pgfpathlineto{\pgfqpoint{5.197400in}{4.713369in}}%
\pgfpathlineto{\pgfqpoint{5.247000in}{4.723051in}}%
\pgfpathlineto{\pgfqpoint{5.274900in}{4.726285in}}%
\pgfpathlineto{\pgfqpoint{5.299700in}{4.727119in}}%
\pgfpathlineto{\pgfqpoint{5.324500in}{4.725925in}}%
\pgfpathlineto{\pgfqpoint{5.352400in}{4.722452in}}%
\pgfpathlineto{\pgfqpoint{5.395800in}{4.714367in}}%
\pgfpathlineto{\pgfqpoint{5.442300in}{4.706076in}}%
\pgfpathlineto{\pgfqpoint{5.473300in}{4.702948in}}%
\pgfpathlineto{\pgfqpoint{5.501200in}{4.702543in}}%
\pgfpathlineto{\pgfqpoint{5.529100in}{4.704466in}}%
\pgfpathlineto{\pgfqpoint{5.563200in}{4.709316in}}%
\pgfpathlineto{\pgfqpoint{5.653100in}{4.723694in}}%
\pgfpathlineto{\pgfqpoint{5.681000in}{4.725223in}}%
\pgfpathlineto{\pgfqpoint{5.708900in}{4.724558in}}%
\pgfpathlineto{\pgfqpoint{5.739900in}{4.721480in}}%
\pgfpathlineto{\pgfqpoint{5.783300in}{4.714625in}}%
\pgfpathlineto{\pgfqpoint{5.836000in}{4.706731in}}%
\pgfpathlineto{\pgfqpoint{5.867000in}{4.704392in}}%
\pgfpathlineto{\pgfqpoint{5.894900in}{4.704386in}}%
\pgfpathlineto{\pgfqpoint{5.925900in}{4.706657in}}%
\pgfpathlineto{\pgfqpoint{5.966200in}{4.712145in}}%
\pgfpathlineto{\pgfqpoint{6.031300in}{4.721351in}}%
\pgfpathlineto{\pgfqpoint{6.062300in}{4.723460in}}%
\pgfpathlineto{\pgfqpoint{6.093300in}{4.723292in}}%
\pgfpathlineto{\pgfqpoint{6.127400in}{4.720608in}}%
\pgfpathlineto{\pgfqpoint{6.177000in}{4.713896in}}%
\pgfpathlineto{\pgfqpoint{6.226600in}{4.707679in}}%
\pgfpathlineto{\pgfqpoint{6.260700in}{4.705703in}}%
\pgfpathlineto{\pgfqpoint{6.291700in}{4.706128in}}%
\pgfpathlineto{\pgfqpoint{6.325800in}{4.708848in}}%
\pgfpathlineto{\pgfqpoint{6.384700in}{4.716389in}}%
\pgfpathlineto{\pgfqpoint{6.428100in}{4.720949in}}%
\pgfpathlineto{\pgfqpoint{6.462200in}{4.722308in}}%
\pgfpathlineto{\pgfqpoint{6.496300in}{4.721288in}}%
\pgfpathlineto{\pgfqpoint{6.536600in}{4.717549in}}%
\pgfpathlineto{\pgfqpoint{6.629600in}{4.707692in}}%
\pgfpathlineto{\pgfqpoint{6.663700in}{4.706884in}}%
\pgfpathlineto{\pgfqpoint{6.697800in}{4.708258in}}%
\pgfpathlineto{\pgfqpoint{6.741200in}{4.712381in}}%
\pgfpathlineto{\pgfqpoint{6.818700in}{4.720082in}}%
\pgfpathlineto{\pgfqpoint{6.855900in}{4.721145in}}%
\pgfpathlineto{\pgfqpoint{6.893100in}{4.719807in}}%
\pgfpathlineto{\pgfqpoint{6.939600in}{4.715637in}}%
\pgfpathlineto{\pgfqpoint{7.014000in}{4.708881in}}%
\pgfpathlineto{\pgfqpoint{7.051200in}{4.707947in}}%
\pgfpathlineto{\pgfqpoint{7.088400in}{4.709230in}}%
\pgfpathlineto{\pgfqpoint{7.138000in}{4.713420in}}%
\pgfpathlineto{\pgfqpoint{7.200000in}{4.718811in}}%
\pgfpathlineto{\pgfqpoint{7.200000in}{4.718811in}}%
\pgfusepath{stroke}%
\end{pgfscope}%
\begin{pgfscope}%
\pgfpathrectangle{\pgfqpoint{1.000000in}{0.600000in}}{\pgfqpoint{6.200000in}{4.800000in}}%
\pgfusepath{clip}%
\pgfsetrectcap%
\pgfsetroundjoin%
\pgfsetlinewidth{1.003750pt}%
\definecolor{currentstroke}{rgb}{0.000000,0.000000,0.000000}%
\pgfsetstrokecolor{currentstroke}%
\pgfsetdash{}{0pt}%
\pgfpathmoveto{\pgfqpoint{1.000000in}{4.786094in}}%
\pgfpathlineto{\pgfqpoint{1.012400in}{4.784725in}}%
\pgfpathlineto{\pgfqpoint{1.024800in}{4.780668in}}%
\pgfpathlineto{\pgfqpoint{1.037200in}{4.774063in}}%
\pgfpathlineto{\pgfqpoint{1.049600in}{4.765148in}}%
\pgfpathlineto{\pgfqpoint{1.065100in}{4.751263in}}%
\pgfpathlineto{\pgfqpoint{1.086800in}{4.728223in}}%
\pgfpathlineto{\pgfqpoint{1.133300in}{4.677009in}}%
\pgfpathlineto{\pgfqpoint{1.148800in}{4.663177in}}%
\pgfpathlineto{\pgfqpoint{1.164300in}{4.652450in}}%
\pgfpathlineto{\pgfqpoint{1.176700in}{4.646525in}}%
\pgfpathlineto{\pgfqpoint{1.189100in}{4.643196in}}%
\pgfpathlineto{\pgfqpoint{1.201500in}{4.642578in}}%
\pgfpathlineto{\pgfqpoint{1.213900in}{4.644694in}}%
\pgfpathlineto{\pgfqpoint{1.226300in}{4.649469in}}%
\pgfpathlineto{\pgfqpoint{1.238700in}{4.656736in}}%
\pgfpathlineto{\pgfqpoint{1.254200in}{4.668915in}}%
\pgfpathlineto{\pgfqpoint{1.272800in}{4.687112in}}%
\pgfpathlineto{\pgfqpoint{1.300700in}{4.718360in}}%
\pgfpathlineto{\pgfqpoint{1.328600in}{4.748800in}}%
\pgfpathlineto{\pgfqpoint{1.347200in}{4.765641in}}%
\pgfpathlineto{\pgfqpoint{1.362700in}{4.776300in}}%
\pgfpathlineto{\pgfqpoint{1.375100in}{4.782163in}}%
\pgfpathlineto{\pgfqpoint{1.387500in}{4.785427in}}%
\pgfpathlineto{\pgfqpoint{1.399900in}{4.785977in}}%
\pgfpathlineto{\pgfqpoint{1.412300in}{4.783795in}}%
\pgfpathlineto{\pgfqpoint{1.424700in}{4.778955in}}%
\pgfpathlineto{\pgfqpoint{1.437100in}{4.771630in}}%
\pgfpathlineto{\pgfqpoint{1.452600in}{4.759388in}}%
\pgfpathlineto{\pgfqpoint{1.471200in}{4.741137in}}%
\pgfpathlineto{\pgfqpoint{1.502200in}{4.706325in}}%
\pgfpathlineto{\pgfqpoint{1.527000in}{4.679464in}}%
\pgfpathlineto{\pgfqpoint{1.545600in}{4.662686in}}%
\pgfpathlineto{\pgfqpoint{1.561100in}{4.652095in}}%
\pgfpathlineto{\pgfqpoint{1.573500in}{4.646293in}}%
\pgfpathlineto{\pgfqpoint{1.585900in}{4.643094in}}%
\pgfpathlineto{\pgfqpoint{1.598300in}{4.642612in}}%
\pgfpathlineto{\pgfqpoint{1.610700in}{4.644861in}}%
\pgfpathlineto{\pgfqpoint{1.623100in}{4.649764in}}%
\pgfpathlineto{\pgfqpoint{1.635500in}{4.657149in}}%
\pgfpathlineto{\pgfqpoint{1.651000in}{4.669453in}}%
\pgfpathlineto{\pgfqpoint{1.669600in}{4.687757in}}%
\pgfpathlineto{\pgfqpoint{1.700600in}{4.722594in}}%
\pgfpathlineto{\pgfqpoint{1.725400in}{4.749413in}}%
\pgfpathlineto{\pgfqpoint{1.744000in}{4.766129in}}%
\pgfpathlineto{\pgfqpoint{1.759500in}{4.776652in}}%
\pgfpathlineto{\pgfqpoint{1.771900in}{4.782392in}}%
\pgfpathlineto{\pgfqpoint{1.784300in}{4.785525in}}%
\pgfpathlineto{\pgfqpoint{1.796700in}{4.785941in}}%
\pgfpathlineto{\pgfqpoint{1.809100in}{4.783624in}}%
\pgfpathlineto{\pgfqpoint{1.821500in}{4.778657in}}%
\pgfpathlineto{\pgfqpoint{1.833900in}{4.771215in}}%
\pgfpathlineto{\pgfqpoint{1.849400in}{4.758849in}}%
\pgfpathlineto{\pgfqpoint{1.868000in}{4.740492in}}%
\pgfpathlineto{\pgfqpoint{1.899000in}{4.705630in}}%
\pgfpathlineto{\pgfqpoint{1.923800in}{4.678853in}}%
\pgfpathlineto{\pgfqpoint{1.939300in}{4.664694in}}%
\pgfpathlineto{\pgfqpoint{1.954800in}{4.653552in}}%
\pgfpathlineto{\pgfqpoint{1.967200in}{4.647248in}}%
\pgfpathlineto{\pgfqpoint{1.979600in}{4.643514in}}%
\pgfpathlineto{\pgfqpoint{1.992000in}{4.642482in}}%
\pgfpathlineto{\pgfqpoint{2.004400in}{4.644186in}}%
\pgfpathlineto{\pgfqpoint{2.016800in}{4.648567in}}%
\pgfpathlineto{\pgfqpoint{2.029200in}{4.655472in}}%
\pgfpathlineto{\pgfqpoint{2.044700in}{4.667265in}}%
\pgfpathlineto{\pgfqpoint{2.060200in}{4.681926in}}%
\pgfpathlineto{\pgfqpoint{2.085000in}{4.709109in}}%
\pgfpathlineto{\pgfqpoint{2.122200in}{4.750022in}}%
\pgfpathlineto{\pgfqpoint{2.137700in}{4.764130in}}%
\pgfpathlineto{\pgfqpoint{2.153200in}{4.775206in}}%
\pgfpathlineto{\pgfqpoint{2.165600in}{4.781450in}}%
\pgfpathlineto{\pgfqpoint{2.178000in}{4.785118in}}%
\pgfpathlineto{\pgfqpoint{2.190400in}{4.786084in}}%
\pgfpathlineto{\pgfqpoint{2.202800in}{4.784313in}}%
\pgfpathlineto{\pgfqpoint{2.215200in}{4.779867in}}%
\pgfpathlineto{\pgfqpoint{2.227600in}{4.772903in}}%
\pgfpathlineto{\pgfqpoint{2.243100in}{4.761046in}}%
\pgfpathlineto{\pgfqpoint{2.258600in}{4.746337in}}%
\pgfpathlineto{\pgfqpoint{2.283400in}{4.719115in}}%
\pgfpathlineto{\pgfqpoint{2.317500in}{4.681354in}}%
\pgfpathlineto{\pgfqpoint{2.336100in}{4.664190in}}%
\pgfpathlineto{\pgfqpoint{2.351600in}{4.653180in}}%
\pgfpathlineto{\pgfqpoint{2.364000in}{4.646997in}}%
\pgfpathlineto{\pgfqpoint{2.376400in}{4.643393in}}%
\pgfpathlineto{\pgfqpoint{2.388800in}{4.642495in}}%
\pgfpathlineto{\pgfqpoint{2.401200in}{4.644332in}}%
\pgfpathlineto{\pgfqpoint{2.413600in}{4.648842in}}%
\pgfpathlineto{\pgfqpoint{2.426000in}{4.655866in}}%
\pgfpathlineto{\pgfqpoint{2.441500in}{4.667787in}}%
\pgfpathlineto{\pgfqpoint{2.460100in}{4.685762in}}%
\pgfpathlineto{\pgfqpoint{2.488000in}{4.716904in}}%
\pgfpathlineto{\pgfqpoint{2.515900in}{4.747527in}}%
\pgfpathlineto{\pgfqpoint{2.534500in}{4.764631in}}%
\pgfpathlineto{\pgfqpoint{2.550000in}{4.775575in}}%
\pgfpathlineto{\pgfqpoint{2.562400in}{4.781697in}}%
\pgfpathlineto{\pgfqpoint{2.574800in}{4.785236in}}%
\pgfpathlineto{\pgfqpoint{2.587200in}{4.786068in}}%
\pgfpathlineto{\pgfqpoint{2.599600in}{4.784164in}}%
\pgfpathlineto{\pgfqpoint{2.612000in}{4.779590in}}%
\pgfpathlineto{\pgfqpoint{2.624400in}{4.772507in}}%
\pgfpathlineto{\pgfqpoint{2.639900in}{4.760523in}}%
\pgfpathlineto{\pgfqpoint{2.658500in}{4.742492in}}%
\pgfpathlineto{\pgfqpoint{2.686400in}{4.711320in}}%
\pgfpathlineto{\pgfqpoint{2.714300in}{4.680736in}}%
\pgfpathlineto{\pgfqpoint{2.732900in}{4.663692in}}%
\pgfpathlineto{\pgfqpoint{2.748400in}{4.652815in}}%
\pgfpathlineto{\pgfqpoint{2.760800in}{4.646753in}}%
\pgfpathlineto{\pgfqpoint{2.773200in}{4.643279in}}%
\pgfpathlineto{\pgfqpoint{2.785600in}{4.642514in}}%
\pgfpathlineto{\pgfqpoint{2.798000in}{4.644485in}}%
\pgfpathlineto{\pgfqpoint{2.810400in}{4.649123in}}%
\pgfpathlineto{\pgfqpoint{2.822800in}{4.656264in}}%
\pgfpathlineto{\pgfqpoint{2.838300in}{4.668311in}}%
\pgfpathlineto{\pgfqpoint{2.856900in}{4.686397in}}%
\pgfpathlineto{\pgfqpoint{2.884800in}{4.717598in}}%
\pgfpathlineto{\pgfqpoint{2.912700in}{4.748143in}}%
\pgfpathlineto{\pgfqpoint{2.931300in}{4.765126in}}%
\pgfpathlineto{\pgfqpoint{2.946800in}{4.775937in}}%
\pgfpathlineto{\pgfqpoint{2.959200in}{4.781938in}}%
\pgfpathlineto{\pgfqpoint{2.971600in}{4.785347in}}%
\pgfpathlineto{\pgfqpoint{2.984000in}{4.786045in}}%
\pgfpathlineto{\pgfqpoint{2.996400in}{4.784008in}}%
\pgfpathlineto{\pgfqpoint{3.008800in}{4.779306in}}%
\pgfpathlineto{\pgfqpoint{3.021200in}{4.772106in}}%
\pgfpathlineto{\pgfqpoint{3.036700in}{4.759996in}}%
\pgfpathlineto{\pgfqpoint{3.055300in}{4.741856in}}%
\pgfpathlineto{\pgfqpoint{3.083200in}{4.710627in}}%
\pgfpathlineto{\pgfqpoint{3.111100in}{4.680122in}}%
\pgfpathlineto{\pgfqpoint{3.129700in}{4.663200in}}%
\pgfpathlineto{\pgfqpoint{3.145200in}{4.652456in}}%
\pgfpathlineto{\pgfqpoint{3.157600in}{4.646516in}}%
\pgfpathlineto{\pgfqpoint{3.170000in}{4.643172in}}%
\pgfpathlineto{\pgfqpoint{3.182400in}{4.642540in}}%
\pgfpathlineto{\pgfqpoint{3.194800in}{4.644644in}}%
\pgfpathlineto{\pgfqpoint{3.207200in}{4.649409in}}%
\pgfpathlineto{\pgfqpoint{3.219600in}{4.656667in}}%
\pgfpathlineto{\pgfqpoint{3.235100in}{4.668840in}}%
\pgfpathlineto{\pgfqpoint{3.253700in}{4.687033in}}%
\pgfpathlineto{\pgfqpoint{3.281600in}{4.718291in}}%
\pgfpathlineto{\pgfqpoint{3.309500in}{4.748754in}}%
\pgfpathlineto{\pgfqpoint{3.328100in}{4.765616in}}%
\pgfpathlineto{\pgfqpoint{3.343600in}{4.776293in}}%
\pgfpathlineto{\pgfqpoint{3.356000in}{4.782172in}}%
\pgfpathlineto{\pgfqpoint{3.368400in}{4.785451in}}%
\pgfpathlineto{\pgfqpoint{3.380800in}{4.786016in}}%
\pgfpathlineto{\pgfqpoint{3.393200in}{4.783846in}}%
\pgfpathlineto{\pgfqpoint{3.405600in}{4.779018in}}%
\pgfpathlineto{\pgfqpoint{3.418000in}{4.771701in}}%
\pgfpathlineto{\pgfqpoint{3.433500in}{4.759467in}}%
\pgfpathlineto{\pgfqpoint{3.452100in}{4.741219in}}%
\pgfpathlineto{\pgfqpoint{3.483100in}{4.706395in}}%
\pgfpathlineto{\pgfqpoint{3.507900in}{4.679513in}}%
\pgfpathlineto{\pgfqpoint{3.526500in}{4.662713in}}%
\pgfpathlineto{\pgfqpoint{3.542000in}{4.652103in}}%
\pgfpathlineto{\pgfqpoint{3.554400in}{4.646285in}}%
\pgfpathlineto{\pgfqpoint{3.566800in}{4.643071in}}%
\pgfpathlineto{\pgfqpoint{3.579200in}{4.642573in}}%
\pgfpathlineto{\pgfqpoint{3.591600in}{4.644809in}}%
\pgfpathlineto{\pgfqpoint{3.604000in}{4.649700in}}%
\pgfpathlineto{\pgfqpoint{3.616400in}{4.657075in}}%
\pgfpathlineto{\pgfqpoint{3.631900in}{4.669371in}}%
\pgfpathlineto{\pgfqpoint{3.650500in}{4.687671in}}%
\pgfpathlineto{\pgfqpoint{3.681500in}{4.722520in}}%
\pgfpathlineto{\pgfqpoint{3.706300in}{4.749361in}}%
\pgfpathlineto{\pgfqpoint{3.724900in}{4.766099in}}%
\pgfpathlineto{\pgfqpoint{3.740400in}{4.776642in}}%
\pgfpathlineto{\pgfqpoint{3.752800in}{4.782399in}}%
\pgfpathlineto{\pgfqpoint{3.765200in}{4.785549in}}%
\pgfpathlineto{\pgfqpoint{3.777600in}{4.785980in}}%
\pgfpathlineto{\pgfqpoint{3.790000in}{4.783678in}}%
\pgfpathlineto{\pgfqpoint{3.802400in}{4.778724in}}%
\pgfpathlineto{\pgfqpoint{3.814800in}{4.771291in}}%
\pgfpathlineto{\pgfqpoint{3.830300in}{4.758934in}}%
\pgfpathlineto{\pgfqpoint{3.848900in}{4.740581in}}%
\pgfpathlineto{\pgfqpoint{3.879900in}{4.705708in}}%
\pgfpathlineto{\pgfqpoint{3.904700in}{4.678908in}}%
\pgfpathlineto{\pgfqpoint{3.920200in}{4.664730in}}%
\pgfpathlineto{\pgfqpoint{3.935700in}{4.653567in}}%
\pgfpathlineto{\pgfqpoint{3.948100in}{4.647246in}}%
\pgfpathlineto{\pgfqpoint{3.960500in}{4.643495in}}%
\pgfpathlineto{\pgfqpoint{3.972900in}{4.642446in}}%
\pgfpathlineto{\pgfqpoint{3.985300in}{4.644134in}}%
\pgfpathlineto{\pgfqpoint{3.997700in}{4.648502in}}%
\pgfpathlineto{\pgfqpoint{4.010100in}{4.655396in}}%
\pgfpathlineto{\pgfqpoint{4.025600in}{4.667179in}}%
\pgfpathlineto{\pgfqpoint{4.041100in}{4.681834in}}%
\pgfpathlineto{\pgfqpoint{4.065900in}{4.709021in}}%
\pgfpathlineto{\pgfqpoint{4.103100in}{4.749964in}}%
\pgfpathlineto{\pgfqpoint{4.118600in}{4.764092in}}%
\pgfpathlineto{\pgfqpoint{4.134100in}{4.775189in}}%
\pgfpathlineto{\pgfqpoint{4.146500in}{4.781451in}}%
\pgfpathlineto{\pgfqpoint{4.158900in}{4.785137in}}%
\pgfpathlineto{\pgfqpoint{4.171300in}{4.786121in}}%
\pgfpathlineto{\pgfqpoint{4.183700in}{4.784366in}}%
\pgfpathlineto{\pgfqpoint{4.196100in}{4.779934in}}%
\pgfpathlineto{\pgfqpoint{4.208500in}{4.772982in}}%
\pgfpathlineto{\pgfqpoint{4.224000in}{4.761135in}}%
\pgfpathlineto{\pgfqpoint{4.239500in}{4.746432in}}%
\pgfpathlineto{\pgfqpoint{4.264300in}{4.719207in}}%
\pgfpathlineto{\pgfqpoint{4.298400in}{4.681419in}}%
\pgfpathlineto{\pgfqpoint{4.317000in}{4.664231in}}%
\pgfpathlineto{\pgfqpoint{4.332500in}{4.653199in}}%
\pgfpathlineto{\pgfqpoint{4.344900in}{4.646997in}}%
\pgfpathlineto{\pgfqpoint{4.357300in}{4.643375in}}%
\pgfpathlineto{\pgfqpoint{4.369700in}{4.642458in}}%
\pgfpathlineto{\pgfqpoint{4.382100in}{4.644279in}}%
\pgfpathlineto{\pgfqpoint{4.394500in}{4.648774in}}%
\pgfpathlineto{\pgfqpoint{4.406900in}{4.655785in}}%
\pgfpathlineto{\pgfqpoint{4.422400in}{4.667694in}}%
\pgfpathlineto{\pgfqpoint{4.441000in}{4.685663in}}%
\pgfpathlineto{\pgfqpoint{4.468900in}{4.716811in}}%
\pgfpathlineto{\pgfqpoint{4.496800in}{4.747459in}}%
\pgfpathlineto{\pgfqpoint{4.515400in}{4.764588in}}%
\pgfpathlineto{\pgfqpoint{4.530900in}{4.775555in}}%
\pgfpathlineto{\pgfqpoint{4.543300in}{4.781696in}}%
\pgfpathlineto{\pgfqpoint{4.555700in}{4.785255in}}%
\pgfpathlineto{\pgfqpoint{4.568100in}{4.786105in}}%
\pgfpathlineto{\pgfqpoint{4.580500in}{4.784219in}}%
\pgfpathlineto{\pgfqpoint{4.592900in}{4.779660in}}%
\pgfpathlineto{\pgfqpoint{4.605300in}{4.772591in}}%
\pgfpathlineto{\pgfqpoint{4.620800in}{4.760618in}}%
\pgfpathlineto{\pgfqpoint{4.639400in}{4.742595in}}%
\pgfpathlineto{\pgfqpoint{4.667300in}{4.711416in}}%
\pgfpathlineto{\pgfqpoint{4.695200in}{4.680806in}}%
\pgfpathlineto{\pgfqpoint{4.713800in}{4.663737in}}%
\pgfpathlineto{\pgfqpoint{4.729300in}{4.652836in}}%
\pgfpathlineto{\pgfqpoint{4.741700in}{4.646755in}}%
\pgfpathlineto{\pgfqpoint{4.754100in}{4.643260in}}%
\pgfpathlineto{\pgfqpoint{4.766500in}{4.642476in}}%
\pgfpathlineto{\pgfqpoint{4.778900in}{4.644429in}}%
\pgfpathlineto{\pgfqpoint{4.791300in}{4.649050in}}%
\pgfpathlineto{\pgfqpoint{4.803700in}{4.656178in}}%
\pgfpathlineto{\pgfqpoint{4.819200in}{4.668213in}}%
\pgfpathlineto{\pgfqpoint{4.837800in}{4.686291in}}%
\pgfpathlineto{\pgfqpoint{4.865700in}{4.717499in}}%
\pgfpathlineto{\pgfqpoint{4.893600in}{4.748069in}}%
\pgfpathlineto{\pgfqpoint{4.912200in}{4.765079in}}%
\pgfpathlineto{\pgfqpoint{4.927700in}{4.775914in}}%
\pgfpathlineto{\pgfqpoint{4.940100in}{4.781936in}}%
\pgfpathlineto{\pgfqpoint{4.952500in}{4.785366in}}%
\pgfpathlineto{\pgfqpoint{4.964900in}{4.786084in}}%
\pgfpathlineto{\pgfqpoint{4.977300in}{4.784065in}}%
\pgfpathlineto{\pgfqpoint{4.989700in}{4.779381in}}%
\pgfpathlineto{\pgfqpoint{5.002100in}{4.772195in}}%
\pgfpathlineto{\pgfqpoint{5.017600in}{4.760098in}}%
\pgfpathlineto{\pgfqpoint{5.036200in}{4.741966in}}%
\pgfpathlineto{\pgfqpoint{5.064100in}{4.710730in}}%
\pgfpathlineto{\pgfqpoint{5.092000in}{4.680199in}}%
\pgfpathlineto{\pgfqpoint{5.110600in}{4.663250in}}%
\pgfpathlineto{\pgfqpoint{5.126100in}{4.652480in}}%
\pgfpathlineto{\pgfqpoint{5.138500in}{4.646519in}}%
\pgfpathlineto{\pgfqpoint{5.150900in}{4.643153in}}%
\pgfpathlineto{\pgfqpoint{5.163300in}{4.642501in}}%
\pgfpathlineto{\pgfqpoint{5.175700in}{4.644585in}}%
\pgfpathlineto{\pgfqpoint{5.188100in}{4.649332in}}%
\pgfpathlineto{\pgfqpoint{5.200500in}{4.656576in}}%
\pgfpathlineto{\pgfqpoint{5.216000in}{4.668735in}}%
\pgfpathlineto{\pgfqpoint{5.234600in}{4.686921in}}%
\pgfpathlineto{\pgfqpoint{5.262500in}{4.718184in}}%
\pgfpathlineto{\pgfqpoint{5.290400in}{4.748675in}}%
\pgfpathlineto{\pgfqpoint{5.309000in}{4.765564in}}%
\pgfpathlineto{\pgfqpoint{5.324500in}{4.776267in}}%
\pgfpathlineto{\pgfqpoint{5.336900in}{4.782168in}}%
\pgfpathlineto{\pgfqpoint{5.349300in}{4.785470in}}%
\pgfpathlineto{\pgfqpoint{5.361700in}{4.786056in}}%
\pgfpathlineto{\pgfqpoint{5.374100in}{4.783906in}}%
\pgfpathlineto{\pgfqpoint{5.386500in}{4.779096in}}%
\pgfpathlineto{\pgfqpoint{5.398900in}{4.771795in}}%
\pgfpathlineto{\pgfqpoint{5.414400in}{4.759574in}}%
\pgfpathlineto{\pgfqpoint{5.433000in}{4.741336in}}%
\pgfpathlineto{\pgfqpoint{5.460900in}{4.710045in}}%
\pgfpathlineto{\pgfqpoint{5.488800in}{4.679595in}}%
\pgfpathlineto{\pgfqpoint{5.507400in}{4.662767in}}%
\pgfpathlineto{\pgfqpoint{5.522900in}{4.652130in}}%
\pgfpathlineto{\pgfqpoint{5.535300in}{4.646289in}}%
\pgfpathlineto{\pgfqpoint{5.547700in}{4.643052in}}%
\pgfpathlineto{\pgfqpoint{5.560100in}{4.642532in}}%
\pgfpathlineto{\pgfqpoint{5.572500in}{4.644747in}}%
\pgfpathlineto{\pgfqpoint{5.584900in}{4.649620in}}%
\pgfpathlineto{\pgfqpoint{5.597300in}{4.656979in}}%
\pgfpathlineto{\pgfqpoint{5.612800in}{4.669260in}}%
\pgfpathlineto{\pgfqpoint{5.631400in}{4.687551in}}%
\pgfpathlineto{\pgfqpoint{5.662400in}{4.722408in}}%
\pgfpathlineto{\pgfqpoint{5.687200in}{4.749276in}}%
\pgfpathlineto{\pgfqpoint{5.705800in}{4.766043in}}%
\pgfpathlineto{\pgfqpoint{5.721300in}{4.776614in}}%
\pgfpathlineto{\pgfqpoint{5.733700in}{4.782395in}}%
\pgfpathlineto{\pgfqpoint{5.746100in}{4.785567in}}%
\pgfpathlineto{\pgfqpoint{5.758500in}{4.786022in}}%
\pgfpathlineto{\pgfqpoint{5.770900in}{4.783741in}}%
\pgfpathlineto{\pgfqpoint{5.783300in}{4.778806in}}%
\pgfpathlineto{\pgfqpoint{5.795700in}{4.771390in}}%
\pgfpathlineto{\pgfqpoint{5.811200in}{4.759048in}}%
\pgfpathlineto{\pgfqpoint{5.829800in}{4.740704in}}%
\pgfpathlineto{\pgfqpoint{5.860800in}{4.705823in}}%
\pgfpathlineto{\pgfqpoint{5.885600in}{4.678996in}}%
\pgfpathlineto{\pgfqpoint{5.904200in}{4.662291in}}%
\pgfpathlineto{\pgfqpoint{5.919700in}{4.651786in}}%
\pgfpathlineto{\pgfqpoint{5.932100in}{4.646066in}}%
\pgfpathlineto{\pgfqpoint{5.944500in}{4.642958in}}%
\pgfpathlineto{\pgfqpoint{5.956900in}{4.642569in}}%
\pgfpathlineto{\pgfqpoint{5.969300in}{4.644915in}}%
\pgfpathlineto{\pgfqpoint{5.981700in}{4.649912in}}%
\pgfpathlineto{\pgfqpoint{5.994100in}{4.657385in}}%
\pgfpathlineto{\pgfqpoint{6.009600in}{4.669788in}}%
\pgfpathlineto{\pgfqpoint{6.028200in}{4.688183in}}%
\pgfpathlineto{\pgfqpoint{6.059200in}{4.723088in}}%
\pgfpathlineto{\pgfqpoint{6.084000in}{4.749873in}}%
\pgfpathlineto{\pgfqpoint{6.099500in}{4.764026in}}%
\pgfpathlineto{\pgfqpoint{6.115000in}{4.775152in}}%
\pgfpathlineto{\pgfqpoint{6.127400in}{4.781438in}}%
\pgfpathlineto{\pgfqpoint{6.139800in}{4.785150in}}%
\pgfpathlineto{\pgfqpoint{6.152200in}{4.786158in}}%
\pgfpathlineto{\pgfqpoint{6.164600in}{4.784426in}}%
\pgfpathlineto{\pgfqpoint{6.177000in}{4.780016in}}%
\pgfpathlineto{\pgfqpoint{6.189400in}{4.773082in}}%
\pgfpathlineto{\pgfqpoint{6.204900in}{4.761253in}}%
\pgfpathlineto{\pgfqpoint{6.220400in}{4.746560in}}%
\pgfpathlineto{\pgfqpoint{6.245200in}{4.719337in}}%
\pgfpathlineto{\pgfqpoint{6.282400in}{4.678401in}}%
\pgfpathlineto{\pgfqpoint{6.297900in}{4.664299in}}%
\pgfpathlineto{\pgfqpoint{6.313400in}{4.653237in}}%
\pgfpathlineto{\pgfqpoint{6.325800in}{4.647010in}}%
\pgfpathlineto{\pgfqpoint{6.338200in}{4.643362in}}%
\pgfpathlineto{\pgfqpoint{6.350600in}{4.642420in}}%
\pgfpathlineto{\pgfqpoint{6.363000in}{4.644216in}}%
\pgfpathlineto{\pgfqpoint{6.375400in}{4.648690in}}%
\pgfpathlineto{\pgfqpoint{6.387800in}{4.655682in}}%
\pgfpathlineto{\pgfqpoint{6.403300in}{4.667573in}}%
\pgfpathlineto{\pgfqpoint{6.421900in}{4.685530in}}%
\pgfpathlineto{\pgfqpoint{6.449800in}{4.716682in}}%
\pgfpathlineto{\pgfqpoint{6.477700in}{4.747359in}}%
\pgfpathlineto{\pgfqpoint{6.496300in}{4.764518in}}%
\pgfpathlineto{\pgfqpoint{6.511800in}{4.775515in}}%
\pgfpathlineto{\pgfqpoint{6.524200in}{4.781683in}}%
\pgfpathlineto{\pgfqpoint{6.536600in}{4.785268in}}%
\pgfpathlineto{\pgfqpoint{6.549000in}{4.786144in}}%
\pgfpathlineto{\pgfqpoint{6.561400in}{4.784282in}}%
\pgfpathlineto{\pgfqpoint{6.573800in}{4.779746in}}%
\pgfpathlineto{\pgfqpoint{6.586200in}{4.772696in}}%
\pgfpathlineto{\pgfqpoint{6.601700in}{4.760742in}}%
\pgfpathlineto{\pgfqpoint{6.620300in}{4.742732in}}%
\pgfpathlineto{\pgfqpoint{6.648200in}{4.711550in}}%
\pgfpathlineto{\pgfqpoint{6.676100in}{4.680910in}}%
\pgfpathlineto{\pgfqpoint{6.694700in}{4.663809in}}%
\pgfpathlineto{\pgfqpoint{6.710200in}{4.652877in}}%
\pgfpathlineto{\pgfqpoint{6.722600in}{4.646769in}}%
\pgfpathlineto{\pgfqpoint{6.735000in}{4.643248in}}%
\pgfpathlineto{\pgfqpoint{6.747400in}{4.642437in}}%
\pgfpathlineto{\pgfqpoint{6.759800in}{4.644364in}}%
\pgfpathlineto{\pgfqpoint{6.772200in}{4.648962in}}%
\pgfpathlineto{\pgfqpoint{6.784600in}{4.656070in}}%
\pgfpathlineto{\pgfqpoint{6.800100in}{4.668086in}}%
\pgfpathlineto{\pgfqpoint{6.818700in}{4.686151in}}%
\pgfpathlineto{\pgfqpoint{6.846600in}{4.717362in}}%
\pgfpathlineto{\pgfqpoint{6.874500in}{4.747963in}}%
\pgfpathlineto{\pgfqpoint{6.893100in}{4.765005in}}%
\pgfpathlineto{\pgfqpoint{6.908600in}{4.775872in}}%
\pgfpathlineto{\pgfqpoint{6.921000in}{4.781921in}}%
\pgfpathlineto{\pgfqpoint{6.933400in}{4.785378in}}%
\pgfpathlineto{\pgfqpoint{6.945800in}{4.786124in}}%
\pgfpathlineto{\pgfqpoint{6.958200in}{4.784132in}}%
\pgfpathlineto{\pgfqpoint{6.970600in}{4.779471in}}%
\pgfpathlineto{\pgfqpoint{6.983000in}{4.772305in}}%
\pgfpathlineto{\pgfqpoint{6.998500in}{4.760228in}}%
\pgfpathlineto{\pgfqpoint{7.017100in}{4.742110in}}%
\pgfpathlineto{\pgfqpoint{7.045000in}{4.710870in}}%
\pgfpathlineto{\pgfqpoint{7.072900in}{4.680308in}}%
\pgfpathlineto{\pgfqpoint{7.091500in}{4.663326in}}%
\pgfpathlineto{\pgfqpoint{7.107000in}{4.652523in}}%
\pgfpathlineto{\pgfqpoint{7.119400in}{4.646534in}}%
\pgfpathlineto{\pgfqpoint{7.131800in}{4.643140in}}%
\pgfpathlineto{\pgfqpoint{7.144200in}{4.642460in}}%
\pgfpathlineto{\pgfqpoint{7.156600in}{4.644517in}}%
\pgfpathlineto{\pgfqpoint{7.169000in}{4.649240in}}%
\pgfpathlineto{\pgfqpoint{7.181400in}{4.656463in}}%
\pgfpathlineto{\pgfqpoint{7.196900in}{4.668601in}}%
\pgfpathlineto{\pgfqpoint{7.200000in}{4.671390in}}%
\pgfpathlineto{\pgfqpoint{7.200000in}{4.671390in}}%
\pgfusepath{stroke}%
\end{pgfscope}%
\begin{pgfscope}%
\pgfsetrectcap%
\pgfsetmiterjoin%
\pgfsetlinewidth{1.003750pt}%
\definecolor{currentstroke}{rgb}{0.000000,0.000000,0.000000}%
\pgfsetstrokecolor{currentstroke}%
\pgfsetdash{}{0pt}%
\pgfpathmoveto{\pgfqpoint{1.000000in}{0.600000in}}%
\pgfpathlineto{\pgfqpoint{1.000000in}{5.400000in}}%
\pgfusepath{stroke}%
\end{pgfscope}%
\begin{pgfscope}%
\pgfsetrectcap%
\pgfsetmiterjoin%
\pgfsetlinewidth{1.003750pt}%
\definecolor{currentstroke}{rgb}{0.000000,0.000000,0.000000}%
\pgfsetstrokecolor{currentstroke}%
\pgfsetdash{}{0pt}%
\pgfpathmoveto{\pgfqpoint{7.200000in}{0.600000in}}%
\pgfpathlineto{\pgfqpoint{7.200000in}{5.400000in}}%
\pgfusepath{stroke}%
\end{pgfscope}%
\begin{pgfscope}%
\pgfsetrectcap%
\pgfsetmiterjoin%
\pgfsetlinewidth{1.003750pt}%
\definecolor{currentstroke}{rgb}{0.000000,0.000000,0.000000}%
\pgfsetstrokecolor{currentstroke}%
\pgfsetdash{}{0pt}%
\pgfpathmoveto{\pgfqpoint{1.000000in}{0.600000in}}%
\pgfpathlineto{\pgfqpoint{7.200000in}{0.600000in}}%
\pgfusepath{stroke}%
\end{pgfscope}%
\begin{pgfscope}%
\pgfsetrectcap%
\pgfsetmiterjoin%
\pgfsetlinewidth{1.003750pt}%
\definecolor{currentstroke}{rgb}{0.000000,0.000000,0.000000}%
\pgfsetstrokecolor{currentstroke}%
\pgfsetdash{}{0pt}%
\pgfpathmoveto{\pgfqpoint{1.000000in}{5.400000in}}%
\pgfpathlineto{\pgfqpoint{7.200000in}{5.400000in}}%
\pgfusepath{stroke}%
\end{pgfscope}%
\begin{pgfscope}%
\pgfpathrectangle{\pgfqpoint{1.000000in}{0.600000in}}{\pgfqpoint{6.200000in}{4.800000in}}%
\pgfusepath{clip}%
\pgfsetbuttcap%
\pgfsetroundjoin%
\pgfsetlinewidth{0.501875pt}%
\definecolor{currentstroke}{rgb}{0.000000,0.000000,0.000000}%
\pgfsetstrokecolor{currentstroke}%
\pgfsetdash{{1.000000pt}{3.000000pt}}{0.000000pt}%
\pgfpathmoveto{\pgfqpoint{1.000000in}{0.600000in}}%
\pgfpathlineto{\pgfqpoint{1.000000in}{5.400000in}}%
\pgfusepath{stroke}%
\end{pgfscope}%
\begin{pgfscope}%
\pgfsetbuttcap%
\pgfsetroundjoin%
\definecolor{currentfill}{rgb}{0.000000,0.000000,0.000000}%
\pgfsetfillcolor{currentfill}%
\pgfsetlinewidth{0.501875pt}%
\definecolor{currentstroke}{rgb}{0.000000,0.000000,0.000000}%
\pgfsetstrokecolor{currentstroke}%
\pgfsetdash{}{0pt}%
\pgfsys@defobject{currentmarker}{\pgfqpoint{0.000000in}{0.000000in}}{\pgfqpoint{0.000000in}{0.055556in}}{%
\pgfpathmoveto{\pgfqpoint{0.000000in}{0.000000in}}%
\pgfpathlineto{\pgfqpoint{0.000000in}{0.055556in}}%
\pgfusepath{stroke,fill}%
}%
\begin{pgfscope}%
\pgfsys@transformshift{1.000000in}{0.600000in}%
\pgfsys@useobject{currentmarker}{}%
\end{pgfscope}%
\end{pgfscope}%
\begin{pgfscope}%
\pgfsetbuttcap%
\pgfsetroundjoin%
\definecolor{currentfill}{rgb}{0.000000,0.000000,0.000000}%
\pgfsetfillcolor{currentfill}%
\pgfsetlinewidth{0.501875pt}%
\definecolor{currentstroke}{rgb}{0.000000,0.000000,0.000000}%
\pgfsetstrokecolor{currentstroke}%
\pgfsetdash{}{0pt}%
\pgfsys@defobject{currentmarker}{\pgfqpoint{0.000000in}{-0.055556in}}{\pgfqpoint{0.000000in}{0.000000in}}{%
\pgfpathmoveto{\pgfqpoint{0.000000in}{0.000000in}}%
\pgfpathlineto{\pgfqpoint{0.000000in}{-0.055556in}}%
\pgfusepath{stroke,fill}%
}%
\begin{pgfscope}%
\pgfsys@transformshift{1.000000in}{5.400000in}%
\pgfsys@useobject{currentmarker}{}%
\end{pgfscope}%
\end{pgfscope}%
\begin{pgfscope}%
\definecolor{textcolor}{rgb}{0.000000,0.000000,0.000000}%
\pgfsetstrokecolor{textcolor}%
\pgfsetfillcolor{textcolor}%
\pgftext[x=1.000000in,y=0.544444in,,top]{\color{textcolor}\rmfamily\fontsize{10.000000}{12.000000}\selectfont \(\displaystyle {0}\)}%
\end{pgfscope}%
\begin{pgfscope}%
\pgfpathrectangle{\pgfqpoint{1.000000in}{0.600000in}}{\pgfqpoint{6.200000in}{4.800000in}}%
\pgfusepath{clip}%
\pgfsetbuttcap%
\pgfsetroundjoin%
\pgfsetlinewidth{0.501875pt}%
\definecolor{currentstroke}{rgb}{0.000000,0.000000,0.000000}%
\pgfsetstrokecolor{currentstroke}%
\pgfsetdash{{1.000000pt}{3.000000pt}}{0.000000pt}%
\pgfpathmoveto{\pgfqpoint{2.240000in}{0.600000in}}%
\pgfpathlineto{\pgfqpoint{2.240000in}{5.400000in}}%
\pgfusepath{stroke}%
\end{pgfscope}%
\begin{pgfscope}%
\pgfsetbuttcap%
\pgfsetroundjoin%
\definecolor{currentfill}{rgb}{0.000000,0.000000,0.000000}%
\pgfsetfillcolor{currentfill}%
\pgfsetlinewidth{0.501875pt}%
\definecolor{currentstroke}{rgb}{0.000000,0.000000,0.000000}%
\pgfsetstrokecolor{currentstroke}%
\pgfsetdash{}{0pt}%
\pgfsys@defobject{currentmarker}{\pgfqpoint{0.000000in}{0.000000in}}{\pgfqpoint{0.000000in}{0.055556in}}{%
\pgfpathmoveto{\pgfqpoint{0.000000in}{0.000000in}}%
\pgfpathlineto{\pgfqpoint{0.000000in}{0.055556in}}%
\pgfusepath{stroke,fill}%
}%
\begin{pgfscope}%
\pgfsys@transformshift{2.240000in}{0.600000in}%
\pgfsys@useobject{currentmarker}{}%
\end{pgfscope}%
\end{pgfscope}%
\begin{pgfscope}%
\pgfsetbuttcap%
\pgfsetroundjoin%
\definecolor{currentfill}{rgb}{0.000000,0.000000,0.000000}%
\pgfsetfillcolor{currentfill}%
\pgfsetlinewidth{0.501875pt}%
\definecolor{currentstroke}{rgb}{0.000000,0.000000,0.000000}%
\pgfsetstrokecolor{currentstroke}%
\pgfsetdash{}{0pt}%
\pgfsys@defobject{currentmarker}{\pgfqpoint{0.000000in}{-0.055556in}}{\pgfqpoint{0.000000in}{0.000000in}}{%
\pgfpathmoveto{\pgfqpoint{0.000000in}{0.000000in}}%
\pgfpathlineto{\pgfqpoint{0.000000in}{-0.055556in}}%
\pgfusepath{stroke,fill}%
}%
\begin{pgfscope}%
\pgfsys@transformshift{2.240000in}{5.400000in}%
\pgfsys@useobject{currentmarker}{}%
\end{pgfscope}%
\end{pgfscope}%
\begin{pgfscope}%
\definecolor{textcolor}{rgb}{0.000000,0.000000,0.000000}%
\pgfsetstrokecolor{textcolor}%
\pgfsetfillcolor{textcolor}%
\pgftext[x=2.240000in,y=0.544444in,,top]{\color{textcolor}\rmfamily\fontsize{10.000000}{12.000000}\selectfont \(\displaystyle {20}\)}%
\end{pgfscope}%
\begin{pgfscope}%
\pgfpathrectangle{\pgfqpoint{1.000000in}{0.600000in}}{\pgfqpoint{6.200000in}{4.800000in}}%
\pgfusepath{clip}%
\pgfsetbuttcap%
\pgfsetroundjoin%
\pgfsetlinewidth{0.501875pt}%
\definecolor{currentstroke}{rgb}{0.000000,0.000000,0.000000}%
\pgfsetstrokecolor{currentstroke}%
\pgfsetdash{{1.000000pt}{3.000000pt}}{0.000000pt}%
\pgfpathmoveto{\pgfqpoint{3.480000in}{0.600000in}}%
\pgfpathlineto{\pgfqpoint{3.480000in}{5.400000in}}%
\pgfusepath{stroke}%
\end{pgfscope}%
\begin{pgfscope}%
\pgfsetbuttcap%
\pgfsetroundjoin%
\definecolor{currentfill}{rgb}{0.000000,0.000000,0.000000}%
\pgfsetfillcolor{currentfill}%
\pgfsetlinewidth{0.501875pt}%
\definecolor{currentstroke}{rgb}{0.000000,0.000000,0.000000}%
\pgfsetstrokecolor{currentstroke}%
\pgfsetdash{}{0pt}%
\pgfsys@defobject{currentmarker}{\pgfqpoint{0.000000in}{0.000000in}}{\pgfqpoint{0.000000in}{0.055556in}}{%
\pgfpathmoveto{\pgfqpoint{0.000000in}{0.000000in}}%
\pgfpathlineto{\pgfqpoint{0.000000in}{0.055556in}}%
\pgfusepath{stroke,fill}%
}%
\begin{pgfscope}%
\pgfsys@transformshift{3.480000in}{0.600000in}%
\pgfsys@useobject{currentmarker}{}%
\end{pgfscope}%
\end{pgfscope}%
\begin{pgfscope}%
\pgfsetbuttcap%
\pgfsetroundjoin%
\definecolor{currentfill}{rgb}{0.000000,0.000000,0.000000}%
\pgfsetfillcolor{currentfill}%
\pgfsetlinewidth{0.501875pt}%
\definecolor{currentstroke}{rgb}{0.000000,0.000000,0.000000}%
\pgfsetstrokecolor{currentstroke}%
\pgfsetdash{}{0pt}%
\pgfsys@defobject{currentmarker}{\pgfqpoint{0.000000in}{-0.055556in}}{\pgfqpoint{0.000000in}{0.000000in}}{%
\pgfpathmoveto{\pgfqpoint{0.000000in}{0.000000in}}%
\pgfpathlineto{\pgfqpoint{0.000000in}{-0.055556in}}%
\pgfusepath{stroke,fill}%
}%
\begin{pgfscope}%
\pgfsys@transformshift{3.480000in}{5.400000in}%
\pgfsys@useobject{currentmarker}{}%
\end{pgfscope}%
\end{pgfscope}%
\begin{pgfscope}%
\definecolor{textcolor}{rgb}{0.000000,0.000000,0.000000}%
\pgfsetstrokecolor{textcolor}%
\pgfsetfillcolor{textcolor}%
\pgftext[x=3.480000in,y=0.544444in,,top]{\color{textcolor}\rmfamily\fontsize{10.000000}{12.000000}\selectfont \(\displaystyle {40}\)}%
\end{pgfscope}%
\begin{pgfscope}%
\pgfpathrectangle{\pgfqpoint{1.000000in}{0.600000in}}{\pgfqpoint{6.200000in}{4.800000in}}%
\pgfusepath{clip}%
\pgfsetbuttcap%
\pgfsetroundjoin%
\pgfsetlinewidth{0.501875pt}%
\definecolor{currentstroke}{rgb}{0.000000,0.000000,0.000000}%
\pgfsetstrokecolor{currentstroke}%
\pgfsetdash{{1.000000pt}{3.000000pt}}{0.000000pt}%
\pgfpathmoveto{\pgfqpoint{4.720000in}{0.600000in}}%
\pgfpathlineto{\pgfqpoint{4.720000in}{5.400000in}}%
\pgfusepath{stroke}%
\end{pgfscope}%
\begin{pgfscope}%
\pgfsetbuttcap%
\pgfsetroundjoin%
\definecolor{currentfill}{rgb}{0.000000,0.000000,0.000000}%
\pgfsetfillcolor{currentfill}%
\pgfsetlinewidth{0.501875pt}%
\definecolor{currentstroke}{rgb}{0.000000,0.000000,0.000000}%
\pgfsetstrokecolor{currentstroke}%
\pgfsetdash{}{0pt}%
\pgfsys@defobject{currentmarker}{\pgfqpoint{0.000000in}{0.000000in}}{\pgfqpoint{0.000000in}{0.055556in}}{%
\pgfpathmoveto{\pgfqpoint{0.000000in}{0.000000in}}%
\pgfpathlineto{\pgfqpoint{0.000000in}{0.055556in}}%
\pgfusepath{stroke,fill}%
}%
\begin{pgfscope}%
\pgfsys@transformshift{4.720000in}{0.600000in}%
\pgfsys@useobject{currentmarker}{}%
\end{pgfscope}%
\end{pgfscope}%
\begin{pgfscope}%
\pgfsetbuttcap%
\pgfsetroundjoin%
\definecolor{currentfill}{rgb}{0.000000,0.000000,0.000000}%
\pgfsetfillcolor{currentfill}%
\pgfsetlinewidth{0.501875pt}%
\definecolor{currentstroke}{rgb}{0.000000,0.000000,0.000000}%
\pgfsetstrokecolor{currentstroke}%
\pgfsetdash{}{0pt}%
\pgfsys@defobject{currentmarker}{\pgfqpoint{0.000000in}{-0.055556in}}{\pgfqpoint{0.000000in}{0.000000in}}{%
\pgfpathmoveto{\pgfqpoint{0.000000in}{0.000000in}}%
\pgfpathlineto{\pgfqpoint{0.000000in}{-0.055556in}}%
\pgfusepath{stroke,fill}%
}%
\begin{pgfscope}%
\pgfsys@transformshift{4.720000in}{5.400000in}%
\pgfsys@useobject{currentmarker}{}%
\end{pgfscope}%
\end{pgfscope}%
\begin{pgfscope}%
\definecolor{textcolor}{rgb}{0.000000,0.000000,0.000000}%
\pgfsetstrokecolor{textcolor}%
\pgfsetfillcolor{textcolor}%
\pgftext[x=4.720000in,y=0.544444in,,top]{\color{textcolor}\rmfamily\fontsize{10.000000}{12.000000}\selectfont \(\displaystyle {60}\)}%
\end{pgfscope}%
\begin{pgfscope}%
\pgfpathrectangle{\pgfqpoint{1.000000in}{0.600000in}}{\pgfqpoint{6.200000in}{4.800000in}}%
\pgfusepath{clip}%
\pgfsetbuttcap%
\pgfsetroundjoin%
\pgfsetlinewidth{0.501875pt}%
\definecolor{currentstroke}{rgb}{0.000000,0.000000,0.000000}%
\pgfsetstrokecolor{currentstroke}%
\pgfsetdash{{1.000000pt}{3.000000pt}}{0.000000pt}%
\pgfpathmoveto{\pgfqpoint{5.960000in}{0.600000in}}%
\pgfpathlineto{\pgfqpoint{5.960000in}{5.400000in}}%
\pgfusepath{stroke}%
\end{pgfscope}%
\begin{pgfscope}%
\pgfsetbuttcap%
\pgfsetroundjoin%
\definecolor{currentfill}{rgb}{0.000000,0.000000,0.000000}%
\pgfsetfillcolor{currentfill}%
\pgfsetlinewidth{0.501875pt}%
\definecolor{currentstroke}{rgb}{0.000000,0.000000,0.000000}%
\pgfsetstrokecolor{currentstroke}%
\pgfsetdash{}{0pt}%
\pgfsys@defobject{currentmarker}{\pgfqpoint{0.000000in}{0.000000in}}{\pgfqpoint{0.000000in}{0.055556in}}{%
\pgfpathmoveto{\pgfqpoint{0.000000in}{0.000000in}}%
\pgfpathlineto{\pgfqpoint{0.000000in}{0.055556in}}%
\pgfusepath{stroke,fill}%
}%
\begin{pgfscope}%
\pgfsys@transformshift{5.960000in}{0.600000in}%
\pgfsys@useobject{currentmarker}{}%
\end{pgfscope}%
\end{pgfscope}%
\begin{pgfscope}%
\pgfsetbuttcap%
\pgfsetroundjoin%
\definecolor{currentfill}{rgb}{0.000000,0.000000,0.000000}%
\pgfsetfillcolor{currentfill}%
\pgfsetlinewidth{0.501875pt}%
\definecolor{currentstroke}{rgb}{0.000000,0.000000,0.000000}%
\pgfsetstrokecolor{currentstroke}%
\pgfsetdash{}{0pt}%
\pgfsys@defobject{currentmarker}{\pgfqpoint{0.000000in}{-0.055556in}}{\pgfqpoint{0.000000in}{0.000000in}}{%
\pgfpathmoveto{\pgfqpoint{0.000000in}{0.000000in}}%
\pgfpathlineto{\pgfqpoint{0.000000in}{-0.055556in}}%
\pgfusepath{stroke,fill}%
}%
\begin{pgfscope}%
\pgfsys@transformshift{5.960000in}{5.400000in}%
\pgfsys@useobject{currentmarker}{}%
\end{pgfscope}%
\end{pgfscope}%
\begin{pgfscope}%
\definecolor{textcolor}{rgb}{0.000000,0.000000,0.000000}%
\pgfsetstrokecolor{textcolor}%
\pgfsetfillcolor{textcolor}%
\pgftext[x=5.960000in,y=0.544444in,,top]{\color{textcolor}\rmfamily\fontsize{10.000000}{12.000000}\selectfont \(\displaystyle {80}\)}%
\end{pgfscope}%
\begin{pgfscope}%
\pgfpathrectangle{\pgfqpoint{1.000000in}{0.600000in}}{\pgfqpoint{6.200000in}{4.800000in}}%
\pgfusepath{clip}%
\pgfsetbuttcap%
\pgfsetroundjoin%
\pgfsetlinewidth{0.501875pt}%
\definecolor{currentstroke}{rgb}{0.000000,0.000000,0.000000}%
\pgfsetstrokecolor{currentstroke}%
\pgfsetdash{{1.000000pt}{3.000000pt}}{0.000000pt}%
\pgfpathmoveto{\pgfqpoint{7.200000in}{0.600000in}}%
\pgfpathlineto{\pgfqpoint{7.200000in}{5.400000in}}%
\pgfusepath{stroke}%
\end{pgfscope}%
\begin{pgfscope}%
\pgfsetbuttcap%
\pgfsetroundjoin%
\definecolor{currentfill}{rgb}{0.000000,0.000000,0.000000}%
\pgfsetfillcolor{currentfill}%
\pgfsetlinewidth{0.501875pt}%
\definecolor{currentstroke}{rgb}{0.000000,0.000000,0.000000}%
\pgfsetstrokecolor{currentstroke}%
\pgfsetdash{}{0pt}%
\pgfsys@defobject{currentmarker}{\pgfqpoint{0.000000in}{0.000000in}}{\pgfqpoint{0.000000in}{0.055556in}}{%
\pgfpathmoveto{\pgfqpoint{0.000000in}{0.000000in}}%
\pgfpathlineto{\pgfqpoint{0.000000in}{0.055556in}}%
\pgfusepath{stroke,fill}%
}%
\begin{pgfscope}%
\pgfsys@transformshift{7.200000in}{0.600000in}%
\pgfsys@useobject{currentmarker}{}%
\end{pgfscope}%
\end{pgfscope}%
\begin{pgfscope}%
\pgfsetbuttcap%
\pgfsetroundjoin%
\definecolor{currentfill}{rgb}{0.000000,0.000000,0.000000}%
\pgfsetfillcolor{currentfill}%
\pgfsetlinewidth{0.501875pt}%
\definecolor{currentstroke}{rgb}{0.000000,0.000000,0.000000}%
\pgfsetstrokecolor{currentstroke}%
\pgfsetdash{}{0pt}%
\pgfsys@defobject{currentmarker}{\pgfqpoint{0.000000in}{-0.055556in}}{\pgfqpoint{0.000000in}{0.000000in}}{%
\pgfpathmoveto{\pgfqpoint{0.000000in}{0.000000in}}%
\pgfpathlineto{\pgfqpoint{0.000000in}{-0.055556in}}%
\pgfusepath{stroke,fill}%
}%
\begin{pgfscope}%
\pgfsys@transformshift{7.200000in}{5.400000in}%
\pgfsys@useobject{currentmarker}{}%
\end{pgfscope}%
\end{pgfscope}%
\begin{pgfscope}%
\definecolor{textcolor}{rgb}{0.000000,0.000000,0.000000}%
\pgfsetstrokecolor{textcolor}%
\pgfsetfillcolor{textcolor}%
\pgftext[x=7.200000in,y=0.544444in,,top]{\color{textcolor}\rmfamily\fontsize{10.000000}{12.000000}\selectfont \(\displaystyle {100}\)}%
\end{pgfscope}%
\begin{pgfscope}%
\definecolor{textcolor}{rgb}{0.000000,0.000000,0.000000}%
\pgfsetstrokecolor{textcolor}%
\pgfsetfillcolor{textcolor}%
\pgftext[x=4.100000in,y=0.351543in,,top]{\color{textcolor}\rmfamily\fontsize{12.000000}{14.400000}\selectfont \(\displaystyle time\ (s)\)}%
\end{pgfscope}%
\begin{pgfscope}%
\pgfpathrectangle{\pgfqpoint{1.000000in}{0.600000in}}{\pgfqpoint{6.200000in}{4.800000in}}%
\pgfusepath{clip}%
\pgfsetbuttcap%
\pgfsetroundjoin%
\pgfsetlinewidth{0.501875pt}%
\definecolor{currentstroke}{rgb}{0.000000,0.000000,0.000000}%
\pgfsetstrokecolor{currentstroke}%
\pgfsetdash{{1.000000pt}{3.000000pt}}{0.000000pt}%
\pgfpathmoveto{\pgfqpoint{1.000000in}{0.600000in}}%
\pgfpathlineto{\pgfqpoint{7.200000in}{0.600000in}}%
\pgfusepath{stroke}%
\end{pgfscope}%
\begin{pgfscope}%
\pgfsetbuttcap%
\pgfsetroundjoin%
\definecolor{currentfill}{rgb}{0.000000,0.000000,0.000000}%
\pgfsetfillcolor{currentfill}%
\pgfsetlinewidth{0.501875pt}%
\definecolor{currentstroke}{rgb}{0.000000,0.000000,0.000000}%
\pgfsetstrokecolor{currentstroke}%
\pgfsetdash{}{0pt}%
\pgfsys@defobject{currentmarker}{\pgfqpoint{0.000000in}{0.000000in}}{\pgfqpoint{0.055556in}{0.000000in}}{%
\pgfpathmoveto{\pgfqpoint{0.000000in}{0.000000in}}%
\pgfpathlineto{\pgfqpoint{0.055556in}{0.000000in}}%
\pgfusepath{stroke,fill}%
}%
\begin{pgfscope}%
\pgfsys@transformshift{1.000000in}{0.600000in}%
\pgfsys@useobject{currentmarker}{}%
\end{pgfscope}%
\end{pgfscope}%
\begin{pgfscope}%
\pgfsetbuttcap%
\pgfsetroundjoin%
\definecolor{currentfill}{rgb}{0.000000,0.000000,0.000000}%
\pgfsetfillcolor{currentfill}%
\pgfsetlinewidth{0.501875pt}%
\definecolor{currentstroke}{rgb}{0.000000,0.000000,0.000000}%
\pgfsetstrokecolor{currentstroke}%
\pgfsetdash{}{0pt}%
\pgfsys@defobject{currentmarker}{\pgfqpoint{-0.055556in}{0.000000in}}{\pgfqpoint{-0.000000in}{0.000000in}}{%
\pgfpathmoveto{\pgfqpoint{-0.000000in}{0.000000in}}%
\pgfpathlineto{\pgfqpoint{-0.055556in}{0.000000in}}%
\pgfusepath{stroke,fill}%
}%
\begin{pgfscope}%
\pgfsys@transformshift{7.200000in}{0.600000in}%
\pgfsys@useobject{currentmarker}{}%
\end{pgfscope}%
\end{pgfscope}%
\begin{pgfscope}%
\definecolor{textcolor}{rgb}{0.000000,0.000000,0.000000}%
\pgfsetstrokecolor{textcolor}%
\pgfsetfillcolor{textcolor}%
\pgftext[x=0.944444in,y=0.600000in,right,]{\color{textcolor}\rmfamily\fontsize{10.000000}{12.000000}\selectfont \(\displaystyle {\ensuremath{-}30}\)}%
\end{pgfscope}%
\begin{pgfscope}%
\pgfpathrectangle{\pgfqpoint{1.000000in}{0.600000in}}{\pgfqpoint{6.200000in}{4.800000in}}%
\pgfusepath{clip}%
\pgfsetbuttcap%
\pgfsetroundjoin%
\pgfsetlinewidth{0.501875pt}%
\definecolor{currentstroke}{rgb}{0.000000,0.000000,0.000000}%
\pgfsetstrokecolor{currentstroke}%
\pgfsetdash{{1.000000pt}{3.000000pt}}{0.000000pt}%
\pgfpathmoveto{\pgfqpoint{1.000000in}{1.285714in}}%
\pgfpathlineto{\pgfqpoint{7.200000in}{1.285714in}}%
\pgfusepath{stroke}%
\end{pgfscope}%
\begin{pgfscope}%
\pgfsetbuttcap%
\pgfsetroundjoin%
\definecolor{currentfill}{rgb}{0.000000,0.000000,0.000000}%
\pgfsetfillcolor{currentfill}%
\pgfsetlinewidth{0.501875pt}%
\definecolor{currentstroke}{rgb}{0.000000,0.000000,0.000000}%
\pgfsetstrokecolor{currentstroke}%
\pgfsetdash{}{0pt}%
\pgfsys@defobject{currentmarker}{\pgfqpoint{0.000000in}{0.000000in}}{\pgfqpoint{0.055556in}{0.000000in}}{%
\pgfpathmoveto{\pgfqpoint{0.000000in}{0.000000in}}%
\pgfpathlineto{\pgfqpoint{0.055556in}{0.000000in}}%
\pgfusepath{stroke,fill}%
}%
\begin{pgfscope}%
\pgfsys@transformshift{1.000000in}{1.285714in}%
\pgfsys@useobject{currentmarker}{}%
\end{pgfscope}%
\end{pgfscope}%
\begin{pgfscope}%
\pgfsetbuttcap%
\pgfsetroundjoin%
\definecolor{currentfill}{rgb}{0.000000,0.000000,0.000000}%
\pgfsetfillcolor{currentfill}%
\pgfsetlinewidth{0.501875pt}%
\definecolor{currentstroke}{rgb}{0.000000,0.000000,0.000000}%
\pgfsetstrokecolor{currentstroke}%
\pgfsetdash{}{0pt}%
\pgfsys@defobject{currentmarker}{\pgfqpoint{-0.055556in}{0.000000in}}{\pgfqpoint{-0.000000in}{0.000000in}}{%
\pgfpathmoveto{\pgfqpoint{-0.000000in}{0.000000in}}%
\pgfpathlineto{\pgfqpoint{-0.055556in}{0.000000in}}%
\pgfusepath{stroke,fill}%
}%
\begin{pgfscope}%
\pgfsys@transformshift{7.200000in}{1.285714in}%
\pgfsys@useobject{currentmarker}{}%
\end{pgfscope}%
\end{pgfscope}%
\begin{pgfscope}%
\definecolor{textcolor}{rgb}{0.000000,0.000000,0.000000}%
\pgfsetstrokecolor{textcolor}%
\pgfsetfillcolor{textcolor}%
\pgftext[x=0.944444in,y=1.285714in,right,]{\color{textcolor}\rmfamily\fontsize{10.000000}{12.000000}\selectfont \(\displaystyle {\ensuremath{-}25}\)}%
\end{pgfscope}%
\begin{pgfscope}%
\pgfpathrectangle{\pgfqpoint{1.000000in}{0.600000in}}{\pgfqpoint{6.200000in}{4.800000in}}%
\pgfusepath{clip}%
\pgfsetbuttcap%
\pgfsetroundjoin%
\pgfsetlinewidth{0.501875pt}%
\definecolor{currentstroke}{rgb}{0.000000,0.000000,0.000000}%
\pgfsetstrokecolor{currentstroke}%
\pgfsetdash{{1.000000pt}{3.000000pt}}{0.000000pt}%
\pgfpathmoveto{\pgfqpoint{1.000000in}{1.971429in}}%
\pgfpathlineto{\pgfqpoint{7.200000in}{1.971429in}}%
\pgfusepath{stroke}%
\end{pgfscope}%
\begin{pgfscope}%
\pgfsetbuttcap%
\pgfsetroundjoin%
\definecolor{currentfill}{rgb}{0.000000,0.000000,0.000000}%
\pgfsetfillcolor{currentfill}%
\pgfsetlinewidth{0.501875pt}%
\definecolor{currentstroke}{rgb}{0.000000,0.000000,0.000000}%
\pgfsetstrokecolor{currentstroke}%
\pgfsetdash{}{0pt}%
\pgfsys@defobject{currentmarker}{\pgfqpoint{0.000000in}{0.000000in}}{\pgfqpoint{0.055556in}{0.000000in}}{%
\pgfpathmoveto{\pgfqpoint{0.000000in}{0.000000in}}%
\pgfpathlineto{\pgfqpoint{0.055556in}{0.000000in}}%
\pgfusepath{stroke,fill}%
}%
\begin{pgfscope}%
\pgfsys@transformshift{1.000000in}{1.971429in}%
\pgfsys@useobject{currentmarker}{}%
\end{pgfscope}%
\end{pgfscope}%
\begin{pgfscope}%
\pgfsetbuttcap%
\pgfsetroundjoin%
\definecolor{currentfill}{rgb}{0.000000,0.000000,0.000000}%
\pgfsetfillcolor{currentfill}%
\pgfsetlinewidth{0.501875pt}%
\definecolor{currentstroke}{rgb}{0.000000,0.000000,0.000000}%
\pgfsetstrokecolor{currentstroke}%
\pgfsetdash{}{0pt}%
\pgfsys@defobject{currentmarker}{\pgfqpoint{-0.055556in}{0.000000in}}{\pgfqpoint{-0.000000in}{0.000000in}}{%
\pgfpathmoveto{\pgfqpoint{-0.000000in}{0.000000in}}%
\pgfpathlineto{\pgfqpoint{-0.055556in}{0.000000in}}%
\pgfusepath{stroke,fill}%
}%
\begin{pgfscope}%
\pgfsys@transformshift{7.200000in}{1.971429in}%
\pgfsys@useobject{currentmarker}{}%
\end{pgfscope}%
\end{pgfscope}%
\begin{pgfscope}%
\definecolor{textcolor}{rgb}{0.000000,0.000000,0.000000}%
\pgfsetstrokecolor{textcolor}%
\pgfsetfillcolor{textcolor}%
\pgftext[x=0.944444in,y=1.971429in,right,]{\color{textcolor}\rmfamily\fontsize{10.000000}{12.000000}\selectfont \(\displaystyle {\ensuremath{-}20}\)}%
\end{pgfscope}%
\begin{pgfscope}%
\pgfpathrectangle{\pgfqpoint{1.000000in}{0.600000in}}{\pgfqpoint{6.200000in}{4.800000in}}%
\pgfusepath{clip}%
\pgfsetbuttcap%
\pgfsetroundjoin%
\pgfsetlinewidth{0.501875pt}%
\definecolor{currentstroke}{rgb}{0.000000,0.000000,0.000000}%
\pgfsetstrokecolor{currentstroke}%
\pgfsetdash{{1.000000pt}{3.000000pt}}{0.000000pt}%
\pgfpathmoveto{\pgfqpoint{1.000000in}{2.657143in}}%
\pgfpathlineto{\pgfqpoint{7.200000in}{2.657143in}}%
\pgfusepath{stroke}%
\end{pgfscope}%
\begin{pgfscope}%
\pgfsetbuttcap%
\pgfsetroundjoin%
\definecolor{currentfill}{rgb}{0.000000,0.000000,0.000000}%
\pgfsetfillcolor{currentfill}%
\pgfsetlinewidth{0.501875pt}%
\definecolor{currentstroke}{rgb}{0.000000,0.000000,0.000000}%
\pgfsetstrokecolor{currentstroke}%
\pgfsetdash{}{0pt}%
\pgfsys@defobject{currentmarker}{\pgfqpoint{0.000000in}{0.000000in}}{\pgfqpoint{0.055556in}{0.000000in}}{%
\pgfpathmoveto{\pgfqpoint{0.000000in}{0.000000in}}%
\pgfpathlineto{\pgfqpoint{0.055556in}{0.000000in}}%
\pgfusepath{stroke,fill}%
}%
\begin{pgfscope}%
\pgfsys@transformshift{1.000000in}{2.657143in}%
\pgfsys@useobject{currentmarker}{}%
\end{pgfscope}%
\end{pgfscope}%
\begin{pgfscope}%
\pgfsetbuttcap%
\pgfsetroundjoin%
\definecolor{currentfill}{rgb}{0.000000,0.000000,0.000000}%
\pgfsetfillcolor{currentfill}%
\pgfsetlinewidth{0.501875pt}%
\definecolor{currentstroke}{rgb}{0.000000,0.000000,0.000000}%
\pgfsetstrokecolor{currentstroke}%
\pgfsetdash{}{0pt}%
\pgfsys@defobject{currentmarker}{\pgfqpoint{-0.055556in}{0.000000in}}{\pgfqpoint{-0.000000in}{0.000000in}}{%
\pgfpathmoveto{\pgfqpoint{-0.000000in}{0.000000in}}%
\pgfpathlineto{\pgfqpoint{-0.055556in}{0.000000in}}%
\pgfusepath{stroke,fill}%
}%
\begin{pgfscope}%
\pgfsys@transformshift{7.200000in}{2.657143in}%
\pgfsys@useobject{currentmarker}{}%
\end{pgfscope}%
\end{pgfscope}%
\begin{pgfscope}%
\definecolor{textcolor}{rgb}{0.000000,0.000000,0.000000}%
\pgfsetstrokecolor{textcolor}%
\pgfsetfillcolor{textcolor}%
\pgftext[x=0.944444in,y=2.657143in,right,]{\color{textcolor}\rmfamily\fontsize{10.000000}{12.000000}\selectfont \(\displaystyle {\ensuremath{-}15}\)}%
\end{pgfscope}%
\begin{pgfscope}%
\pgfpathrectangle{\pgfqpoint{1.000000in}{0.600000in}}{\pgfqpoint{6.200000in}{4.800000in}}%
\pgfusepath{clip}%
\pgfsetbuttcap%
\pgfsetroundjoin%
\pgfsetlinewidth{0.501875pt}%
\definecolor{currentstroke}{rgb}{0.000000,0.000000,0.000000}%
\pgfsetstrokecolor{currentstroke}%
\pgfsetdash{{1.000000pt}{3.000000pt}}{0.000000pt}%
\pgfpathmoveto{\pgfqpoint{1.000000in}{3.342857in}}%
\pgfpathlineto{\pgfqpoint{7.200000in}{3.342857in}}%
\pgfusepath{stroke}%
\end{pgfscope}%
\begin{pgfscope}%
\pgfsetbuttcap%
\pgfsetroundjoin%
\definecolor{currentfill}{rgb}{0.000000,0.000000,0.000000}%
\pgfsetfillcolor{currentfill}%
\pgfsetlinewidth{0.501875pt}%
\definecolor{currentstroke}{rgb}{0.000000,0.000000,0.000000}%
\pgfsetstrokecolor{currentstroke}%
\pgfsetdash{}{0pt}%
\pgfsys@defobject{currentmarker}{\pgfqpoint{0.000000in}{0.000000in}}{\pgfqpoint{0.055556in}{0.000000in}}{%
\pgfpathmoveto{\pgfqpoint{0.000000in}{0.000000in}}%
\pgfpathlineto{\pgfqpoint{0.055556in}{0.000000in}}%
\pgfusepath{stroke,fill}%
}%
\begin{pgfscope}%
\pgfsys@transformshift{1.000000in}{3.342857in}%
\pgfsys@useobject{currentmarker}{}%
\end{pgfscope}%
\end{pgfscope}%
\begin{pgfscope}%
\pgfsetbuttcap%
\pgfsetroundjoin%
\definecolor{currentfill}{rgb}{0.000000,0.000000,0.000000}%
\pgfsetfillcolor{currentfill}%
\pgfsetlinewidth{0.501875pt}%
\definecolor{currentstroke}{rgb}{0.000000,0.000000,0.000000}%
\pgfsetstrokecolor{currentstroke}%
\pgfsetdash{}{0pt}%
\pgfsys@defobject{currentmarker}{\pgfqpoint{-0.055556in}{0.000000in}}{\pgfqpoint{-0.000000in}{0.000000in}}{%
\pgfpathmoveto{\pgfqpoint{-0.000000in}{0.000000in}}%
\pgfpathlineto{\pgfqpoint{-0.055556in}{0.000000in}}%
\pgfusepath{stroke,fill}%
}%
\begin{pgfscope}%
\pgfsys@transformshift{7.200000in}{3.342857in}%
\pgfsys@useobject{currentmarker}{}%
\end{pgfscope}%
\end{pgfscope}%
\begin{pgfscope}%
\definecolor{textcolor}{rgb}{0.000000,0.000000,0.000000}%
\pgfsetstrokecolor{textcolor}%
\pgfsetfillcolor{textcolor}%
\pgftext[x=0.944444in,y=3.342857in,right,]{\color{textcolor}\rmfamily\fontsize{10.000000}{12.000000}\selectfont \(\displaystyle {\ensuremath{-}10}\)}%
\end{pgfscope}%
\begin{pgfscope}%
\pgfpathrectangle{\pgfqpoint{1.000000in}{0.600000in}}{\pgfqpoint{6.200000in}{4.800000in}}%
\pgfusepath{clip}%
\pgfsetbuttcap%
\pgfsetroundjoin%
\pgfsetlinewidth{0.501875pt}%
\definecolor{currentstroke}{rgb}{0.000000,0.000000,0.000000}%
\pgfsetstrokecolor{currentstroke}%
\pgfsetdash{{1.000000pt}{3.000000pt}}{0.000000pt}%
\pgfpathmoveto{\pgfqpoint{1.000000in}{4.028571in}}%
\pgfpathlineto{\pgfqpoint{7.200000in}{4.028571in}}%
\pgfusepath{stroke}%
\end{pgfscope}%
\begin{pgfscope}%
\pgfsetbuttcap%
\pgfsetroundjoin%
\definecolor{currentfill}{rgb}{0.000000,0.000000,0.000000}%
\pgfsetfillcolor{currentfill}%
\pgfsetlinewidth{0.501875pt}%
\definecolor{currentstroke}{rgb}{0.000000,0.000000,0.000000}%
\pgfsetstrokecolor{currentstroke}%
\pgfsetdash{}{0pt}%
\pgfsys@defobject{currentmarker}{\pgfqpoint{0.000000in}{0.000000in}}{\pgfqpoint{0.055556in}{0.000000in}}{%
\pgfpathmoveto{\pgfqpoint{0.000000in}{0.000000in}}%
\pgfpathlineto{\pgfqpoint{0.055556in}{0.000000in}}%
\pgfusepath{stroke,fill}%
}%
\begin{pgfscope}%
\pgfsys@transformshift{1.000000in}{4.028571in}%
\pgfsys@useobject{currentmarker}{}%
\end{pgfscope}%
\end{pgfscope}%
\begin{pgfscope}%
\pgfsetbuttcap%
\pgfsetroundjoin%
\definecolor{currentfill}{rgb}{0.000000,0.000000,0.000000}%
\pgfsetfillcolor{currentfill}%
\pgfsetlinewidth{0.501875pt}%
\definecolor{currentstroke}{rgb}{0.000000,0.000000,0.000000}%
\pgfsetstrokecolor{currentstroke}%
\pgfsetdash{}{0pt}%
\pgfsys@defobject{currentmarker}{\pgfqpoint{-0.055556in}{0.000000in}}{\pgfqpoint{-0.000000in}{0.000000in}}{%
\pgfpathmoveto{\pgfqpoint{-0.000000in}{0.000000in}}%
\pgfpathlineto{\pgfqpoint{-0.055556in}{0.000000in}}%
\pgfusepath{stroke,fill}%
}%
\begin{pgfscope}%
\pgfsys@transformshift{7.200000in}{4.028571in}%
\pgfsys@useobject{currentmarker}{}%
\end{pgfscope}%
\end{pgfscope}%
\begin{pgfscope}%
\definecolor{textcolor}{rgb}{0.000000,0.000000,0.000000}%
\pgfsetstrokecolor{textcolor}%
\pgfsetfillcolor{textcolor}%
\pgftext[x=0.944444in,y=4.028571in,right,]{\color{textcolor}\rmfamily\fontsize{10.000000}{12.000000}\selectfont \(\displaystyle {\ensuremath{-}5}\)}%
\end{pgfscope}%
\begin{pgfscope}%
\pgfpathrectangle{\pgfqpoint{1.000000in}{0.600000in}}{\pgfqpoint{6.200000in}{4.800000in}}%
\pgfusepath{clip}%
\pgfsetbuttcap%
\pgfsetroundjoin%
\pgfsetlinewidth{0.501875pt}%
\definecolor{currentstroke}{rgb}{0.000000,0.000000,0.000000}%
\pgfsetstrokecolor{currentstroke}%
\pgfsetdash{{1.000000pt}{3.000000pt}}{0.000000pt}%
\pgfpathmoveto{\pgfqpoint{1.000000in}{4.714286in}}%
\pgfpathlineto{\pgfqpoint{7.200000in}{4.714286in}}%
\pgfusepath{stroke}%
\end{pgfscope}%
\begin{pgfscope}%
\pgfsetbuttcap%
\pgfsetroundjoin%
\definecolor{currentfill}{rgb}{0.000000,0.000000,0.000000}%
\pgfsetfillcolor{currentfill}%
\pgfsetlinewidth{0.501875pt}%
\definecolor{currentstroke}{rgb}{0.000000,0.000000,0.000000}%
\pgfsetstrokecolor{currentstroke}%
\pgfsetdash{}{0pt}%
\pgfsys@defobject{currentmarker}{\pgfqpoint{0.000000in}{0.000000in}}{\pgfqpoint{0.055556in}{0.000000in}}{%
\pgfpathmoveto{\pgfqpoint{0.000000in}{0.000000in}}%
\pgfpathlineto{\pgfqpoint{0.055556in}{0.000000in}}%
\pgfusepath{stroke,fill}%
}%
\begin{pgfscope}%
\pgfsys@transformshift{1.000000in}{4.714286in}%
\pgfsys@useobject{currentmarker}{}%
\end{pgfscope}%
\end{pgfscope}%
\begin{pgfscope}%
\pgfsetbuttcap%
\pgfsetroundjoin%
\definecolor{currentfill}{rgb}{0.000000,0.000000,0.000000}%
\pgfsetfillcolor{currentfill}%
\pgfsetlinewidth{0.501875pt}%
\definecolor{currentstroke}{rgb}{0.000000,0.000000,0.000000}%
\pgfsetstrokecolor{currentstroke}%
\pgfsetdash{}{0pt}%
\pgfsys@defobject{currentmarker}{\pgfqpoint{-0.055556in}{0.000000in}}{\pgfqpoint{-0.000000in}{0.000000in}}{%
\pgfpathmoveto{\pgfqpoint{-0.000000in}{0.000000in}}%
\pgfpathlineto{\pgfqpoint{-0.055556in}{0.000000in}}%
\pgfusepath{stroke,fill}%
}%
\begin{pgfscope}%
\pgfsys@transformshift{7.200000in}{4.714286in}%
\pgfsys@useobject{currentmarker}{}%
\end{pgfscope}%
\end{pgfscope}%
\begin{pgfscope}%
\definecolor{textcolor}{rgb}{0.000000,0.000000,0.000000}%
\pgfsetstrokecolor{textcolor}%
\pgfsetfillcolor{textcolor}%
\pgftext[x=0.944444in,y=4.714286in,right,]{\color{textcolor}\rmfamily\fontsize{10.000000}{12.000000}\selectfont \(\displaystyle {0}\)}%
\end{pgfscope}%
\begin{pgfscope}%
\pgfpathrectangle{\pgfqpoint{1.000000in}{0.600000in}}{\pgfqpoint{6.200000in}{4.800000in}}%
\pgfusepath{clip}%
\pgfsetbuttcap%
\pgfsetroundjoin%
\pgfsetlinewidth{0.501875pt}%
\definecolor{currentstroke}{rgb}{0.000000,0.000000,0.000000}%
\pgfsetstrokecolor{currentstroke}%
\pgfsetdash{{1.000000pt}{3.000000pt}}{0.000000pt}%
\pgfpathmoveto{\pgfqpoint{1.000000in}{5.400000in}}%
\pgfpathlineto{\pgfqpoint{7.200000in}{5.400000in}}%
\pgfusepath{stroke}%
\end{pgfscope}%
\begin{pgfscope}%
\pgfsetbuttcap%
\pgfsetroundjoin%
\definecolor{currentfill}{rgb}{0.000000,0.000000,0.000000}%
\pgfsetfillcolor{currentfill}%
\pgfsetlinewidth{0.501875pt}%
\definecolor{currentstroke}{rgb}{0.000000,0.000000,0.000000}%
\pgfsetstrokecolor{currentstroke}%
\pgfsetdash{}{0pt}%
\pgfsys@defobject{currentmarker}{\pgfqpoint{0.000000in}{0.000000in}}{\pgfqpoint{0.055556in}{0.000000in}}{%
\pgfpathmoveto{\pgfqpoint{0.000000in}{0.000000in}}%
\pgfpathlineto{\pgfqpoint{0.055556in}{0.000000in}}%
\pgfusepath{stroke,fill}%
}%
\begin{pgfscope}%
\pgfsys@transformshift{1.000000in}{5.400000in}%
\pgfsys@useobject{currentmarker}{}%
\end{pgfscope}%
\end{pgfscope}%
\begin{pgfscope}%
\pgfsetbuttcap%
\pgfsetroundjoin%
\definecolor{currentfill}{rgb}{0.000000,0.000000,0.000000}%
\pgfsetfillcolor{currentfill}%
\pgfsetlinewidth{0.501875pt}%
\definecolor{currentstroke}{rgb}{0.000000,0.000000,0.000000}%
\pgfsetstrokecolor{currentstroke}%
\pgfsetdash{}{0pt}%
\pgfsys@defobject{currentmarker}{\pgfqpoint{-0.055556in}{0.000000in}}{\pgfqpoint{-0.000000in}{0.000000in}}{%
\pgfpathmoveto{\pgfqpoint{-0.000000in}{0.000000in}}%
\pgfpathlineto{\pgfqpoint{-0.055556in}{0.000000in}}%
\pgfusepath{stroke,fill}%
}%
\begin{pgfscope}%
\pgfsys@transformshift{7.200000in}{5.400000in}%
\pgfsys@useobject{currentmarker}{}%
\end{pgfscope}%
\end{pgfscope}%
\begin{pgfscope}%
\definecolor{textcolor}{rgb}{0.000000,0.000000,0.000000}%
\pgfsetstrokecolor{textcolor}%
\pgfsetfillcolor{textcolor}%
\pgftext[x=0.944444in,y=5.400000in,right,]{\color{textcolor}\rmfamily\fontsize{10.000000}{12.000000}\selectfont \(\displaystyle {5}\)}%
\end{pgfscope}%
\begin{pgfscope}%
\definecolor{textcolor}{rgb}{0.000000,0.000000,0.000000}%
\pgfsetstrokecolor{textcolor}%
\pgfsetfillcolor{textcolor}%
\pgftext[x=0.628086in,y=3.000000in,,bottom,rotate=90.000000]{\color{textcolor}\rmfamily\fontsize{12.000000}{14.400000}\selectfont \(\displaystyle \theta\ (rad)\)}%
\end{pgfscope}%
\begin{pgfscope}%
\definecolor{textcolor}{rgb}{0.000000,0.000000,0.000000}%
\pgfsetstrokecolor{textcolor}%
\pgfsetfillcolor{textcolor}%
\pgftext[x=4.100000in,y=5.469444in,,base]{\color{textcolor}\rmfamily\fontsize{12.000000}{14.400000}\selectfont \(\displaystyle Simple\ pendulum\ solution\ (time\ step = 0.05\ s)\)}%
\end{pgfscope}%
\begin{pgfscope}%
\pgfsetbuttcap%
\pgfsetmiterjoin%
\definecolor{currentfill}{rgb}{1.000000,1.000000,1.000000}%
\pgfsetfillcolor{currentfill}%
\pgfsetlinewidth{1.003750pt}%
\definecolor{currentstroke}{rgb}{0.000000,0.000000,0.000000}%
\pgfsetstrokecolor{currentstroke}%
\pgfsetdash{}{0pt}%
\pgfpathmoveto{\pgfqpoint{1.083333in}{0.683333in}}%
\pgfpathlineto{\pgfqpoint{3.093110in}{0.683333in}}%
\pgfpathlineto{\pgfqpoint{3.093110in}{1.430555in}}%
\pgfpathlineto{\pgfqpoint{1.083333in}{1.430555in}}%
\pgfpathlineto{\pgfqpoint{1.083333in}{0.683333in}}%
\pgfpathclose%
\pgfusepath{stroke,fill}%
\end{pgfscope}%
\begin{pgfscope}%
\pgfsetrectcap%
\pgfsetroundjoin%
\pgfsetlinewidth{1.003750pt}%
\definecolor{currentstroke}{rgb}{1.000000,0.000000,0.000000}%
\pgfsetstrokecolor{currentstroke}%
\pgfsetdash{}{0pt}%
\pgfpathmoveto{\pgfqpoint{1.200000in}{1.305555in}}%
\pgfpathlineto{\pgfqpoint{1.433333in}{1.305555in}}%
\pgfusepath{stroke}%
\end{pgfscope}%
\begin{pgfscope}%
\definecolor{textcolor}{rgb}{0.000000,0.000000,0.000000}%
\pgfsetstrokecolor{textcolor}%
\pgfsetfillcolor{textcolor}%
\pgftext[x=1.616667in,y=1.247221in,left,base]{\color{textcolor}\rmfamily\fontsize{12.000000}{14.400000}\selectfont \(\displaystyle euler\ explicit\)}%
\end{pgfscope}%
\begin{pgfscope}%
\pgfsetrectcap%
\pgfsetroundjoin%
\pgfsetlinewidth{1.003750pt}%
\definecolor{currentstroke}{rgb}{0.000000,0.000000,1.000000}%
\pgfsetstrokecolor{currentstroke}%
\pgfsetdash{}{0pt}%
\pgfpathmoveto{\pgfqpoint{1.200000in}{1.073147in}}%
\pgfpathlineto{\pgfqpoint{1.433333in}{1.073147in}}%
\pgfusepath{stroke}%
\end{pgfscope}%
\begin{pgfscope}%
\definecolor{textcolor}{rgb}{0.000000,0.000000,0.000000}%
\pgfsetstrokecolor{textcolor}%
\pgfsetfillcolor{textcolor}%
\pgftext[x=1.616667in,y=1.014814in,left,base]{\color{textcolor}\rmfamily\fontsize{12.000000}{14.400000}\selectfont \(\displaystyle euler\ implicit\)}%
\end{pgfscope}%
\begin{pgfscope}%
\pgfsetrectcap%
\pgfsetroundjoin%
\pgfsetlinewidth{1.003750pt}%
\definecolor{currentstroke}{rgb}{0.000000,0.000000,0.000000}%
\pgfsetstrokecolor{currentstroke}%
\pgfsetdash{}{0pt}%
\pgfpathmoveto{\pgfqpoint{1.200000in}{0.840740in}}%
\pgfpathlineto{\pgfqpoint{1.433333in}{0.840740in}}%
\pgfusepath{stroke}%
\end{pgfscope}%
\begin{pgfscope}%
\definecolor{textcolor}{rgb}{0.000000,0.000000,0.000000}%
\pgfsetstrokecolor{textcolor}%
\pgfsetfillcolor{textcolor}%
\pgftext[x=1.616667in,y=0.782407in,left,base]{\color{textcolor}\rmfamily\fontsize{12.000000}{14.400000}\selectfont \(\displaystyle trapezoidal\ scheme\)}%
\end{pgfscope}%
\end{pgfpicture}%
\makeatother%
\endgroup%
}
    \end{figure}

    \begin{figure}[ht!]
    \centering
    \resizebox{0.9\linewidth}{!}{%% Creator: Matplotlib, PGF backend
%%
%% To include the figure in your LaTeX document, write
%%   \input{<filename>.pgf}
%%
%% Make sure the required packages are loaded in your preamble
%%   \usepackage{pgf}
%%
%% Also ensure that all the required font packages are loaded; for instance,
%% the lmodern package is sometimes necessary when using math font.
%%   \usepackage{lmodern}
%%
%% Figures using additional raster images can only be included by \input if
%% they are in the same directory as the main LaTeX file. For loading figures
%% from other directories you can use the `import` package
%%   \usepackage{import}
%%
%% and then include the figures with
%%   \import{<path to file>}{<filename>.pgf}
%%
%% Matplotlib used the following preamble
%%
\begingroup%
\makeatletter%
\begin{pgfpicture}%
\pgfpathrectangle{\pgfpointorigin}{\pgfqpoint{8.000000in}{6.000000in}}%
\pgfusepath{use as bounding box, clip}%
\begin{pgfscope}%
\pgfsetbuttcap%
\pgfsetmiterjoin%
\definecolor{currentfill}{rgb}{1.000000,1.000000,1.000000}%
\pgfsetfillcolor{currentfill}%
\pgfsetlinewidth{0.000000pt}%
\definecolor{currentstroke}{rgb}{1.000000,1.000000,1.000000}%
\pgfsetstrokecolor{currentstroke}%
\pgfsetdash{}{0pt}%
\pgfpathmoveto{\pgfqpoint{0.000000in}{0.000000in}}%
\pgfpathlineto{\pgfqpoint{8.000000in}{0.000000in}}%
\pgfpathlineto{\pgfqpoint{8.000000in}{6.000000in}}%
\pgfpathlineto{\pgfqpoint{0.000000in}{6.000000in}}%
\pgfpathlineto{\pgfqpoint{0.000000in}{0.000000in}}%
\pgfpathclose%
\pgfusepath{fill}%
\end{pgfscope}%
\begin{pgfscope}%
\pgfsetbuttcap%
\pgfsetmiterjoin%
\definecolor{currentfill}{rgb}{1.000000,1.000000,1.000000}%
\pgfsetfillcolor{currentfill}%
\pgfsetlinewidth{0.000000pt}%
\definecolor{currentstroke}{rgb}{0.000000,0.000000,0.000000}%
\pgfsetstrokecolor{currentstroke}%
\pgfsetstrokeopacity{0.000000}%
\pgfsetdash{}{0pt}%
\pgfpathmoveto{\pgfqpoint{1.000000in}{0.600000in}}%
\pgfpathlineto{\pgfqpoint{7.200000in}{0.600000in}}%
\pgfpathlineto{\pgfqpoint{7.200000in}{5.400000in}}%
\pgfpathlineto{\pgfqpoint{1.000000in}{5.400000in}}%
\pgfpathlineto{\pgfqpoint{1.000000in}{0.600000in}}%
\pgfpathclose%
\pgfusepath{fill}%
\end{pgfscope}%
\begin{pgfscope}%
\pgfpathrectangle{\pgfqpoint{1.000000in}{0.600000in}}{\pgfqpoint{6.200000in}{4.800000in}}%
\pgfusepath{clip}%
\pgfsetrectcap%
\pgfsetroundjoin%
\pgfsetlinewidth{1.003750pt}%
\definecolor{currentstroke}{rgb}{1.000000,0.000000,0.000000}%
\pgfsetstrokecolor{currentstroke}%
\pgfsetdash{}{0pt}%
\pgfpathmoveto{\pgfqpoint{1.000000in}{4.815708in}}%
\pgfpathlineto{\pgfqpoint{1.024800in}{4.814809in}}%
\pgfpathlineto{\pgfqpoint{1.049600in}{4.811600in}}%
\pgfpathlineto{\pgfqpoint{1.080600in}{4.804947in}}%
\pgfpathlineto{\pgfqpoint{1.161200in}{4.785242in}}%
\pgfpathlineto{\pgfqpoint{1.186000in}{4.782213in}}%
\pgfpathlineto{\pgfqpoint{1.210800in}{4.781767in}}%
\pgfpathlineto{\pgfqpoint{1.235600in}{4.784053in}}%
\pgfpathlineto{\pgfqpoint{1.260400in}{4.788854in}}%
\pgfpathlineto{\pgfqpoint{1.291400in}{4.797454in}}%
\pgfpathlineto{\pgfqpoint{1.353400in}{4.815510in}}%
\pgfpathlineto{\pgfqpoint{1.378200in}{4.819912in}}%
\pgfpathlineto{\pgfqpoint{1.396800in}{4.821387in}}%
\pgfpathlineto{\pgfqpoint{1.415400in}{4.821126in}}%
\pgfpathlineto{\pgfqpoint{1.434000in}{4.819106in}}%
\pgfpathlineto{\pgfqpoint{1.458800in}{4.813871in}}%
\pgfpathlineto{\pgfqpoint{1.483600in}{4.806309in}}%
\pgfpathlineto{\pgfqpoint{1.558000in}{4.781267in}}%
\pgfpathlineto{\pgfqpoint{1.582800in}{4.776445in}}%
\pgfpathlineto{\pgfqpoint{1.601400in}{4.774943in}}%
\pgfpathlineto{\pgfqpoint{1.620000in}{4.775416in}}%
\pgfpathlineto{\pgfqpoint{1.638600in}{4.777876in}}%
\pgfpathlineto{\pgfqpoint{1.657200in}{4.782209in}}%
\pgfpathlineto{\pgfqpoint{1.682000in}{4.790435in}}%
\pgfpathlineto{\pgfqpoint{1.725400in}{4.808301in}}%
\pgfpathlineto{\pgfqpoint{1.756400in}{4.819908in}}%
\pgfpathlineto{\pgfqpoint{1.775000in}{4.825032in}}%
\pgfpathlineto{\pgfqpoint{1.793600in}{4.828263in}}%
\pgfpathlineto{\pgfqpoint{1.812200in}{4.829365in}}%
\pgfpathlineto{\pgfqpoint{1.830800in}{4.828242in}}%
\pgfpathlineto{\pgfqpoint{1.849400in}{4.824912in}}%
\pgfpathlineto{\pgfqpoint{1.868000in}{4.819520in}}%
\pgfpathlineto{\pgfqpoint{1.892800in}{4.809671in}}%
\pgfpathlineto{\pgfqpoint{1.973400in}{4.773791in}}%
\pgfpathlineto{\pgfqpoint{1.992000in}{4.768849in}}%
\pgfpathlineto{\pgfqpoint{2.010600in}{4.766154in}}%
\pgfpathlineto{\pgfqpoint{2.029200in}{4.765865in}}%
\pgfpathlineto{\pgfqpoint{2.047800in}{4.768027in}}%
\pgfpathlineto{\pgfqpoint{2.066400in}{4.772586in}}%
\pgfpathlineto{\pgfqpoint{2.085000in}{4.779363in}}%
\pgfpathlineto{\pgfqpoint{2.109800in}{4.791204in}}%
\pgfpathlineto{\pgfqpoint{2.178000in}{4.826989in}}%
\pgfpathlineto{\pgfqpoint{2.196600in}{4.833741in}}%
\pgfpathlineto{\pgfqpoint{2.215200in}{4.838157in}}%
\pgfpathlineto{\pgfqpoint{2.233800in}{4.840034in}}%
\pgfpathlineto{\pgfqpoint{2.252400in}{4.839299in}}%
\pgfpathlineto{\pgfqpoint{2.271000in}{4.835948in}}%
\pgfpathlineto{\pgfqpoint{2.289600in}{4.830047in}}%
\pgfpathlineto{\pgfqpoint{2.308200in}{4.821786in}}%
\pgfpathlineto{\pgfqpoint{2.333000in}{4.807825in}}%
\pgfpathlineto{\pgfqpoint{2.395000in}{4.770422in}}%
\pgfpathlineto{\pgfqpoint{2.413600in}{4.762293in}}%
\pgfpathlineto{\pgfqpoint{2.432200in}{4.756637in}}%
\pgfpathlineto{\pgfqpoint{2.450800in}{4.753629in}}%
\pgfpathlineto{\pgfqpoint{2.469400in}{4.753315in}}%
\pgfpathlineto{\pgfqpoint{2.488000in}{4.755701in}}%
\pgfpathlineto{\pgfqpoint{2.506600in}{4.760780in}}%
\pgfpathlineto{\pgfqpoint{2.525200in}{4.768496in}}%
\pgfpathlineto{\pgfqpoint{2.543800in}{4.778651in}}%
\pgfpathlineto{\pgfqpoint{2.568600in}{4.795122in}}%
\pgfpathlineto{\pgfqpoint{2.618200in}{4.829548in}}%
\pgfpathlineto{\pgfqpoint{2.636800in}{4.839736in}}%
\pgfpathlineto{\pgfqpoint{2.655400in}{4.847414in}}%
\pgfpathlineto{\pgfqpoint{2.674000in}{4.852410in}}%
\pgfpathlineto{\pgfqpoint{2.692600in}{4.854732in}}%
\pgfpathlineto{\pgfqpoint{2.711200in}{4.854426in}}%
\pgfpathlineto{\pgfqpoint{2.729800in}{4.851504in}}%
\pgfpathlineto{\pgfqpoint{2.748400in}{4.845930in}}%
\pgfpathlineto{\pgfqpoint{2.767000in}{4.837664in}}%
\pgfpathlineto{\pgfqpoint{2.785600in}{4.826780in}}%
\pgfpathlineto{\pgfqpoint{2.810400in}{4.808853in}}%
\pgfpathlineto{\pgfqpoint{2.866200in}{4.765619in}}%
\pgfpathlineto{\pgfqpoint{2.884800in}{4.754375in}}%
\pgfpathlineto{\pgfqpoint{2.903400in}{4.745765in}}%
\pgfpathlineto{\pgfqpoint{2.922000in}{4.739813in}}%
\pgfpathlineto{\pgfqpoint{2.940600in}{4.736369in}}%
\pgfpathlineto{\pgfqpoint{2.959200in}{4.735272in}}%
\pgfpathlineto{\pgfqpoint{2.977800in}{4.736438in}}%
\pgfpathlineto{\pgfqpoint{2.996400in}{4.739886in}}%
\pgfpathlineto{\pgfqpoint{3.015000in}{4.745733in}}%
\pgfpathlineto{\pgfqpoint{3.033600in}{4.754146in}}%
\pgfpathlineto{\pgfqpoint{3.052200in}{4.765237in}}%
\pgfpathlineto{\pgfqpoint{3.070800in}{4.778891in}}%
\pgfpathlineto{\pgfqpoint{3.095600in}{4.800050in}}%
\pgfpathlineto{\pgfqpoint{3.132800in}{4.832272in}}%
\pgfpathlineto{\pgfqpoint{3.151400in}{4.845824in}}%
\pgfpathlineto{\pgfqpoint{3.170000in}{4.856763in}}%
\pgfpathlineto{\pgfqpoint{3.188600in}{4.865082in}}%
\pgfpathlineto{\pgfqpoint{3.207200in}{4.871055in}}%
\pgfpathlineto{\pgfqpoint{3.225800in}{4.875029in}}%
\pgfpathlineto{\pgfqpoint{3.250600in}{4.877720in}}%
\pgfpathlineto{\pgfqpoint{3.275400in}{4.877814in}}%
\pgfpathlineto{\pgfqpoint{3.300200in}{4.875411in}}%
\pgfpathlineto{\pgfqpoint{3.325000in}{4.870274in}}%
\pgfpathlineto{\pgfqpoint{3.343600in}{4.864309in}}%
\pgfpathlineto{\pgfqpoint{3.362200in}{4.856187in}}%
\pgfpathlineto{\pgfqpoint{3.380800in}{4.845588in}}%
\pgfpathlineto{\pgfqpoint{3.399400in}{4.832323in}}%
\pgfpathlineto{\pgfqpoint{3.418000in}{4.816554in}}%
\pgfpathlineto{\pgfqpoint{3.492400in}{4.749573in}}%
\pgfpathlineto{\pgfqpoint{3.511000in}{4.737597in}}%
\pgfpathlineto{\pgfqpoint{3.529600in}{4.728102in}}%
\pgfpathlineto{\pgfqpoint{3.554400in}{4.718474in}}%
\pgfpathlineto{\pgfqpoint{3.585400in}{4.709526in}}%
\pgfpathlineto{\pgfqpoint{3.659800in}{4.689928in}}%
\pgfpathlineto{\pgfqpoint{3.684600in}{4.680474in}}%
\pgfpathlineto{\pgfqpoint{3.703200in}{4.671288in}}%
\pgfpathlineto{\pgfqpoint{3.721800in}{4.659787in}}%
\pgfpathlineto{\pgfqpoint{3.740400in}{4.645654in}}%
\pgfpathlineto{\pgfqpoint{3.759000in}{4.628967in}}%
\pgfpathlineto{\pgfqpoint{3.833400in}{4.558140in}}%
\pgfpathlineto{\pgfqpoint{3.852000in}{4.545276in}}%
\pgfpathlineto{\pgfqpoint{3.870600in}{4.534755in}}%
\pgfpathlineto{\pgfqpoint{3.895400in}{4.523278in}}%
\pgfpathlineto{\pgfqpoint{3.963600in}{4.493955in}}%
\pgfpathlineto{\pgfqpoint{3.982200in}{4.483401in}}%
\pgfpathlineto{\pgfqpoint{4.000800in}{4.470584in}}%
\pgfpathlineto{\pgfqpoint{4.019400in}{4.455141in}}%
\pgfpathlineto{\pgfqpoint{4.044200in}{4.430777in}}%
\pgfpathlineto{\pgfqpoint{4.093800in}{4.379729in}}%
\pgfpathlineto{\pgfqpoint{4.112400in}{4.364043in}}%
\pgfpathlineto{\pgfqpoint{4.131000in}{4.350940in}}%
\pgfpathlineto{\pgfqpoint{4.155800in}{4.336534in}}%
\pgfpathlineto{\pgfqpoint{4.224000in}{4.299853in}}%
\pgfpathlineto{\pgfqpoint{4.242600in}{4.286803in}}%
\pgfpathlineto{\pgfqpoint{4.261200in}{4.271229in}}%
\pgfpathlineto{\pgfqpoint{4.279800in}{4.253049in}}%
\pgfpathlineto{\pgfqpoint{4.348000in}{4.181564in}}%
\pgfpathlineto{\pgfqpoint{4.366600in}{4.166456in}}%
\pgfpathlineto{\pgfqpoint{4.391400in}{4.149732in}}%
\pgfpathlineto{\pgfqpoint{4.459600in}{4.107450in}}%
\pgfpathlineto{\pgfqpoint{4.478200in}{4.092596in}}%
\pgfpathlineto{\pgfqpoint{4.496800in}{4.075124in}}%
\pgfpathlineto{\pgfqpoint{4.521600in}{4.048229in}}%
\pgfpathlineto{\pgfqpoint{4.558800in}{4.006684in}}%
\pgfpathlineto{\pgfqpoint{4.577400in}{3.988731in}}%
\pgfpathlineto{\pgfqpoint{4.596000in}{3.973384in}}%
\pgfpathlineto{\pgfqpoint{4.620800in}{3.955957in}}%
\pgfpathlineto{\pgfqpoint{4.670400in}{3.922486in}}%
\pgfpathlineto{\pgfqpoint{4.689000in}{3.907306in}}%
\pgfpathlineto{\pgfqpoint{4.707600in}{3.889569in}}%
\pgfpathlineto{\pgfqpoint{4.732400in}{3.862161in}}%
\pgfpathlineto{\pgfqpoint{4.775800in}{3.812651in}}%
\pgfpathlineto{\pgfqpoint{4.794400in}{3.794735in}}%
\pgfpathlineto{\pgfqpoint{4.819200in}{3.774604in}}%
\pgfpathlineto{\pgfqpoint{4.887400in}{3.723515in}}%
\pgfpathlineto{\pgfqpoint{4.906000in}{3.705791in}}%
\pgfpathlineto{\pgfqpoint{4.930800in}{3.678218in}}%
\pgfpathlineto{\pgfqpoint{4.980400in}{3.620358in}}%
\pgfpathlineto{\pgfqpoint{4.999000in}{3.602271in}}%
\pgfpathlineto{\pgfqpoint{5.023800in}{3.581676in}}%
\pgfpathlineto{\pgfqpoint{5.079600in}{3.537508in}}%
\pgfpathlineto{\pgfqpoint{5.098200in}{3.519461in}}%
\pgfpathlineto{\pgfqpoint{5.123000in}{3.491411in}}%
\pgfpathlineto{\pgfqpoint{5.172600in}{3.431991in}}%
\pgfpathlineto{\pgfqpoint{5.191200in}{3.413237in}}%
\pgfpathlineto{\pgfqpoint{5.216000in}{3.391717in}}%
\pgfpathlineto{\pgfqpoint{5.265600in}{3.350462in}}%
\pgfpathlineto{\pgfqpoint{5.284200in}{3.331961in}}%
\pgfpathlineto{\pgfqpoint{5.309000in}{3.303295in}}%
\pgfpathlineto{\pgfqpoint{5.358600in}{3.242540in}}%
\pgfpathlineto{\pgfqpoint{5.377200in}{3.223301in}}%
\pgfpathlineto{\pgfqpoint{5.402000in}{3.201031in}}%
\pgfpathlineto{\pgfqpoint{5.445400in}{3.163299in}}%
\pgfpathlineto{\pgfqpoint{5.464000in}{3.144356in}}%
\pgfpathlineto{\pgfqpoint{5.488800in}{3.115096in}}%
\pgfpathlineto{\pgfqpoint{5.538400in}{3.053081in}}%
\pgfpathlineto{\pgfqpoint{5.557000in}{3.033387in}}%
\pgfpathlineto{\pgfqpoint{5.581800in}{3.010403in}}%
\pgfpathlineto{\pgfqpoint{5.619000in}{2.976803in}}%
\pgfpathlineto{\pgfqpoint{5.637600in}{2.957533in}}%
\pgfpathlineto{\pgfqpoint{5.656200in}{2.935654in}}%
\pgfpathlineto{\pgfqpoint{5.724400in}{2.850752in}}%
\pgfpathlineto{\pgfqpoint{5.749200in}{2.826267in}}%
\pgfpathlineto{\pgfqpoint{5.792600in}{2.785434in}}%
\pgfpathlineto{\pgfqpoint{5.811200in}{2.765204in}}%
\pgfpathlineto{\pgfqpoint{5.836000in}{2.734280in}}%
\pgfpathlineto{\pgfqpoint{5.879400in}{2.677694in}}%
\pgfpathlineto{\pgfqpoint{5.898000in}{2.656748in}}%
\pgfpathlineto{\pgfqpoint{5.922800in}{2.632125in}}%
\pgfpathlineto{\pgfqpoint{5.960000in}{2.595738in}}%
\pgfpathlineto{\pgfqpoint{5.978600in}{2.574817in}}%
\pgfpathlineto{\pgfqpoint{6.003400in}{2.543020in}}%
\pgfpathlineto{\pgfqpoint{6.046800in}{2.485764in}}%
\pgfpathlineto{\pgfqpoint{6.071600in}{2.458226in}}%
\pgfpathlineto{\pgfqpoint{6.133600in}{2.394488in}}%
\pgfpathlineto{\pgfqpoint{6.158400in}{2.363236in}}%
\pgfpathlineto{\pgfqpoint{6.214200in}{2.289181in}}%
\pgfpathlineto{\pgfqpoint{6.239000in}{2.262079in}}%
\pgfpathlineto{\pgfqpoint{6.288600in}{2.210000in}}%
\pgfpathlineto{\pgfqpoint{6.307200in}{2.187001in}}%
\pgfpathlineto{\pgfqpoint{6.338200in}{2.144230in}}%
\pgfpathlineto{\pgfqpoint{6.363000in}{2.110846in}}%
\pgfpathlineto{\pgfqpoint{6.387800in}{2.081774in}}%
\pgfpathlineto{\pgfqpoint{6.449800in}{2.014225in}}%
\pgfpathlineto{\pgfqpoint{6.474600in}{1.981351in}}%
\pgfpathlineto{\pgfqpoint{6.518000in}{1.921080in}}%
\pgfpathlineto{\pgfqpoint{6.542800in}{1.891598in}}%
\pgfpathlineto{\pgfqpoint{6.598600in}{1.829785in}}%
\pgfpathlineto{\pgfqpoint{6.623400in}{1.796931in}}%
\pgfpathlineto{\pgfqpoint{6.673000in}{1.727495in}}%
\pgfpathlineto{\pgfqpoint{6.697800in}{1.698099in}}%
\pgfpathlineto{\pgfqpoint{6.747400in}{1.641950in}}%
\pgfpathlineto{\pgfqpoint{6.772200in}{1.608658in}}%
\pgfpathlineto{\pgfqpoint{6.821800in}{1.538296in}}%
\pgfpathlineto{\pgfqpoint{6.846600in}{1.508428in}}%
\pgfpathlineto{\pgfqpoint{6.890000in}{1.458654in}}%
\pgfpathlineto{\pgfqpoint{6.914800in}{1.425596in}}%
\pgfpathlineto{\pgfqpoint{6.976800in}{1.337951in}}%
\pgfpathlineto{\pgfqpoint{7.007800in}{1.301692in}}%
\pgfpathlineto{\pgfqpoint{7.038800in}{1.264538in}}%
\pgfpathlineto{\pgfqpoint{7.063600in}{1.230170in}}%
\pgfpathlineto{\pgfqpoint{7.113200in}{1.158227in}}%
\pgfpathlineto{\pgfqpoint{7.138000in}{1.127639in}}%
\pgfpathlineto{\pgfqpoint{7.181400in}{1.075333in}}%
\pgfpathlineto{\pgfqpoint{7.200000in}{1.049522in}}%
\pgfpathlineto{\pgfqpoint{7.200000in}{1.049522in}}%
\pgfusepath{stroke}%
\end{pgfscope}%
\begin{pgfscope}%
\pgfpathrectangle{\pgfqpoint{1.000000in}{0.600000in}}{\pgfqpoint{6.200000in}{4.800000in}}%
\pgfusepath{clip}%
\pgfsetrectcap%
\pgfsetroundjoin%
\pgfsetlinewidth{1.003750pt}%
\definecolor{currentstroke}{rgb}{0.000000,0.000000,1.000000}%
\pgfsetstrokecolor{currentstroke}%
\pgfsetdash{}{0pt}%
\pgfpathmoveto{\pgfqpoint{1.000000in}{4.815708in}}%
\pgfpathlineto{\pgfqpoint{1.024800in}{4.814233in}}%
\pgfpathlineto{\pgfqpoint{1.049600in}{4.810652in}}%
\pgfpathlineto{\pgfqpoint{1.086800in}{4.802708in}}%
\pgfpathlineto{\pgfqpoint{1.136400in}{4.791908in}}%
\pgfpathlineto{\pgfqpoint{1.167400in}{4.787692in}}%
\pgfpathlineto{\pgfqpoint{1.192200in}{4.786533in}}%
\pgfpathlineto{\pgfqpoint{1.217000in}{4.787466in}}%
\pgfpathlineto{\pgfqpoint{1.248000in}{4.791227in}}%
\pgfpathlineto{\pgfqpoint{1.291400in}{4.799403in}}%
\pgfpathlineto{\pgfqpoint{1.334800in}{4.807306in}}%
\pgfpathlineto{\pgfqpoint{1.365800in}{4.810731in}}%
\pgfpathlineto{\pgfqpoint{1.390600in}{4.811531in}}%
\pgfpathlineto{\pgfqpoint{1.421600in}{4.810021in}}%
\pgfpathlineto{\pgfqpoint{1.452600in}{4.806192in}}%
\pgfpathlineto{\pgfqpoint{1.551800in}{4.791454in}}%
\pgfpathlineto{\pgfqpoint{1.582800in}{4.790124in}}%
\pgfpathlineto{\pgfqpoint{1.613800in}{4.791212in}}%
\pgfpathlineto{\pgfqpoint{1.651000in}{4.795158in}}%
\pgfpathlineto{\pgfqpoint{1.744000in}{4.807135in}}%
\pgfpathlineto{\pgfqpoint{1.775000in}{4.808443in}}%
\pgfpathlineto{\pgfqpoint{1.806000in}{4.807666in}}%
\pgfpathlineto{\pgfqpoint{1.843200in}{4.804409in}}%
\pgfpathlineto{\pgfqpoint{1.948600in}{4.793327in}}%
\pgfpathlineto{\pgfqpoint{1.985800in}{4.792908in}}%
\pgfpathlineto{\pgfqpoint{2.023000in}{4.794934in}}%
\pgfpathlineto{\pgfqpoint{2.085000in}{4.801255in}}%
\pgfpathlineto{\pgfqpoint{2.134600in}{4.805341in}}%
\pgfpathlineto{\pgfqpoint{2.171800in}{4.806185in}}%
\pgfpathlineto{\pgfqpoint{2.209000in}{4.804865in}}%
\pgfpathlineto{\pgfqpoint{2.264800in}{4.800229in}}%
\pgfpathlineto{\pgfqpoint{2.326800in}{4.795508in}}%
\pgfpathlineto{\pgfqpoint{2.370200in}{4.794746in}}%
\pgfpathlineto{\pgfqpoint{2.413600in}{4.796448in}}%
\pgfpathlineto{\pgfqpoint{2.550000in}{4.804532in}}%
\pgfpathlineto{\pgfqpoint{2.593400in}{4.803709in}}%
\pgfpathlineto{\pgfqpoint{2.655400in}{4.799967in}}%
\pgfpathlineto{\pgfqpoint{2.723600in}{4.796430in}}%
\pgfpathlineto{\pgfqpoint{2.773200in}{4.796406in}}%
\pgfpathlineto{\pgfqpoint{2.829000in}{4.798857in}}%
\pgfpathlineto{\pgfqpoint{2.922000in}{4.803150in}}%
\pgfpathlineto{\pgfqpoint{2.971600in}{4.802971in}}%
\pgfpathlineto{\pgfqpoint{3.039800in}{4.800122in}}%
\pgfpathlineto{\pgfqpoint{3.114200in}{4.797325in}}%
\pgfpathlineto{\pgfqpoint{3.170000in}{4.797563in}}%
\pgfpathlineto{\pgfqpoint{3.256800in}{4.800865in}}%
\pgfpathlineto{\pgfqpoint{3.325000in}{4.802437in}}%
\pgfpathlineto{\pgfqpoint{3.387000in}{4.801485in}}%
\pgfpathlineto{\pgfqpoint{3.529600in}{4.797919in}}%
\pgfpathlineto{\pgfqpoint{3.604000in}{4.799405in}}%
\pgfpathlineto{\pgfqpoint{3.715600in}{4.801791in}}%
\pgfpathlineto{\pgfqpoint{3.790000in}{4.800705in}}%
\pgfpathlineto{\pgfqpoint{3.914000in}{4.798469in}}%
\pgfpathlineto{\pgfqpoint{4.000800in}{4.799787in}}%
\pgfpathlineto{\pgfqpoint{4.112400in}{4.801300in}}%
\pgfpathlineto{\pgfqpoint{4.211600in}{4.799842in}}%
\pgfpathlineto{\pgfqpoint{4.317000in}{4.798932in}}%
\pgfpathlineto{\pgfqpoint{4.608400in}{4.799749in}}%
\pgfpathlineto{\pgfqpoint{4.720000in}{4.799309in}}%
\pgfpathlineto{\pgfqpoint{4.968000in}{4.800101in}}%
\pgfpathlineto{\pgfqpoint{5.116800in}{4.799550in}}%
\pgfpathlineto{\pgfqpoint{5.352400in}{4.800094in}}%
\pgfpathlineto{\pgfqpoint{5.526000in}{4.799792in}}%
\pgfpathlineto{\pgfqpoint{5.743000in}{4.800048in}}%
\pgfpathlineto{\pgfqpoint{5.941400in}{4.799983in}}%
\pgfpathlineto{\pgfqpoint{6.152200in}{4.799944in}}%
\pgfpathlineto{\pgfqpoint{6.381600in}{4.800153in}}%
\pgfpathlineto{\pgfqpoint{6.871400in}{4.800088in}}%
\pgfpathlineto{\pgfqpoint{7.200000in}{4.800110in}}%
\pgfpathlineto{\pgfqpoint{7.200000in}{4.800110in}}%
\pgfusepath{stroke}%
\end{pgfscope}%
\begin{pgfscope}%
\pgfpathrectangle{\pgfqpoint{1.000000in}{0.600000in}}{\pgfqpoint{6.200000in}{4.800000in}}%
\pgfusepath{clip}%
\pgfsetrectcap%
\pgfsetroundjoin%
\pgfsetlinewidth{1.003750pt}%
\definecolor{currentstroke}{rgb}{0.000000,0.000000,0.000000}%
\pgfsetstrokecolor{currentstroke}%
\pgfsetdash{}{0pt}%
\pgfpathmoveto{\pgfqpoint{1.000000in}{4.815708in}}%
\pgfpathlineto{\pgfqpoint{1.024800in}{4.814519in}}%
\pgfpathlineto{\pgfqpoint{1.049600in}{4.811117in}}%
\pgfpathlineto{\pgfqpoint{1.080600in}{4.804533in}}%
\pgfpathlineto{\pgfqpoint{1.148800in}{4.788787in}}%
\pgfpathlineto{\pgfqpoint{1.173600in}{4.785427in}}%
\pgfpathlineto{\pgfqpoint{1.198400in}{4.784287in}}%
\pgfpathlineto{\pgfqpoint{1.223200in}{4.785526in}}%
\pgfpathlineto{\pgfqpoint{1.248000in}{4.788972in}}%
\pgfpathlineto{\pgfqpoint{1.279000in}{4.795592in}}%
\pgfpathlineto{\pgfqpoint{1.347200in}{4.811309in}}%
\pgfpathlineto{\pgfqpoint{1.372000in}{4.814627in}}%
\pgfpathlineto{\pgfqpoint{1.396800in}{4.815718in}}%
\pgfpathlineto{\pgfqpoint{1.421600in}{4.814428in}}%
\pgfpathlineto{\pgfqpoint{1.446400in}{4.810938in}}%
\pgfpathlineto{\pgfqpoint{1.477400in}{4.804283in}}%
\pgfpathlineto{\pgfqpoint{1.545600in}{4.788596in}}%
\pgfpathlineto{\pgfqpoint{1.570400in}{4.785320in}}%
\pgfpathlineto{\pgfqpoint{1.595200in}{4.784279in}}%
\pgfpathlineto{\pgfqpoint{1.620000in}{4.785619in}}%
\pgfpathlineto{\pgfqpoint{1.644800in}{4.789152in}}%
\pgfpathlineto{\pgfqpoint{1.675800in}{4.795842in}}%
\pgfpathlineto{\pgfqpoint{1.744000in}{4.811498in}}%
\pgfpathlineto{\pgfqpoint{1.768800in}{4.814732in}}%
\pgfpathlineto{\pgfqpoint{1.793600in}{4.815723in}}%
\pgfpathlineto{\pgfqpoint{1.818400in}{4.814333in}}%
\pgfpathlineto{\pgfqpoint{1.843200in}{4.810757in}}%
\pgfpathlineto{\pgfqpoint{1.874200in}{4.804033in}}%
\pgfpathlineto{\pgfqpoint{1.942400in}{4.788409in}}%
\pgfpathlineto{\pgfqpoint{1.967200in}{4.785218in}}%
\pgfpathlineto{\pgfqpoint{1.992000in}{4.784276in}}%
\pgfpathlineto{\pgfqpoint{2.016800in}{4.785715in}}%
\pgfpathlineto{\pgfqpoint{2.041600in}{4.789334in}}%
\pgfpathlineto{\pgfqpoint{2.072600in}{4.796092in}}%
\pgfpathlineto{\pgfqpoint{2.134600in}{4.810588in}}%
\pgfpathlineto{\pgfqpoint{2.159400in}{4.814243in}}%
\pgfpathlineto{\pgfqpoint{2.184200in}{4.815725in}}%
\pgfpathlineto{\pgfqpoint{2.209000in}{4.814826in}}%
\pgfpathlineto{\pgfqpoint{2.233800in}{4.811670in}}%
\pgfpathlineto{\pgfqpoint{2.264800in}{4.805275in}}%
\pgfpathlineto{\pgfqpoint{2.339200in}{4.788228in}}%
\pgfpathlineto{\pgfqpoint{2.364000in}{4.785120in}}%
\pgfpathlineto{\pgfqpoint{2.388800in}{4.784277in}}%
\pgfpathlineto{\pgfqpoint{2.413600in}{4.785814in}}%
\pgfpathlineto{\pgfqpoint{2.438400in}{4.789517in}}%
\pgfpathlineto{\pgfqpoint{2.475600in}{4.797872in}}%
\pgfpathlineto{\pgfqpoint{2.531400in}{4.810785in}}%
\pgfpathlineto{\pgfqpoint{2.556200in}{4.814361in}}%
\pgfpathlineto{\pgfqpoint{2.581000in}{4.815747in}}%
\pgfpathlineto{\pgfqpoint{2.605800in}{4.814749in}}%
\pgfpathlineto{\pgfqpoint{2.630600in}{4.811505in}}%
\pgfpathlineto{\pgfqpoint{2.661600in}{4.805036in}}%
\pgfpathlineto{\pgfqpoint{2.736000in}{4.788050in}}%
\pgfpathlineto{\pgfqpoint{2.760800in}{4.785028in}}%
\pgfpathlineto{\pgfqpoint{2.785600in}{4.784281in}}%
\pgfpathlineto{\pgfqpoint{2.810400in}{4.785915in}}%
\pgfpathlineto{\pgfqpoint{2.835200in}{4.789700in}}%
\pgfpathlineto{\pgfqpoint{2.872400in}{4.798124in}}%
\pgfpathlineto{\pgfqpoint{2.922000in}{4.809812in}}%
\pgfpathlineto{\pgfqpoint{2.946800in}{4.813793in}}%
\pgfpathlineto{\pgfqpoint{2.971600in}{4.815665in}}%
\pgfpathlineto{\pgfqpoint{2.996400in}{4.815164in}}%
\pgfpathlineto{\pgfqpoint{3.021200in}{4.812360in}}%
\pgfpathlineto{\pgfqpoint{3.052200in}{4.806257in}}%
\pgfpathlineto{\pgfqpoint{3.139000in}{4.786946in}}%
\pgfpathlineto{\pgfqpoint{3.163800in}{4.784553in}}%
\pgfpathlineto{\pgfqpoint{3.188600in}{4.784503in}}%
\pgfpathlineto{\pgfqpoint{3.213400in}{4.786802in}}%
\pgfpathlineto{\pgfqpoint{3.244400in}{4.792454in}}%
\pgfpathlineto{\pgfqpoint{3.343600in}{4.813923in}}%
\pgfpathlineto{\pgfqpoint{3.368400in}{4.815703in}}%
\pgfpathlineto{\pgfqpoint{3.393200in}{4.815105in}}%
\pgfpathlineto{\pgfqpoint{3.418000in}{4.812212in}}%
\pgfpathlineto{\pgfqpoint{3.449000in}{4.806029in}}%
\pgfpathlineto{\pgfqpoint{3.529600in}{4.787710in}}%
\pgfpathlineto{\pgfqpoint{3.554400in}{4.784856in}}%
\pgfpathlineto{\pgfqpoint{3.579200in}{4.784302in}}%
\pgfpathlineto{\pgfqpoint{3.604000in}{4.786126in}}%
\pgfpathlineto{\pgfqpoint{3.628800in}{4.790071in}}%
\pgfpathlineto{\pgfqpoint{3.666000in}{4.798624in}}%
\pgfpathlineto{\pgfqpoint{3.715600in}{4.810215in}}%
\pgfpathlineto{\pgfqpoint{3.740400in}{4.814049in}}%
\pgfpathlineto{\pgfqpoint{3.765200in}{4.815738in}}%
\pgfpathlineto{\pgfqpoint{3.790000in}{4.815043in}}%
\pgfpathlineto{\pgfqpoint{3.814800in}{4.812063in}}%
\pgfpathlineto{\pgfqpoint{3.845800in}{4.805802in}}%
\pgfpathlineto{\pgfqpoint{3.926400in}{4.787547in}}%
\pgfpathlineto{\pgfqpoint{3.951200in}{4.784777in}}%
\pgfpathlineto{\pgfqpoint{3.976000in}{4.784318in}}%
\pgfpathlineto{\pgfqpoint{4.000800in}{4.786235in}}%
\pgfpathlineto{\pgfqpoint{4.025600in}{4.790257in}}%
\pgfpathlineto{\pgfqpoint{4.062800in}{4.798871in}}%
\pgfpathlineto{\pgfqpoint{4.112400in}{4.810410in}}%
\pgfpathlineto{\pgfqpoint{4.137200in}{4.814170in}}%
\pgfpathlineto{\pgfqpoint{4.162000in}{4.815768in}}%
\pgfpathlineto{\pgfqpoint{4.186800in}{4.814978in}}%
\pgfpathlineto{\pgfqpoint{4.211600in}{4.811911in}}%
\pgfpathlineto{\pgfqpoint{4.242600in}{4.805576in}}%
\pgfpathlineto{\pgfqpoint{4.317000in}{4.788384in}}%
\pgfpathlineto{\pgfqpoint{4.341800in}{4.785165in}}%
\pgfpathlineto{\pgfqpoint{4.366600in}{4.784203in}}%
\pgfpathlineto{\pgfqpoint{4.391400in}{4.785632in}}%
\pgfpathlineto{\pgfqpoint{4.416200in}{4.789252in}}%
\pgfpathlineto{\pgfqpoint{4.447200in}{4.796029in}}%
\pgfpathlineto{\pgfqpoint{4.509200in}{4.810601in}}%
\pgfpathlineto{\pgfqpoint{4.534000in}{4.814287in}}%
\pgfpathlineto{\pgfqpoint{4.558800in}{4.815794in}}%
\pgfpathlineto{\pgfqpoint{4.583600in}{4.814910in}}%
\pgfpathlineto{\pgfqpoint{4.608400in}{4.811759in}}%
\pgfpathlineto{\pgfqpoint{4.639400in}{4.805350in}}%
\pgfpathlineto{\pgfqpoint{4.713800in}{4.788212in}}%
\pgfpathlineto{\pgfqpoint{4.738600in}{4.785072in}}%
\pgfpathlineto{\pgfqpoint{4.763400in}{4.784202in}}%
\pgfpathlineto{\pgfqpoint{4.788200in}{4.785723in}}%
\pgfpathlineto{\pgfqpoint{4.813000in}{4.789423in}}%
\pgfpathlineto{\pgfqpoint{4.850200in}{4.797800in}}%
\pgfpathlineto{\pgfqpoint{4.906000in}{4.810787in}}%
\pgfpathlineto{\pgfqpoint{4.930800in}{4.814400in}}%
\pgfpathlineto{\pgfqpoint{4.955600in}{4.815817in}}%
\pgfpathlineto{\pgfqpoint{4.980400in}{4.814840in}}%
\pgfpathlineto{\pgfqpoint{5.005200in}{4.811605in}}%
\pgfpathlineto{\pgfqpoint{5.036200in}{4.805126in}}%
\pgfpathlineto{\pgfqpoint{5.110600in}{4.788043in}}%
\pgfpathlineto{\pgfqpoint{5.135400in}{4.784983in}}%
\pgfpathlineto{\pgfqpoint{5.160200in}{4.784204in}}%
\pgfpathlineto{\pgfqpoint{5.185000in}{4.785817in}}%
\pgfpathlineto{\pgfqpoint{5.209800in}{4.789595in}}%
\pgfpathlineto{\pgfqpoint{5.247000in}{4.798037in}}%
\pgfpathlineto{\pgfqpoint{5.302800in}{4.810970in}}%
\pgfpathlineto{\pgfqpoint{5.327600in}{4.814508in}}%
\pgfpathlineto{\pgfqpoint{5.352400in}{4.815836in}}%
\pgfpathlineto{\pgfqpoint{5.377200in}{4.814766in}}%
\pgfpathlineto{\pgfqpoint{5.402000in}{4.811450in}}%
\pgfpathlineto{\pgfqpoint{5.433000in}{4.804902in}}%
\pgfpathlineto{\pgfqpoint{5.507400in}{4.787880in}}%
\pgfpathlineto{\pgfqpoint{5.532200in}{4.784898in}}%
\pgfpathlineto{\pgfqpoint{5.557000in}{4.784211in}}%
\pgfpathlineto{\pgfqpoint{5.581800in}{4.785914in}}%
\pgfpathlineto{\pgfqpoint{5.606600in}{4.789768in}}%
\pgfpathlineto{\pgfqpoint{5.643800in}{4.798273in}}%
\pgfpathlineto{\pgfqpoint{5.693400in}{4.809986in}}%
\pgfpathlineto{\pgfqpoint{5.718200in}{4.813941in}}%
\pgfpathlineto{\pgfqpoint{5.743000in}{4.815765in}}%
\pgfpathlineto{\pgfqpoint{5.767800in}{4.815203in}}%
\pgfpathlineto{\pgfqpoint{5.792600in}{4.812331in}}%
\pgfpathlineto{\pgfqpoint{5.823600in}{4.806151in}}%
\pgfpathlineto{\pgfqpoint{5.904200in}{4.787720in}}%
\pgfpathlineto{\pgfqpoint{5.929000in}{4.784817in}}%
\pgfpathlineto{\pgfqpoint{5.953800in}{4.784220in}}%
\pgfpathlineto{\pgfqpoint{5.978600in}{4.786012in}}%
\pgfpathlineto{\pgfqpoint{6.003400in}{4.789941in}}%
\pgfpathlineto{\pgfqpoint{6.040600in}{4.798507in}}%
\pgfpathlineto{\pgfqpoint{6.090200in}{4.810175in}}%
\pgfpathlineto{\pgfqpoint{6.115000in}{4.814061in}}%
\pgfpathlineto{\pgfqpoint{6.139800in}{4.815799in}}%
\pgfpathlineto{\pgfqpoint{6.164600in}{4.815146in}}%
\pgfpathlineto{\pgfqpoint{6.189400in}{4.812193in}}%
\pgfpathlineto{\pgfqpoint{6.220400in}{4.805939in}}%
\pgfpathlineto{\pgfqpoint{6.301000in}{4.787565in}}%
\pgfpathlineto{\pgfqpoint{6.325800in}{4.784740in}}%
\pgfpathlineto{\pgfqpoint{6.350600in}{4.784233in}}%
\pgfpathlineto{\pgfqpoint{6.375400in}{4.786112in}}%
\pgfpathlineto{\pgfqpoint{6.400200in}{4.790114in}}%
\pgfpathlineto{\pgfqpoint{6.437400in}{4.798738in}}%
\pgfpathlineto{\pgfqpoint{6.487000in}{4.810360in}}%
\pgfpathlineto{\pgfqpoint{6.511800in}{4.814177in}}%
\pgfpathlineto{\pgfqpoint{6.536600in}{4.815830in}}%
\pgfpathlineto{\pgfqpoint{6.561400in}{4.815088in}}%
\pgfpathlineto{\pgfqpoint{6.586200in}{4.812053in}}%
\pgfpathlineto{\pgfqpoint{6.617200in}{4.805728in}}%
\pgfpathlineto{\pgfqpoint{6.697800in}{4.787413in}}%
\pgfpathlineto{\pgfqpoint{6.722600in}{4.784667in}}%
\pgfpathlineto{\pgfqpoint{6.747400in}{4.784248in}}%
\pgfpathlineto{\pgfqpoint{6.772200in}{4.786214in}}%
\pgfpathlineto{\pgfqpoint{6.797000in}{4.790288in}}%
\pgfpathlineto{\pgfqpoint{6.834200in}{4.798968in}}%
\pgfpathlineto{\pgfqpoint{6.883800in}{4.810541in}}%
\pgfpathlineto{\pgfqpoint{6.908600in}{4.814289in}}%
\pgfpathlineto{\pgfqpoint{6.933400in}{4.815858in}}%
\pgfpathlineto{\pgfqpoint{6.958200in}{4.815027in}}%
\pgfpathlineto{\pgfqpoint{6.983000in}{4.811912in}}%
\pgfpathlineto{\pgfqpoint{7.014000in}{4.805517in}}%
\pgfpathlineto{\pgfqpoint{7.088400in}{4.788261in}}%
\pgfpathlineto{\pgfqpoint{7.113200in}{4.785055in}}%
\pgfpathlineto{\pgfqpoint{7.138000in}{4.784122in}}%
\pgfpathlineto{\pgfqpoint{7.162800in}{4.785592in}}%
\pgfpathlineto{\pgfqpoint{7.187600in}{4.789258in}}%
\pgfpathlineto{\pgfqpoint{7.200000in}{4.791760in}}%
\pgfpathlineto{\pgfqpoint{7.200000in}{4.791760in}}%
\pgfusepath{stroke}%
\end{pgfscope}%
\begin{pgfscope}%
\pgfsetrectcap%
\pgfsetmiterjoin%
\pgfsetlinewidth{1.003750pt}%
\definecolor{currentstroke}{rgb}{0.000000,0.000000,0.000000}%
\pgfsetstrokecolor{currentstroke}%
\pgfsetdash{}{0pt}%
\pgfpathmoveto{\pgfqpoint{1.000000in}{0.600000in}}%
\pgfpathlineto{\pgfqpoint{1.000000in}{5.400000in}}%
\pgfusepath{stroke}%
\end{pgfscope}%
\begin{pgfscope}%
\pgfsetrectcap%
\pgfsetmiterjoin%
\pgfsetlinewidth{1.003750pt}%
\definecolor{currentstroke}{rgb}{0.000000,0.000000,0.000000}%
\pgfsetstrokecolor{currentstroke}%
\pgfsetdash{}{0pt}%
\pgfpathmoveto{\pgfqpoint{7.200000in}{0.600000in}}%
\pgfpathlineto{\pgfqpoint{7.200000in}{5.400000in}}%
\pgfusepath{stroke}%
\end{pgfscope}%
\begin{pgfscope}%
\pgfsetrectcap%
\pgfsetmiterjoin%
\pgfsetlinewidth{1.003750pt}%
\definecolor{currentstroke}{rgb}{0.000000,0.000000,0.000000}%
\pgfsetstrokecolor{currentstroke}%
\pgfsetdash{}{0pt}%
\pgfpathmoveto{\pgfqpoint{1.000000in}{0.600000in}}%
\pgfpathlineto{\pgfqpoint{7.200000in}{0.600000in}}%
\pgfusepath{stroke}%
\end{pgfscope}%
\begin{pgfscope}%
\pgfsetrectcap%
\pgfsetmiterjoin%
\pgfsetlinewidth{1.003750pt}%
\definecolor{currentstroke}{rgb}{0.000000,0.000000,0.000000}%
\pgfsetstrokecolor{currentstroke}%
\pgfsetdash{}{0pt}%
\pgfpathmoveto{\pgfqpoint{1.000000in}{5.400000in}}%
\pgfpathlineto{\pgfqpoint{7.200000in}{5.400000in}}%
\pgfusepath{stroke}%
\end{pgfscope}%
\begin{pgfscope}%
\pgfpathrectangle{\pgfqpoint{1.000000in}{0.600000in}}{\pgfqpoint{6.200000in}{4.800000in}}%
\pgfusepath{clip}%
\pgfsetbuttcap%
\pgfsetroundjoin%
\pgfsetlinewidth{0.501875pt}%
\definecolor{currentstroke}{rgb}{0.000000,0.000000,0.000000}%
\pgfsetstrokecolor{currentstroke}%
\pgfsetdash{{1.000000pt}{3.000000pt}}{0.000000pt}%
\pgfpathmoveto{\pgfqpoint{1.000000in}{0.600000in}}%
\pgfpathlineto{\pgfqpoint{1.000000in}{5.400000in}}%
\pgfusepath{stroke}%
\end{pgfscope}%
\begin{pgfscope}%
\pgfsetbuttcap%
\pgfsetroundjoin%
\definecolor{currentfill}{rgb}{0.000000,0.000000,0.000000}%
\pgfsetfillcolor{currentfill}%
\pgfsetlinewidth{0.501875pt}%
\definecolor{currentstroke}{rgb}{0.000000,0.000000,0.000000}%
\pgfsetstrokecolor{currentstroke}%
\pgfsetdash{}{0pt}%
\pgfsys@defobject{currentmarker}{\pgfqpoint{0.000000in}{0.000000in}}{\pgfqpoint{0.000000in}{0.055556in}}{%
\pgfpathmoveto{\pgfqpoint{0.000000in}{0.000000in}}%
\pgfpathlineto{\pgfqpoint{0.000000in}{0.055556in}}%
\pgfusepath{stroke,fill}%
}%
\begin{pgfscope}%
\pgfsys@transformshift{1.000000in}{0.600000in}%
\pgfsys@useobject{currentmarker}{}%
\end{pgfscope}%
\end{pgfscope}%
\begin{pgfscope}%
\pgfsetbuttcap%
\pgfsetroundjoin%
\definecolor{currentfill}{rgb}{0.000000,0.000000,0.000000}%
\pgfsetfillcolor{currentfill}%
\pgfsetlinewidth{0.501875pt}%
\definecolor{currentstroke}{rgb}{0.000000,0.000000,0.000000}%
\pgfsetstrokecolor{currentstroke}%
\pgfsetdash{}{0pt}%
\pgfsys@defobject{currentmarker}{\pgfqpoint{0.000000in}{-0.055556in}}{\pgfqpoint{0.000000in}{0.000000in}}{%
\pgfpathmoveto{\pgfqpoint{0.000000in}{0.000000in}}%
\pgfpathlineto{\pgfqpoint{0.000000in}{-0.055556in}}%
\pgfusepath{stroke,fill}%
}%
\begin{pgfscope}%
\pgfsys@transformshift{1.000000in}{5.400000in}%
\pgfsys@useobject{currentmarker}{}%
\end{pgfscope}%
\end{pgfscope}%
\begin{pgfscope}%
\definecolor{textcolor}{rgb}{0.000000,0.000000,0.000000}%
\pgfsetstrokecolor{textcolor}%
\pgfsetfillcolor{textcolor}%
\pgftext[x=1.000000in,y=0.544444in,,top]{\color{textcolor}\rmfamily\fontsize{10.000000}{12.000000}\selectfont \(\displaystyle {0}\)}%
\end{pgfscope}%
\begin{pgfscope}%
\pgfpathrectangle{\pgfqpoint{1.000000in}{0.600000in}}{\pgfqpoint{6.200000in}{4.800000in}}%
\pgfusepath{clip}%
\pgfsetbuttcap%
\pgfsetroundjoin%
\pgfsetlinewidth{0.501875pt}%
\definecolor{currentstroke}{rgb}{0.000000,0.000000,0.000000}%
\pgfsetstrokecolor{currentstroke}%
\pgfsetdash{{1.000000pt}{3.000000pt}}{0.000000pt}%
\pgfpathmoveto{\pgfqpoint{2.240000in}{0.600000in}}%
\pgfpathlineto{\pgfqpoint{2.240000in}{5.400000in}}%
\pgfusepath{stroke}%
\end{pgfscope}%
\begin{pgfscope}%
\pgfsetbuttcap%
\pgfsetroundjoin%
\definecolor{currentfill}{rgb}{0.000000,0.000000,0.000000}%
\pgfsetfillcolor{currentfill}%
\pgfsetlinewidth{0.501875pt}%
\definecolor{currentstroke}{rgb}{0.000000,0.000000,0.000000}%
\pgfsetstrokecolor{currentstroke}%
\pgfsetdash{}{0pt}%
\pgfsys@defobject{currentmarker}{\pgfqpoint{0.000000in}{0.000000in}}{\pgfqpoint{0.000000in}{0.055556in}}{%
\pgfpathmoveto{\pgfqpoint{0.000000in}{0.000000in}}%
\pgfpathlineto{\pgfqpoint{0.000000in}{0.055556in}}%
\pgfusepath{stroke,fill}%
}%
\begin{pgfscope}%
\pgfsys@transformshift{2.240000in}{0.600000in}%
\pgfsys@useobject{currentmarker}{}%
\end{pgfscope}%
\end{pgfscope}%
\begin{pgfscope}%
\pgfsetbuttcap%
\pgfsetroundjoin%
\definecolor{currentfill}{rgb}{0.000000,0.000000,0.000000}%
\pgfsetfillcolor{currentfill}%
\pgfsetlinewidth{0.501875pt}%
\definecolor{currentstroke}{rgb}{0.000000,0.000000,0.000000}%
\pgfsetstrokecolor{currentstroke}%
\pgfsetdash{}{0pt}%
\pgfsys@defobject{currentmarker}{\pgfqpoint{0.000000in}{-0.055556in}}{\pgfqpoint{0.000000in}{0.000000in}}{%
\pgfpathmoveto{\pgfqpoint{0.000000in}{0.000000in}}%
\pgfpathlineto{\pgfqpoint{0.000000in}{-0.055556in}}%
\pgfusepath{stroke,fill}%
}%
\begin{pgfscope}%
\pgfsys@transformshift{2.240000in}{5.400000in}%
\pgfsys@useobject{currentmarker}{}%
\end{pgfscope}%
\end{pgfscope}%
\begin{pgfscope}%
\definecolor{textcolor}{rgb}{0.000000,0.000000,0.000000}%
\pgfsetstrokecolor{textcolor}%
\pgfsetfillcolor{textcolor}%
\pgftext[x=2.240000in,y=0.544444in,,top]{\color{textcolor}\rmfamily\fontsize{10.000000}{12.000000}\selectfont \(\displaystyle {20}\)}%
\end{pgfscope}%
\begin{pgfscope}%
\pgfpathrectangle{\pgfqpoint{1.000000in}{0.600000in}}{\pgfqpoint{6.200000in}{4.800000in}}%
\pgfusepath{clip}%
\pgfsetbuttcap%
\pgfsetroundjoin%
\pgfsetlinewidth{0.501875pt}%
\definecolor{currentstroke}{rgb}{0.000000,0.000000,0.000000}%
\pgfsetstrokecolor{currentstroke}%
\pgfsetdash{{1.000000pt}{3.000000pt}}{0.000000pt}%
\pgfpathmoveto{\pgfqpoint{3.480000in}{0.600000in}}%
\pgfpathlineto{\pgfqpoint{3.480000in}{5.400000in}}%
\pgfusepath{stroke}%
\end{pgfscope}%
\begin{pgfscope}%
\pgfsetbuttcap%
\pgfsetroundjoin%
\definecolor{currentfill}{rgb}{0.000000,0.000000,0.000000}%
\pgfsetfillcolor{currentfill}%
\pgfsetlinewidth{0.501875pt}%
\definecolor{currentstroke}{rgb}{0.000000,0.000000,0.000000}%
\pgfsetstrokecolor{currentstroke}%
\pgfsetdash{}{0pt}%
\pgfsys@defobject{currentmarker}{\pgfqpoint{0.000000in}{0.000000in}}{\pgfqpoint{0.000000in}{0.055556in}}{%
\pgfpathmoveto{\pgfqpoint{0.000000in}{0.000000in}}%
\pgfpathlineto{\pgfqpoint{0.000000in}{0.055556in}}%
\pgfusepath{stroke,fill}%
}%
\begin{pgfscope}%
\pgfsys@transformshift{3.480000in}{0.600000in}%
\pgfsys@useobject{currentmarker}{}%
\end{pgfscope}%
\end{pgfscope}%
\begin{pgfscope}%
\pgfsetbuttcap%
\pgfsetroundjoin%
\definecolor{currentfill}{rgb}{0.000000,0.000000,0.000000}%
\pgfsetfillcolor{currentfill}%
\pgfsetlinewidth{0.501875pt}%
\definecolor{currentstroke}{rgb}{0.000000,0.000000,0.000000}%
\pgfsetstrokecolor{currentstroke}%
\pgfsetdash{}{0pt}%
\pgfsys@defobject{currentmarker}{\pgfqpoint{0.000000in}{-0.055556in}}{\pgfqpoint{0.000000in}{0.000000in}}{%
\pgfpathmoveto{\pgfqpoint{0.000000in}{0.000000in}}%
\pgfpathlineto{\pgfqpoint{0.000000in}{-0.055556in}}%
\pgfusepath{stroke,fill}%
}%
\begin{pgfscope}%
\pgfsys@transformshift{3.480000in}{5.400000in}%
\pgfsys@useobject{currentmarker}{}%
\end{pgfscope}%
\end{pgfscope}%
\begin{pgfscope}%
\definecolor{textcolor}{rgb}{0.000000,0.000000,0.000000}%
\pgfsetstrokecolor{textcolor}%
\pgfsetfillcolor{textcolor}%
\pgftext[x=3.480000in,y=0.544444in,,top]{\color{textcolor}\rmfamily\fontsize{10.000000}{12.000000}\selectfont \(\displaystyle {40}\)}%
\end{pgfscope}%
\begin{pgfscope}%
\pgfpathrectangle{\pgfqpoint{1.000000in}{0.600000in}}{\pgfqpoint{6.200000in}{4.800000in}}%
\pgfusepath{clip}%
\pgfsetbuttcap%
\pgfsetroundjoin%
\pgfsetlinewidth{0.501875pt}%
\definecolor{currentstroke}{rgb}{0.000000,0.000000,0.000000}%
\pgfsetstrokecolor{currentstroke}%
\pgfsetdash{{1.000000pt}{3.000000pt}}{0.000000pt}%
\pgfpathmoveto{\pgfqpoint{4.720000in}{0.600000in}}%
\pgfpathlineto{\pgfqpoint{4.720000in}{5.400000in}}%
\pgfusepath{stroke}%
\end{pgfscope}%
\begin{pgfscope}%
\pgfsetbuttcap%
\pgfsetroundjoin%
\definecolor{currentfill}{rgb}{0.000000,0.000000,0.000000}%
\pgfsetfillcolor{currentfill}%
\pgfsetlinewidth{0.501875pt}%
\definecolor{currentstroke}{rgb}{0.000000,0.000000,0.000000}%
\pgfsetstrokecolor{currentstroke}%
\pgfsetdash{}{0pt}%
\pgfsys@defobject{currentmarker}{\pgfqpoint{0.000000in}{0.000000in}}{\pgfqpoint{0.000000in}{0.055556in}}{%
\pgfpathmoveto{\pgfqpoint{0.000000in}{0.000000in}}%
\pgfpathlineto{\pgfqpoint{0.000000in}{0.055556in}}%
\pgfusepath{stroke,fill}%
}%
\begin{pgfscope}%
\pgfsys@transformshift{4.720000in}{0.600000in}%
\pgfsys@useobject{currentmarker}{}%
\end{pgfscope}%
\end{pgfscope}%
\begin{pgfscope}%
\pgfsetbuttcap%
\pgfsetroundjoin%
\definecolor{currentfill}{rgb}{0.000000,0.000000,0.000000}%
\pgfsetfillcolor{currentfill}%
\pgfsetlinewidth{0.501875pt}%
\definecolor{currentstroke}{rgb}{0.000000,0.000000,0.000000}%
\pgfsetstrokecolor{currentstroke}%
\pgfsetdash{}{0pt}%
\pgfsys@defobject{currentmarker}{\pgfqpoint{0.000000in}{-0.055556in}}{\pgfqpoint{0.000000in}{0.000000in}}{%
\pgfpathmoveto{\pgfqpoint{0.000000in}{0.000000in}}%
\pgfpathlineto{\pgfqpoint{0.000000in}{-0.055556in}}%
\pgfusepath{stroke,fill}%
}%
\begin{pgfscope}%
\pgfsys@transformshift{4.720000in}{5.400000in}%
\pgfsys@useobject{currentmarker}{}%
\end{pgfscope}%
\end{pgfscope}%
\begin{pgfscope}%
\definecolor{textcolor}{rgb}{0.000000,0.000000,0.000000}%
\pgfsetstrokecolor{textcolor}%
\pgfsetfillcolor{textcolor}%
\pgftext[x=4.720000in,y=0.544444in,,top]{\color{textcolor}\rmfamily\fontsize{10.000000}{12.000000}\selectfont \(\displaystyle {60}\)}%
\end{pgfscope}%
\begin{pgfscope}%
\pgfpathrectangle{\pgfqpoint{1.000000in}{0.600000in}}{\pgfqpoint{6.200000in}{4.800000in}}%
\pgfusepath{clip}%
\pgfsetbuttcap%
\pgfsetroundjoin%
\pgfsetlinewidth{0.501875pt}%
\definecolor{currentstroke}{rgb}{0.000000,0.000000,0.000000}%
\pgfsetstrokecolor{currentstroke}%
\pgfsetdash{{1.000000pt}{3.000000pt}}{0.000000pt}%
\pgfpathmoveto{\pgfqpoint{5.960000in}{0.600000in}}%
\pgfpathlineto{\pgfqpoint{5.960000in}{5.400000in}}%
\pgfusepath{stroke}%
\end{pgfscope}%
\begin{pgfscope}%
\pgfsetbuttcap%
\pgfsetroundjoin%
\definecolor{currentfill}{rgb}{0.000000,0.000000,0.000000}%
\pgfsetfillcolor{currentfill}%
\pgfsetlinewidth{0.501875pt}%
\definecolor{currentstroke}{rgb}{0.000000,0.000000,0.000000}%
\pgfsetstrokecolor{currentstroke}%
\pgfsetdash{}{0pt}%
\pgfsys@defobject{currentmarker}{\pgfqpoint{0.000000in}{0.000000in}}{\pgfqpoint{0.000000in}{0.055556in}}{%
\pgfpathmoveto{\pgfqpoint{0.000000in}{0.000000in}}%
\pgfpathlineto{\pgfqpoint{0.000000in}{0.055556in}}%
\pgfusepath{stroke,fill}%
}%
\begin{pgfscope}%
\pgfsys@transformshift{5.960000in}{0.600000in}%
\pgfsys@useobject{currentmarker}{}%
\end{pgfscope}%
\end{pgfscope}%
\begin{pgfscope}%
\pgfsetbuttcap%
\pgfsetroundjoin%
\definecolor{currentfill}{rgb}{0.000000,0.000000,0.000000}%
\pgfsetfillcolor{currentfill}%
\pgfsetlinewidth{0.501875pt}%
\definecolor{currentstroke}{rgb}{0.000000,0.000000,0.000000}%
\pgfsetstrokecolor{currentstroke}%
\pgfsetdash{}{0pt}%
\pgfsys@defobject{currentmarker}{\pgfqpoint{0.000000in}{-0.055556in}}{\pgfqpoint{0.000000in}{0.000000in}}{%
\pgfpathmoveto{\pgfqpoint{0.000000in}{0.000000in}}%
\pgfpathlineto{\pgfqpoint{0.000000in}{-0.055556in}}%
\pgfusepath{stroke,fill}%
}%
\begin{pgfscope}%
\pgfsys@transformshift{5.960000in}{5.400000in}%
\pgfsys@useobject{currentmarker}{}%
\end{pgfscope}%
\end{pgfscope}%
\begin{pgfscope}%
\definecolor{textcolor}{rgb}{0.000000,0.000000,0.000000}%
\pgfsetstrokecolor{textcolor}%
\pgfsetfillcolor{textcolor}%
\pgftext[x=5.960000in,y=0.544444in,,top]{\color{textcolor}\rmfamily\fontsize{10.000000}{12.000000}\selectfont \(\displaystyle {80}\)}%
\end{pgfscope}%
\begin{pgfscope}%
\pgfpathrectangle{\pgfqpoint{1.000000in}{0.600000in}}{\pgfqpoint{6.200000in}{4.800000in}}%
\pgfusepath{clip}%
\pgfsetbuttcap%
\pgfsetroundjoin%
\pgfsetlinewidth{0.501875pt}%
\definecolor{currentstroke}{rgb}{0.000000,0.000000,0.000000}%
\pgfsetstrokecolor{currentstroke}%
\pgfsetdash{{1.000000pt}{3.000000pt}}{0.000000pt}%
\pgfpathmoveto{\pgfqpoint{7.200000in}{0.600000in}}%
\pgfpathlineto{\pgfqpoint{7.200000in}{5.400000in}}%
\pgfusepath{stroke}%
\end{pgfscope}%
\begin{pgfscope}%
\pgfsetbuttcap%
\pgfsetroundjoin%
\definecolor{currentfill}{rgb}{0.000000,0.000000,0.000000}%
\pgfsetfillcolor{currentfill}%
\pgfsetlinewidth{0.501875pt}%
\definecolor{currentstroke}{rgb}{0.000000,0.000000,0.000000}%
\pgfsetstrokecolor{currentstroke}%
\pgfsetdash{}{0pt}%
\pgfsys@defobject{currentmarker}{\pgfqpoint{0.000000in}{0.000000in}}{\pgfqpoint{0.000000in}{0.055556in}}{%
\pgfpathmoveto{\pgfqpoint{0.000000in}{0.000000in}}%
\pgfpathlineto{\pgfqpoint{0.000000in}{0.055556in}}%
\pgfusepath{stroke,fill}%
}%
\begin{pgfscope}%
\pgfsys@transformshift{7.200000in}{0.600000in}%
\pgfsys@useobject{currentmarker}{}%
\end{pgfscope}%
\end{pgfscope}%
\begin{pgfscope}%
\pgfsetbuttcap%
\pgfsetroundjoin%
\definecolor{currentfill}{rgb}{0.000000,0.000000,0.000000}%
\pgfsetfillcolor{currentfill}%
\pgfsetlinewidth{0.501875pt}%
\definecolor{currentstroke}{rgb}{0.000000,0.000000,0.000000}%
\pgfsetstrokecolor{currentstroke}%
\pgfsetdash{}{0pt}%
\pgfsys@defobject{currentmarker}{\pgfqpoint{0.000000in}{-0.055556in}}{\pgfqpoint{0.000000in}{0.000000in}}{%
\pgfpathmoveto{\pgfqpoint{0.000000in}{0.000000in}}%
\pgfpathlineto{\pgfqpoint{0.000000in}{-0.055556in}}%
\pgfusepath{stroke,fill}%
}%
\begin{pgfscope}%
\pgfsys@transformshift{7.200000in}{5.400000in}%
\pgfsys@useobject{currentmarker}{}%
\end{pgfscope}%
\end{pgfscope}%
\begin{pgfscope}%
\definecolor{textcolor}{rgb}{0.000000,0.000000,0.000000}%
\pgfsetstrokecolor{textcolor}%
\pgfsetfillcolor{textcolor}%
\pgftext[x=7.200000in,y=0.544444in,,top]{\color{textcolor}\rmfamily\fontsize{10.000000}{12.000000}\selectfont \(\displaystyle {100}\)}%
\end{pgfscope}%
\begin{pgfscope}%
\definecolor{textcolor}{rgb}{0.000000,0.000000,0.000000}%
\pgfsetstrokecolor{textcolor}%
\pgfsetfillcolor{textcolor}%
\pgftext[x=4.100000in,y=0.351543in,,top]{\color{textcolor}\rmfamily\fontsize{12.000000}{14.400000}\selectfont \(\displaystyle time\ (s)\)}%
\end{pgfscope}%
\begin{pgfscope}%
\pgfpathrectangle{\pgfqpoint{1.000000in}{0.600000in}}{\pgfqpoint{6.200000in}{4.800000in}}%
\pgfusepath{clip}%
\pgfsetbuttcap%
\pgfsetroundjoin%
\pgfsetlinewidth{0.501875pt}%
\definecolor{currentstroke}{rgb}{0.000000,0.000000,0.000000}%
\pgfsetstrokecolor{currentstroke}%
\pgfsetdash{{1.000000pt}{3.000000pt}}{0.000000pt}%
\pgfpathmoveto{\pgfqpoint{1.000000in}{0.600000in}}%
\pgfpathlineto{\pgfqpoint{7.200000in}{0.600000in}}%
\pgfusepath{stroke}%
\end{pgfscope}%
\begin{pgfscope}%
\pgfsetbuttcap%
\pgfsetroundjoin%
\definecolor{currentfill}{rgb}{0.000000,0.000000,0.000000}%
\pgfsetfillcolor{currentfill}%
\pgfsetlinewidth{0.501875pt}%
\definecolor{currentstroke}{rgb}{0.000000,0.000000,0.000000}%
\pgfsetstrokecolor{currentstroke}%
\pgfsetdash{}{0pt}%
\pgfsys@defobject{currentmarker}{\pgfqpoint{0.000000in}{0.000000in}}{\pgfqpoint{0.055556in}{0.000000in}}{%
\pgfpathmoveto{\pgfqpoint{0.000000in}{0.000000in}}%
\pgfpathlineto{\pgfqpoint{0.055556in}{0.000000in}}%
\pgfusepath{stroke,fill}%
}%
\begin{pgfscope}%
\pgfsys@transformshift{1.000000in}{0.600000in}%
\pgfsys@useobject{currentmarker}{}%
\end{pgfscope}%
\end{pgfscope}%
\begin{pgfscope}%
\pgfsetbuttcap%
\pgfsetroundjoin%
\definecolor{currentfill}{rgb}{0.000000,0.000000,0.000000}%
\pgfsetfillcolor{currentfill}%
\pgfsetlinewidth{0.501875pt}%
\definecolor{currentstroke}{rgb}{0.000000,0.000000,0.000000}%
\pgfsetstrokecolor{currentstroke}%
\pgfsetdash{}{0pt}%
\pgfsys@defobject{currentmarker}{\pgfqpoint{-0.055556in}{0.000000in}}{\pgfqpoint{-0.000000in}{0.000000in}}{%
\pgfpathmoveto{\pgfqpoint{-0.000000in}{0.000000in}}%
\pgfpathlineto{\pgfqpoint{-0.055556in}{0.000000in}}%
\pgfusepath{stroke,fill}%
}%
\begin{pgfscope}%
\pgfsys@transformshift{7.200000in}{0.600000in}%
\pgfsys@useobject{currentmarker}{}%
\end{pgfscope}%
\end{pgfscope}%
\begin{pgfscope}%
\definecolor{textcolor}{rgb}{0.000000,0.000000,0.000000}%
\pgfsetstrokecolor{textcolor}%
\pgfsetfillcolor{textcolor}%
\pgftext[x=0.944444in,y=0.600000in,right,]{\color{textcolor}\rmfamily\fontsize{10.000000}{12.000000}\selectfont \(\displaystyle {\ensuremath{-}140}\)}%
\end{pgfscope}%
\begin{pgfscope}%
\pgfpathrectangle{\pgfqpoint{1.000000in}{0.600000in}}{\pgfqpoint{6.200000in}{4.800000in}}%
\pgfusepath{clip}%
\pgfsetbuttcap%
\pgfsetroundjoin%
\pgfsetlinewidth{0.501875pt}%
\definecolor{currentstroke}{rgb}{0.000000,0.000000,0.000000}%
\pgfsetstrokecolor{currentstroke}%
\pgfsetdash{{1.000000pt}{3.000000pt}}{0.000000pt}%
\pgfpathmoveto{\pgfqpoint{1.000000in}{1.200000in}}%
\pgfpathlineto{\pgfqpoint{7.200000in}{1.200000in}}%
\pgfusepath{stroke}%
\end{pgfscope}%
\begin{pgfscope}%
\pgfsetbuttcap%
\pgfsetroundjoin%
\definecolor{currentfill}{rgb}{0.000000,0.000000,0.000000}%
\pgfsetfillcolor{currentfill}%
\pgfsetlinewidth{0.501875pt}%
\definecolor{currentstroke}{rgb}{0.000000,0.000000,0.000000}%
\pgfsetstrokecolor{currentstroke}%
\pgfsetdash{}{0pt}%
\pgfsys@defobject{currentmarker}{\pgfqpoint{0.000000in}{0.000000in}}{\pgfqpoint{0.055556in}{0.000000in}}{%
\pgfpathmoveto{\pgfqpoint{0.000000in}{0.000000in}}%
\pgfpathlineto{\pgfqpoint{0.055556in}{0.000000in}}%
\pgfusepath{stroke,fill}%
}%
\begin{pgfscope}%
\pgfsys@transformshift{1.000000in}{1.200000in}%
\pgfsys@useobject{currentmarker}{}%
\end{pgfscope}%
\end{pgfscope}%
\begin{pgfscope}%
\pgfsetbuttcap%
\pgfsetroundjoin%
\definecolor{currentfill}{rgb}{0.000000,0.000000,0.000000}%
\pgfsetfillcolor{currentfill}%
\pgfsetlinewidth{0.501875pt}%
\definecolor{currentstroke}{rgb}{0.000000,0.000000,0.000000}%
\pgfsetstrokecolor{currentstroke}%
\pgfsetdash{}{0pt}%
\pgfsys@defobject{currentmarker}{\pgfqpoint{-0.055556in}{0.000000in}}{\pgfqpoint{-0.000000in}{0.000000in}}{%
\pgfpathmoveto{\pgfqpoint{-0.000000in}{0.000000in}}%
\pgfpathlineto{\pgfqpoint{-0.055556in}{0.000000in}}%
\pgfusepath{stroke,fill}%
}%
\begin{pgfscope}%
\pgfsys@transformshift{7.200000in}{1.200000in}%
\pgfsys@useobject{currentmarker}{}%
\end{pgfscope}%
\end{pgfscope}%
\begin{pgfscope}%
\definecolor{textcolor}{rgb}{0.000000,0.000000,0.000000}%
\pgfsetstrokecolor{textcolor}%
\pgfsetfillcolor{textcolor}%
\pgftext[x=0.944444in,y=1.200000in,right,]{\color{textcolor}\rmfamily\fontsize{10.000000}{12.000000}\selectfont \(\displaystyle {\ensuremath{-}120}\)}%
\end{pgfscope}%
\begin{pgfscope}%
\pgfpathrectangle{\pgfqpoint{1.000000in}{0.600000in}}{\pgfqpoint{6.200000in}{4.800000in}}%
\pgfusepath{clip}%
\pgfsetbuttcap%
\pgfsetroundjoin%
\pgfsetlinewidth{0.501875pt}%
\definecolor{currentstroke}{rgb}{0.000000,0.000000,0.000000}%
\pgfsetstrokecolor{currentstroke}%
\pgfsetdash{{1.000000pt}{3.000000pt}}{0.000000pt}%
\pgfpathmoveto{\pgfqpoint{1.000000in}{1.800000in}}%
\pgfpathlineto{\pgfqpoint{7.200000in}{1.800000in}}%
\pgfusepath{stroke}%
\end{pgfscope}%
\begin{pgfscope}%
\pgfsetbuttcap%
\pgfsetroundjoin%
\definecolor{currentfill}{rgb}{0.000000,0.000000,0.000000}%
\pgfsetfillcolor{currentfill}%
\pgfsetlinewidth{0.501875pt}%
\definecolor{currentstroke}{rgb}{0.000000,0.000000,0.000000}%
\pgfsetstrokecolor{currentstroke}%
\pgfsetdash{}{0pt}%
\pgfsys@defobject{currentmarker}{\pgfqpoint{0.000000in}{0.000000in}}{\pgfqpoint{0.055556in}{0.000000in}}{%
\pgfpathmoveto{\pgfqpoint{0.000000in}{0.000000in}}%
\pgfpathlineto{\pgfqpoint{0.055556in}{0.000000in}}%
\pgfusepath{stroke,fill}%
}%
\begin{pgfscope}%
\pgfsys@transformshift{1.000000in}{1.800000in}%
\pgfsys@useobject{currentmarker}{}%
\end{pgfscope}%
\end{pgfscope}%
\begin{pgfscope}%
\pgfsetbuttcap%
\pgfsetroundjoin%
\definecolor{currentfill}{rgb}{0.000000,0.000000,0.000000}%
\pgfsetfillcolor{currentfill}%
\pgfsetlinewidth{0.501875pt}%
\definecolor{currentstroke}{rgb}{0.000000,0.000000,0.000000}%
\pgfsetstrokecolor{currentstroke}%
\pgfsetdash{}{0pt}%
\pgfsys@defobject{currentmarker}{\pgfqpoint{-0.055556in}{0.000000in}}{\pgfqpoint{-0.000000in}{0.000000in}}{%
\pgfpathmoveto{\pgfqpoint{-0.000000in}{0.000000in}}%
\pgfpathlineto{\pgfqpoint{-0.055556in}{0.000000in}}%
\pgfusepath{stroke,fill}%
}%
\begin{pgfscope}%
\pgfsys@transformshift{7.200000in}{1.800000in}%
\pgfsys@useobject{currentmarker}{}%
\end{pgfscope}%
\end{pgfscope}%
\begin{pgfscope}%
\definecolor{textcolor}{rgb}{0.000000,0.000000,0.000000}%
\pgfsetstrokecolor{textcolor}%
\pgfsetfillcolor{textcolor}%
\pgftext[x=0.944444in,y=1.800000in,right,]{\color{textcolor}\rmfamily\fontsize{10.000000}{12.000000}\selectfont \(\displaystyle {\ensuremath{-}100}\)}%
\end{pgfscope}%
\begin{pgfscope}%
\pgfpathrectangle{\pgfqpoint{1.000000in}{0.600000in}}{\pgfqpoint{6.200000in}{4.800000in}}%
\pgfusepath{clip}%
\pgfsetbuttcap%
\pgfsetroundjoin%
\pgfsetlinewidth{0.501875pt}%
\definecolor{currentstroke}{rgb}{0.000000,0.000000,0.000000}%
\pgfsetstrokecolor{currentstroke}%
\pgfsetdash{{1.000000pt}{3.000000pt}}{0.000000pt}%
\pgfpathmoveto{\pgfqpoint{1.000000in}{2.400000in}}%
\pgfpathlineto{\pgfqpoint{7.200000in}{2.400000in}}%
\pgfusepath{stroke}%
\end{pgfscope}%
\begin{pgfscope}%
\pgfsetbuttcap%
\pgfsetroundjoin%
\definecolor{currentfill}{rgb}{0.000000,0.000000,0.000000}%
\pgfsetfillcolor{currentfill}%
\pgfsetlinewidth{0.501875pt}%
\definecolor{currentstroke}{rgb}{0.000000,0.000000,0.000000}%
\pgfsetstrokecolor{currentstroke}%
\pgfsetdash{}{0pt}%
\pgfsys@defobject{currentmarker}{\pgfqpoint{0.000000in}{0.000000in}}{\pgfqpoint{0.055556in}{0.000000in}}{%
\pgfpathmoveto{\pgfqpoint{0.000000in}{0.000000in}}%
\pgfpathlineto{\pgfqpoint{0.055556in}{0.000000in}}%
\pgfusepath{stroke,fill}%
}%
\begin{pgfscope}%
\pgfsys@transformshift{1.000000in}{2.400000in}%
\pgfsys@useobject{currentmarker}{}%
\end{pgfscope}%
\end{pgfscope}%
\begin{pgfscope}%
\pgfsetbuttcap%
\pgfsetroundjoin%
\definecolor{currentfill}{rgb}{0.000000,0.000000,0.000000}%
\pgfsetfillcolor{currentfill}%
\pgfsetlinewidth{0.501875pt}%
\definecolor{currentstroke}{rgb}{0.000000,0.000000,0.000000}%
\pgfsetstrokecolor{currentstroke}%
\pgfsetdash{}{0pt}%
\pgfsys@defobject{currentmarker}{\pgfqpoint{-0.055556in}{0.000000in}}{\pgfqpoint{-0.000000in}{0.000000in}}{%
\pgfpathmoveto{\pgfqpoint{-0.000000in}{0.000000in}}%
\pgfpathlineto{\pgfqpoint{-0.055556in}{0.000000in}}%
\pgfusepath{stroke,fill}%
}%
\begin{pgfscope}%
\pgfsys@transformshift{7.200000in}{2.400000in}%
\pgfsys@useobject{currentmarker}{}%
\end{pgfscope}%
\end{pgfscope}%
\begin{pgfscope}%
\definecolor{textcolor}{rgb}{0.000000,0.000000,0.000000}%
\pgfsetstrokecolor{textcolor}%
\pgfsetfillcolor{textcolor}%
\pgftext[x=0.944444in,y=2.400000in,right,]{\color{textcolor}\rmfamily\fontsize{10.000000}{12.000000}\selectfont \(\displaystyle {\ensuremath{-}80}\)}%
\end{pgfscope}%
\begin{pgfscope}%
\pgfpathrectangle{\pgfqpoint{1.000000in}{0.600000in}}{\pgfqpoint{6.200000in}{4.800000in}}%
\pgfusepath{clip}%
\pgfsetbuttcap%
\pgfsetroundjoin%
\pgfsetlinewidth{0.501875pt}%
\definecolor{currentstroke}{rgb}{0.000000,0.000000,0.000000}%
\pgfsetstrokecolor{currentstroke}%
\pgfsetdash{{1.000000pt}{3.000000pt}}{0.000000pt}%
\pgfpathmoveto{\pgfqpoint{1.000000in}{3.000000in}}%
\pgfpathlineto{\pgfqpoint{7.200000in}{3.000000in}}%
\pgfusepath{stroke}%
\end{pgfscope}%
\begin{pgfscope}%
\pgfsetbuttcap%
\pgfsetroundjoin%
\definecolor{currentfill}{rgb}{0.000000,0.000000,0.000000}%
\pgfsetfillcolor{currentfill}%
\pgfsetlinewidth{0.501875pt}%
\definecolor{currentstroke}{rgb}{0.000000,0.000000,0.000000}%
\pgfsetstrokecolor{currentstroke}%
\pgfsetdash{}{0pt}%
\pgfsys@defobject{currentmarker}{\pgfqpoint{0.000000in}{0.000000in}}{\pgfqpoint{0.055556in}{0.000000in}}{%
\pgfpathmoveto{\pgfqpoint{0.000000in}{0.000000in}}%
\pgfpathlineto{\pgfqpoint{0.055556in}{0.000000in}}%
\pgfusepath{stroke,fill}%
}%
\begin{pgfscope}%
\pgfsys@transformshift{1.000000in}{3.000000in}%
\pgfsys@useobject{currentmarker}{}%
\end{pgfscope}%
\end{pgfscope}%
\begin{pgfscope}%
\pgfsetbuttcap%
\pgfsetroundjoin%
\definecolor{currentfill}{rgb}{0.000000,0.000000,0.000000}%
\pgfsetfillcolor{currentfill}%
\pgfsetlinewidth{0.501875pt}%
\definecolor{currentstroke}{rgb}{0.000000,0.000000,0.000000}%
\pgfsetstrokecolor{currentstroke}%
\pgfsetdash{}{0pt}%
\pgfsys@defobject{currentmarker}{\pgfqpoint{-0.055556in}{0.000000in}}{\pgfqpoint{-0.000000in}{0.000000in}}{%
\pgfpathmoveto{\pgfqpoint{-0.000000in}{0.000000in}}%
\pgfpathlineto{\pgfqpoint{-0.055556in}{0.000000in}}%
\pgfusepath{stroke,fill}%
}%
\begin{pgfscope}%
\pgfsys@transformshift{7.200000in}{3.000000in}%
\pgfsys@useobject{currentmarker}{}%
\end{pgfscope}%
\end{pgfscope}%
\begin{pgfscope}%
\definecolor{textcolor}{rgb}{0.000000,0.000000,0.000000}%
\pgfsetstrokecolor{textcolor}%
\pgfsetfillcolor{textcolor}%
\pgftext[x=0.944444in,y=3.000000in,right,]{\color{textcolor}\rmfamily\fontsize{10.000000}{12.000000}\selectfont \(\displaystyle {\ensuremath{-}60}\)}%
\end{pgfscope}%
\begin{pgfscope}%
\pgfpathrectangle{\pgfqpoint{1.000000in}{0.600000in}}{\pgfqpoint{6.200000in}{4.800000in}}%
\pgfusepath{clip}%
\pgfsetbuttcap%
\pgfsetroundjoin%
\pgfsetlinewidth{0.501875pt}%
\definecolor{currentstroke}{rgb}{0.000000,0.000000,0.000000}%
\pgfsetstrokecolor{currentstroke}%
\pgfsetdash{{1.000000pt}{3.000000pt}}{0.000000pt}%
\pgfpathmoveto{\pgfqpoint{1.000000in}{3.600000in}}%
\pgfpathlineto{\pgfqpoint{7.200000in}{3.600000in}}%
\pgfusepath{stroke}%
\end{pgfscope}%
\begin{pgfscope}%
\pgfsetbuttcap%
\pgfsetroundjoin%
\definecolor{currentfill}{rgb}{0.000000,0.000000,0.000000}%
\pgfsetfillcolor{currentfill}%
\pgfsetlinewidth{0.501875pt}%
\definecolor{currentstroke}{rgb}{0.000000,0.000000,0.000000}%
\pgfsetstrokecolor{currentstroke}%
\pgfsetdash{}{0pt}%
\pgfsys@defobject{currentmarker}{\pgfqpoint{0.000000in}{0.000000in}}{\pgfqpoint{0.055556in}{0.000000in}}{%
\pgfpathmoveto{\pgfqpoint{0.000000in}{0.000000in}}%
\pgfpathlineto{\pgfqpoint{0.055556in}{0.000000in}}%
\pgfusepath{stroke,fill}%
}%
\begin{pgfscope}%
\pgfsys@transformshift{1.000000in}{3.600000in}%
\pgfsys@useobject{currentmarker}{}%
\end{pgfscope}%
\end{pgfscope}%
\begin{pgfscope}%
\pgfsetbuttcap%
\pgfsetroundjoin%
\definecolor{currentfill}{rgb}{0.000000,0.000000,0.000000}%
\pgfsetfillcolor{currentfill}%
\pgfsetlinewidth{0.501875pt}%
\definecolor{currentstroke}{rgb}{0.000000,0.000000,0.000000}%
\pgfsetstrokecolor{currentstroke}%
\pgfsetdash{}{0pt}%
\pgfsys@defobject{currentmarker}{\pgfqpoint{-0.055556in}{0.000000in}}{\pgfqpoint{-0.000000in}{0.000000in}}{%
\pgfpathmoveto{\pgfqpoint{-0.000000in}{0.000000in}}%
\pgfpathlineto{\pgfqpoint{-0.055556in}{0.000000in}}%
\pgfusepath{stroke,fill}%
}%
\begin{pgfscope}%
\pgfsys@transformshift{7.200000in}{3.600000in}%
\pgfsys@useobject{currentmarker}{}%
\end{pgfscope}%
\end{pgfscope}%
\begin{pgfscope}%
\definecolor{textcolor}{rgb}{0.000000,0.000000,0.000000}%
\pgfsetstrokecolor{textcolor}%
\pgfsetfillcolor{textcolor}%
\pgftext[x=0.944444in,y=3.600000in,right,]{\color{textcolor}\rmfamily\fontsize{10.000000}{12.000000}\selectfont \(\displaystyle {\ensuremath{-}40}\)}%
\end{pgfscope}%
\begin{pgfscope}%
\pgfpathrectangle{\pgfqpoint{1.000000in}{0.600000in}}{\pgfqpoint{6.200000in}{4.800000in}}%
\pgfusepath{clip}%
\pgfsetbuttcap%
\pgfsetroundjoin%
\pgfsetlinewidth{0.501875pt}%
\definecolor{currentstroke}{rgb}{0.000000,0.000000,0.000000}%
\pgfsetstrokecolor{currentstroke}%
\pgfsetdash{{1.000000pt}{3.000000pt}}{0.000000pt}%
\pgfpathmoveto{\pgfqpoint{1.000000in}{4.200000in}}%
\pgfpathlineto{\pgfqpoint{7.200000in}{4.200000in}}%
\pgfusepath{stroke}%
\end{pgfscope}%
\begin{pgfscope}%
\pgfsetbuttcap%
\pgfsetroundjoin%
\definecolor{currentfill}{rgb}{0.000000,0.000000,0.000000}%
\pgfsetfillcolor{currentfill}%
\pgfsetlinewidth{0.501875pt}%
\definecolor{currentstroke}{rgb}{0.000000,0.000000,0.000000}%
\pgfsetstrokecolor{currentstroke}%
\pgfsetdash{}{0pt}%
\pgfsys@defobject{currentmarker}{\pgfqpoint{0.000000in}{0.000000in}}{\pgfqpoint{0.055556in}{0.000000in}}{%
\pgfpathmoveto{\pgfqpoint{0.000000in}{0.000000in}}%
\pgfpathlineto{\pgfqpoint{0.055556in}{0.000000in}}%
\pgfusepath{stroke,fill}%
}%
\begin{pgfscope}%
\pgfsys@transformshift{1.000000in}{4.200000in}%
\pgfsys@useobject{currentmarker}{}%
\end{pgfscope}%
\end{pgfscope}%
\begin{pgfscope}%
\pgfsetbuttcap%
\pgfsetroundjoin%
\definecolor{currentfill}{rgb}{0.000000,0.000000,0.000000}%
\pgfsetfillcolor{currentfill}%
\pgfsetlinewidth{0.501875pt}%
\definecolor{currentstroke}{rgb}{0.000000,0.000000,0.000000}%
\pgfsetstrokecolor{currentstroke}%
\pgfsetdash{}{0pt}%
\pgfsys@defobject{currentmarker}{\pgfqpoint{-0.055556in}{0.000000in}}{\pgfqpoint{-0.000000in}{0.000000in}}{%
\pgfpathmoveto{\pgfqpoint{-0.000000in}{0.000000in}}%
\pgfpathlineto{\pgfqpoint{-0.055556in}{0.000000in}}%
\pgfusepath{stroke,fill}%
}%
\begin{pgfscope}%
\pgfsys@transformshift{7.200000in}{4.200000in}%
\pgfsys@useobject{currentmarker}{}%
\end{pgfscope}%
\end{pgfscope}%
\begin{pgfscope}%
\definecolor{textcolor}{rgb}{0.000000,0.000000,0.000000}%
\pgfsetstrokecolor{textcolor}%
\pgfsetfillcolor{textcolor}%
\pgftext[x=0.944444in,y=4.200000in,right,]{\color{textcolor}\rmfamily\fontsize{10.000000}{12.000000}\selectfont \(\displaystyle {\ensuremath{-}20}\)}%
\end{pgfscope}%
\begin{pgfscope}%
\pgfpathrectangle{\pgfqpoint{1.000000in}{0.600000in}}{\pgfqpoint{6.200000in}{4.800000in}}%
\pgfusepath{clip}%
\pgfsetbuttcap%
\pgfsetroundjoin%
\pgfsetlinewidth{0.501875pt}%
\definecolor{currentstroke}{rgb}{0.000000,0.000000,0.000000}%
\pgfsetstrokecolor{currentstroke}%
\pgfsetdash{{1.000000pt}{3.000000pt}}{0.000000pt}%
\pgfpathmoveto{\pgfqpoint{1.000000in}{4.800000in}}%
\pgfpathlineto{\pgfqpoint{7.200000in}{4.800000in}}%
\pgfusepath{stroke}%
\end{pgfscope}%
\begin{pgfscope}%
\pgfsetbuttcap%
\pgfsetroundjoin%
\definecolor{currentfill}{rgb}{0.000000,0.000000,0.000000}%
\pgfsetfillcolor{currentfill}%
\pgfsetlinewidth{0.501875pt}%
\definecolor{currentstroke}{rgb}{0.000000,0.000000,0.000000}%
\pgfsetstrokecolor{currentstroke}%
\pgfsetdash{}{0pt}%
\pgfsys@defobject{currentmarker}{\pgfqpoint{0.000000in}{0.000000in}}{\pgfqpoint{0.055556in}{0.000000in}}{%
\pgfpathmoveto{\pgfqpoint{0.000000in}{0.000000in}}%
\pgfpathlineto{\pgfqpoint{0.055556in}{0.000000in}}%
\pgfusepath{stroke,fill}%
}%
\begin{pgfscope}%
\pgfsys@transformshift{1.000000in}{4.800000in}%
\pgfsys@useobject{currentmarker}{}%
\end{pgfscope}%
\end{pgfscope}%
\begin{pgfscope}%
\pgfsetbuttcap%
\pgfsetroundjoin%
\definecolor{currentfill}{rgb}{0.000000,0.000000,0.000000}%
\pgfsetfillcolor{currentfill}%
\pgfsetlinewidth{0.501875pt}%
\definecolor{currentstroke}{rgb}{0.000000,0.000000,0.000000}%
\pgfsetstrokecolor{currentstroke}%
\pgfsetdash{}{0pt}%
\pgfsys@defobject{currentmarker}{\pgfqpoint{-0.055556in}{0.000000in}}{\pgfqpoint{-0.000000in}{0.000000in}}{%
\pgfpathmoveto{\pgfqpoint{-0.000000in}{0.000000in}}%
\pgfpathlineto{\pgfqpoint{-0.055556in}{0.000000in}}%
\pgfusepath{stroke,fill}%
}%
\begin{pgfscope}%
\pgfsys@transformshift{7.200000in}{4.800000in}%
\pgfsys@useobject{currentmarker}{}%
\end{pgfscope}%
\end{pgfscope}%
\begin{pgfscope}%
\definecolor{textcolor}{rgb}{0.000000,0.000000,0.000000}%
\pgfsetstrokecolor{textcolor}%
\pgfsetfillcolor{textcolor}%
\pgftext[x=0.944444in,y=4.800000in,right,]{\color{textcolor}\rmfamily\fontsize{10.000000}{12.000000}\selectfont \(\displaystyle {0}\)}%
\end{pgfscope}%
\begin{pgfscope}%
\pgfpathrectangle{\pgfqpoint{1.000000in}{0.600000in}}{\pgfqpoint{6.200000in}{4.800000in}}%
\pgfusepath{clip}%
\pgfsetbuttcap%
\pgfsetroundjoin%
\pgfsetlinewidth{0.501875pt}%
\definecolor{currentstroke}{rgb}{0.000000,0.000000,0.000000}%
\pgfsetstrokecolor{currentstroke}%
\pgfsetdash{{1.000000pt}{3.000000pt}}{0.000000pt}%
\pgfpathmoveto{\pgfqpoint{1.000000in}{5.400000in}}%
\pgfpathlineto{\pgfqpoint{7.200000in}{5.400000in}}%
\pgfusepath{stroke}%
\end{pgfscope}%
\begin{pgfscope}%
\pgfsetbuttcap%
\pgfsetroundjoin%
\definecolor{currentfill}{rgb}{0.000000,0.000000,0.000000}%
\pgfsetfillcolor{currentfill}%
\pgfsetlinewidth{0.501875pt}%
\definecolor{currentstroke}{rgb}{0.000000,0.000000,0.000000}%
\pgfsetstrokecolor{currentstroke}%
\pgfsetdash{}{0pt}%
\pgfsys@defobject{currentmarker}{\pgfqpoint{0.000000in}{0.000000in}}{\pgfqpoint{0.055556in}{0.000000in}}{%
\pgfpathmoveto{\pgfqpoint{0.000000in}{0.000000in}}%
\pgfpathlineto{\pgfqpoint{0.055556in}{0.000000in}}%
\pgfusepath{stroke,fill}%
}%
\begin{pgfscope}%
\pgfsys@transformshift{1.000000in}{5.400000in}%
\pgfsys@useobject{currentmarker}{}%
\end{pgfscope}%
\end{pgfscope}%
\begin{pgfscope}%
\pgfsetbuttcap%
\pgfsetroundjoin%
\definecolor{currentfill}{rgb}{0.000000,0.000000,0.000000}%
\pgfsetfillcolor{currentfill}%
\pgfsetlinewidth{0.501875pt}%
\definecolor{currentstroke}{rgb}{0.000000,0.000000,0.000000}%
\pgfsetstrokecolor{currentstroke}%
\pgfsetdash{}{0pt}%
\pgfsys@defobject{currentmarker}{\pgfqpoint{-0.055556in}{0.000000in}}{\pgfqpoint{-0.000000in}{0.000000in}}{%
\pgfpathmoveto{\pgfqpoint{-0.000000in}{0.000000in}}%
\pgfpathlineto{\pgfqpoint{-0.055556in}{0.000000in}}%
\pgfusepath{stroke,fill}%
}%
\begin{pgfscope}%
\pgfsys@transformshift{7.200000in}{5.400000in}%
\pgfsys@useobject{currentmarker}{}%
\end{pgfscope}%
\end{pgfscope}%
\begin{pgfscope}%
\definecolor{textcolor}{rgb}{0.000000,0.000000,0.000000}%
\pgfsetstrokecolor{textcolor}%
\pgfsetfillcolor{textcolor}%
\pgftext[x=0.944444in,y=5.400000in,right,]{\color{textcolor}\rmfamily\fontsize{10.000000}{12.000000}\selectfont \(\displaystyle {20}\)}%
\end{pgfscope}%
\begin{pgfscope}%
\definecolor{textcolor}{rgb}{0.000000,0.000000,0.000000}%
\pgfsetstrokecolor{textcolor}%
\pgfsetfillcolor{textcolor}%
\pgftext[x=0.558641in,y=3.000000in,,bottom,rotate=90.000000]{\color{textcolor}\rmfamily\fontsize{12.000000}{14.400000}\selectfont \(\displaystyle \theta\ (rad)\)}%
\end{pgfscope}%
\begin{pgfscope}%
\definecolor{textcolor}{rgb}{0.000000,0.000000,0.000000}%
\pgfsetstrokecolor{textcolor}%
\pgfsetfillcolor{textcolor}%
\pgftext[x=4.100000in,y=5.469444in,,base]{\color{textcolor}\rmfamily\fontsize{12.000000}{14.400000}\selectfont \(\displaystyle Simple\ pendulum\ using\ Euler's\ methods\ (time\ step = 0.1\ (s))\)}%
\end{pgfscope}%
\begin{pgfscope}%
\pgfsetbuttcap%
\pgfsetmiterjoin%
\definecolor{currentfill}{rgb}{1.000000,1.000000,1.000000}%
\pgfsetfillcolor{currentfill}%
\pgfsetlinewidth{1.003750pt}%
\definecolor{currentstroke}{rgb}{0.000000,0.000000,0.000000}%
\pgfsetstrokecolor{currentstroke}%
\pgfsetdash{}{0pt}%
\pgfpathmoveto{\pgfqpoint{1.083333in}{0.683333in}}%
\pgfpathlineto{\pgfqpoint{3.093110in}{0.683333in}}%
\pgfpathlineto{\pgfqpoint{3.093110in}{1.430555in}}%
\pgfpathlineto{\pgfqpoint{1.083333in}{1.430555in}}%
\pgfpathlineto{\pgfqpoint{1.083333in}{0.683333in}}%
\pgfpathclose%
\pgfusepath{stroke,fill}%
\end{pgfscope}%
\begin{pgfscope}%
\pgfsetrectcap%
\pgfsetroundjoin%
\pgfsetlinewidth{1.003750pt}%
\definecolor{currentstroke}{rgb}{1.000000,0.000000,0.000000}%
\pgfsetstrokecolor{currentstroke}%
\pgfsetdash{}{0pt}%
\pgfpathmoveto{\pgfqpoint{1.200000in}{1.305555in}}%
\pgfpathlineto{\pgfqpoint{1.433333in}{1.305555in}}%
\pgfusepath{stroke}%
\end{pgfscope}%
\begin{pgfscope}%
\definecolor{textcolor}{rgb}{0.000000,0.000000,0.000000}%
\pgfsetstrokecolor{textcolor}%
\pgfsetfillcolor{textcolor}%
\pgftext[x=1.616667in,y=1.247221in,left,base]{\color{textcolor}\rmfamily\fontsize{12.000000}{14.400000}\selectfont \(\displaystyle euler\ explicit\)}%
\end{pgfscope}%
\begin{pgfscope}%
\pgfsetrectcap%
\pgfsetroundjoin%
\pgfsetlinewidth{1.003750pt}%
\definecolor{currentstroke}{rgb}{0.000000,0.000000,1.000000}%
\pgfsetstrokecolor{currentstroke}%
\pgfsetdash{}{0pt}%
\pgfpathmoveto{\pgfqpoint{1.200000in}{1.073147in}}%
\pgfpathlineto{\pgfqpoint{1.433333in}{1.073147in}}%
\pgfusepath{stroke}%
\end{pgfscope}%
\begin{pgfscope}%
\definecolor{textcolor}{rgb}{0.000000,0.000000,0.000000}%
\pgfsetstrokecolor{textcolor}%
\pgfsetfillcolor{textcolor}%
\pgftext[x=1.616667in,y=1.014814in,left,base]{\color{textcolor}\rmfamily\fontsize{12.000000}{14.400000}\selectfont \(\displaystyle euler\ implicit\)}%
\end{pgfscope}%
\begin{pgfscope}%
\pgfsetrectcap%
\pgfsetroundjoin%
\pgfsetlinewidth{1.003750pt}%
\definecolor{currentstroke}{rgb}{0.000000,0.000000,0.000000}%
\pgfsetstrokecolor{currentstroke}%
\pgfsetdash{}{0pt}%
\pgfpathmoveto{\pgfqpoint{1.200000in}{0.840740in}}%
\pgfpathlineto{\pgfqpoint{1.433333in}{0.840740in}}%
\pgfusepath{stroke}%
\end{pgfscope}%
\begin{pgfscope}%
\definecolor{textcolor}{rgb}{0.000000,0.000000,0.000000}%
\pgfsetstrokecolor{textcolor}%
\pgfsetfillcolor{textcolor}%
\pgftext[x=1.616667in,y=0.782407in,left,base]{\color{textcolor}\rmfamily\fontsize{12.000000}{14.400000}\selectfont \(\displaystyle trapezoidal\ scheme\)}%
\end{pgfscope}%
\end{pgfpicture}%
\makeatother%
\endgroup%
}
    \end{figure}

    \begin{figure}[ht!]
    \centering
    \resizebox{0.9\linewidth}{!}{%% Creator: Matplotlib, PGF backend
%%
%% To include the figure in your LaTeX document, write
%%   \input{<filename>.pgf}
%%
%% Make sure the required packages are loaded in your preamble
%%   \usepackage{pgf}
%%
%% Also ensure that all the required font packages are loaded; for instance,
%% the lmodern package is sometimes necessary when using math font.
%%   \usepackage{lmodern}
%%
%% Figures using additional raster images can only be included by \input if
%% they are in the same directory as the main LaTeX file. For loading figures
%% from other directories you can use the `import` package
%%   \usepackage{import}
%%
%% and then include the figures with
%%   \import{<path to file>}{<filename>.pgf}
%%
%% Matplotlib used the following preamble
%%
\begingroup%
\makeatletter%
\begin{pgfpicture}%
\pgfpathrectangle{\pgfpointorigin}{\pgfqpoint{8.000000in}{6.000000in}}%
\pgfusepath{use as bounding box, clip}%
\begin{pgfscope}%
\pgfsetbuttcap%
\pgfsetmiterjoin%
\definecolor{currentfill}{rgb}{1.000000,1.000000,1.000000}%
\pgfsetfillcolor{currentfill}%
\pgfsetlinewidth{0.000000pt}%
\definecolor{currentstroke}{rgb}{1.000000,1.000000,1.000000}%
\pgfsetstrokecolor{currentstroke}%
\pgfsetdash{}{0pt}%
\pgfpathmoveto{\pgfqpoint{0.000000in}{0.000000in}}%
\pgfpathlineto{\pgfqpoint{8.000000in}{0.000000in}}%
\pgfpathlineto{\pgfqpoint{8.000000in}{6.000000in}}%
\pgfpathlineto{\pgfqpoint{0.000000in}{6.000000in}}%
\pgfpathlineto{\pgfqpoint{0.000000in}{0.000000in}}%
\pgfpathclose%
\pgfusepath{fill}%
\end{pgfscope}%
\begin{pgfscope}%
\pgfsetbuttcap%
\pgfsetmiterjoin%
\definecolor{currentfill}{rgb}{1.000000,1.000000,1.000000}%
\pgfsetfillcolor{currentfill}%
\pgfsetlinewidth{0.000000pt}%
\definecolor{currentstroke}{rgb}{0.000000,0.000000,0.000000}%
\pgfsetstrokecolor{currentstroke}%
\pgfsetstrokeopacity{0.000000}%
\pgfsetdash{}{0pt}%
\pgfpathmoveto{\pgfqpoint{1.000000in}{0.600000in}}%
\pgfpathlineto{\pgfqpoint{7.200000in}{0.600000in}}%
\pgfpathlineto{\pgfqpoint{7.200000in}{5.400000in}}%
\pgfpathlineto{\pgfqpoint{1.000000in}{5.400000in}}%
\pgfpathlineto{\pgfqpoint{1.000000in}{0.600000in}}%
\pgfpathclose%
\pgfusepath{fill}%
\end{pgfscope}%
\begin{pgfscope}%
\pgfpathrectangle{\pgfqpoint{1.000000in}{0.600000in}}{\pgfqpoint{6.200000in}{4.800000in}}%
\pgfusepath{clip}%
\pgfsetrectcap%
\pgfsetroundjoin%
\pgfsetlinewidth{1.003750pt}%
\definecolor{currentstroke}{rgb}{1.000000,0.000000,0.000000}%
\pgfsetstrokecolor{currentstroke}%
\pgfsetdash{}{0pt}%
\pgfpathmoveto{\pgfqpoint{1.000000in}{1.408378in}}%
\pgfpathlineto{\pgfqpoint{1.037200in}{1.407418in}}%
\pgfpathlineto{\pgfqpoint{1.074400in}{1.403747in}}%
\pgfpathlineto{\pgfqpoint{1.186000in}{1.389159in}}%
\pgfpathlineto{\pgfqpoint{1.210800in}{1.388522in}}%
\pgfpathlineto{\pgfqpoint{1.235600in}{1.389542in}}%
\pgfpathlineto{\pgfqpoint{1.260400in}{1.392194in}}%
\pgfpathlineto{\pgfqpoint{1.297600in}{1.398646in}}%
\pgfpathlineto{\pgfqpoint{1.359600in}{1.410940in}}%
\pgfpathlineto{\pgfqpoint{1.384400in}{1.414190in}}%
\pgfpathlineto{\pgfqpoint{1.409200in}{1.415624in}}%
\pgfpathlineto{\pgfqpoint{1.434000in}{1.414993in}}%
\pgfpathlineto{\pgfqpoint{1.458800in}{1.412257in}}%
\pgfpathlineto{\pgfqpoint{1.483600in}{1.407586in}}%
\pgfpathlineto{\pgfqpoint{1.520800in}{1.398008in}}%
\pgfpathlineto{\pgfqpoint{1.570400in}{1.385135in}}%
\pgfpathlineto{\pgfqpoint{1.595200in}{1.380844in}}%
\pgfpathlineto{\pgfqpoint{1.620000in}{1.378789in}}%
\pgfpathlineto{\pgfqpoint{1.644800in}{1.379178in}}%
\pgfpathlineto{\pgfqpoint{1.669600in}{1.382048in}}%
\pgfpathlineto{\pgfqpoint{1.694400in}{1.387319in}}%
\pgfpathlineto{\pgfqpoint{1.719200in}{1.394682in}}%
\pgfpathlineto{\pgfqpoint{1.793600in}{1.419987in}}%
\pgfpathlineto{\pgfqpoint{1.818400in}{1.425538in}}%
\pgfpathlineto{\pgfqpoint{1.843200in}{1.428574in}}%
\pgfpathlineto{\pgfqpoint{1.868000in}{1.429076in}}%
\pgfpathlineto{\pgfqpoint{1.892800in}{1.427093in}}%
\pgfpathlineto{\pgfqpoint{1.917600in}{1.422601in}}%
\pgfpathlineto{\pgfqpoint{1.942400in}{1.415561in}}%
\pgfpathlineto{\pgfqpoint{1.967200in}{1.406162in}}%
\pgfpathlineto{\pgfqpoint{2.041600in}{1.374476in}}%
\pgfpathlineto{\pgfqpoint{2.066400in}{1.367295in}}%
\pgfpathlineto{\pgfqpoint{2.091200in}{1.362557in}}%
\pgfpathlineto{\pgfqpoint{2.116000in}{1.359876in}}%
\pgfpathlineto{\pgfqpoint{2.140800in}{1.358862in}}%
\pgfpathlineto{\pgfqpoint{2.178000in}{1.360016in}}%
\pgfpathlineto{\pgfqpoint{2.202800in}{1.362581in}}%
\pgfpathlineto{\pgfqpoint{2.227600in}{1.366825in}}%
\pgfpathlineto{\pgfqpoint{2.252400in}{1.373110in}}%
\pgfpathlineto{\pgfqpoint{2.277200in}{1.381805in}}%
\pgfpathlineto{\pgfqpoint{2.302000in}{1.392984in}}%
\pgfpathlineto{\pgfqpoint{2.376400in}{1.430306in}}%
\pgfpathlineto{\pgfqpoint{2.401200in}{1.439143in}}%
\pgfpathlineto{\pgfqpoint{2.438400in}{1.448814in}}%
\pgfpathlineto{\pgfqpoint{2.500400in}{1.462914in}}%
\pgfpathlineto{\pgfqpoint{2.525200in}{1.470343in}}%
\pgfpathlineto{\pgfqpoint{2.550000in}{1.479919in}}%
\pgfpathlineto{\pgfqpoint{2.574800in}{1.491997in}}%
\pgfpathlineto{\pgfqpoint{2.649200in}{1.532879in}}%
\pgfpathlineto{\pgfqpoint{2.674000in}{1.542972in}}%
\pgfpathlineto{\pgfqpoint{2.723600in}{1.559033in}}%
\pgfpathlineto{\pgfqpoint{2.760800in}{1.572278in}}%
\pgfpathlineto{\pgfqpoint{2.785600in}{1.583640in}}%
\pgfpathlineto{\pgfqpoint{2.810400in}{1.597464in}}%
\pgfpathlineto{\pgfqpoint{2.860000in}{1.627530in}}%
\pgfpathlineto{\pgfqpoint{2.884800in}{1.639999in}}%
\pgfpathlineto{\pgfqpoint{2.922000in}{1.654988in}}%
\pgfpathlineto{\pgfqpoint{2.959200in}{1.669578in}}%
\pgfpathlineto{\pgfqpoint{2.984000in}{1.681343in}}%
\pgfpathlineto{\pgfqpoint{3.008800in}{1.695577in}}%
\pgfpathlineto{\pgfqpoint{3.070800in}{1.734455in}}%
\pgfpathlineto{\pgfqpoint{3.095600in}{1.746818in}}%
\pgfpathlineto{\pgfqpoint{3.170000in}{1.780906in}}%
\pgfpathlineto{\pgfqpoint{3.194800in}{1.795787in}}%
\pgfpathlineto{\pgfqpoint{3.256800in}{1.836403in}}%
\pgfpathlineto{\pgfqpoint{3.281600in}{1.849466in}}%
\pgfpathlineto{\pgfqpoint{3.343600in}{1.880109in}}%
\pgfpathlineto{\pgfqpoint{3.368400in}{1.895499in}}%
\pgfpathlineto{\pgfqpoint{3.430400in}{1.937741in}}%
\pgfpathlineto{\pgfqpoint{3.455200in}{1.951511in}}%
\pgfpathlineto{\pgfqpoint{3.504800in}{1.977409in}}%
\pgfpathlineto{\pgfqpoint{3.529600in}{1.992921in}}%
\pgfpathlineto{\pgfqpoint{3.604000in}{2.044445in}}%
\pgfpathlineto{\pgfqpoint{3.678400in}{2.086887in}}%
\pgfpathlineto{\pgfqpoint{3.703200in}{2.104426in}}%
\pgfpathlineto{\pgfqpoint{3.740400in}{2.132090in}}%
\pgfpathlineto{\pgfqpoint{3.765200in}{2.148185in}}%
\pgfpathlineto{\pgfqpoint{3.827200in}{2.184904in}}%
\pgfpathlineto{\pgfqpoint{3.852000in}{2.202650in}}%
\pgfpathlineto{\pgfqpoint{3.901600in}{2.239943in}}%
\pgfpathlineto{\pgfqpoint{3.938800in}{2.263049in}}%
\pgfpathlineto{\pgfqpoint{3.976000in}{2.286476in}}%
\pgfpathlineto{\pgfqpoint{4.000800in}{2.304885in}}%
\pgfpathlineto{\pgfqpoint{4.050400in}{2.342921in}}%
\pgfpathlineto{\pgfqpoint{4.087600in}{2.366636in}}%
\pgfpathlineto{\pgfqpoint{4.112400in}{2.382684in}}%
\pgfpathlineto{\pgfqpoint{4.137200in}{2.400987in}}%
\pgfpathlineto{\pgfqpoint{4.186800in}{2.440339in}}%
\pgfpathlineto{\pgfqpoint{4.224000in}{2.465236in}}%
\pgfpathlineto{\pgfqpoint{4.261200in}{2.490607in}}%
\pgfpathlineto{\pgfqpoint{4.298400in}{2.520666in}}%
\pgfpathlineto{\pgfqpoint{4.323200in}{2.540481in}}%
\pgfpathlineto{\pgfqpoint{4.360400in}{2.566199in}}%
\pgfpathlineto{\pgfqpoint{4.385200in}{2.583175in}}%
\pgfpathlineto{\pgfqpoint{4.410000in}{2.602332in}}%
\pgfpathlineto{\pgfqpoint{4.459600in}{2.643302in}}%
\pgfpathlineto{\pgfqpoint{4.496800in}{2.669507in}}%
\pgfpathlineto{\pgfqpoint{4.521600in}{2.687247in}}%
\pgfpathlineto{\pgfqpoint{4.546400in}{2.707298in}}%
\pgfpathlineto{\pgfqpoint{4.583600in}{2.738956in}}%
\pgfpathlineto{\pgfqpoint{4.608400in}{2.757671in}}%
\pgfpathlineto{\pgfqpoint{4.658000in}{2.793972in}}%
\pgfpathlineto{\pgfqpoint{4.744800in}{2.865135in}}%
\pgfpathlineto{\pgfqpoint{4.782000in}{2.893119in}}%
\pgfpathlineto{\pgfqpoint{4.819200in}{2.925601in}}%
\pgfpathlineto{\pgfqpoint{4.844000in}{2.946751in}}%
\pgfpathlineto{\pgfqpoint{4.918400in}{3.004940in}}%
\pgfpathlineto{\pgfqpoint{4.968000in}{3.048761in}}%
\pgfpathlineto{\pgfqpoint{5.042400in}{3.108431in}}%
\pgfpathlineto{\pgfqpoint{5.092000in}{3.152639in}}%
\pgfpathlineto{\pgfqpoint{5.154000in}{3.202660in}}%
\pgfpathlineto{\pgfqpoint{5.216000in}{3.258259in}}%
\pgfpathlineto{\pgfqpoint{5.265600in}{3.298646in}}%
\pgfpathlineto{\pgfqpoint{5.340000in}{3.365504in}}%
\pgfpathlineto{\pgfqpoint{5.377200in}{3.396249in}}%
\pgfpathlineto{\pgfqpoint{5.464000in}{3.474366in}}%
\pgfpathlineto{\pgfqpoint{5.488800in}{3.495362in}}%
\pgfpathlineto{\pgfqpoint{5.526000in}{3.530727in}}%
\pgfpathlineto{\pgfqpoint{5.550800in}{3.553712in}}%
\pgfpathlineto{\pgfqpoint{5.612800in}{3.607449in}}%
\pgfpathlineto{\pgfqpoint{5.662400in}{3.654978in}}%
\pgfpathlineto{\pgfqpoint{5.724400in}{3.709657in}}%
\pgfpathlineto{\pgfqpoint{5.774000in}{3.757608in}}%
\pgfpathlineto{\pgfqpoint{5.823600in}{3.801314in}}%
\pgfpathlineto{\pgfqpoint{5.898000in}{3.872420in}}%
\pgfpathlineto{\pgfqpoint{5.935200in}{3.906177in}}%
\pgfpathlineto{\pgfqpoint{5.997200in}{3.966742in}}%
\pgfpathlineto{\pgfqpoint{6.034400in}{4.000330in}}%
\pgfpathlineto{\pgfqpoint{6.121200in}{4.084257in}}%
\pgfpathlineto{\pgfqpoint{6.146000in}{4.107771in}}%
\pgfpathlineto{\pgfqpoint{6.208000in}{4.169554in}}%
\pgfpathlineto{\pgfqpoint{6.245200in}{4.204183in}}%
\pgfpathlineto{\pgfqpoint{6.319600in}{4.278293in}}%
\pgfpathlineto{\pgfqpoint{6.344400in}{4.301706in}}%
\pgfpathlineto{\pgfqpoint{6.431200in}{4.388312in}}%
\pgfpathlineto{\pgfqpoint{6.456000in}{4.412886in}}%
\pgfpathlineto{\pgfqpoint{6.505600in}{4.464395in}}%
\pgfpathlineto{\pgfqpoint{6.555200in}{4.512548in}}%
\pgfpathlineto{\pgfqpoint{6.604800in}{4.564580in}}%
\pgfpathlineto{\pgfqpoint{6.654400in}{4.613247in}}%
\pgfpathlineto{\pgfqpoint{6.704000in}{4.665720in}}%
\pgfpathlineto{\pgfqpoint{6.753600in}{4.714970in}}%
\pgfpathlineto{\pgfqpoint{6.803200in}{4.767799in}}%
\pgfpathlineto{\pgfqpoint{6.852800in}{4.817715in}}%
\pgfpathlineto{\pgfqpoint{6.902400in}{4.870799in}}%
\pgfpathlineto{\pgfqpoint{6.939600in}{4.908057in}}%
\pgfpathlineto{\pgfqpoint{7.014000in}{4.986980in}}%
\pgfpathlineto{\pgfqpoint{7.038800in}{5.012584in}}%
\pgfpathlineto{\pgfqpoint{7.100800in}{5.079474in}}%
\pgfpathlineto{\pgfqpoint{7.138000in}{5.118144in}}%
\pgfpathlineto{\pgfqpoint{7.187600in}{5.172591in}}%
\pgfpathlineto{\pgfqpoint{7.200000in}{5.185133in}}%
\pgfpathlineto{\pgfqpoint{7.200000in}{5.185133in}}%
\pgfusepath{stroke}%
\end{pgfscope}%
\begin{pgfscope}%
\pgfpathrectangle{\pgfqpoint{1.000000in}{0.600000in}}{\pgfqpoint{6.200000in}{4.800000in}}%
\pgfusepath{clip}%
\pgfsetrectcap%
\pgfsetroundjoin%
\pgfsetlinewidth{1.003750pt}%
\definecolor{currentstroke}{rgb}{0.000000,0.000000,1.000000}%
\pgfsetstrokecolor{currentstroke}%
\pgfsetdash{}{0pt}%
\pgfpathmoveto{\pgfqpoint{1.000000in}{1.408378in}}%
\pgfpathlineto{\pgfqpoint{1.037200in}{1.406525in}}%
\pgfpathlineto{\pgfqpoint{1.086800in}{1.401152in}}%
\pgfpathlineto{\pgfqpoint{1.136400in}{1.395869in}}%
\pgfpathlineto{\pgfqpoint{1.173600in}{1.393905in}}%
\pgfpathlineto{\pgfqpoint{1.210800in}{1.394233in}}%
\pgfpathlineto{\pgfqpoint{1.260400in}{1.397537in}}%
\pgfpathlineto{\pgfqpoint{1.334800in}{1.403455in}}%
\pgfpathlineto{\pgfqpoint{1.372000in}{1.404603in}}%
\pgfpathlineto{\pgfqpoint{1.421600in}{1.403517in}}%
\pgfpathlineto{\pgfqpoint{1.570400in}{1.396574in}}%
\pgfpathlineto{\pgfqpoint{1.620000in}{1.397694in}}%
\pgfpathlineto{\pgfqpoint{1.756400in}{1.402540in}}%
\pgfpathlineto{\pgfqpoint{1.818400in}{1.401450in}}%
\pgfpathlineto{\pgfqpoint{1.942400in}{1.398126in}}%
\pgfpathlineto{\pgfqpoint{2.016800in}{1.399137in}}%
\pgfpathlineto{\pgfqpoint{2.140800in}{1.401395in}}%
\pgfpathlineto{\pgfqpoint{2.240000in}{1.399976in}}%
\pgfpathlineto{\pgfqpoint{2.339200in}{1.398983in}}%
\pgfpathlineto{\pgfqpoint{2.674000in}{1.399519in}}%
\pgfpathlineto{\pgfqpoint{2.810400in}{1.400020in}}%
\pgfpathlineto{\pgfqpoint{2.959200in}{1.400228in}}%
\pgfpathlineto{\pgfqpoint{3.182400in}{1.399979in}}%
\pgfpathlineto{\pgfqpoint{3.405600in}{1.399913in}}%
\pgfpathlineto{\pgfqpoint{3.852000in}{1.399906in}}%
\pgfpathlineto{\pgfqpoint{7.200000in}{1.400000in}}%
\pgfpathlineto{\pgfqpoint{7.200000in}{1.400000in}}%
\pgfusepath{stroke}%
\end{pgfscope}%
\begin{pgfscope}%
\pgfpathrectangle{\pgfqpoint{1.000000in}{0.600000in}}{\pgfqpoint{6.200000in}{4.800000in}}%
\pgfusepath{clip}%
\pgfsetrectcap%
\pgfsetroundjoin%
\pgfsetlinewidth{1.003750pt}%
\definecolor{currentstroke}{rgb}{0.000000,0.000000,0.000000}%
\pgfsetstrokecolor{currentstroke}%
\pgfsetdash{}{0pt}%
\pgfpathmoveto{\pgfqpoint{1.000000in}{1.408378in}}%
\pgfpathlineto{\pgfqpoint{1.037200in}{1.406963in}}%
\pgfpathlineto{\pgfqpoint{1.074400in}{1.403160in}}%
\pgfpathlineto{\pgfqpoint{1.161200in}{1.392920in}}%
\pgfpathlineto{\pgfqpoint{1.198400in}{1.391599in}}%
\pgfpathlineto{\pgfqpoint{1.235600in}{1.393122in}}%
\pgfpathlineto{\pgfqpoint{1.272800in}{1.397006in}}%
\pgfpathlineto{\pgfqpoint{1.359600in}{1.407200in}}%
\pgfpathlineto{\pgfqpoint{1.396800in}{1.408420in}}%
\pgfpathlineto{\pgfqpoint{1.434000in}{1.406790in}}%
\pgfpathlineto{\pgfqpoint{1.471200in}{1.402829in}}%
\pgfpathlineto{\pgfqpoint{1.545600in}{1.393656in}}%
\pgfpathlineto{\pgfqpoint{1.582800in}{1.391619in}}%
\pgfpathlineto{\pgfqpoint{1.620000in}{1.392419in}}%
\pgfpathlineto{\pgfqpoint{1.657200in}{1.395805in}}%
\pgfpathlineto{\pgfqpoint{1.756400in}{1.407427in}}%
\pgfpathlineto{\pgfqpoint{1.793600in}{1.408447in}}%
\pgfpathlineto{\pgfqpoint{1.830800in}{1.406610in}}%
\pgfpathlineto{\pgfqpoint{1.880400in}{1.400848in}}%
\pgfpathlineto{\pgfqpoint{1.942400in}{1.393385in}}%
\pgfpathlineto{\pgfqpoint{1.979600in}{1.391529in}}%
\pgfpathlineto{\pgfqpoint{2.016800in}{1.392539in}}%
\pgfpathlineto{\pgfqpoint{2.054000in}{1.396096in}}%
\pgfpathlineto{\pgfqpoint{2.153200in}{1.407637in}}%
\pgfpathlineto{\pgfqpoint{2.190400in}{1.408461in}}%
\pgfpathlineto{\pgfqpoint{2.227600in}{1.406422in}}%
\pgfpathlineto{\pgfqpoint{2.277200in}{1.400498in}}%
\pgfpathlineto{\pgfqpoint{2.339200in}{1.393129in}}%
\pgfpathlineto{\pgfqpoint{2.376400in}{1.391454in}}%
\pgfpathlineto{\pgfqpoint{2.413600in}{1.392668in}}%
\pgfpathlineto{\pgfqpoint{2.450800in}{1.396387in}}%
\pgfpathlineto{\pgfqpoint{2.537600in}{1.406992in}}%
\pgfpathlineto{\pgfqpoint{2.574800in}{1.408578in}}%
\pgfpathlineto{\pgfqpoint{2.612000in}{1.407265in}}%
\pgfpathlineto{\pgfqpoint{2.649200in}{1.403468in}}%
\pgfpathlineto{\pgfqpoint{2.736000in}{1.392890in}}%
\pgfpathlineto{\pgfqpoint{2.773200in}{1.391393in}}%
\pgfpathlineto{\pgfqpoint{2.810400in}{1.392804in}}%
\pgfpathlineto{\pgfqpoint{2.847600in}{1.396677in}}%
\pgfpathlineto{\pgfqpoint{2.934400in}{1.407224in}}%
\pgfpathlineto{\pgfqpoint{2.971600in}{1.408633in}}%
\pgfpathlineto{\pgfqpoint{3.008800in}{1.407126in}}%
\pgfpathlineto{\pgfqpoint{3.046000in}{1.403180in}}%
\pgfpathlineto{\pgfqpoint{3.132800in}{1.392666in}}%
\pgfpathlineto{\pgfqpoint{3.170000in}{1.391344in}}%
\pgfpathlineto{\pgfqpoint{3.207200in}{1.392946in}}%
\pgfpathlineto{\pgfqpoint{3.244400in}{1.396963in}}%
\pgfpathlineto{\pgfqpoint{3.318800in}{1.406420in}}%
\pgfpathlineto{\pgfqpoint{3.356000in}{1.408589in}}%
\pgfpathlineto{\pgfqpoint{3.393200in}{1.407858in}}%
\pgfpathlineto{\pgfqpoint{3.430400in}{1.404453in}}%
\pgfpathlineto{\pgfqpoint{3.529600in}{1.392456in}}%
\pgfpathlineto{\pgfqpoint{3.566800in}{1.391307in}}%
\pgfpathlineto{\pgfqpoint{3.604000in}{1.393092in}}%
\pgfpathlineto{\pgfqpoint{3.641200in}{1.397246in}}%
\pgfpathlineto{\pgfqpoint{3.715600in}{1.406667in}}%
\pgfpathlineto{\pgfqpoint{3.752800in}{1.408682in}}%
\pgfpathlineto{\pgfqpoint{3.790000in}{1.407767in}}%
\pgfpathlineto{\pgfqpoint{3.827200in}{1.404208in}}%
\pgfpathlineto{\pgfqpoint{3.926400in}{1.392261in}}%
\pgfpathlineto{\pgfqpoint{3.963600in}{1.391280in}}%
\pgfpathlineto{\pgfqpoint{4.000800in}{1.393242in}}%
\pgfpathlineto{\pgfqpoint{4.050400in}{1.399233in}}%
\pgfpathlineto{\pgfqpoint{4.112400in}{1.406900in}}%
\pgfpathlineto{\pgfqpoint{4.149600in}{1.408763in}}%
\pgfpathlineto{\pgfqpoint{4.186800in}{1.407671in}}%
\pgfpathlineto{\pgfqpoint{4.224000in}{1.403966in}}%
\pgfpathlineto{\pgfqpoint{4.323200in}{1.392079in}}%
\pgfpathlineto{\pgfqpoint{4.360400in}{1.391262in}}%
\pgfpathlineto{\pgfqpoint{4.397600in}{1.393393in}}%
\pgfpathlineto{\pgfqpoint{4.447200in}{1.399525in}}%
\pgfpathlineto{\pgfqpoint{4.496800in}{1.405954in}}%
\pgfpathlineto{\pgfqpoint{4.534000in}{1.408585in}}%
\pgfpathlineto{\pgfqpoint{4.571200in}{1.408318in}}%
\pgfpathlineto{\pgfqpoint{4.608400in}{1.405233in}}%
\pgfpathlineto{\pgfqpoint{4.720000in}{1.391910in}}%
\pgfpathlineto{\pgfqpoint{4.757200in}{1.391252in}}%
\pgfpathlineto{\pgfqpoint{4.794400in}{1.393545in}}%
\pgfpathlineto{\pgfqpoint{4.844000in}{1.399807in}}%
\pgfpathlineto{\pgfqpoint{4.893600in}{1.406196in}}%
\pgfpathlineto{\pgfqpoint{4.930800in}{1.408699in}}%
\pgfpathlineto{\pgfqpoint{4.968000in}{1.408270in}}%
\pgfpathlineto{\pgfqpoint{5.005200in}{1.405038in}}%
\pgfpathlineto{\pgfqpoint{5.116800in}{1.391753in}}%
\pgfpathlineto{\pgfqpoint{5.154000in}{1.391249in}}%
\pgfpathlineto{\pgfqpoint{5.191200in}{1.393696in}}%
\pgfpathlineto{\pgfqpoint{5.240800in}{1.400079in}}%
\pgfpathlineto{\pgfqpoint{5.290400in}{1.406424in}}%
\pgfpathlineto{\pgfqpoint{5.327600in}{1.408804in}}%
\pgfpathlineto{\pgfqpoint{5.364800in}{1.408218in}}%
\pgfpathlineto{\pgfqpoint{5.402000in}{1.404846in}}%
\pgfpathlineto{\pgfqpoint{5.513600in}{1.391607in}}%
\pgfpathlineto{\pgfqpoint{5.550800in}{1.391251in}}%
\pgfpathlineto{\pgfqpoint{5.588000in}{1.393845in}}%
\pgfpathlineto{\pgfqpoint{5.637600in}{1.400340in}}%
\pgfpathlineto{\pgfqpoint{5.687200in}{1.406639in}}%
\pgfpathlineto{\pgfqpoint{5.724400in}{1.408899in}}%
\pgfpathlineto{\pgfqpoint{5.761600in}{1.408163in}}%
\pgfpathlineto{\pgfqpoint{5.798800in}{1.404659in}}%
\pgfpathlineto{\pgfqpoint{5.898000in}{1.392214in}}%
\pgfpathlineto{\pgfqpoint{5.935200in}{1.390984in}}%
\pgfpathlineto{\pgfqpoint{5.972400in}{1.392787in}}%
\pgfpathlineto{\pgfqpoint{6.009600in}{1.397055in}}%
\pgfpathlineto{\pgfqpoint{6.084000in}{1.406842in}}%
\pgfpathlineto{\pgfqpoint{6.121200in}{1.408986in}}%
\pgfpathlineto{\pgfqpoint{6.158400in}{1.408108in}}%
\pgfpathlineto{\pgfqpoint{6.195600in}{1.404479in}}%
\pgfpathlineto{\pgfqpoint{6.294800in}{1.392050in}}%
\pgfpathlineto{\pgfqpoint{6.332000in}{1.390947in}}%
\pgfpathlineto{\pgfqpoint{6.369200in}{1.392887in}}%
\pgfpathlineto{\pgfqpoint{6.406400in}{1.397261in}}%
\pgfpathlineto{\pgfqpoint{6.480800in}{1.407033in}}%
\pgfpathlineto{\pgfqpoint{6.518000in}{1.409065in}}%
\pgfpathlineto{\pgfqpoint{6.555200in}{1.408052in}}%
\pgfpathlineto{\pgfqpoint{6.592400in}{1.404306in}}%
\pgfpathlineto{\pgfqpoint{6.691600in}{1.391896in}}%
\pgfpathlineto{\pgfqpoint{6.728800in}{1.390913in}}%
\pgfpathlineto{\pgfqpoint{6.766000in}{1.392984in}}%
\pgfpathlineto{\pgfqpoint{6.803200in}{1.397457in}}%
\pgfpathlineto{\pgfqpoint{6.877600in}{1.407212in}}%
\pgfpathlineto{\pgfqpoint{6.914800in}{1.409139in}}%
\pgfpathlineto{\pgfqpoint{6.952000in}{1.407998in}}%
\pgfpathlineto{\pgfqpoint{6.989200in}{1.404141in}}%
\pgfpathlineto{\pgfqpoint{7.076000in}{1.392703in}}%
\pgfpathlineto{\pgfqpoint{7.113200in}{1.390826in}}%
\pgfpathlineto{\pgfqpoint{7.150400in}{1.392028in}}%
\pgfpathlineto{\pgfqpoint{7.187600in}{1.395938in}}%
\pgfpathlineto{\pgfqpoint{7.200000in}{1.397643in}}%
\pgfpathlineto{\pgfqpoint{7.200000in}{1.397643in}}%
\pgfusepath{stroke}%
\end{pgfscope}%
\begin{pgfscope}%
\pgfsetrectcap%
\pgfsetmiterjoin%
\pgfsetlinewidth{1.003750pt}%
\definecolor{currentstroke}{rgb}{0.000000,0.000000,0.000000}%
\pgfsetstrokecolor{currentstroke}%
\pgfsetdash{}{0pt}%
\pgfpathmoveto{\pgfqpoint{1.000000in}{0.600000in}}%
\pgfpathlineto{\pgfqpoint{1.000000in}{5.400000in}}%
\pgfusepath{stroke}%
\end{pgfscope}%
\begin{pgfscope}%
\pgfsetrectcap%
\pgfsetmiterjoin%
\pgfsetlinewidth{1.003750pt}%
\definecolor{currentstroke}{rgb}{0.000000,0.000000,0.000000}%
\pgfsetstrokecolor{currentstroke}%
\pgfsetdash{}{0pt}%
\pgfpathmoveto{\pgfqpoint{7.200000in}{0.600000in}}%
\pgfpathlineto{\pgfqpoint{7.200000in}{5.400000in}}%
\pgfusepath{stroke}%
\end{pgfscope}%
\begin{pgfscope}%
\pgfsetrectcap%
\pgfsetmiterjoin%
\pgfsetlinewidth{1.003750pt}%
\definecolor{currentstroke}{rgb}{0.000000,0.000000,0.000000}%
\pgfsetstrokecolor{currentstroke}%
\pgfsetdash{}{0pt}%
\pgfpathmoveto{\pgfqpoint{1.000000in}{0.600000in}}%
\pgfpathlineto{\pgfqpoint{7.200000in}{0.600000in}}%
\pgfusepath{stroke}%
\end{pgfscope}%
\begin{pgfscope}%
\pgfsetrectcap%
\pgfsetmiterjoin%
\pgfsetlinewidth{1.003750pt}%
\definecolor{currentstroke}{rgb}{0.000000,0.000000,0.000000}%
\pgfsetstrokecolor{currentstroke}%
\pgfsetdash{}{0pt}%
\pgfpathmoveto{\pgfqpoint{1.000000in}{5.400000in}}%
\pgfpathlineto{\pgfqpoint{7.200000in}{5.400000in}}%
\pgfusepath{stroke}%
\end{pgfscope}%
\begin{pgfscope}%
\pgfpathrectangle{\pgfqpoint{1.000000in}{0.600000in}}{\pgfqpoint{6.200000in}{4.800000in}}%
\pgfusepath{clip}%
\pgfsetbuttcap%
\pgfsetroundjoin%
\pgfsetlinewidth{0.501875pt}%
\definecolor{currentstroke}{rgb}{0.000000,0.000000,0.000000}%
\pgfsetstrokecolor{currentstroke}%
\pgfsetdash{{1.000000pt}{3.000000pt}}{0.000000pt}%
\pgfpathmoveto{\pgfqpoint{1.000000in}{0.600000in}}%
\pgfpathlineto{\pgfqpoint{1.000000in}{5.400000in}}%
\pgfusepath{stroke}%
\end{pgfscope}%
\begin{pgfscope}%
\pgfsetbuttcap%
\pgfsetroundjoin%
\definecolor{currentfill}{rgb}{0.000000,0.000000,0.000000}%
\pgfsetfillcolor{currentfill}%
\pgfsetlinewidth{0.501875pt}%
\definecolor{currentstroke}{rgb}{0.000000,0.000000,0.000000}%
\pgfsetstrokecolor{currentstroke}%
\pgfsetdash{}{0pt}%
\pgfsys@defobject{currentmarker}{\pgfqpoint{0.000000in}{0.000000in}}{\pgfqpoint{0.000000in}{0.055556in}}{%
\pgfpathmoveto{\pgfqpoint{0.000000in}{0.000000in}}%
\pgfpathlineto{\pgfqpoint{0.000000in}{0.055556in}}%
\pgfusepath{stroke,fill}%
}%
\begin{pgfscope}%
\pgfsys@transformshift{1.000000in}{0.600000in}%
\pgfsys@useobject{currentmarker}{}%
\end{pgfscope}%
\end{pgfscope}%
\begin{pgfscope}%
\pgfsetbuttcap%
\pgfsetroundjoin%
\definecolor{currentfill}{rgb}{0.000000,0.000000,0.000000}%
\pgfsetfillcolor{currentfill}%
\pgfsetlinewidth{0.501875pt}%
\definecolor{currentstroke}{rgb}{0.000000,0.000000,0.000000}%
\pgfsetstrokecolor{currentstroke}%
\pgfsetdash{}{0pt}%
\pgfsys@defobject{currentmarker}{\pgfqpoint{0.000000in}{-0.055556in}}{\pgfqpoint{0.000000in}{0.000000in}}{%
\pgfpathmoveto{\pgfqpoint{0.000000in}{0.000000in}}%
\pgfpathlineto{\pgfqpoint{0.000000in}{-0.055556in}}%
\pgfusepath{stroke,fill}%
}%
\begin{pgfscope}%
\pgfsys@transformshift{1.000000in}{5.400000in}%
\pgfsys@useobject{currentmarker}{}%
\end{pgfscope}%
\end{pgfscope}%
\begin{pgfscope}%
\definecolor{textcolor}{rgb}{0.000000,0.000000,0.000000}%
\pgfsetstrokecolor{textcolor}%
\pgfsetfillcolor{textcolor}%
\pgftext[x=1.000000in,y=0.544444in,,top]{\color{textcolor}\rmfamily\fontsize{10.000000}{12.000000}\selectfont \(\displaystyle {0}\)}%
\end{pgfscope}%
\begin{pgfscope}%
\pgfpathrectangle{\pgfqpoint{1.000000in}{0.600000in}}{\pgfqpoint{6.200000in}{4.800000in}}%
\pgfusepath{clip}%
\pgfsetbuttcap%
\pgfsetroundjoin%
\pgfsetlinewidth{0.501875pt}%
\definecolor{currentstroke}{rgb}{0.000000,0.000000,0.000000}%
\pgfsetstrokecolor{currentstroke}%
\pgfsetdash{{1.000000pt}{3.000000pt}}{0.000000pt}%
\pgfpathmoveto{\pgfqpoint{2.240000in}{0.600000in}}%
\pgfpathlineto{\pgfqpoint{2.240000in}{5.400000in}}%
\pgfusepath{stroke}%
\end{pgfscope}%
\begin{pgfscope}%
\pgfsetbuttcap%
\pgfsetroundjoin%
\definecolor{currentfill}{rgb}{0.000000,0.000000,0.000000}%
\pgfsetfillcolor{currentfill}%
\pgfsetlinewidth{0.501875pt}%
\definecolor{currentstroke}{rgb}{0.000000,0.000000,0.000000}%
\pgfsetstrokecolor{currentstroke}%
\pgfsetdash{}{0pt}%
\pgfsys@defobject{currentmarker}{\pgfqpoint{0.000000in}{0.000000in}}{\pgfqpoint{0.000000in}{0.055556in}}{%
\pgfpathmoveto{\pgfqpoint{0.000000in}{0.000000in}}%
\pgfpathlineto{\pgfqpoint{0.000000in}{0.055556in}}%
\pgfusepath{stroke,fill}%
}%
\begin{pgfscope}%
\pgfsys@transformshift{2.240000in}{0.600000in}%
\pgfsys@useobject{currentmarker}{}%
\end{pgfscope}%
\end{pgfscope}%
\begin{pgfscope}%
\pgfsetbuttcap%
\pgfsetroundjoin%
\definecolor{currentfill}{rgb}{0.000000,0.000000,0.000000}%
\pgfsetfillcolor{currentfill}%
\pgfsetlinewidth{0.501875pt}%
\definecolor{currentstroke}{rgb}{0.000000,0.000000,0.000000}%
\pgfsetstrokecolor{currentstroke}%
\pgfsetdash{}{0pt}%
\pgfsys@defobject{currentmarker}{\pgfqpoint{0.000000in}{-0.055556in}}{\pgfqpoint{0.000000in}{0.000000in}}{%
\pgfpathmoveto{\pgfqpoint{0.000000in}{0.000000in}}%
\pgfpathlineto{\pgfqpoint{0.000000in}{-0.055556in}}%
\pgfusepath{stroke,fill}%
}%
\begin{pgfscope}%
\pgfsys@transformshift{2.240000in}{5.400000in}%
\pgfsys@useobject{currentmarker}{}%
\end{pgfscope}%
\end{pgfscope}%
\begin{pgfscope}%
\definecolor{textcolor}{rgb}{0.000000,0.000000,0.000000}%
\pgfsetstrokecolor{textcolor}%
\pgfsetfillcolor{textcolor}%
\pgftext[x=2.240000in,y=0.544444in,,top]{\color{textcolor}\rmfamily\fontsize{10.000000}{12.000000}\selectfont \(\displaystyle {20}\)}%
\end{pgfscope}%
\begin{pgfscope}%
\pgfpathrectangle{\pgfqpoint{1.000000in}{0.600000in}}{\pgfqpoint{6.200000in}{4.800000in}}%
\pgfusepath{clip}%
\pgfsetbuttcap%
\pgfsetroundjoin%
\pgfsetlinewidth{0.501875pt}%
\definecolor{currentstroke}{rgb}{0.000000,0.000000,0.000000}%
\pgfsetstrokecolor{currentstroke}%
\pgfsetdash{{1.000000pt}{3.000000pt}}{0.000000pt}%
\pgfpathmoveto{\pgfqpoint{3.480000in}{0.600000in}}%
\pgfpathlineto{\pgfqpoint{3.480000in}{5.400000in}}%
\pgfusepath{stroke}%
\end{pgfscope}%
\begin{pgfscope}%
\pgfsetbuttcap%
\pgfsetroundjoin%
\definecolor{currentfill}{rgb}{0.000000,0.000000,0.000000}%
\pgfsetfillcolor{currentfill}%
\pgfsetlinewidth{0.501875pt}%
\definecolor{currentstroke}{rgb}{0.000000,0.000000,0.000000}%
\pgfsetstrokecolor{currentstroke}%
\pgfsetdash{}{0pt}%
\pgfsys@defobject{currentmarker}{\pgfqpoint{0.000000in}{0.000000in}}{\pgfqpoint{0.000000in}{0.055556in}}{%
\pgfpathmoveto{\pgfqpoint{0.000000in}{0.000000in}}%
\pgfpathlineto{\pgfqpoint{0.000000in}{0.055556in}}%
\pgfusepath{stroke,fill}%
}%
\begin{pgfscope}%
\pgfsys@transformshift{3.480000in}{0.600000in}%
\pgfsys@useobject{currentmarker}{}%
\end{pgfscope}%
\end{pgfscope}%
\begin{pgfscope}%
\pgfsetbuttcap%
\pgfsetroundjoin%
\definecolor{currentfill}{rgb}{0.000000,0.000000,0.000000}%
\pgfsetfillcolor{currentfill}%
\pgfsetlinewidth{0.501875pt}%
\definecolor{currentstroke}{rgb}{0.000000,0.000000,0.000000}%
\pgfsetstrokecolor{currentstroke}%
\pgfsetdash{}{0pt}%
\pgfsys@defobject{currentmarker}{\pgfqpoint{0.000000in}{-0.055556in}}{\pgfqpoint{0.000000in}{0.000000in}}{%
\pgfpathmoveto{\pgfqpoint{0.000000in}{0.000000in}}%
\pgfpathlineto{\pgfqpoint{0.000000in}{-0.055556in}}%
\pgfusepath{stroke,fill}%
}%
\begin{pgfscope}%
\pgfsys@transformshift{3.480000in}{5.400000in}%
\pgfsys@useobject{currentmarker}{}%
\end{pgfscope}%
\end{pgfscope}%
\begin{pgfscope}%
\definecolor{textcolor}{rgb}{0.000000,0.000000,0.000000}%
\pgfsetstrokecolor{textcolor}%
\pgfsetfillcolor{textcolor}%
\pgftext[x=3.480000in,y=0.544444in,,top]{\color{textcolor}\rmfamily\fontsize{10.000000}{12.000000}\selectfont \(\displaystyle {40}\)}%
\end{pgfscope}%
\begin{pgfscope}%
\pgfpathrectangle{\pgfqpoint{1.000000in}{0.600000in}}{\pgfqpoint{6.200000in}{4.800000in}}%
\pgfusepath{clip}%
\pgfsetbuttcap%
\pgfsetroundjoin%
\pgfsetlinewidth{0.501875pt}%
\definecolor{currentstroke}{rgb}{0.000000,0.000000,0.000000}%
\pgfsetstrokecolor{currentstroke}%
\pgfsetdash{{1.000000pt}{3.000000pt}}{0.000000pt}%
\pgfpathmoveto{\pgfqpoint{4.720000in}{0.600000in}}%
\pgfpathlineto{\pgfqpoint{4.720000in}{5.400000in}}%
\pgfusepath{stroke}%
\end{pgfscope}%
\begin{pgfscope}%
\pgfsetbuttcap%
\pgfsetroundjoin%
\definecolor{currentfill}{rgb}{0.000000,0.000000,0.000000}%
\pgfsetfillcolor{currentfill}%
\pgfsetlinewidth{0.501875pt}%
\definecolor{currentstroke}{rgb}{0.000000,0.000000,0.000000}%
\pgfsetstrokecolor{currentstroke}%
\pgfsetdash{}{0pt}%
\pgfsys@defobject{currentmarker}{\pgfqpoint{0.000000in}{0.000000in}}{\pgfqpoint{0.000000in}{0.055556in}}{%
\pgfpathmoveto{\pgfqpoint{0.000000in}{0.000000in}}%
\pgfpathlineto{\pgfqpoint{0.000000in}{0.055556in}}%
\pgfusepath{stroke,fill}%
}%
\begin{pgfscope}%
\pgfsys@transformshift{4.720000in}{0.600000in}%
\pgfsys@useobject{currentmarker}{}%
\end{pgfscope}%
\end{pgfscope}%
\begin{pgfscope}%
\pgfsetbuttcap%
\pgfsetroundjoin%
\definecolor{currentfill}{rgb}{0.000000,0.000000,0.000000}%
\pgfsetfillcolor{currentfill}%
\pgfsetlinewidth{0.501875pt}%
\definecolor{currentstroke}{rgb}{0.000000,0.000000,0.000000}%
\pgfsetstrokecolor{currentstroke}%
\pgfsetdash{}{0pt}%
\pgfsys@defobject{currentmarker}{\pgfqpoint{0.000000in}{-0.055556in}}{\pgfqpoint{0.000000in}{0.000000in}}{%
\pgfpathmoveto{\pgfqpoint{0.000000in}{0.000000in}}%
\pgfpathlineto{\pgfqpoint{0.000000in}{-0.055556in}}%
\pgfusepath{stroke,fill}%
}%
\begin{pgfscope}%
\pgfsys@transformshift{4.720000in}{5.400000in}%
\pgfsys@useobject{currentmarker}{}%
\end{pgfscope}%
\end{pgfscope}%
\begin{pgfscope}%
\definecolor{textcolor}{rgb}{0.000000,0.000000,0.000000}%
\pgfsetstrokecolor{textcolor}%
\pgfsetfillcolor{textcolor}%
\pgftext[x=4.720000in,y=0.544444in,,top]{\color{textcolor}\rmfamily\fontsize{10.000000}{12.000000}\selectfont \(\displaystyle {60}\)}%
\end{pgfscope}%
\begin{pgfscope}%
\pgfpathrectangle{\pgfqpoint{1.000000in}{0.600000in}}{\pgfqpoint{6.200000in}{4.800000in}}%
\pgfusepath{clip}%
\pgfsetbuttcap%
\pgfsetroundjoin%
\pgfsetlinewidth{0.501875pt}%
\definecolor{currentstroke}{rgb}{0.000000,0.000000,0.000000}%
\pgfsetstrokecolor{currentstroke}%
\pgfsetdash{{1.000000pt}{3.000000pt}}{0.000000pt}%
\pgfpathmoveto{\pgfqpoint{5.960000in}{0.600000in}}%
\pgfpathlineto{\pgfqpoint{5.960000in}{5.400000in}}%
\pgfusepath{stroke}%
\end{pgfscope}%
\begin{pgfscope}%
\pgfsetbuttcap%
\pgfsetroundjoin%
\definecolor{currentfill}{rgb}{0.000000,0.000000,0.000000}%
\pgfsetfillcolor{currentfill}%
\pgfsetlinewidth{0.501875pt}%
\definecolor{currentstroke}{rgb}{0.000000,0.000000,0.000000}%
\pgfsetstrokecolor{currentstroke}%
\pgfsetdash{}{0pt}%
\pgfsys@defobject{currentmarker}{\pgfqpoint{0.000000in}{0.000000in}}{\pgfqpoint{0.000000in}{0.055556in}}{%
\pgfpathmoveto{\pgfqpoint{0.000000in}{0.000000in}}%
\pgfpathlineto{\pgfqpoint{0.000000in}{0.055556in}}%
\pgfusepath{stroke,fill}%
}%
\begin{pgfscope}%
\pgfsys@transformshift{5.960000in}{0.600000in}%
\pgfsys@useobject{currentmarker}{}%
\end{pgfscope}%
\end{pgfscope}%
\begin{pgfscope}%
\pgfsetbuttcap%
\pgfsetroundjoin%
\definecolor{currentfill}{rgb}{0.000000,0.000000,0.000000}%
\pgfsetfillcolor{currentfill}%
\pgfsetlinewidth{0.501875pt}%
\definecolor{currentstroke}{rgb}{0.000000,0.000000,0.000000}%
\pgfsetstrokecolor{currentstroke}%
\pgfsetdash{}{0pt}%
\pgfsys@defobject{currentmarker}{\pgfqpoint{0.000000in}{-0.055556in}}{\pgfqpoint{0.000000in}{0.000000in}}{%
\pgfpathmoveto{\pgfqpoint{0.000000in}{0.000000in}}%
\pgfpathlineto{\pgfqpoint{0.000000in}{-0.055556in}}%
\pgfusepath{stroke,fill}%
}%
\begin{pgfscope}%
\pgfsys@transformshift{5.960000in}{5.400000in}%
\pgfsys@useobject{currentmarker}{}%
\end{pgfscope}%
\end{pgfscope}%
\begin{pgfscope}%
\definecolor{textcolor}{rgb}{0.000000,0.000000,0.000000}%
\pgfsetstrokecolor{textcolor}%
\pgfsetfillcolor{textcolor}%
\pgftext[x=5.960000in,y=0.544444in,,top]{\color{textcolor}\rmfamily\fontsize{10.000000}{12.000000}\selectfont \(\displaystyle {80}\)}%
\end{pgfscope}%
\begin{pgfscope}%
\pgfpathrectangle{\pgfqpoint{1.000000in}{0.600000in}}{\pgfqpoint{6.200000in}{4.800000in}}%
\pgfusepath{clip}%
\pgfsetbuttcap%
\pgfsetroundjoin%
\pgfsetlinewidth{0.501875pt}%
\definecolor{currentstroke}{rgb}{0.000000,0.000000,0.000000}%
\pgfsetstrokecolor{currentstroke}%
\pgfsetdash{{1.000000pt}{3.000000pt}}{0.000000pt}%
\pgfpathmoveto{\pgfqpoint{7.200000in}{0.600000in}}%
\pgfpathlineto{\pgfqpoint{7.200000in}{5.400000in}}%
\pgfusepath{stroke}%
\end{pgfscope}%
\begin{pgfscope}%
\pgfsetbuttcap%
\pgfsetroundjoin%
\definecolor{currentfill}{rgb}{0.000000,0.000000,0.000000}%
\pgfsetfillcolor{currentfill}%
\pgfsetlinewidth{0.501875pt}%
\definecolor{currentstroke}{rgb}{0.000000,0.000000,0.000000}%
\pgfsetstrokecolor{currentstroke}%
\pgfsetdash{}{0pt}%
\pgfsys@defobject{currentmarker}{\pgfqpoint{0.000000in}{0.000000in}}{\pgfqpoint{0.000000in}{0.055556in}}{%
\pgfpathmoveto{\pgfqpoint{0.000000in}{0.000000in}}%
\pgfpathlineto{\pgfqpoint{0.000000in}{0.055556in}}%
\pgfusepath{stroke,fill}%
}%
\begin{pgfscope}%
\pgfsys@transformshift{7.200000in}{0.600000in}%
\pgfsys@useobject{currentmarker}{}%
\end{pgfscope}%
\end{pgfscope}%
\begin{pgfscope}%
\pgfsetbuttcap%
\pgfsetroundjoin%
\definecolor{currentfill}{rgb}{0.000000,0.000000,0.000000}%
\pgfsetfillcolor{currentfill}%
\pgfsetlinewidth{0.501875pt}%
\definecolor{currentstroke}{rgb}{0.000000,0.000000,0.000000}%
\pgfsetstrokecolor{currentstroke}%
\pgfsetdash{}{0pt}%
\pgfsys@defobject{currentmarker}{\pgfqpoint{0.000000in}{-0.055556in}}{\pgfqpoint{0.000000in}{0.000000in}}{%
\pgfpathmoveto{\pgfqpoint{0.000000in}{0.000000in}}%
\pgfpathlineto{\pgfqpoint{0.000000in}{-0.055556in}}%
\pgfusepath{stroke,fill}%
}%
\begin{pgfscope}%
\pgfsys@transformshift{7.200000in}{5.400000in}%
\pgfsys@useobject{currentmarker}{}%
\end{pgfscope}%
\end{pgfscope}%
\begin{pgfscope}%
\definecolor{textcolor}{rgb}{0.000000,0.000000,0.000000}%
\pgfsetstrokecolor{textcolor}%
\pgfsetfillcolor{textcolor}%
\pgftext[x=7.200000in,y=0.544444in,,top]{\color{textcolor}\rmfamily\fontsize{10.000000}{12.000000}\selectfont \(\displaystyle {100}\)}%
\end{pgfscope}%
\begin{pgfscope}%
\definecolor{textcolor}{rgb}{0.000000,0.000000,0.000000}%
\pgfsetstrokecolor{textcolor}%
\pgfsetfillcolor{textcolor}%
\pgftext[x=4.100000in,y=0.351543in,,top]{\color{textcolor}\rmfamily\fontsize{12.000000}{14.400000}\selectfont \(\displaystyle time\ (s)\)}%
\end{pgfscope}%
\begin{pgfscope}%
\pgfpathrectangle{\pgfqpoint{1.000000in}{0.600000in}}{\pgfqpoint{6.200000in}{4.800000in}}%
\pgfusepath{clip}%
\pgfsetbuttcap%
\pgfsetroundjoin%
\pgfsetlinewidth{0.501875pt}%
\definecolor{currentstroke}{rgb}{0.000000,0.000000,0.000000}%
\pgfsetstrokecolor{currentstroke}%
\pgfsetdash{{1.000000pt}{3.000000pt}}{0.000000pt}%
\pgfpathmoveto{\pgfqpoint{1.000000in}{0.600000in}}%
\pgfpathlineto{\pgfqpoint{7.200000in}{0.600000in}}%
\pgfusepath{stroke}%
\end{pgfscope}%
\begin{pgfscope}%
\pgfsetbuttcap%
\pgfsetroundjoin%
\definecolor{currentfill}{rgb}{0.000000,0.000000,0.000000}%
\pgfsetfillcolor{currentfill}%
\pgfsetlinewidth{0.501875pt}%
\definecolor{currentstroke}{rgb}{0.000000,0.000000,0.000000}%
\pgfsetstrokecolor{currentstroke}%
\pgfsetdash{}{0pt}%
\pgfsys@defobject{currentmarker}{\pgfqpoint{0.000000in}{0.000000in}}{\pgfqpoint{0.055556in}{0.000000in}}{%
\pgfpathmoveto{\pgfqpoint{0.000000in}{0.000000in}}%
\pgfpathlineto{\pgfqpoint{0.055556in}{0.000000in}}%
\pgfusepath{stroke,fill}%
}%
\begin{pgfscope}%
\pgfsys@transformshift{1.000000in}{0.600000in}%
\pgfsys@useobject{currentmarker}{}%
\end{pgfscope}%
\end{pgfscope}%
\begin{pgfscope}%
\pgfsetbuttcap%
\pgfsetroundjoin%
\definecolor{currentfill}{rgb}{0.000000,0.000000,0.000000}%
\pgfsetfillcolor{currentfill}%
\pgfsetlinewidth{0.501875pt}%
\definecolor{currentstroke}{rgb}{0.000000,0.000000,0.000000}%
\pgfsetstrokecolor{currentstroke}%
\pgfsetdash{}{0pt}%
\pgfsys@defobject{currentmarker}{\pgfqpoint{-0.055556in}{0.000000in}}{\pgfqpoint{-0.000000in}{0.000000in}}{%
\pgfpathmoveto{\pgfqpoint{-0.000000in}{0.000000in}}%
\pgfpathlineto{\pgfqpoint{-0.055556in}{0.000000in}}%
\pgfusepath{stroke,fill}%
}%
\begin{pgfscope}%
\pgfsys@transformshift{7.200000in}{0.600000in}%
\pgfsys@useobject{currentmarker}{}%
\end{pgfscope}%
\end{pgfscope}%
\begin{pgfscope}%
\definecolor{textcolor}{rgb}{0.000000,0.000000,0.000000}%
\pgfsetstrokecolor{textcolor}%
\pgfsetfillcolor{textcolor}%
\pgftext[x=0.944444in,y=0.600000in,right,]{\color{textcolor}\rmfamily\fontsize{10.000000}{12.000000}\selectfont \(\displaystyle {\ensuremath{-}50}\)}%
\end{pgfscope}%
\begin{pgfscope}%
\pgfpathrectangle{\pgfqpoint{1.000000in}{0.600000in}}{\pgfqpoint{6.200000in}{4.800000in}}%
\pgfusepath{clip}%
\pgfsetbuttcap%
\pgfsetroundjoin%
\pgfsetlinewidth{0.501875pt}%
\definecolor{currentstroke}{rgb}{0.000000,0.000000,0.000000}%
\pgfsetstrokecolor{currentstroke}%
\pgfsetdash{{1.000000pt}{3.000000pt}}{0.000000pt}%
\pgfpathmoveto{\pgfqpoint{1.000000in}{1.400000in}}%
\pgfpathlineto{\pgfqpoint{7.200000in}{1.400000in}}%
\pgfusepath{stroke}%
\end{pgfscope}%
\begin{pgfscope}%
\pgfsetbuttcap%
\pgfsetroundjoin%
\definecolor{currentfill}{rgb}{0.000000,0.000000,0.000000}%
\pgfsetfillcolor{currentfill}%
\pgfsetlinewidth{0.501875pt}%
\definecolor{currentstroke}{rgb}{0.000000,0.000000,0.000000}%
\pgfsetstrokecolor{currentstroke}%
\pgfsetdash{}{0pt}%
\pgfsys@defobject{currentmarker}{\pgfqpoint{0.000000in}{0.000000in}}{\pgfqpoint{0.055556in}{0.000000in}}{%
\pgfpathmoveto{\pgfqpoint{0.000000in}{0.000000in}}%
\pgfpathlineto{\pgfqpoint{0.055556in}{0.000000in}}%
\pgfusepath{stroke,fill}%
}%
\begin{pgfscope}%
\pgfsys@transformshift{1.000000in}{1.400000in}%
\pgfsys@useobject{currentmarker}{}%
\end{pgfscope}%
\end{pgfscope}%
\begin{pgfscope}%
\pgfsetbuttcap%
\pgfsetroundjoin%
\definecolor{currentfill}{rgb}{0.000000,0.000000,0.000000}%
\pgfsetfillcolor{currentfill}%
\pgfsetlinewidth{0.501875pt}%
\definecolor{currentstroke}{rgb}{0.000000,0.000000,0.000000}%
\pgfsetstrokecolor{currentstroke}%
\pgfsetdash{}{0pt}%
\pgfsys@defobject{currentmarker}{\pgfqpoint{-0.055556in}{0.000000in}}{\pgfqpoint{-0.000000in}{0.000000in}}{%
\pgfpathmoveto{\pgfqpoint{-0.000000in}{0.000000in}}%
\pgfpathlineto{\pgfqpoint{-0.055556in}{0.000000in}}%
\pgfusepath{stroke,fill}%
}%
\begin{pgfscope}%
\pgfsys@transformshift{7.200000in}{1.400000in}%
\pgfsys@useobject{currentmarker}{}%
\end{pgfscope}%
\end{pgfscope}%
\begin{pgfscope}%
\definecolor{textcolor}{rgb}{0.000000,0.000000,0.000000}%
\pgfsetstrokecolor{textcolor}%
\pgfsetfillcolor{textcolor}%
\pgftext[x=0.944444in,y=1.400000in,right,]{\color{textcolor}\rmfamily\fontsize{10.000000}{12.000000}\selectfont \(\displaystyle {0}\)}%
\end{pgfscope}%
\begin{pgfscope}%
\pgfpathrectangle{\pgfqpoint{1.000000in}{0.600000in}}{\pgfqpoint{6.200000in}{4.800000in}}%
\pgfusepath{clip}%
\pgfsetbuttcap%
\pgfsetroundjoin%
\pgfsetlinewidth{0.501875pt}%
\definecolor{currentstroke}{rgb}{0.000000,0.000000,0.000000}%
\pgfsetstrokecolor{currentstroke}%
\pgfsetdash{{1.000000pt}{3.000000pt}}{0.000000pt}%
\pgfpathmoveto{\pgfqpoint{1.000000in}{2.200000in}}%
\pgfpathlineto{\pgfqpoint{7.200000in}{2.200000in}}%
\pgfusepath{stroke}%
\end{pgfscope}%
\begin{pgfscope}%
\pgfsetbuttcap%
\pgfsetroundjoin%
\definecolor{currentfill}{rgb}{0.000000,0.000000,0.000000}%
\pgfsetfillcolor{currentfill}%
\pgfsetlinewidth{0.501875pt}%
\definecolor{currentstroke}{rgb}{0.000000,0.000000,0.000000}%
\pgfsetstrokecolor{currentstroke}%
\pgfsetdash{}{0pt}%
\pgfsys@defobject{currentmarker}{\pgfqpoint{0.000000in}{0.000000in}}{\pgfqpoint{0.055556in}{0.000000in}}{%
\pgfpathmoveto{\pgfqpoint{0.000000in}{0.000000in}}%
\pgfpathlineto{\pgfqpoint{0.055556in}{0.000000in}}%
\pgfusepath{stroke,fill}%
}%
\begin{pgfscope}%
\pgfsys@transformshift{1.000000in}{2.200000in}%
\pgfsys@useobject{currentmarker}{}%
\end{pgfscope}%
\end{pgfscope}%
\begin{pgfscope}%
\pgfsetbuttcap%
\pgfsetroundjoin%
\definecolor{currentfill}{rgb}{0.000000,0.000000,0.000000}%
\pgfsetfillcolor{currentfill}%
\pgfsetlinewidth{0.501875pt}%
\definecolor{currentstroke}{rgb}{0.000000,0.000000,0.000000}%
\pgfsetstrokecolor{currentstroke}%
\pgfsetdash{}{0pt}%
\pgfsys@defobject{currentmarker}{\pgfqpoint{-0.055556in}{0.000000in}}{\pgfqpoint{-0.000000in}{0.000000in}}{%
\pgfpathmoveto{\pgfqpoint{-0.000000in}{0.000000in}}%
\pgfpathlineto{\pgfqpoint{-0.055556in}{0.000000in}}%
\pgfusepath{stroke,fill}%
}%
\begin{pgfscope}%
\pgfsys@transformshift{7.200000in}{2.200000in}%
\pgfsys@useobject{currentmarker}{}%
\end{pgfscope}%
\end{pgfscope}%
\begin{pgfscope}%
\definecolor{textcolor}{rgb}{0.000000,0.000000,0.000000}%
\pgfsetstrokecolor{textcolor}%
\pgfsetfillcolor{textcolor}%
\pgftext[x=0.944444in,y=2.200000in,right,]{\color{textcolor}\rmfamily\fontsize{10.000000}{12.000000}\selectfont \(\displaystyle {50}\)}%
\end{pgfscope}%
\begin{pgfscope}%
\pgfpathrectangle{\pgfqpoint{1.000000in}{0.600000in}}{\pgfqpoint{6.200000in}{4.800000in}}%
\pgfusepath{clip}%
\pgfsetbuttcap%
\pgfsetroundjoin%
\pgfsetlinewidth{0.501875pt}%
\definecolor{currentstroke}{rgb}{0.000000,0.000000,0.000000}%
\pgfsetstrokecolor{currentstroke}%
\pgfsetdash{{1.000000pt}{3.000000pt}}{0.000000pt}%
\pgfpathmoveto{\pgfqpoint{1.000000in}{3.000000in}}%
\pgfpathlineto{\pgfqpoint{7.200000in}{3.000000in}}%
\pgfusepath{stroke}%
\end{pgfscope}%
\begin{pgfscope}%
\pgfsetbuttcap%
\pgfsetroundjoin%
\definecolor{currentfill}{rgb}{0.000000,0.000000,0.000000}%
\pgfsetfillcolor{currentfill}%
\pgfsetlinewidth{0.501875pt}%
\definecolor{currentstroke}{rgb}{0.000000,0.000000,0.000000}%
\pgfsetstrokecolor{currentstroke}%
\pgfsetdash{}{0pt}%
\pgfsys@defobject{currentmarker}{\pgfqpoint{0.000000in}{0.000000in}}{\pgfqpoint{0.055556in}{0.000000in}}{%
\pgfpathmoveto{\pgfqpoint{0.000000in}{0.000000in}}%
\pgfpathlineto{\pgfqpoint{0.055556in}{0.000000in}}%
\pgfusepath{stroke,fill}%
}%
\begin{pgfscope}%
\pgfsys@transformshift{1.000000in}{3.000000in}%
\pgfsys@useobject{currentmarker}{}%
\end{pgfscope}%
\end{pgfscope}%
\begin{pgfscope}%
\pgfsetbuttcap%
\pgfsetroundjoin%
\definecolor{currentfill}{rgb}{0.000000,0.000000,0.000000}%
\pgfsetfillcolor{currentfill}%
\pgfsetlinewidth{0.501875pt}%
\definecolor{currentstroke}{rgb}{0.000000,0.000000,0.000000}%
\pgfsetstrokecolor{currentstroke}%
\pgfsetdash{}{0pt}%
\pgfsys@defobject{currentmarker}{\pgfqpoint{-0.055556in}{0.000000in}}{\pgfqpoint{-0.000000in}{0.000000in}}{%
\pgfpathmoveto{\pgfqpoint{-0.000000in}{0.000000in}}%
\pgfpathlineto{\pgfqpoint{-0.055556in}{0.000000in}}%
\pgfusepath{stroke,fill}%
}%
\begin{pgfscope}%
\pgfsys@transformshift{7.200000in}{3.000000in}%
\pgfsys@useobject{currentmarker}{}%
\end{pgfscope}%
\end{pgfscope}%
\begin{pgfscope}%
\definecolor{textcolor}{rgb}{0.000000,0.000000,0.000000}%
\pgfsetstrokecolor{textcolor}%
\pgfsetfillcolor{textcolor}%
\pgftext[x=0.944444in,y=3.000000in,right,]{\color{textcolor}\rmfamily\fontsize{10.000000}{12.000000}\selectfont \(\displaystyle {100}\)}%
\end{pgfscope}%
\begin{pgfscope}%
\pgfpathrectangle{\pgfqpoint{1.000000in}{0.600000in}}{\pgfqpoint{6.200000in}{4.800000in}}%
\pgfusepath{clip}%
\pgfsetbuttcap%
\pgfsetroundjoin%
\pgfsetlinewidth{0.501875pt}%
\definecolor{currentstroke}{rgb}{0.000000,0.000000,0.000000}%
\pgfsetstrokecolor{currentstroke}%
\pgfsetdash{{1.000000pt}{3.000000pt}}{0.000000pt}%
\pgfpathmoveto{\pgfqpoint{1.000000in}{3.800000in}}%
\pgfpathlineto{\pgfqpoint{7.200000in}{3.800000in}}%
\pgfusepath{stroke}%
\end{pgfscope}%
\begin{pgfscope}%
\pgfsetbuttcap%
\pgfsetroundjoin%
\definecolor{currentfill}{rgb}{0.000000,0.000000,0.000000}%
\pgfsetfillcolor{currentfill}%
\pgfsetlinewidth{0.501875pt}%
\definecolor{currentstroke}{rgb}{0.000000,0.000000,0.000000}%
\pgfsetstrokecolor{currentstroke}%
\pgfsetdash{}{0pt}%
\pgfsys@defobject{currentmarker}{\pgfqpoint{0.000000in}{0.000000in}}{\pgfqpoint{0.055556in}{0.000000in}}{%
\pgfpathmoveto{\pgfqpoint{0.000000in}{0.000000in}}%
\pgfpathlineto{\pgfqpoint{0.055556in}{0.000000in}}%
\pgfusepath{stroke,fill}%
}%
\begin{pgfscope}%
\pgfsys@transformshift{1.000000in}{3.800000in}%
\pgfsys@useobject{currentmarker}{}%
\end{pgfscope}%
\end{pgfscope}%
\begin{pgfscope}%
\pgfsetbuttcap%
\pgfsetroundjoin%
\definecolor{currentfill}{rgb}{0.000000,0.000000,0.000000}%
\pgfsetfillcolor{currentfill}%
\pgfsetlinewidth{0.501875pt}%
\definecolor{currentstroke}{rgb}{0.000000,0.000000,0.000000}%
\pgfsetstrokecolor{currentstroke}%
\pgfsetdash{}{0pt}%
\pgfsys@defobject{currentmarker}{\pgfqpoint{-0.055556in}{0.000000in}}{\pgfqpoint{-0.000000in}{0.000000in}}{%
\pgfpathmoveto{\pgfqpoint{-0.000000in}{0.000000in}}%
\pgfpathlineto{\pgfqpoint{-0.055556in}{0.000000in}}%
\pgfusepath{stroke,fill}%
}%
\begin{pgfscope}%
\pgfsys@transformshift{7.200000in}{3.800000in}%
\pgfsys@useobject{currentmarker}{}%
\end{pgfscope}%
\end{pgfscope}%
\begin{pgfscope}%
\definecolor{textcolor}{rgb}{0.000000,0.000000,0.000000}%
\pgfsetstrokecolor{textcolor}%
\pgfsetfillcolor{textcolor}%
\pgftext[x=0.944444in,y=3.800000in,right,]{\color{textcolor}\rmfamily\fontsize{10.000000}{12.000000}\selectfont \(\displaystyle {150}\)}%
\end{pgfscope}%
\begin{pgfscope}%
\pgfpathrectangle{\pgfqpoint{1.000000in}{0.600000in}}{\pgfqpoint{6.200000in}{4.800000in}}%
\pgfusepath{clip}%
\pgfsetbuttcap%
\pgfsetroundjoin%
\pgfsetlinewidth{0.501875pt}%
\definecolor{currentstroke}{rgb}{0.000000,0.000000,0.000000}%
\pgfsetstrokecolor{currentstroke}%
\pgfsetdash{{1.000000pt}{3.000000pt}}{0.000000pt}%
\pgfpathmoveto{\pgfqpoint{1.000000in}{4.600000in}}%
\pgfpathlineto{\pgfqpoint{7.200000in}{4.600000in}}%
\pgfusepath{stroke}%
\end{pgfscope}%
\begin{pgfscope}%
\pgfsetbuttcap%
\pgfsetroundjoin%
\definecolor{currentfill}{rgb}{0.000000,0.000000,0.000000}%
\pgfsetfillcolor{currentfill}%
\pgfsetlinewidth{0.501875pt}%
\definecolor{currentstroke}{rgb}{0.000000,0.000000,0.000000}%
\pgfsetstrokecolor{currentstroke}%
\pgfsetdash{}{0pt}%
\pgfsys@defobject{currentmarker}{\pgfqpoint{0.000000in}{0.000000in}}{\pgfqpoint{0.055556in}{0.000000in}}{%
\pgfpathmoveto{\pgfqpoint{0.000000in}{0.000000in}}%
\pgfpathlineto{\pgfqpoint{0.055556in}{0.000000in}}%
\pgfusepath{stroke,fill}%
}%
\begin{pgfscope}%
\pgfsys@transformshift{1.000000in}{4.600000in}%
\pgfsys@useobject{currentmarker}{}%
\end{pgfscope}%
\end{pgfscope}%
\begin{pgfscope}%
\pgfsetbuttcap%
\pgfsetroundjoin%
\definecolor{currentfill}{rgb}{0.000000,0.000000,0.000000}%
\pgfsetfillcolor{currentfill}%
\pgfsetlinewidth{0.501875pt}%
\definecolor{currentstroke}{rgb}{0.000000,0.000000,0.000000}%
\pgfsetstrokecolor{currentstroke}%
\pgfsetdash{}{0pt}%
\pgfsys@defobject{currentmarker}{\pgfqpoint{-0.055556in}{0.000000in}}{\pgfqpoint{-0.000000in}{0.000000in}}{%
\pgfpathmoveto{\pgfqpoint{-0.000000in}{0.000000in}}%
\pgfpathlineto{\pgfqpoint{-0.055556in}{0.000000in}}%
\pgfusepath{stroke,fill}%
}%
\begin{pgfscope}%
\pgfsys@transformshift{7.200000in}{4.600000in}%
\pgfsys@useobject{currentmarker}{}%
\end{pgfscope}%
\end{pgfscope}%
\begin{pgfscope}%
\definecolor{textcolor}{rgb}{0.000000,0.000000,0.000000}%
\pgfsetstrokecolor{textcolor}%
\pgfsetfillcolor{textcolor}%
\pgftext[x=0.944444in,y=4.600000in,right,]{\color{textcolor}\rmfamily\fontsize{10.000000}{12.000000}\selectfont \(\displaystyle {200}\)}%
\end{pgfscope}%
\begin{pgfscope}%
\pgfpathrectangle{\pgfqpoint{1.000000in}{0.600000in}}{\pgfqpoint{6.200000in}{4.800000in}}%
\pgfusepath{clip}%
\pgfsetbuttcap%
\pgfsetroundjoin%
\pgfsetlinewidth{0.501875pt}%
\definecolor{currentstroke}{rgb}{0.000000,0.000000,0.000000}%
\pgfsetstrokecolor{currentstroke}%
\pgfsetdash{{1.000000pt}{3.000000pt}}{0.000000pt}%
\pgfpathmoveto{\pgfqpoint{1.000000in}{5.400000in}}%
\pgfpathlineto{\pgfqpoint{7.200000in}{5.400000in}}%
\pgfusepath{stroke}%
\end{pgfscope}%
\begin{pgfscope}%
\pgfsetbuttcap%
\pgfsetroundjoin%
\definecolor{currentfill}{rgb}{0.000000,0.000000,0.000000}%
\pgfsetfillcolor{currentfill}%
\pgfsetlinewidth{0.501875pt}%
\definecolor{currentstroke}{rgb}{0.000000,0.000000,0.000000}%
\pgfsetstrokecolor{currentstroke}%
\pgfsetdash{}{0pt}%
\pgfsys@defobject{currentmarker}{\pgfqpoint{0.000000in}{0.000000in}}{\pgfqpoint{0.055556in}{0.000000in}}{%
\pgfpathmoveto{\pgfqpoint{0.000000in}{0.000000in}}%
\pgfpathlineto{\pgfqpoint{0.055556in}{0.000000in}}%
\pgfusepath{stroke,fill}%
}%
\begin{pgfscope}%
\pgfsys@transformshift{1.000000in}{5.400000in}%
\pgfsys@useobject{currentmarker}{}%
\end{pgfscope}%
\end{pgfscope}%
\begin{pgfscope}%
\pgfsetbuttcap%
\pgfsetroundjoin%
\definecolor{currentfill}{rgb}{0.000000,0.000000,0.000000}%
\pgfsetfillcolor{currentfill}%
\pgfsetlinewidth{0.501875pt}%
\definecolor{currentstroke}{rgb}{0.000000,0.000000,0.000000}%
\pgfsetstrokecolor{currentstroke}%
\pgfsetdash{}{0pt}%
\pgfsys@defobject{currentmarker}{\pgfqpoint{-0.055556in}{0.000000in}}{\pgfqpoint{-0.000000in}{0.000000in}}{%
\pgfpathmoveto{\pgfqpoint{-0.000000in}{0.000000in}}%
\pgfpathlineto{\pgfqpoint{-0.055556in}{0.000000in}}%
\pgfusepath{stroke,fill}%
}%
\begin{pgfscope}%
\pgfsys@transformshift{7.200000in}{5.400000in}%
\pgfsys@useobject{currentmarker}{}%
\end{pgfscope}%
\end{pgfscope}%
\begin{pgfscope}%
\definecolor{textcolor}{rgb}{0.000000,0.000000,0.000000}%
\pgfsetstrokecolor{textcolor}%
\pgfsetfillcolor{textcolor}%
\pgftext[x=0.944444in,y=5.400000in,right,]{\color{textcolor}\rmfamily\fontsize{10.000000}{12.000000}\selectfont \(\displaystyle {250}\)}%
\end{pgfscope}%
\begin{pgfscope}%
\definecolor{textcolor}{rgb}{0.000000,0.000000,0.000000}%
\pgfsetstrokecolor{textcolor}%
\pgfsetfillcolor{textcolor}%
\pgftext[x=0.628086in,y=3.000000in,,bottom,rotate=90.000000]{\color{textcolor}\rmfamily\fontsize{12.000000}{14.400000}\selectfont \(\displaystyle \theta\ (rad)\)}%
\end{pgfscope}%
\begin{pgfscope}%
\definecolor{textcolor}{rgb}{0.000000,0.000000,0.000000}%
\pgfsetstrokecolor{textcolor}%
\pgfsetfillcolor{textcolor}%
\pgftext[x=4.100000in,y=5.469444in,,base]{\color{textcolor}\rmfamily\fontsize{12.000000}{14.400000}\selectfont \(\displaystyle Simple\ pendulum\ using\ Euler's\ methods\ (time\ step = 0.2\ (s))\)}%
\end{pgfscope}%
\begin{pgfscope}%
\pgfsetbuttcap%
\pgfsetmiterjoin%
\definecolor{currentfill}{rgb}{1.000000,1.000000,1.000000}%
\pgfsetfillcolor{currentfill}%
\pgfsetlinewidth{1.003750pt}%
\definecolor{currentstroke}{rgb}{0.000000,0.000000,0.000000}%
\pgfsetstrokecolor{currentstroke}%
\pgfsetdash{}{0pt}%
\pgfpathmoveto{\pgfqpoint{1.083333in}{4.569445in}}%
\pgfpathlineto{\pgfqpoint{3.093110in}{4.569445in}}%
\pgfpathlineto{\pgfqpoint{3.093110in}{5.316667in}}%
\pgfpathlineto{\pgfqpoint{1.083333in}{5.316667in}}%
\pgfpathlineto{\pgfqpoint{1.083333in}{4.569445in}}%
\pgfpathclose%
\pgfusepath{stroke,fill}%
\end{pgfscope}%
\begin{pgfscope}%
\pgfsetrectcap%
\pgfsetroundjoin%
\pgfsetlinewidth{1.003750pt}%
\definecolor{currentstroke}{rgb}{1.000000,0.000000,0.000000}%
\pgfsetstrokecolor{currentstroke}%
\pgfsetdash{}{0pt}%
\pgfpathmoveto{\pgfqpoint{1.200000in}{5.191667in}}%
\pgfpathlineto{\pgfqpoint{1.433333in}{5.191667in}}%
\pgfusepath{stroke}%
\end{pgfscope}%
\begin{pgfscope}%
\definecolor{textcolor}{rgb}{0.000000,0.000000,0.000000}%
\pgfsetstrokecolor{textcolor}%
\pgfsetfillcolor{textcolor}%
\pgftext[x=1.616667in,y=5.133333in,left,base]{\color{textcolor}\rmfamily\fontsize{12.000000}{14.400000}\selectfont \(\displaystyle euler\ explicit\)}%
\end{pgfscope}%
\begin{pgfscope}%
\pgfsetrectcap%
\pgfsetroundjoin%
\pgfsetlinewidth{1.003750pt}%
\definecolor{currentstroke}{rgb}{0.000000,0.000000,1.000000}%
\pgfsetstrokecolor{currentstroke}%
\pgfsetdash{}{0pt}%
\pgfpathmoveto{\pgfqpoint{1.200000in}{4.959260in}}%
\pgfpathlineto{\pgfqpoint{1.433333in}{4.959260in}}%
\pgfusepath{stroke}%
\end{pgfscope}%
\begin{pgfscope}%
\definecolor{textcolor}{rgb}{0.000000,0.000000,0.000000}%
\pgfsetstrokecolor{textcolor}%
\pgfsetfillcolor{textcolor}%
\pgftext[x=1.616667in,y=4.900926in,left,base]{\color{textcolor}\rmfamily\fontsize{12.000000}{14.400000}\selectfont \(\displaystyle euler\ implicit\)}%
\end{pgfscope}%
\begin{pgfscope}%
\pgfsetrectcap%
\pgfsetroundjoin%
\pgfsetlinewidth{1.003750pt}%
\definecolor{currentstroke}{rgb}{0.000000,0.000000,0.000000}%
\pgfsetstrokecolor{currentstroke}%
\pgfsetdash{}{0pt}%
\pgfpathmoveto{\pgfqpoint{1.200000in}{4.726852in}}%
\pgfpathlineto{\pgfqpoint{1.433333in}{4.726852in}}%
\pgfusepath{stroke}%
\end{pgfscope}%
\begin{pgfscope}%
\definecolor{textcolor}{rgb}{0.000000,0.000000,0.000000}%
\pgfsetstrokecolor{textcolor}%
\pgfsetfillcolor{textcolor}%
\pgftext[x=1.616667in,y=4.668519in,left,base]{\color{textcolor}\rmfamily\fontsize{12.000000}{14.400000}\selectfont \(\displaystyle trapezoidal\ scheme\)}%
\end{pgfscope}%
\end{pgfpicture}%
\makeatother%
\endgroup%
}
    \end{figure}

    \begin{figure}[ht!]
    \centering
    \resizebox{0.9\linewidth}{!}{%% Creator: Matplotlib, PGF backend
%%
%% To include the figure in your LaTeX document, write
%%   \input{<filename>.pgf}
%%
%% Make sure the required packages are loaded in your preamble
%%   \usepackage{pgf}
%%
%% Also ensure that all the required font packages are loaded; for instance,
%% the lmodern package is sometimes necessary when using math font.
%%   \usepackage{lmodern}
%%
%% Figures using additional raster images can only be included by \input if
%% they are in the same directory as the main LaTeX file. For loading figures
%% from other directories you can use the `import` package
%%   \usepackage{import}
%%
%% and then include the figures with
%%   \import{<path to file>}{<filename>.pgf}
%%
%% Matplotlib used the following preamble
%%
\begingroup%
\makeatletter%
\begin{pgfpicture}%
\pgfpathrectangle{\pgfpointorigin}{\pgfqpoint{8.000000in}{6.000000in}}%
\pgfusepath{use as bounding box, clip}%
\begin{pgfscope}%
\pgfsetbuttcap%
\pgfsetmiterjoin%
\definecolor{currentfill}{rgb}{1.000000,1.000000,1.000000}%
\pgfsetfillcolor{currentfill}%
\pgfsetlinewidth{0.000000pt}%
\definecolor{currentstroke}{rgb}{1.000000,1.000000,1.000000}%
\pgfsetstrokecolor{currentstroke}%
\pgfsetdash{}{0pt}%
\pgfpathmoveto{\pgfqpoint{0.000000in}{0.000000in}}%
\pgfpathlineto{\pgfqpoint{8.000000in}{0.000000in}}%
\pgfpathlineto{\pgfqpoint{8.000000in}{6.000000in}}%
\pgfpathlineto{\pgfqpoint{0.000000in}{6.000000in}}%
\pgfpathlineto{\pgfqpoint{0.000000in}{0.000000in}}%
\pgfpathclose%
\pgfusepath{fill}%
\end{pgfscope}%
\begin{pgfscope}%
\pgfsetbuttcap%
\pgfsetmiterjoin%
\definecolor{currentfill}{rgb}{1.000000,1.000000,1.000000}%
\pgfsetfillcolor{currentfill}%
\pgfsetlinewidth{0.000000pt}%
\definecolor{currentstroke}{rgb}{0.000000,0.000000,0.000000}%
\pgfsetstrokecolor{currentstroke}%
\pgfsetstrokeopacity{0.000000}%
\pgfsetdash{}{0pt}%
\pgfpathmoveto{\pgfqpoint{1.000000in}{0.600000in}}%
\pgfpathlineto{\pgfqpoint{7.200000in}{0.600000in}}%
\pgfpathlineto{\pgfqpoint{7.200000in}{5.400000in}}%
\pgfpathlineto{\pgfqpoint{1.000000in}{5.400000in}}%
\pgfpathlineto{\pgfqpoint{1.000000in}{0.600000in}}%
\pgfpathclose%
\pgfusepath{fill}%
\end{pgfscope}%
\begin{pgfscope}%
\pgfpathrectangle{\pgfqpoint{1.000000in}{0.600000in}}{\pgfqpoint{6.200000in}{4.800000in}}%
\pgfusepath{clip}%
\pgfsetrectcap%
\pgfsetroundjoin%
\pgfsetlinewidth{1.003750pt}%
\definecolor{currentstroke}{rgb}{1.000000,0.000000,0.000000}%
\pgfsetstrokecolor{currentstroke}%
\pgfsetdash{}{0pt}%
\pgfpathmoveto{\pgfqpoint{1.000000in}{4.872252in}}%
\pgfpathlineto{\pgfqpoint{1.031000in}{4.872252in}}%
\pgfpathlineto{\pgfqpoint{1.062000in}{4.870918in}}%
\pgfpathlineto{\pgfqpoint{1.124000in}{4.864550in}}%
\pgfpathlineto{\pgfqpoint{1.186000in}{4.856883in}}%
\pgfpathlineto{\pgfqpoint{1.217000in}{4.854779in}}%
\pgfpathlineto{\pgfqpoint{1.248000in}{4.854792in}}%
\pgfpathlineto{\pgfqpoint{1.279000in}{4.857199in}}%
\pgfpathlineto{\pgfqpoint{1.310000in}{4.861999in}}%
\pgfpathlineto{\pgfqpoint{1.372000in}{4.876864in}}%
\pgfpathlineto{\pgfqpoint{1.403000in}{4.884316in}}%
\pgfpathlineto{\pgfqpoint{1.434000in}{4.889589in}}%
\pgfpathlineto{\pgfqpoint{1.465000in}{4.892205in}}%
\pgfpathlineto{\pgfqpoint{1.496000in}{4.892587in}}%
\pgfpathlineto{\pgfqpoint{1.527000in}{4.891158in}}%
\pgfpathlineto{\pgfqpoint{1.558000in}{4.887986in}}%
\pgfpathlineto{\pgfqpoint{1.589000in}{4.882820in}}%
\pgfpathlineto{\pgfqpoint{1.620000in}{4.875227in}}%
\pgfpathlineto{\pgfqpoint{1.651000in}{4.864971in}}%
\pgfpathlineto{\pgfqpoint{1.713000in}{4.841047in}}%
\pgfpathlineto{\pgfqpoint{1.744000in}{4.831866in}}%
\pgfpathlineto{\pgfqpoint{1.806000in}{4.816777in}}%
\pgfpathlineto{\pgfqpoint{1.837000in}{4.807134in}}%
\pgfpathlineto{\pgfqpoint{1.868000in}{4.794826in}}%
\pgfpathlineto{\pgfqpoint{1.930000in}{4.767931in}}%
\pgfpathlineto{\pgfqpoint{2.023000in}{4.736176in}}%
\pgfpathlineto{\pgfqpoint{2.116000in}{4.693666in}}%
\pgfpathlineto{\pgfqpoint{2.147000in}{4.682438in}}%
\pgfpathlineto{\pgfqpoint{2.178000in}{4.669906in}}%
\pgfpathlineto{\pgfqpoint{2.271000in}{4.624492in}}%
\pgfpathlineto{\pgfqpoint{2.302000in}{4.612021in}}%
\pgfpathlineto{\pgfqpoint{2.333000in}{4.597806in}}%
\pgfpathlineto{\pgfqpoint{2.395000in}{4.564493in}}%
\pgfpathlineto{\pgfqpoint{2.457000in}{4.536549in}}%
\pgfpathlineto{\pgfqpoint{2.550000in}{4.486562in}}%
\pgfpathlineto{\pgfqpoint{2.581000in}{4.472053in}}%
\pgfpathlineto{\pgfqpoint{2.612000in}{4.455196in}}%
\pgfpathlineto{\pgfqpoint{2.643000in}{4.436613in}}%
\pgfpathlineto{\pgfqpoint{2.674000in}{4.420078in}}%
\pgfpathlineto{\pgfqpoint{2.705000in}{4.404876in}}%
\pgfpathlineto{\pgfqpoint{2.736000in}{4.387392in}}%
\pgfpathlineto{\pgfqpoint{2.767000in}{4.368211in}}%
\pgfpathlineto{\pgfqpoint{2.860000in}{4.316736in}}%
\pgfpathlineto{\pgfqpoint{2.891000in}{4.297121in}}%
\pgfpathlineto{\pgfqpoint{2.953000in}{4.263049in}}%
\pgfpathlineto{\pgfqpoint{3.015000in}{4.223785in}}%
\pgfpathlineto{\pgfqpoint{3.046000in}{4.206742in}}%
\pgfpathlineto{\pgfqpoint{3.077000in}{4.188238in}}%
\pgfpathlineto{\pgfqpoint{3.108000in}{4.167543in}}%
\pgfpathlineto{\pgfqpoint{3.201000in}{4.110667in}}%
\pgfpathlineto{\pgfqpoint{3.232000in}{4.089966in}}%
\pgfpathlineto{\pgfqpoint{3.263000in}{4.071884in}}%
\pgfpathlineto{\pgfqpoint{3.294000in}{4.052388in}}%
\pgfpathlineto{\pgfqpoint{3.325000in}{4.030824in}}%
\pgfpathlineto{\pgfqpoint{3.387000in}{3.992480in}}%
\pgfpathlineto{\pgfqpoint{3.449000in}{3.950013in}}%
\pgfpathlineto{\pgfqpoint{3.480000in}{3.931131in}}%
\pgfpathlineto{\pgfqpoint{3.573000in}{3.868578in}}%
\pgfpathlineto{\pgfqpoint{3.604000in}{3.847634in}}%
\pgfpathlineto{\pgfqpoint{3.635000in}{3.825019in}}%
\pgfpathlineto{\pgfqpoint{3.697000in}{3.784210in}}%
\pgfpathlineto{\pgfqpoint{3.728000in}{3.761217in}}%
\pgfpathlineto{\pgfqpoint{3.790000in}{3.719666in}}%
\pgfpathlineto{\pgfqpoint{3.821000in}{3.696430in}}%
\pgfpathlineto{\pgfqpoint{3.914000in}{3.630662in}}%
\pgfpathlineto{\pgfqpoint{4.069000in}{3.520287in}}%
\pgfpathlineto{\pgfqpoint{4.100000in}{3.496323in}}%
\pgfpathlineto{\pgfqpoint{4.255000in}{3.383088in}}%
\pgfpathlineto{\pgfqpoint{4.286000in}{3.358480in}}%
\pgfpathlineto{\pgfqpoint{4.348000in}{3.313235in}}%
\pgfpathlineto{\pgfqpoint{4.379000in}{3.288337in}}%
\pgfpathlineto{\pgfqpoint{4.441000in}{3.242521in}}%
\pgfpathlineto{\pgfqpoint{4.472000in}{3.217446in}}%
\pgfpathlineto{\pgfqpoint{4.503000in}{3.195023in}}%
\pgfpathlineto{\pgfqpoint{4.782000in}{2.977514in}}%
\pgfpathlineto{\pgfqpoint{4.813000in}{2.951532in}}%
\pgfpathlineto{\pgfqpoint{4.875000in}{2.903280in}}%
\pgfpathlineto{\pgfqpoint{4.906000in}{2.877354in}}%
\pgfpathlineto{\pgfqpoint{4.937000in}{2.853988in}}%
\pgfpathlineto{\pgfqpoint{4.999000in}{2.802912in}}%
\pgfpathlineto{\pgfqpoint{5.030000in}{2.778978in}}%
\pgfpathlineto{\pgfqpoint{5.061000in}{2.752474in}}%
\pgfpathlineto{\pgfqpoint{5.123000in}{2.703029in}}%
\pgfpathlineto{\pgfqpoint{5.154000in}{2.676484in}}%
\pgfpathlineto{\pgfqpoint{5.185000in}{2.652548in}}%
\pgfpathlineto{\pgfqpoint{5.247000in}{2.600465in}}%
\pgfpathlineto{\pgfqpoint{5.278000in}{2.575870in}}%
\pgfpathlineto{\pgfqpoint{5.309000in}{2.548817in}}%
\pgfpathlineto{\pgfqpoint{5.371000in}{2.498199in}}%
\pgfpathlineto{\pgfqpoint{5.402000in}{2.471421in}}%
\pgfpathlineto{\pgfqpoint{5.433000in}{2.446901in}}%
\pgfpathlineto{\pgfqpoint{5.464000in}{2.419720in}}%
\pgfpathlineto{\pgfqpoint{5.526000in}{2.368448in}}%
\pgfpathlineto{\pgfqpoint{5.557000in}{2.341045in}}%
\pgfpathlineto{\pgfqpoint{5.588000in}{2.316276in}}%
\pgfpathlineto{\pgfqpoint{5.650000in}{2.262738in}}%
\pgfpathlineto{\pgfqpoint{5.681000in}{2.237077in}}%
\pgfpathlineto{\pgfqpoint{5.712000in}{2.209285in}}%
\pgfpathlineto{\pgfqpoint{5.743000in}{2.184148in}}%
\pgfpathlineto{\pgfqpoint{5.805000in}{2.129978in}}%
\pgfpathlineto{\pgfqpoint{5.836000in}{2.104201in}}%
\pgfpathlineto{\pgfqpoint{5.867000in}{2.076116in}}%
\pgfpathlineto{\pgfqpoint{5.898000in}{2.050644in}}%
\pgfpathlineto{\pgfqpoint{5.960000in}{1.995940in}}%
\pgfpathlineto{\pgfqpoint{5.991000in}{1.969905in}}%
\pgfpathlineto{\pgfqpoint{6.022000in}{1.941562in}}%
\pgfpathlineto{\pgfqpoint{6.053000in}{1.915844in}}%
\pgfpathlineto{\pgfqpoint{6.115000in}{1.860674in}}%
\pgfpathlineto{\pgfqpoint{6.146000in}{1.834250in}}%
\pgfpathlineto{\pgfqpoint{6.177000in}{1.805687in}}%
\pgfpathlineto{\pgfqpoint{6.208000in}{1.779789in}}%
\pgfpathlineto{\pgfqpoint{6.270000in}{1.724245in}}%
\pgfpathlineto{\pgfqpoint{6.301000in}{1.697285in}}%
\pgfpathlineto{\pgfqpoint{6.332000in}{1.668591in}}%
\pgfpathlineto{\pgfqpoint{6.363000in}{1.642482in}}%
\pgfpathlineto{\pgfqpoint{6.394000in}{1.613760in}}%
\pgfpathlineto{\pgfqpoint{6.456000in}{1.559051in}}%
\pgfpathlineto{\pgfqpoint{6.487000in}{1.530410in}}%
\pgfpathlineto{\pgfqpoint{6.518000in}{1.503905in}}%
\pgfpathlineto{\pgfqpoint{6.549000in}{1.474779in}}%
\pgfpathlineto{\pgfqpoint{6.580000in}{1.448030in}}%
\pgfpathlineto{\pgfqpoint{6.704000in}{1.334762in}}%
\pgfpathlineto{\pgfqpoint{6.735000in}{1.308108in}}%
\pgfpathlineto{\pgfqpoint{6.797000in}{1.251260in}}%
\pgfpathlineto{\pgfqpoint{7.045000in}{1.024378in}}%
\pgfpathlineto{\pgfqpoint{7.076000in}{0.994797in}}%
\pgfpathlineto{\pgfqpoint{7.107000in}{0.967553in}}%
\pgfpathlineto{\pgfqpoint{7.138000in}{0.937666in}}%
\pgfpathlineto{\pgfqpoint{7.169000in}{0.910176in}}%
\pgfpathlineto{\pgfqpoint{7.200000in}{0.880816in}}%
\pgfpathlineto{\pgfqpoint{7.200000in}{0.880816in}}%
\pgfusepath{stroke}%
\end{pgfscope}%
\begin{pgfscope}%
\pgfpathrectangle{\pgfqpoint{1.000000in}{0.600000in}}{\pgfqpoint{6.200000in}{4.800000in}}%
\pgfusepath{clip}%
\pgfsetrectcap%
\pgfsetroundjoin%
\pgfsetlinewidth{1.003750pt}%
\definecolor{currentstroke}{rgb}{0.000000,0.000000,1.000000}%
\pgfsetstrokecolor{currentstroke}%
\pgfsetdash{}{0pt}%
\pgfpathmoveto{\pgfqpoint{1.000000in}{4.872252in}}%
\pgfpathlineto{\pgfqpoint{1.062000in}{4.868550in}}%
\pgfpathlineto{\pgfqpoint{1.124000in}{4.864176in}}%
\pgfpathlineto{\pgfqpoint{1.155000in}{4.863438in}}%
\pgfpathlineto{\pgfqpoint{1.217000in}{4.865000in}}%
\pgfpathlineto{\pgfqpoint{1.310000in}{4.868434in}}%
\pgfpathlineto{\pgfqpoint{1.372000in}{4.867906in}}%
\pgfpathlineto{\pgfqpoint{1.496000in}{4.865605in}}%
\pgfpathlineto{\pgfqpoint{1.775000in}{4.866473in}}%
\pgfpathlineto{\pgfqpoint{1.930000in}{4.866735in}}%
\pgfpathlineto{\pgfqpoint{2.240000in}{4.866656in}}%
\pgfpathlineto{\pgfqpoint{2.984000in}{4.866674in}}%
\pgfpathlineto{\pgfqpoint{7.200000in}{4.866667in}}%
\pgfpathlineto{\pgfqpoint{7.200000in}{4.866667in}}%
\pgfusepath{stroke}%
\end{pgfscope}%
\begin{pgfscope}%
\pgfpathrectangle{\pgfqpoint{1.000000in}{0.600000in}}{\pgfqpoint{6.200000in}{4.800000in}}%
\pgfusepath{clip}%
\pgfsetrectcap%
\pgfsetroundjoin%
\pgfsetlinewidth{1.003750pt}%
\definecolor{currentstroke}{rgb}{0.000000,0.000000,0.000000}%
\pgfsetstrokecolor{currentstroke}%
\pgfsetdash{}{0pt}%
\pgfpathmoveto{\pgfqpoint{1.000000in}{4.872252in}}%
\pgfpathlineto{\pgfqpoint{1.031000in}{4.871585in}}%
\pgfpathlineto{\pgfqpoint{1.093000in}{4.866923in}}%
\pgfpathlineto{\pgfqpoint{1.155000in}{4.861869in}}%
\pgfpathlineto{\pgfqpoint{1.186000in}{4.860856in}}%
\pgfpathlineto{\pgfqpoint{1.217000in}{4.861285in}}%
\pgfpathlineto{\pgfqpoint{1.279000in}{4.865826in}}%
\pgfpathlineto{\pgfqpoint{1.341000in}{4.871326in}}%
\pgfpathlineto{\pgfqpoint{1.372000in}{4.872654in}}%
\pgfpathlineto{\pgfqpoint{1.403000in}{4.872504in}}%
\pgfpathlineto{\pgfqpoint{1.465000in}{4.868172in}}%
\pgfpathlineto{\pgfqpoint{1.527000in}{4.862244in}}%
\pgfpathlineto{\pgfqpoint{1.558000in}{4.860572in}}%
\pgfpathlineto{\pgfqpoint{1.589000in}{4.860398in}}%
\pgfpathlineto{\pgfqpoint{1.620000in}{4.861775in}}%
\pgfpathlineto{\pgfqpoint{1.744000in}{4.872778in}}%
\pgfpathlineto{\pgfqpoint{1.775000in}{4.873325in}}%
\pgfpathlineto{\pgfqpoint{1.806000in}{4.872241in}}%
\pgfpathlineto{\pgfqpoint{1.868000in}{4.866410in}}%
\pgfpathlineto{\pgfqpoint{1.930000in}{4.860655in}}%
\pgfpathlineto{\pgfqpoint{1.961000in}{4.859688in}}%
\pgfpathlineto{\pgfqpoint{1.992000in}{4.860413in}}%
\pgfpathlineto{\pgfqpoint{2.054000in}{4.866033in}}%
\pgfpathlineto{\pgfqpoint{2.116000in}{4.872433in}}%
\pgfpathlineto{\pgfqpoint{2.147000in}{4.873866in}}%
\pgfpathlineto{\pgfqpoint{2.178000in}{4.873568in}}%
\pgfpathlineto{\pgfqpoint{2.209000in}{4.871582in}}%
\pgfpathlineto{\pgfqpoint{2.333000in}{4.859382in}}%
\pgfpathlineto{\pgfqpoint{2.364000in}{4.859183in}}%
\pgfpathlineto{\pgfqpoint{2.395000in}{4.860782in}}%
\pgfpathlineto{\pgfqpoint{2.457000in}{4.867729in}}%
\pgfpathlineto{\pgfqpoint{2.488000in}{4.871369in}}%
\pgfpathlineto{\pgfqpoint{2.519000in}{4.873855in}}%
\pgfpathlineto{\pgfqpoint{2.550000in}{4.874623in}}%
\pgfpathlineto{\pgfqpoint{2.581000in}{4.873508in}}%
\pgfpathlineto{\pgfqpoint{2.643000in}{4.866833in}}%
\pgfpathlineto{\pgfqpoint{2.705000in}{4.859805in}}%
\pgfpathlineto{\pgfqpoint{2.736000in}{4.858402in}}%
\pgfpathlineto{\pgfqpoint{2.767000in}{4.858933in}}%
\pgfpathlineto{\pgfqpoint{2.798000in}{4.861316in}}%
\pgfpathlineto{\pgfqpoint{2.891000in}{4.872916in}}%
\pgfpathlineto{\pgfqpoint{2.922000in}{4.875013in}}%
\pgfpathlineto{\pgfqpoint{2.953000in}{4.875166in}}%
\pgfpathlineto{\pgfqpoint{2.984000in}{4.873334in}}%
\pgfpathlineto{\pgfqpoint{3.046000in}{4.865487in}}%
\pgfpathlineto{\pgfqpoint{3.077000in}{4.861371in}}%
\pgfpathlineto{\pgfqpoint{3.108000in}{4.858541in}}%
\pgfpathlineto{\pgfqpoint{3.139000in}{4.857608in}}%
\pgfpathlineto{\pgfqpoint{3.170000in}{4.858749in}}%
\pgfpathlineto{\pgfqpoint{3.201000in}{4.861780in}}%
\pgfpathlineto{\pgfqpoint{3.294000in}{4.874188in}}%
\pgfpathlineto{\pgfqpoint{3.325000in}{4.875985in}}%
\pgfpathlineto{\pgfqpoint{3.356000in}{4.875672in}}%
\pgfpathlineto{\pgfqpoint{3.387000in}{4.873285in}}%
\pgfpathlineto{\pgfqpoint{3.511000in}{4.857497in}}%
\pgfpathlineto{\pgfqpoint{3.542000in}{4.856855in}}%
\pgfpathlineto{\pgfqpoint{3.573000in}{4.858406in}}%
\pgfpathlineto{\pgfqpoint{3.604000in}{4.861912in}}%
\pgfpathlineto{\pgfqpoint{3.697000in}{4.875162in}}%
\pgfpathlineto{\pgfqpoint{3.728000in}{4.876868in}}%
\pgfpathlineto{\pgfqpoint{3.759000in}{4.876331in}}%
\pgfpathlineto{\pgfqpoint{3.790000in}{4.873616in}}%
\pgfpathlineto{\pgfqpoint{3.914000in}{4.856647in}}%
\pgfpathlineto{\pgfqpoint{3.945000in}{4.856012in}}%
\pgfpathlineto{\pgfqpoint{3.976000in}{4.857686in}}%
\pgfpathlineto{\pgfqpoint{4.007000in}{4.861436in}}%
\pgfpathlineto{\pgfqpoint{4.100000in}{4.875765in}}%
\pgfpathlineto{\pgfqpoint{4.131000in}{4.877703in}}%
\pgfpathlineto{\pgfqpoint{4.162000in}{4.877288in}}%
\pgfpathlineto{\pgfqpoint{4.193000in}{4.874565in}}%
\pgfpathlineto{\pgfqpoint{4.255000in}{4.864412in}}%
\pgfpathlineto{\pgfqpoint{4.286000in}{4.859389in}}%
\pgfpathlineto{\pgfqpoint{4.317000in}{4.856075in}}%
\pgfpathlineto{\pgfqpoint{4.348000in}{4.855048in}}%
\pgfpathlineto{\pgfqpoint{4.379000in}{4.856443in}}%
\pgfpathlineto{\pgfqpoint{4.410000in}{4.860098in}}%
\pgfpathlineto{\pgfqpoint{4.503000in}{4.875765in}}%
\pgfpathlineto{\pgfqpoint{4.534000in}{4.878365in}}%
\pgfpathlineto{\pgfqpoint{4.565000in}{4.878541in}}%
\pgfpathlineto{\pgfqpoint{4.596000in}{4.876266in}}%
\pgfpathlineto{\pgfqpoint{4.627000in}{4.871816in}}%
\pgfpathlineto{\pgfqpoint{4.689000in}{4.860301in}}%
\pgfpathlineto{\pgfqpoint{4.720000in}{4.856093in}}%
\pgfpathlineto{\pgfqpoint{4.751000in}{4.854165in}}%
\pgfpathlineto{\pgfqpoint{4.782000in}{4.854742in}}%
\pgfpathlineto{\pgfqpoint{4.813000in}{4.857775in}}%
\pgfpathlineto{\pgfqpoint{4.875000in}{4.869021in}}%
\pgfpathlineto{\pgfqpoint{4.906000in}{4.874637in}}%
\pgfpathlineto{\pgfqpoint{4.937000in}{4.878415in}}%
\pgfpathlineto{\pgfqpoint{4.968000in}{4.879767in}}%
\pgfpathlineto{\pgfqpoint{4.999000in}{4.878566in}}%
\pgfpathlineto{\pgfqpoint{5.030000in}{4.874909in}}%
\pgfpathlineto{\pgfqpoint{5.123000in}{4.857411in}}%
\pgfpathlineto{\pgfqpoint{5.154000in}{4.853983in}}%
\pgfpathlineto{\pgfqpoint{5.185000in}{4.853072in}}%
\pgfpathlineto{\pgfqpoint{5.216000in}{4.854747in}}%
\pgfpathlineto{\pgfqpoint{5.247000in}{4.858881in}}%
\pgfpathlineto{\pgfqpoint{5.340000in}{4.876897in}}%
\pgfpathlineto{\pgfqpoint{5.371000in}{4.880106in}}%
\pgfpathlineto{\pgfqpoint{5.402000in}{4.880741in}}%
\pgfpathlineto{\pgfqpoint{5.433000in}{4.878762in}}%
\pgfpathlineto{\pgfqpoint{5.464000in}{4.874309in}}%
\pgfpathlineto{\pgfqpoint{5.557000in}{4.855771in}}%
\pgfpathlineto{\pgfqpoint{5.588000in}{4.852613in}}%
\pgfpathlineto{\pgfqpoint{5.619000in}{4.852065in}}%
\pgfpathlineto{\pgfqpoint{5.650000in}{4.854153in}}%
\pgfpathlineto{\pgfqpoint{5.681000in}{4.858752in}}%
\pgfpathlineto{\pgfqpoint{5.774000in}{4.877891in}}%
\pgfpathlineto{\pgfqpoint{5.805000in}{4.881197in}}%
\pgfpathlineto{\pgfqpoint{5.836000in}{4.881871in}}%
\pgfpathlineto{\pgfqpoint{5.867000in}{4.879892in}}%
\pgfpathlineto{\pgfqpoint{5.898000in}{4.875348in}}%
\pgfpathlineto{\pgfqpoint{5.991000in}{4.855523in}}%
\pgfpathlineto{\pgfqpoint{6.022000in}{4.851847in}}%
\pgfpathlineto{\pgfqpoint{6.053000in}{4.850811in}}%
\pgfpathlineto{\pgfqpoint{6.084000in}{4.852437in}}%
\pgfpathlineto{\pgfqpoint{6.115000in}{4.856684in}}%
\pgfpathlineto{\pgfqpoint{6.208000in}{4.877180in}}%
\pgfpathlineto{\pgfqpoint{6.239000in}{4.881472in}}%
\pgfpathlineto{\pgfqpoint{6.270000in}{4.883124in}}%
\pgfpathlineto{\pgfqpoint{6.301000in}{4.882117in}}%
\pgfpathlineto{\pgfqpoint{6.332000in}{4.878452in}}%
\pgfpathlineto{\pgfqpoint{6.363000in}{4.872340in}}%
\pgfpathlineto{\pgfqpoint{6.425000in}{4.857556in}}%
\pgfpathlineto{\pgfqpoint{6.456000in}{4.852390in}}%
\pgfpathlineto{\pgfqpoint{6.487000in}{4.849843in}}%
\pgfpathlineto{\pgfqpoint{6.518000in}{4.849951in}}%
\pgfpathlineto{\pgfqpoint{6.549000in}{4.852716in}}%
\pgfpathlineto{\pgfqpoint{6.580000in}{4.858089in}}%
\pgfpathlineto{\pgfqpoint{6.673000in}{4.879571in}}%
\pgfpathlineto{\pgfqpoint{6.704000in}{4.883318in}}%
\pgfpathlineto{\pgfqpoint{6.735000in}{4.884407in}}%
\pgfpathlineto{\pgfqpoint{6.766000in}{4.882865in}}%
\pgfpathlineto{\pgfqpoint{6.797000in}{4.878662in}}%
\pgfpathlineto{\pgfqpoint{6.828000in}{4.871992in}}%
\pgfpathlineto{\pgfqpoint{6.890000in}{4.856444in}}%
\pgfpathlineto{\pgfqpoint{6.921000in}{4.851177in}}%
\pgfpathlineto{\pgfqpoint{6.952000in}{4.848567in}}%
\pgfpathlineto{\pgfqpoint{6.983000in}{4.848572in}}%
\pgfpathlineto{\pgfqpoint{7.014000in}{4.851192in}}%
\pgfpathlineto{\pgfqpoint{7.045000in}{4.856469in}}%
\pgfpathlineto{\pgfqpoint{7.138000in}{4.879352in}}%
\pgfpathlineto{\pgfqpoint{7.169000in}{4.883861in}}%
\pgfpathlineto{\pgfqpoint{7.200000in}{4.885721in}}%
\pgfpathlineto{\pgfqpoint{7.200000in}{4.885721in}}%
\pgfusepath{stroke}%
\end{pgfscope}%
\begin{pgfscope}%
\pgfsetrectcap%
\pgfsetmiterjoin%
\pgfsetlinewidth{1.003750pt}%
\definecolor{currentstroke}{rgb}{0.000000,0.000000,0.000000}%
\pgfsetstrokecolor{currentstroke}%
\pgfsetdash{}{0pt}%
\pgfpathmoveto{\pgfqpoint{1.000000in}{0.600000in}}%
\pgfpathlineto{\pgfqpoint{1.000000in}{5.400000in}}%
\pgfusepath{stroke}%
\end{pgfscope}%
\begin{pgfscope}%
\pgfsetrectcap%
\pgfsetmiterjoin%
\pgfsetlinewidth{1.003750pt}%
\definecolor{currentstroke}{rgb}{0.000000,0.000000,0.000000}%
\pgfsetstrokecolor{currentstroke}%
\pgfsetdash{}{0pt}%
\pgfpathmoveto{\pgfqpoint{7.200000in}{0.600000in}}%
\pgfpathlineto{\pgfqpoint{7.200000in}{5.400000in}}%
\pgfusepath{stroke}%
\end{pgfscope}%
\begin{pgfscope}%
\pgfsetrectcap%
\pgfsetmiterjoin%
\pgfsetlinewidth{1.003750pt}%
\definecolor{currentstroke}{rgb}{0.000000,0.000000,0.000000}%
\pgfsetstrokecolor{currentstroke}%
\pgfsetdash{}{0pt}%
\pgfpathmoveto{\pgfqpoint{1.000000in}{0.600000in}}%
\pgfpathlineto{\pgfqpoint{7.200000in}{0.600000in}}%
\pgfusepath{stroke}%
\end{pgfscope}%
\begin{pgfscope}%
\pgfsetrectcap%
\pgfsetmiterjoin%
\pgfsetlinewidth{1.003750pt}%
\definecolor{currentstroke}{rgb}{0.000000,0.000000,0.000000}%
\pgfsetstrokecolor{currentstroke}%
\pgfsetdash{}{0pt}%
\pgfpathmoveto{\pgfqpoint{1.000000in}{5.400000in}}%
\pgfpathlineto{\pgfqpoint{7.200000in}{5.400000in}}%
\pgfusepath{stroke}%
\end{pgfscope}%
\begin{pgfscope}%
\pgfpathrectangle{\pgfqpoint{1.000000in}{0.600000in}}{\pgfqpoint{6.200000in}{4.800000in}}%
\pgfusepath{clip}%
\pgfsetbuttcap%
\pgfsetroundjoin%
\pgfsetlinewidth{0.501875pt}%
\definecolor{currentstroke}{rgb}{0.000000,0.000000,0.000000}%
\pgfsetstrokecolor{currentstroke}%
\pgfsetdash{{1.000000pt}{3.000000pt}}{0.000000pt}%
\pgfpathmoveto{\pgfqpoint{1.000000in}{0.600000in}}%
\pgfpathlineto{\pgfqpoint{1.000000in}{5.400000in}}%
\pgfusepath{stroke}%
\end{pgfscope}%
\begin{pgfscope}%
\pgfsetbuttcap%
\pgfsetroundjoin%
\definecolor{currentfill}{rgb}{0.000000,0.000000,0.000000}%
\pgfsetfillcolor{currentfill}%
\pgfsetlinewidth{0.501875pt}%
\definecolor{currentstroke}{rgb}{0.000000,0.000000,0.000000}%
\pgfsetstrokecolor{currentstroke}%
\pgfsetdash{}{0pt}%
\pgfsys@defobject{currentmarker}{\pgfqpoint{0.000000in}{0.000000in}}{\pgfqpoint{0.000000in}{0.055556in}}{%
\pgfpathmoveto{\pgfqpoint{0.000000in}{0.000000in}}%
\pgfpathlineto{\pgfqpoint{0.000000in}{0.055556in}}%
\pgfusepath{stroke,fill}%
}%
\begin{pgfscope}%
\pgfsys@transformshift{1.000000in}{0.600000in}%
\pgfsys@useobject{currentmarker}{}%
\end{pgfscope}%
\end{pgfscope}%
\begin{pgfscope}%
\pgfsetbuttcap%
\pgfsetroundjoin%
\definecolor{currentfill}{rgb}{0.000000,0.000000,0.000000}%
\pgfsetfillcolor{currentfill}%
\pgfsetlinewidth{0.501875pt}%
\definecolor{currentstroke}{rgb}{0.000000,0.000000,0.000000}%
\pgfsetstrokecolor{currentstroke}%
\pgfsetdash{}{0pt}%
\pgfsys@defobject{currentmarker}{\pgfqpoint{0.000000in}{-0.055556in}}{\pgfqpoint{0.000000in}{0.000000in}}{%
\pgfpathmoveto{\pgfqpoint{0.000000in}{0.000000in}}%
\pgfpathlineto{\pgfqpoint{0.000000in}{-0.055556in}}%
\pgfusepath{stroke,fill}%
}%
\begin{pgfscope}%
\pgfsys@transformshift{1.000000in}{5.400000in}%
\pgfsys@useobject{currentmarker}{}%
\end{pgfscope}%
\end{pgfscope}%
\begin{pgfscope}%
\definecolor{textcolor}{rgb}{0.000000,0.000000,0.000000}%
\pgfsetstrokecolor{textcolor}%
\pgfsetfillcolor{textcolor}%
\pgftext[x=1.000000in,y=0.544444in,,top]{\color{textcolor}\rmfamily\fontsize{10.000000}{12.000000}\selectfont \(\displaystyle {0}\)}%
\end{pgfscope}%
\begin{pgfscope}%
\pgfpathrectangle{\pgfqpoint{1.000000in}{0.600000in}}{\pgfqpoint{6.200000in}{4.800000in}}%
\pgfusepath{clip}%
\pgfsetbuttcap%
\pgfsetroundjoin%
\pgfsetlinewidth{0.501875pt}%
\definecolor{currentstroke}{rgb}{0.000000,0.000000,0.000000}%
\pgfsetstrokecolor{currentstroke}%
\pgfsetdash{{1.000000pt}{3.000000pt}}{0.000000pt}%
\pgfpathmoveto{\pgfqpoint{2.240000in}{0.600000in}}%
\pgfpathlineto{\pgfqpoint{2.240000in}{5.400000in}}%
\pgfusepath{stroke}%
\end{pgfscope}%
\begin{pgfscope}%
\pgfsetbuttcap%
\pgfsetroundjoin%
\definecolor{currentfill}{rgb}{0.000000,0.000000,0.000000}%
\pgfsetfillcolor{currentfill}%
\pgfsetlinewidth{0.501875pt}%
\definecolor{currentstroke}{rgb}{0.000000,0.000000,0.000000}%
\pgfsetstrokecolor{currentstroke}%
\pgfsetdash{}{0pt}%
\pgfsys@defobject{currentmarker}{\pgfqpoint{0.000000in}{0.000000in}}{\pgfqpoint{0.000000in}{0.055556in}}{%
\pgfpathmoveto{\pgfqpoint{0.000000in}{0.000000in}}%
\pgfpathlineto{\pgfqpoint{0.000000in}{0.055556in}}%
\pgfusepath{stroke,fill}%
}%
\begin{pgfscope}%
\pgfsys@transformshift{2.240000in}{0.600000in}%
\pgfsys@useobject{currentmarker}{}%
\end{pgfscope}%
\end{pgfscope}%
\begin{pgfscope}%
\pgfsetbuttcap%
\pgfsetroundjoin%
\definecolor{currentfill}{rgb}{0.000000,0.000000,0.000000}%
\pgfsetfillcolor{currentfill}%
\pgfsetlinewidth{0.501875pt}%
\definecolor{currentstroke}{rgb}{0.000000,0.000000,0.000000}%
\pgfsetstrokecolor{currentstroke}%
\pgfsetdash{}{0pt}%
\pgfsys@defobject{currentmarker}{\pgfqpoint{0.000000in}{-0.055556in}}{\pgfqpoint{0.000000in}{0.000000in}}{%
\pgfpathmoveto{\pgfqpoint{0.000000in}{0.000000in}}%
\pgfpathlineto{\pgfqpoint{0.000000in}{-0.055556in}}%
\pgfusepath{stroke,fill}%
}%
\begin{pgfscope}%
\pgfsys@transformshift{2.240000in}{5.400000in}%
\pgfsys@useobject{currentmarker}{}%
\end{pgfscope}%
\end{pgfscope}%
\begin{pgfscope}%
\definecolor{textcolor}{rgb}{0.000000,0.000000,0.000000}%
\pgfsetstrokecolor{textcolor}%
\pgfsetfillcolor{textcolor}%
\pgftext[x=2.240000in,y=0.544444in,,top]{\color{textcolor}\rmfamily\fontsize{10.000000}{12.000000}\selectfont \(\displaystyle {20}\)}%
\end{pgfscope}%
\begin{pgfscope}%
\pgfpathrectangle{\pgfqpoint{1.000000in}{0.600000in}}{\pgfqpoint{6.200000in}{4.800000in}}%
\pgfusepath{clip}%
\pgfsetbuttcap%
\pgfsetroundjoin%
\pgfsetlinewidth{0.501875pt}%
\definecolor{currentstroke}{rgb}{0.000000,0.000000,0.000000}%
\pgfsetstrokecolor{currentstroke}%
\pgfsetdash{{1.000000pt}{3.000000pt}}{0.000000pt}%
\pgfpathmoveto{\pgfqpoint{3.480000in}{0.600000in}}%
\pgfpathlineto{\pgfqpoint{3.480000in}{5.400000in}}%
\pgfusepath{stroke}%
\end{pgfscope}%
\begin{pgfscope}%
\pgfsetbuttcap%
\pgfsetroundjoin%
\definecolor{currentfill}{rgb}{0.000000,0.000000,0.000000}%
\pgfsetfillcolor{currentfill}%
\pgfsetlinewidth{0.501875pt}%
\definecolor{currentstroke}{rgb}{0.000000,0.000000,0.000000}%
\pgfsetstrokecolor{currentstroke}%
\pgfsetdash{}{0pt}%
\pgfsys@defobject{currentmarker}{\pgfqpoint{0.000000in}{0.000000in}}{\pgfqpoint{0.000000in}{0.055556in}}{%
\pgfpathmoveto{\pgfqpoint{0.000000in}{0.000000in}}%
\pgfpathlineto{\pgfqpoint{0.000000in}{0.055556in}}%
\pgfusepath{stroke,fill}%
}%
\begin{pgfscope}%
\pgfsys@transformshift{3.480000in}{0.600000in}%
\pgfsys@useobject{currentmarker}{}%
\end{pgfscope}%
\end{pgfscope}%
\begin{pgfscope}%
\pgfsetbuttcap%
\pgfsetroundjoin%
\definecolor{currentfill}{rgb}{0.000000,0.000000,0.000000}%
\pgfsetfillcolor{currentfill}%
\pgfsetlinewidth{0.501875pt}%
\definecolor{currentstroke}{rgb}{0.000000,0.000000,0.000000}%
\pgfsetstrokecolor{currentstroke}%
\pgfsetdash{}{0pt}%
\pgfsys@defobject{currentmarker}{\pgfqpoint{0.000000in}{-0.055556in}}{\pgfqpoint{0.000000in}{0.000000in}}{%
\pgfpathmoveto{\pgfqpoint{0.000000in}{0.000000in}}%
\pgfpathlineto{\pgfqpoint{0.000000in}{-0.055556in}}%
\pgfusepath{stroke,fill}%
}%
\begin{pgfscope}%
\pgfsys@transformshift{3.480000in}{5.400000in}%
\pgfsys@useobject{currentmarker}{}%
\end{pgfscope}%
\end{pgfscope}%
\begin{pgfscope}%
\definecolor{textcolor}{rgb}{0.000000,0.000000,0.000000}%
\pgfsetstrokecolor{textcolor}%
\pgfsetfillcolor{textcolor}%
\pgftext[x=3.480000in,y=0.544444in,,top]{\color{textcolor}\rmfamily\fontsize{10.000000}{12.000000}\selectfont \(\displaystyle {40}\)}%
\end{pgfscope}%
\begin{pgfscope}%
\pgfpathrectangle{\pgfqpoint{1.000000in}{0.600000in}}{\pgfqpoint{6.200000in}{4.800000in}}%
\pgfusepath{clip}%
\pgfsetbuttcap%
\pgfsetroundjoin%
\pgfsetlinewidth{0.501875pt}%
\definecolor{currentstroke}{rgb}{0.000000,0.000000,0.000000}%
\pgfsetstrokecolor{currentstroke}%
\pgfsetdash{{1.000000pt}{3.000000pt}}{0.000000pt}%
\pgfpathmoveto{\pgfqpoint{4.720000in}{0.600000in}}%
\pgfpathlineto{\pgfqpoint{4.720000in}{5.400000in}}%
\pgfusepath{stroke}%
\end{pgfscope}%
\begin{pgfscope}%
\pgfsetbuttcap%
\pgfsetroundjoin%
\definecolor{currentfill}{rgb}{0.000000,0.000000,0.000000}%
\pgfsetfillcolor{currentfill}%
\pgfsetlinewidth{0.501875pt}%
\definecolor{currentstroke}{rgb}{0.000000,0.000000,0.000000}%
\pgfsetstrokecolor{currentstroke}%
\pgfsetdash{}{0pt}%
\pgfsys@defobject{currentmarker}{\pgfqpoint{0.000000in}{0.000000in}}{\pgfqpoint{0.000000in}{0.055556in}}{%
\pgfpathmoveto{\pgfqpoint{0.000000in}{0.000000in}}%
\pgfpathlineto{\pgfqpoint{0.000000in}{0.055556in}}%
\pgfusepath{stroke,fill}%
}%
\begin{pgfscope}%
\pgfsys@transformshift{4.720000in}{0.600000in}%
\pgfsys@useobject{currentmarker}{}%
\end{pgfscope}%
\end{pgfscope}%
\begin{pgfscope}%
\pgfsetbuttcap%
\pgfsetroundjoin%
\definecolor{currentfill}{rgb}{0.000000,0.000000,0.000000}%
\pgfsetfillcolor{currentfill}%
\pgfsetlinewidth{0.501875pt}%
\definecolor{currentstroke}{rgb}{0.000000,0.000000,0.000000}%
\pgfsetstrokecolor{currentstroke}%
\pgfsetdash{}{0pt}%
\pgfsys@defobject{currentmarker}{\pgfqpoint{0.000000in}{-0.055556in}}{\pgfqpoint{0.000000in}{0.000000in}}{%
\pgfpathmoveto{\pgfqpoint{0.000000in}{0.000000in}}%
\pgfpathlineto{\pgfqpoint{0.000000in}{-0.055556in}}%
\pgfusepath{stroke,fill}%
}%
\begin{pgfscope}%
\pgfsys@transformshift{4.720000in}{5.400000in}%
\pgfsys@useobject{currentmarker}{}%
\end{pgfscope}%
\end{pgfscope}%
\begin{pgfscope}%
\definecolor{textcolor}{rgb}{0.000000,0.000000,0.000000}%
\pgfsetstrokecolor{textcolor}%
\pgfsetfillcolor{textcolor}%
\pgftext[x=4.720000in,y=0.544444in,,top]{\color{textcolor}\rmfamily\fontsize{10.000000}{12.000000}\selectfont \(\displaystyle {60}\)}%
\end{pgfscope}%
\begin{pgfscope}%
\pgfpathrectangle{\pgfqpoint{1.000000in}{0.600000in}}{\pgfqpoint{6.200000in}{4.800000in}}%
\pgfusepath{clip}%
\pgfsetbuttcap%
\pgfsetroundjoin%
\pgfsetlinewidth{0.501875pt}%
\definecolor{currentstroke}{rgb}{0.000000,0.000000,0.000000}%
\pgfsetstrokecolor{currentstroke}%
\pgfsetdash{{1.000000pt}{3.000000pt}}{0.000000pt}%
\pgfpathmoveto{\pgfqpoint{5.960000in}{0.600000in}}%
\pgfpathlineto{\pgfqpoint{5.960000in}{5.400000in}}%
\pgfusepath{stroke}%
\end{pgfscope}%
\begin{pgfscope}%
\pgfsetbuttcap%
\pgfsetroundjoin%
\definecolor{currentfill}{rgb}{0.000000,0.000000,0.000000}%
\pgfsetfillcolor{currentfill}%
\pgfsetlinewidth{0.501875pt}%
\definecolor{currentstroke}{rgb}{0.000000,0.000000,0.000000}%
\pgfsetstrokecolor{currentstroke}%
\pgfsetdash{}{0pt}%
\pgfsys@defobject{currentmarker}{\pgfqpoint{0.000000in}{0.000000in}}{\pgfqpoint{0.000000in}{0.055556in}}{%
\pgfpathmoveto{\pgfqpoint{0.000000in}{0.000000in}}%
\pgfpathlineto{\pgfqpoint{0.000000in}{0.055556in}}%
\pgfusepath{stroke,fill}%
}%
\begin{pgfscope}%
\pgfsys@transformshift{5.960000in}{0.600000in}%
\pgfsys@useobject{currentmarker}{}%
\end{pgfscope}%
\end{pgfscope}%
\begin{pgfscope}%
\pgfsetbuttcap%
\pgfsetroundjoin%
\definecolor{currentfill}{rgb}{0.000000,0.000000,0.000000}%
\pgfsetfillcolor{currentfill}%
\pgfsetlinewidth{0.501875pt}%
\definecolor{currentstroke}{rgb}{0.000000,0.000000,0.000000}%
\pgfsetstrokecolor{currentstroke}%
\pgfsetdash{}{0pt}%
\pgfsys@defobject{currentmarker}{\pgfqpoint{0.000000in}{-0.055556in}}{\pgfqpoint{0.000000in}{0.000000in}}{%
\pgfpathmoveto{\pgfqpoint{0.000000in}{0.000000in}}%
\pgfpathlineto{\pgfqpoint{0.000000in}{-0.055556in}}%
\pgfusepath{stroke,fill}%
}%
\begin{pgfscope}%
\pgfsys@transformshift{5.960000in}{5.400000in}%
\pgfsys@useobject{currentmarker}{}%
\end{pgfscope}%
\end{pgfscope}%
\begin{pgfscope}%
\definecolor{textcolor}{rgb}{0.000000,0.000000,0.000000}%
\pgfsetstrokecolor{textcolor}%
\pgfsetfillcolor{textcolor}%
\pgftext[x=5.960000in,y=0.544444in,,top]{\color{textcolor}\rmfamily\fontsize{10.000000}{12.000000}\selectfont \(\displaystyle {80}\)}%
\end{pgfscope}%
\begin{pgfscope}%
\pgfpathrectangle{\pgfqpoint{1.000000in}{0.600000in}}{\pgfqpoint{6.200000in}{4.800000in}}%
\pgfusepath{clip}%
\pgfsetbuttcap%
\pgfsetroundjoin%
\pgfsetlinewidth{0.501875pt}%
\definecolor{currentstroke}{rgb}{0.000000,0.000000,0.000000}%
\pgfsetstrokecolor{currentstroke}%
\pgfsetdash{{1.000000pt}{3.000000pt}}{0.000000pt}%
\pgfpathmoveto{\pgfqpoint{7.200000in}{0.600000in}}%
\pgfpathlineto{\pgfqpoint{7.200000in}{5.400000in}}%
\pgfusepath{stroke}%
\end{pgfscope}%
\begin{pgfscope}%
\pgfsetbuttcap%
\pgfsetroundjoin%
\definecolor{currentfill}{rgb}{0.000000,0.000000,0.000000}%
\pgfsetfillcolor{currentfill}%
\pgfsetlinewidth{0.501875pt}%
\definecolor{currentstroke}{rgb}{0.000000,0.000000,0.000000}%
\pgfsetstrokecolor{currentstroke}%
\pgfsetdash{}{0pt}%
\pgfsys@defobject{currentmarker}{\pgfqpoint{0.000000in}{0.000000in}}{\pgfqpoint{0.000000in}{0.055556in}}{%
\pgfpathmoveto{\pgfqpoint{0.000000in}{0.000000in}}%
\pgfpathlineto{\pgfqpoint{0.000000in}{0.055556in}}%
\pgfusepath{stroke,fill}%
}%
\begin{pgfscope}%
\pgfsys@transformshift{7.200000in}{0.600000in}%
\pgfsys@useobject{currentmarker}{}%
\end{pgfscope}%
\end{pgfscope}%
\begin{pgfscope}%
\pgfsetbuttcap%
\pgfsetroundjoin%
\definecolor{currentfill}{rgb}{0.000000,0.000000,0.000000}%
\pgfsetfillcolor{currentfill}%
\pgfsetlinewidth{0.501875pt}%
\definecolor{currentstroke}{rgb}{0.000000,0.000000,0.000000}%
\pgfsetstrokecolor{currentstroke}%
\pgfsetdash{}{0pt}%
\pgfsys@defobject{currentmarker}{\pgfqpoint{0.000000in}{-0.055556in}}{\pgfqpoint{0.000000in}{0.000000in}}{%
\pgfpathmoveto{\pgfqpoint{0.000000in}{0.000000in}}%
\pgfpathlineto{\pgfqpoint{0.000000in}{-0.055556in}}%
\pgfusepath{stroke,fill}%
}%
\begin{pgfscope}%
\pgfsys@transformshift{7.200000in}{5.400000in}%
\pgfsys@useobject{currentmarker}{}%
\end{pgfscope}%
\end{pgfscope}%
\begin{pgfscope}%
\definecolor{textcolor}{rgb}{0.000000,0.000000,0.000000}%
\pgfsetstrokecolor{textcolor}%
\pgfsetfillcolor{textcolor}%
\pgftext[x=7.200000in,y=0.544444in,,top]{\color{textcolor}\rmfamily\fontsize{10.000000}{12.000000}\selectfont \(\displaystyle {100}\)}%
\end{pgfscope}%
\begin{pgfscope}%
\definecolor{textcolor}{rgb}{0.000000,0.000000,0.000000}%
\pgfsetstrokecolor{textcolor}%
\pgfsetfillcolor{textcolor}%
\pgftext[x=4.100000in,y=0.351543in,,top]{\color{textcolor}\rmfamily\fontsize{12.000000}{14.400000}\selectfont \(\displaystyle time\ (s)\)}%
\end{pgfscope}%
\begin{pgfscope}%
\pgfpathrectangle{\pgfqpoint{1.000000in}{0.600000in}}{\pgfqpoint{6.200000in}{4.800000in}}%
\pgfusepath{clip}%
\pgfsetbuttcap%
\pgfsetroundjoin%
\pgfsetlinewidth{0.501875pt}%
\definecolor{currentstroke}{rgb}{0.000000,0.000000,0.000000}%
\pgfsetstrokecolor{currentstroke}%
\pgfsetdash{{1.000000pt}{3.000000pt}}{0.000000pt}%
\pgfpathmoveto{\pgfqpoint{1.000000in}{0.600000in}}%
\pgfpathlineto{\pgfqpoint{7.200000in}{0.600000in}}%
\pgfusepath{stroke}%
\end{pgfscope}%
\begin{pgfscope}%
\pgfsetbuttcap%
\pgfsetroundjoin%
\definecolor{currentfill}{rgb}{0.000000,0.000000,0.000000}%
\pgfsetfillcolor{currentfill}%
\pgfsetlinewidth{0.501875pt}%
\definecolor{currentstroke}{rgb}{0.000000,0.000000,0.000000}%
\pgfsetstrokecolor{currentstroke}%
\pgfsetdash{}{0pt}%
\pgfsys@defobject{currentmarker}{\pgfqpoint{0.000000in}{0.000000in}}{\pgfqpoint{0.055556in}{0.000000in}}{%
\pgfpathmoveto{\pgfqpoint{0.000000in}{0.000000in}}%
\pgfpathlineto{\pgfqpoint{0.055556in}{0.000000in}}%
\pgfusepath{stroke,fill}%
}%
\begin{pgfscope}%
\pgfsys@transformshift{1.000000in}{0.600000in}%
\pgfsys@useobject{currentmarker}{}%
\end{pgfscope}%
\end{pgfscope}%
\begin{pgfscope}%
\pgfsetbuttcap%
\pgfsetroundjoin%
\definecolor{currentfill}{rgb}{0.000000,0.000000,0.000000}%
\pgfsetfillcolor{currentfill}%
\pgfsetlinewidth{0.501875pt}%
\definecolor{currentstroke}{rgb}{0.000000,0.000000,0.000000}%
\pgfsetstrokecolor{currentstroke}%
\pgfsetdash{}{0pt}%
\pgfsys@defobject{currentmarker}{\pgfqpoint{-0.055556in}{0.000000in}}{\pgfqpoint{-0.000000in}{0.000000in}}{%
\pgfpathmoveto{\pgfqpoint{-0.000000in}{0.000000in}}%
\pgfpathlineto{\pgfqpoint{-0.055556in}{0.000000in}}%
\pgfusepath{stroke,fill}%
}%
\begin{pgfscope}%
\pgfsys@transformshift{7.200000in}{0.600000in}%
\pgfsys@useobject{currentmarker}{}%
\end{pgfscope}%
\end{pgfscope}%
\begin{pgfscope}%
\definecolor{textcolor}{rgb}{0.000000,0.000000,0.000000}%
\pgfsetstrokecolor{textcolor}%
\pgfsetfillcolor{textcolor}%
\pgftext[x=0.944444in,y=0.600000in,right,]{\color{textcolor}\rmfamily\fontsize{10.000000}{12.000000}\selectfont \(\displaystyle {\ensuremath{-}400}\)}%
\end{pgfscope}%
\begin{pgfscope}%
\pgfpathrectangle{\pgfqpoint{1.000000in}{0.600000in}}{\pgfqpoint{6.200000in}{4.800000in}}%
\pgfusepath{clip}%
\pgfsetbuttcap%
\pgfsetroundjoin%
\pgfsetlinewidth{0.501875pt}%
\definecolor{currentstroke}{rgb}{0.000000,0.000000,0.000000}%
\pgfsetstrokecolor{currentstroke}%
\pgfsetdash{{1.000000pt}{3.000000pt}}{0.000000pt}%
\pgfpathmoveto{\pgfqpoint{1.000000in}{1.133333in}}%
\pgfpathlineto{\pgfqpoint{7.200000in}{1.133333in}}%
\pgfusepath{stroke}%
\end{pgfscope}%
\begin{pgfscope}%
\pgfsetbuttcap%
\pgfsetroundjoin%
\definecolor{currentfill}{rgb}{0.000000,0.000000,0.000000}%
\pgfsetfillcolor{currentfill}%
\pgfsetlinewidth{0.501875pt}%
\definecolor{currentstroke}{rgb}{0.000000,0.000000,0.000000}%
\pgfsetstrokecolor{currentstroke}%
\pgfsetdash{}{0pt}%
\pgfsys@defobject{currentmarker}{\pgfqpoint{0.000000in}{0.000000in}}{\pgfqpoint{0.055556in}{0.000000in}}{%
\pgfpathmoveto{\pgfqpoint{0.000000in}{0.000000in}}%
\pgfpathlineto{\pgfqpoint{0.055556in}{0.000000in}}%
\pgfusepath{stroke,fill}%
}%
\begin{pgfscope}%
\pgfsys@transformshift{1.000000in}{1.133333in}%
\pgfsys@useobject{currentmarker}{}%
\end{pgfscope}%
\end{pgfscope}%
\begin{pgfscope}%
\pgfsetbuttcap%
\pgfsetroundjoin%
\definecolor{currentfill}{rgb}{0.000000,0.000000,0.000000}%
\pgfsetfillcolor{currentfill}%
\pgfsetlinewidth{0.501875pt}%
\definecolor{currentstroke}{rgb}{0.000000,0.000000,0.000000}%
\pgfsetstrokecolor{currentstroke}%
\pgfsetdash{}{0pt}%
\pgfsys@defobject{currentmarker}{\pgfqpoint{-0.055556in}{0.000000in}}{\pgfqpoint{-0.000000in}{0.000000in}}{%
\pgfpathmoveto{\pgfqpoint{-0.000000in}{0.000000in}}%
\pgfpathlineto{\pgfqpoint{-0.055556in}{0.000000in}}%
\pgfusepath{stroke,fill}%
}%
\begin{pgfscope}%
\pgfsys@transformshift{7.200000in}{1.133333in}%
\pgfsys@useobject{currentmarker}{}%
\end{pgfscope}%
\end{pgfscope}%
\begin{pgfscope}%
\definecolor{textcolor}{rgb}{0.000000,0.000000,0.000000}%
\pgfsetstrokecolor{textcolor}%
\pgfsetfillcolor{textcolor}%
\pgftext[x=0.944444in,y=1.133333in,right,]{\color{textcolor}\rmfamily\fontsize{10.000000}{12.000000}\selectfont \(\displaystyle {\ensuremath{-}350}\)}%
\end{pgfscope}%
\begin{pgfscope}%
\pgfpathrectangle{\pgfqpoint{1.000000in}{0.600000in}}{\pgfqpoint{6.200000in}{4.800000in}}%
\pgfusepath{clip}%
\pgfsetbuttcap%
\pgfsetroundjoin%
\pgfsetlinewidth{0.501875pt}%
\definecolor{currentstroke}{rgb}{0.000000,0.000000,0.000000}%
\pgfsetstrokecolor{currentstroke}%
\pgfsetdash{{1.000000pt}{3.000000pt}}{0.000000pt}%
\pgfpathmoveto{\pgfqpoint{1.000000in}{1.666667in}}%
\pgfpathlineto{\pgfqpoint{7.200000in}{1.666667in}}%
\pgfusepath{stroke}%
\end{pgfscope}%
\begin{pgfscope}%
\pgfsetbuttcap%
\pgfsetroundjoin%
\definecolor{currentfill}{rgb}{0.000000,0.000000,0.000000}%
\pgfsetfillcolor{currentfill}%
\pgfsetlinewidth{0.501875pt}%
\definecolor{currentstroke}{rgb}{0.000000,0.000000,0.000000}%
\pgfsetstrokecolor{currentstroke}%
\pgfsetdash{}{0pt}%
\pgfsys@defobject{currentmarker}{\pgfqpoint{0.000000in}{0.000000in}}{\pgfqpoint{0.055556in}{0.000000in}}{%
\pgfpathmoveto{\pgfqpoint{0.000000in}{0.000000in}}%
\pgfpathlineto{\pgfqpoint{0.055556in}{0.000000in}}%
\pgfusepath{stroke,fill}%
}%
\begin{pgfscope}%
\pgfsys@transformshift{1.000000in}{1.666667in}%
\pgfsys@useobject{currentmarker}{}%
\end{pgfscope}%
\end{pgfscope}%
\begin{pgfscope}%
\pgfsetbuttcap%
\pgfsetroundjoin%
\definecolor{currentfill}{rgb}{0.000000,0.000000,0.000000}%
\pgfsetfillcolor{currentfill}%
\pgfsetlinewidth{0.501875pt}%
\definecolor{currentstroke}{rgb}{0.000000,0.000000,0.000000}%
\pgfsetstrokecolor{currentstroke}%
\pgfsetdash{}{0pt}%
\pgfsys@defobject{currentmarker}{\pgfqpoint{-0.055556in}{0.000000in}}{\pgfqpoint{-0.000000in}{0.000000in}}{%
\pgfpathmoveto{\pgfqpoint{-0.000000in}{0.000000in}}%
\pgfpathlineto{\pgfqpoint{-0.055556in}{0.000000in}}%
\pgfusepath{stroke,fill}%
}%
\begin{pgfscope}%
\pgfsys@transformshift{7.200000in}{1.666667in}%
\pgfsys@useobject{currentmarker}{}%
\end{pgfscope}%
\end{pgfscope}%
\begin{pgfscope}%
\definecolor{textcolor}{rgb}{0.000000,0.000000,0.000000}%
\pgfsetstrokecolor{textcolor}%
\pgfsetfillcolor{textcolor}%
\pgftext[x=0.944444in,y=1.666667in,right,]{\color{textcolor}\rmfamily\fontsize{10.000000}{12.000000}\selectfont \(\displaystyle {\ensuremath{-}300}\)}%
\end{pgfscope}%
\begin{pgfscope}%
\pgfpathrectangle{\pgfqpoint{1.000000in}{0.600000in}}{\pgfqpoint{6.200000in}{4.800000in}}%
\pgfusepath{clip}%
\pgfsetbuttcap%
\pgfsetroundjoin%
\pgfsetlinewidth{0.501875pt}%
\definecolor{currentstroke}{rgb}{0.000000,0.000000,0.000000}%
\pgfsetstrokecolor{currentstroke}%
\pgfsetdash{{1.000000pt}{3.000000pt}}{0.000000pt}%
\pgfpathmoveto{\pgfqpoint{1.000000in}{2.200000in}}%
\pgfpathlineto{\pgfqpoint{7.200000in}{2.200000in}}%
\pgfusepath{stroke}%
\end{pgfscope}%
\begin{pgfscope}%
\pgfsetbuttcap%
\pgfsetroundjoin%
\definecolor{currentfill}{rgb}{0.000000,0.000000,0.000000}%
\pgfsetfillcolor{currentfill}%
\pgfsetlinewidth{0.501875pt}%
\definecolor{currentstroke}{rgb}{0.000000,0.000000,0.000000}%
\pgfsetstrokecolor{currentstroke}%
\pgfsetdash{}{0pt}%
\pgfsys@defobject{currentmarker}{\pgfqpoint{0.000000in}{0.000000in}}{\pgfqpoint{0.055556in}{0.000000in}}{%
\pgfpathmoveto{\pgfqpoint{0.000000in}{0.000000in}}%
\pgfpathlineto{\pgfqpoint{0.055556in}{0.000000in}}%
\pgfusepath{stroke,fill}%
}%
\begin{pgfscope}%
\pgfsys@transformshift{1.000000in}{2.200000in}%
\pgfsys@useobject{currentmarker}{}%
\end{pgfscope}%
\end{pgfscope}%
\begin{pgfscope}%
\pgfsetbuttcap%
\pgfsetroundjoin%
\definecolor{currentfill}{rgb}{0.000000,0.000000,0.000000}%
\pgfsetfillcolor{currentfill}%
\pgfsetlinewidth{0.501875pt}%
\definecolor{currentstroke}{rgb}{0.000000,0.000000,0.000000}%
\pgfsetstrokecolor{currentstroke}%
\pgfsetdash{}{0pt}%
\pgfsys@defobject{currentmarker}{\pgfqpoint{-0.055556in}{0.000000in}}{\pgfqpoint{-0.000000in}{0.000000in}}{%
\pgfpathmoveto{\pgfqpoint{-0.000000in}{0.000000in}}%
\pgfpathlineto{\pgfqpoint{-0.055556in}{0.000000in}}%
\pgfusepath{stroke,fill}%
}%
\begin{pgfscope}%
\pgfsys@transformshift{7.200000in}{2.200000in}%
\pgfsys@useobject{currentmarker}{}%
\end{pgfscope}%
\end{pgfscope}%
\begin{pgfscope}%
\definecolor{textcolor}{rgb}{0.000000,0.000000,0.000000}%
\pgfsetstrokecolor{textcolor}%
\pgfsetfillcolor{textcolor}%
\pgftext[x=0.944444in,y=2.200000in,right,]{\color{textcolor}\rmfamily\fontsize{10.000000}{12.000000}\selectfont \(\displaystyle {\ensuremath{-}250}\)}%
\end{pgfscope}%
\begin{pgfscope}%
\pgfpathrectangle{\pgfqpoint{1.000000in}{0.600000in}}{\pgfqpoint{6.200000in}{4.800000in}}%
\pgfusepath{clip}%
\pgfsetbuttcap%
\pgfsetroundjoin%
\pgfsetlinewidth{0.501875pt}%
\definecolor{currentstroke}{rgb}{0.000000,0.000000,0.000000}%
\pgfsetstrokecolor{currentstroke}%
\pgfsetdash{{1.000000pt}{3.000000pt}}{0.000000pt}%
\pgfpathmoveto{\pgfqpoint{1.000000in}{2.733333in}}%
\pgfpathlineto{\pgfqpoint{7.200000in}{2.733333in}}%
\pgfusepath{stroke}%
\end{pgfscope}%
\begin{pgfscope}%
\pgfsetbuttcap%
\pgfsetroundjoin%
\definecolor{currentfill}{rgb}{0.000000,0.000000,0.000000}%
\pgfsetfillcolor{currentfill}%
\pgfsetlinewidth{0.501875pt}%
\definecolor{currentstroke}{rgb}{0.000000,0.000000,0.000000}%
\pgfsetstrokecolor{currentstroke}%
\pgfsetdash{}{0pt}%
\pgfsys@defobject{currentmarker}{\pgfqpoint{0.000000in}{0.000000in}}{\pgfqpoint{0.055556in}{0.000000in}}{%
\pgfpathmoveto{\pgfqpoint{0.000000in}{0.000000in}}%
\pgfpathlineto{\pgfqpoint{0.055556in}{0.000000in}}%
\pgfusepath{stroke,fill}%
}%
\begin{pgfscope}%
\pgfsys@transformshift{1.000000in}{2.733333in}%
\pgfsys@useobject{currentmarker}{}%
\end{pgfscope}%
\end{pgfscope}%
\begin{pgfscope}%
\pgfsetbuttcap%
\pgfsetroundjoin%
\definecolor{currentfill}{rgb}{0.000000,0.000000,0.000000}%
\pgfsetfillcolor{currentfill}%
\pgfsetlinewidth{0.501875pt}%
\definecolor{currentstroke}{rgb}{0.000000,0.000000,0.000000}%
\pgfsetstrokecolor{currentstroke}%
\pgfsetdash{}{0pt}%
\pgfsys@defobject{currentmarker}{\pgfqpoint{-0.055556in}{0.000000in}}{\pgfqpoint{-0.000000in}{0.000000in}}{%
\pgfpathmoveto{\pgfqpoint{-0.000000in}{0.000000in}}%
\pgfpathlineto{\pgfqpoint{-0.055556in}{0.000000in}}%
\pgfusepath{stroke,fill}%
}%
\begin{pgfscope}%
\pgfsys@transformshift{7.200000in}{2.733333in}%
\pgfsys@useobject{currentmarker}{}%
\end{pgfscope}%
\end{pgfscope}%
\begin{pgfscope}%
\definecolor{textcolor}{rgb}{0.000000,0.000000,0.000000}%
\pgfsetstrokecolor{textcolor}%
\pgfsetfillcolor{textcolor}%
\pgftext[x=0.944444in,y=2.733333in,right,]{\color{textcolor}\rmfamily\fontsize{10.000000}{12.000000}\selectfont \(\displaystyle {\ensuremath{-}200}\)}%
\end{pgfscope}%
\begin{pgfscope}%
\pgfpathrectangle{\pgfqpoint{1.000000in}{0.600000in}}{\pgfqpoint{6.200000in}{4.800000in}}%
\pgfusepath{clip}%
\pgfsetbuttcap%
\pgfsetroundjoin%
\pgfsetlinewidth{0.501875pt}%
\definecolor{currentstroke}{rgb}{0.000000,0.000000,0.000000}%
\pgfsetstrokecolor{currentstroke}%
\pgfsetdash{{1.000000pt}{3.000000pt}}{0.000000pt}%
\pgfpathmoveto{\pgfqpoint{1.000000in}{3.266667in}}%
\pgfpathlineto{\pgfqpoint{7.200000in}{3.266667in}}%
\pgfusepath{stroke}%
\end{pgfscope}%
\begin{pgfscope}%
\pgfsetbuttcap%
\pgfsetroundjoin%
\definecolor{currentfill}{rgb}{0.000000,0.000000,0.000000}%
\pgfsetfillcolor{currentfill}%
\pgfsetlinewidth{0.501875pt}%
\definecolor{currentstroke}{rgb}{0.000000,0.000000,0.000000}%
\pgfsetstrokecolor{currentstroke}%
\pgfsetdash{}{0pt}%
\pgfsys@defobject{currentmarker}{\pgfqpoint{0.000000in}{0.000000in}}{\pgfqpoint{0.055556in}{0.000000in}}{%
\pgfpathmoveto{\pgfqpoint{0.000000in}{0.000000in}}%
\pgfpathlineto{\pgfqpoint{0.055556in}{0.000000in}}%
\pgfusepath{stroke,fill}%
}%
\begin{pgfscope}%
\pgfsys@transformshift{1.000000in}{3.266667in}%
\pgfsys@useobject{currentmarker}{}%
\end{pgfscope}%
\end{pgfscope}%
\begin{pgfscope}%
\pgfsetbuttcap%
\pgfsetroundjoin%
\definecolor{currentfill}{rgb}{0.000000,0.000000,0.000000}%
\pgfsetfillcolor{currentfill}%
\pgfsetlinewidth{0.501875pt}%
\definecolor{currentstroke}{rgb}{0.000000,0.000000,0.000000}%
\pgfsetstrokecolor{currentstroke}%
\pgfsetdash{}{0pt}%
\pgfsys@defobject{currentmarker}{\pgfqpoint{-0.055556in}{0.000000in}}{\pgfqpoint{-0.000000in}{0.000000in}}{%
\pgfpathmoveto{\pgfqpoint{-0.000000in}{0.000000in}}%
\pgfpathlineto{\pgfqpoint{-0.055556in}{0.000000in}}%
\pgfusepath{stroke,fill}%
}%
\begin{pgfscope}%
\pgfsys@transformshift{7.200000in}{3.266667in}%
\pgfsys@useobject{currentmarker}{}%
\end{pgfscope}%
\end{pgfscope}%
\begin{pgfscope}%
\definecolor{textcolor}{rgb}{0.000000,0.000000,0.000000}%
\pgfsetstrokecolor{textcolor}%
\pgfsetfillcolor{textcolor}%
\pgftext[x=0.944444in,y=3.266667in,right,]{\color{textcolor}\rmfamily\fontsize{10.000000}{12.000000}\selectfont \(\displaystyle {\ensuremath{-}150}\)}%
\end{pgfscope}%
\begin{pgfscope}%
\pgfpathrectangle{\pgfqpoint{1.000000in}{0.600000in}}{\pgfqpoint{6.200000in}{4.800000in}}%
\pgfusepath{clip}%
\pgfsetbuttcap%
\pgfsetroundjoin%
\pgfsetlinewidth{0.501875pt}%
\definecolor{currentstroke}{rgb}{0.000000,0.000000,0.000000}%
\pgfsetstrokecolor{currentstroke}%
\pgfsetdash{{1.000000pt}{3.000000pt}}{0.000000pt}%
\pgfpathmoveto{\pgfqpoint{1.000000in}{3.800000in}}%
\pgfpathlineto{\pgfqpoint{7.200000in}{3.800000in}}%
\pgfusepath{stroke}%
\end{pgfscope}%
\begin{pgfscope}%
\pgfsetbuttcap%
\pgfsetroundjoin%
\definecolor{currentfill}{rgb}{0.000000,0.000000,0.000000}%
\pgfsetfillcolor{currentfill}%
\pgfsetlinewidth{0.501875pt}%
\definecolor{currentstroke}{rgb}{0.000000,0.000000,0.000000}%
\pgfsetstrokecolor{currentstroke}%
\pgfsetdash{}{0pt}%
\pgfsys@defobject{currentmarker}{\pgfqpoint{0.000000in}{0.000000in}}{\pgfqpoint{0.055556in}{0.000000in}}{%
\pgfpathmoveto{\pgfqpoint{0.000000in}{0.000000in}}%
\pgfpathlineto{\pgfqpoint{0.055556in}{0.000000in}}%
\pgfusepath{stroke,fill}%
}%
\begin{pgfscope}%
\pgfsys@transformshift{1.000000in}{3.800000in}%
\pgfsys@useobject{currentmarker}{}%
\end{pgfscope}%
\end{pgfscope}%
\begin{pgfscope}%
\pgfsetbuttcap%
\pgfsetroundjoin%
\definecolor{currentfill}{rgb}{0.000000,0.000000,0.000000}%
\pgfsetfillcolor{currentfill}%
\pgfsetlinewidth{0.501875pt}%
\definecolor{currentstroke}{rgb}{0.000000,0.000000,0.000000}%
\pgfsetstrokecolor{currentstroke}%
\pgfsetdash{}{0pt}%
\pgfsys@defobject{currentmarker}{\pgfqpoint{-0.055556in}{0.000000in}}{\pgfqpoint{-0.000000in}{0.000000in}}{%
\pgfpathmoveto{\pgfqpoint{-0.000000in}{0.000000in}}%
\pgfpathlineto{\pgfqpoint{-0.055556in}{0.000000in}}%
\pgfusepath{stroke,fill}%
}%
\begin{pgfscope}%
\pgfsys@transformshift{7.200000in}{3.800000in}%
\pgfsys@useobject{currentmarker}{}%
\end{pgfscope}%
\end{pgfscope}%
\begin{pgfscope}%
\definecolor{textcolor}{rgb}{0.000000,0.000000,0.000000}%
\pgfsetstrokecolor{textcolor}%
\pgfsetfillcolor{textcolor}%
\pgftext[x=0.944444in,y=3.800000in,right,]{\color{textcolor}\rmfamily\fontsize{10.000000}{12.000000}\selectfont \(\displaystyle {\ensuremath{-}100}\)}%
\end{pgfscope}%
\begin{pgfscope}%
\pgfpathrectangle{\pgfqpoint{1.000000in}{0.600000in}}{\pgfqpoint{6.200000in}{4.800000in}}%
\pgfusepath{clip}%
\pgfsetbuttcap%
\pgfsetroundjoin%
\pgfsetlinewidth{0.501875pt}%
\definecolor{currentstroke}{rgb}{0.000000,0.000000,0.000000}%
\pgfsetstrokecolor{currentstroke}%
\pgfsetdash{{1.000000pt}{3.000000pt}}{0.000000pt}%
\pgfpathmoveto{\pgfqpoint{1.000000in}{4.333333in}}%
\pgfpathlineto{\pgfqpoint{7.200000in}{4.333333in}}%
\pgfusepath{stroke}%
\end{pgfscope}%
\begin{pgfscope}%
\pgfsetbuttcap%
\pgfsetroundjoin%
\definecolor{currentfill}{rgb}{0.000000,0.000000,0.000000}%
\pgfsetfillcolor{currentfill}%
\pgfsetlinewidth{0.501875pt}%
\definecolor{currentstroke}{rgb}{0.000000,0.000000,0.000000}%
\pgfsetstrokecolor{currentstroke}%
\pgfsetdash{}{0pt}%
\pgfsys@defobject{currentmarker}{\pgfqpoint{0.000000in}{0.000000in}}{\pgfqpoint{0.055556in}{0.000000in}}{%
\pgfpathmoveto{\pgfqpoint{0.000000in}{0.000000in}}%
\pgfpathlineto{\pgfqpoint{0.055556in}{0.000000in}}%
\pgfusepath{stroke,fill}%
}%
\begin{pgfscope}%
\pgfsys@transformshift{1.000000in}{4.333333in}%
\pgfsys@useobject{currentmarker}{}%
\end{pgfscope}%
\end{pgfscope}%
\begin{pgfscope}%
\pgfsetbuttcap%
\pgfsetroundjoin%
\definecolor{currentfill}{rgb}{0.000000,0.000000,0.000000}%
\pgfsetfillcolor{currentfill}%
\pgfsetlinewidth{0.501875pt}%
\definecolor{currentstroke}{rgb}{0.000000,0.000000,0.000000}%
\pgfsetstrokecolor{currentstroke}%
\pgfsetdash{}{0pt}%
\pgfsys@defobject{currentmarker}{\pgfqpoint{-0.055556in}{0.000000in}}{\pgfqpoint{-0.000000in}{0.000000in}}{%
\pgfpathmoveto{\pgfqpoint{-0.000000in}{0.000000in}}%
\pgfpathlineto{\pgfqpoint{-0.055556in}{0.000000in}}%
\pgfusepath{stroke,fill}%
}%
\begin{pgfscope}%
\pgfsys@transformshift{7.200000in}{4.333333in}%
\pgfsys@useobject{currentmarker}{}%
\end{pgfscope}%
\end{pgfscope}%
\begin{pgfscope}%
\definecolor{textcolor}{rgb}{0.000000,0.000000,0.000000}%
\pgfsetstrokecolor{textcolor}%
\pgfsetfillcolor{textcolor}%
\pgftext[x=0.944444in,y=4.333333in,right,]{\color{textcolor}\rmfamily\fontsize{10.000000}{12.000000}\selectfont \(\displaystyle {\ensuremath{-}50}\)}%
\end{pgfscope}%
\begin{pgfscope}%
\pgfpathrectangle{\pgfqpoint{1.000000in}{0.600000in}}{\pgfqpoint{6.200000in}{4.800000in}}%
\pgfusepath{clip}%
\pgfsetbuttcap%
\pgfsetroundjoin%
\pgfsetlinewidth{0.501875pt}%
\definecolor{currentstroke}{rgb}{0.000000,0.000000,0.000000}%
\pgfsetstrokecolor{currentstroke}%
\pgfsetdash{{1.000000pt}{3.000000pt}}{0.000000pt}%
\pgfpathmoveto{\pgfqpoint{1.000000in}{4.866667in}}%
\pgfpathlineto{\pgfqpoint{7.200000in}{4.866667in}}%
\pgfusepath{stroke}%
\end{pgfscope}%
\begin{pgfscope}%
\pgfsetbuttcap%
\pgfsetroundjoin%
\definecolor{currentfill}{rgb}{0.000000,0.000000,0.000000}%
\pgfsetfillcolor{currentfill}%
\pgfsetlinewidth{0.501875pt}%
\definecolor{currentstroke}{rgb}{0.000000,0.000000,0.000000}%
\pgfsetstrokecolor{currentstroke}%
\pgfsetdash{}{0pt}%
\pgfsys@defobject{currentmarker}{\pgfqpoint{0.000000in}{0.000000in}}{\pgfqpoint{0.055556in}{0.000000in}}{%
\pgfpathmoveto{\pgfqpoint{0.000000in}{0.000000in}}%
\pgfpathlineto{\pgfqpoint{0.055556in}{0.000000in}}%
\pgfusepath{stroke,fill}%
}%
\begin{pgfscope}%
\pgfsys@transformshift{1.000000in}{4.866667in}%
\pgfsys@useobject{currentmarker}{}%
\end{pgfscope}%
\end{pgfscope}%
\begin{pgfscope}%
\pgfsetbuttcap%
\pgfsetroundjoin%
\definecolor{currentfill}{rgb}{0.000000,0.000000,0.000000}%
\pgfsetfillcolor{currentfill}%
\pgfsetlinewidth{0.501875pt}%
\definecolor{currentstroke}{rgb}{0.000000,0.000000,0.000000}%
\pgfsetstrokecolor{currentstroke}%
\pgfsetdash{}{0pt}%
\pgfsys@defobject{currentmarker}{\pgfqpoint{-0.055556in}{0.000000in}}{\pgfqpoint{-0.000000in}{0.000000in}}{%
\pgfpathmoveto{\pgfqpoint{-0.000000in}{0.000000in}}%
\pgfpathlineto{\pgfqpoint{-0.055556in}{0.000000in}}%
\pgfusepath{stroke,fill}%
}%
\begin{pgfscope}%
\pgfsys@transformshift{7.200000in}{4.866667in}%
\pgfsys@useobject{currentmarker}{}%
\end{pgfscope}%
\end{pgfscope}%
\begin{pgfscope}%
\definecolor{textcolor}{rgb}{0.000000,0.000000,0.000000}%
\pgfsetstrokecolor{textcolor}%
\pgfsetfillcolor{textcolor}%
\pgftext[x=0.944444in,y=4.866667in,right,]{\color{textcolor}\rmfamily\fontsize{10.000000}{12.000000}\selectfont \(\displaystyle {0}\)}%
\end{pgfscope}%
\begin{pgfscope}%
\pgfpathrectangle{\pgfqpoint{1.000000in}{0.600000in}}{\pgfqpoint{6.200000in}{4.800000in}}%
\pgfusepath{clip}%
\pgfsetbuttcap%
\pgfsetroundjoin%
\pgfsetlinewidth{0.501875pt}%
\definecolor{currentstroke}{rgb}{0.000000,0.000000,0.000000}%
\pgfsetstrokecolor{currentstroke}%
\pgfsetdash{{1.000000pt}{3.000000pt}}{0.000000pt}%
\pgfpathmoveto{\pgfqpoint{1.000000in}{5.400000in}}%
\pgfpathlineto{\pgfqpoint{7.200000in}{5.400000in}}%
\pgfusepath{stroke}%
\end{pgfscope}%
\begin{pgfscope}%
\pgfsetbuttcap%
\pgfsetroundjoin%
\definecolor{currentfill}{rgb}{0.000000,0.000000,0.000000}%
\pgfsetfillcolor{currentfill}%
\pgfsetlinewidth{0.501875pt}%
\definecolor{currentstroke}{rgb}{0.000000,0.000000,0.000000}%
\pgfsetstrokecolor{currentstroke}%
\pgfsetdash{}{0pt}%
\pgfsys@defobject{currentmarker}{\pgfqpoint{0.000000in}{0.000000in}}{\pgfqpoint{0.055556in}{0.000000in}}{%
\pgfpathmoveto{\pgfqpoint{0.000000in}{0.000000in}}%
\pgfpathlineto{\pgfqpoint{0.055556in}{0.000000in}}%
\pgfusepath{stroke,fill}%
}%
\begin{pgfscope}%
\pgfsys@transformshift{1.000000in}{5.400000in}%
\pgfsys@useobject{currentmarker}{}%
\end{pgfscope}%
\end{pgfscope}%
\begin{pgfscope}%
\pgfsetbuttcap%
\pgfsetroundjoin%
\definecolor{currentfill}{rgb}{0.000000,0.000000,0.000000}%
\pgfsetfillcolor{currentfill}%
\pgfsetlinewidth{0.501875pt}%
\definecolor{currentstroke}{rgb}{0.000000,0.000000,0.000000}%
\pgfsetstrokecolor{currentstroke}%
\pgfsetdash{}{0pt}%
\pgfsys@defobject{currentmarker}{\pgfqpoint{-0.055556in}{0.000000in}}{\pgfqpoint{-0.000000in}{0.000000in}}{%
\pgfpathmoveto{\pgfqpoint{-0.000000in}{0.000000in}}%
\pgfpathlineto{\pgfqpoint{-0.055556in}{0.000000in}}%
\pgfusepath{stroke,fill}%
}%
\begin{pgfscope}%
\pgfsys@transformshift{7.200000in}{5.400000in}%
\pgfsys@useobject{currentmarker}{}%
\end{pgfscope}%
\end{pgfscope}%
\begin{pgfscope}%
\definecolor{textcolor}{rgb}{0.000000,0.000000,0.000000}%
\pgfsetstrokecolor{textcolor}%
\pgfsetfillcolor{textcolor}%
\pgftext[x=0.944444in,y=5.400000in,right,]{\color{textcolor}\rmfamily\fontsize{10.000000}{12.000000}\selectfont \(\displaystyle {50}\)}%
\end{pgfscope}%
\begin{pgfscope}%
\definecolor{textcolor}{rgb}{0.000000,0.000000,0.000000}%
\pgfsetstrokecolor{textcolor}%
\pgfsetfillcolor{textcolor}%
\pgftext[x=0.558641in,y=3.000000in,,bottom,rotate=90.000000]{\color{textcolor}\rmfamily\fontsize{12.000000}{14.400000}\selectfont \(\displaystyle \theta\ (rad)\)}%
\end{pgfscope}%
\begin{pgfscope}%
\definecolor{textcolor}{rgb}{0.000000,0.000000,0.000000}%
\pgfsetstrokecolor{textcolor}%
\pgfsetfillcolor{textcolor}%
\pgftext[x=4.100000in,y=5.469444in,,base]{\color{textcolor}\rmfamily\fontsize{12.000000}{14.400000}\selectfont \(\displaystyle Simple\ pendulum\ solution\ (time\ step = 0.5\ (s))\)}%
\end{pgfscope}%
\begin{pgfscope}%
\pgfsetbuttcap%
\pgfsetmiterjoin%
\definecolor{currentfill}{rgb}{1.000000,1.000000,1.000000}%
\pgfsetfillcolor{currentfill}%
\pgfsetlinewidth{1.003750pt}%
\definecolor{currentstroke}{rgb}{0.000000,0.000000,0.000000}%
\pgfsetstrokecolor{currentstroke}%
\pgfsetdash{}{0pt}%
\pgfpathmoveto{\pgfqpoint{1.083333in}{0.683333in}}%
\pgfpathlineto{\pgfqpoint{3.093110in}{0.683333in}}%
\pgfpathlineto{\pgfqpoint{3.093110in}{1.430555in}}%
\pgfpathlineto{\pgfqpoint{1.083333in}{1.430555in}}%
\pgfpathlineto{\pgfqpoint{1.083333in}{0.683333in}}%
\pgfpathclose%
\pgfusepath{stroke,fill}%
\end{pgfscope}%
\begin{pgfscope}%
\pgfsetrectcap%
\pgfsetroundjoin%
\pgfsetlinewidth{1.003750pt}%
\definecolor{currentstroke}{rgb}{1.000000,0.000000,0.000000}%
\pgfsetstrokecolor{currentstroke}%
\pgfsetdash{}{0pt}%
\pgfpathmoveto{\pgfqpoint{1.200000in}{1.305555in}}%
\pgfpathlineto{\pgfqpoint{1.433333in}{1.305555in}}%
\pgfusepath{stroke}%
\end{pgfscope}%
\begin{pgfscope}%
\definecolor{textcolor}{rgb}{0.000000,0.000000,0.000000}%
\pgfsetstrokecolor{textcolor}%
\pgfsetfillcolor{textcolor}%
\pgftext[x=1.616667in,y=1.247221in,left,base]{\color{textcolor}\rmfamily\fontsize{12.000000}{14.400000}\selectfont \(\displaystyle euler\ explicit\)}%
\end{pgfscope}%
\begin{pgfscope}%
\pgfsetrectcap%
\pgfsetroundjoin%
\pgfsetlinewidth{1.003750pt}%
\definecolor{currentstroke}{rgb}{0.000000,0.000000,1.000000}%
\pgfsetstrokecolor{currentstroke}%
\pgfsetdash{}{0pt}%
\pgfpathmoveto{\pgfqpoint{1.200000in}{1.073147in}}%
\pgfpathlineto{\pgfqpoint{1.433333in}{1.073147in}}%
\pgfusepath{stroke}%
\end{pgfscope}%
\begin{pgfscope}%
\definecolor{textcolor}{rgb}{0.000000,0.000000,0.000000}%
\pgfsetstrokecolor{textcolor}%
\pgfsetfillcolor{textcolor}%
\pgftext[x=1.616667in,y=1.014814in,left,base]{\color{textcolor}\rmfamily\fontsize{12.000000}{14.400000}\selectfont \(\displaystyle euler\ implicit\)}%
\end{pgfscope}%
\begin{pgfscope}%
\pgfsetrectcap%
\pgfsetroundjoin%
\pgfsetlinewidth{1.003750pt}%
\definecolor{currentstroke}{rgb}{0.000000,0.000000,0.000000}%
\pgfsetstrokecolor{currentstroke}%
\pgfsetdash{}{0pt}%
\pgfpathmoveto{\pgfqpoint{1.200000in}{0.840740in}}%
\pgfpathlineto{\pgfqpoint{1.433333in}{0.840740in}}%
\pgfusepath{stroke}%
\end{pgfscope}%
\begin{pgfscope}%
\definecolor{textcolor}{rgb}{0.000000,0.000000,0.000000}%
\pgfsetstrokecolor{textcolor}%
\pgfsetfillcolor{textcolor}%
\pgftext[x=1.616667in,y=0.782407in,left,base]{\color{textcolor}\rmfamily\fontsize{12.000000}{14.400000}\selectfont \(\displaystyle trapezoidal\ scheme\)}%
\end{pgfscope}%
\end{pgfpicture}%
\makeatother%
\endgroup%
}
    \end{figure}

    \begin{figure}[ht!]
    \centering
    \resizebox{0.9\linewidth}{!}{%% Creator: Matplotlib, PGF backend
%%
%% To include the figure in your LaTeX document, write
%%   \input{<filename>.pgf}
%%
%% Make sure the required packages are loaded in your preamble
%%   \usepackage{pgf}
%%
%% Also ensure that all the required font packages are loaded; for instance,
%% the lmodern package is sometimes necessary when using math font.
%%   \usepackage{lmodern}
%%
%% Figures using additional raster images can only be included by \input if
%% they are in the same directory as the main LaTeX file. For loading figures
%% from other directories you can use the `import` package
%%   \usepackage{import}
%%
%% and then include the figures with
%%   \import{<path to file>}{<filename>.pgf}
%%
%% Matplotlib used the following preamble
%%
\begingroup%
\makeatletter%
\begin{pgfpicture}%
\pgfpathrectangle{\pgfpointorigin}{\pgfqpoint{8.000000in}{6.000000in}}%
\pgfusepath{use as bounding box, clip}%
\begin{pgfscope}%
\pgfsetbuttcap%
\pgfsetmiterjoin%
\definecolor{currentfill}{rgb}{1.000000,1.000000,1.000000}%
\pgfsetfillcolor{currentfill}%
\pgfsetlinewidth{0.000000pt}%
\definecolor{currentstroke}{rgb}{1.000000,1.000000,1.000000}%
\pgfsetstrokecolor{currentstroke}%
\pgfsetdash{}{0pt}%
\pgfpathmoveto{\pgfqpoint{0.000000in}{0.000000in}}%
\pgfpathlineto{\pgfqpoint{8.000000in}{0.000000in}}%
\pgfpathlineto{\pgfqpoint{8.000000in}{6.000000in}}%
\pgfpathlineto{\pgfqpoint{0.000000in}{6.000000in}}%
\pgfpathlineto{\pgfqpoint{0.000000in}{0.000000in}}%
\pgfpathclose%
\pgfusepath{fill}%
\end{pgfscope}%
\begin{pgfscope}%
\pgfsetbuttcap%
\pgfsetmiterjoin%
\definecolor{currentfill}{rgb}{1.000000,1.000000,1.000000}%
\pgfsetfillcolor{currentfill}%
\pgfsetlinewidth{0.000000pt}%
\definecolor{currentstroke}{rgb}{0.000000,0.000000,0.000000}%
\pgfsetstrokecolor{currentstroke}%
\pgfsetstrokeopacity{0.000000}%
\pgfsetdash{}{0pt}%
\pgfpathmoveto{\pgfqpoint{1.000000in}{0.600000in}}%
\pgfpathlineto{\pgfqpoint{7.200000in}{0.600000in}}%
\pgfpathlineto{\pgfqpoint{7.200000in}{5.400000in}}%
\pgfpathlineto{\pgfqpoint{1.000000in}{5.400000in}}%
\pgfpathlineto{\pgfqpoint{1.000000in}{0.600000in}}%
\pgfpathclose%
\pgfusepath{fill}%
\end{pgfscope}%
\begin{pgfscope}%
\pgfpathrectangle{\pgfqpoint{1.000000in}{0.600000in}}{\pgfqpoint{6.200000in}{4.800000in}}%
\pgfusepath{clip}%
\pgfsetrectcap%
\pgfsetroundjoin%
\pgfsetlinewidth{1.003750pt}%
\definecolor{currentstroke}{rgb}{1.000000,0.000000,0.000000}%
\pgfsetstrokecolor{currentstroke}%
\pgfsetdash{}{0pt}%
\pgfpathmoveto{\pgfqpoint{1.000000in}{3.004189in}}%
\pgfpathlineto{\pgfqpoint{1.062000in}{3.004189in}}%
\pgfpathlineto{\pgfqpoint{1.124000in}{3.000189in}}%
\pgfpathlineto{\pgfqpoint{1.186000in}{2.992189in}}%
\pgfpathlineto{\pgfqpoint{1.248000in}{2.984000in}}%
\pgfpathlineto{\pgfqpoint{1.310000in}{2.982439in}}%
\pgfpathlineto{\pgfqpoint{1.372000in}{2.988153in}}%
\pgfpathlineto{\pgfqpoint{1.434000in}{3.000357in}}%
\pgfpathlineto{\pgfqpoint{1.496000in}{3.020529in}}%
\pgfpathlineto{\pgfqpoint{1.558000in}{3.040344in}}%
\pgfpathlineto{\pgfqpoint{1.620000in}{3.055806in}}%
\pgfpathlineto{\pgfqpoint{1.682000in}{3.078834in}}%
\pgfpathlineto{\pgfqpoint{1.744000in}{3.096754in}}%
\pgfpathlineto{\pgfqpoint{1.806000in}{3.118006in}}%
\pgfpathlineto{\pgfqpoint{1.868000in}{3.142895in}}%
\pgfpathlineto{\pgfqpoint{1.930000in}{3.161243in}}%
\pgfpathlineto{\pgfqpoint{1.992000in}{3.186270in}}%
\pgfpathlineto{\pgfqpoint{2.054000in}{3.203575in}}%
\pgfpathlineto{\pgfqpoint{2.116000in}{3.228573in}}%
\pgfpathlineto{\pgfqpoint{2.178000in}{3.251099in}}%
\pgfpathlineto{\pgfqpoint{2.240000in}{3.275968in}}%
\pgfpathlineto{\pgfqpoint{2.302000in}{3.301066in}}%
\pgfpathlineto{\pgfqpoint{2.364000in}{3.325672in}}%
\pgfpathlineto{\pgfqpoint{2.426000in}{3.350805in}}%
\pgfpathlineto{\pgfqpoint{2.488000in}{3.374887in}}%
\pgfpathlineto{\pgfqpoint{2.550000in}{3.400020in}}%
\pgfpathlineto{\pgfqpoint{2.612000in}{3.423073in}}%
\pgfpathlineto{\pgfqpoint{2.674000in}{3.448206in}}%
\pgfpathlineto{\pgfqpoint{2.736000in}{3.469344in}}%
\pgfpathlineto{\pgfqpoint{2.798000in}{3.494476in}}%
\pgfpathlineto{\pgfqpoint{2.860000in}{3.512783in}}%
\pgfpathlineto{\pgfqpoint{2.922000in}{3.537915in}}%
\pgfpathlineto{\pgfqpoint{2.984000in}{3.555417in}}%
\pgfpathlineto{\pgfqpoint{3.046000in}{3.580550in}}%
\pgfpathlineto{\pgfqpoint{3.108000in}{3.603226in}}%
\pgfpathlineto{\pgfqpoint{3.170000in}{3.628359in}}%
\pgfpathlineto{\pgfqpoint{3.232000in}{3.653451in}}%
\pgfpathlineto{\pgfqpoint{3.294000in}{3.678584in}}%
\pgfpathlineto{\pgfqpoint{3.356000in}{3.703717in}}%
\pgfpathlineto{\pgfqpoint{3.418000in}{3.728849in}}%
\pgfpathlineto{\pgfqpoint{3.480000in}{3.753982in}}%
\pgfpathlineto{\pgfqpoint{3.542000in}{3.779115in}}%
\pgfpathlineto{\pgfqpoint{3.604000in}{3.804248in}}%
\pgfpathlineto{\pgfqpoint{3.666000in}{3.829380in}}%
\pgfpathlineto{\pgfqpoint{3.728000in}{3.854513in}}%
\pgfpathlineto{\pgfqpoint{3.790000in}{3.879646in}}%
\pgfpathlineto{\pgfqpoint{3.852000in}{3.904779in}}%
\pgfpathlineto{\pgfqpoint{3.914000in}{3.929911in}}%
\pgfpathlineto{\pgfqpoint{3.976000in}{3.955044in}}%
\pgfpathlineto{\pgfqpoint{4.038000in}{3.980177in}}%
\pgfpathlineto{\pgfqpoint{4.100000in}{4.005310in}}%
\pgfpathlineto{\pgfqpoint{4.162000in}{4.030442in}}%
\pgfpathlineto{\pgfqpoint{4.224000in}{4.055575in}}%
\pgfpathlineto{\pgfqpoint{4.286000in}{4.080708in}}%
\pgfpathlineto{\pgfqpoint{4.348000in}{4.105841in}}%
\pgfpathlineto{\pgfqpoint{4.410000in}{4.130973in}}%
\pgfpathlineto{\pgfqpoint{4.472000in}{4.156106in}}%
\pgfpathlineto{\pgfqpoint{4.534000in}{4.181239in}}%
\pgfpathlineto{\pgfqpoint{4.596000in}{4.206372in}}%
\pgfpathlineto{\pgfqpoint{4.658000in}{4.231504in}}%
\pgfpathlineto{\pgfqpoint{4.720000in}{4.256637in}}%
\pgfpathlineto{\pgfqpoint{4.782000in}{4.281770in}}%
\pgfpathlineto{\pgfqpoint{4.844000in}{4.306902in}}%
\pgfpathlineto{\pgfqpoint{4.906000in}{4.332035in}}%
\pgfpathlineto{\pgfqpoint{4.968000in}{4.357168in}}%
\pgfpathlineto{\pgfqpoint{5.030000in}{4.382301in}}%
\pgfpathlineto{\pgfqpoint{5.092000in}{4.407433in}}%
\pgfpathlineto{\pgfqpoint{5.154000in}{4.432566in}}%
\pgfpathlineto{\pgfqpoint{5.216000in}{4.457699in}}%
\pgfpathlineto{\pgfqpoint{5.278000in}{4.482831in}}%
\pgfpathlineto{\pgfqpoint{5.340000in}{4.507964in}}%
\pgfpathlineto{\pgfqpoint{5.402000in}{4.533095in}}%
\pgfpathlineto{\pgfqpoint{5.464000in}{4.558228in}}%
\pgfpathlineto{\pgfqpoint{5.526000in}{4.583359in}}%
\pgfpathlineto{\pgfqpoint{5.588000in}{4.608492in}}%
\pgfpathlineto{\pgfqpoint{5.650000in}{4.633621in}}%
\pgfpathlineto{\pgfqpoint{5.712000in}{4.658754in}}%
\pgfpathlineto{\pgfqpoint{5.774000in}{4.683879in}}%
\pgfpathlineto{\pgfqpoint{5.836000in}{4.709012in}}%
\pgfpathlineto{\pgfqpoint{5.898000in}{4.734130in}}%
\pgfpathlineto{\pgfqpoint{5.960000in}{4.759263in}}%
\pgfpathlineto{\pgfqpoint{6.022000in}{4.784367in}}%
\pgfpathlineto{\pgfqpoint{6.084000in}{4.809500in}}%
\pgfpathlineto{\pgfqpoint{6.146000in}{4.834575in}}%
\pgfpathlineto{\pgfqpoint{6.208000in}{4.859708in}}%
\pgfpathlineto{\pgfqpoint{6.270000in}{4.884726in}}%
\pgfpathlineto{\pgfqpoint{6.332000in}{4.909859in}}%
\pgfpathlineto{\pgfqpoint{6.394000in}{4.934762in}}%
\pgfpathlineto{\pgfqpoint{6.456000in}{4.959895in}}%
\pgfpathlineto{\pgfqpoint{6.518000in}{4.984569in}}%
\pgfpathlineto{\pgfqpoint{6.580000in}{5.009702in}}%
\pgfpathlineto{\pgfqpoint{6.642000in}{5.033920in}}%
\pgfpathlineto{\pgfqpoint{6.704000in}{5.059052in}}%
\pgfpathlineto{\pgfqpoint{6.766000in}{5.082369in}}%
\pgfpathlineto{\pgfqpoint{6.828000in}{5.107501in}}%
\pgfpathlineto{\pgfqpoint{6.890000in}{5.129110in}}%
\pgfpathlineto{\pgfqpoint{6.952000in}{5.154243in}}%
\pgfpathlineto{\pgfqpoint{7.014000in}{5.173126in}}%
\pgfpathlineto{\pgfqpoint{7.076000in}{5.198259in}}%
\pgfpathlineto{\pgfqpoint{7.138000in}{5.215437in}}%
\pgfpathlineto{\pgfqpoint{7.200000in}{5.240570in}}%
\pgfusepath{stroke}%
\end{pgfscope}%
\begin{pgfscope}%
\pgfpathrectangle{\pgfqpoint{1.000000in}{0.600000in}}{\pgfqpoint{6.200000in}{4.800000in}}%
\pgfusepath{clip}%
\pgfsetrectcap%
\pgfsetroundjoin%
\pgfsetlinewidth{1.003750pt}%
\definecolor{currentstroke}{rgb}{0.000000,0.000000,1.000000}%
\pgfsetstrokecolor{currentstroke}%
\pgfsetdash{}{0pt}%
\pgfpathmoveto{\pgfqpoint{1.000000in}{3.004189in}}%
\pgfpathlineto{\pgfqpoint{1.062000in}{3.000189in}}%
\pgfpathlineto{\pgfqpoint{1.124000in}{2.996000in}}%
\pgfpathlineto{\pgfqpoint{1.186000in}{2.999504in}}%
\pgfpathlineto{\pgfqpoint{1.248000in}{3.003791in}}%
\pgfpathlineto{\pgfqpoint{1.310000in}{3.000729in}}%
\pgfpathlineto{\pgfqpoint{1.372000in}{2.996425in}}%
\pgfpathlineto{\pgfqpoint{1.434000in}{2.999094in}}%
\pgfpathlineto{\pgfqpoint{1.496000in}{3.003360in}}%
\pgfpathlineto{\pgfqpoint{1.558000in}{3.001042in}}%
\pgfpathlineto{\pgfqpoint{1.620000in}{2.996848in}}%
\pgfpathlineto{\pgfqpoint{1.682000in}{2.998852in}}%
\pgfpathlineto{\pgfqpoint{1.744000in}{3.002955in}}%
\pgfpathlineto{\pgfqpoint{1.806000in}{3.001230in}}%
\pgfpathlineto{\pgfqpoint{1.868000in}{2.997229in}}%
\pgfpathlineto{\pgfqpoint{1.930000in}{2.998705in}}%
\pgfpathlineto{\pgfqpoint{1.992000in}{3.002599in}}%
\pgfpathlineto{\pgfqpoint{2.054000in}{3.001346in}}%
\pgfpathlineto{\pgfqpoint{2.116000in}{2.997560in}}%
\pgfpathlineto{\pgfqpoint{2.178000in}{2.998613in}}%
\pgfpathlineto{\pgfqpoint{2.240000in}{3.002293in}}%
\pgfpathlineto{\pgfqpoint{2.302000in}{3.001420in}}%
\pgfpathlineto{\pgfqpoint{2.364000in}{2.997843in}}%
\pgfpathlineto{\pgfqpoint{2.426000in}{2.998552in}}%
\pgfpathlineto{\pgfqpoint{2.488000in}{3.002031in}}%
\pgfpathlineto{\pgfqpoint{2.550000in}{3.001470in}}%
\pgfpathlineto{\pgfqpoint{2.612000in}{2.998086in}}%
\pgfpathlineto{\pgfqpoint{2.674000in}{2.998512in}}%
\pgfpathlineto{\pgfqpoint{2.736000in}{3.001806in}}%
\pgfpathlineto{\pgfqpoint{2.798000in}{3.001504in}}%
\pgfpathlineto{\pgfqpoint{2.860000in}{2.998296in}}%
\pgfpathlineto{\pgfqpoint{2.922000in}{2.998483in}}%
\pgfpathlineto{\pgfqpoint{2.984000in}{3.001610in}}%
\pgfpathlineto{\pgfqpoint{3.046000in}{3.001527in}}%
\pgfpathlineto{\pgfqpoint{3.108000in}{2.998478in}}%
\pgfpathlineto{\pgfqpoint{3.170000in}{2.998463in}}%
\pgfpathlineto{\pgfqpoint{3.232000in}{3.001439in}}%
\pgfpathlineto{\pgfqpoint{3.294000in}{3.001544in}}%
\pgfpathlineto{\pgfqpoint{3.356000in}{2.998639in}}%
\pgfpathlineto{\pgfqpoint{3.418000in}{2.998449in}}%
\pgfpathlineto{\pgfqpoint{3.480000in}{3.001288in}}%
\pgfpathlineto{\pgfqpoint{3.542000in}{3.001556in}}%
\pgfpathlineto{\pgfqpoint{3.604000in}{2.998781in}}%
\pgfpathlineto{\pgfqpoint{3.666000in}{2.998439in}}%
\pgfpathlineto{\pgfqpoint{3.728000in}{3.001154in}}%
\pgfpathlineto{\pgfqpoint{3.790000in}{3.001565in}}%
\pgfpathlineto{\pgfqpoint{3.852000in}{2.998908in}}%
\pgfpathlineto{\pgfqpoint{3.914000in}{2.998432in}}%
\pgfpathlineto{\pgfqpoint{3.976000in}{3.001034in}}%
\pgfpathlineto{\pgfqpoint{4.038000in}{3.001570in}}%
\pgfpathlineto{\pgfqpoint{4.100000in}{2.999022in}}%
\pgfpathlineto{\pgfqpoint{4.162000in}{2.998428in}}%
\pgfpathlineto{\pgfqpoint{4.224000in}{3.000925in}}%
\pgfpathlineto{\pgfqpoint{4.286000in}{3.001574in}}%
\pgfpathlineto{\pgfqpoint{4.348000in}{2.999125in}}%
\pgfpathlineto{\pgfqpoint{4.410000in}{2.998425in}}%
\pgfpathlineto{\pgfqpoint{4.472000in}{3.000826in}}%
\pgfpathlineto{\pgfqpoint{4.534000in}{3.001575in}}%
\pgfpathlineto{\pgfqpoint{4.596000in}{2.999220in}}%
\pgfpathlineto{\pgfqpoint{4.658000in}{2.998425in}}%
\pgfpathlineto{\pgfqpoint{4.720000in}{3.000736in}}%
\pgfpathlineto{\pgfqpoint{4.782000in}{3.001575in}}%
\pgfpathlineto{\pgfqpoint{4.844000in}{2.999306in}}%
\pgfpathlineto{\pgfqpoint{4.906000in}{2.998426in}}%
\pgfpathlineto{\pgfqpoint{4.968000in}{3.000654in}}%
\pgfpathlineto{\pgfqpoint{5.030000in}{3.001573in}}%
\pgfpathlineto{\pgfqpoint{5.092000in}{2.999385in}}%
\pgfpathlineto{\pgfqpoint{5.154000in}{2.998429in}}%
\pgfpathlineto{\pgfqpoint{5.216000in}{3.000578in}}%
\pgfpathlineto{\pgfqpoint{5.278000in}{3.001569in}}%
\pgfpathlineto{\pgfqpoint{5.340000in}{2.999458in}}%
\pgfpathlineto{\pgfqpoint{5.402000in}{2.998433in}}%
\pgfpathlineto{\pgfqpoint{5.464000in}{3.000508in}}%
\pgfpathlineto{\pgfqpoint{5.526000in}{3.001564in}}%
\pgfpathlineto{\pgfqpoint{5.588000in}{2.999525in}}%
\pgfpathlineto{\pgfqpoint{5.650000in}{2.998439in}}%
\pgfpathlineto{\pgfqpoint{5.712000in}{3.000443in}}%
\pgfpathlineto{\pgfqpoint{5.774000in}{3.001558in}}%
\pgfpathlineto{\pgfqpoint{5.836000in}{2.999587in}}%
\pgfpathlineto{\pgfqpoint{5.898000in}{2.998446in}}%
\pgfpathlineto{\pgfqpoint{5.960000in}{3.000383in}}%
\pgfpathlineto{\pgfqpoint{6.022000in}{3.001550in}}%
\pgfpathlineto{\pgfqpoint{6.084000in}{2.999645in}}%
\pgfpathlineto{\pgfqpoint{6.146000in}{2.998454in}}%
\pgfpathlineto{\pgfqpoint{6.208000in}{3.000327in}}%
\pgfpathlineto{\pgfqpoint{6.270000in}{3.001541in}}%
\pgfpathlineto{\pgfqpoint{6.332000in}{2.999699in}}%
\pgfpathlineto{\pgfqpoint{6.394000in}{2.998463in}}%
\pgfpathlineto{\pgfqpoint{6.456000in}{3.000275in}}%
\pgfpathlineto{\pgfqpoint{6.518000in}{3.001532in}}%
\pgfpathlineto{\pgfqpoint{6.580000in}{2.999749in}}%
\pgfpathlineto{\pgfqpoint{6.642000in}{2.998474in}}%
\pgfpathlineto{\pgfqpoint{6.704000in}{3.000227in}}%
\pgfpathlineto{\pgfqpoint{6.766000in}{3.001521in}}%
\pgfpathlineto{\pgfqpoint{6.828000in}{2.999796in}}%
\pgfpathlineto{\pgfqpoint{6.890000in}{2.998485in}}%
\pgfpathlineto{\pgfqpoint{6.952000in}{3.000182in}}%
\pgfpathlineto{\pgfqpoint{7.014000in}{3.001509in}}%
\pgfpathlineto{\pgfqpoint{7.076000in}{2.999840in}}%
\pgfpathlineto{\pgfqpoint{7.138000in}{2.998497in}}%
\pgfpathlineto{\pgfqpoint{7.200000in}{3.000139in}}%
\pgfusepath{stroke}%
\end{pgfscope}%
\begin{pgfscope}%
\pgfpathrectangle{\pgfqpoint{1.000000in}{0.600000in}}{\pgfqpoint{6.200000in}{4.800000in}}%
\pgfusepath{clip}%
\pgfsetrectcap%
\pgfsetroundjoin%
\pgfsetlinewidth{1.003750pt}%
\definecolor{currentstroke}{rgb}{0.000000,0.000000,0.000000}%
\pgfsetstrokecolor{currentstroke}%
\pgfsetdash{}{0pt}%
\pgfpathmoveto{\pgfqpoint{1.000000in}{3.004189in}}%
\pgfpathlineto{\pgfqpoint{1.062000in}{3.002189in}}%
\pgfpathlineto{\pgfqpoint{1.124000in}{2.997108in}}%
\pgfpathlineto{\pgfqpoint{1.186000in}{2.994340in}}%
\pgfpathlineto{\pgfqpoint{1.248000in}{2.997265in}}%
\pgfpathlineto{\pgfqpoint{1.310000in}{3.004006in}}%
\pgfpathlineto{\pgfqpoint{1.372000in}{3.007518in}}%
\pgfpathlineto{\pgfqpoint{1.434000in}{3.004102in}}%
\pgfpathlineto{\pgfqpoint{1.496000in}{2.995550in}}%
\pgfpathlineto{\pgfqpoint{1.558000in}{2.990334in}}%
\pgfpathlineto{\pgfqpoint{1.620000in}{2.992838in}}%
\pgfpathlineto{\pgfqpoint{1.682000in}{3.002377in}}%
\pgfpathlineto{\pgfqpoint{1.744000in}{3.011117in}}%
\pgfpathlineto{\pgfqpoint{1.806000in}{3.011924in}}%
\pgfpathlineto{\pgfqpoint{1.868000in}{3.005078in}}%
\pgfpathlineto{\pgfqpoint{1.930000in}{2.992237in}}%
\pgfpathlineto{\pgfqpoint{1.992000in}{2.985193in}}%
\pgfpathlineto{\pgfqpoint{2.054000in}{2.985070in}}%
\pgfpathlineto{\pgfqpoint{2.116000in}{2.991629in}}%
\pgfpathlineto{\pgfqpoint{2.178000in}{3.005647in}}%
\pgfpathlineto{\pgfqpoint{2.240000in}{3.015991in}}%
\pgfpathlineto{\pgfqpoint{2.302000in}{3.019775in}}%
\pgfpathlineto{\pgfqpoint{2.364000in}{3.020225in}}%
\pgfpathlineto{\pgfqpoint{2.426000in}{3.017177in}}%
\pgfpathlineto{\pgfqpoint{2.488000in}{3.008178in}}%
\pgfpathlineto{\pgfqpoint{2.550000in}{2.991799in}}%
\pgfpathlineto{\pgfqpoint{2.612000in}{2.981093in}}%
\pgfpathlineto{\pgfqpoint{2.674000in}{2.974570in}}%
\pgfpathlineto{\pgfqpoint{2.736000in}{2.966384in}}%
\pgfpathlineto{\pgfqpoint{2.798000in}{2.951255in}}%
\pgfpathlineto{\pgfqpoint{2.860000in}{2.933026in}}%
\pgfpathlineto{\pgfqpoint{2.922000in}{2.921918in}}%
\pgfpathlineto{\pgfqpoint{2.984000in}{2.906710in}}%
\pgfpathlineto{\pgfqpoint{3.046000in}{2.884848in}}%
\pgfpathlineto{\pgfqpoint{3.108000in}{2.870817in}}%
\pgfpathlineto{\pgfqpoint{3.170000in}{2.851892in}}%
\pgfpathlineto{\pgfqpoint{3.232000in}{2.829561in}}%
\pgfpathlineto{\pgfqpoint{3.294000in}{2.812772in}}%
\pgfpathlineto{\pgfqpoint{3.356000in}{2.788083in}}%
\pgfpathlineto{\pgfqpoint{3.418000in}{2.770342in}}%
\pgfpathlineto{\pgfqpoint{3.480000in}{2.747738in}}%
\pgfpathlineto{\pgfqpoint{3.542000in}{2.725229in}}%
\pgfpathlineto{\pgfqpoint{3.604000in}{2.704167in}}%
\pgfpathlineto{\pgfqpoint{3.666000in}{2.678159in}}%
\pgfpathlineto{\pgfqpoint{3.728000in}{2.657676in}}%
\pgfpathlineto{\pgfqpoint{3.790000in}{2.630330in}}%
\pgfpathlineto{\pgfqpoint{3.852000in}{2.610082in}}%
\pgfpathlineto{\pgfqpoint{3.914000in}{2.582222in}}%
\pgfpathlineto{\pgfqpoint{3.976000in}{2.562042in}}%
\pgfpathlineto{\pgfqpoint{4.038000in}{2.534075in}}%
\pgfpathlineto{\pgfqpoint{4.100000in}{2.513815in}}%
\pgfpathlineto{\pgfqpoint{4.162000in}{2.486022in}}%
\pgfpathlineto{\pgfqpoint{4.224000in}{2.465488in}}%
\pgfpathlineto{\pgfqpoint{4.286000in}{2.438086in}}%
\pgfpathlineto{\pgfqpoint{4.348000in}{2.417023in}}%
\pgfpathlineto{\pgfqpoint{4.410000in}{2.390149in}}%
\pgfpathlineto{\pgfqpoint{4.472000in}{2.368307in}}%
\pgfpathlineto{\pgfqpoint{4.534000in}{2.341981in}}%
\pgfpathlineto{\pgfqpoint{4.596000in}{2.319249in}}%
\pgfpathlineto{\pgfqpoint{4.658000in}{2.293387in}}%
\pgfpathlineto{\pgfqpoint{4.720000in}{2.269846in}}%
\pgfpathlineto{\pgfqpoint{4.782000in}{2.244323in}}%
\pgfpathlineto{\pgfqpoint{4.844000in}{2.220162in}}%
\pgfpathlineto{\pgfqpoint{4.906000in}{2.194862in}}%
\pgfpathlineto{\pgfqpoint{4.968000in}{2.170271in}}%
\pgfpathlineto{\pgfqpoint{5.030000in}{2.145110in}}%
\pgfpathlineto{\pgfqpoint{5.092000in}{2.120238in}}%
\pgfpathlineto{\pgfqpoint{5.154000in}{2.095157in}}%
\pgfpathlineto{\pgfqpoint{5.216000in}{2.070111in}}%
\pgfpathlineto{\pgfqpoint{5.278000in}{2.045072in}}%
\pgfpathlineto{\pgfqpoint{5.340000in}{2.019922in}}%
\pgfpathlineto{\pgfqpoint{5.402000in}{1.994901in}}%
\pgfpathlineto{\pgfqpoint{5.464000in}{1.969694in}}%
\pgfpathlineto{\pgfqpoint{5.526000in}{1.944675in}}%
\pgfpathlineto{\pgfqpoint{5.588000in}{1.919440in}}%
\pgfpathlineto{\pgfqpoint{5.650000in}{1.894416in}}%
\pgfpathlineto{\pgfqpoint{5.712000in}{1.869172in}}%
\pgfpathlineto{\pgfqpoint{5.774000in}{1.844139in}}%
\pgfpathlineto{\pgfqpoint{5.836000in}{1.818896in}}%
\pgfpathlineto{\pgfqpoint{5.898000in}{1.793851in}}%
\pgfpathlineto{\pgfqpoint{5.960000in}{1.768617in}}%
\pgfpathlineto{\pgfqpoint{6.022000in}{1.743559in}}%
\pgfpathlineto{\pgfqpoint{6.084000in}{1.718336in}}%
\pgfpathlineto{\pgfqpoint{6.146000in}{1.693266in}}%
\pgfpathlineto{\pgfqpoint{6.208000in}{1.668055in}}%
\pgfpathlineto{\pgfqpoint{6.270000in}{1.642975in}}%
\pgfpathlineto{\pgfqpoint{6.332000in}{1.617775in}}%
\pgfpathlineto{\pgfqpoint{6.394000in}{1.592686in}}%
\pgfpathlineto{\pgfqpoint{6.456000in}{1.567497in}}%
\pgfpathlineto{\pgfqpoint{6.518000in}{1.542400in}}%
\pgfpathlineto{\pgfqpoint{6.580000in}{1.517220in}}%
\pgfpathlineto{\pgfqpoint{6.642000in}{1.492117in}}%
\pgfpathlineto{\pgfqpoint{6.704000in}{1.466945in}}%
\pgfpathlineto{\pgfqpoint{6.766000in}{1.441836in}}%
\pgfpathlineto{\pgfqpoint{6.828000in}{1.416672in}}%
\pgfpathlineto{\pgfqpoint{6.890000in}{1.391558in}}%
\pgfpathlineto{\pgfqpoint{6.952000in}{1.366400in}}%
\pgfpathlineto{\pgfqpoint{7.014000in}{1.341282in}}%
\pgfpathlineto{\pgfqpoint{7.076000in}{1.316129in}}%
\pgfpathlineto{\pgfqpoint{7.138000in}{1.291008in}}%
\pgfpathlineto{\pgfqpoint{7.200000in}{1.265859in}}%
\pgfusepath{stroke}%
\end{pgfscope}%
\begin{pgfscope}%
\pgfsetrectcap%
\pgfsetmiterjoin%
\pgfsetlinewidth{1.003750pt}%
\definecolor{currentstroke}{rgb}{0.000000,0.000000,0.000000}%
\pgfsetstrokecolor{currentstroke}%
\pgfsetdash{}{0pt}%
\pgfpathmoveto{\pgfqpoint{1.000000in}{0.600000in}}%
\pgfpathlineto{\pgfqpoint{1.000000in}{5.400000in}}%
\pgfusepath{stroke}%
\end{pgfscope}%
\begin{pgfscope}%
\pgfsetrectcap%
\pgfsetmiterjoin%
\pgfsetlinewidth{1.003750pt}%
\definecolor{currentstroke}{rgb}{0.000000,0.000000,0.000000}%
\pgfsetstrokecolor{currentstroke}%
\pgfsetdash{}{0pt}%
\pgfpathmoveto{\pgfqpoint{7.200000in}{0.600000in}}%
\pgfpathlineto{\pgfqpoint{7.200000in}{5.400000in}}%
\pgfusepath{stroke}%
\end{pgfscope}%
\begin{pgfscope}%
\pgfsetrectcap%
\pgfsetmiterjoin%
\pgfsetlinewidth{1.003750pt}%
\definecolor{currentstroke}{rgb}{0.000000,0.000000,0.000000}%
\pgfsetstrokecolor{currentstroke}%
\pgfsetdash{}{0pt}%
\pgfpathmoveto{\pgfqpoint{1.000000in}{0.600000in}}%
\pgfpathlineto{\pgfqpoint{7.200000in}{0.600000in}}%
\pgfusepath{stroke}%
\end{pgfscope}%
\begin{pgfscope}%
\pgfsetrectcap%
\pgfsetmiterjoin%
\pgfsetlinewidth{1.003750pt}%
\definecolor{currentstroke}{rgb}{0.000000,0.000000,0.000000}%
\pgfsetstrokecolor{currentstroke}%
\pgfsetdash{}{0pt}%
\pgfpathmoveto{\pgfqpoint{1.000000in}{5.400000in}}%
\pgfpathlineto{\pgfqpoint{7.200000in}{5.400000in}}%
\pgfusepath{stroke}%
\end{pgfscope}%
\begin{pgfscope}%
\pgfpathrectangle{\pgfqpoint{1.000000in}{0.600000in}}{\pgfqpoint{6.200000in}{4.800000in}}%
\pgfusepath{clip}%
\pgfsetbuttcap%
\pgfsetroundjoin%
\pgfsetlinewidth{0.501875pt}%
\definecolor{currentstroke}{rgb}{0.000000,0.000000,0.000000}%
\pgfsetstrokecolor{currentstroke}%
\pgfsetdash{{1.000000pt}{3.000000pt}}{0.000000pt}%
\pgfpathmoveto{\pgfqpoint{1.000000in}{0.600000in}}%
\pgfpathlineto{\pgfqpoint{1.000000in}{5.400000in}}%
\pgfusepath{stroke}%
\end{pgfscope}%
\begin{pgfscope}%
\pgfsetbuttcap%
\pgfsetroundjoin%
\definecolor{currentfill}{rgb}{0.000000,0.000000,0.000000}%
\pgfsetfillcolor{currentfill}%
\pgfsetlinewidth{0.501875pt}%
\definecolor{currentstroke}{rgb}{0.000000,0.000000,0.000000}%
\pgfsetstrokecolor{currentstroke}%
\pgfsetdash{}{0pt}%
\pgfsys@defobject{currentmarker}{\pgfqpoint{0.000000in}{0.000000in}}{\pgfqpoint{0.000000in}{0.055556in}}{%
\pgfpathmoveto{\pgfqpoint{0.000000in}{0.000000in}}%
\pgfpathlineto{\pgfqpoint{0.000000in}{0.055556in}}%
\pgfusepath{stroke,fill}%
}%
\begin{pgfscope}%
\pgfsys@transformshift{1.000000in}{0.600000in}%
\pgfsys@useobject{currentmarker}{}%
\end{pgfscope}%
\end{pgfscope}%
\begin{pgfscope}%
\pgfsetbuttcap%
\pgfsetroundjoin%
\definecolor{currentfill}{rgb}{0.000000,0.000000,0.000000}%
\pgfsetfillcolor{currentfill}%
\pgfsetlinewidth{0.501875pt}%
\definecolor{currentstroke}{rgb}{0.000000,0.000000,0.000000}%
\pgfsetstrokecolor{currentstroke}%
\pgfsetdash{}{0pt}%
\pgfsys@defobject{currentmarker}{\pgfqpoint{0.000000in}{-0.055556in}}{\pgfqpoint{0.000000in}{0.000000in}}{%
\pgfpathmoveto{\pgfqpoint{0.000000in}{0.000000in}}%
\pgfpathlineto{\pgfqpoint{0.000000in}{-0.055556in}}%
\pgfusepath{stroke,fill}%
}%
\begin{pgfscope}%
\pgfsys@transformshift{1.000000in}{5.400000in}%
\pgfsys@useobject{currentmarker}{}%
\end{pgfscope}%
\end{pgfscope}%
\begin{pgfscope}%
\definecolor{textcolor}{rgb}{0.000000,0.000000,0.000000}%
\pgfsetstrokecolor{textcolor}%
\pgfsetfillcolor{textcolor}%
\pgftext[x=1.000000in,y=0.544444in,,top]{\color{textcolor}\rmfamily\fontsize{10.000000}{12.000000}\selectfont \(\displaystyle {0}\)}%
\end{pgfscope}%
\begin{pgfscope}%
\pgfpathrectangle{\pgfqpoint{1.000000in}{0.600000in}}{\pgfqpoint{6.200000in}{4.800000in}}%
\pgfusepath{clip}%
\pgfsetbuttcap%
\pgfsetroundjoin%
\pgfsetlinewidth{0.501875pt}%
\definecolor{currentstroke}{rgb}{0.000000,0.000000,0.000000}%
\pgfsetstrokecolor{currentstroke}%
\pgfsetdash{{1.000000pt}{3.000000pt}}{0.000000pt}%
\pgfpathmoveto{\pgfqpoint{2.240000in}{0.600000in}}%
\pgfpathlineto{\pgfqpoint{2.240000in}{5.400000in}}%
\pgfusepath{stroke}%
\end{pgfscope}%
\begin{pgfscope}%
\pgfsetbuttcap%
\pgfsetroundjoin%
\definecolor{currentfill}{rgb}{0.000000,0.000000,0.000000}%
\pgfsetfillcolor{currentfill}%
\pgfsetlinewidth{0.501875pt}%
\definecolor{currentstroke}{rgb}{0.000000,0.000000,0.000000}%
\pgfsetstrokecolor{currentstroke}%
\pgfsetdash{}{0pt}%
\pgfsys@defobject{currentmarker}{\pgfqpoint{0.000000in}{0.000000in}}{\pgfqpoint{0.000000in}{0.055556in}}{%
\pgfpathmoveto{\pgfqpoint{0.000000in}{0.000000in}}%
\pgfpathlineto{\pgfqpoint{0.000000in}{0.055556in}}%
\pgfusepath{stroke,fill}%
}%
\begin{pgfscope}%
\pgfsys@transformshift{2.240000in}{0.600000in}%
\pgfsys@useobject{currentmarker}{}%
\end{pgfscope}%
\end{pgfscope}%
\begin{pgfscope}%
\pgfsetbuttcap%
\pgfsetroundjoin%
\definecolor{currentfill}{rgb}{0.000000,0.000000,0.000000}%
\pgfsetfillcolor{currentfill}%
\pgfsetlinewidth{0.501875pt}%
\definecolor{currentstroke}{rgb}{0.000000,0.000000,0.000000}%
\pgfsetstrokecolor{currentstroke}%
\pgfsetdash{}{0pt}%
\pgfsys@defobject{currentmarker}{\pgfqpoint{0.000000in}{-0.055556in}}{\pgfqpoint{0.000000in}{0.000000in}}{%
\pgfpathmoveto{\pgfqpoint{0.000000in}{0.000000in}}%
\pgfpathlineto{\pgfqpoint{0.000000in}{-0.055556in}}%
\pgfusepath{stroke,fill}%
}%
\begin{pgfscope}%
\pgfsys@transformshift{2.240000in}{5.400000in}%
\pgfsys@useobject{currentmarker}{}%
\end{pgfscope}%
\end{pgfscope}%
\begin{pgfscope}%
\definecolor{textcolor}{rgb}{0.000000,0.000000,0.000000}%
\pgfsetstrokecolor{textcolor}%
\pgfsetfillcolor{textcolor}%
\pgftext[x=2.240000in,y=0.544444in,,top]{\color{textcolor}\rmfamily\fontsize{10.000000}{12.000000}\selectfont \(\displaystyle {20}\)}%
\end{pgfscope}%
\begin{pgfscope}%
\pgfpathrectangle{\pgfqpoint{1.000000in}{0.600000in}}{\pgfqpoint{6.200000in}{4.800000in}}%
\pgfusepath{clip}%
\pgfsetbuttcap%
\pgfsetroundjoin%
\pgfsetlinewidth{0.501875pt}%
\definecolor{currentstroke}{rgb}{0.000000,0.000000,0.000000}%
\pgfsetstrokecolor{currentstroke}%
\pgfsetdash{{1.000000pt}{3.000000pt}}{0.000000pt}%
\pgfpathmoveto{\pgfqpoint{3.480000in}{0.600000in}}%
\pgfpathlineto{\pgfqpoint{3.480000in}{5.400000in}}%
\pgfusepath{stroke}%
\end{pgfscope}%
\begin{pgfscope}%
\pgfsetbuttcap%
\pgfsetroundjoin%
\definecolor{currentfill}{rgb}{0.000000,0.000000,0.000000}%
\pgfsetfillcolor{currentfill}%
\pgfsetlinewidth{0.501875pt}%
\definecolor{currentstroke}{rgb}{0.000000,0.000000,0.000000}%
\pgfsetstrokecolor{currentstroke}%
\pgfsetdash{}{0pt}%
\pgfsys@defobject{currentmarker}{\pgfqpoint{0.000000in}{0.000000in}}{\pgfqpoint{0.000000in}{0.055556in}}{%
\pgfpathmoveto{\pgfqpoint{0.000000in}{0.000000in}}%
\pgfpathlineto{\pgfqpoint{0.000000in}{0.055556in}}%
\pgfusepath{stroke,fill}%
}%
\begin{pgfscope}%
\pgfsys@transformshift{3.480000in}{0.600000in}%
\pgfsys@useobject{currentmarker}{}%
\end{pgfscope}%
\end{pgfscope}%
\begin{pgfscope}%
\pgfsetbuttcap%
\pgfsetroundjoin%
\definecolor{currentfill}{rgb}{0.000000,0.000000,0.000000}%
\pgfsetfillcolor{currentfill}%
\pgfsetlinewidth{0.501875pt}%
\definecolor{currentstroke}{rgb}{0.000000,0.000000,0.000000}%
\pgfsetstrokecolor{currentstroke}%
\pgfsetdash{}{0pt}%
\pgfsys@defobject{currentmarker}{\pgfqpoint{0.000000in}{-0.055556in}}{\pgfqpoint{0.000000in}{0.000000in}}{%
\pgfpathmoveto{\pgfqpoint{0.000000in}{0.000000in}}%
\pgfpathlineto{\pgfqpoint{0.000000in}{-0.055556in}}%
\pgfusepath{stroke,fill}%
}%
\begin{pgfscope}%
\pgfsys@transformshift{3.480000in}{5.400000in}%
\pgfsys@useobject{currentmarker}{}%
\end{pgfscope}%
\end{pgfscope}%
\begin{pgfscope}%
\definecolor{textcolor}{rgb}{0.000000,0.000000,0.000000}%
\pgfsetstrokecolor{textcolor}%
\pgfsetfillcolor{textcolor}%
\pgftext[x=3.480000in,y=0.544444in,,top]{\color{textcolor}\rmfamily\fontsize{10.000000}{12.000000}\selectfont \(\displaystyle {40}\)}%
\end{pgfscope}%
\begin{pgfscope}%
\pgfpathrectangle{\pgfqpoint{1.000000in}{0.600000in}}{\pgfqpoint{6.200000in}{4.800000in}}%
\pgfusepath{clip}%
\pgfsetbuttcap%
\pgfsetroundjoin%
\pgfsetlinewidth{0.501875pt}%
\definecolor{currentstroke}{rgb}{0.000000,0.000000,0.000000}%
\pgfsetstrokecolor{currentstroke}%
\pgfsetdash{{1.000000pt}{3.000000pt}}{0.000000pt}%
\pgfpathmoveto{\pgfqpoint{4.720000in}{0.600000in}}%
\pgfpathlineto{\pgfqpoint{4.720000in}{5.400000in}}%
\pgfusepath{stroke}%
\end{pgfscope}%
\begin{pgfscope}%
\pgfsetbuttcap%
\pgfsetroundjoin%
\definecolor{currentfill}{rgb}{0.000000,0.000000,0.000000}%
\pgfsetfillcolor{currentfill}%
\pgfsetlinewidth{0.501875pt}%
\definecolor{currentstroke}{rgb}{0.000000,0.000000,0.000000}%
\pgfsetstrokecolor{currentstroke}%
\pgfsetdash{}{0pt}%
\pgfsys@defobject{currentmarker}{\pgfqpoint{0.000000in}{0.000000in}}{\pgfqpoint{0.000000in}{0.055556in}}{%
\pgfpathmoveto{\pgfqpoint{0.000000in}{0.000000in}}%
\pgfpathlineto{\pgfqpoint{0.000000in}{0.055556in}}%
\pgfusepath{stroke,fill}%
}%
\begin{pgfscope}%
\pgfsys@transformshift{4.720000in}{0.600000in}%
\pgfsys@useobject{currentmarker}{}%
\end{pgfscope}%
\end{pgfscope}%
\begin{pgfscope}%
\pgfsetbuttcap%
\pgfsetroundjoin%
\definecolor{currentfill}{rgb}{0.000000,0.000000,0.000000}%
\pgfsetfillcolor{currentfill}%
\pgfsetlinewidth{0.501875pt}%
\definecolor{currentstroke}{rgb}{0.000000,0.000000,0.000000}%
\pgfsetstrokecolor{currentstroke}%
\pgfsetdash{}{0pt}%
\pgfsys@defobject{currentmarker}{\pgfqpoint{0.000000in}{-0.055556in}}{\pgfqpoint{0.000000in}{0.000000in}}{%
\pgfpathmoveto{\pgfqpoint{0.000000in}{0.000000in}}%
\pgfpathlineto{\pgfqpoint{0.000000in}{-0.055556in}}%
\pgfusepath{stroke,fill}%
}%
\begin{pgfscope}%
\pgfsys@transformshift{4.720000in}{5.400000in}%
\pgfsys@useobject{currentmarker}{}%
\end{pgfscope}%
\end{pgfscope}%
\begin{pgfscope}%
\definecolor{textcolor}{rgb}{0.000000,0.000000,0.000000}%
\pgfsetstrokecolor{textcolor}%
\pgfsetfillcolor{textcolor}%
\pgftext[x=4.720000in,y=0.544444in,,top]{\color{textcolor}\rmfamily\fontsize{10.000000}{12.000000}\selectfont \(\displaystyle {60}\)}%
\end{pgfscope}%
\begin{pgfscope}%
\pgfpathrectangle{\pgfqpoint{1.000000in}{0.600000in}}{\pgfqpoint{6.200000in}{4.800000in}}%
\pgfusepath{clip}%
\pgfsetbuttcap%
\pgfsetroundjoin%
\pgfsetlinewidth{0.501875pt}%
\definecolor{currentstroke}{rgb}{0.000000,0.000000,0.000000}%
\pgfsetstrokecolor{currentstroke}%
\pgfsetdash{{1.000000pt}{3.000000pt}}{0.000000pt}%
\pgfpathmoveto{\pgfqpoint{5.960000in}{0.600000in}}%
\pgfpathlineto{\pgfqpoint{5.960000in}{5.400000in}}%
\pgfusepath{stroke}%
\end{pgfscope}%
\begin{pgfscope}%
\pgfsetbuttcap%
\pgfsetroundjoin%
\definecolor{currentfill}{rgb}{0.000000,0.000000,0.000000}%
\pgfsetfillcolor{currentfill}%
\pgfsetlinewidth{0.501875pt}%
\definecolor{currentstroke}{rgb}{0.000000,0.000000,0.000000}%
\pgfsetstrokecolor{currentstroke}%
\pgfsetdash{}{0pt}%
\pgfsys@defobject{currentmarker}{\pgfqpoint{0.000000in}{0.000000in}}{\pgfqpoint{0.000000in}{0.055556in}}{%
\pgfpathmoveto{\pgfqpoint{0.000000in}{0.000000in}}%
\pgfpathlineto{\pgfqpoint{0.000000in}{0.055556in}}%
\pgfusepath{stroke,fill}%
}%
\begin{pgfscope}%
\pgfsys@transformshift{5.960000in}{0.600000in}%
\pgfsys@useobject{currentmarker}{}%
\end{pgfscope}%
\end{pgfscope}%
\begin{pgfscope}%
\pgfsetbuttcap%
\pgfsetroundjoin%
\definecolor{currentfill}{rgb}{0.000000,0.000000,0.000000}%
\pgfsetfillcolor{currentfill}%
\pgfsetlinewidth{0.501875pt}%
\definecolor{currentstroke}{rgb}{0.000000,0.000000,0.000000}%
\pgfsetstrokecolor{currentstroke}%
\pgfsetdash{}{0pt}%
\pgfsys@defobject{currentmarker}{\pgfqpoint{0.000000in}{-0.055556in}}{\pgfqpoint{0.000000in}{0.000000in}}{%
\pgfpathmoveto{\pgfqpoint{0.000000in}{0.000000in}}%
\pgfpathlineto{\pgfqpoint{0.000000in}{-0.055556in}}%
\pgfusepath{stroke,fill}%
}%
\begin{pgfscope}%
\pgfsys@transformshift{5.960000in}{5.400000in}%
\pgfsys@useobject{currentmarker}{}%
\end{pgfscope}%
\end{pgfscope}%
\begin{pgfscope}%
\definecolor{textcolor}{rgb}{0.000000,0.000000,0.000000}%
\pgfsetstrokecolor{textcolor}%
\pgfsetfillcolor{textcolor}%
\pgftext[x=5.960000in,y=0.544444in,,top]{\color{textcolor}\rmfamily\fontsize{10.000000}{12.000000}\selectfont \(\displaystyle {80}\)}%
\end{pgfscope}%
\begin{pgfscope}%
\pgfpathrectangle{\pgfqpoint{1.000000in}{0.600000in}}{\pgfqpoint{6.200000in}{4.800000in}}%
\pgfusepath{clip}%
\pgfsetbuttcap%
\pgfsetroundjoin%
\pgfsetlinewidth{0.501875pt}%
\definecolor{currentstroke}{rgb}{0.000000,0.000000,0.000000}%
\pgfsetstrokecolor{currentstroke}%
\pgfsetdash{{1.000000pt}{3.000000pt}}{0.000000pt}%
\pgfpathmoveto{\pgfqpoint{7.200000in}{0.600000in}}%
\pgfpathlineto{\pgfqpoint{7.200000in}{5.400000in}}%
\pgfusepath{stroke}%
\end{pgfscope}%
\begin{pgfscope}%
\pgfsetbuttcap%
\pgfsetroundjoin%
\definecolor{currentfill}{rgb}{0.000000,0.000000,0.000000}%
\pgfsetfillcolor{currentfill}%
\pgfsetlinewidth{0.501875pt}%
\definecolor{currentstroke}{rgb}{0.000000,0.000000,0.000000}%
\pgfsetstrokecolor{currentstroke}%
\pgfsetdash{}{0pt}%
\pgfsys@defobject{currentmarker}{\pgfqpoint{0.000000in}{0.000000in}}{\pgfqpoint{0.000000in}{0.055556in}}{%
\pgfpathmoveto{\pgfqpoint{0.000000in}{0.000000in}}%
\pgfpathlineto{\pgfqpoint{0.000000in}{0.055556in}}%
\pgfusepath{stroke,fill}%
}%
\begin{pgfscope}%
\pgfsys@transformshift{7.200000in}{0.600000in}%
\pgfsys@useobject{currentmarker}{}%
\end{pgfscope}%
\end{pgfscope}%
\begin{pgfscope}%
\pgfsetbuttcap%
\pgfsetroundjoin%
\definecolor{currentfill}{rgb}{0.000000,0.000000,0.000000}%
\pgfsetfillcolor{currentfill}%
\pgfsetlinewidth{0.501875pt}%
\definecolor{currentstroke}{rgb}{0.000000,0.000000,0.000000}%
\pgfsetstrokecolor{currentstroke}%
\pgfsetdash{}{0pt}%
\pgfsys@defobject{currentmarker}{\pgfqpoint{0.000000in}{-0.055556in}}{\pgfqpoint{0.000000in}{0.000000in}}{%
\pgfpathmoveto{\pgfqpoint{0.000000in}{0.000000in}}%
\pgfpathlineto{\pgfqpoint{0.000000in}{-0.055556in}}%
\pgfusepath{stroke,fill}%
}%
\begin{pgfscope}%
\pgfsys@transformshift{7.200000in}{5.400000in}%
\pgfsys@useobject{currentmarker}{}%
\end{pgfscope}%
\end{pgfscope}%
\begin{pgfscope}%
\definecolor{textcolor}{rgb}{0.000000,0.000000,0.000000}%
\pgfsetstrokecolor{textcolor}%
\pgfsetfillcolor{textcolor}%
\pgftext[x=7.200000in,y=0.544444in,,top]{\color{textcolor}\rmfamily\fontsize{10.000000}{12.000000}\selectfont \(\displaystyle {100}\)}%
\end{pgfscope}%
\begin{pgfscope}%
\definecolor{textcolor}{rgb}{0.000000,0.000000,0.000000}%
\pgfsetstrokecolor{textcolor}%
\pgfsetfillcolor{textcolor}%
\pgftext[x=4.100000in,y=0.351543in,,top]{\color{textcolor}\rmfamily\fontsize{12.000000}{14.400000}\selectfont \(\displaystyle time\ (s)\)}%
\end{pgfscope}%
\begin{pgfscope}%
\pgfpathrectangle{\pgfqpoint{1.000000in}{0.600000in}}{\pgfqpoint{6.200000in}{4.800000in}}%
\pgfusepath{clip}%
\pgfsetbuttcap%
\pgfsetroundjoin%
\pgfsetlinewidth{0.501875pt}%
\definecolor{currentstroke}{rgb}{0.000000,0.000000,0.000000}%
\pgfsetstrokecolor{currentstroke}%
\pgfsetdash{{1.000000pt}{3.000000pt}}{0.000000pt}%
\pgfpathmoveto{\pgfqpoint{1.000000in}{0.600000in}}%
\pgfpathlineto{\pgfqpoint{7.200000in}{0.600000in}}%
\pgfusepath{stroke}%
\end{pgfscope}%
\begin{pgfscope}%
\pgfsetbuttcap%
\pgfsetroundjoin%
\definecolor{currentfill}{rgb}{0.000000,0.000000,0.000000}%
\pgfsetfillcolor{currentfill}%
\pgfsetlinewidth{0.501875pt}%
\definecolor{currentstroke}{rgb}{0.000000,0.000000,0.000000}%
\pgfsetstrokecolor{currentstroke}%
\pgfsetdash{}{0pt}%
\pgfsys@defobject{currentmarker}{\pgfqpoint{0.000000in}{0.000000in}}{\pgfqpoint{0.055556in}{0.000000in}}{%
\pgfpathmoveto{\pgfqpoint{0.000000in}{0.000000in}}%
\pgfpathlineto{\pgfqpoint{0.055556in}{0.000000in}}%
\pgfusepath{stroke,fill}%
}%
\begin{pgfscope}%
\pgfsys@transformshift{1.000000in}{0.600000in}%
\pgfsys@useobject{currentmarker}{}%
\end{pgfscope}%
\end{pgfscope}%
\begin{pgfscope}%
\pgfsetbuttcap%
\pgfsetroundjoin%
\definecolor{currentfill}{rgb}{0.000000,0.000000,0.000000}%
\pgfsetfillcolor{currentfill}%
\pgfsetlinewidth{0.501875pt}%
\definecolor{currentstroke}{rgb}{0.000000,0.000000,0.000000}%
\pgfsetstrokecolor{currentstroke}%
\pgfsetdash{}{0pt}%
\pgfsys@defobject{currentmarker}{\pgfqpoint{-0.055556in}{0.000000in}}{\pgfqpoint{-0.000000in}{0.000000in}}{%
\pgfpathmoveto{\pgfqpoint{-0.000000in}{0.000000in}}%
\pgfpathlineto{\pgfqpoint{-0.055556in}{0.000000in}}%
\pgfusepath{stroke,fill}%
}%
\begin{pgfscope}%
\pgfsys@transformshift{7.200000in}{0.600000in}%
\pgfsys@useobject{currentmarker}{}%
\end{pgfscope}%
\end{pgfscope}%
\begin{pgfscope}%
\definecolor{textcolor}{rgb}{0.000000,0.000000,0.000000}%
\pgfsetstrokecolor{textcolor}%
\pgfsetfillcolor{textcolor}%
\pgftext[x=0.944444in,y=0.600000in,right,]{\color{textcolor}\rmfamily\fontsize{10.000000}{12.000000}\selectfont \(\displaystyle {\ensuremath{-}300}\)}%
\end{pgfscope}%
\begin{pgfscope}%
\pgfpathrectangle{\pgfqpoint{1.000000in}{0.600000in}}{\pgfqpoint{6.200000in}{4.800000in}}%
\pgfusepath{clip}%
\pgfsetbuttcap%
\pgfsetroundjoin%
\pgfsetlinewidth{0.501875pt}%
\definecolor{currentstroke}{rgb}{0.000000,0.000000,0.000000}%
\pgfsetstrokecolor{currentstroke}%
\pgfsetdash{{1.000000pt}{3.000000pt}}{0.000000pt}%
\pgfpathmoveto{\pgfqpoint{1.000000in}{1.400000in}}%
\pgfpathlineto{\pgfqpoint{7.200000in}{1.400000in}}%
\pgfusepath{stroke}%
\end{pgfscope}%
\begin{pgfscope}%
\pgfsetbuttcap%
\pgfsetroundjoin%
\definecolor{currentfill}{rgb}{0.000000,0.000000,0.000000}%
\pgfsetfillcolor{currentfill}%
\pgfsetlinewidth{0.501875pt}%
\definecolor{currentstroke}{rgb}{0.000000,0.000000,0.000000}%
\pgfsetstrokecolor{currentstroke}%
\pgfsetdash{}{0pt}%
\pgfsys@defobject{currentmarker}{\pgfqpoint{0.000000in}{0.000000in}}{\pgfqpoint{0.055556in}{0.000000in}}{%
\pgfpathmoveto{\pgfqpoint{0.000000in}{0.000000in}}%
\pgfpathlineto{\pgfqpoint{0.055556in}{0.000000in}}%
\pgfusepath{stroke,fill}%
}%
\begin{pgfscope}%
\pgfsys@transformshift{1.000000in}{1.400000in}%
\pgfsys@useobject{currentmarker}{}%
\end{pgfscope}%
\end{pgfscope}%
\begin{pgfscope}%
\pgfsetbuttcap%
\pgfsetroundjoin%
\definecolor{currentfill}{rgb}{0.000000,0.000000,0.000000}%
\pgfsetfillcolor{currentfill}%
\pgfsetlinewidth{0.501875pt}%
\definecolor{currentstroke}{rgb}{0.000000,0.000000,0.000000}%
\pgfsetstrokecolor{currentstroke}%
\pgfsetdash{}{0pt}%
\pgfsys@defobject{currentmarker}{\pgfqpoint{-0.055556in}{0.000000in}}{\pgfqpoint{-0.000000in}{0.000000in}}{%
\pgfpathmoveto{\pgfqpoint{-0.000000in}{0.000000in}}%
\pgfpathlineto{\pgfqpoint{-0.055556in}{0.000000in}}%
\pgfusepath{stroke,fill}%
}%
\begin{pgfscope}%
\pgfsys@transformshift{7.200000in}{1.400000in}%
\pgfsys@useobject{currentmarker}{}%
\end{pgfscope}%
\end{pgfscope}%
\begin{pgfscope}%
\definecolor{textcolor}{rgb}{0.000000,0.000000,0.000000}%
\pgfsetstrokecolor{textcolor}%
\pgfsetfillcolor{textcolor}%
\pgftext[x=0.944444in,y=1.400000in,right,]{\color{textcolor}\rmfamily\fontsize{10.000000}{12.000000}\selectfont \(\displaystyle {\ensuremath{-}200}\)}%
\end{pgfscope}%
\begin{pgfscope}%
\pgfpathrectangle{\pgfqpoint{1.000000in}{0.600000in}}{\pgfqpoint{6.200000in}{4.800000in}}%
\pgfusepath{clip}%
\pgfsetbuttcap%
\pgfsetroundjoin%
\pgfsetlinewidth{0.501875pt}%
\definecolor{currentstroke}{rgb}{0.000000,0.000000,0.000000}%
\pgfsetstrokecolor{currentstroke}%
\pgfsetdash{{1.000000pt}{3.000000pt}}{0.000000pt}%
\pgfpathmoveto{\pgfqpoint{1.000000in}{2.200000in}}%
\pgfpathlineto{\pgfqpoint{7.200000in}{2.200000in}}%
\pgfusepath{stroke}%
\end{pgfscope}%
\begin{pgfscope}%
\pgfsetbuttcap%
\pgfsetroundjoin%
\definecolor{currentfill}{rgb}{0.000000,0.000000,0.000000}%
\pgfsetfillcolor{currentfill}%
\pgfsetlinewidth{0.501875pt}%
\definecolor{currentstroke}{rgb}{0.000000,0.000000,0.000000}%
\pgfsetstrokecolor{currentstroke}%
\pgfsetdash{}{0pt}%
\pgfsys@defobject{currentmarker}{\pgfqpoint{0.000000in}{0.000000in}}{\pgfqpoint{0.055556in}{0.000000in}}{%
\pgfpathmoveto{\pgfqpoint{0.000000in}{0.000000in}}%
\pgfpathlineto{\pgfqpoint{0.055556in}{0.000000in}}%
\pgfusepath{stroke,fill}%
}%
\begin{pgfscope}%
\pgfsys@transformshift{1.000000in}{2.200000in}%
\pgfsys@useobject{currentmarker}{}%
\end{pgfscope}%
\end{pgfscope}%
\begin{pgfscope}%
\pgfsetbuttcap%
\pgfsetroundjoin%
\definecolor{currentfill}{rgb}{0.000000,0.000000,0.000000}%
\pgfsetfillcolor{currentfill}%
\pgfsetlinewidth{0.501875pt}%
\definecolor{currentstroke}{rgb}{0.000000,0.000000,0.000000}%
\pgfsetstrokecolor{currentstroke}%
\pgfsetdash{}{0pt}%
\pgfsys@defobject{currentmarker}{\pgfqpoint{-0.055556in}{0.000000in}}{\pgfqpoint{-0.000000in}{0.000000in}}{%
\pgfpathmoveto{\pgfqpoint{-0.000000in}{0.000000in}}%
\pgfpathlineto{\pgfqpoint{-0.055556in}{0.000000in}}%
\pgfusepath{stroke,fill}%
}%
\begin{pgfscope}%
\pgfsys@transformshift{7.200000in}{2.200000in}%
\pgfsys@useobject{currentmarker}{}%
\end{pgfscope}%
\end{pgfscope}%
\begin{pgfscope}%
\definecolor{textcolor}{rgb}{0.000000,0.000000,0.000000}%
\pgfsetstrokecolor{textcolor}%
\pgfsetfillcolor{textcolor}%
\pgftext[x=0.944444in,y=2.200000in,right,]{\color{textcolor}\rmfamily\fontsize{10.000000}{12.000000}\selectfont \(\displaystyle {\ensuremath{-}100}\)}%
\end{pgfscope}%
\begin{pgfscope}%
\pgfpathrectangle{\pgfqpoint{1.000000in}{0.600000in}}{\pgfqpoint{6.200000in}{4.800000in}}%
\pgfusepath{clip}%
\pgfsetbuttcap%
\pgfsetroundjoin%
\pgfsetlinewidth{0.501875pt}%
\definecolor{currentstroke}{rgb}{0.000000,0.000000,0.000000}%
\pgfsetstrokecolor{currentstroke}%
\pgfsetdash{{1.000000pt}{3.000000pt}}{0.000000pt}%
\pgfpathmoveto{\pgfqpoint{1.000000in}{3.000000in}}%
\pgfpathlineto{\pgfqpoint{7.200000in}{3.000000in}}%
\pgfusepath{stroke}%
\end{pgfscope}%
\begin{pgfscope}%
\pgfsetbuttcap%
\pgfsetroundjoin%
\definecolor{currentfill}{rgb}{0.000000,0.000000,0.000000}%
\pgfsetfillcolor{currentfill}%
\pgfsetlinewidth{0.501875pt}%
\definecolor{currentstroke}{rgb}{0.000000,0.000000,0.000000}%
\pgfsetstrokecolor{currentstroke}%
\pgfsetdash{}{0pt}%
\pgfsys@defobject{currentmarker}{\pgfqpoint{0.000000in}{0.000000in}}{\pgfqpoint{0.055556in}{0.000000in}}{%
\pgfpathmoveto{\pgfqpoint{0.000000in}{0.000000in}}%
\pgfpathlineto{\pgfqpoint{0.055556in}{0.000000in}}%
\pgfusepath{stroke,fill}%
}%
\begin{pgfscope}%
\pgfsys@transformshift{1.000000in}{3.000000in}%
\pgfsys@useobject{currentmarker}{}%
\end{pgfscope}%
\end{pgfscope}%
\begin{pgfscope}%
\pgfsetbuttcap%
\pgfsetroundjoin%
\definecolor{currentfill}{rgb}{0.000000,0.000000,0.000000}%
\pgfsetfillcolor{currentfill}%
\pgfsetlinewidth{0.501875pt}%
\definecolor{currentstroke}{rgb}{0.000000,0.000000,0.000000}%
\pgfsetstrokecolor{currentstroke}%
\pgfsetdash{}{0pt}%
\pgfsys@defobject{currentmarker}{\pgfqpoint{-0.055556in}{0.000000in}}{\pgfqpoint{-0.000000in}{0.000000in}}{%
\pgfpathmoveto{\pgfqpoint{-0.000000in}{0.000000in}}%
\pgfpathlineto{\pgfqpoint{-0.055556in}{0.000000in}}%
\pgfusepath{stroke,fill}%
}%
\begin{pgfscope}%
\pgfsys@transformshift{7.200000in}{3.000000in}%
\pgfsys@useobject{currentmarker}{}%
\end{pgfscope}%
\end{pgfscope}%
\begin{pgfscope}%
\definecolor{textcolor}{rgb}{0.000000,0.000000,0.000000}%
\pgfsetstrokecolor{textcolor}%
\pgfsetfillcolor{textcolor}%
\pgftext[x=0.944444in,y=3.000000in,right,]{\color{textcolor}\rmfamily\fontsize{10.000000}{12.000000}\selectfont \(\displaystyle {0}\)}%
\end{pgfscope}%
\begin{pgfscope}%
\pgfpathrectangle{\pgfqpoint{1.000000in}{0.600000in}}{\pgfqpoint{6.200000in}{4.800000in}}%
\pgfusepath{clip}%
\pgfsetbuttcap%
\pgfsetroundjoin%
\pgfsetlinewidth{0.501875pt}%
\definecolor{currentstroke}{rgb}{0.000000,0.000000,0.000000}%
\pgfsetstrokecolor{currentstroke}%
\pgfsetdash{{1.000000pt}{3.000000pt}}{0.000000pt}%
\pgfpathmoveto{\pgfqpoint{1.000000in}{3.800000in}}%
\pgfpathlineto{\pgfqpoint{7.200000in}{3.800000in}}%
\pgfusepath{stroke}%
\end{pgfscope}%
\begin{pgfscope}%
\pgfsetbuttcap%
\pgfsetroundjoin%
\definecolor{currentfill}{rgb}{0.000000,0.000000,0.000000}%
\pgfsetfillcolor{currentfill}%
\pgfsetlinewidth{0.501875pt}%
\definecolor{currentstroke}{rgb}{0.000000,0.000000,0.000000}%
\pgfsetstrokecolor{currentstroke}%
\pgfsetdash{}{0pt}%
\pgfsys@defobject{currentmarker}{\pgfqpoint{0.000000in}{0.000000in}}{\pgfqpoint{0.055556in}{0.000000in}}{%
\pgfpathmoveto{\pgfqpoint{0.000000in}{0.000000in}}%
\pgfpathlineto{\pgfqpoint{0.055556in}{0.000000in}}%
\pgfusepath{stroke,fill}%
}%
\begin{pgfscope}%
\pgfsys@transformshift{1.000000in}{3.800000in}%
\pgfsys@useobject{currentmarker}{}%
\end{pgfscope}%
\end{pgfscope}%
\begin{pgfscope}%
\pgfsetbuttcap%
\pgfsetroundjoin%
\definecolor{currentfill}{rgb}{0.000000,0.000000,0.000000}%
\pgfsetfillcolor{currentfill}%
\pgfsetlinewidth{0.501875pt}%
\definecolor{currentstroke}{rgb}{0.000000,0.000000,0.000000}%
\pgfsetstrokecolor{currentstroke}%
\pgfsetdash{}{0pt}%
\pgfsys@defobject{currentmarker}{\pgfqpoint{-0.055556in}{0.000000in}}{\pgfqpoint{-0.000000in}{0.000000in}}{%
\pgfpathmoveto{\pgfqpoint{-0.000000in}{0.000000in}}%
\pgfpathlineto{\pgfqpoint{-0.055556in}{0.000000in}}%
\pgfusepath{stroke,fill}%
}%
\begin{pgfscope}%
\pgfsys@transformshift{7.200000in}{3.800000in}%
\pgfsys@useobject{currentmarker}{}%
\end{pgfscope}%
\end{pgfscope}%
\begin{pgfscope}%
\definecolor{textcolor}{rgb}{0.000000,0.000000,0.000000}%
\pgfsetstrokecolor{textcolor}%
\pgfsetfillcolor{textcolor}%
\pgftext[x=0.944444in,y=3.800000in,right,]{\color{textcolor}\rmfamily\fontsize{10.000000}{12.000000}\selectfont \(\displaystyle {100}\)}%
\end{pgfscope}%
\begin{pgfscope}%
\pgfpathrectangle{\pgfqpoint{1.000000in}{0.600000in}}{\pgfqpoint{6.200000in}{4.800000in}}%
\pgfusepath{clip}%
\pgfsetbuttcap%
\pgfsetroundjoin%
\pgfsetlinewidth{0.501875pt}%
\definecolor{currentstroke}{rgb}{0.000000,0.000000,0.000000}%
\pgfsetstrokecolor{currentstroke}%
\pgfsetdash{{1.000000pt}{3.000000pt}}{0.000000pt}%
\pgfpathmoveto{\pgfqpoint{1.000000in}{4.600000in}}%
\pgfpathlineto{\pgfqpoint{7.200000in}{4.600000in}}%
\pgfusepath{stroke}%
\end{pgfscope}%
\begin{pgfscope}%
\pgfsetbuttcap%
\pgfsetroundjoin%
\definecolor{currentfill}{rgb}{0.000000,0.000000,0.000000}%
\pgfsetfillcolor{currentfill}%
\pgfsetlinewidth{0.501875pt}%
\definecolor{currentstroke}{rgb}{0.000000,0.000000,0.000000}%
\pgfsetstrokecolor{currentstroke}%
\pgfsetdash{}{0pt}%
\pgfsys@defobject{currentmarker}{\pgfqpoint{0.000000in}{0.000000in}}{\pgfqpoint{0.055556in}{0.000000in}}{%
\pgfpathmoveto{\pgfqpoint{0.000000in}{0.000000in}}%
\pgfpathlineto{\pgfqpoint{0.055556in}{0.000000in}}%
\pgfusepath{stroke,fill}%
}%
\begin{pgfscope}%
\pgfsys@transformshift{1.000000in}{4.600000in}%
\pgfsys@useobject{currentmarker}{}%
\end{pgfscope}%
\end{pgfscope}%
\begin{pgfscope}%
\pgfsetbuttcap%
\pgfsetroundjoin%
\definecolor{currentfill}{rgb}{0.000000,0.000000,0.000000}%
\pgfsetfillcolor{currentfill}%
\pgfsetlinewidth{0.501875pt}%
\definecolor{currentstroke}{rgb}{0.000000,0.000000,0.000000}%
\pgfsetstrokecolor{currentstroke}%
\pgfsetdash{}{0pt}%
\pgfsys@defobject{currentmarker}{\pgfqpoint{-0.055556in}{0.000000in}}{\pgfqpoint{-0.000000in}{0.000000in}}{%
\pgfpathmoveto{\pgfqpoint{-0.000000in}{0.000000in}}%
\pgfpathlineto{\pgfqpoint{-0.055556in}{0.000000in}}%
\pgfusepath{stroke,fill}%
}%
\begin{pgfscope}%
\pgfsys@transformshift{7.200000in}{4.600000in}%
\pgfsys@useobject{currentmarker}{}%
\end{pgfscope}%
\end{pgfscope}%
\begin{pgfscope}%
\definecolor{textcolor}{rgb}{0.000000,0.000000,0.000000}%
\pgfsetstrokecolor{textcolor}%
\pgfsetfillcolor{textcolor}%
\pgftext[x=0.944444in,y=4.600000in,right,]{\color{textcolor}\rmfamily\fontsize{10.000000}{12.000000}\selectfont \(\displaystyle {200}\)}%
\end{pgfscope}%
\begin{pgfscope}%
\pgfpathrectangle{\pgfqpoint{1.000000in}{0.600000in}}{\pgfqpoint{6.200000in}{4.800000in}}%
\pgfusepath{clip}%
\pgfsetbuttcap%
\pgfsetroundjoin%
\pgfsetlinewidth{0.501875pt}%
\definecolor{currentstroke}{rgb}{0.000000,0.000000,0.000000}%
\pgfsetstrokecolor{currentstroke}%
\pgfsetdash{{1.000000pt}{3.000000pt}}{0.000000pt}%
\pgfpathmoveto{\pgfqpoint{1.000000in}{5.400000in}}%
\pgfpathlineto{\pgfqpoint{7.200000in}{5.400000in}}%
\pgfusepath{stroke}%
\end{pgfscope}%
\begin{pgfscope}%
\pgfsetbuttcap%
\pgfsetroundjoin%
\definecolor{currentfill}{rgb}{0.000000,0.000000,0.000000}%
\pgfsetfillcolor{currentfill}%
\pgfsetlinewidth{0.501875pt}%
\definecolor{currentstroke}{rgb}{0.000000,0.000000,0.000000}%
\pgfsetstrokecolor{currentstroke}%
\pgfsetdash{}{0pt}%
\pgfsys@defobject{currentmarker}{\pgfqpoint{0.000000in}{0.000000in}}{\pgfqpoint{0.055556in}{0.000000in}}{%
\pgfpathmoveto{\pgfqpoint{0.000000in}{0.000000in}}%
\pgfpathlineto{\pgfqpoint{0.055556in}{0.000000in}}%
\pgfusepath{stroke,fill}%
}%
\begin{pgfscope}%
\pgfsys@transformshift{1.000000in}{5.400000in}%
\pgfsys@useobject{currentmarker}{}%
\end{pgfscope}%
\end{pgfscope}%
\begin{pgfscope}%
\pgfsetbuttcap%
\pgfsetroundjoin%
\definecolor{currentfill}{rgb}{0.000000,0.000000,0.000000}%
\pgfsetfillcolor{currentfill}%
\pgfsetlinewidth{0.501875pt}%
\definecolor{currentstroke}{rgb}{0.000000,0.000000,0.000000}%
\pgfsetstrokecolor{currentstroke}%
\pgfsetdash{}{0pt}%
\pgfsys@defobject{currentmarker}{\pgfqpoint{-0.055556in}{0.000000in}}{\pgfqpoint{-0.000000in}{0.000000in}}{%
\pgfpathmoveto{\pgfqpoint{-0.000000in}{0.000000in}}%
\pgfpathlineto{\pgfqpoint{-0.055556in}{0.000000in}}%
\pgfusepath{stroke,fill}%
}%
\begin{pgfscope}%
\pgfsys@transformshift{7.200000in}{5.400000in}%
\pgfsys@useobject{currentmarker}{}%
\end{pgfscope}%
\end{pgfscope}%
\begin{pgfscope}%
\definecolor{textcolor}{rgb}{0.000000,0.000000,0.000000}%
\pgfsetstrokecolor{textcolor}%
\pgfsetfillcolor{textcolor}%
\pgftext[x=0.944444in,y=5.400000in,right,]{\color{textcolor}\rmfamily\fontsize{10.000000}{12.000000}\selectfont \(\displaystyle {300}\)}%
\end{pgfscope}%
\begin{pgfscope}%
\definecolor{textcolor}{rgb}{0.000000,0.000000,0.000000}%
\pgfsetstrokecolor{textcolor}%
\pgfsetfillcolor{textcolor}%
\pgftext[x=0.558641in,y=3.000000in,,bottom,rotate=90.000000]{\color{textcolor}\rmfamily\fontsize{12.000000}{14.400000}\selectfont \(\displaystyle \theta\ (rad)\)}%
\end{pgfscope}%
\begin{pgfscope}%
\definecolor{textcolor}{rgb}{0.000000,0.000000,0.000000}%
\pgfsetstrokecolor{textcolor}%
\pgfsetfillcolor{textcolor}%
\pgftext[x=4.100000in,y=5.469444in,,base]{\color{textcolor}\rmfamily\fontsize{12.000000}{14.400000}\selectfont \(\displaystyle Simple\ pendulum\ solution\ (time\ step = 1\ s)\)}%
\end{pgfscope}%
\begin{pgfscope}%
\pgfsetbuttcap%
\pgfsetmiterjoin%
\definecolor{currentfill}{rgb}{1.000000,1.000000,1.000000}%
\pgfsetfillcolor{currentfill}%
\pgfsetlinewidth{1.003750pt}%
\definecolor{currentstroke}{rgb}{0.000000,0.000000,0.000000}%
\pgfsetstrokecolor{currentstroke}%
\pgfsetdash{}{0pt}%
\pgfpathmoveto{\pgfqpoint{1.083333in}{4.569445in}}%
\pgfpathlineto{\pgfqpoint{3.093110in}{4.569445in}}%
\pgfpathlineto{\pgfqpoint{3.093110in}{5.316667in}}%
\pgfpathlineto{\pgfqpoint{1.083333in}{5.316667in}}%
\pgfpathlineto{\pgfqpoint{1.083333in}{4.569445in}}%
\pgfpathclose%
\pgfusepath{stroke,fill}%
\end{pgfscope}%
\begin{pgfscope}%
\pgfsetrectcap%
\pgfsetroundjoin%
\pgfsetlinewidth{1.003750pt}%
\definecolor{currentstroke}{rgb}{1.000000,0.000000,0.000000}%
\pgfsetstrokecolor{currentstroke}%
\pgfsetdash{}{0pt}%
\pgfpathmoveto{\pgfqpoint{1.200000in}{5.191667in}}%
\pgfpathlineto{\pgfqpoint{1.433333in}{5.191667in}}%
\pgfusepath{stroke}%
\end{pgfscope}%
\begin{pgfscope}%
\definecolor{textcolor}{rgb}{0.000000,0.000000,0.000000}%
\pgfsetstrokecolor{textcolor}%
\pgfsetfillcolor{textcolor}%
\pgftext[x=1.616667in,y=5.133333in,left,base]{\color{textcolor}\rmfamily\fontsize{12.000000}{14.400000}\selectfont \(\displaystyle euler\ explicit\)}%
\end{pgfscope}%
\begin{pgfscope}%
\pgfsetrectcap%
\pgfsetroundjoin%
\pgfsetlinewidth{1.003750pt}%
\definecolor{currentstroke}{rgb}{0.000000,0.000000,1.000000}%
\pgfsetstrokecolor{currentstroke}%
\pgfsetdash{}{0pt}%
\pgfpathmoveto{\pgfqpoint{1.200000in}{4.959260in}}%
\pgfpathlineto{\pgfqpoint{1.433333in}{4.959260in}}%
\pgfusepath{stroke}%
\end{pgfscope}%
\begin{pgfscope}%
\definecolor{textcolor}{rgb}{0.000000,0.000000,0.000000}%
\pgfsetstrokecolor{textcolor}%
\pgfsetfillcolor{textcolor}%
\pgftext[x=1.616667in,y=4.900926in,left,base]{\color{textcolor}\rmfamily\fontsize{12.000000}{14.400000}\selectfont \(\displaystyle euler\ implicit\)}%
\end{pgfscope}%
\begin{pgfscope}%
\pgfsetrectcap%
\pgfsetroundjoin%
\pgfsetlinewidth{1.003750pt}%
\definecolor{currentstroke}{rgb}{0.000000,0.000000,0.000000}%
\pgfsetstrokecolor{currentstroke}%
\pgfsetdash{}{0pt}%
\pgfpathmoveto{\pgfqpoint{1.200000in}{4.726852in}}%
\pgfpathlineto{\pgfqpoint{1.433333in}{4.726852in}}%
\pgfusepath{stroke}%
\end{pgfscope}%
\begin{pgfscope}%
\definecolor{textcolor}{rgb}{0.000000,0.000000,0.000000}%
\pgfsetstrokecolor{textcolor}%
\pgfsetfillcolor{textcolor}%
\pgftext[x=1.616667in,y=4.668519in,left,base]{\color{textcolor}\rmfamily\fontsize{12.000000}{14.400000}\selectfont \(\displaystyle trapezoidal\ scheme\)}%
\end{pgfscope}%
\end{pgfpicture}%
\makeatother%
\endgroup%
}
    \end{figure}
    
    \begin{figure}[ht!]
    \centering
    \resizebox{0.9\linewidth}{!}{%% Creator: Matplotlib, PGF backend
%%
%% To include the figure in your LaTeX document, write
%%   \input{<filename>.pgf}
%%
%% Make sure the required packages are loaded in your preamble
%%   \usepackage{pgf}
%%
%% Also ensure that all the required font packages are loaded; for instance,
%% the lmodern package is sometimes necessary when using math font.
%%   \usepackage{lmodern}
%%
%% Figures using additional raster images can only be included by \input if
%% they are in the same directory as the main LaTeX file. For loading figures
%% from other directories you can use the `import` package
%%   \usepackage{import}
%%
%% and then include the figures with
%%   \import{<path to file>}{<filename>.pgf}
%%
%% Matplotlib used the following preamble
%%
\begingroup%
\makeatletter%
\begin{pgfpicture}%
\pgfpathrectangle{\pgfpointorigin}{\pgfqpoint{8.000000in}{6.000000in}}%
\pgfusepath{use as bounding box, clip}%
\begin{pgfscope}%
\pgfsetbuttcap%
\pgfsetmiterjoin%
\definecolor{currentfill}{rgb}{1.000000,1.000000,1.000000}%
\pgfsetfillcolor{currentfill}%
\pgfsetlinewidth{0.000000pt}%
\definecolor{currentstroke}{rgb}{1.000000,1.000000,1.000000}%
\pgfsetstrokecolor{currentstroke}%
\pgfsetdash{}{0pt}%
\pgfpathmoveto{\pgfqpoint{0.000000in}{0.000000in}}%
\pgfpathlineto{\pgfqpoint{8.000000in}{0.000000in}}%
\pgfpathlineto{\pgfqpoint{8.000000in}{6.000000in}}%
\pgfpathlineto{\pgfqpoint{0.000000in}{6.000000in}}%
\pgfpathlineto{\pgfqpoint{0.000000in}{0.000000in}}%
\pgfpathclose%
\pgfusepath{fill}%
\end{pgfscope}%
\begin{pgfscope}%
\pgfsetbuttcap%
\pgfsetmiterjoin%
\definecolor{currentfill}{rgb}{1.000000,1.000000,1.000000}%
\pgfsetfillcolor{currentfill}%
\pgfsetlinewidth{0.000000pt}%
\definecolor{currentstroke}{rgb}{0.000000,0.000000,0.000000}%
\pgfsetstrokecolor{currentstroke}%
\pgfsetstrokeopacity{0.000000}%
\pgfsetdash{}{0pt}%
\pgfpathmoveto{\pgfqpoint{1.000000in}{0.600000in}}%
\pgfpathlineto{\pgfqpoint{7.200000in}{0.600000in}}%
\pgfpathlineto{\pgfqpoint{7.200000in}{5.400000in}}%
\pgfpathlineto{\pgfqpoint{1.000000in}{5.400000in}}%
\pgfpathlineto{\pgfqpoint{1.000000in}{0.600000in}}%
\pgfpathclose%
\pgfusepath{fill}%
\end{pgfscope}%
\begin{pgfscope}%
\pgfpathrectangle{\pgfqpoint{1.000000in}{0.600000in}}{\pgfqpoint{6.200000in}{4.800000in}}%
\pgfusepath{clip}%
\pgfsetrectcap%
\pgfsetroundjoin%
\pgfsetlinewidth{1.003750pt}%
\definecolor{currentstroke}{rgb}{1.000000,0.000000,0.000000}%
\pgfsetstrokecolor{currentstroke}%
\pgfsetdash{}{0pt}%
\pgfpathmoveto{\pgfqpoint{1.000000in}{1.289305in}}%
\pgfpathlineto{\pgfqpoint{1.124000in}{1.289305in}}%
\pgfpathlineto{\pgfqpoint{1.248000in}{1.275590in}}%
\pgfpathlineto{\pgfqpoint{1.372000in}{1.248162in}}%
\pgfpathlineto{\pgfqpoint{1.496000in}{1.248040in}}%
\pgfpathlineto{\pgfqpoint{1.620000in}{1.228112in}}%
\pgfpathlineto{\pgfqpoint{1.744000in}{1.188721in}}%
\pgfpathlineto{\pgfqpoint{1.868000in}{1.172766in}}%
\pgfpathlineto{\pgfqpoint{1.992000in}{1.184238in}}%
\pgfpathlineto{\pgfqpoint{2.116000in}{1.176741in}}%
\pgfpathlineto{\pgfqpoint{2.240000in}{1.190889in}}%
\pgfpathlineto{\pgfqpoint{2.364000in}{1.200019in}}%
\pgfpathlineto{\pgfqpoint{2.488000in}{1.235284in}}%
\pgfpathlineto{\pgfqpoint{2.612000in}{1.268653in}}%
\pgfpathlineto{\pgfqpoint{2.736000in}{1.326099in}}%
\pgfpathlineto{\pgfqpoint{2.860000in}{1.400221in}}%
\pgfpathlineto{\pgfqpoint{2.984000in}{1.484865in}}%
\pgfpathlineto{\pgfqpoint{3.108000in}{1.592455in}}%
\pgfpathlineto{\pgfqpoint{3.232000in}{1.719110in}}%
\pgfpathlineto{\pgfqpoint{3.356000in}{1.827054in}}%
\pgfpathlineto{\pgfqpoint{3.480000in}{1.925037in}}%
\pgfpathlineto{\pgfqpoint{3.604000in}{2.033842in}}%
\pgfpathlineto{\pgfqpoint{3.728000in}{2.165920in}}%
\pgfpathlineto{\pgfqpoint{3.852000in}{2.277322in}}%
\pgfpathlineto{\pgfqpoint{3.976000in}{2.376994in}}%
\pgfpathlineto{\pgfqpoint{4.100000in}{2.474028in}}%
\pgfpathlineto{\pgfqpoint{4.224000in}{2.546919in}}%
\pgfpathlineto{\pgfqpoint{4.348000in}{2.633157in}}%
\pgfpathlineto{\pgfqpoint{4.472000in}{2.692245in}}%
\pgfpathlineto{\pgfqpoint{4.596000in}{2.724222in}}%
\pgfpathlineto{\pgfqpoint{4.720000in}{2.777946in}}%
\pgfpathlineto{\pgfqpoint{4.844000in}{2.813910in}}%
\pgfpathlineto{\pgfqpoint{4.968000in}{2.870431in}}%
\pgfpathlineto{\pgfqpoint{5.092000in}{2.921740in}}%
\pgfpathlineto{\pgfqpoint{5.216000in}{2.999944in}}%
\pgfpathlineto{\pgfqpoint{5.340000in}{3.082896in}}%
\pgfpathlineto{\pgfqpoint{5.464000in}{3.192520in}}%
\pgfpathlineto{\pgfqpoint{5.588000in}{3.328825in}}%
\pgfpathlineto{\pgfqpoint{5.712000in}{3.437729in}}%
\pgfpathlineto{\pgfqpoint{5.836000in}{3.533539in}}%
\pgfpathlineto{\pgfqpoint{5.960000in}{3.638076in}}%
\pgfpathlineto{\pgfqpoint{6.084000in}{3.718394in}}%
\pgfpathlineto{\pgfqpoint{6.208000in}{3.814631in}}%
\pgfpathlineto{\pgfqpoint{6.332000in}{3.904504in}}%
\pgfpathlineto{\pgfqpoint{6.456000in}{4.020264in}}%
\pgfpathlineto{\pgfqpoint{6.580000in}{4.162888in}}%
\pgfpathlineto{\pgfqpoint{6.704000in}{4.300230in}}%
\pgfpathlineto{\pgfqpoint{6.828000in}{4.464531in}}%
\pgfpathlineto{\pgfqpoint{6.952000in}{4.634447in}}%
\pgfpathlineto{\pgfqpoint{7.076000in}{4.831284in}}%
\pgfpathlineto{\pgfqpoint{7.200000in}{5.055197in}}%
\pgfusepath{stroke}%
\end{pgfscope}%
\begin{pgfscope}%
\pgfpathrectangle{\pgfqpoint{1.000000in}{0.600000in}}{\pgfqpoint{6.200000in}{4.800000in}}%
\pgfusepath{clip}%
\pgfsetrectcap%
\pgfsetroundjoin%
\pgfsetlinewidth{1.003750pt}%
\definecolor{currentstroke}{rgb}{0.000000,0.000000,1.000000}%
\pgfsetstrokecolor{currentstroke}%
\pgfsetdash{}{0pt}%
\pgfpathmoveto{\pgfqpoint{1.000000in}{1.289305in}}%
\pgfpathlineto{\pgfqpoint{1.124000in}{1.275590in}}%
\pgfpathlineto{\pgfqpoint{1.248000in}{1.289183in}}%
\pgfpathlineto{\pgfqpoint{1.372000in}{1.253166in}}%
\pgfpathlineto{\pgfqpoint{1.496000in}{1.211994in}}%
\pgfpathlineto{\pgfqpoint{1.620000in}{1.184046in}}%
\pgfpathlineto{\pgfqpoint{1.744000in}{1.176478in}}%
\pgfpathlineto{\pgfqpoint{1.868000in}{1.146257in}}%
\pgfpathlineto{\pgfqpoint{1.992000in}{1.165037in}}%
\pgfpathlineto{\pgfqpoint{2.116000in}{1.141283in}}%
\pgfpathlineto{\pgfqpoint{2.240000in}{1.138098in}}%
\pgfpathlineto{\pgfqpoint{2.364000in}{1.114466in}}%
\pgfpathlineto{\pgfqpoint{2.488000in}{1.101602in}}%
\pgfpathlineto{\pgfqpoint{2.612000in}{1.144455in}}%
\pgfpathlineto{\pgfqpoint{2.736000in}{1.172239in}}%
\pgfpathlineto{\pgfqpoint{2.860000in}{1.179975in}}%
\pgfpathlineto{\pgfqpoint{2.984000in}{1.212478in}}%
\pgfpathlineto{\pgfqpoint{3.108000in}{1.192401in}}%
\pgfpathlineto{\pgfqpoint{3.232000in}{1.211894in}}%
\pgfpathlineto{\pgfqpoint{3.356000in}{1.208339in}}%
\pgfpathlineto{\pgfqpoint{3.480000in}{1.229713in}}%
\pgfpathlineto{\pgfqpoint{3.604000in}{1.278015in}}%
\pgfpathlineto{\pgfqpoint{3.728000in}{1.298152in}}%
\pgfpathlineto{\pgfqpoint{3.852000in}{1.293707in}}%
\pgfpathlineto{\pgfqpoint{3.976000in}{1.316575in}}%
\pgfpathlineto{\pgfqpoint{4.100000in}{1.365452in}}%
\pgfpathlineto{\pgfqpoint{4.224000in}{1.382456in}}%
\pgfpathlineto{\pgfqpoint{4.348000in}{1.377181in}}%
\pgfpathlineto{\pgfqpoint{4.472000in}{1.407561in}}%
\pgfpathlineto{\pgfqpoint{4.596000in}{1.454121in}}%
\pgfpathlineto{\pgfqpoint{4.720000in}{1.468896in}}%
\pgfpathlineto{\pgfqpoint{4.844000in}{1.456595in}}%
\pgfpathlineto{\pgfqpoint{4.968000in}{1.494012in}}%
\pgfpathlineto{\pgfqpoint{5.092000in}{1.575387in}}%
\pgfpathlineto{\pgfqpoint{5.216000in}{1.635758in}}%
\pgfpathlineto{\pgfqpoint{5.340000in}{1.649202in}}%
\pgfpathlineto{\pgfqpoint{5.464000in}{1.689424in}}%
\pgfpathlineto{\pgfqpoint{5.588000in}{1.738730in}}%
\pgfpathlineto{\pgfqpoint{5.712000in}{1.814257in}}%
\pgfpathlineto{\pgfqpoint{5.836000in}{1.833476in}}%
\pgfpathlineto{\pgfqpoint{5.960000in}{1.884120in}}%
\pgfpathlineto{\pgfqpoint{6.084000in}{1.898425in}}%
\pgfpathlineto{\pgfqpoint{6.208000in}{1.893493in}}%
\pgfpathlineto{\pgfqpoint{6.332000in}{1.925777in}}%
\pgfpathlineto{\pgfqpoint{6.456000in}{1.969238in}}%
\pgfpathlineto{\pgfqpoint{6.580000in}{1.991032in}}%
\pgfpathlineto{\pgfqpoint{6.704000in}{1.989491in}}%
\pgfpathlineto{\pgfqpoint{6.828000in}{2.010471in}}%
\pgfpathlineto{\pgfqpoint{6.952000in}{2.059809in}}%
\pgfpathlineto{\pgfqpoint{7.076000in}{2.073136in}}%
\pgfpathlineto{\pgfqpoint{7.200000in}{2.032122in}}%
\pgfusepath{stroke}%
\end{pgfscope}%
\begin{pgfscope}%
\pgfpathrectangle{\pgfqpoint{1.000000in}{0.600000in}}{\pgfqpoint{6.200000in}{4.800000in}}%
\pgfusepath{clip}%
\pgfsetrectcap%
\pgfsetroundjoin%
\pgfsetlinewidth{1.003750pt}%
\definecolor{currentstroke}{rgb}{0.000000,0.000000,0.000000}%
\pgfsetstrokecolor{currentstroke}%
\pgfsetdash{}{0pt}%
\pgfpathmoveto{\pgfqpoint{1.000000in}{1.289305in}}%
\pgfpathlineto{\pgfqpoint{1.124000in}{1.282448in}}%
\pgfpathlineto{\pgfqpoint{1.248000in}{1.275022in}}%
\pgfpathlineto{\pgfqpoint{1.372000in}{1.289775in}}%
\pgfpathlineto{\pgfqpoint{1.496000in}{1.310405in}}%
\pgfpathlineto{\pgfqpoint{1.620000in}{1.350826in}}%
\pgfpathlineto{\pgfqpoint{1.744000in}{1.382230in}}%
\pgfpathlineto{\pgfqpoint{1.868000in}{1.386514in}}%
\pgfpathlineto{\pgfqpoint{1.992000in}{1.390651in}}%
\pgfpathlineto{\pgfqpoint{2.116000in}{1.402552in}}%
\pgfpathlineto{\pgfqpoint{2.240000in}{1.439483in}}%
\pgfpathlineto{\pgfqpoint{2.364000in}{1.468487in}}%
\pgfpathlineto{\pgfqpoint{2.488000in}{1.475153in}}%
\pgfpathlineto{\pgfqpoint{2.612000in}{1.486924in}}%
\pgfpathlineto{\pgfqpoint{2.736000in}{1.521162in}}%
\pgfpathlineto{\pgfqpoint{2.860000in}{1.539778in}}%
\pgfpathlineto{\pgfqpoint{2.984000in}{1.569519in}}%
\pgfpathlineto{\pgfqpoint{3.108000in}{1.597993in}}%
\pgfpathlineto{\pgfqpoint{3.232000in}{1.605972in}}%
\pgfpathlineto{\pgfqpoint{3.356000in}{1.622079in}}%
\pgfpathlineto{\pgfqpoint{3.480000in}{1.656337in}}%
\pgfpathlineto{\pgfqpoint{3.604000in}{1.685881in}}%
\pgfpathlineto{\pgfqpoint{3.728000in}{1.694228in}}%
\pgfpathlineto{\pgfqpoint{3.852000in}{1.714258in}}%
\pgfpathlineto{\pgfqpoint{3.976000in}{1.740254in}}%
\pgfpathlineto{\pgfqpoint{4.100000in}{1.765835in}}%
\pgfpathlineto{\pgfqpoint{4.224000in}{1.776787in}}%
\pgfpathlineto{\pgfqpoint{4.348000in}{1.790506in}}%
\pgfpathlineto{\pgfqpoint{4.472000in}{1.825210in}}%
\pgfpathlineto{\pgfqpoint{4.596000in}{1.848487in}}%
\pgfpathlineto{\pgfqpoint{4.720000in}{1.864952in}}%
\pgfpathlineto{\pgfqpoint{4.844000in}{1.882156in}}%
\pgfpathlineto{\pgfqpoint{4.968000in}{1.914780in}}%
\pgfpathlineto{\pgfqpoint{5.092000in}{1.943737in}}%
\pgfpathlineto{\pgfqpoint{5.216000in}{1.952169in}}%
\pgfpathlineto{\pgfqpoint{5.340000in}{1.971313in}}%
\pgfpathlineto{\pgfqpoint{5.464000in}{1.999706in}}%
\pgfpathlineto{\pgfqpoint{5.588000in}{2.026447in}}%
\pgfpathlineto{\pgfqpoint{5.712000in}{2.035823in}}%
\pgfpathlineto{\pgfqpoint{5.836000in}{2.051094in}}%
\pgfpathlineto{\pgfqpoint{5.960000in}{2.085474in}}%
\pgfpathlineto{\pgfqpoint{6.084000in}{2.111394in}}%
\pgfpathlineto{\pgfqpoint{6.208000in}{2.121947in}}%
\pgfpathlineto{\pgfqpoint{6.332000in}{2.137569in}}%
\pgfpathlineto{\pgfqpoint{6.456000in}{2.171522in}}%
\pgfpathlineto{\pgfqpoint{6.580000in}{2.196793in}}%
\pgfpathlineto{\pgfqpoint{6.704000in}{2.208625in}}%
\pgfpathlineto{\pgfqpoint{6.828000in}{2.226087in}}%
\pgfpathlineto{\pgfqpoint{6.952000in}{2.257612in}}%
\pgfpathlineto{\pgfqpoint{7.076000in}{2.280472in}}%
\pgfpathlineto{\pgfqpoint{7.200000in}{2.298222in}}%
\pgfusepath{stroke}%
\end{pgfscope}%
\begin{pgfscope}%
\pgfsetrectcap%
\pgfsetmiterjoin%
\pgfsetlinewidth{1.003750pt}%
\definecolor{currentstroke}{rgb}{0.000000,0.000000,0.000000}%
\pgfsetstrokecolor{currentstroke}%
\pgfsetdash{}{0pt}%
\pgfpathmoveto{\pgfqpoint{1.000000in}{0.600000in}}%
\pgfpathlineto{\pgfqpoint{1.000000in}{5.400000in}}%
\pgfusepath{stroke}%
\end{pgfscope}%
\begin{pgfscope}%
\pgfsetrectcap%
\pgfsetmiterjoin%
\pgfsetlinewidth{1.003750pt}%
\definecolor{currentstroke}{rgb}{0.000000,0.000000,0.000000}%
\pgfsetstrokecolor{currentstroke}%
\pgfsetdash{}{0pt}%
\pgfpathmoveto{\pgfqpoint{7.200000in}{0.600000in}}%
\pgfpathlineto{\pgfqpoint{7.200000in}{5.400000in}}%
\pgfusepath{stroke}%
\end{pgfscope}%
\begin{pgfscope}%
\pgfsetrectcap%
\pgfsetmiterjoin%
\pgfsetlinewidth{1.003750pt}%
\definecolor{currentstroke}{rgb}{0.000000,0.000000,0.000000}%
\pgfsetstrokecolor{currentstroke}%
\pgfsetdash{}{0pt}%
\pgfpathmoveto{\pgfqpoint{1.000000in}{0.600000in}}%
\pgfpathlineto{\pgfqpoint{7.200000in}{0.600000in}}%
\pgfusepath{stroke}%
\end{pgfscope}%
\begin{pgfscope}%
\pgfsetrectcap%
\pgfsetmiterjoin%
\pgfsetlinewidth{1.003750pt}%
\definecolor{currentstroke}{rgb}{0.000000,0.000000,0.000000}%
\pgfsetstrokecolor{currentstroke}%
\pgfsetdash{}{0pt}%
\pgfpathmoveto{\pgfqpoint{1.000000in}{5.400000in}}%
\pgfpathlineto{\pgfqpoint{7.200000in}{5.400000in}}%
\pgfusepath{stroke}%
\end{pgfscope}%
\begin{pgfscope}%
\pgfpathrectangle{\pgfqpoint{1.000000in}{0.600000in}}{\pgfqpoint{6.200000in}{4.800000in}}%
\pgfusepath{clip}%
\pgfsetbuttcap%
\pgfsetroundjoin%
\pgfsetlinewidth{0.501875pt}%
\definecolor{currentstroke}{rgb}{0.000000,0.000000,0.000000}%
\pgfsetstrokecolor{currentstroke}%
\pgfsetdash{{1.000000pt}{3.000000pt}}{0.000000pt}%
\pgfpathmoveto{\pgfqpoint{1.000000in}{0.600000in}}%
\pgfpathlineto{\pgfqpoint{1.000000in}{5.400000in}}%
\pgfusepath{stroke}%
\end{pgfscope}%
\begin{pgfscope}%
\pgfsetbuttcap%
\pgfsetroundjoin%
\definecolor{currentfill}{rgb}{0.000000,0.000000,0.000000}%
\pgfsetfillcolor{currentfill}%
\pgfsetlinewidth{0.501875pt}%
\definecolor{currentstroke}{rgb}{0.000000,0.000000,0.000000}%
\pgfsetstrokecolor{currentstroke}%
\pgfsetdash{}{0pt}%
\pgfsys@defobject{currentmarker}{\pgfqpoint{0.000000in}{0.000000in}}{\pgfqpoint{0.000000in}{0.055556in}}{%
\pgfpathmoveto{\pgfqpoint{0.000000in}{0.000000in}}%
\pgfpathlineto{\pgfqpoint{0.000000in}{0.055556in}}%
\pgfusepath{stroke,fill}%
}%
\begin{pgfscope}%
\pgfsys@transformshift{1.000000in}{0.600000in}%
\pgfsys@useobject{currentmarker}{}%
\end{pgfscope}%
\end{pgfscope}%
\begin{pgfscope}%
\pgfsetbuttcap%
\pgfsetroundjoin%
\definecolor{currentfill}{rgb}{0.000000,0.000000,0.000000}%
\pgfsetfillcolor{currentfill}%
\pgfsetlinewidth{0.501875pt}%
\definecolor{currentstroke}{rgb}{0.000000,0.000000,0.000000}%
\pgfsetstrokecolor{currentstroke}%
\pgfsetdash{}{0pt}%
\pgfsys@defobject{currentmarker}{\pgfqpoint{0.000000in}{-0.055556in}}{\pgfqpoint{0.000000in}{0.000000in}}{%
\pgfpathmoveto{\pgfqpoint{0.000000in}{0.000000in}}%
\pgfpathlineto{\pgfqpoint{0.000000in}{-0.055556in}}%
\pgfusepath{stroke,fill}%
}%
\begin{pgfscope}%
\pgfsys@transformshift{1.000000in}{5.400000in}%
\pgfsys@useobject{currentmarker}{}%
\end{pgfscope}%
\end{pgfscope}%
\begin{pgfscope}%
\definecolor{textcolor}{rgb}{0.000000,0.000000,0.000000}%
\pgfsetstrokecolor{textcolor}%
\pgfsetfillcolor{textcolor}%
\pgftext[x=1.000000in,y=0.544444in,,top]{\color{textcolor}\rmfamily\fontsize{10.000000}{12.000000}\selectfont \(\displaystyle {0}\)}%
\end{pgfscope}%
\begin{pgfscope}%
\pgfpathrectangle{\pgfqpoint{1.000000in}{0.600000in}}{\pgfqpoint{6.200000in}{4.800000in}}%
\pgfusepath{clip}%
\pgfsetbuttcap%
\pgfsetroundjoin%
\pgfsetlinewidth{0.501875pt}%
\definecolor{currentstroke}{rgb}{0.000000,0.000000,0.000000}%
\pgfsetstrokecolor{currentstroke}%
\pgfsetdash{{1.000000pt}{3.000000pt}}{0.000000pt}%
\pgfpathmoveto{\pgfqpoint{2.240000in}{0.600000in}}%
\pgfpathlineto{\pgfqpoint{2.240000in}{5.400000in}}%
\pgfusepath{stroke}%
\end{pgfscope}%
\begin{pgfscope}%
\pgfsetbuttcap%
\pgfsetroundjoin%
\definecolor{currentfill}{rgb}{0.000000,0.000000,0.000000}%
\pgfsetfillcolor{currentfill}%
\pgfsetlinewidth{0.501875pt}%
\definecolor{currentstroke}{rgb}{0.000000,0.000000,0.000000}%
\pgfsetstrokecolor{currentstroke}%
\pgfsetdash{}{0pt}%
\pgfsys@defobject{currentmarker}{\pgfqpoint{0.000000in}{0.000000in}}{\pgfqpoint{0.000000in}{0.055556in}}{%
\pgfpathmoveto{\pgfqpoint{0.000000in}{0.000000in}}%
\pgfpathlineto{\pgfqpoint{0.000000in}{0.055556in}}%
\pgfusepath{stroke,fill}%
}%
\begin{pgfscope}%
\pgfsys@transformshift{2.240000in}{0.600000in}%
\pgfsys@useobject{currentmarker}{}%
\end{pgfscope}%
\end{pgfscope}%
\begin{pgfscope}%
\pgfsetbuttcap%
\pgfsetroundjoin%
\definecolor{currentfill}{rgb}{0.000000,0.000000,0.000000}%
\pgfsetfillcolor{currentfill}%
\pgfsetlinewidth{0.501875pt}%
\definecolor{currentstroke}{rgb}{0.000000,0.000000,0.000000}%
\pgfsetstrokecolor{currentstroke}%
\pgfsetdash{}{0pt}%
\pgfsys@defobject{currentmarker}{\pgfqpoint{0.000000in}{-0.055556in}}{\pgfqpoint{0.000000in}{0.000000in}}{%
\pgfpathmoveto{\pgfqpoint{0.000000in}{0.000000in}}%
\pgfpathlineto{\pgfqpoint{0.000000in}{-0.055556in}}%
\pgfusepath{stroke,fill}%
}%
\begin{pgfscope}%
\pgfsys@transformshift{2.240000in}{5.400000in}%
\pgfsys@useobject{currentmarker}{}%
\end{pgfscope}%
\end{pgfscope}%
\begin{pgfscope}%
\definecolor{textcolor}{rgb}{0.000000,0.000000,0.000000}%
\pgfsetstrokecolor{textcolor}%
\pgfsetfillcolor{textcolor}%
\pgftext[x=2.240000in,y=0.544444in,,top]{\color{textcolor}\rmfamily\fontsize{10.000000}{12.000000}\selectfont \(\displaystyle {20}\)}%
\end{pgfscope}%
\begin{pgfscope}%
\pgfpathrectangle{\pgfqpoint{1.000000in}{0.600000in}}{\pgfqpoint{6.200000in}{4.800000in}}%
\pgfusepath{clip}%
\pgfsetbuttcap%
\pgfsetroundjoin%
\pgfsetlinewidth{0.501875pt}%
\definecolor{currentstroke}{rgb}{0.000000,0.000000,0.000000}%
\pgfsetstrokecolor{currentstroke}%
\pgfsetdash{{1.000000pt}{3.000000pt}}{0.000000pt}%
\pgfpathmoveto{\pgfqpoint{3.480000in}{0.600000in}}%
\pgfpathlineto{\pgfqpoint{3.480000in}{5.400000in}}%
\pgfusepath{stroke}%
\end{pgfscope}%
\begin{pgfscope}%
\pgfsetbuttcap%
\pgfsetroundjoin%
\definecolor{currentfill}{rgb}{0.000000,0.000000,0.000000}%
\pgfsetfillcolor{currentfill}%
\pgfsetlinewidth{0.501875pt}%
\definecolor{currentstroke}{rgb}{0.000000,0.000000,0.000000}%
\pgfsetstrokecolor{currentstroke}%
\pgfsetdash{}{0pt}%
\pgfsys@defobject{currentmarker}{\pgfqpoint{0.000000in}{0.000000in}}{\pgfqpoint{0.000000in}{0.055556in}}{%
\pgfpathmoveto{\pgfqpoint{0.000000in}{0.000000in}}%
\pgfpathlineto{\pgfqpoint{0.000000in}{0.055556in}}%
\pgfusepath{stroke,fill}%
}%
\begin{pgfscope}%
\pgfsys@transformshift{3.480000in}{0.600000in}%
\pgfsys@useobject{currentmarker}{}%
\end{pgfscope}%
\end{pgfscope}%
\begin{pgfscope}%
\pgfsetbuttcap%
\pgfsetroundjoin%
\definecolor{currentfill}{rgb}{0.000000,0.000000,0.000000}%
\pgfsetfillcolor{currentfill}%
\pgfsetlinewidth{0.501875pt}%
\definecolor{currentstroke}{rgb}{0.000000,0.000000,0.000000}%
\pgfsetstrokecolor{currentstroke}%
\pgfsetdash{}{0pt}%
\pgfsys@defobject{currentmarker}{\pgfqpoint{0.000000in}{-0.055556in}}{\pgfqpoint{0.000000in}{0.000000in}}{%
\pgfpathmoveto{\pgfqpoint{0.000000in}{0.000000in}}%
\pgfpathlineto{\pgfqpoint{0.000000in}{-0.055556in}}%
\pgfusepath{stroke,fill}%
}%
\begin{pgfscope}%
\pgfsys@transformshift{3.480000in}{5.400000in}%
\pgfsys@useobject{currentmarker}{}%
\end{pgfscope}%
\end{pgfscope}%
\begin{pgfscope}%
\definecolor{textcolor}{rgb}{0.000000,0.000000,0.000000}%
\pgfsetstrokecolor{textcolor}%
\pgfsetfillcolor{textcolor}%
\pgftext[x=3.480000in,y=0.544444in,,top]{\color{textcolor}\rmfamily\fontsize{10.000000}{12.000000}\selectfont \(\displaystyle {40}\)}%
\end{pgfscope}%
\begin{pgfscope}%
\pgfpathrectangle{\pgfqpoint{1.000000in}{0.600000in}}{\pgfqpoint{6.200000in}{4.800000in}}%
\pgfusepath{clip}%
\pgfsetbuttcap%
\pgfsetroundjoin%
\pgfsetlinewidth{0.501875pt}%
\definecolor{currentstroke}{rgb}{0.000000,0.000000,0.000000}%
\pgfsetstrokecolor{currentstroke}%
\pgfsetdash{{1.000000pt}{3.000000pt}}{0.000000pt}%
\pgfpathmoveto{\pgfqpoint{4.720000in}{0.600000in}}%
\pgfpathlineto{\pgfqpoint{4.720000in}{5.400000in}}%
\pgfusepath{stroke}%
\end{pgfscope}%
\begin{pgfscope}%
\pgfsetbuttcap%
\pgfsetroundjoin%
\definecolor{currentfill}{rgb}{0.000000,0.000000,0.000000}%
\pgfsetfillcolor{currentfill}%
\pgfsetlinewidth{0.501875pt}%
\definecolor{currentstroke}{rgb}{0.000000,0.000000,0.000000}%
\pgfsetstrokecolor{currentstroke}%
\pgfsetdash{}{0pt}%
\pgfsys@defobject{currentmarker}{\pgfqpoint{0.000000in}{0.000000in}}{\pgfqpoint{0.000000in}{0.055556in}}{%
\pgfpathmoveto{\pgfqpoint{0.000000in}{0.000000in}}%
\pgfpathlineto{\pgfqpoint{0.000000in}{0.055556in}}%
\pgfusepath{stroke,fill}%
}%
\begin{pgfscope}%
\pgfsys@transformshift{4.720000in}{0.600000in}%
\pgfsys@useobject{currentmarker}{}%
\end{pgfscope}%
\end{pgfscope}%
\begin{pgfscope}%
\pgfsetbuttcap%
\pgfsetroundjoin%
\definecolor{currentfill}{rgb}{0.000000,0.000000,0.000000}%
\pgfsetfillcolor{currentfill}%
\pgfsetlinewidth{0.501875pt}%
\definecolor{currentstroke}{rgb}{0.000000,0.000000,0.000000}%
\pgfsetstrokecolor{currentstroke}%
\pgfsetdash{}{0pt}%
\pgfsys@defobject{currentmarker}{\pgfqpoint{0.000000in}{-0.055556in}}{\pgfqpoint{0.000000in}{0.000000in}}{%
\pgfpathmoveto{\pgfqpoint{0.000000in}{0.000000in}}%
\pgfpathlineto{\pgfqpoint{0.000000in}{-0.055556in}}%
\pgfusepath{stroke,fill}%
}%
\begin{pgfscope}%
\pgfsys@transformshift{4.720000in}{5.400000in}%
\pgfsys@useobject{currentmarker}{}%
\end{pgfscope}%
\end{pgfscope}%
\begin{pgfscope}%
\definecolor{textcolor}{rgb}{0.000000,0.000000,0.000000}%
\pgfsetstrokecolor{textcolor}%
\pgfsetfillcolor{textcolor}%
\pgftext[x=4.720000in,y=0.544444in,,top]{\color{textcolor}\rmfamily\fontsize{10.000000}{12.000000}\selectfont \(\displaystyle {60}\)}%
\end{pgfscope}%
\begin{pgfscope}%
\pgfpathrectangle{\pgfqpoint{1.000000in}{0.600000in}}{\pgfqpoint{6.200000in}{4.800000in}}%
\pgfusepath{clip}%
\pgfsetbuttcap%
\pgfsetroundjoin%
\pgfsetlinewidth{0.501875pt}%
\definecolor{currentstroke}{rgb}{0.000000,0.000000,0.000000}%
\pgfsetstrokecolor{currentstroke}%
\pgfsetdash{{1.000000pt}{3.000000pt}}{0.000000pt}%
\pgfpathmoveto{\pgfqpoint{5.960000in}{0.600000in}}%
\pgfpathlineto{\pgfqpoint{5.960000in}{5.400000in}}%
\pgfusepath{stroke}%
\end{pgfscope}%
\begin{pgfscope}%
\pgfsetbuttcap%
\pgfsetroundjoin%
\definecolor{currentfill}{rgb}{0.000000,0.000000,0.000000}%
\pgfsetfillcolor{currentfill}%
\pgfsetlinewidth{0.501875pt}%
\definecolor{currentstroke}{rgb}{0.000000,0.000000,0.000000}%
\pgfsetstrokecolor{currentstroke}%
\pgfsetdash{}{0pt}%
\pgfsys@defobject{currentmarker}{\pgfqpoint{0.000000in}{0.000000in}}{\pgfqpoint{0.000000in}{0.055556in}}{%
\pgfpathmoveto{\pgfqpoint{0.000000in}{0.000000in}}%
\pgfpathlineto{\pgfqpoint{0.000000in}{0.055556in}}%
\pgfusepath{stroke,fill}%
}%
\begin{pgfscope}%
\pgfsys@transformshift{5.960000in}{0.600000in}%
\pgfsys@useobject{currentmarker}{}%
\end{pgfscope}%
\end{pgfscope}%
\begin{pgfscope}%
\pgfsetbuttcap%
\pgfsetroundjoin%
\definecolor{currentfill}{rgb}{0.000000,0.000000,0.000000}%
\pgfsetfillcolor{currentfill}%
\pgfsetlinewidth{0.501875pt}%
\definecolor{currentstroke}{rgb}{0.000000,0.000000,0.000000}%
\pgfsetstrokecolor{currentstroke}%
\pgfsetdash{}{0pt}%
\pgfsys@defobject{currentmarker}{\pgfqpoint{0.000000in}{-0.055556in}}{\pgfqpoint{0.000000in}{0.000000in}}{%
\pgfpathmoveto{\pgfqpoint{0.000000in}{0.000000in}}%
\pgfpathlineto{\pgfqpoint{0.000000in}{-0.055556in}}%
\pgfusepath{stroke,fill}%
}%
\begin{pgfscope}%
\pgfsys@transformshift{5.960000in}{5.400000in}%
\pgfsys@useobject{currentmarker}{}%
\end{pgfscope}%
\end{pgfscope}%
\begin{pgfscope}%
\definecolor{textcolor}{rgb}{0.000000,0.000000,0.000000}%
\pgfsetstrokecolor{textcolor}%
\pgfsetfillcolor{textcolor}%
\pgftext[x=5.960000in,y=0.544444in,,top]{\color{textcolor}\rmfamily\fontsize{10.000000}{12.000000}\selectfont \(\displaystyle {80}\)}%
\end{pgfscope}%
\begin{pgfscope}%
\pgfpathrectangle{\pgfqpoint{1.000000in}{0.600000in}}{\pgfqpoint{6.200000in}{4.800000in}}%
\pgfusepath{clip}%
\pgfsetbuttcap%
\pgfsetroundjoin%
\pgfsetlinewidth{0.501875pt}%
\definecolor{currentstroke}{rgb}{0.000000,0.000000,0.000000}%
\pgfsetstrokecolor{currentstroke}%
\pgfsetdash{{1.000000pt}{3.000000pt}}{0.000000pt}%
\pgfpathmoveto{\pgfqpoint{7.200000in}{0.600000in}}%
\pgfpathlineto{\pgfqpoint{7.200000in}{5.400000in}}%
\pgfusepath{stroke}%
\end{pgfscope}%
\begin{pgfscope}%
\pgfsetbuttcap%
\pgfsetroundjoin%
\definecolor{currentfill}{rgb}{0.000000,0.000000,0.000000}%
\pgfsetfillcolor{currentfill}%
\pgfsetlinewidth{0.501875pt}%
\definecolor{currentstroke}{rgb}{0.000000,0.000000,0.000000}%
\pgfsetstrokecolor{currentstroke}%
\pgfsetdash{}{0pt}%
\pgfsys@defobject{currentmarker}{\pgfqpoint{0.000000in}{0.000000in}}{\pgfqpoint{0.000000in}{0.055556in}}{%
\pgfpathmoveto{\pgfqpoint{0.000000in}{0.000000in}}%
\pgfpathlineto{\pgfqpoint{0.000000in}{0.055556in}}%
\pgfusepath{stroke,fill}%
}%
\begin{pgfscope}%
\pgfsys@transformshift{7.200000in}{0.600000in}%
\pgfsys@useobject{currentmarker}{}%
\end{pgfscope}%
\end{pgfscope}%
\begin{pgfscope}%
\pgfsetbuttcap%
\pgfsetroundjoin%
\definecolor{currentfill}{rgb}{0.000000,0.000000,0.000000}%
\pgfsetfillcolor{currentfill}%
\pgfsetlinewidth{0.501875pt}%
\definecolor{currentstroke}{rgb}{0.000000,0.000000,0.000000}%
\pgfsetstrokecolor{currentstroke}%
\pgfsetdash{}{0pt}%
\pgfsys@defobject{currentmarker}{\pgfqpoint{0.000000in}{-0.055556in}}{\pgfqpoint{0.000000in}{0.000000in}}{%
\pgfpathmoveto{\pgfqpoint{0.000000in}{0.000000in}}%
\pgfpathlineto{\pgfqpoint{0.000000in}{-0.055556in}}%
\pgfusepath{stroke,fill}%
}%
\begin{pgfscope}%
\pgfsys@transformshift{7.200000in}{5.400000in}%
\pgfsys@useobject{currentmarker}{}%
\end{pgfscope}%
\end{pgfscope}%
\begin{pgfscope}%
\definecolor{textcolor}{rgb}{0.000000,0.000000,0.000000}%
\pgfsetstrokecolor{textcolor}%
\pgfsetfillcolor{textcolor}%
\pgftext[x=7.200000in,y=0.544444in,,top]{\color{textcolor}\rmfamily\fontsize{10.000000}{12.000000}\selectfont \(\displaystyle {100}\)}%
\end{pgfscope}%
\begin{pgfscope}%
\definecolor{textcolor}{rgb}{0.000000,0.000000,0.000000}%
\pgfsetstrokecolor{textcolor}%
\pgfsetfillcolor{textcolor}%
\pgftext[x=4.100000in,y=0.351543in,,top]{\color{textcolor}\rmfamily\fontsize{12.000000}{14.400000}\selectfont \(\displaystyle time\ (s)\)}%
\end{pgfscope}%
\begin{pgfscope}%
\pgfpathrectangle{\pgfqpoint{1.000000in}{0.600000in}}{\pgfqpoint{6.200000in}{4.800000in}}%
\pgfusepath{clip}%
\pgfsetbuttcap%
\pgfsetroundjoin%
\pgfsetlinewidth{0.501875pt}%
\definecolor{currentstroke}{rgb}{0.000000,0.000000,0.000000}%
\pgfsetstrokecolor{currentstroke}%
\pgfsetdash{{1.000000pt}{3.000000pt}}{0.000000pt}%
\pgfpathmoveto{\pgfqpoint{1.000000in}{0.600000in}}%
\pgfpathlineto{\pgfqpoint{7.200000in}{0.600000in}}%
\pgfusepath{stroke}%
\end{pgfscope}%
\begin{pgfscope}%
\pgfsetbuttcap%
\pgfsetroundjoin%
\definecolor{currentfill}{rgb}{0.000000,0.000000,0.000000}%
\pgfsetfillcolor{currentfill}%
\pgfsetlinewidth{0.501875pt}%
\definecolor{currentstroke}{rgb}{0.000000,0.000000,0.000000}%
\pgfsetstrokecolor{currentstroke}%
\pgfsetdash{}{0pt}%
\pgfsys@defobject{currentmarker}{\pgfqpoint{0.000000in}{0.000000in}}{\pgfqpoint{0.055556in}{0.000000in}}{%
\pgfpathmoveto{\pgfqpoint{0.000000in}{0.000000in}}%
\pgfpathlineto{\pgfqpoint{0.055556in}{0.000000in}}%
\pgfusepath{stroke,fill}%
}%
\begin{pgfscope}%
\pgfsys@transformshift{1.000000in}{0.600000in}%
\pgfsys@useobject{currentmarker}{}%
\end{pgfscope}%
\end{pgfscope}%
\begin{pgfscope}%
\pgfsetbuttcap%
\pgfsetroundjoin%
\definecolor{currentfill}{rgb}{0.000000,0.000000,0.000000}%
\pgfsetfillcolor{currentfill}%
\pgfsetlinewidth{0.501875pt}%
\definecolor{currentstroke}{rgb}{0.000000,0.000000,0.000000}%
\pgfsetstrokecolor{currentstroke}%
\pgfsetdash{}{0pt}%
\pgfsys@defobject{currentmarker}{\pgfqpoint{-0.055556in}{0.000000in}}{\pgfqpoint{-0.000000in}{0.000000in}}{%
\pgfpathmoveto{\pgfqpoint{-0.000000in}{0.000000in}}%
\pgfpathlineto{\pgfqpoint{-0.055556in}{0.000000in}}%
\pgfusepath{stroke,fill}%
}%
\begin{pgfscope}%
\pgfsys@transformshift{7.200000in}{0.600000in}%
\pgfsys@useobject{currentmarker}{}%
\end{pgfscope}%
\end{pgfscope}%
\begin{pgfscope}%
\definecolor{textcolor}{rgb}{0.000000,0.000000,0.000000}%
\pgfsetstrokecolor{textcolor}%
\pgfsetfillcolor{textcolor}%
\pgftext[x=0.944444in,y=0.600000in,right,]{\color{textcolor}\rmfamily\fontsize{10.000000}{12.000000}\selectfont \(\displaystyle {\ensuremath{-}100}\)}%
\end{pgfscope}%
\begin{pgfscope}%
\pgfpathrectangle{\pgfqpoint{1.000000in}{0.600000in}}{\pgfqpoint{6.200000in}{4.800000in}}%
\pgfusepath{clip}%
\pgfsetbuttcap%
\pgfsetroundjoin%
\pgfsetlinewidth{0.501875pt}%
\definecolor{currentstroke}{rgb}{0.000000,0.000000,0.000000}%
\pgfsetstrokecolor{currentstroke}%
\pgfsetdash{{1.000000pt}{3.000000pt}}{0.000000pt}%
\pgfpathmoveto{\pgfqpoint{1.000000in}{1.285714in}}%
\pgfpathlineto{\pgfqpoint{7.200000in}{1.285714in}}%
\pgfusepath{stroke}%
\end{pgfscope}%
\begin{pgfscope}%
\pgfsetbuttcap%
\pgfsetroundjoin%
\definecolor{currentfill}{rgb}{0.000000,0.000000,0.000000}%
\pgfsetfillcolor{currentfill}%
\pgfsetlinewidth{0.501875pt}%
\definecolor{currentstroke}{rgb}{0.000000,0.000000,0.000000}%
\pgfsetstrokecolor{currentstroke}%
\pgfsetdash{}{0pt}%
\pgfsys@defobject{currentmarker}{\pgfqpoint{0.000000in}{0.000000in}}{\pgfqpoint{0.055556in}{0.000000in}}{%
\pgfpathmoveto{\pgfqpoint{0.000000in}{0.000000in}}%
\pgfpathlineto{\pgfqpoint{0.055556in}{0.000000in}}%
\pgfusepath{stroke,fill}%
}%
\begin{pgfscope}%
\pgfsys@transformshift{1.000000in}{1.285714in}%
\pgfsys@useobject{currentmarker}{}%
\end{pgfscope}%
\end{pgfscope}%
\begin{pgfscope}%
\pgfsetbuttcap%
\pgfsetroundjoin%
\definecolor{currentfill}{rgb}{0.000000,0.000000,0.000000}%
\pgfsetfillcolor{currentfill}%
\pgfsetlinewidth{0.501875pt}%
\definecolor{currentstroke}{rgb}{0.000000,0.000000,0.000000}%
\pgfsetstrokecolor{currentstroke}%
\pgfsetdash{}{0pt}%
\pgfsys@defobject{currentmarker}{\pgfqpoint{-0.055556in}{0.000000in}}{\pgfqpoint{-0.000000in}{0.000000in}}{%
\pgfpathmoveto{\pgfqpoint{-0.000000in}{0.000000in}}%
\pgfpathlineto{\pgfqpoint{-0.055556in}{0.000000in}}%
\pgfusepath{stroke,fill}%
}%
\begin{pgfscope}%
\pgfsys@transformshift{7.200000in}{1.285714in}%
\pgfsys@useobject{currentmarker}{}%
\end{pgfscope}%
\end{pgfscope}%
\begin{pgfscope}%
\definecolor{textcolor}{rgb}{0.000000,0.000000,0.000000}%
\pgfsetstrokecolor{textcolor}%
\pgfsetfillcolor{textcolor}%
\pgftext[x=0.944444in,y=1.285714in,right,]{\color{textcolor}\rmfamily\fontsize{10.000000}{12.000000}\selectfont \(\displaystyle {0}\)}%
\end{pgfscope}%
\begin{pgfscope}%
\pgfpathrectangle{\pgfqpoint{1.000000in}{0.600000in}}{\pgfqpoint{6.200000in}{4.800000in}}%
\pgfusepath{clip}%
\pgfsetbuttcap%
\pgfsetroundjoin%
\pgfsetlinewidth{0.501875pt}%
\definecolor{currentstroke}{rgb}{0.000000,0.000000,0.000000}%
\pgfsetstrokecolor{currentstroke}%
\pgfsetdash{{1.000000pt}{3.000000pt}}{0.000000pt}%
\pgfpathmoveto{\pgfqpoint{1.000000in}{1.971429in}}%
\pgfpathlineto{\pgfqpoint{7.200000in}{1.971429in}}%
\pgfusepath{stroke}%
\end{pgfscope}%
\begin{pgfscope}%
\pgfsetbuttcap%
\pgfsetroundjoin%
\definecolor{currentfill}{rgb}{0.000000,0.000000,0.000000}%
\pgfsetfillcolor{currentfill}%
\pgfsetlinewidth{0.501875pt}%
\definecolor{currentstroke}{rgb}{0.000000,0.000000,0.000000}%
\pgfsetstrokecolor{currentstroke}%
\pgfsetdash{}{0pt}%
\pgfsys@defobject{currentmarker}{\pgfqpoint{0.000000in}{0.000000in}}{\pgfqpoint{0.055556in}{0.000000in}}{%
\pgfpathmoveto{\pgfqpoint{0.000000in}{0.000000in}}%
\pgfpathlineto{\pgfqpoint{0.055556in}{0.000000in}}%
\pgfusepath{stroke,fill}%
}%
\begin{pgfscope}%
\pgfsys@transformshift{1.000000in}{1.971429in}%
\pgfsys@useobject{currentmarker}{}%
\end{pgfscope}%
\end{pgfscope}%
\begin{pgfscope}%
\pgfsetbuttcap%
\pgfsetroundjoin%
\definecolor{currentfill}{rgb}{0.000000,0.000000,0.000000}%
\pgfsetfillcolor{currentfill}%
\pgfsetlinewidth{0.501875pt}%
\definecolor{currentstroke}{rgb}{0.000000,0.000000,0.000000}%
\pgfsetstrokecolor{currentstroke}%
\pgfsetdash{}{0pt}%
\pgfsys@defobject{currentmarker}{\pgfqpoint{-0.055556in}{0.000000in}}{\pgfqpoint{-0.000000in}{0.000000in}}{%
\pgfpathmoveto{\pgfqpoint{-0.000000in}{0.000000in}}%
\pgfpathlineto{\pgfqpoint{-0.055556in}{0.000000in}}%
\pgfusepath{stroke,fill}%
}%
\begin{pgfscope}%
\pgfsys@transformshift{7.200000in}{1.971429in}%
\pgfsys@useobject{currentmarker}{}%
\end{pgfscope}%
\end{pgfscope}%
\begin{pgfscope}%
\definecolor{textcolor}{rgb}{0.000000,0.000000,0.000000}%
\pgfsetstrokecolor{textcolor}%
\pgfsetfillcolor{textcolor}%
\pgftext[x=0.944444in,y=1.971429in,right,]{\color{textcolor}\rmfamily\fontsize{10.000000}{12.000000}\selectfont \(\displaystyle {100}\)}%
\end{pgfscope}%
\begin{pgfscope}%
\pgfpathrectangle{\pgfqpoint{1.000000in}{0.600000in}}{\pgfqpoint{6.200000in}{4.800000in}}%
\pgfusepath{clip}%
\pgfsetbuttcap%
\pgfsetroundjoin%
\pgfsetlinewidth{0.501875pt}%
\definecolor{currentstroke}{rgb}{0.000000,0.000000,0.000000}%
\pgfsetstrokecolor{currentstroke}%
\pgfsetdash{{1.000000pt}{3.000000pt}}{0.000000pt}%
\pgfpathmoveto{\pgfqpoint{1.000000in}{2.657143in}}%
\pgfpathlineto{\pgfqpoint{7.200000in}{2.657143in}}%
\pgfusepath{stroke}%
\end{pgfscope}%
\begin{pgfscope}%
\pgfsetbuttcap%
\pgfsetroundjoin%
\definecolor{currentfill}{rgb}{0.000000,0.000000,0.000000}%
\pgfsetfillcolor{currentfill}%
\pgfsetlinewidth{0.501875pt}%
\definecolor{currentstroke}{rgb}{0.000000,0.000000,0.000000}%
\pgfsetstrokecolor{currentstroke}%
\pgfsetdash{}{0pt}%
\pgfsys@defobject{currentmarker}{\pgfqpoint{0.000000in}{0.000000in}}{\pgfqpoint{0.055556in}{0.000000in}}{%
\pgfpathmoveto{\pgfqpoint{0.000000in}{0.000000in}}%
\pgfpathlineto{\pgfqpoint{0.055556in}{0.000000in}}%
\pgfusepath{stroke,fill}%
}%
\begin{pgfscope}%
\pgfsys@transformshift{1.000000in}{2.657143in}%
\pgfsys@useobject{currentmarker}{}%
\end{pgfscope}%
\end{pgfscope}%
\begin{pgfscope}%
\pgfsetbuttcap%
\pgfsetroundjoin%
\definecolor{currentfill}{rgb}{0.000000,0.000000,0.000000}%
\pgfsetfillcolor{currentfill}%
\pgfsetlinewidth{0.501875pt}%
\definecolor{currentstroke}{rgb}{0.000000,0.000000,0.000000}%
\pgfsetstrokecolor{currentstroke}%
\pgfsetdash{}{0pt}%
\pgfsys@defobject{currentmarker}{\pgfqpoint{-0.055556in}{0.000000in}}{\pgfqpoint{-0.000000in}{0.000000in}}{%
\pgfpathmoveto{\pgfqpoint{-0.000000in}{0.000000in}}%
\pgfpathlineto{\pgfqpoint{-0.055556in}{0.000000in}}%
\pgfusepath{stroke,fill}%
}%
\begin{pgfscope}%
\pgfsys@transformshift{7.200000in}{2.657143in}%
\pgfsys@useobject{currentmarker}{}%
\end{pgfscope}%
\end{pgfscope}%
\begin{pgfscope}%
\definecolor{textcolor}{rgb}{0.000000,0.000000,0.000000}%
\pgfsetstrokecolor{textcolor}%
\pgfsetfillcolor{textcolor}%
\pgftext[x=0.944444in,y=2.657143in,right,]{\color{textcolor}\rmfamily\fontsize{10.000000}{12.000000}\selectfont \(\displaystyle {200}\)}%
\end{pgfscope}%
\begin{pgfscope}%
\pgfpathrectangle{\pgfqpoint{1.000000in}{0.600000in}}{\pgfqpoint{6.200000in}{4.800000in}}%
\pgfusepath{clip}%
\pgfsetbuttcap%
\pgfsetroundjoin%
\pgfsetlinewidth{0.501875pt}%
\definecolor{currentstroke}{rgb}{0.000000,0.000000,0.000000}%
\pgfsetstrokecolor{currentstroke}%
\pgfsetdash{{1.000000pt}{3.000000pt}}{0.000000pt}%
\pgfpathmoveto{\pgfqpoint{1.000000in}{3.342857in}}%
\pgfpathlineto{\pgfqpoint{7.200000in}{3.342857in}}%
\pgfusepath{stroke}%
\end{pgfscope}%
\begin{pgfscope}%
\pgfsetbuttcap%
\pgfsetroundjoin%
\definecolor{currentfill}{rgb}{0.000000,0.000000,0.000000}%
\pgfsetfillcolor{currentfill}%
\pgfsetlinewidth{0.501875pt}%
\definecolor{currentstroke}{rgb}{0.000000,0.000000,0.000000}%
\pgfsetstrokecolor{currentstroke}%
\pgfsetdash{}{0pt}%
\pgfsys@defobject{currentmarker}{\pgfqpoint{0.000000in}{0.000000in}}{\pgfqpoint{0.055556in}{0.000000in}}{%
\pgfpathmoveto{\pgfqpoint{0.000000in}{0.000000in}}%
\pgfpathlineto{\pgfqpoint{0.055556in}{0.000000in}}%
\pgfusepath{stroke,fill}%
}%
\begin{pgfscope}%
\pgfsys@transformshift{1.000000in}{3.342857in}%
\pgfsys@useobject{currentmarker}{}%
\end{pgfscope}%
\end{pgfscope}%
\begin{pgfscope}%
\pgfsetbuttcap%
\pgfsetroundjoin%
\definecolor{currentfill}{rgb}{0.000000,0.000000,0.000000}%
\pgfsetfillcolor{currentfill}%
\pgfsetlinewidth{0.501875pt}%
\definecolor{currentstroke}{rgb}{0.000000,0.000000,0.000000}%
\pgfsetstrokecolor{currentstroke}%
\pgfsetdash{}{0pt}%
\pgfsys@defobject{currentmarker}{\pgfqpoint{-0.055556in}{0.000000in}}{\pgfqpoint{-0.000000in}{0.000000in}}{%
\pgfpathmoveto{\pgfqpoint{-0.000000in}{0.000000in}}%
\pgfpathlineto{\pgfqpoint{-0.055556in}{0.000000in}}%
\pgfusepath{stroke,fill}%
}%
\begin{pgfscope}%
\pgfsys@transformshift{7.200000in}{3.342857in}%
\pgfsys@useobject{currentmarker}{}%
\end{pgfscope}%
\end{pgfscope}%
\begin{pgfscope}%
\definecolor{textcolor}{rgb}{0.000000,0.000000,0.000000}%
\pgfsetstrokecolor{textcolor}%
\pgfsetfillcolor{textcolor}%
\pgftext[x=0.944444in,y=3.342857in,right,]{\color{textcolor}\rmfamily\fontsize{10.000000}{12.000000}\selectfont \(\displaystyle {300}\)}%
\end{pgfscope}%
\begin{pgfscope}%
\pgfpathrectangle{\pgfqpoint{1.000000in}{0.600000in}}{\pgfqpoint{6.200000in}{4.800000in}}%
\pgfusepath{clip}%
\pgfsetbuttcap%
\pgfsetroundjoin%
\pgfsetlinewidth{0.501875pt}%
\definecolor{currentstroke}{rgb}{0.000000,0.000000,0.000000}%
\pgfsetstrokecolor{currentstroke}%
\pgfsetdash{{1.000000pt}{3.000000pt}}{0.000000pt}%
\pgfpathmoveto{\pgfqpoint{1.000000in}{4.028571in}}%
\pgfpathlineto{\pgfqpoint{7.200000in}{4.028571in}}%
\pgfusepath{stroke}%
\end{pgfscope}%
\begin{pgfscope}%
\pgfsetbuttcap%
\pgfsetroundjoin%
\definecolor{currentfill}{rgb}{0.000000,0.000000,0.000000}%
\pgfsetfillcolor{currentfill}%
\pgfsetlinewidth{0.501875pt}%
\definecolor{currentstroke}{rgb}{0.000000,0.000000,0.000000}%
\pgfsetstrokecolor{currentstroke}%
\pgfsetdash{}{0pt}%
\pgfsys@defobject{currentmarker}{\pgfqpoint{0.000000in}{0.000000in}}{\pgfqpoint{0.055556in}{0.000000in}}{%
\pgfpathmoveto{\pgfqpoint{0.000000in}{0.000000in}}%
\pgfpathlineto{\pgfqpoint{0.055556in}{0.000000in}}%
\pgfusepath{stroke,fill}%
}%
\begin{pgfscope}%
\pgfsys@transformshift{1.000000in}{4.028571in}%
\pgfsys@useobject{currentmarker}{}%
\end{pgfscope}%
\end{pgfscope}%
\begin{pgfscope}%
\pgfsetbuttcap%
\pgfsetroundjoin%
\definecolor{currentfill}{rgb}{0.000000,0.000000,0.000000}%
\pgfsetfillcolor{currentfill}%
\pgfsetlinewidth{0.501875pt}%
\definecolor{currentstroke}{rgb}{0.000000,0.000000,0.000000}%
\pgfsetstrokecolor{currentstroke}%
\pgfsetdash{}{0pt}%
\pgfsys@defobject{currentmarker}{\pgfqpoint{-0.055556in}{0.000000in}}{\pgfqpoint{-0.000000in}{0.000000in}}{%
\pgfpathmoveto{\pgfqpoint{-0.000000in}{0.000000in}}%
\pgfpathlineto{\pgfqpoint{-0.055556in}{0.000000in}}%
\pgfusepath{stroke,fill}%
}%
\begin{pgfscope}%
\pgfsys@transformshift{7.200000in}{4.028571in}%
\pgfsys@useobject{currentmarker}{}%
\end{pgfscope}%
\end{pgfscope}%
\begin{pgfscope}%
\definecolor{textcolor}{rgb}{0.000000,0.000000,0.000000}%
\pgfsetstrokecolor{textcolor}%
\pgfsetfillcolor{textcolor}%
\pgftext[x=0.944444in,y=4.028571in,right,]{\color{textcolor}\rmfamily\fontsize{10.000000}{12.000000}\selectfont \(\displaystyle {400}\)}%
\end{pgfscope}%
\begin{pgfscope}%
\pgfpathrectangle{\pgfqpoint{1.000000in}{0.600000in}}{\pgfqpoint{6.200000in}{4.800000in}}%
\pgfusepath{clip}%
\pgfsetbuttcap%
\pgfsetroundjoin%
\pgfsetlinewidth{0.501875pt}%
\definecolor{currentstroke}{rgb}{0.000000,0.000000,0.000000}%
\pgfsetstrokecolor{currentstroke}%
\pgfsetdash{{1.000000pt}{3.000000pt}}{0.000000pt}%
\pgfpathmoveto{\pgfqpoint{1.000000in}{4.714286in}}%
\pgfpathlineto{\pgfqpoint{7.200000in}{4.714286in}}%
\pgfusepath{stroke}%
\end{pgfscope}%
\begin{pgfscope}%
\pgfsetbuttcap%
\pgfsetroundjoin%
\definecolor{currentfill}{rgb}{0.000000,0.000000,0.000000}%
\pgfsetfillcolor{currentfill}%
\pgfsetlinewidth{0.501875pt}%
\definecolor{currentstroke}{rgb}{0.000000,0.000000,0.000000}%
\pgfsetstrokecolor{currentstroke}%
\pgfsetdash{}{0pt}%
\pgfsys@defobject{currentmarker}{\pgfqpoint{0.000000in}{0.000000in}}{\pgfqpoint{0.055556in}{0.000000in}}{%
\pgfpathmoveto{\pgfqpoint{0.000000in}{0.000000in}}%
\pgfpathlineto{\pgfqpoint{0.055556in}{0.000000in}}%
\pgfusepath{stroke,fill}%
}%
\begin{pgfscope}%
\pgfsys@transformshift{1.000000in}{4.714286in}%
\pgfsys@useobject{currentmarker}{}%
\end{pgfscope}%
\end{pgfscope}%
\begin{pgfscope}%
\pgfsetbuttcap%
\pgfsetroundjoin%
\definecolor{currentfill}{rgb}{0.000000,0.000000,0.000000}%
\pgfsetfillcolor{currentfill}%
\pgfsetlinewidth{0.501875pt}%
\definecolor{currentstroke}{rgb}{0.000000,0.000000,0.000000}%
\pgfsetstrokecolor{currentstroke}%
\pgfsetdash{}{0pt}%
\pgfsys@defobject{currentmarker}{\pgfqpoint{-0.055556in}{0.000000in}}{\pgfqpoint{-0.000000in}{0.000000in}}{%
\pgfpathmoveto{\pgfqpoint{-0.000000in}{0.000000in}}%
\pgfpathlineto{\pgfqpoint{-0.055556in}{0.000000in}}%
\pgfusepath{stroke,fill}%
}%
\begin{pgfscope}%
\pgfsys@transformshift{7.200000in}{4.714286in}%
\pgfsys@useobject{currentmarker}{}%
\end{pgfscope}%
\end{pgfscope}%
\begin{pgfscope}%
\definecolor{textcolor}{rgb}{0.000000,0.000000,0.000000}%
\pgfsetstrokecolor{textcolor}%
\pgfsetfillcolor{textcolor}%
\pgftext[x=0.944444in,y=4.714286in,right,]{\color{textcolor}\rmfamily\fontsize{10.000000}{12.000000}\selectfont \(\displaystyle {500}\)}%
\end{pgfscope}%
\begin{pgfscope}%
\pgfpathrectangle{\pgfqpoint{1.000000in}{0.600000in}}{\pgfqpoint{6.200000in}{4.800000in}}%
\pgfusepath{clip}%
\pgfsetbuttcap%
\pgfsetroundjoin%
\pgfsetlinewidth{0.501875pt}%
\definecolor{currentstroke}{rgb}{0.000000,0.000000,0.000000}%
\pgfsetstrokecolor{currentstroke}%
\pgfsetdash{{1.000000pt}{3.000000pt}}{0.000000pt}%
\pgfpathmoveto{\pgfqpoint{1.000000in}{5.400000in}}%
\pgfpathlineto{\pgfqpoint{7.200000in}{5.400000in}}%
\pgfusepath{stroke}%
\end{pgfscope}%
\begin{pgfscope}%
\pgfsetbuttcap%
\pgfsetroundjoin%
\definecolor{currentfill}{rgb}{0.000000,0.000000,0.000000}%
\pgfsetfillcolor{currentfill}%
\pgfsetlinewidth{0.501875pt}%
\definecolor{currentstroke}{rgb}{0.000000,0.000000,0.000000}%
\pgfsetstrokecolor{currentstroke}%
\pgfsetdash{}{0pt}%
\pgfsys@defobject{currentmarker}{\pgfqpoint{0.000000in}{0.000000in}}{\pgfqpoint{0.055556in}{0.000000in}}{%
\pgfpathmoveto{\pgfqpoint{0.000000in}{0.000000in}}%
\pgfpathlineto{\pgfqpoint{0.055556in}{0.000000in}}%
\pgfusepath{stroke,fill}%
}%
\begin{pgfscope}%
\pgfsys@transformshift{1.000000in}{5.400000in}%
\pgfsys@useobject{currentmarker}{}%
\end{pgfscope}%
\end{pgfscope}%
\begin{pgfscope}%
\pgfsetbuttcap%
\pgfsetroundjoin%
\definecolor{currentfill}{rgb}{0.000000,0.000000,0.000000}%
\pgfsetfillcolor{currentfill}%
\pgfsetlinewidth{0.501875pt}%
\definecolor{currentstroke}{rgb}{0.000000,0.000000,0.000000}%
\pgfsetstrokecolor{currentstroke}%
\pgfsetdash{}{0pt}%
\pgfsys@defobject{currentmarker}{\pgfqpoint{-0.055556in}{0.000000in}}{\pgfqpoint{-0.000000in}{0.000000in}}{%
\pgfpathmoveto{\pgfqpoint{-0.000000in}{0.000000in}}%
\pgfpathlineto{\pgfqpoint{-0.055556in}{0.000000in}}%
\pgfusepath{stroke,fill}%
}%
\begin{pgfscope}%
\pgfsys@transformshift{7.200000in}{5.400000in}%
\pgfsys@useobject{currentmarker}{}%
\end{pgfscope}%
\end{pgfscope}%
\begin{pgfscope}%
\definecolor{textcolor}{rgb}{0.000000,0.000000,0.000000}%
\pgfsetstrokecolor{textcolor}%
\pgfsetfillcolor{textcolor}%
\pgftext[x=0.944444in,y=5.400000in,right,]{\color{textcolor}\rmfamily\fontsize{10.000000}{12.000000}\selectfont \(\displaystyle {600}\)}%
\end{pgfscope}%
\begin{pgfscope}%
\definecolor{textcolor}{rgb}{0.000000,0.000000,0.000000}%
\pgfsetstrokecolor{textcolor}%
\pgfsetfillcolor{textcolor}%
\pgftext[x=0.558641in,y=3.000000in,,bottom,rotate=90.000000]{\color{textcolor}\rmfamily\fontsize{12.000000}{14.400000}\selectfont \(\displaystyle \theta\ (rad)\)}%
\end{pgfscope}%
\begin{pgfscope}%
\definecolor{textcolor}{rgb}{0.000000,0.000000,0.000000}%
\pgfsetstrokecolor{textcolor}%
\pgfsetfillcolor{textcolor}%
\pgftext[x=4.100000in,y=5.469444in,,base]{\color{textcolor}\rmfamily\fontsize{12.000000}{14.400000}\selectfont \(\displaystyle Simple\ pendulum\ using\ Euler's\ methods\ (time\ step = 2\ (s))\)}%
\end{pgfscope}%
\begin{pgfscope}%
\pgfsetbuttcap%
\pgfsetmiterjoin%
\definecolor{currentfill}{rgb}{1.000000,1.000000,1.000000}%
\pgfsetfillcolor{currentfill}%
\pgfsetlinewidth{1.003750pt}%
\definecolor{currentstroke}{rgb}{0.000000,0.000000,0.000000}%
\pgfsetstrokecolor{currentstroke}%
\pgfsetdash{}{0pt}%
\pgfpathmoveto{\pgfqpoint{1.083333in}{4.569445in}}%
\pgfpathlineto{\pgfqpoint{3.093110in}{4.569445in}}%
\pgfpathlineto{\pgfqpoint{3.093110in}{5.316667in}}%
\pgfpathlineto{\pgfqpoint{1.083333in}{5.316667in}}%
\pgfpathlineto{\pgfqpoint{1.083333in}{4.569445in}}%
\pgfpathclose%
\pgfusepath{stroke,fill}%
\end{pgfscope}%
\begin{pgfscope}%
\pgfsetrectcap%
\pgfsetroundjoin%
\pgfsetlinewidth{1.003750pt}%
\definecolor{currentstroke}{rgb}{1.000000,0.000000,0.000000}%
\pgfsetstrokecolor{currentstroke}%
\pgfsetdash{}{0pt}%
\pgfpathmoveto{\pgfqpoint{1.200000in}{5.191667in}}%
\pgfpathlineto{\pgfqpoint{1.433333in}{5.191667in}}%
\pgfusepath{stroke}%
\end{pgfscope}%
\begin{pgfscope}%
\definecolor{textcolor}{rgb}{0.000000,0.000000,0.000000}%
\pgfsetstrokecolor{textcolor}%
\pgfsetfillcolor{textcolor}%
\pgftext[x=1.616667in,y=5.133333in,left,base]{\color{textcolor}\rmfamily\fontsize{12.000000}{14.400000}\selectfont \(\displaystyle euler\ explicit\)}%
\end{pgfscope}%
\begin{pgfscope}%
\pgfsetrectcap%
\pgfsetroundjoin%
\pgfsetlinewidth{1.003750pt}%
\definecolor{currentstroke}{rgb}{0.000000,0.000000,1.000000}%
\pgfsetstrokecolor{currentstroke}%
\pgfsetdash{}{0pt}%
\pgfpathmoveto{\pgfqpoint{1.200000in}{4.959260in}}%
\pgfpathlineto{\pgfqpoint{1.433333in}{4.959260in}}%
\pgfusepath{stroke}%
\end{pgfscope}%
\begin{pgfscope}%
\definecolor{textcolor}{rgb}{0.000000,0.000000,0.000000}%
\pgfsetstrokecolor{textcolor}%
\pgfsetfillcolor{textcolor}%
\pgftext[x=1.616667in,y=4.900926in,left,base]{\color{textcolor}\rmfamily\fontsize{12.000000}{14.400000}\selectfont \(\displaystyle euler\ implicit\)}%
\end{pgfscope}%
\begin{pgfscope}%
\pgfsetrectcap%
\pgfsetroundjoin%
\pgfsetlinewidth{1.003750pt}%
\definecolor{currentstroke}{rgb}{0.000000,0.000000,0.000000}%
\pgfsetstrokecolor{currentstroke}%
\pgfsetdash{}{0pt}%
\pgfpathmoveto{\pgfqpoint{1.200000in}{4.726852in}}%
\pgfpathlineto{\pgfqpoint{1.433333in}{4.726852in}}%
\pgfusepath{stroke}%
\end{pgfscope}%
\begin{pgfscope}%
\definecolor{textcolor}{rgb}{0.000000,0.000000,0.000000}%
\pgfsetstrokecolor{textcolor}%
\pgfsetfillcolor{textcolor}%
\pgftext[x=1.616667in,y=4.668519in,left,base]{\color{textcolor}\rmfamily\fontsize{12.000000}{14.400000}\selectfont \(\displaystyle trapezoidal\ scheme\)}%
\end{pgfscope}%
\end{pgfpicture}%
\makeatother%
\endgroup%
}
    \end{figure}

    \begin{figure}[ht!]
    \centering
    \resizebox{0.9\linewidth}{!}{%% Creator: Matplotlib, PGF backend
%%
%% To include the figure in your LaTeX document, write
%%   \input{<filename>.pgf}
%%
%% Make sure the required packages are loaded in your preamble
%%   \usepackage{pgf}
%%
%% Also ensure that all the required font packages are loaded; for instance,
%% the lmodern package is sometimes necessary when using math font.
%%   \usepackage{lmodern}
%%
%% Figures using additional raster images can only be included by \input if
%% they are in the same directory as the main LaTeX file. For loading figures
%% from other directories you can use the `import` package
%%   \usepackage{import}
%%
%% and then include the figures with
%%   \import{<path to file>}{<filename>.pgf}
%%
%% Matplotlib used the following preamble
%%
\begingroup%
\makeatletter%
\begin{pgfpicture}%
\pgfpathrectangle{\pgfpointorigin}{\pgfqpoint{8.000000in}{6.000000in}}%
\pgfusepath{use as bounding box, clip}%
\begin{pgfscope}%
\pgfsetbuttcap%
\pgfsetmiterjoin%
\definecolor{currentfill}{rgb}{1.000000,1.000000,1.000000}%
\pgfsetfillcolor{currentfill}%
\pgfsetlinewidth{0.000000pt}%
\definecolor{currentstroke}{rgb}{1.000000,1.000000,1.000000}%
\pgfsetstrokecolor{currentstroke}%
\pgfsetdash{}{0pt}%
\pgfpathmoveto{\pgfqpoint{0.000000in}{0.000000in}}%
\pgfpathlineto{\pgfqpoint{8.000000in}{0.000000in}}%
\pgfpathlineto{\pgfqpoint{8.000000in}{6.000000in}}%
\pgfpathlineto{\pgfqpoint{0.000000in}{6.000000in}}%
\pgfpathlineto{\pgfqpoint{0.000000in}{0.000000in}}%
\pgfpathclose%
\pgfusepath{fill}%
\end{pgfscope}%
\begin{pgfscope}%
\pgfsetbuttcap%
\pgfsetmiterjoin%
\definecolor{currentfill}{rgb}{1.000000,1.000000,1.000000}%
\pgfsetfillcolor{currentfill}%
\pgfsetlinewidth{0.000000pt}%
\definecolor{currentstroke}{rgb}{0.000000,0.000000,0.000000}%
\pgfsetstrokecolor{currentstroke}%
\pgfsetstrokeopacity{0.000000}%
\pgfsetdash{}{0pt}%
\pgfpathmoveto{\pgfqpoint{1.000000in}{0.600000in}}%
\pgfpathlineto{\pgfqpoint{7.200000in}{0.600000in}}%
\pgfpathlineto{\pgfqpoint{7.200000in}{5.400000in}}%
\pgfpathlineto{\pgfqpoint{1.000000in}{5.400000in}}%
\pgfpathlineto{\pgfqpoint{1.000000in}{0.600000in}}%
\pgfpathclose%
\pgfusepath{fill}%
\end{pgfscope}%
\begin{pgfscope}%
\pgfpathrectangle{\pgfqpoint{1.000000in}{0.600000in}}{\pgfqpoint{6.200000in}{4.800000in}}%
\pgfusepath{clip}%
\pgfsetrectcap%
\pgfsetroundjoin%
\pgfsetlinewidth{1.003750pt}%
\definecolor{currentstroke}{rgb}{1.000000,0.000000,0.000000}%
\pgfsetstrokecolor{currentstroke}%
\pgfsetdash{}{0pt}%
\pgfpathmoveto{\pgfqpoint{1.000000in}{4.602094in}}%
\pgfpathlineto{\pgfqpoint{1.310000in}{4.602094in}}%
\pgfpathlineto{\pgfqpoint{1.620000in}{4.552094in}}%
\pgfpathlineto{\pgfqpoint{1.930000in}{4.452094in}}%
\pgfpathlineto{\pgfqpoint{2.240000in}{4.296461in}}%
\pgfpathlineto{\pgfqpoint{2.550000in}{4.074685in}}%
\pgfpathlineto{\pgfqpoint{2.860000in}{3.899669in}}%
\pgfpathlineto{\pgfqpoint{3.170000in}{3.666695in}}%
\pgfpathlineto{\pgfqpoint{3.480000in}{3.358827in}}%
\pgfpathlineto{\pgfqpoint{3.790000in}{3.125974in}}%
\pgfpathlineto{\pgfqpoint{4.100000in}{2.959393in}}%
\pgfpathlineto{\pgfqpoint{4.410000in}{2.712050in}}%
\pgfpathlineto{\pgfqpoint{4.720000in}{2.563198in}}%
\pgfpathlineto{\pgfqpoint{5.030000in}{2.482412in}}%
\pgfpathlineto{\pgfqpoint{5.340000in}{2.427574in}}%
\pgfpathlineto{\pgfqpoint{5.650000in}{2.472661in}}%
\pgfpathlineto{\pgfqpoint{5.960000in}{2.555671in}}%
\pgfpathlineto{\pgfqpoint{6.270000in}{2.559999in}}%
\pgfpathlineto{\pgfqpoint{6.580000in}{2.648326in}}%
\pgfpathlineto{\pgfqpoint{6.890000in}{2.824001in}}%
\pgfpathlineto{\pgfqpoint{7.200000in}{2.917094in}}%
\pgfusepath{stroke}%
\end{pgfscope}%
\begin{pgfscope}%
\pgfpathrectangle{\pgfqpoint{1.000000in}{0.600000in}}{\pgfqpoint{6.200000in}{4.800000in}}%
\pgfusepath{clip}%
\pgfsetrectcap%
\pgfsetroundjoin%
\pgfsetlinewidth{1.003750pt}%
\definecolor{currentstroke}{rgb}{0.000000,0.000000,1.000000}%
\pgfsetstrokecolor{currentstroke}%
\pgfsetdash{}{0pt}%
\pgfpathmoveto{\pgfqpoint{1.000000in}{4.602094in}}%
\pgfpathlineto{\pgfqpoint{1.310000in}{4.552094in}}%
\pgfpathlineto{\pgfqpoint{1.620000in}{4.446461in}}%
\pgfpathlineto{\pgfqpoint{1.930000in}{4.398759in}}%
\pgfpathlineto{\pgfqpoint{2.240000in}{4.275612in}}%
\pgfpathlineto{\pgfqpoint{2.550000in}{4.143982in}}%
\pgfpathlineto{\pgfqpoint{2.860000in}{4.115053in}}%
\pgfpathlineto{\pgfqpoint{3.170000in}{4.145990in}}%
\pgfpathlineto{\pgfqpoint{3.480000in}{4.053105in}}%
\pgfpathlineto{\pgfqpoint{3.790000in}{3.908908in}}%
\pgfpathlineto{\pgfqpoint{4.100000in}{3.849028in}}%
\pgfpathlineto{\pgfqpoint{4.410000in}{3.681578in}}%
\pgfpathlineto{\pgfqpoint{4.720000in}{3.466214in}}%
\pgfpathlineto{\pgfqpoint{5.030000in}{3.375302in}}%
\pgfpathlineto{\pgfqpoint{5.340000in}{3.215334in}}%
\pgfpathlineto{\pgfqpoint{5.650000in}{3.290894in}}%
\pgfpathlineto{\pgfqpoint{5.960000in}{3.456958in}}%
\pgfpathlineto{\pgfqpoint{6.270000in}{3.544425in}}%
\pgfpathlineto{\pgfqpoint{6.580000in}{3.623161in}}%
\pgfpathlineto{\pgfqpoint{6.890000in}{3.553761in}}%
\pgfpathlineto{\pgfqpoint{7.200000in}{3.395586in}}%
\pgfusepath{stroke}%
\end{pgfscope}%
\begin{pgfscope}%
\pgfpathrectangle{\pgfqpoint{1.000000in}{0.600000in}}{\pgfqpoint{6.200000in}{4.800000in}}%
\pgfusepath{clip}%
\pgfsetrectcap%
\pgfsetroundjoin%
\pgfsetlinewidth{1.003750pt}%
\definecolor{currentstroke}{rgb}{0.000000,0.000000,0.000000}%
\pgfsetstrokecolor{currentstroke}%
\pgfsetdash{}{0pt}%
\pgfpathmoveto{\pgfqpoint{1.000000in}{4.602094in}}%
\pgfpathlineto{\pgfqpoint{1.310000in}{4.577094in}}%
\pgfpathlineto{\pgfqpoint{1.620000in}{4.500671in}}%
\pgfpathlineto{\pgfqpoint{1.930000in}{4.380264in}}%
\pgfpathlineto{\pgfqpoint{2.240000in}{4.251074in}}%
\pgfpathlineto{\pgfqpoint{2.550000in}{4.057980in}}%
\pgfpathlineto{\pgfqpoint{2.860000in}{3.894439in}}%
\pgfpathlineto{\pgfqpoint{3.170000in}{3.802803in}}%
\pgfpathlineto{\pgfqpoint{3.480000in}{3.631551in}}%
\pgfpathlineto{\pgfqpoint{3.790000in}{3.425690in}}%
\pgfpathlineto{\pgfqpoint{4.100000in}{3.216941in}}%
\pgfpathlineto{\pgfqpoint{4.410000in}{3.037919in}}%
\pgfpathlineto{\pgfqpoint{4.720000in}{2.891231in}}%
\pgfpathlineto{\pgfqpoint{5.030000in}{2.706507in}}%
\pgfpathlineto{\pgfqpoint{5.340000in}{2.612616in}}%
\pgfpathlineto{\pgfqpoint{5.650000in}{2.491613in}}%
\pgfpathlineto{\pgfqpoint{5.960000in}{2.291063in}}%
\pgfpathlineto{\pgfqpoint{6.270000in}{2.024098in}}%
\pgfpathlineto{\pgfqpoint{6.580000in}{1.771775in}}%
\pgfpathlineto{\pgfqpoint{6.890000in}{1.472500in}}%
\pgfpathlineto{\pgfqpoint{7.200000in}{1.206698in}}%
\pgfusepath{stroke}%
\end{pgfscope}%
\begin{pgfscope}%
\pgfsetrectcap%
\pgfsetmiterjoin%
\pgfsetlinewidth{1.003750pt}%
\definecolor{currentstroke}{rgb}{0.000000,0.000000,0.000000}%
\pgfsetstrokecolor{currentstroke}%
\pgfsetdash{}{0pt}%
\pgfpathmoveto{\pgfqpoint{1.000000in}{0.600000in}}%
\pgfpathlineto{\pgfqpoint{1.000000in}{5.400000in}}%
\pgfusepath{stroke}%
\end{pgfscope}%
\begin{pgfscope}%
\pgfsetrectcap%
\pgfsetmiterjoin%
\pgfsetlinewidth{1.003750pt}%
\definecolor{currentstroke}{rgb}{0.000000,0.000000,0.000000}%
\pgfsetstrokecolor{currentstroke}%
\pgfsetdash{}{0pt}%
\pgfpathmoveto{\pgfqpoint{7.200000in}{0.600000in}}%
\pgfpathlineto{\pgfqpoint{7.200000in}{5.400000in}}%
\pgfusepath{stroke}%
\end{pgfscope}%
\begin{pgfscope}%
\pgfsetrectcap%
\pgfsetmiterjoin%
\pgfsetlinewidth{1.003750pt}%
\definecolor{currentstroke}{rgb}{0.000000,0.000000,0.000000}%
\pgfsetstrokecolor{currentstroke}%
\pgfsetdash{}{0pt}%
\pgfpathmoveto{\pgfqpoint{1.000000in}{0.600000in}}%
\pgfpathlineto{\pgfqpoint{7.200000in}{0.600000in}}%
\pgfusepath{stroke}%
\end{pgfscope}%
\begin{pgfscope}%
\pgfsetrectcap%
\pgfsetmiterjoin%
\pgfsetlinewidth{1.003750pt}%
\definecolor{currentstroke}{rgb}{0.000000,0.000000,0.000000}%
\pgfsetstrokecolor{currentstroke}%
\pgfsetdash{}{0pt}%
\pgfpathmoveto{\pgfqpoint{1.000000in}{5.400000in}}%
\pgfpathlineto{\pgfqpoint{7.200000in}{5.400000in}}%
\pgfusepath{stroke}%
\end{pgfscope}%
\begin{pgfscope}%
\pgfpathrectangle{\pgfqpoint{1.000000in}{0.600000in}}{\pgfqpoint{6.200000in}{4.800000in}}%
\pgfusepath{clip}%
\pgfsetbuttcap%
\pgfsetroundjoin%
\pgfsetlinewidth{0.501875pt}%
\definecolor{currentstroke}{rgb}{0.000000,0.000000,0.000000}%
\pgfsetstrokecolor{currentstroke}%
\pgfsetdash{{1.000000pt}{3.000000pt}}{0.000000pt}%
\pgfpathmoveto{\pgfqpoint{1.000000in}{0.600000in}}%
\pgfpathlineto{\pgfqpoint{1.000000in}{5.400000in}}%
\pgfusepath{stroke}%
\end{pgfscope}%
\begin{pgfscope}%
\pgfsetbuttcap%
\pgfsetroundjoin%
\definecolor{currentfill}{rgb}{0.000000,0.000000,0.000000}%
\pgfsetfillcolor{currentfill}%
\pgfsetlinewidth{0.501875pt}%
\definecolor{currentstroke}{rgb}{0.000000,0.000000,0.000000}%
\pgfsetstrokecolor{currentstroke}%
\pgfsetdash{}{0pt}%
\pgfsys@defobject{currentmarker}{\pgfqpoint{0.000000in}{0.000000in}}{\pgfqpoint{0.000000in}{0.055556in}}{%
\pgfpathmoveto{\pgfqpoint{0.000000in}{0.000000in}}%
\pgfpathlineto{\pgfqpoint{0.000000in}{0.055556in}}%
\pgfusepath{stroke,fill}%
}%
\begin{pgfscope}%
\pgfsys@transformshift{1.000000in}{0.600000in}%
\pgfsys@useobject{currentmarker}{}%
\end{pgfscope}%
\end{pgfscope}%
\begin{pgfscope}%
\pgfsetbuttcap%
\pgfsetroundjoin%
\definecolor{currentfill}{rgb}{0.000000,0.000000,0.000000}%
\pgfsetfillcolor{currentfill}%
\pgfsetlinewidth{0.501875pt}%
\definecolor{currentstroke}{rgb}{0.000000,0.000000,0.000000}%
\pgfsetstrokecolor{currentstroke}%
\pgfsetdash{}{0pt}%
\pgfsys@defobject{currentmarker}{\pgfqpoint{0.000000in}{-0.055556in}}{\pgfqpoint{0.000000in}{0.000000in}}{%
\pgfpathmoveto{\pgfqpoint{0.000000in}{0.000000in}}%
\pgfpathlineto{\pgfqpoint{0.000000in}{-0.055556in}}%
\pgfusepath{stroke,fill}%
}%
\begin{pgfscope}%
\pgfsys@transformshift{1.000000in}{5.400000in}%
\pgfsys@useobject{currentmarker}{}%
\end{pgfscope}%
\end{pgfscope}%
\begin{pgfscope}%
\definecolor{textcolor}{rgb}{0.000000,0.000000,0.000000}%
\pgfsetstrokecolor{textcolor}%
\pgfsetfillcolor{textcolor}%
\pgftext[x=1.000000in,y=0.544444in,,top]{\color{textcolor}\rmfamily\fontsize{10.000000}{12.000000}\selectfont \(\displaystyle {0}\)}%
\end{pgfscope}%
\begin{pgfscope}%
\pgfpathrectangle{\pgfqpoint{1.000000in}{0.600000in}}{\pgfqpoint{6.200000in}{4.800000in}}%
\pgfusepath{clip}%
\pgfsetbuttcap%
\pgfsetroundjoin%
\pgfsetlinewidth{0.501875pt}%
\definecolor{currentstroke}{rgb}{0.000000,0.000000,0.000000}%
\pgfsetstrokecolor{currentstroke}%
\pgfsetdash{{1.000000pt}{3.000000pt}}{0.000000pt}%
\pgfpathmoveto{\pgfqpoint{2.240000in}{0.600000in}}%
\pgfpathlineto{\pgfqpoint{2.240000in}{5.400000in}}%
\pgfusepath{stroke}%
\end{pgfscope}%
\begin{pgfscope}%
\pgfsetbuttcap%
\pgfsetroundjoin%
\definecolor{currentfill}{rgb}{0.000000,0.000000,0.000000}%
\pgfsetfillcolor{currentfill}%
\pgfsetlinewidth{0.501875pt}%
\definecolor{currentstroke}{rgb}{0.000000,0.000000,0.000000}%
\pgfsetstrokecolor{currentstroke}%
\pgfsetdash{}{0pt}%
\pgfsys@defobject{currentmarker}{\pgfqpoint{0.000000in}{0.000000in}}{\pgfqpoint{0.000000in}{0.055556in}}{%
\pgfpathmoveto{\pgfqpoint{0.000000in}{0.000000in}}%
\pgfpathlineto{\pgfqpoint{0.000000in}{0.055556in}}%
\pgfusepath{stroke,fill}%
}%
\begin{pgfscope}%
\pgfsys@transformshift{2.240000in}{0.600000in}%
\pgfsys@useobject{currentmarker}{}%
\end{pgfscope}%
\end{pgfscope}%
\begin{pgfscope}%
\pgfsetbuttcap%
\pgfsetroundjoin%
\definecolor{currentfill}{rgb}{0.000000,0.000000,0.000000}%
\pgfsetfillcolor{currentfill}%
\pgfsetlinewidth{0.501875pt}%
\definecolor{currentstroke}{rgb}{0.000000,0.000000,0.000000}%
\pgfsetstrokecolor{currentstroke}%
\pgfsetdash{}{0pt}%
\pgfsys@defobject{currentmarker}{\pgfqpoint{0.000000in}{-0.055556in}}{\pgfqpoint{0.000000in}{0.000000in}}{%
\pgfpathmoveto{\pgfqpoint{0.000000in}{0.000000in}}%
\pgfpathlineto{\pgfqpoint{0.000000in}{-0.055556in}}%
\pgfusepath{stroke,fill}%
}%
\begin{pgfscope}%
\pgfsys@transformshift{2.240000in}{5.400000in}%
\pgfsys@useobject{currentmarker}{}%
\end{pgfscope}%
\end{pgfscope}%
\begin{pgfscope}%
\definecolor{textcolor}{rgb}{0.000000,0.000000,0.000000}%
\pgfsetstrokecolor{textcolor}%
\pgfsetfillcolor{textcolor}%
\pgftext[x=2.240000in,y=0.544444in,,top]{\color{textcolor}\rmfamily\fontsize{10.000000}{12.000000}\selectfont \(\displaystyle {20}\)}%
\end{pgfscope}%
\begin{pgfscope}%
\pgfpathrectangle{\pgfqpoint{1.000000in}{0.600000in}}{\pgfqpoint{6.200000in}{4.800000in}}%
\pgfusepath{clip}%
\pgfsetbuttcap%
\pgfsetroundjoin%
\pgfsetlinewidth{0.501875pt}%
\definecolor{currentstroke}{rgb}{0.000000,0.000000,0.000000}%
\pgfsetstrokecolor{currentstroke}%
\pgfsetdash{{1.000000pt}{3.000000pt}}{0.000000pt}%
\pgfpathmoveto{\pgfqpoint{3.480000in}{0.600000in}}%
\pgfpathlineto{\pgfqpoint{3.480000in}{5.400000in}}%
\pgfusepath{stroke}%
\end{pgfscope}%
\begin{pgfscope}%
\pgfsetbuttcap%
\pgfsetroundjoin%
\definecolor{currentfill}{rgb}{0.000000,0.000000,0.000000}%
\pgfsetfillcolor{currentfill}%
\pgfsetlinewidth{0.501875pt}%
\definecolor{currentstroke}{rgb}{0.000000,0.000000,0.000000}%
\pgfsetstrokecolor{currentstroke}%
\pgfsetdash{}{0pt}%
\pgfsys@defobject{currentmarker}{\pgfqpoint{0.000000in}{0.000000in}}{\pgfqpoint{0.000000in}{0.055556in}}{%
\pgfpathmoveto{\pgfqpoint{0.000000in}{0.000000in}}%
\pgfpathlineto{\pgfqpoint{0.000000in}{0.055556in}}%
\pgfusepath{stroke,fill}%
}%
\begin{pgfscope}%
\pgfsys@transformshift{3.480000in}{0.600000in}%
\pgfsys@useobject{currentmarker}{}%
\end{pgfscope}%
\end{pgfscope}%
\begin{pgfscope}%
\pgfsetbuttcap%
\pgfsetroundjoin%
\definecolor{currentfill}{rgb}{0.000000,0.000000,0.000000}%
\pgfsetfillcolor{currentfill}%
\pgfsetlinewidth{0.501875pt}%
\definecolor{currentstroke}{rgb}{0.000000,0.000000,0.000000}%
\pgfsetstrokecolor{currentstroke}%
\pgfsetdash{}{0pt}%
\pgfsys@defobject{currentmarker}{\pgfqpoint{0.000000in}{-0.055556in}}{\pgfqpoint{0.000000in}{0.000000in}}{%
\pgfpathmoveto{\pgfqpoint{0.000000in}{0.000000in}}%
\pgfpathlineto{\pgfqpoint{0.000000in}{-0.055556in}}%
\pgfusepath{stroke,fill}%
}%
\begin{pgfscope}%
\pgfsys@transformshift{3.480000in}{5.400000in}%
\pgfsys@useobject{currentmarker}{}%
\end{pgfscope}%
\end{pgfscope}%
\begin{pgfscope}%
\definecolor{textcolor}{rgb}{0.000000,0.000000,0.000000}%
\pgfsetstrokecolor{textcolor}%
\pgfsetfillcolor{textcolor}%
\pgftext[x=3.480000in,y=0.544444in,,top]{\color{textcolor}\rmfamily\fontsize{10.000000}{12.000000}\selectfont \(\displaystyle {40}\)}%
\end{pgfscope}%
\begin{pgfscope}%
\pgfpathrectangle{\pgfqpoint{1.000000in}{0.600000in}}{\pgfqpoint{6.200000in}{4.800000in}}%
\pgfusepath{clip}%
\pgfsetbuttcap%
\pgfsetroundjoin%
\pgfsetlinewidth{0.501875pt}%
\definecolor{currentstroke}{rgb}{0.000000,0.000000,0.000000}%
\pgfsetstrokecolor{currentstroke}%
\pgfsetdash{{1.000000pt}{3.000000pt}}{0.000000pt}%
\pgfpathmoveto{\pgfqpoint{4.720000in}{0.600000in}}%
\pgfpathlineto{\pgfqpoint{4.720000in}{5.400000in}}%
\pgfusepath{stroke}%
\end{pgfscope}%
\begin{pgfscope}%
\pgfsetbuttcap%
\pgfsetroundjoin%
\definecolor{currentfill}{rgb}{0.000000,0.000000,0.000000}%
\pgfsetfillcolor{currentfill}%
\pgfsetlinewidth{0.501875pt}%
\definecolor{currentstroke}{rgb}{0.000000,0.000000,0.000000}%
\pgfsetstrokecolor{currentstroke}%
\pgfsetdash{}{0pt}%
\pgfsys@defobject{currentmarker}{\pgfqpoint{0.000000in}{0.000000in}}{\pgfqpoint{0.000000in}{0.055556in}}{%
\pgfpathmoveto{\pgfqpoint{0.000000in}{0.000000in}}%
\pgfpathlineto{\pgfqpoint{0.000000in}{0.055556in}}%
\pgfusepath{stroke,fill}%
}%
\begin{pgfscope}%
\pgfsys@transformshift{4.720000in}{0.600000in}%
\pgfsys@useobject{currentmarker}{}%
\end{pgfscope}%
\end{pgfscope}%
\begin{pgfscope}%
\pgfsetbuttcap%
\pgfsetroundjoin%
\definecolor{currentfill}{rgb}{0.000000,0.000000,0.000000}%
\pgfsetfillcolor{currentfill}%
\pgfsetlinewidth{0.501875pt}%
\definecolor{currentstroke}{rgb}{0.000000,0.000000,0.000000}%
\pgfsetstrokecolor{currentstroke}%
\pgfsetdash{}{0pt}%
\pgfsys@defobject{currentmarker}{\pgfqpoint{0.000000in}{-0.055556in}}{\pgfqpoint{0.000000in}{0.000000in}}{%
\pgfpathmoveto{\pgfqpoint{0.000000in}{0.000000in}}%
\pgfpathlineto{\pgfqpoint{0.000000in}{-0.055556in}}%
\pgfusepath{stroke,fill}%
}%
\begin{pgfscope}%
\pgfsys@transformshift{4.720000in}{5.400000in}%
\pgfsys@useobject{currentmarker}{}%
\end{pgfscope}%
\end{pgfscope}%
\begin{pgfscope}%
\definecolor{textcolor}{rgb}{0.000000,0.000000,0.000000}%
\pgfsetstrokecolor{textcolor}%
\pgfsetfillcolor{textcolor}%
\pgftext[x=4.720000in,y=0.544444in,,top]{\color{textcolor}\rmfamily\fontsize{10.000000}{12.000000}\selectfont \(\displaystyle {60}\)}%
\end{pgfscope}%
\begin{pgfscope}%
\pgfpathrectangle{\pgfqpoint{1.000000in}{0.600000in}}{\pgfqpoint{6.200000in}{4.800000in}}%
\pgfusepath{clip}%
\pgfsetbuttcap%
\pgfsetroundjoin%
\pgfsetlinewidth{0.501875pt}%
\definecolor{currentstroke}{rgb}{0.000000,0.000000,0.000000}%
\pgfsetstrokecolor{currentstroke}%
\pgfsetdash{{1.000000pt}{3.000000pt}}{0.000000pt}%
\pgfpathmoveto{\pgfqpoint{5.960000in}{0.600000in}}%
\pgfpathlineto{\pgfqpoint{5.960000in}{5.400000in}}%
\pgfusepath{stroke}%
\end{pgfscope}%
\begin{pgfscope}%
\pgfsetbuttcap%
\pgfsetroundjoin%
\definecolor{currentfill}{rgb}{0.000000,0.000000,0.000000}%
\pgfsetfillcolor{currentfill}%
\pgfsetlinewidth{0.501875pt}%
\definecolor{currentstroke}{rgb}{0.000000,0.000000,0.000000}%
\pgfsetstrokecolor{currentstroke}%
\pgfsetdash{}{0pt}%
\pgfsys@defobject{currentmarker}{\pgfqpoint{0.000000in}{0.000000in}}{\pgfqpoint{0.000000in}{0.055556in}}{%
\pgfpathmoveto{\pgfqpoint{0.000000in}{0.000000in}}%
\pgfpathlineto{\pgfqpoint{0.000000in}{0.055556in}}%
\pgfusepath{stroke,fill}%
}%
\begin{pgfscope}%
\pgfsys@transformshift{5.960000in}{0.600000in}%
\pgfsys@useobject{currentmarker}{}%
\end{pgfscope}%
\end{pgfscope}%
\begin{pgfscope}%
\pgfsetbuttcap%
\pgfsetroundjoin%
\definecolor{currentfill}{rgb}{0.000000,0.000000,0.000000}%
\pgfsetfillcolor{currentfill}%
\pgfsetlinewidth{0.501875pt}%
\definecolor{currentstroke}{rgb}{0.000000,0.000000,0.000000}%
\pgfsetstrokecolor{currentstroke}%
\pgfsetdash{}{0pt}%
\pgfsys@defobject{currentmarker}{\pgfqpoint{0.000000in}{-0.055556in}}{\pgfqpoint{0.000000in}{0.000000in}}{%
\pgfpathmoveto{\pgfqpoint{0.000000in}{0.000000in}}%
\pgfpathlineto{\pgfqpoint{0.000000in}{-0.055556in}}%
\pgfusepath{stroke,fill}%
}%
\begin{pgfscope}%
\pgfsys@transformshift{5.960000in}{5.400000in}%
\pgfsys@useobject{currentmarker}{}%
\end{pgfscope}%
\end{pgfscope}%
\begin{pgfscope}%
\definecolor{textcolor}{rgb}{0.000000,0.000000,0.000000}%
\pgfsetstrokecolor{textcolor}%
\pgfsetfillcolor{textcolor}%
\pgftext[x=5.960000in,y=0.544444in,,top]{\color{textcolor}\rmfamily\fontsize{10.000000}{12.000000}\selectfont \(\displaystyle {80}\)}%
\end{pgfscope}%
\begin{pgfscope}%
\pgfpathrectangle{\pgfqpoint{1.000000in}{0.600000in}}{\pgfqpoint{6.200000in}{4.800000in}}%
\pgfusepath{clip}%
\pgfsetbuttcap%
\pgfsetroundjoin%
\pgfsetlinewidth{0.501875pt}%
\definecolor{currentstroke}{rgb}{0.000000,0.000000,0.000000}%
\pgfsetstrokecolor{currentstroke}%
\pgfsetdash{{1.000000pt}{3.000000pt}}{0.000000pt}%
\pgfpathmoveto{\pgfqpoint{7.200000in}{0.600000in}}%
\pgfpathlineto{\pgfqpoint{7.200000in}{5.400000in}}%
\pgfusepath{stroke}%
\end{pgfscope}%
\begin{pgfscope}%
\pgfsetbuttcap%
\pgfsetroundjoin%
\definecolor{currentfill}{rgb}{0.000000,0.000000,0.000000}%
\pgfsetfillcolor{currentfill}%
\pgfsetlinewidth{0.501875pt}%
\definecolor{currentstroke}{rgb}{0.000000,0.000000,0.000000}%
\pgfsetstrokecolor{currentstroke}%
\pgfsetdash{}{0pt}%
\pgfsys@defobject{currentmarker}{\pgfqpoint{0.000000in}{0.000000in}}{\pgfqpoint{0.000000in}{0.055556in}}{%
\pgfpathmoveto{\pgfqpoint{0.000000in}{0.000000in}}%
\pgfpathlineto{\pgfqpoint{0.000000in}{0.055556in}}%
\pgfusepath{stroke,fill}%
}%
\begin{pgfscope}%
\pgfsys@transformshift{7.200000in}{0.600000in}%
\pgfsys@useobject{currentmarker}{}%
\end{pgfscope}%
\end{pgfscope}%
\begin{pgfscope}%
\pgfsetbuttcap%
\pgfsetroundjoin%
\definecolor{currentfill}{rgb}{0.000000,0.000000,0.000000}%
\pgfsetfillcolor{currentfill}%
\pgfsetlinewidth{0.501875pt}%
\definecolor{currentstroke}{rgb}{0.000000,0.000000,0.000000}%
\pgfsetstrokecolor{currentstroke}%
\pgfsetdash{}{0pt}%
\pgfsys@defobject{currentmarker}{\pgfqpoint{0.000000in}{-0.055556in}}{\pgfqpoint{0.000000in}{0.000000in}}{%
\pgfpathmoveto{\pgfqpoint{0.000000in}{0.000000in}}%
\pgfpathlineto{\pgfqpoint{0.000000in}{-0.055556in}}%
\pgfusepath{stroke,fill}%
}%
\begin{pgfscope}%
\pgfsys@transformshift{7.200000in}{5.400000in}%
\pgfsys@useobject{currentmarker}{}%
\end{pgfscope}%
\end{pgfscope}%
\begin{pgfscope}%
\definecolor{textcolor}{rgb}{0.000000,0.000000,0.000000}%
\pgfsetstrokecolor{textcolor}%
\pgfsetfillcolor{textcolor}%
\pgftext[x=7.200000in,y=0.544444in,,top]{\color{textcolor}\rmfamily\fontsize{10.000000}{12.000000}\selectfont \(\displaystyle {100}\)}%
\end{pgfscope}%
\begin{pgfscope}%
\definecolor{textcolor}{rgb}{0.000000,0.000000,0.000000}%
\pgfsetstrokecolor{textcolor}%
\pgfsetfillcolor{textcolor}%
\pgftext[x=4.100000in,y=0.351543in,,top]{\color{textcolor}\rmfamily\fontsize{12.000000}{14.400000}\selectfont \(\displaystyle time\ (s)\)}%
\end{pgfscope}%
\begin{pgfscope}%
\pgfpathrectangle{\pgfqpoint{1.000000in}{0.600000in}}{\pgfqpoint{6.200000in}{4.800000in}}%
\pgfusepath{clip}%
\pgfsetbuttcap%
\pgfsetroundjoin%
\pgfsetlinewidth{0.501875pt}%
\definecolor{currentstroke}{rgb}{0.000000,0.000000,0.000000}%
\pgfsetstrokecolor{currentstroke}%
\pgfsetdash{{1.000000pt}{3.000000pt}}{0.000000pt}%
\pgfpathmoveto{\pgfqpoint{1.000000in}{0.600000in}}%
\pgfpathlineto{\pgfqpoint{7.200000in}{0.600000in}}%
\pgfusepath{stroke}%
\end{pgfscope}%
\begin{pgfscope}%
\pgfsetbuttcap%
\pgfsetroundjoin%
\definecolor{currentfill}{rgb}{0.000000,0.000000,0.000000}%
\pgfsetfillcolor{currentfill}%
\pgfsetlinewidth{0.501875pt}%
\definecolor{currentstroke}{rgb}{0.000000,0.000000,0.000000}%
\pgfsetstrokecolor{currentstroke}%
\pgfsetdash{}{0pt}%
\pgfsys@defobject{currentmarker}{\pgfqpoint{0.000000in}{0.000000in}}{\pgfqpoint{0.055556in}{0.000000in}}{%
\pgfpathmoveto{\pgfqpoint{0.000000in}{0.000000in}}%
\pgfpathlineto{\pgfqpoint{0.055556in}{0.000000in}}%
\pgfusepath{stroke,fill}%
}%
\begin{pgfscope}%
\pgfsys@transformshift{1.000000in}{0.600000in}%
\pgfsys@useobject{currentmarker}{}%
\end{pgfscope}%
\end{pgfscope}%
\begin{pgfscope}%
\pgfsetbuttcap%
\pgfsetroundjoin%
\definecolor{currentfill}{rgb}{0.000000,0.000000,0.000000}%
\pgfsetfillcolor{currentfill}%
\pgfsetlinewidth{0.501875pt}%
\definecolor{currentstroke}{rgb}{0.000000,0.000000,0.000000}%
\pgfsetstrokecolor{currentstroke}%
\pgfsetdash{}{0pt}%
\pgfsys@defobject{currentmarker}{\pgfqpoint{-0.055556in}{0.000000in}}{\pgfqpoint{-0.000000in}{0.000000in}}{%
\pgfpathmoveto{\pgfqpoint{-0.000000in}{0.000000in}}%
\pgfpathlineto{\pgfqpoint{-0.055556in}{0.000000in}}%
\pgfusepath{stroke,fill}%
}%
\begin{pgfscope}%
\pgfsys@transformshift{7.200000in}{0.600000in}%
\pgfsys@useobject{currentmarker}{}%
\end{pgfscope}%
\end{pgfscope}%
\begin{pgfscope}%
\definecolor{textcolor}{rgb}{0.000000,0.000000,0.000000}%
\pgfsetstrokecolor{textcolor}%
\pgfsetfillcolor{textcolor}%
\pgftext[x=0.944444in,y=0.600000in,right,]{\color{textcolor}\rmfamily\fontsize{10.000000}{12.000000}\selectfont \(\displaystyle {\ensuremath{-}1000}\)}%
\end{pgfscope}%
\begin{pgfscope}%
\pgfpathrectangle{\pgfqpoint{1.000000in}{0.600000in}}{\pgfqpoint{6.200000in}{4.800000in}}%
\pgfusepath{clip}%
\pgfsetbuttcap%
\pgfsetroundjoin%
\pgfsetlinewidth{0.501875pt}%
\definecolor{currentstroke}{rgb}{0.000000,0.000000,0.000000}%
\pgfsetstrokecolor{currentstroke}%
\pgfsetdash{{1.000000pt}{3.000000pt}}{0.000000pt}%
\pgfpathmoveto{\pgfqpoint{1.000000in}{1.400000in}}%
\pgfpathlineto{\pgfqpoint{7.200000in}{1.400000in}}%
\pgfusepath{stroke}%
\end{pgfscope}%
\begin{pgfscope}%
\pgfsetbuttcap%
\pgfsetroundjoin%
\definecolor{currentfill}{rgb}{0.000000,0.000000,0.000000}%
\pgfsetfillcolor{currentfill}%
\pgfsetlinewidth{0.501875pt}%
\definecolor{currentstroke}{rgb}{0.000000,0.000000,0.000000}%
\pgfsetstrokecolor{currentstroke}%
\pgfsetdash{}{0pt}%
\pgfsys@defobject{currentmarker}{\pgfqpoint{0.000000in}{0.000000in}}{\pgfqpoint{0.055556in}{0.000000in}}{%
\pgfpathmoveto{\pgfqpoint{0.000000in}{0.000000in}}%
\pgfpathlineto{\pgfqpoint{0.055556in}{0.000000in}}%
\pgfusepath{stroke,fill}%
}%
\begin{pgfscope}%
\pgfsys@transformshift{1.000000in}{1.400000in}%
\pgfsys@useobject{currentmarker}{}%
\end{pgfscope}%
\end{pgfscope}%
\begin{pgfscope}%
\pgfsetbuttcap%
\pgfsetroundjoin%
\definecolor{currentfill}{rgb}{0.000000,0.000000,0.000000}%
\pgfsetfillcolor{currentfill}%
\pgfsetlinewidth{0.501875pt}%
\definecolor{currentstroke}{rgb}{0.000000,0.000000,0.000000}%
\pgfsetstrokecolor{currentstroke}%
\pgfsetdash{}{0pt}%
\pgfsys@defobject{currentmarker}{\pgfqpoint{-0.055556in}{0.000000in}}{\pgfqpoint{-0.000000in}{0.000000in}}{%
\pgfpathmoveto{\pgfqpoint{-0.000000in}{0.000000in}}%
\pgfpathlineto{\pgfqpoint{-0.055556in}{0.000000in}}%
\pgfusepath{stroke,fill}%
}%
\begin{pgfscope}%
\pgfsys@transformshift{7.200000in}{1.400000in}%
\pgfsys@useobject{currentmarker}{}%
\end{pgfscope}%
\end{pgfscope}%
\begin{pgfscope}%
\definecolor{textcolor}{rgb}{0.000000,0.000000,0.000000}%
\pgfsetstrokecolor{textcolor}%
\pgfsetfillcolor{textcolor}%
\pgftext[x=0.944444in,y=1.400000in,right,]{\color{textcolor}\rmfamily\fontsize{10.000000}{12.000000}\selectfont \(\displaystyle {\ensuremath{-}800}\)}%
\end{pgfscope}%
\begin{pgfscope}%
\pgfpathrectangle{\pgfqpoint{1.000000in}{0.600000in}}{\pgfqpoint{6.200000in}{4.800000in}}%
\pgfusepath{clip}%
\pgfsetbuttcap%
\pgfsetroundjoin%
\pgfsetlinewidth{0.501875pt}%
\definecolor{currentstroke}{rgb}{0.000000,0.000000,0.000000}%
\pgfsetstrokecolor{currentstroke}%
\pgfsetdash{{1.000000pt}{3.000000pt}}{0.000000pt}%
\pgfpathmoveto{\pgfqpoint{1.000000in}{2.200000in}}%
\pgfpathlineto{\pgfqpoint{7.200000in}{2.200000in}}%
\pgfusepath{stroke}%
\end{pgfscope}%
\begin{pgfscope}%
\pgfsetbuttcap%
\pgfsetroundjoin%
\definecolor{currentfill}{rgb}{0.000000,0.000000,0.000000}%
\pgfsetfillcolor{currentfill}%
\pgfsetlinewidth{0.501875pt}%
\definecolor{currentstroke}{rgb}{0.000000,0.000000,0.000000}%
\pgfsetstrokecolor{currentstroke}%
\pgfsetdash{}{0pt}%
\pgfsys@defobject{currentmarker}{\pgfqpoint{0.000000in}{0.000000in}}{\pgfqpoint{0.055556in}{0.000000in}}{%
\pgfpathmoveto{\pgfqpoint{0.000000in}{0.000000in}}%
\pgfpathlineto{\pgfqpoint{0.055556in}{0.000000in}}%
\pgfusepath{stroke,fill}%
}%
\begin{pgfscope}%
\pgfsys@transformshift{1.000000in}{2.200000in}%
\pgfsys@useobject{currentmarker}{}%
\end{pgfscope}%
\end{pgfscope}%
\begin{pgfscope}%
\pgfsetbuttcap%
\pgfsetroundjoin%
\definecolor{currentfill}{rgb}{0.000000,0.000000,0.000000}%
\pgfsetfillcolor{currentfill}%
\pgfsetlinewidth{0.501875pt}%
\definecolor{currentstroke}{rgb}{0.000000,0.000000,0.000000}%
\pgfsetstrokecolor{currentstroke}%
\pgfsetdash{}{0pt}%
\pgfsys@defobject{currentmarker}{\pgfqpoint{-0.055556in}{0.000000in}}{\pgfqpoint{-0.000000in}{0.000000in}}{%
\pgfpathmoveto{\pgfqpoint{-0.000000in}{0.000000in}}%
\pgfpathlineto{\pgfqpoint{-0.055556in}{0.000000in}}%
\pgfusepath{stroke,fill}%
}%
\begin{pgfscope}%
\pgfsys@transformshift{7.200000in}{2.200000in}%
\pgfsys@useobject{currentmarker}{}%
\end{pgfscope}%
\end{pgfscope}%
\begin{pgfscope}%
\definecolor{textcolor}{rgb}{0.000000,0.000000,0.000000}%
\pgfsetstrokecolor{textcolor}%
\pgfsetfillcolor{textcolor}%
\pgftext[x=0.944444in,y=2.200000in,right,]{\color{textcolor}\rmfamily\fontsize{10.000000}{12.000000}\selectfont \(\displaystyle {\ensuremath{-}600}\)}%
\end{pgfscope}%
\begin{pgfscope}%
\pgfpathrectangle{\pgfqpoint{1.000000in}{0.600000in}}{\pgfqpoint{6.200000in}{4.800000in}}%
\pgfusepath{clip}%
\pgfsetbuttcap%
\pgfsetroundjoin%
\pgfsetlinewidth{0.501875pt}%
\definecolor{currentstroke}{rgb}{0.000000,0.000000,0.000000}%
\pgfsetstrokecolor{currentstroke}%
\pgfsetdash{{1.000000pt}{3.000000pt}}{0.000000pt}%
\pgfpathmoveto{\pgfqpoint{1.000000in}{3.000000in}}%
\pgfpathlineto{\pgfqpoint{7.200000in}{3.000000in}}%
\pgfusepath{stroke}%
\end{pgfscope}%
\begin{pgfscope}%
\pgfsetbuttcap%
\pgfsetroundjoin%
\definecolor{currentfill}{rgb}{0.000000,0.000000,0.000000}%
\pgfsetfillcolor{currentfill}%
\pgfsetlinewidth{0.501875pt}%
\definecolor{currentstroke}{rgb}{0.000000,0.000000,0.000000}%
\pgfsetstrokecolor{currentstroke}%
\pgfsetdash{}{0pt}%
\pgfsys@defobject{currentmarker}{\pgfqpoint{0.000000in}{0.000000in}}{\pgfqpoint{0.055556in}{0.000000in}}{%
\pgfpathmoveto{\pgfqpoint{0.000000in}{0.000000in}}%
\pgfpathlineto{\pgfqpoint{0.055556in}{0.000000in}}%
\pgfusepath{stroke,fill}%
}%
\begin{pgfscope}%
\pgfsys@transformshift{1.000000in}{3.000000in}%
\pgfsys@useobject{currentmarker}{}%
\end{pgfscope}%
\end{pgfscope}%
\begin{pgfscope}%
\pgfsetbuttcap%
\pgfsetroundjoin%
\definecolor{currentfill}{rgb}{0.000000,0.000000,0.000000}%
\pgfsetfillcolor{currentfill}%
\pgfsetlinewidth{0.501875pt}%
\definecolor{currentstroke}{rgb}{0.000000,0.000000,0.000000}%
\pgfsetstrokecolor{currentstroke}%
\pgfsetdash{}{0pt}%
\pgfsys@defobject{currentmarker}{\pgfqpoint{-0.055556in}{0.000000in}}{\pgfqpoint{-0.000000in}{0.000000in}}{%
\pgfpathmoveto{\pgfqpoint{-0.000000in}{0.000000in}}%
\pgfpathlineto{\pgfqpoint{-0.055556in}{0.000000in}}%
\pgfusepath{stroke,fill}%
}%
\begin{pgfscope}%
\pgfsys@transformshift{7.200000in}{3.000000in}%
\pgfsys@useobject{currentmarker}{}%
\end{pgfscope}%
\end{pgfscope}%
\begin{pgfscope}%
\definecolor{textcolor}{rgb}{0.000000,0.000000,0.000000}%
\pgfsetstrokecolor{textcolor}%
\pgfsetfillcolor{textcolor}%
\pgftext[x=0.944444in,y=3.000000in,right,]{\color{textcolor}\rmfamily\fontsize{10.000000}{12.000000}\selectfont \(\displaystyle {\ensuremath{-}400}\)}%
\end{pgfscope}%
\begin{pgfscope}%
\pgfpathrectangle{\pgfqpoint{1.000000in}{0.600000in}}{\pgfqpoint{6.200000in}{4.800000in}}%
\pgfusepath{clip}%
\pgfsetbuttcap%
\pgfsetroundjoin%
\pgfsetlinewidth{0.501875pt}%
\definecolor{currentstroke}{rgb}{0.000000,0.000000,0.000000}%
\pgfsetstrokecolor{currentstroke}%
\pgfsetdash{{1.000000pt}{3.000000pt}}{0.000000pt}%
\pgfpathmoveto{\pgfqpoint{1.000000in}{3.800000in}}%
\pgfpathlineto{\pgfqpoint{7.200000in}{3.800000in}}%
\pgfusepath{stroke}%
\end{pgfscope}%
\begin{pgfscope}%
\pgfsetbuttcap%
\pgfsetroundjoin%
\definecolor{currentfill}{rgb}{0.000000,0.000000,0.000000}%
\pgfsetfillcolor{currentfill}%
\pgfsetlinewidth{0.501875pt}%
\definecolor{currentstroke}{rgb}{0.000000,0.000000,0.000000}%
\pgfsetstrokecolor{currentstroke}%
\pgfsetdash{}{0pt}%
\pgfsys@defobject{currentmarker}{\pgfqpoint{0.000000in}{0.000000in}}{\pgfqpoint{0.055556in}{0.000000in}}{%
\pgfpathmoveto{\pgfqpoint{0.000000in}{0.000000in}}%
\pgfpathlineto{\pgfqpoint{0.055556in}{0.000000in}}%
\pgfusepath{stroke,fill}%
}%
\begin{pgfscope}%
\pgfsys@transformshift{1.000000in}{3.800000in}%
\pgfsys@useobject{currentmarker}{}%
\end{pgfscope}%
\end{pgfscope}%
\begin{pgfscope}%
\pgfsetbuttcap%
\pgfsetroundjoin%
\definecolor{currentfill}{rgb}{0.000000,0.000000,0.000000}%
\pgfsetfillcolor{currentfill}%
\pgfsetlinewidth{0.501875pt}%
\definecolor{currentstroke}{rgb}{0.000000,0.000000,0.000000}%
\pgfsetstrokecolor{currentstroke}%
\pgfsetdash{}{0pt}%
\pgfsys@defobject{currentmarker}{\pgfqpoint{-0.055556in}{0.000000in}}{\pgfqpoint{-0.000000in}{0.000000in}}{%
\pgfpathmoveto{\pgfqpoint{-0.000000in}{0.000000in}}%
\pgfpathlineto{\pgfqpoint{-0.055556in}{0.000000in}}%
\pgfusepath{stroke,fill}%
}%
\begin{pgfscope}%
\pgfsys@transformshift{7.200000in}{3.800000in}%
\pgfsys@useobject{currentmarker}{}%
\end{pgfscope}%
\end{pgfscope}%
\begin{pgfscope}%
\definecolor{textcolor}{rgb}{0.000000,0.000000,0.000000}%
\pgfsetstrokecolor{textcolor}%
\pgfsetfillcolor{textcolor}%
\pgftext[x=0.944444in,y=3.800000in,right,]{\color{textcolor}\rmfamily\fontsize{10.000000}{12.000000}\selectfont \(\displaystyle {\ensuremath{-}200}\)}%
\end{pgfscope}%
\begin{pgfscope}%
\pgfpathrectangle{\pgfqpoint{1.000000in}{0.600000in}}{\pgfqpoint{6.200000in}{4.800000in}}%
\pgfusepath{clip}%
\pgfsetbuttcap%
\pgfsetroundjoin%
\pgfsetlinewidth{0.501875pt}%
\definecolor{currentstroke}{rgb}{0.000000,0.000000,0.000000}%
\pgfsetstrokecolor{currentstroke}%
\pgfsetdash{{1.000000pt}{3.000000pt}}{0.000000pt}%
\pgfpathmoveto{\pgfqpoint{1.000000in}{4.600000in}}%
\pgfpathlineto{\pgfqpoint{7.200000in}{4.600000in}}%
\pgfusepath{stroke}%
\end{pgfscope}%
\begin{pgfscope}%
\pgfsetbuttcap%
\pgfsetroundjoin%
\definecolor{currentfill}{rgb}{0.000000,0.000000,0.000000}%
\pgfsetfillcolor{currentfill}%
\pgfsetlinewidth{0.501875pt}%
\definecolor{currentstroke}{rgb}{0.000000,0.000000,0.000000}%
\pgfsetstrokecolor{currentstroke}%
\pgfsetdash{}{0pt}%
\pgfsys@defobject{currentmarker}{\pgfqpoint{0.000000in}{0.000000in}}{\pgfqpoint{0.055556in}{0.000000in}}{%
\pgfpathmoveto{\pgfqpoint{0.000000in}{0.000000in}}%
\pgfpathlineto{\pgfqpoint{0.055556in}{0.000000in}}%
\pgfusepath{stroke,fill}%
}%
\begin{pgfscope}%
\pgfsys@transformshift{1.000000in}{4.600000in}%
\pgfsys@useobject{currentmarker}{}%
\end{pgfscope}%
\end{pgfscope}%
\begin{pgfscope}%
\pgfsetbuttcap%
\pgfsetroundjoin%
\definecolor{currentfill}{rgb}{0.000000,0.000000,0.000000}%
\pgfsetfillcolor{currentfill}%
\pgfsetlinewidth{0.501875pt}%
\definecolor{currentstroke}{rgb}{0.000000,0.000000,0.000000}%
\pgfsetstrokecolor{currentstroke}%
\pgfsetdash{}{0pt}%
\pgfsys@defobject{currentmarker}{\pgfqpoint{-0.055556in}{0.000000in}}{\pgfqpoint{-0.000000in}{0.000000in}}{%
\pgfpathmoveto{\pgfqpoint{-0.000000in}{0.000000in}}%
\pgfpathlineto{\pgfqpoint{-0.055556in}{0.000000in}}%
\pgfusepath{stroke,fill}%
}%
\begin{pgfscope}%
\pgfsys@transformshift{7.200000in}{4.600000in}%
\pgfsys@useobject{currentmarker}{}%
\end{pgfscope}%
\end{pgfscope}%
\begin{pgfscope}%
\definecolor{textcolor}{rgb}{0.000000,0.000000,0.000000}%
\pgfsetstrokecolor{textcolor}%
\pgfsetfillcolor{textcolor}%
\pgftext[x=0.944444in,y=4.600000in,right,]{\color{textcolor}\rmfamily\fontsize{10.000000}{12.000000}\selectfont \(\displaystyle {0}\)}%
\end{pgfscope}%
\begin{pgfscope}%
\pgfpathrectangle{\pgfqpoint{1.000000in}{0.600000in}}{\pgfqpoint{6.200000in}{4.800000in}}%
\pgfusepath{clip}%
\pgfsetbuttcap%
\pgfsetroundjoin%
\pgfsetlinewidth{0.501875pt}%
\definecolor{currentstroke}{rgb}{0.000000,0.000000,0.000000}%
\pgfsetstrokecolor{currentstroke}%
\pgfsetdash{{1.000000pt}{3.000000pt}}{0.000000pt}%
\pgfpathmoveto{\pgfqpoint{1.000000in}{5.400000in}}%
\pgfpathlineto{\pgfqpoint{7.200000in}{5.400000in}}%
\pgfusepath{stroke}%
\end{pgfscope}%
\begin{pgfscope}%
\pgfsetbuttcap%
\pgfsetroundjoin%
\definecolor{currentfill}{rgb}{0.000000,0.000000,0.000000}%
\pgfsetfillcolor{currentfill}%
\pgfsetlinewidth{0.501875pt}%
\definecolor{currentstroke}{rgb}{0.000000,0.000000,0.000000}%
\pgfsetstrokecolor{currentstroke}%
\pgfsetdash{}{0pt}%
\pgfsys@defobject{currentmarker}{\pgfqpoint{0.000000in}{0.000000in}}{\pgfqpoint{0.055556in}{0.000000in}}{%
\pgfpathmoveto{\pgfqpoint{0.000000in}{0.000000in}}%
\pgfpathlineto{\pgfqpoint{0.055556in}{0.000000in}}%
\pgfusepath{stroke,fill}%
}%
\begin{pgfscope}%
\pgfsys@transformshift{1.000000in}{5.400000in}%
\pgfsys@useobject{currentmarker}{}%
\end{pgfscope}%
\end{pgfscope}%
\begin{pgfscope}%
\pgfsetbuttcap%
\pgfsetroundjoin%
\definecolor{currentfill}{rgb}{0.000000,0.000000,0.000000}%
\pgfsetfillcolor{currentfill}%
\pgfsetlinewidth{0.501875pt}%
\definecolor{currentstroke}{rgb}{0.000000,0.000000,0.000000}%
\pgfsetstrokecolor{currentstroke}%
\pgfsetdash{}{0pt}%
\pgfsys@defobject{currentmarker}{\pgfqpoint{-0.055556in}{0.000000in}}{\pgfqpoint{-0.000000in}{0.000000in}}{%
\pgfpathmoveto{\pgfqpoint{-0.000000in}{0.000000in}}%
\pgfpathlineto{\pgfqpoint{-0.055556in}{0.000000in}}%
\pgfusepath{stroke,fill}%
}%
\begin{pgfscope}%
\pgfsys@transformshift{7.200000in}{5.400000in}%
\pgfsys@useobject{currentmarker}{}%
\end{pgfscope}%
\end{pgfscope}%
\begin{pgfscope}%
\definecolor{textcolor}{rgb}{0.000000,0.000000,0.000000}%
\pgfsetstrokecolor{textcolor}%
\pgfsetfillcolor{textcolor}%
\pgftext[x=0.944444in,y=5.400000in,right,]{\color{textcolor}\rmfamily\fontsize{10.000000}{12.000000}\selectfont \(\displaystyle {200}\)}%
\end{pgfscope}%
\begin{pgfscope}%
\definecolor{textcolor}{rgb}{0.000000,0.000000,0.000000}%
\pgfsetstrokecolor{textcolor}%
\pgfsetfillcolor{textcolor}%
\pgftext[x=0.489196in,y=3.000000in,,bottom,rotate=90.000000]{\color{textcolor}\rmfamily\fontsize{12.000000}{14.400000}\selectfont \(\displaystyle \theta\ (rad)\)}%
\end{pgfscope}%
\begin{pgfscope}%
\definecolor{textcolor}{rgb}{0.000000,0.000000,0.000000}%
\pgfsetstrokecolor{textcolor}%
\pgfsetfillcolor{textcolor}%
\pgftext[x=4.100000in,y=5.469444in,,base]{\color{textcolor}\rmfamily\fontsize{12.000000}{14.400000}\selectfont \(\displaystyle Simple\ pendulum\ solution\ (time\ step = 5\ (s))\)}%
\end{pgfscope}%
\begin{pgfscope}%
\pgfsetbuttcap%
\pgfsetmiterjoin%
\definecolor{currentfill}{rgb}{1.000000,1.000000,1.000000}%
\pgfsetfillcolor{currentfill}%
\pgfsetlinewidth{1.003750pt}%
\definecolor{currentstroke}{rgb}{0.000000,0.000000,0.000000}%
\pgfsetstrokecolor{currentstroke}%
\pgfsetdash{}{0pt}%
\pgfpathmoveto{\pgfqpoint{5.106890in}{4.569445in}}%
\pgfpathlineto{\pgfqpoint{7.116667in}{4.569445in}}%
\pgfpathlineto{\pgfqpoint{7.116667in}{5.316667in}}%
\pgfpathlineto{\pgfqpoint{5.106890in}{5.316667in}}%
\pgfpathlineto{\pgfqpoint{5.106890in}{4.569445in}}%
\pgfpathclose%
\pgfusepath{stroke,fill}%
\end{pgfscope}%
\begin{pgfscope}%
\pgfsetrectcap%
\pgfsetroundjoin%
\pgfsetlinewidth{1.003750pt}%
\definecolor{currentstroke}{rgb}{1.000000,0.000000,0.000000}%
\pgfsetstrokecolor{currentstroke}%
\pgfsetdash{}{0pt}%
\pgfpathmoveto{\pgfqpoint{5.223556in}{5.191667in}}%
\pgfpathlineto{\pgfqpoint{5.456890in}{5.191667in}}%
\pgfusepath{stroke}%
\end{pgfscope}%
\begin{pgfscope}%
\definecolor{textcolor}{rgb}{0.000000,0.000000,0.000000}%
\pgfsetstrokecolor{textcolor}%
\pgfsetfillcolor{textcolor}%
\pgftext[x=5.640223in,y=5.133333in,left,base]{\color{textcolor}\rmfamily\fontsize{12.000000}{14.400000}\selectfont \(\displaystyle euler\ explicit\)}%
\end{pgfscope}%
\begin{pgfscope}%
\pgfsetrectcap%
\pgfsetroundjoin%
\pgfsetlinewidth{1.003750pt}%
\definecolor{currentstroke}{rgb}{0.000000,0.000000,1.000000}%
\pgfsetstrokecolor{currentstroke}%
\pgfsetdash{}{0pt}%
\pgfpathmoveto{\pgfqpoint{5.223556in}{4.959260in}}%
\pgfpathlineto{\pgfqpoint{5.456890in}{4.959260in}}%
\pgfusepath{stroke}%
\end{pgfscope}%
\begin{pgfscope}%
\definecolor{textcolor}{rgb}{0.000000,0.000000,0.000000}%
\pgfsetstrokecolor{textcolor}%
\pgfsetfillcolor{textcolor}%
\pgftext[x=5.640223in,y=4.900926in,left,base]{\color{textcolor}\rmfamily\fontsize{12.000000}{14.400000}\selectfont \(\displaystyle euler\ implicit\)}%
\end{pgfscope}%
\begin{pgfscope}%
\pgfsetrectcap%
\pgfsetroundjoin%
\pgfsetlinewidth{1.003750pt}%
\definecolor{currentstroke}{rgb}{0.000000,0.000000,0.000000}%
\pgfsetstrokecolor{currentstroke}%
\pgfsetdash{}{0pt}%
\pgfpathmoveto{\pgfqpoint{5.223556in}{4.726852in}}%
\pgfpathlineto{\pgfqpoint{5.456890in}{4.726852in}}%
\pgfusepath{stroke}%
\end{pgfscope}%
\begin{pgfscope}%
\definecolor{textcolor}{rgb}{0.000000,0.000000,0.000000}%
\pgfsetstrokecolor{textcolor}%
\pgfsetfillcolor{textcolor}%
\pgftext[x=5.640223in,y=4.668519in,left,base]{\color{textcolor}\rmfamily\fontsize{12.000000}{14.400000}\selectfont \(\displaystyle trapezoidal\ scheme\)}%
\end{pgfscope}%
\end{pgfpicture}%
\makeatother%
\endgroup%
}
    \end{figure}

    \begin{figure}[ht!]
    \centering
    \resizebox{0.9\linewidth}{!}{%% Creator: Matplotlib, PGF backend
%%
%% To include the figure in your LaTeX document, write
%%   \input{<filename>.pgf}
%%
%% Make sure the required packages are loaded in your preamble
%%   \usepackage{pgf}
%%
%% Also ensure that all the required font packages are loaded; for instance,
%% the lmodern package is sometimes necessary when using math font.
%%   \usepackage{lmodern}
%%
%% Figures using additional raster images can only be included by \input if
%% they are in the same directory as the main LaTeX file. For loading figures
%% from other directories you can use the `import` package
%%   \usepackage{import}
%%
%% and then include the figures with
%%   \import{<path to file>}{<filename>.pgf}
%%
%% Matplotlib used the following preamble
%%
\begingroup%
\makeatletter%
\begin{pgfpicture}%
\pgfpathrectangle{\pgfpointorigin}{\pgfqpoint{8.000000in}{6.000000in}}%
\pgfusepath{use as bounding box, clip}%
\begin{pgfscope}%
\pgfsetbuttcap%
\pgfsetmiterjoin%
\definecolor{currentfill}{rgb}{1.000000,1.000000,1.000000}%
\pgfsetfillcolor{currentfill}%
\pgfsetlinewidth{0.000000pt}%
\definecolor{currentstroke}{rgb}{1.000000,1.000000,1.000000}%
\pgfsetstrokecolor{currentstroke}%
\pgfsetdash{}{0pt}%
\pgfpathmoveto{\pgfqpoint{0.000000in}{0.000000in}}%
\pgfpathlineto{\pgfqpoint{8.000000in}{0.000000in}}%
\pgfpathlineto{\pgfqpoint{8.000000in}{6.000000in}}%
\pgfpathlineto{\pgfqpoint{0.000000in}{6.000000in}}%
\pgfpathlineto{\pgfqpoint{0.000000in}{0.000000in}}%
\pgfpathclose%
\pgfusepath{fill}%
\end{pgfscope}%
\begin{pgfscope}%
\pgfsetbuttcap%
\pgfsetmiterjoin%
\definecolor{currentfill}{rgb}{1.000000,1.000000,1.000000}%
\pgfsetfillcolor{currentfill}%
\pgfsetlinewidth{0.000000pt}%
\definecolor{currentstroke}{rgb}{0.000000,0.000000,0.000000}%
\pgfsetstrokecolor{currentstroke}%
\pgfsetstrokeopacity{0.000000}%
\pgfsetdash{}{0pt}%
\pgfpathmoveto{\pgfqpoint{1.000000in}{0.600000in}}%
\pgfpathlineto{\pgfqpoint{7.200000in}{0.600000in}}%
\pgfpathlineto{\pgfqpoint{7.200000in}{5.400000in}}%
\pgfpathlineto{\pgfqpoint{1.000000in}{5.400000in}}%
\pgfpathlineto{\pgfqpoint{1.000000in}{0.600000in}}%
\pgfpathclose%
\pgfusepath{fill}%
\end{pgfscope}%
\begin{pgfscope}%
\pgfpathrectangle{\pgfqpoint{1.000000in}{0.600000in}}{\pgfqpoint{6.200000in}{4.800000in}}%
\pgfusepath{clip}%
\pgfsetrectcap%
\pgfsetroundjoin%
\pgfsetlinewidth{1.003750pt}%
\definecolor{currentstroke}{rgb}{1.000000,0.000000,0.000000}%
\pgfsetstrokecolor{currentstroke}%
\pgfsetdash{}{0pt}%
\pgfpathmoveto{\pgfqpoint{1.000000in}{4.715004in}}%
\pgfpathlineto{\pgfqpoint{1.620000in}{4.715004in}}%
\pgfpathlineto{\pgfqpoint{2.240000in}{4.646432in}}%
\pgfpathlineto{\pgfqpoint{2.860000in}{4.509290in}}%
\pgfpathlineto{\pgfqpoint{3.480000in}{4.274815in}}%
\pgfpathlineto{\pgfqpoint{4.100000in}{3.907488in}}%
\pgfpathlineto{\pgfqpoint{4.720000in}{3.540788in}}%
\pgfpathlineto{\pgfqpoint{5.340000in}{3.074522in}}%
\pgfpathlineto{\pgfqpoint{5.960000in}{2.734116in}}%
\pgfpathlineto{\pgfqpoint{6.580000in}{2.525323in}}%
\pgfpathlineto{\pgfqpoint{7.200000in}{2.186005in}}%
\pgfusepath{stroke}%
\end{pgfscope}%
\begin{pgfscope}%
\pgfpathrectangle{\pgfqpoint{1.000000in}{0.600000in}}{\pgfqpoint{6.200000in}{4.800000in}}%
\pgfusepath{clip}%
\pgfsetrectcap%
\pgfsetroundjoin%
\pgfsetlinewidth{1.003750pt}%
\definecolor{currentstroke}{rgb}{0.000000,0.000000,1.000000}%
\pgfsetstrokecolor{currentstroke}%
\pgfsetdash{}{0pt}%
\pgfpathmoveto{\pgfqpoint{1.000000in}{4.715004in}}%
\pgfpathlineto{\pgfqpoint{1.620000in}{4.646432in}}%
\pgfpathlineto{\pgfqpoint{2.240000in}{4.480530in}}%
\pgfpathlineto{\pgfqpoint{2.860000in}{4.391166in}}%
\pgfpathlineto{\pgfqpoint{3.480000in}{4.143897in}}%
\pgfpathlineto{\pgfqpoint{4.100000in}{4.123181in}}%
\pgfpathlineto{\pgfqpoint{4.720000in}{3.866317in}}%
\pgfpathlineto{\pgfqpoint{5.340000in}{3.898967in}}%
\pgfpathlineto{\pgfqpoint{5.960000in}{3.854570in}}%
\pgfpathlineto{\pgfqpoint{6.580000in}{3.780033in}}%
\pgfpathlineto{\pgfqpoint{7.200000in}{3.803721in}}%
\pgfusepath{stroke}%
\end{pgfscope}%
\begin{pgfscope}%
\pgfpathrectangle{\pgfqpoint{1.000000in}{0.600000in}}{\pgfqpoint{6.200000in}{4.800000in}}%
\pgfusepath{clip}%
\pgfsetrectcap%
\pgfsetroundjoin%
\pgfsetlinewidth{1.003750pt}%
\definecolor{currentstroke}{rgb}{0.000000,0.000000,0.000000}%
\pgfsetstrokecolor{currentstroke}%
\pgfsetdash{}{0pt}%
\pgfpathmoveto{\pgfqpoint{1.000000in}{4.715004in}}%
\pgfpathlineto{\pgfqpoint{1.620000in}{4.680718in}}%
\pgfpathlineto{\pgfqpoint{2.240000in}{4.570303in}}%
\pgfpathlineto{\pgfqpoint{2.860000in}{4.338918in}}%
\pgfpathlineto{\pgfqpoint{3.480000in}{4.029633in}}%
\pgfpathlineto{\pgfqpoint{4.100000in}{3.761209in}}%
\pgfpathlineto{\pgfqpoint{4.720000in}{3.398110in}}%
\pgfpathlineto{\pgfqpoint{5.340000in}{2.915308in}}%
\pgfpathlineto{\pgfqpoint{5.960000in}{2.303041in}}%
\pgfpathlineto{\pgfqpoint{6.580000in}{1.607061in}}%
\pgfpathlineto{\pgfqpoint{7.200000in}{0.859255in}}%
\pgfusepath{stroke}%
\end{pgfscope}%
\begin{pgfscope}%
\pgfsetrectcap%
\pgfsetmiterjoin%
\pgfsetlinewidth{1.003750pt}%
\definecolor{currentstroke}{rgb}{0.000000,0.000000,0.000000}%
\pgfsetstrokecolor{currentstroke}%
\pgfsetdash{}{0pt}%
\pgfpathmoveto{\pgfqpoint{1.000000in}{0.600000in}}%
\pgfpathlineto{\pgfqpoint{1.000000in}{5.400000in}}%
\pgfusepath{stroke}%
\end{pgfscope}%
\begin{pgfscope}%
\pgfsetrectcap%
\pgfsetmiterjoin%
\pgfsetlinewidth{1.003750pt}%
\definecolor{currentstroke}{rgb}{0.000000,0.000000,0.000000}%
\pgfsetstrokecolor{currentstroke}%
\pgfsetdash{}{0pt}%
\pgfpathmoveto{\pgfqpoint{7.200000in}{0.600000in}}%
\pgfpathlineto{\pgfqpoint{7.200000in}{5.400000in}}%
\pgfusepath{stroke}%
\end{pgfscope}%
\begin{pgfscope}%
\pgfsetrectcap%
\pgfsetmiterjoin%
\pgfsetlinewidth{1.003750pt}%
\definecolor{currentstroke}{rgb}{0.000000,0.000000,0.000000}%
\pgfsetstrokecolor{currentstroke}%
\pgfsetdash{}{0pt}%
\pgfpathmoveto{\pgfqpoint{1.000000in}{0.600000in}}%
\pgfpathlineto{\pgfqpoint{7.200000in}{0.600000in}}%
\pgfusepath{stroke}%
\end{pgfscope}%
\begin{pgfscope}%
\pgfsetrectcap%
\pgfsetmiterjoin%
\pgfsetlinewidth{1.003750pt}%
\definecolor{currentstroke}{rgb}{0.000000,0.000000,0.000000}%
\pgfsetstrokecolor{currentstroke}%
\pgfsetdash{}{0pt}%
\pgfpathmoveto{\pgfqpoint{1.000000in}{5.400000in}}%
\pgfpathlineto{\pgfqpoint{7.200000in}{5.400000in}}%
\pgfusepath{stroke}%
\end{pgfscope}%
\begin{pgfscope}%
\pgfpathrectangle{\pgfqpoint{1.000000in}{0.600000in}}{\pgfqpoint{6.200000in}{4.800000in}}%
\pgfusepath{clip}%
\pgfsetbuttcap%
\pgfsetroundjoin%
\pgfsetlinewidth{0.501875pt}%
\definecolor{currentstroke}{rgb}{0.000000,0.000000,0.000000}%
\pgfsetstrokecolor{currentstroke}%
\pgfsetdash{{1.000000pt}{3.000000pt}}{0.000000pt}%
\pgfpathmoveto{\pgfqpoint{1.000000in}{0.600000in}}%
\pgfpathlineto{\pgfqpoint{1.000000in}{5.400000in}}%
\pgfusepath{stroke}%
\end{pgfscope}%
\begin{pgfscope}%
\pgfsetbuttcap%
\pgfsetroundjoin%
\definecolor{currentfill}{rgb}{0.000000,0.000000,0.000000}%
\pgfsetfillcolor{currentfill}%
\pgfsetlinewidth{0.501875pt}%
\definecolor{currentstroke}{rgb}{0.000000,0.000000,0.000000}%
\pgfsetstrokecolor{currentstroke}%
\pgfsetdash{}{0pt}%
\pgfsys@defobject{currentmarker}{\pgfqpoint{0.000000in}{0.000000in}}{\pgfqpoint{0.000000in}{0.055556in}}{%
\pgfpathmoveto{\pgfqpoint{0.000000in}{0.000000in}}%
\pgfpathlineto{\pgfqpoint{0.000000in}{0.055556in}}%
\pgfusepath{stroke,fill}%
}%
\begin{pgfscope}%
\pgfsys@transformshift{1.000000in}{0.600000in}%
\pgfsys@useobject{currentmarker}{}%
\end{pgfscope}%
\end{pgfscope}%
\begin{pgfscope}%
\pgfsetbuttcap%
\pgfsetroundjoin%
\definecolor{currentfill}{rgb}{0.000000,0.000000,0.000000}%
\pgfsetfillcolor{currentfill}%
\pgfsetlinewidth{0.501875pt}%
\definecolor{currentstroke}{rgb}{0.000000,0.000000,0.000000}%
\pgfsetstrokecolor{currentstroke}%
\pgfsetdash{}{0pt}%
\pgfsys@defobject{currentmarker}{\pgfqpoint{0.000000in}{-0.055556in}}{\pgfqpoint{0.000000in}{0.000000in}}{%
\pgfpathmoveto{\pgfqpoint{0.000000in}{0.000000in}}%
\pgfpathlineto{\pgfqpoint{0.000000in}{-0.055556in}}%
\pgfusepath{stroke,fill}%
}%
\begin{pgfscope}%
\pgfsys@transformshift{1.000000in}{5.400000in}%
\pgfsys@useobject{currentmarker}{}%
\end{pgfscope}%
\end{pgfscope}%
\begin{pgfscope}%
\definecolor{textcolor}{rgb}{0.000000,0.000000,0.000000}%
\pgfsetstrokecolor{textcolor}%
\pgfsetfillcolor{textcolor}%
\pgftext[x=1.000000in,y=0.544444in,,top]{\color{textcolor}\rmfamily\fontsize{10.000000}{12.000000}\selectfont \(\displaystyle {0}\)}%
\end{pgfscope}%
\begin{pgfscope}%
\pgfpathrectangle{\pgfqpoint{1.000000in}{0.600000in}}{\pgfqpoint{6.200000in}{4.800000in}}%
\pgfusepath{clip}%
\pgfsetbuttcap%
\pgfsetroundjoin%
\pgfsetlinewidth{0.501875pt}%
\definecolor{currentstroke}{rgb}{0.000000,0.000000,0.000000}%
\pgfsetstrokecolor{currentstroke}%
\pgfsetdash{{1.000000pt}{3.000000pt}}{0.000000pt}%
\pgfpathmoveto{\pgfqpoint{2.240000in}{0.600000in}}%
\pgfpathlineto{\pgfqpoint{2.240000in}{5.400000in}}%
\pgfusepath{stroke}%
\end{pgfscope}%
\begin{pgfscope}%
\pgfsetbuttcap%
\pgfsetroundjoin%
\definecolor{currentfill}{rgb}{0.000000,0.000000,0.000000}%
\pgfsetfillcolor{currentfill}%
\pgfsetlinewidth{0.501875pt}%
\definecolor{currentstroke}{rgb}{0.000000,0.000000,0.000000}%
\pgfsetstrokecolor{currentstroke}%
\pgfsetdash{}{0pt}%
\pgfsys@defobject{currentmarker}{\pgfqpoint{0.000000in}{0.000000in}}{\pgfqpoint{0.000000in}{0.055556in}}{%
\pgfpathmoveto{\pgfqpoint{0.000000in}{0.000000in}}%
\pgfpathlineto{\pgfqpoint{0.000000in}{0.055556in}}%
\pgfusepath{stroke,fill}%
}%
\begin{pgfscope}%
\pgfsys@transformshift{2.240000in}{0.600000in}%
\pgfsys@useobject{currentmarker}{}%
\end{pgfscope}%
\end{pgfscope}%
\begin{pgfscope}%
\pgfsetbuttcap%
\pgfsetroundjoin%
\definecolor{currentfill}{rgb}{0.000000,0.000000,0.000000}%
\pgfsetfillcolor{currentfill}%
\pgfsetlinewidth{0.501875pt}%
\definecolor{currentstroke}{rgb}{0.000000,0.000000,0.000000}%
\pgfsetstrokecolor{currentstroke}%
\pgfsetdash{}{0pt}%
\pgfsys@defobject{currentmarker}{\pgfqpoint{0.000000in}{-0.055556in}}{\pgfqpoint{0.000000in}{0.000000in}}{%
\pgfpathmoveto{\pgfqpoint{0.000000in}{0.000000in}}%
\pgfpathlineto{\pgfqpoint{0.000000in}{-0.055556in}}%
\pgfusepath{stroke,fill}%
}%
\begin{pgfscope}%
\pgfsys@transformshift{2.240000in}{5.400000in}%
\pgfsys@useobject{currentmarker}{}%
\end{pgfscope}%
\end{pgfscope}%
\begin{pgfscope}%
\definecolor{textcolor}{rgb}{0.000000,0.000000,0.000000}%
\pgfsetstrokecolor{textcolor}%
\pgfsetfillcolor{textcolor}%
\pgftext[x=2.240000in,y=0.544444in,,top]{\color{textcolor}\rmfamily\fontsize{10.000000}{12.000000}\selectfont \(\displaystyle {20}\)}%
\end{pgfscope}%
\begin{pgfscope}%
\pgfpathrectangle{\pgfqpoint{1.000000in}{0.600000in}}{\pgfqpoint{6.200000in}{4.800000in}}%
\pgfusepath{clip}%
\pgfsetbuttcap%
\pgfsetroundjoin%
\pgfsetlinewidth{0.501875pt}%
\definecolor{currentstroke}{rgb}{0.000000,0.000000,0.000000}%
\pgfsetstrokecolor{currentstroke}%
\pgfsetdash{{1.000000pt}{3.000000pt}}{0.000000pt}%
\pgfpathmoveto{\pgfqpoint{3.480000in}{0.600000in}}%
\pgfpathlineto{\pgfqpoint{3.480000in}{5.400000in}}%
\pgfusepath{stroke}%
\end{pgfscope}%
\begin{pgfscope}%
\pgfsetbuttcap%
\pgfsetroundjoin%
\definecolor{currentfill}{rgb}{0.000000,0.000000,0.000000}%
\pgfsetfillcolor{currentfill}%
\pgfsetlinewidth{0.501875pt}%
\definecolor{currentstroke}{rgb}{0.000000,0.000000,0.000000}%
\pgfsetstrokecolor{currentstroke}%
\pgfsetdash{}{0pt}%
\pgfsys@defobject{currentmarker}{\pgfqpoint{0.000000in}{0.000000in}}{\pgfqpoint{0.000000in}{0.055556in}}{%
\pgfpathmoveto{\pgfqpoint{0.000000in}{0.000000in}}%
\pgfpathlineto{\pgfqpoint{0.000000in}{0.055556in}}%
\pgfusepath{stroke,fill}%
}%
\begin{pgfscope}%
\pgfsys@transformshift{3.480000in}{0.600000in}%
\pgfsys@useobject{currentmarker}{}%
\end{pgfscope}%
\end{pgfscope}%
\begin{pgfscope}%
\pgfsetbuttcap%
\pgfsetroundjoin%
\definecolor{currentfill}{rgb}{0.000000,0.000000,0.000000}%
\pgfsetfillcolor{currentfill}%
\pgfsetlinewidth{0.501875pt}%
\definecolor{currentstroke}{rgb}{0.000000,0.000000,0.000000}%
\pgfsetstrokecolor{currentstroke}%
\pgfsetdash{}{0pt}%
\pgfsys@defobject{currentmarker}{\pgfqpoint{0.000000in}{-0.055556in}}{\pgfqpoint{0.000000in}{0.000000in}}{%
\pgfpathmoveto{\pgfqpoint{0.000000in}{0.000000in}}%
\pgfpathlineto{\pgfqpoint{0.000000in}{-0.055556in}}%
\pgfusepath{stroke,fill}%
}%
\begin{pgfscope}%
\pgfsys@transformshift{3.480000in}{5.400000in}%
\pgfsys@useobject{currentmarker}{}%
\end{pgfscope}%
\end{pgfscope}%
\begin{pgfscope}%
\definecolor{textcolor}{rgb}{0.000000,0.000000,0.000000}%
\pgfsetstrokecolor{textcolor}%
\pgfsetfillcolor{textcolor}%
\pgftext[x=3.480000in,y=0.544444in,,top]{\color{textcolor}\rmfamily\fontsize{10.000000}{12.000000}\selectfont \(\displaystyle {40}\)}%
\end{pgfscope}%
\begin{pgfscope}%
\pgfpathrectangle{\pgfqpoint{1.000000in}{0.600000in}}{\pgfqpoint{6.200000in}{4.800000in}}%
\pgfusepath{clip}%
\pgfsetbuttcap%
\pgfsetroundjoin%
\pgfsetlinewidth{0.501875pt}%
\definecolor{currentstroke}{rgb}{0.000000,0.000000,0.000000}%
\pgfsetstrokecolor{currentstroke}%
\pgfsetdash{{1.000000pt}{3.000000pt}}{0.000000pt}%
\pgfpathmoveto{\pgfqpoint{4.720000in}{0.600000in}}%
\pgfpathlineto{\pgfqpoint{4.720000in}{5.400000in}}%
\pgfusepath{stroke}%
\end{pgfscope}%
\begin{pgfscope}%
\pgfsetbuttcap%
\pgfsetroundjoin%
\definecolor{currentfill}{rgb}{0.000000,0.000000,0.000000}%
\pgfsetfillcolor{currentfill}%
\pgfsetlinewidth{0.501875pt}%
\definecolor{currentstroke}{rgb}{0.000000,0.000000,0.000000}%
\pgfsetstrokecolor{currentstroke}%
\pgfsetdash{}{0pt}%
\pgfsys@defobject{currentmarker}{\pgfqpoint{0.000000in}{0.000000in}}{\pgfqpoint{0.000000in}{0.055556in}}{%
\pgfpathmoveto{\pgfqpoint{0.000000in}{0.000000in}}%
\pgfpathlineto{\pgfqpoint{0.000000in}{0.055556in}}%
\pgfusepath{stroke,fill}%
}%
\begin{pgfscope}%
\pgfsys@transformshift{4.720000in}{0.600000in}%
\pgfsys@useobject{currentmarker}{}%
\end{pgfscope}%
\end{pgfscope}%
\begin{pgfscope}%
\pgfsetbuttcap%
\pgfsetroundjoin%
\definecolor{currentfill}{rgb}{0.000000,0.000000,0.000000}%
\pgfsetfillcolor{currentfill}%
\pgfsetlinewidth{0.501875pt}%
\definecolor{currentstroke}{rgb}{0.000000,0.000000,0.000000}%
\pgfsetstrokecolor{currentstroke}%
\pgfsetdash{}{0pt}%
\pgfsys@defobject{currentmarker}{\pgfqpoint{0.000000in}{-0.055556in}}{\pgfqpoint{0.000000in}{0.000000in}}{%
\pgfpathmoveto{\pgfqpoint{0.000000in}{0.000000in}}%
\pgfpathlineto{\pgfqpoint{0.000000in}{-0.055556in}}%
\pgfusepath{stroke,fill}%
}%
\begin{pgfscope}%
\pgfsys@transformshift{4.720000in}{5.400000in}%
\pgfsys@useobject{currentmarker}{}%
\end{pgfscope}%
\end{pgfscope}%
\begin{pgfscope}%
\definecolor{textcolor}{rgb}{0.000000,0.000000,0.000000}%
\pgfsetstrokecolor{textcolor}%
\pgfsetfillcolor{textcolor}%
\pgftext[x=4.720000in,y=0.544444in,,top]{\color{textcolor}\rmfamily\fontsize{10.000000}{12.000000}\selectfont \(\displaystyle {60}\)}%
\end{pgfscope}%
\begin{pgfscope}%
\pgfpathrectangle{\pgfqpoint{1.000000in}{0.600000in}}{\pgfqpoint{6.200000in}{4.800000in}}%
\pgfusepath{clip}%
\pgfsetbuttcap%
\pgfsetroundjoin%
\pgfsetlinewidth{0.501875pt}%
\definecolor{currentstroke}{rgb}{0.000000,0.000000,0.000000}%
\pgfsetstrokecolor{currentstroke}%
\pgfsetdash{{1.000000pt}{3.000000pt}}{0.000000pt}%
\pgfpathmoveto{\pgfqpoint{5.960000in}{0.600000in}}%
\pgfpathlineto{\pgfqpoint{5.960000in}{5.400000in}}%
\pgfusepath{stroke}%
\end{pgfscope}%
\begin{pgfscope}%
\pgfsetbuttcap%
\pgfsetroundjoin%
\definecolor{currentfill}{rgb}{0.000000,0.000000,0.000000}%
\pgfsetfillcolor{currentfill}%
\pgfsetlinewidth{0.501875pt}%
\definecolor{currentstroke}{rgb}{0.000000,0.000000,0.000000}%
\pgfsetstrokecolor{currentstroke}%
\pgfsetdash{}{0pt}%
\pgfsys@defobject{currentmarker}{\pgfqpoint{0.000000in}{0.000000in}}{\pgfqpoint{0.000000in}{0.055556in}}{%
\pgfpathmoveto{\pgfqpoint{0.000000in}{0.000000in}}%
\pgfpathlineto{\pgfqpoint{0.000000in}{0.055556in}}%
\pgfusepath{stroke,fill}%
}%
\begin{pgfscope}%
\pgfsys@transformshift{5.960000in}{0.600000in}%
\pgfsys@useobject{currentmarker}{}%
\end{pgfscope}%
\end{pgfscope}%
\begin{pgfscope}%
\pgfsetbuttcap%
\pgfsetroundjoin%
\definecolor{currentfill}{rgb}{0.000000,0.000000,0.000000}%
\pgfsetfillcolor{currentfill}%
\pgfsetlinewidth{0.501875pt}%
\definecolor{currentstroke}{rgb}{0.000000,0.000000,0.000000}%
\pgfsetstrokecolor{currentstroke}%
\pgfsetdash{}{0pt}%
\pgfsys@defobject{currentmarker}{\pgfqpoint{0.000000in}{-0.055556in}}{\pgfqpoint{0.000000in}{0.000000in}}{%
\pgfpathmoveto{\pgfqpoint{0.000000in}{0.000000in}}%
\pgfpathlineto{\pgfqpoint{0.000000in}{-0.055556in}}%
\pgfusepath{stroke,fill}%
}%
\begin{pgfscope}%
\pgfsys@transformshift{5.960000in}{5.400000in}%
\pgfsys@useobject{currentmarker}{}%
\end{pgfscope}%
\end{pgfscope}%
\begin{pgfscope}%
\definecolor{textcolor}{rgb}{0.000000,0.000000,0.000000}%
\pgfsetstrokecolor{textcolor}%
\pgfsetfillcolor{textcolor}%
\pgftext[x=5.960000in,y=0.544444in,,top]{\color{textcolor}\rmfamily\fontsize{10.000000}{12.000000}\selectfont \(\displaystyle {80}\)}%
\end{pgfscope}%
\begin{pgfscope}%
\pgfpathrectangle{\pgfqpoint{1.000000in}{0.600000in}}{\pgfqpoint{6.200000in}{4.800000in}}%
\pgfusepath{clip}%
\pgfsetbuttcap%
\pgfsetroundjoin%
\pgfsetlinewidth{0.501875pt}%
\definecolor{currentstroke}{rgb}{0.000000,0.000000,0.000000}%
\pgfsetstrokecolor{currentstroke}%
\pgfsetdash{{1.000000pt}{3.000000pt}}{0.000000pt}%
\pgfpathmoveto{\pgfqpoint{7.200000in}{0.600000in}}%
\pgfpathlineto{\pgfqpoint{7.200000in}{5.400000in}}%
\pgfusepath{stroke}%
\end{pgfscope}%
\begin{pgfscope}%
\pgfsetbuttcap%
\pgfsetroundjoin%
\definecolor{currentfill}{rgb}{0.000000,0.000000,0.000000}%
\pgfsetfillcolor{currentfill}%
\pgfsetlinewidth{0.501875pt}%
\definecolor{currentstroke}{rgb}{0.000000,0.000000,0.000000}%
\pgfsetstrokecolor{currentstroke}%
\pgfsetdash{}{0pt}%
\pgfsys@defobject{currentmarker}{\pgfqpoint{0.000000in}{0.000000in}}{\pgfqpoint{0.000000in}{0.055556in}}{%
\pgfpathmoveto{\pgfqpoint{0.000000in}{0.000000in}}%
\pgfpathlineto{\pgfqpoint{0.000000in}{0.055556in}}%
\pgfusepath{stroke,fill}%
}%
\begin{pgfscope}%
\pgfsys@transformshift{7.200000in}{0.600000in}%
\pgfsys@useobject{currentmarker}{}%
\end{pgfscope}%
\end{pgfscope}%
\begin{pgfscope}%
\pgfsetbuttcap%
\pgfsetroundjoin%
\definecolor{currentfill}{rgb}{0.000000,0.000000,0.000000}%
\pgfsetfillcolor{currentfill}%
\pgfsetlinewidth{0.501875pt}%
\definecolor{currentstroke}{rgb}{0.000000,0.000000,0.000000}%
\pgfsetstrokecolor{currentstroke}%
\pgfsetdash{}{0pt}%
\pgfsys@defobject{currentmarker}{\pgfqpoint{0.000000in}{-0.055556in}}{\pgfqpoint{0.000000in}{0.000000in}}{%
\pgfpathmoveto{\pgfqpoint{0.000000in}{0.000000in}}%
\pgfpathlineto{\pgfqpoint{0.000000in}{-0.055556in}}%
\pgfusepath{stroke,fill}%
}%
\begin{pgfscope}%
\pgfsys@transformshift{7.200000in}{5.400000in}%
\pgfsys@useobject{currentmarker}{}%
\end{pgfscope}%
\end{pgfscope}%
\begin{pgfscope}%
\definecolor{textcolor}{rgb}{0.000000,0.000000,0.000000}%
\pgfsetstrokecolor{textcolor}%
\pgfsetfillcolor{textcolor}%
\pgftext[x=7.200000in,y=0.544444in,,top]{\color{textcolor}\rmfamily\fontsize{10.000000}{12.000000}\selectfont \(\displaystyle {100}\)}%
\end{pgfscope}%
\begin{pgfscope}%
\definecolor{textcolor}{rgb}{0.000000,0.000000,0.000000}%
\pgfsetstrokecolor{textcolor}%
\pgfsetfillcolor{textcolor}%
\pgftext[x=4.100000in,y=0.351543in,,top]{\color{textcolor}\rmfamily\fontsize{12.000000}{14.400000}\selectfont \(\displaystyle time\ (s)\)}%
\end{pgfscope}%
\begin{pgfscope}%
\pgfpathrectangle{\pgfqpoint{1.000000in}{0.600000in}}{\pgfqpoint{6.200000in}{4.800000in}}%
\pgfusepath{clip}%
\pgfsetbuttcap%
\pgfsetroundjoin%
\pgfsetlinewidth{0.501875pt}%
\definecolor{currentstroke}{rgb}{0.000000,0.000000,0.000000}%
\pgfsetstrokecolor{currentstroke}%
\pgfsetdash{{1.000000pt}{3.000000pt}}{0.000000pt}%
\pgfpathmoveto{\pgfqpoint{1.000000in}{0.600000in}}%
\pgfpathlineto{\pgfqpoint{7.200000in}{0.600000in}}%
\pgfusepath{stroke}%
\end{pgfscope}%
\begin{pgfscope}%
\pgfsetbuttcap%
\pgfsetroundjoin%
\definecolor{currentfill}{rgb}{0.000000,0.000000,0.000000}%
\pgfsetfillcolor{currentfill}%
\pgfsetlinewidth{0.501875pt}%
\definecolor{currentstroke}{rgb}{0.000000,0.000000,0.000000}%
\pgfsetstrokecolor{currentstroke}%
\pgfsetdash{}{0pt}%
\pgfsys@defobject{currentmarker}{\pgfqpoint{0.000000in}{0.000000in}}{\pgfqpoint{0.055556in}{0.000000in}}{%
\pgfpathmoveto{\pgfqpoint{0.000000in}{0.000000in}}%
\pgfpathlineto{\pgfqpoint{0.055556in}{0.000000in}}%
\pgfusepath{stroke,fill}%
}%
\begin{pgfscope}%
\pgfsys@transformshift{1.000000in}{0.600000in}%
\pgfsys@useobject{currentmarker}{}%
\end{pgfscope}%
\end{pgfscope}%
\begin{pgfscope}%
\pgfsetbuttcap%
\pgfsetroundjoin%
\definecolor{currentfill}{rgb}{0.000000,0.000000,0.000000}%
\pgfsetfillcolor{currentfill}%
\pgfsetlinewidth{0.501875pt}%
\definecolor{currentstroke}{rgb}{0.000000,0.000000,0.000000}%
\pgfsetstrokecolor{currentstroke}%
\pgfsetdash{}{0pt}%
\pgfsys@defobject{currentmarker}{\pgfqpoint{-0.055556in}{0.000000in}}{\pgfqpoint{-0.000000in}{0.000000in}}{%
\pgfpathmoveto{\pgfqpoint{-0.000000in}{0.000000in}}%
\pgfpathlineto{\pgfqpoint{-0.055556in}{0.000000in}}%
\pgfusepath{stroke,fill}%
}%
\begin{pgfscope}%
\pgfsys@transformshift{7.200000in}{0.600000in}%
\pgfsys@useobject{currentmarker}{}%
\end{pgfscope}%
\end{pgfscope}%
\begin{pgfscope}%
\definecolor{textcolor}{rgb}{0.000000,0.000000,0.000000}%
\pgfsetstrokecolor{textcolor}%
\pgfsetfillcolor{textcolor}%
\pgftext[x=0.944444in,y=0.600000in,right,]{\color{textcolor}\rmfamily\fontsize{10.000000}{12.000000}\selectfont \(\displaystyle {\ensuremath{-}3000}\)}%
\end{pgfscope}%
\begin{pgfscope}%
\pgfpathrectangle{\pgfqpoint{1.000000in}{0.600000in}}{\pgfqpoint{6.200000in}{4.800000in}}%
\pgfusepath{clip}%
\pgfsetbuttcap%
\pgfsetroundjoin%
\pgfsetlinewidth{0.501875pt}%
\definecolor{currentstroke}{rgb}{0.000000,0.000000,0.000000}%
\pgfsetstrokecolor{currentstroke}%
\pgfsetdash{{1.000000pt}{3.000000pt}}{0.000000pt}%
\pgfpathmoveto{\pgfqpoint{1.000000in}{1.285714in}}%
\pgfpathlineto{\pgfqpoint{7.200000in}{1.285714in}}%
\pgfusepath{stroke}%
\end{pgfscope}%
\begin{pgfscope}%
\pgfsetbuttcap%
\pgfsetroundjoin%
\definecolor{currentfill}{rgb}{0.000000,0.000000,0.000000}%
\pgfsetfillcolor{currentfill}%
\pgfsetlinewidth{0.501875pt}%
\definecolor{currentstroke}{rgb}{0.000000,0.000000,0.000000}%
\pgfsetstrokecolor{currentstroke}%
\pgfsetdash{}{0pt}%
\pgfsys@defobject{currentmarker}{\pgfqpoint{0.000000in}{0.000000in}}{\pgfqpoint{0.055556in}{0.000000in}}{%
\pgfpathmoveto{\pgfqpoint{0.000000in}{0.000000in}}%
\pgfpathlineto{\pgfqpoint{0.055556in}{0.000000in}}%
\pgfusepath{stroke,fill}%
}%
\begin{pgfscope}%
\pgfsys@transformshift{1.000000in}{1.285714in}%
\pgfsys@useobject{currentmarker}{}%
\end{pgfscope}%
\end{pgfscope}%
\begin{pgfscope}%
\pgfsetbuttcap%
\pgfsetroundjoin%
\definecolor{currentfill}{rgb}{0.000000,0.000000,0.000000}%
\pgfsetfillcolor{currentfill}%
\pgfsetlinewidth{0.501875pt}%
\definecolor{currentstroke}{rgb}{0.000000,0.000000,0.000000}%
\pgfsetstrokecolor{currentstroke}%
\pgfsetdash{}{0pt}%
\pgfsys@defobject{currentmarker}{\pgfqpoint{-0.055556in}{0.000000in}}{\pgfqpoint{-0.000000in}{0.000000in}}{%
\pgfpathmoveto{\pgfqpoint{-0.000000in}{0.000000in}}%
\pgfpathlineto{\pgfqpoint{-0.055556in}{0.000000in}}%
\pgfusepath{stroke,fill}%
}%
\begin{pgfscope}%
\pgfsys@transformshift{7.200000in}{1.285714in}%
\pgfsys@useobject{currentmarker}{}%
\end{pgfscope}%
\end{pgfscope}%
\begin{pgfscope}%
\definecolor{textcolor}{rgb}{0.000000,0.000000,0.000000}%
\pgfsetstrokecolor{textcolor}%
\pgfsetfillcolor{textcolor}%
\pgftext[x=0.944444in,y=1.285714in,right,]{\color{textcolor}\rmfamily\fontsize{10.000000}{12.000000}\selectfont \(\displaystyle {\ensuremath{-}2500}\)}%
\end{pgfscope}%
\begin{pgfscope}%
\pgfpathrectangle{\pgfqpoint{1.000000in}{0.600000in}}{\pgfqpoint{6.200000in}{4.800000in}}%
\pgfusepath{clip}%
\pgfsetbuttcap%
\pgfsetroundjoin%
\pgfsetlinewidth{0.501875pt}%
\definecolor{currentstroke}{rgb}{0.000000,0.000000,0.000000}%
\pgfsetstrokecolor{currentstroke}%
\pgfsetdash{{1.000000pt}{3.000000pt}}{0.000000pt}%
\pgfpathmoveto{\pgfqpoint{1.000000in}{1.971429in}}%
\pgfpathlineto{\pgfqpoint{7.200000in}{1.971429in}}%
\pgfusepath{stroke}%
\end{pgfscope}%
\begin{pgfscope}%
\pgfsetbuttcap%
\pgfsetroundjoin%
\definecolor{currentfill}{rgb}{0.000000,0.000000,0.000000}%
\pgfsetfillcolor{currentfill}%
\pgfsetlinewidth{0.501875pt}%
\definecolor{currentstroke}{rgb}{0.000000,0.000000,0.000000}%
\pgfsetstrokecolor{currentstroke}%
\pgfsetdash{}{0pt}%
\pgfsys@defobject{currentmarker}{\pgfqpoint{0.000000in}{0.000000in}}{\pgfqpoint{0.055556in}{0.000000in}}{%
\pgfpathmoveto{\pgfqpoint{0.000000in}{0.000000in}}%
\pgfpathlineto{\pgfqpoint{0.055556in}{0.000000in}}%
\pgfusepath{stroke,fill}%
}%
\begin{pgfscope}%
\pgfsys@transformshift{1.000000in}{1.971429in}%
\pgfsys@useobject{currentmarker}{}%
\end{pgfscope}%
\end{pgfscope}%
\begin{pgfscope}%
\pgfsetbuttcap%
\pgfsetroundjoin%
\definecolor{currentfill}{rgb}{0.000000,0.000000,0.000000}%
\pgfsetfillcolor{currentfill}%
\pgfsetlinewidth{0.501875pt}%
\definecolor{currentstroke}{rgb}{0.000000,0.000000,0.000000}%
\pgfsetstrokecolor{currentstroke}%
\pgfsetdash{}{0pt}%
\pgfsys@defobject{currentmarker}{\pgfqpoint{-0.055556in}{0.000000in}}{\pgfqpoint{-0.000000in}{0.000000in}}{%
\pgfpathmoveto{\pgfqpoint{-0.000000in}{0.000000in}}%
\pgfpathlineto{\pgfqpoint{-0.055556in}{0.000000in}}%
\pgfusepath{stroke,fill}%
}%
\begin{pgfscope}%
\pgfsys@transformshift{7.200000in}{1.971429in}%
\pgfsys@useobject{currentmarker}{}%
\end{pgfscope}%
\end{pgfscope}%
\begin{pgfscope}%
\definecolor{textcolor}{rgb}{0.000000,0.000000,0.000000}%
\pgfsetstrokecolor{textcolor}%
\pgfsetfillcolor{textcolor}%
\pgftext[x=0.944444in,y=1.971429in,right,]{\color{textcolor}\rmfamily\fontsize{10.000000}{12.000000}\selectfont \(\displaystyle {\ensuremath{-}2000}\)}%
\end{pgfscope}%
\begin{pgfscope}%
\pgfpathrectangle{\pgfqpoint{1.000000in}{0.600000in}}{\pgfqpoint{6.200000in}{4.800000in}}%
\pgfusepath{clip}%
\pgfsetbuttcap%
\pgfsetroundjoin%
\pgfsetlinewidth{0.501875pt}%
\definecolor{currentstroke}{rgb}{0.000000,0.000000,0.000000}%
\pgfsetstrokecolor{currentstroke}%
\pgfsetdash{{1.000000pt}{3.000000pt}}{0.000000pt}%
\pgfpathmoveto{\pgfqpoint{1.000000in}{2.657143in}}%
\pgfpathlineto{\pgfqpoint{7.200000in}{2.657143in}}%
\pgfusepath{stroke}%
\end{pgfscope}%
\begin{pgfscope}%
\pgfsetbuttcap%
\pgfsetroundjoin%
\definecolor{currentfill}{rgb}{0.000000,0.000000,0.000000}%
\pgfsetfillcolor{currentfill}%
\pgfsetlinewidth{0.501875pt}%
\definecolor{currentstroke}{rgb}{0.000000,0.000000,0.000000}%
\pgfsetstrokecolor{currentstroke}%
\pgfsetdash{}{0pt}%
\pgfsys@defobject{currentmarker}{\pgfqpoint{0.000000in}{0.000000in}}{\pgfqpoint{0.055556in}{0.000000in}}{%
\pgfpathmoveto{\pgfqpoint{0.000000in}{0.000000in}}%
\pgfpathlineto{\pgfqpoint{0.055556in}{0.000000in}}%
\pgfusepath{stroke,fill}%
}%
\begin{pgfscope}%
\pgfsys@transformshift{1.000000in}{2.657143in}%
\pgfsys@useobject{currentmarker}{}%
\end{pgfscope}%
\end{pgfscope}%
\begin{pgfscope}%
\pgfsetbuttcap%
\pgfsetroundjoin%
\definecolor{currentfill}{rgb}{0.000000,0.000000,0.000000}%
\pgfsetfillcolor{currentfill}%
\pgfsetlinewidth{0.501875pt}%
\definecolor{currentstroke}{rgb}{0.000000,0.000000,0.000000}%
\pgfsetstrokecolor{currentstroke}%
\pgfsetdash{}{0pt}%
\pgfsys@defobject{currentmarker}{\pgfqpoint{-0.055556in}{0.000000in}}{\pgfqpoint{-0.000000in}{0.000000in}}{%
\pgfpathmoveto{\pgfqpoint{-0.000000in}{0.000000in}}%
\pgfpathlineto{\pgfqpoint{-0.055556in}{0.000000in}}%
\pgfusepath{stroke,fill}%
}%
\begin{pgfscope}%
\pgfsys@transformshift{7.200000in}{2.657143in}%
\pgfsys@useobject{currentmarker}{}%
\end{pgfscope}%
\end{pgfscope}%
\begin{pgfscope}%
\definecolor{textcolor}{rgb}{0.000000,0.000000,0.000000}%
\pgfsetstrokecolor{textcolor}%
\pgfsetfillcolor{textcolor}%
\pgftext[x=0.944444in,y=2.657143in,right,]{\color{textcolor}\rmfamily\fontsize{10.000000}{12.000000}\selectfont \(\displaystyle {\ensuremath{-}1500}\)}%
\end{pgfscope}%
\begin{pgfscope}%
\pgfpathrectangle{\pgfqpoint{1.000000in}{0.600000in}}{\pgfqpoint{6.200000in}{4.800000in}}%
\pgfusepath{clip}%
\pgfsetbuttcap%
\pgfsetroundjoin%
\pgfsetlinewidth{0.501875pt}%
\definecolor{currentstroke}{rgb}{0.000000,0.000000,0.000000}%
\pgfsetstrokecolor{currentstroke}%
\pgfsetdash{{1.000000pt}{3.000000pt}}{0.000000pt}%
\pgfpathmoveto{\pgfqpoint{1.000000in}{3.342857in}}%
\pgfpathlineto{\pgfqpoint{7.200000in}{3.342857in}}%
\pgfusepath{stroke}%
\end{pgfscope}%
\begin{pgfscope}%
\pgfsetbuttcap%
\pgfsetroundjoin%
\definecolor{currentfill}{rgb}{0.000000,0.000000,0.000000}%
\pgfsetfillcolor{currentfill}%
\pgfsetlinewidth{0.501875pt}%
\definecolor{currentstroke}{rgb}{0.000000,0.000000,0.000000}%
\pgfsetstrokecolor{currentstroke}%
\pgfsetdash{}{0pt}%
\pgfsys@defobject{currentmarker}{\pgfqpoint{0.000000in}{0.000000in}}{\pgfqpoint{0.055556in}{0.000000in}}{%
\pgfpathmoveto{\pgfqpoint{0.000000in}{0.000000in}}%
\pgfpathlineto{\pgfqpoint{0.055556in}{0.000000in}}%
\pgfusepath{stroke,fill}%
}%
\begin{pgfscope}%
\pgfsys@transformshift{1.000000in}{3.342857in}%
\pgfsys@useobject{currentmarker}{}%
\end{pgfscope}%
\end{pgfscope}%
\begin{pgfscope}%
\pgfsetbuttcap%
\pgfsetroundjoin%
\definecolor{currentfill}{rgb}{0.000000,0.000000,0.000000}%
\pgfsetfillcolor{currentfill}%
\pgfsetlinewidth{0.501875pt}%
\definecolor{currentstroke}{rgb}{0.000000,0.000000,0.000000}%
\pgfsetstrokecolor{currentstroke}%
\pgfsetdash{}{0pt}%
\pgfsys@defobject{currentmarker}{\pgfqpoint{-0.055556in}{0.000000in}}{\pgfqpoint{-0.000000in}{0.000000in}}{%
\pgfpathmoveto{\pgfqpoint{-0.000000in}{0.000000in}}%
\pgfpathlineto{\pgfqpoint{-0.055556in}{0.000000in}}%
\pgfusepath{stroke,fill}%
}%
\begin{pgfscope}%
\pgfsys@transformshift{7.200000in}{3.342857in}%
\pgfsys@useobject{currentmarker}{}%
\end{pgfscope}%
\end{pgfscope}%
\begin{pgfscope}%
\definecolor{textcolor}{rgb}{0.000000,0.000000,0.000000}%
\pgfsetstrokecolor{textcolor}%
\pgfsetfillcolor{textcolor}%
\pgftext[x=0.944444in,y=3.342857in,right,]{\color{textcolor}\rmfamily\fontsize{10.000000}{12.000000}\selectfont \(\displaystyle {\ensuremath{-}1000}\)}%
\end{pgfscope}%
\begin{pgfscope}%
\pgfpathrectangle{\pgfqpoint{1.000000in}{0.600000in}}{\pgfqpoint{6.200000in}{4.800000in}}%
\pgfusepath{clip}%
\pgfsetbuttcap%
\pgfsetroundjoin%
\pgfsetlinewidth{0.501875pt}%
\definecolor{currentstroke}{rgb}{0.000000,0.000000,0.000000}%
\pgfsetstrokecolor{currentstroke}%
\pgfsetdash{{1.000000pt}{3.000000pt}}{0.000000pt}%
\pgfpathmoveto{\pgfqpoint{1.000000in}{4.028571in}}%
\pgfpathlineto{\pgfqpoint{7.200000in}{4.028571in}}%
\pgfusepath{stroke}%
\end{pgfscope}%
\begin{pgfscope}%
\pgfsetbuttcap%
\pgfsetroundjoin%
\definecolor{currentfill}{rgb}{0.000000,0.000000,0.000000}%
\pgfsetfillcolor{currentfill}%
\pgfsetlinewidth{0.501875pt}%
\definecolor{currentstroke}{rgb}{0.000000,0.000000,0.000000}%
\pgfsetstrokecolor{currentstroke}%
\pgfsetdash{}{0pt}%
\pgfsys@defobject{currentmarker}{\pgfqpoint{0.000000in}{0.000000in}}{\pgfqpoint{0.055556in}{0.000000in}}{%
\pgfpathmoveto{\pgfqpoint{0.000000in}{0.000000in}}%
\pgfpathlineto{\pgfqpoint{0.055556in}{0.000000in}}%
\pgfusepath{stroke,fill}%
}%
\begin{pgfscope}%
\pgfsys@transformshift{1.000000in}{4.028571in}%
\pgfsys@useobject{currentmarker}{}%
\end{pgfscope}%
\end{pgfscope}%
\begin{pgfscope}%
\pgfsetbuttcap%
\pgfsetroundjoin%
\definecolor{currentfill}{rgb}{0.000000,0.000000,0.000000}%
\pgfsetfillcolor{currentfill}%
\pgfsetlinewidth{0.501875pt}%
\definecolor{currentstroke}{rgb}{0.000000,0.000000,0.000000}%
\pgfsetstrokecolor{currentstroke}%
\pgfsetdash{}{0pt}%
\pgfsys@defobject{currentmarker}{\pgfqpoint{-0.055556in}{0.000000in}}{\pgfqpoint{-0.000000in}{0.000000in}}{%
\pgfpathmoveto{\pgfqpoint{-0.000000in}{0.000000in}}%
\pgfpathlineto{\pgfqpoint{-0.055556in}{0.000000in}}%
\pgfusepath{stroke,fill}%
}%
\begin{pgfscope}%
\pgfsys@transformshift{7.200000in}{4.028571in}%
\pgfsys@useobject{currentmarker}{}%
\end{pgfscope}%
\end{pgfscope}%
\begin{pgfscope}%
\definecolor{textcolor}{rgb}{0.000000,0.000000,0.000000}%
\pgfsetstrokecolor{textcolor}%
\pgfsetfillcolor{textcolor}%
\pgftext[x=0.944444in,y=4.028571in,right,]{\color{textcolor}\rmfamily\fontsize{10.000000}{12.000000}\selectfont \(\displaystyle {\ensuremath{-}500}\)}%
\end{pgfscope}%
\begin{pgfscope}%
\pgfpathrectangle{\pgfqpoint{1.000000in}{0.600000in}}{\pgfqpoint{6.200000in}{4.800000in}}%
\pgfusepath{clip}%
\pgfsetbuttcap%
\pgfsetroundjoin%
\pgfsetlinewidth{0.501875pt}%
\definecolor{currentstroke}{rgb}{0.000000,0.000000,0.000000}%
\pgfsetstrokecolor{currentstroke}%
\pgfsetdash{{1.000000pt}{3.000000pt}}{0.000000pt}%
\pgfpathmoveto{\pgfqpoint{1.000000in}{4.714286in}}%
\pgfpathlineto{\pgfqpoint{7.200000in}{4.714286in}}%
\pgfusepath{stroke}%
\end{pgfscope}%
\begin{pgfscope}%
\pgfsetbuttcap%
\pgfsetroundjoin%
\definecolor{currentfill}{rgb}{0.000000,0.000000,0.000000}%
\pgfsetfillcolor{currentfill}%
\pgfsetlinewidth{0.501875pt}%
\definecolor{currentstroke}{rgb}{0.000000,0.000000,0.000000}%
\pgfsetstrokecolor{currentstroke}%
\pgfsetdash{}{0pt}%
\pgfsys@defobject{currentmarker}{\pgfqpoint{0.000000in}{0.000000in}}{\pgfqpoint{0.055556in}{0.000000in}}{%
\pgfpathmoveto{\pgfqpoint{0.000000in}{0.000000in}}%
\pgfpathlineto{\pgfqpoint{0.055556in}{0.000000in}}%
\pgfusepath{stroke,fill}%
}%
\begin{pgfscope}%
\pgfsys@transformshift{1.000000in}{4.714286in}%
\pgfsys@useobject{currentmarker}{}%
\end{pgfscope}%
\end{pgfscope}%
\begin{pgfscope}%
\pgfsetbuttcap%
\pgfsetroundjoin%
\definecolor{currentfill}{rgb}{0.000000,0.000000,0.000000}%
\pgfsetfillcolor{currentfill}%
\pgfsetlinewidth{0.501875pt}%
\definecolor{currentstroke}{rgb}{0.000000,0.000000,0.000000}%
\pgfsetstrokecolor{currentstroke}%
\pgfsetdash{}{0pt}%
\pgfsys@defobject{currentmarker}{\pgfqpoint{-0.055556in}{0.000000in}}{\pgfqpoint{-0.000000in}{0.000000in}}{%
\pgfpathmoveto{\pgfqpoint{-0.000000in}{0.000000in}}%
\pgfpathlineto{\pgfqpoint{-0.055556in}{0.000000in}}%
\pgfusepath{stroke,fill}%
}%
\begin{pgfscope}%
\pgfsys@transformshift{7.200000in}{4.714286in}%
\pgfsys@useobject{currentmarker}{}%
\end{pgfscope}%
\end{pgfscope}%
\begin{pgfscope}%
\definecolor{textcolor}{rgb}{0.000000,0.000000,0.000000}%
\pgfsetstrokecolor{textcolor}%
\pgfsetfillcolor{textcolor}%
\pgftext[x=0.944444in,y=4.714286in,right,]{\color{textcolor}\rmfamily\fontsize{10.000000}{12.000000}\selectfont \(\displaystyle {0}\)}%
\end{pgfscope}%
\begin{pgfscope}%
\pgfpathrectangle{\pgfqpoint{1.000000in}{0.600000in}}{\pgfqpoint{6.200000in}{4.800000in}}%
\pgfusepath{clip}%
\pgfsetbuttcap%
\pgfsetroundjoin%
\pgfsetlinewidth{0.501875pt}%
\definecolor{currentstroke}{rgb}{0.000000,0.000000,0.000000}%
\pgfsetstrokecolor{currentstroke}%
\pgfsetdash{{1.000000pt}{3.000000pt}}{0.000000pt}%
\pgfpathmoveto{\pgfqpoint{1.000000in}{5.400000in}}%
\pgfpathlineto{\pgfqpoint{7.200000in}{5.400000in}}%
\pgfusepath{stroke}%
\end{pgfscope}%
\begin{pgfscope}%
\pgfsetbuttcap%
\pgfsetroundjoin%
\definecolor{currentfill}{rgb}{0.000000,0.000000,0.000000}%
\pgfsetfillcolor{currentfill}%
\pgfsetlinewidth{0.501875pt}%
\definecolor{currentstroke}{rgb}{0.000000,0.000000,0.000000}%
\pgfsetstrokecolor{currentstroke}%
\pgfsetdash{}{0pt}%
\pgfsys@defobject{currentmarker}{\pgfqpoint{0.000000in}{0.000000in}}{\pgfqpoint{0.055556in}{0.000000in}}{%
\pgfpathmoveto{\pgfqpoint{0.000000in}{0.000000in}}%
\pgfpathlineto{\pgfqpoint{0.055556in}{0.000000in}}%
\pgfusepath{stroke,fill}%
}%
\begin{pgfscope}%
\pgfsys@transformshift{1.000000in}{5.400000in}%
\pgfsys@useobject{currentmarker}{}%
\end{pgfscope}%
\end{pgfscope}%
\begin{pgfscope}%
\pgfsetbuttcap%
\pgfsetroundjoin%
\definecolor{currentfill}{rgb}{0.000000,0.000000,0.000000}%
\pgfsetfillcolor{currentfill}%
\pgfsetlinewidth{0.501875pt}%
\definecolor{currentstroke}{rgb}{0.000000,0.000000,0.000000}%
\pgfsetstrokecolor{currentstroke}%
\pgfsetdash{}{0pt}%
\pgfsys@defobject{currentmarker}{\pgfqpoint{-0.055556in}{0.000000in}}{\pgfqpoint{-0.000000in}{0.000000in}}{%
\pgfpathmoveto{\pgfqpoint{-0.000000in}{0.000000in}}%
\pgfpathlineto{\pgfqpoint{-0.055556in}{0.000000in}}%
\pgfusepath{stroke,fill}%
}%
\begin{pgfscope}%
\pgfsys@transformshift{7.200000in}{5.400000in}%
\pgfsys@useobject{currentmarker}{}%
\end{pgfscope}%
\end{pgfscope}%
\begin{pgfscope}%
\definecolor{textcolor}{rgb}{0.000000,0.000000,0.000000}%
\pgfsetstrokecolor{textcolor}%
\pgfsetfillcolor{textcolor}%
\pgftext[x=0.944444in,y=5.400000in,right,]{\color{textcolor}\rmfamily\fontsize{10.000000}{12.000000}\selectfont \(\displaystyle {500}\)}%
\end{pgfscope}%
\begin{pgfscope}%
\definecolor{textcolor}{rgb}{0.000000,0.000000,0.000000}%
\pgfsetstrokecolor{textcolor}%
\pgfsetfillcolor{textcolor}%
\pgftext[x=0.489196in,y=3.000000in,,bottom,rotate=90.000000]{\color{textcolor}\rmfamily\fontsize{12.000000}{14.400000}\selectfont \(\displaystyle \theta\ (rad)\)}%
\end{pgfscope}%
\begin{pgfscope}%
\definecolor{textcolor}{rgb}{0.000000,0.000000,0.000000}%
\pgfsetstrokecolor{textcolor}%
\pgfsetfillcolor{textcolor}%
\pgftext[x=4.100000in,y=5.469444in,,base]{\color{textcolor}\rmfamily\fontsize{12.000000}{14.400000}\selectfont \(\displaystyle Simple\ pendulum\ solution\ (time\ step = 10\ s)\)}%
\end{pgfscope}%
\begin{pgfscope}%
\pgfsetbuttcap%
\pgfsetmiterjoin%
\definecolor{currentfill}{rgb}{1.000000,1.000000,1.000000}%
\pgfsetfillcolor{currentfill}%
\pgfsetlinewidth{1.003750pt}%
\definecolor{currentstroke}{rgb}{0.000000,0.000000,0.000000}%
\pgfsetstrokecolor{currentstroke}%
\pgfsetdash{}{0pt}%
\pgfpathmoveto{\pgfqpoint{5.106890in}{4.569445in}}%
\pgfpathlineto{\pgfqpoint{7.116667in}{4.569445in}}%
\pgfpathlineto{\pgfqpoint{7.116667in}{5.316667in}}%
\pgfpathlineto{\pgfqpoint{5.106890in}{5.316667in}}%
\pgfpathlineto{\pgfqpoint{5.106890in}{4.569445in}}%
\pgfpathclose%
\pgfusepath{stroke,fill}%
\end{pgfscope}%
\begin{pgfscope}%
\pgfsetrectcap%
\pgfsetroundjoin%
\pgfsetlinewidth{1.003750pt}%
\definecolor{currentstroke}{rgb}{1.000000,0.000000,0.000000}%
\pgfsetstrokecolor{currentstroke}%
\pgfsetdash{}{0pt}%
\pgfpathmoveto{\pgfqpoint{5.223556in}{5.191667in}}%
\pgfpathlineto{\pgfqpoint{5.456890in}{5.191667in}}%
\pgfusepath{stroke}%
\end{pgfscope}%
\begin{pgfscope}%
\definecolor{textcolor}{rgb}{0.000000,0.000000,0.000000}%
\pgfsetstrokecolor{textcolor}%
\pgfsetfillcolor{textcolor}%
\pgftext[x=5.640223in,y=5.133333in,left,base]{\color{textcolor}\rmfamily\fontsize{12.000000}{14.400000}\selectfont \(\displaystyle euler\ explicit\)}%
\end{pgfscope}%
\begin{pgfscope}%
\pgfsetrectcap%
\pgfsetroundjoin%
\pgfsetlinewidth{1.003750pt}%
\definecolor{currentstroke}{rgb}{0.000000,0.000000,1.000000}%
\pgfsetstrokecolor{currentstroke}%
\pgfsetdash{}{0pt}%
\pgfpathmoveto{\pgfqpoint{5.223556in}{4.959260in}}%
\pgfpathlineto{\pgfqpoint{5.456890in}{4.959260in}}%
\pgfusepath{stroke}%
\end{pgfscope}%
\begin{pgfscope}%
\definecolor{textcolor}{rgb}{0.000000,0.000000,0.000000}%
\pgfsetstrokecolor{textcolor}%
\pgfsetfillcolor{textcolor}%
\pgftext[x=5.640223in,y=4.900926in,left,base]{\color{textcolor}\rmfamily\fontsize{12.000000}{14.400000}\selectfont \(\displaystyle euler\ implicit\)}%
\end{pgfscope}%
\begin{pgfscope}%
\pgfsetrectcap%
\pgfsetroundjoin%
\pgfsetlinewidth{1.003750pt}%
\definecolor{currentstroke}{rgb}{0.000000,0.000000,0.000000}%
\pgfsetstrokecolor{currentstroke}%
\pgfsetdash{}{0pt}%
\pgfpathmoveto{\pgfqpoint{5.223556in}{4.726852in}}%
\pgfpathlineto{\pgfqpoint{5.456890in}{4.726852in}}%
\pgfusepath{stroke}%
\end{pgfscope}%
\begin{pgfscope}%
\definecolor{textcolor}{rgb}{0.000000,0.000000,0.000000}%
\pgfsetstrokecolor{textcolor}%
\pgfsetfillcolor{textcolor}%
\pgftext[x=5.640223in,y=4.668519in,left,base]{\color{textcolor}\rmfamily\fontsize{12.000000}{14.400000}\selectfont \(\displaystyle trapezoidal\ scheme\)}%
\end{pgfscope}%
\end{pgfpicture}%
\makeatother%
\endgroup%
}
    \end{figure}

    \begin{figure}[ht!]
    \centering
    \resizebox{0.9\linewidth}{!}{%% Creator: Matplotlib, PGF backend
%%
%% To include the figure in your LaTeX document, write
%%   \input{<filename>.pgf}
%%
%% Make sure the required packages are loaded in your preamble
%%   \usepackage{pgf}
%%
%% Also ensure that all the required font packages are loaded; for instance,
%% the lmodern package is sometimes necessary when using math font.
%%   \usepackage{lmodern}
%%
%% Figures using additional raster images can only be included by \input if
%% they are in the same directory as the main LaTeX file. For loading figures
%% from other directories you can use the `import` package
%%   \usepackage{import}
%%
%% and then include the figures with
%%   \import{<path to file>}{<filename>.pgf}
%%
%% Matplotlib used the following preamble
%%
\begingroup%
\makeatletter%
\begin{pgfpicture}%
\pgfpathrectangle{\pgfpointorigin}{\pgfqpoint{8.000000in}{6.000000in}}%
\pgfusepath{use as bounding box, clip}%
\begin{pgfscope}%
\pgfsetbuttcap%
\pgfsetmiterjoin%
\definecolor{currentfill}{rgb}{1.000000,1.000000,1.000000}%
\pgfsetfillcolor{currentfill}%
\pgfsetlinewidth{0.000000pt}%
\definecolor{currentstroke}{rgb}{1.000000,1.000000,1.000000}%
\pgfsetstrokecolor{currentstroke}%
\pgfsetdash{}{0pt}%
\pgfpathmoveto{\pgfqpoint{0.000000in}{0.000000in}}%
\pgfpathlineto{\pgfqpoint{8.000000in}{0.000000in}}%
\pgfpathlineto{\pgfqpoint{8.000000in}{6.000000in}}%
\pgfpathlineto{\pgfqpoint{0.000000in}{6.000000in}}%
\pgfpathlineto{\pgfqpoint{0.000000in}{0.000000in}}%
\pgfpathclose%
\pgfusepath{fill}%
\end{pgfscope}%
\begin{pgfscope}%
\pgfsetbuttcap%
\pgfsetmiterjoin%
\definecolor{currentfill}{rgb}{1.000000,1.000000,1.000000}%
\pgfsetfillcolor{currentfill}%
\pgfsetlinewidth{0.000000pt}%
\definecolor{currentstroke}{rgb}{0.000000,0.000000,0.000000}%
\pgfsetstrokecolor{currentstroke}%
\pgfsetstrokeopacity{0.000000}%
\pgfsetdash{}{0pt}%
\pgfpathmoveto{\pgfqpoint{1.000000in}{0.600000in}}%
\pgfpathlineto{\pgfqpoint{7.200000in}{0.600000in}}%
\pgfpathlineto{\pgfqpoint{7.200000in}{5.400000in}}%
\pgfpathlineto{\pgfqpoint{1.000000in}{5.400000in}}%
\pgfpathlineto{\pgfqpoint{1.000000in}{0.600000in}}%
\pgfpathclose%
\pgfusepath{fill}%
\end{pgfscope}%
\begin{pgfscope}%
\pgfpathrectangle{\pgfqpoint{1.000000in}{0.600000in}}{\pgfqpoint{6.200000in}{4.800000in}}%
\pgfusepath{clip}%
\pgfsetrectcap%
\pgfsetroundjoin%
\pgfsetlinewidth{1.003750pt}%
\definecolor{currentstroke}{rgb}{1.000000,0.000000,0.000000}%
\pgfsetstrokecolor{currentstroke}%
\pgfsetdash{}{0pt}%
\pgfpathmoveto{\pgfqpoint{1.000000in}{4.441005in}}%
\pgfpathlineto{\pgfqpoint{2.240000in}{4.441005in}}%
\pgfpathlineto{\pgfqpoint{3.480000in}{4.057005in}}%
\pgfpathlineto{\pgfqpoint{4.720000in}{3.289005in}}%
\pgfpathlineto{\pgfqpoint{5.960000in}{1.753089in}}%
\pgfpathlineto{\pgfqpoint{7.200000in}{0.630183in}}%
\pgfusepath{stroke}%
\end{pgfscope}%
\begin{pgfscope}%
\pgfpathrectangle{\pgfqpoint{1.000000in}{0.600000in}}{\pgfqpoint{6.200000in}{4.800000in}}%
\pgfusepath{clip}%
\pgfsetrectcap%
\pgfsetroundjoin%
\pgfsetlinewidth{1.003750pt}%
\definecolor{currentstroke}{rgb}{0.000000,0.000000,1.000000}%
\pgfsetstrokecolor{currentstroke}%
\pgfsetdash{}{0pt}%
\pgfpathmoveto{\pgfqpoint{1.000000in}{4.441005in}}%
\pgfpathlineto{\pgfqpoint{2.240000in}{4.057005in}}%
\pgfpathlineto{\pgfqpoint{3.480000in}{2.905089in}}%
\pgfpathlineto{\pgfqpoint{4.720000in}{2.920832in}}%
\pgfpathlineto{\pgfqpoint{5.960000in}{2.607182in}}%
\pgfpathlineto{\pgfqpoint{7.200000in}{2.870209in}}%
\pgfusepath{stroke}%
\end{pgfscope}%
\begin{pgfscope}%
\pgfpathrectangle{\pgfqpoint{1.000000in}{0.600000in}}{\pgfqpoint{6.200000in}{4.800000in}}%
\pgfusepath{clip}%
\pgfsetrectcap%
\pgfsetroundjoin%
\pgfsetlinewidth{1.003750pt}%
\definecolor{currentstroke}{rgb}{0.000000,0.000000,0.000000}%
\pgfsetstrokecolor{currentstroke}%
\pgfsetdash{}{0pt}%
\pgfpathmoveto{\pgfqpoint{1.000000in}{4.441005in}}%
\pgfpathlineto{\pgfqpoint{2.240000in}{4.249005in}}%
\pgfpathlineto{\pgfqpoint{3.480000in}{3.531046in}}%
\pgfpathlineto{\pgfqpoint{4.720000in}{2.800983in}}%
\pgfpathlineto{\pgfqpoint{5.960000in}{1.939529in}}%
\pgfpathlineto{\pgfqpoint{7.200000in}{1.809859in}}%
\pgfusepath{stroke}%
\end{pgfscope}%
\begin{pgfscope}%
\pgfsetrectcap%
\pgfsetmiterjoin%
\pgfsetlinewidth{1.003750pt}%
\definecolor{currentstroke}{rgb}{0.000000,0.000000,0.000000}%
\pgfsetstrokecolor{currentstroke}%
\pgfsetdash{}{0pt}%
\pgfpathmoveto{\pgfqpoint{1.000000in}{0.600000in}}%
\pgfpathlineto{\pgfqpoint{1.000000in}{5.400000in}}%
\pgfusepath{stroke}%
\end{pgfscope}%
\begin{pgfscope}%
\pgfsetrectcap%
\pgfsetmiterjoin%
\pgfsetlinewidth{1.003750pt}%
\definecolor{currentstroke}{rgb}{0.000000,0.000000,0.000000}%
\pgfsetstrokecolor{currentstroke}%
\pgfsetdash{}{0pt}%
\pgfpathmoveto{\pgfqpoint{7.200000in}{0.600000in}}%
\pgfpathlineto{\pgfqpoint{7.200000in}{5.400000in}}%
\pgfusepath{stroke}%
\end{pgfscope}%
\begin{pgfscope}%
\pgfsetrectcap%
\pgfsetmiterjoin%
\pgfsetlinewidth{1.003750pt}%
\definecolor{currentstroke}{rgb}{0.000000,0.000000,0.000000}%
\pgfsetstrokecolor{currentstroke}%
\pgfsetdash{}{0pt}%
\pgfpathmoveto{\pgfqpoint{1.000000in}{0.600000in}}%
\pgfpathlineto{\pgfqpoint{7.200000in}{0.600000in}}%
\pgfusepath{stroke}%
\end{pgfscope}%
\begin{pgfscope}%
\pgfsetrectcap%
\pgfsetmiterjoin%
\pgfsetlinewidth{1.003750pt}%
\definecolor{currentstroke}{rgb}{0.000000,0.000000,0.000000}%
\pgfsetstrokecolor{currentstroke}%
\pgfsetdash{}{0pt}%
\pgfpathmoveto{\pgfqpoint{1.000000in}{5.400000in}}%
\pgfpathlineto{\pgfqpoint{7.200000in}{5.400000in}}%
\pgfusepath{stroke}%
\end{pgfscope}%
\begin{pgfscope}%
\pgfpathrectangle{\pgfqpoint{1.000000in}{0.600000in}}{\pgfqpoint{6.200000in}{4.800000in}}%
\pgfusepath{clip}%
\pgfsetbuttcap%
\pgfsetroundjoin%
\pgfsetlinewidth{0.501875pt}%
\definecolor{currentstroke}{rgb}{0.000000,0.000000,0.000000}%
\pgfsetstrokecolor{currentstroke}%
\pgfsetdash{{1.000000pt}{3.000000pt}}{0.000000pt}%
\pgfpathmoveto{\pgfqpoint{1.000000in}{0.600000in}}%
\pgfpathlineto{\pgfqpoint{1.000000in}{5.400000in}}%
\pgfusepath{stroke}%
\end{pgfscope}%
\begin{pgfscope}%
\pgfsetbuttcap%
\pgfsetroundjoin%
\definecolor{currentfill}{rgb}{0.000000,0.000000,0.000000}%
\pgfsetfillcolor{currentfill}%
\pgfsetlinewidth{0.501875pt}%
\definecolor{currentstroke}{rgb}{0.000000,0.000000,0.000000}%
\pgfsetstrokecolor{currentstroke}%
\pgfsetdash{}{0pt}%
\pgfsys@defobject{currentmarker}{\pgfqpoint{0.000000in}{0.000000in}}{\pgfqpoint{0.000000in}{0.055556in}}{%
\pgfpathmoveto{\pgfqpoint{0.000000in}{0.000000in}}%
\pgfpathlineto{\pgfqpoint{0.000000in}{0.055556in}}%
\pgfusepath{stroke,fill}%
}%
\begin{pgfscope}%
\pgfsys@transformshift{1.000000in}{0.600000in}%
\pgfsys@useobject{currentmarker}{}%
\end{pgfscope}%
\end{pgfscope}%
\begin{pgfscope}%
\pgfsetbuttcap%
\pgfsetroundjoin%
\definecolor{currentfill}{rgb}{0.000000,0.000000,0.000000}%
\pgfsetfillcolor{currentfill}%
\pgfsetlinewidth{0.501875pt}%
\definecolor{currentstroke}{rgb}{0.000000,0.000000,0.000000}%
\pgfsetstrokecolor{currentstroke}%
\pgfsetdash{}{0pt}%
\pgfsys@defobject{currentmarker}{\pgfqpoint{0.000000in}{-0.055556in}}{\pgfqpoint{0.000000in}{0.000000in}}{%
\pgfpathmoveto{\pgfqpoint{0.000000in}{0.000000in}}%
\pgfpathlineto{\pgfqpoint{0.000000in}{-0.055556in}}%
\pgfusepath{stroke,fill}%
}%
\begin{pgfscope}%
\pgfsys@transformshift{1.000000in}{5.400000in}%
\pgfsys@useobject{currentmarker}{}%
\end{pgfscope}%
\end{pgfscope}%
\begin{pgfscope}%
\definecolor{textcolor}{rgb}{0.000000,0.000000,0.000000}%
\pgfsetstrokecolor{textcolor}%
\pgfsetfillcolor{textcolor}%
\pgftext[x=1.000000in,y=0.544444in,,top]{\color{textcolor}\rmfamily\fontsize{10.000000}{12.000000}\selectfont \(\displaystyle {0}\)}%
\end{pgfscope}%
\begin{pgfscope}%
\pgfpathrectangle{\pgfqpoint{1.000000in}{0.600000in}}{\pgfqpoint{6.200000in}{4.800000in}}%
\pgfusepath{clip}%
\pgfsetbuttcap%
\pgfsetroundjoin%
\pgfsetlinewidth{0.501875pt}%
\definecolor{currentstroke}{rgb}{0.000000,0.000000,0.000000}%
\pgfsetstrokecolor{currentstroke}%
\pgfsetdash{{1.000000pt}{3.000000pt}}{0.000000pt}%
\pgfpathmoveto{\pgfqpoint{2.240000in}{0.600000in}}%
\pgfpathlineto{\pgfqpoint{2.240000in}{5.400000in}}%
\pgfusepath{stroke}%
\end{pgfscope}%
\begin{pgfscope}%
\pgfsetbuttcap%
\pgfsetroundjoin%
\definecolor{currentfill}{rgb}{0.000000,0.000000,0.000000}%
\pgfsetfillcolor{currentfill}%
\pgfsetlinewidth{0.501875pt}%
\definecolor{currentstroke}{rgb}{0.000000,0.000000,0.000000}%
\pgfsetstrokecolor{currentstroke}%
\pgfsetdash{}{0pt}%
\pgfsys@defobject{currentmarker}{\pgfqpoint{0.000000in}{0.000000in}}{\pgfqpoint{0.000000in}{0.055556in}}{%
\pgfpathmoveto{\pgfqpoint{0.000000in}{0.000000in}}%
\pgfpathlineto{\pgfqpoint{0.000000in}{0.055556in}}%
\pgfusepath{stroke,fill}%
}%
\begin{pgfscope}%
\pgfsys@transformshift{2.240000in}{0.600000in}%
\pgfsys@useobject{currentmarker}{}%
\end{pgfscope}%
\end{pgfscope}%
\begin{pgfscope}%
\pgfsetbuttcap%
\pgfsetroundjoin%
\definecolor{currentfill}{rgb}{0.000000,0.000000,0.000000}%
\pgfsetfillcolor{currentfill}%
\pgfsetlinewidth{0.501875pt}%
\definecolor{currentstroke}{rgb}{0.000000,0.000000,0.000000}%
\pgfsetstrokecolor{currentstroke}%
\pgfsetdash{}{0pt}%
\pgfsys@defobject{currentmarker}{\pgfqpoint{0.000000in}{-0.055556in}}{\pgfqpoint{0.000000in}{0.000000in}}{%
\pgfpathmoveto{\pgfqpoint{0.000000in}{0.000000in}}%
\pgfpathlineto{\pgfqpoint{0.000000in}{-0.055556in}}%
\pgfusepath{stroke,fill}%
}%
\begin{pgfscope}%
\pgfsys@transformshift{2.240000in}{5.400000in}%
\pgfsys@useobject{currentmarker}{}%
\end{pgfscope}%
\end{pgfscope}%
\begin{pgfscope}%
\definecolor{textcolor}{rgb}{0.000000,0.000000,0.000000}%
\pgfsetstrokecolor{textcolor}%
\pgfsetfillcolor{textcolor}%
\pgftext[x=2.240000in,y=0.544444in,,top]{\color{textcolor}\rmfamily\fontsize{10.000000}{12.000000}\selectfont \(\displaystyle {20}\)}%
\end{pgfscope}%
\begin{pgfscope}%
\pgfpathrectangle{\pgfqpoint{1.000000in}{0.600000in}}{\pgfqpoint{6.200000in}{4.800000in}}%
\pgfusepath{clip}%
\pgfsetbuttcap%
\pgfsetroundjoin%
\pgfsetlinewidth{0.501875pt}%
\definecolor{currentstroke}{rgb}{0.000000,0.000000,0.000000}%
\pgfsetstrokecolor{currentstroke}%
\pgfsetdash{{1.000000pt}{3.000000pt}}{0.000000pt}%
\pgfpathmoveto{\pgfqpoint{3.480000in}{0.600000in}}%
\pgfpathlineto{\pgfqpoint{3.480000in}{5.400000in}}%
\pgfusepath{stroke}%
\end{pgfscope}%
\begin{pgfscope}%
\pgfsetbuttcap%
\pgfsetroundjoin%
\definecolor{currentfill}{rgb}{0.000000,0.000000,0.000000}%
\pgfsetfillcolor{currentfill}%
\pgfsetlinewidth{0.501875pt}%
\definecolor{currentstroke}{rgb}{0.000000,0.000000,0.000000}%
\pgfsetstrokecolor{currentstroke}%
\pgfsetdash{}{0pt}%
\pgfsys@defobject{currentmarker}{\pgfqpoint{0.000000in}{0.000000in}}{\pgfqpoint{0.000000in}{0.055556in}}{%
\pgfpathmoveto{\pgfqpoint{0.000000in}{0.000000in}}%
\pgfpathlineto{\pgfqpoint{0.000000in}{0.055556in}}%
\pgfusepath{stroke,fill}%
}%
\begin{pgfscope}%
\pgfsys@transformshift{3.480000in}{0.600000in}%
\pgfsys@useobject{currentmarker}{}%
\end{pgfscope}%
\end{pgfscope}%
\begin{pgfscope}%
\pgfsetbuttcap%
\pgfsetroundjoin%
\definecolor{currentfill}{rgb}{0.000000,0.000000,0.000000}%
\pgfsetfillcolor{currentfill}%
\pgfsetlinewidth{0.501875pt}%
\definecolor{currentstroke}{rgb}{0.000000,0.000000,0.000000}%
\pgfsetstrokecolor{currentstroke}%
\pgfsetdash{}{0pt}%
\pgfsys@defobject{currentmarker}{\pgfqpoint{0.000000in}{-0.055556in}}{\pgfqpoint{0.000000in}{0.000000in}}{%
\pgfpathmoveto{\pgfqpoint{0.000000in}{0.000000in}}%
\pgfpathlineto{\pgfqpoint{0.000000in}{-0.055556in}}%
\pgfusepath{stroke,fill}%
}%
\begin{pgfscope}%
\pgfsys@transformshift{3.480000in}{5.400000in}%
\pgfsys@useobject{currentmarker}{}%
\end{pgfscope}%
\end{pgfscope}%
\begin{pgfscope}%
\definecolor{textcolor}{rgb}{0.000000,0.000000,0.000000}%
\pgfsetstrokecolor{textcolor}%
\pgfsetfillcolor{textcolor}%
\pgftext[x=3.480000in,y=0.544444in,,top]{\color{textcolor}\rmfamily\fontsize{10.000000}{12.000000}\selectfont \(\displaystyle {40}\)}%
\end{pgfscope}%
\begin{pgfscope}%
\pgfpathrectangle{\pgfqpoint{1.000000in}{0.600000in}}{\pgfqpoint{6.200000in}{4.800000in}}%
\pgfusepath{clip}%
\pgfsetbuttcap%
\pgfsetroundjoin%
\pgfsetlinewidth{0.501875pt}%
\definecolor{currentstroke}{rgb}{0.000000,0.000000,0.000000}%
\pgfsetstrokecolor{currentstroke}%
\pgfsetdash{{1.000000pt}{3.000000pt}}{0.000000pt}%
\pgfpathmoveto{\pgfqpoint{4.720000in}{0.600000in}}%
\pgfpathlineto{\pgfqpoint{4.720000in}{5.400000in}}%
\pgfusepath{stroke}%
\end{pgfscope}%
\begin{pgfscope}%
\pgfsetbuttcap%
\pgfsetroundjoin%
\definecolor{currentfill}{rgb}{0.000000,0.000000,0.000000}%
\pgfsetfillcolor{currentfill}%
\pgfsetlinewidth{0.501875pt}%
\definecolor{currentstroke}{rgb}{0.000000,0.000000,0.000000}%
\pgfsetstrokecolor{currentstroke}%
\pgfsetdash{}{0pt}%
\pgfsys@defobject{currentmarker}{\pgfqpoint{0.000000in}{0.000000in}}{\pgfqpoint{0.000000in}{0.055556in}}{%
\pgfpathmoveto{\pgfqpoint{0.000000in}{0.000000in}}%
\pgfpathlineto{\pgfqpoint{0.000000in}{0.055556in}}%
\pgfusepath{stroke,fill}%
}%
\begin{pgfscope}%
\pgfsys@transformshift{4.720000in}{0.600000in}%
\pgfsys@useobject{currentmarker}{}%
\end{pgfscope}%
\end{pgfscope}%
\begin{pgfscope}%
\pgfsetbuttcap%
\pgfsetroundjoin%
\definecolor{currentfill}{rgb}{0.000000,0.000000,0.000000}%
\pgfsetfillcolor{currentfill}%
\pgfsetlinewidth{0.501875pt}%
\definecolor{currentstroke}{rgb}{0.000000,0.000000,0.000000}%
\pgfsetstrokecolor{currentstroke}%
\pgfsetdash{}{0pt}%
\pgfsys@defobject{currentmarker}{\pgfqpoint{0.000000in}{-0.055556in}}{\pgfqpoint{0.000000in}{0.000000in}}{%
\pgfpathmoveto{\pgfqpoint{0.000000in}{0.000000in}}%
\pgfpathlineto{\pgfqpoint{0.000000in}{-0.055556in}}%
\pgfusepath{stroke,fill}%
}%
\begin{pgfscope}%
\pgfsys@transformshift{4.720000in}{5.400000in}%
\pgfsys@useobject{currentmarker}{}%
\end{pgfscope}%
\end{pgfscope}%
\begin{pgfscope}%
\definecolor{textcolor}{rgb}{0.000000,0.000000,0.000000}%
\pgfsetstrokecolor{textcolor}%
\pgfsetfillcolor{textcolor}%
\pgftext[x=4.720000in,y=0.544444in,,top]{\color{textcolor}\rmfamily\fontsize{10.000000}{12.000000}\selectfont \(\displaystyle {60}\)}%
\end{pgfscope}%
\begin{pgfscope}%
\pgfpathrectangle{\pgfqpoint{1.000000in}{0.600000in}}{\pgfqpoint{6.200000in}{4.800000in}}%
\pgfusepath{clip}%
\pgfsetbuttcap%
\pgfsetroundjoin%
\pgfsetlinewidth{0.501875pt}%
\definecolor{currentstroke}{rgb}{0.000000,0.000000,0.000000}%
\pgfsetstrokecolor{currentstroke}%
\pgfsetdash{{1.000000pt}{3.000000pt}}{0.000000pt}%
\pgfpathmoveto{\pgfqpoint{5.960000in}{0.600000in}}%
\pgfpathlineto{\pgfqpoint{5.960000in}{5.400000in}}%
\pgfusepath{stroke}%
\end{pgfscope}%
\begin{pgfscope}%
\pgfsetbuttcap%
\pgfsetroundjoin%
\definecolor{currentfill}{rgb}{0.000000,0.000000,0.000000}%
\pgfsetfillcolor{currentfill}%
\pgfsetlinewidth{0.501875pt}%
\definecolor{currentstroke}{rgb}{0.000000,0.000000,0.000000}%
\pgfsetstrokecolor{currentstroke}%
\pgfsetdash{}{0pt}%
\pgfsys@defobject{currentmarker}{\pgfqpoint{0.000000in}{0.000000in}}{\pgfqpoint{0.000000in}{0.055556in}}{%
\pgfpathmoveto{\pgfqpoint{0.000000in}{0.000000in}}%
\pgfpathlineto{\pgfqpoint{0.000000in}{0.055556in}}%
\pgfusepath{stroke,fill}%
}%
\begin{pgfscope}%
\pgfsys@transformshift{5.960000in}{0.600000in}%
\pgfsys@useobject{currentmarker}{}%
\end{pgfscope}%
\end{pgfscope}%
\begin{pgfscope}%
\pgfsetbuttcap%
\pgfsetroundjoin%
\definecolor{currentfill}{rgb}{0.000000,0.000000,0.000000}%
\pgfsetfillcolor{currentfill}%
\pgfsetlinewidth{0.501875pt}%
\definecolor{currentstroke}{rgb}{0.000000,0.000000,0.000000}%
\pgfsetstrokecolor{currentstroke}%
\pgfsetdash{}{0pt}%
\pgfsys@defobject{currentmarker}{\pgfqpoint{0.000000in}{-0.055556in}}{\pgfqpoint{0.000000in}{0.000000in}}{%
\pgfpathmoveto{\pgfqpoint{0.000000in}{0.000000in}}%
\pgfpathlineto{\pgfqpoint{0.000000in}{-0.055556in}}%
\pgfusepath{stroke,fill}%
}%
\begin{pgfscope}%
\pgfsys@transformshift{5.960000in}{5.400000in}%
\pgfsys@useobject{currentmarker}{}%
\end{pgfscope}%
\end{pgfscope}%
\begin{pgfscope}%
\definecolor{textcolor}{rgb}{0.000000,0.000000,0.000000}%
\pgfsetstrokecolor{textcolor}%
\pgfsetfillcolor{textcolor}%
\pgftext[x=5.960000in,y=0.544444in,,top]{\color{textcolor}\rmfamily\fontsize{10.000000}{12.000000}\selectfont \(\displaystyle {80}\)}%
\end{pgfscope}%
\begin{pgfscope}%
\pgfpathrectangle{\pgfqpoint{1.000000in}{0.600000in}}{\pgfqpoint{6.200000in}{4.800000in}}%
\pgfusepath{clip}%
\pgfsetbuttcap%
\pgfsetroundjoin%
\pgfsetlinewidth{0.501875pt}%
\definecolor{currentstroke}{rgb}{0.000000,0.000000,0.000000}%
\pgfsetstrokecolor{currentstroke}%
\pgfsetdash{{1.000000pt}{3.000000pt}}{0.000000pt}%
\pgfpathmoveto{\pgfqpoint{7.200000in}{0.600000in}}%
\pgfpathlineto{\pgfqpoint{7.200000in}{5.400000in}}%
\pgfusepath{stroke}%
\end{pgfscope}%
\begin{pgfscope}%
\pgfsetbuttcap%
\pgfsetroundjoin%
\definecolor{currentfill}{rgb}{0.000000,0.000000,0.000000}%
\pgfsetfillcolor{currentfill}%
\pgfsetlinewidth{0.501875pt}%
\definecolor{currentstroke}{rgb}{0.000000,0.000000,0.000000}%
\pgfsetstrokecolor{currentstroke}%
\pgfsetdash{}{0pt}%
\pgfsys@defobject{currentmarker}{\pgfqpoint{0.000000in}{0.000000in}}{\pgfqpoint{0.000000in}{0.055556in}}{%
\pgfpathmoveto{\pgfqpoint{0.000000in}{0.000000in}}%
\pgfpathlineto{\pgfqpoint{0.000000in}{0.055556in}}%
\pgfusepath{stroke,fill}%
}%
\begin{pgfscope}%
\pgfsys@transformshift{7.200000in}{0.600000in}%
\pgfsys@useobject{currentmarker}{}%
\end{pgfscope}%
\end{pgfscope}%
\begin{pgfscope}%
\pgfsetbuttcap%
\pgfsetroundjoin%
\definecolor{currentfill}{rgb}{0.000000,0.000000,0.000000}%
\pgfsetfillcolor{currentfill}%
\pgfsetlinewidth{0.501875pt}%
\definecolor{currentstroke}{rgb}{0.000000,0.000000,0.000000}%
\pgfsetstrokecolor{currentstroke}%
\pgfsetdash{}{0pt}%
\pgfsys@defobject{currentmarker}{\pgfqpoint{0.000000in}{-0.055556in}}{\pgfqpoint{0.000000in}{0.000000in}}{%
\pgfpathmoveto{\pgfqpoint{0.000000in}{0.000000in}}%
\pgfpathlineto{\pgfqpoint{0.000000in}{-0.055556in}}%
\pgfusepath{stroke,fill}%
}%
\begin{pgfscope}%
\pgfsys@transformshift{7.200000in}{5.400000in}%
\pgfsys@useobject{currentmarker}{}%
\end{pgfscope}%
\end{pgfscope}%
\begin{pgfscope}%
\definecolor{textcolor}{rgb}{0.000000,0.000000,0.000000}%
\pgfsetstrokecolor{textcolor}%
\pgfsetfillcolor{textcolor}%
\pgftext[x=7.200000in,y=0.544444in,,top]{\color{textcolor}\rmfamily\fontsize{10.000000}{12.000000}\selectfont \(\displaystyle {100}\)}%
\end{pgfscope}%
\begin{pgfscope}%
\definecolor{textcolor}{rgb}{0.000000,0.000000,0.000000}%
\pgfsetstrokecolor{textcolor}%
\pgfsetfillcolor{textcolor}%
\pgftext[x=4.100000in,y=0.351543in,,top]{\color{textcolor}\rmfamily\fontsize{12.000000}{14.400000}\selectfont \(\displaystyle time\ (s)\)}%
\end{pgfscope}%
\begin{pgfscope}%
\pgfpathrectangle{\pgfqpoint{1.000000in}{0.600000in}}{\pgfqpoint{6.200000in}{4.800000in}}%
\pgfusepath{clip}%
\pgfsetbuttcap%
\pgfsetroundjoin%
\pgfsetlinewidth{0.501875pt}%
\definecolor{currentstroke}{rgb}{0.000000,0.000000,0.000000}%
\pgfsetstrokecolor{currentstroke}%
\pgfsetdash{{1.000000pt}{3.000000pt}}{0.000000pt}%
\pgfpathmoveto{\pgfqpoint{1.000000in}{0.600000in}}%
\pgfpathlineto{\pgfqpoint{7.200000in}{0.600000in}}%
\pgfusepath{stroke}%
\end{pgfscope}%
\begin{pgfscope}%
\pgfsetbuttcap%
\pgfsetroundjoin%
\definecolor{currentfill}{rgb}{0.000000,0.000000,0.000000}%
\pgfsetfillcolor{currentfill}%
\pgfsetlinewidth{0.501875pt}%
\definecolor{currentstroke}{rgb}{0.000000,0.000000,0.000000}%
\pgfsetstrokecolor{currentstroke}%
\pgfsetdash{}{0pt}%
\pgfsys@defobject{currentmarker}{\pgfqpoint{0.000000in}{0.000000in}}{\pgfqpoint{0.055556in}{0.000000in}}{%
\pgfpathmoveto{\pgfqpoint{0.000000in}{0.000000in}}%
\pgfpathlineto{\pgfqpoint{0.055556in}{0.000000in}}%
\pgfusepath{stroke,fill}%
}%
\begin{pgfscope}%
\pgfsys@transformshift{1.000000in}{0.600000in}%
\pgfsys@useobject{currentmarker}{}%
\end{pgfscope}%
\end{pgfscope}%
\begin{pgfscope}%
\pgfsetbuttcap%
\pgfsetroundjoin%
\definecolor{currentfill}{rgb}{0.000000,0.000000,0.000000}%
\pgfsetfillcolor{currentfill}%
\pgfsetlinewidth{0.501875pt}%
\definecolor{currentstroke}{rgb}{0.000000,0.000000,0.000000}%
\pgfsetstrokecolor{currentstroke}%
\pgfsetdash{}{0pt}%
\pgfsys@defobject{currentmarker}{\pgfqpoint{-0.055556in}{0.000000in}}{\pgfqpoint{-0.000000in}{0.000000in}}{%
\pgfpathmoveto{\pgfqpoint{-0.000000in}{0.000000in}}%
\pgfpathlineto{\pgfqpoint{-0.055556in}{0.000000in}}%
\pgfusepath{stroke,fill}%
}%
\begin{pgfscope}%
\pgfsys@transformshift{7.200000in}{0.600000in}%
\pgfsys@useobject{currentmarker}{}%
\end{pgfscope}%
\end{pgfscope}%
\begin{pgfscope}%
\definecolor{textcolor}{rgb}{0.000000,0.000000,0.000000}%
\pgfsetstrokecolor{textcolor}%
\pgfsetfillcolor{textcolor}%
\pgftext[x=0.944444in,y=0.600000in,right,]{\color{textcolor}\rmfamily\fontsize{10.000000}{12.000000}\selectfont \(\displaystyle {\ensuremath{-}2000}\)}%
\end{pgfscope}%
\begin{pgfscope}%
\pgfpathrectangle{\pgfqpoint{1.000000in}{0.600000in}}{\pgfqpoint{6.200000in}{4.800000in}}%
\pgfusepath{clip}%
\pgfsetbuttcap%
\pgfsetroundjoin%
\pgfsetlinewidth{0.501875pt}%
\definecolor{currentstroke}{rgb}{0.000000,0.000000,0.000000}%
\pgfsetstrokecolor{currentstroke}%
\pgfsetdash{{1.000000pt}{3.000000pt}}{0.000000pt}%
\pgfpathmoveto{\pgfqpoint{1.000000in}{1.560000in}}%
\pgfpathlineto{\pgfqpoint{7.200000in}{1.560000in}}%
\pgfusepath{stroke}%
\end{pgfscope}%
\begin{pgfscope}%
\pgfsetbuttcap%
\pgfsetroundjoin%
\definecolor{currentfill}{rgb}{0.000000,0.000000,0.000000}%
\pgfsetfillcolor{currentfill}%
\pgfsetlinewidth{0.501875pt}%
\definecolor{currentstroke}{rgb}{0.000000,0.000000,0.000000}%
\pgfsetstrokecolor{currentstroke}%
\pgfsetdash{}{0pt}%
\pgfsys@defobject{currentmarker}{\pgfqpoint{0.000000in}{0.000000in}}{\pgfqpoint{0.055556in}{0.000000in}}{%
\pgfpathmoveto{\pgfqpoint{0.000000in}{0.000000in}}%
\pgfpathlineto{\pgfqpoint{0.055556in}{0.000000in}}%
\pgfusepath{stroke,fill}%
}%
\begin{pgfscope}%
\pgfsys@transformshift{1.000000in}{1.560000in}%
\pgfsys@useobject{currentmarker}{}%
\end{pgfscope}%
\end{pgfscope}%
\begin{pgfscope}%
\pgfsetbuttcap%
\pgfsetroundjoin%
\definecolor{currentfill}{rgb}{0.000000,0.000000,0.000000}%
\pgfsetfillcolor{currentfill}%
\pgfsetlinewidth{0.501875pt}%
\definecolor{currentstroke}{rgb}{0.000000,0.000000,0.000000}%
\pgfsetstrokecolor{currentstroke}%
\pgfsetdash{}{0pt}%
\pgfsys@defobject{currentmarker}{\pgfqpoint{-0.055556in}{0.000000in}}{\pgfqpoint{-0.000000in}{0.000000in}}{%
\pgfpathmoveto{\pgfqpoint{-0.000000in}{0.000000in}}%
\pgfpathlineto{\pgfqpoint{-0.055556in}{0.000000in}}%
\pgfusepath{stroke,fill}%
}%
\begin{pgfscope}%
\pgfsys@transformshift{7.200000in}{1.560000in}%
\pgfsys@useobject{currentmarker}{}%
\end{pgfscope}%
\end{pgfscope}%
\begin{pgfscope}%
\definecolor{textcolor}{rgb}{0.000000,0.000000,0.000000}%
\pgfsetstrokecolor{textcolor}%
\pgfsetfillcolor{textcolor}%
\pgftext[x=0.944444in,y=1.560000in,right,]{\color{textcolor}\rmfamily\fontsize{10.000000}{12.000000}\selectfont \(\displaystyle {\ensuremath{-}1500}\)}%
\end{pgfscope}%
\begin{pgfscope}%
\pgfpathrectangle{\pgfqpoint{1.000000in}{0.600000in}}{\pgfqpoint{6.200000in}{4.800000in}}%
\pgfusepath{clip}%
\pgfsetbuttcap%
\pgfsetroundjoin%
\pgfsetlinewidth{0.501875pt}%
\definecolor{currentstroke}{rgb}{0.000000,0.000000,0.000000}%
\pgfsetstrokecolor{currentstroke}%
\pgfsetdash{{1.000000pt}{3.000000pt}}{0.000000pt}%
\pgfpathmoveto{\pgfqpoint{1.000000in}{2.520000in}}%
\pgfpathlineto{\pgfqpoint{7.200000in}{2.520000in}}%
\pgfusepath{stroke}%
\end{pgfscope}%
\begin{pgfscope}%
\pgfsetbuttcap%
\pgfsetroundjoin%
\definecolor{currentfill}{rgb}{0.000000,0.000000,0.000000}%
\pgfsetfillcolor{currentfill}%
\pgfsetlinewidth{0.501875pt}%
\definecolor{currentstroke}{rgb}{0.000000,0.000000,0.000000}%
\pgfsetstrokecolor{currentstroke}%
\pgfsetdash{}{0pt}%
\pgfsys@defobject{currentmarker}{\pgfqpoint{0.000000in}{0.000000in}}{\pgfqpoint{0.055556in}{0.000000in}}{%
\pgfpathmoveto{\pgfqpoint{0.000000in}{0.000000in}}%
\pgfpathlineto{\pgfqpoint{0.055556in}{0.000000in}}%
\pgfusepath{stroke,fill}%
}%
\begin{pgfscope}%
\pgfsys@transformshift{1.000000in}{2.520000in}%
\pgfsys@useobject{currentmarker}{}%
\end{pgfscope}%
\end{pgfscope}%
\begin{pgfscope}%
\pgfsetbuttcap%
\pgfsetroundjoin%
\definecolor{currentfill}{rgb}{0.000000,0.000000,0.000000}%
\pgfsetfillcolor{currentfill}%
\pgfsetlinewidth{0.501875pt}%
\definecolor{currentstroke}{rgb}{0.000000,0.000000,0.000000}%
\pgfsetstrokecolor{currentstroke}%
\pgfsetdash{}{0pt}%
\pgfsys@defobject{currentmarker}{\pgfqpoint{-0.055556in}{0.000000in}}{\pgfqpoint{-0.000000in}{0.000000in}}{%
\pgfpathmoveto{\pgfqpoint{-0.000000in}{0.000000in}}%
\pgfpathlineto{\pgfqpoint{-0.055556in}{0.000000in}}%
\pgfusepath{stroke,fill}%
}%
\begin{pgfscope}%
\pgfsys@transformshift{7.200000in}{2.520000in}%
\pgfsys@useobject{currentmarker}{}%
\end{pgfscope}%
\end{pgfscope}%
\begin{pgfscope}%
\definecolor{textcolor}{rgb}{0.000000,0.000000,0.000000}%
\pgfsetstrokecolor{textcolor}%
\pgfsetfillcolor{textcolor}%
\pgftext[x=0.944444in,y=2.520000in,right,]{\color{textcolor}\rmfamily\fontsize{10.000000}{12.000000}\selectfont \(\displaystyle {\ensuremath{-}1000}\)}%
\end{pgfscope}%
\begin{pgfscope}%
\pgfpathrectangle{\pgfqpoint{1.000000in}{0.600000in}}{\pgfqpoint{6.200000in}{4.800000in}}%
\pgfusepath{clip}%
\pgfsetbuttcap%
\pgfsetroundjoin%
\pgfsetlinewidth{0.501875pt}%
\definecolor{currentstroke}{rgb}{0.000000,0.000000,0.000000}%
\pgfsetstrokecolor{currentstroke}%
\pgfsetdash{{1.000000pt}{3.000000pt}}{0.000000pt}%
\pgfpathmoveto{\pgfqpoint{1.000000in}{3.480000in}}%
\pgfpathlineto{\pgfqpoint{7.200000in}{3.480000in}}%
\pgfusepath{stroke}%
\end{pgfscope}%
\begin{pgfscope}%
\pgfsetbuttcap%
\pgfsetroundjoin%
\definecolor{currentfill}{rgb}{0.000000,0.000000,0.000000}%
\pgfsetfillcolor{currentfill}%
\pgfsetlinewidth{0.501875pt}%
\definecolor{currentstroke}{rgb}{0.000000,0.000000,0.000000}%
\pgfsetstrokecolor{currentstroke}%
\pgfsetdash{}{0pt}%
\pgfsys@defobject{currentmarker}{\pgfqpoint{0.000000in}{0.000000in}}{\pgfqpoint{0.055556in}{0.000000in}}{%
\pgfpathmoveto{\pgfqpoint{0.000000in}{0.000000in}}%
\pgfpathlineto{\pgfqpoint{0.055556in}{0.000000in}}%
\pgfusepath{stroke,fill}%
}%
\begin{pgfscope}%
\pgfsys@transformshift{1.000000in}{3.480000in}%
\pgfsys@useobject{currentmarker}{}%
\end{pgfscope}%
\end{pgfscope}%
\begin{pgfscope}%
\pgfsetbuttcap%
\pgfsetroundjoin%
\definecolor{currentfill}{rgb}{0.000000,0.000000,0.000000}%
\pgfsetfillcolor{currentfill}%
\pgfsetlinewidth{0.501875pt}%
\definecolor{currentstroke}{rgb}{0.000000,0.000000,0.000000}%
\pgfsetstrokecolor{currentstroke}%
\pgfsetdash{}{0pt}%
\pgfsys@defobject{currentmarker}{\pgfqpoint{-0.055556in}{0.000000in}}{\pgfqpoint{-0.000000in}{0.000000in}}{%
\pgfpathmoveto{\pgfqpoint{-0.000000in}{0.000000in}}%
\pgfpathlineto{\pgfqpoint{-0.055556in}{0.000000in}}%
\pgfusepath{stroke,fill}%
}%
\begin{pgfscope}%
\pgfsys@transformshift{7.200000in}{3.480000in}%
\pgfsys@useobject{currentmarker}{}%
\end{pgfscope}%
\end{pgfscope}%
\begin{pgfscope}%
\definecolor{textcolor}{rgb}{0.000000,0.000000,0.000000}%
\pgfsetstrokecolor{textcolor}%
\pgfsetfillcolor{textcolor}%
\pgftext[x=0.944444in,y=3.480000in,right,]{\color{textcolor}\rmfamily\fontsize{10.000000}{12.000000}\selectfont \(\displaystyle {\ensuremath{-}500}\)}%
\end{pgfscope}%
\begin{pgfscope}%
\pgfpathrectangle{\pgfqpoint{1.000000in}{0.600000in}}{\pgfqpoint{6.200000in}{4.800000in}}%
\pgfusepath{clip}%
\pgfsetbuttcap%
\pgfsetroundjoin%
\pgfsetlinewidth{0.501875pt}%
\definecolor{currentstroke}{rgb}{0.000000,0.000000,0.000000}%
\pgfsetstrokecolor{currentstroke}%
\pgfsetdash{{1.000000pt}{3.000000pt}}{0.000000pt}%
\pgfpathmoveto{\pgfqpoint{1.000000in}{4.440000in}}%
\pgfpathlineto{\pgfqpoint{7.200000in}{4.440000in}}%
\pgfusepath{stroke}%
\end{pgfscope}%
\begin{pgfscope}%
\pgfsetbuttcap%
\pgfsetroundjoin%
\definecolor{currentfill}{rgb}{0.000000,0.000000,0.000000}%
\pgfsetfillcolor{currentfill}%
\pgfsetlinewidth{0.501875pt}%
\definecolor{currentstroke}{rgb}{0.000000,0.000000,0.000000}%
\pgfsetstrokecolor{currentstroke}%
\pgfsetdash{}{0pt}%
\pgfsys@defobject{currentmarker}{\pgfqpoint{0.000000in}{0.000000in}}{\pgfqpoint{0.055556in}{0.000000in}}{%
\pgfpathmoveto{\pgfqpoint{0.000000in}{0.000000in}}%
\pgfpathlineto{\pgfqpoint{0.055556in}{0.000000in}}%
\pgfusepath{stroke,fill}%
}%
\begin{pgfscope}%
\pgfsys@transformshift{1.000000in}{4.440000in}%
\pgfsys@useobject{currentmarker}{}%
\end{pgfscope}%
\end{pgfscope}%
\begin{pgfscope}%
\pgfsetbuttcap%
\pgfsetroundjoin%
\definecolor{currentfill}{rgb}{0.000000,0.000000,0.000000}%
\pgfsetfillcolor{currentfill}%
\pgfsetlinewidth{0.501875pt}%
\definecolor{currentstroke}{rgb}{0.000000,0.000000,0.000000}%
\pgfsetstrokecolor{currentstroke}%
\pgfsetdash{}{0pt}%
\pgfsys@defobject{currentmarker}{\pgfqpoint{-0.055556in}{0.000000in}}{\pgfqpoint{-0.000000in}{0.000000in}}{%
\pgfpathmoveto{\pgfqpoint{-0.000000in}{0.000000in}}%
\pgfpathlineto{\pgfqpoint{-0.055556in}{0.000000in}}%
\pgfusepath{stroke,fill}%
}%
\begin{pgfscope}%
\pgfsys@transformshift{7.200000in}{4.440000in}%
\pgfsys@useobject{currentmarker}{}%
\end{pgfscope}%
\end{pgfscope}%
\begin{pgfscope}%
\definecolor{textcolor}{rgb}{0.000000,0.000000,0.000000}%
\pgfsetstrokecolor{textcolor}%
\pgfsetfillcolor{textcolor}%
\pgftext[x=0.944444in,y=4.440000in,right,]{\color{textcolor}\rmfamily\fontsize{10.000000}{12.000000}\selectfont \(\displaystyle {0}\)}%
\end{pgfscope}%
\begin{pgfscope}%
\pgfpathrectangle{\pgfqpoint{1.000000in}{0.600000in}}{\pgfqpoint{6.200000in}{4.800000in}}%
\pgfusepath{clip}%
\pgfsetbuttcap%
\pgfsetroundjoin%
\pgfsetlinewidth{0.501875pt}%
\definecolor{currentstroke}{rgb}{0.000000,0.000000,0.000000}%
\pgfsetstrokecolor{currentstroke}%
\pgfsetdash{{1.000000pt}{3.000000pt}}{0.000000pt}%
\pgfpathmoveto{\pgfqpoint{1.000000in}{5.400000in}}%
\pgfpathlineto{\pgfqpoint{7.200000in}{5.400000in}}%
\pgfusepath{stroke}%
\end{pgfscope}%
\begin{pgfscope}%
\pgfsetbuttcap%
\pgfsetroundjoin%
\definecolor{currentfill}{rgb}{0.000000,0.000000,0.000000}%
\pgfsetfillcolor{currentfill}%
\pgfsetlinewidth{0.501875pt}%
\definecolor{currentstroke}{rgb}{0.000000,0.000000,0.000000}%
\pgfsetstrokecolor{currentstroke}%
\pgfsetdash{}{0pt}%
\pgfsys@defobject{currentmarker}{\pgfqpoint{0.000000in}{0.000000in}}{\pgfqpoint{0.055556in}{0.000000in}}{%
\pgfpathmoveto{\pgfqpoint{0.000000in}{0.000000in}}%
\pgfpathlineto{\pgfqpoint{0.055556in}{0.000000in}}%
\pgfusepath{stroke,fill}%
}%
\begin{pgfscope}%
\pgfsys@transformshift{1.000000in}{5.400000in}%
\pgfsys@useobject{currentmarker}{}%
\end{pgfscope}%
\end{pgfscope}%
\begin{pgfscope}%
\pgfsetbuttcap%
\pgfsetroundjoin%
\definecolor{currentfill}{rgb}{0.000000,0.000000,0.000000}%
\pgfsetfillcolor{currentfill}%
\pgfsetlinewidth{0.501875pt}%
\definecolor{currentstroke}{rgb}{0.000000,0.000000,0.000000}%
\pgfsetstrokecolor{currentstroke}%
\pgfsetdash{}{0pt}%
\pgfsys@defobject{currentmarker}{\pgfqpoint{-0.055556in}{0.000000in}}{\pgfqpoint{-0.000000in}{0.000000in}}{%
\pgfpathmoveto{\pgfqpoint{-0.000000in}{0.000000in}}%
\pgfpathlineto{\pgfqpoint{-0.055556in}{0.000000in}}%
\pgfusepath{stroke,fill}%
}%
\begin{pgfscope}%
\pgfsys@transformshift{7.200000in}{5.400000in}%
\pgfsys@useobject{currentmarker}{}%
\end{pgfscope}%
\end{pgfscope}%
\begin{pgfscope}%
\definecolor{textcolor}{rgb}{0.000000,0.000000,0.000000}%
\pgfsetstrokecolor{textcolor}%
\pgfsetfillcolor{textcolor}%
\pgftext[x=0.944444in,y=5.400000in,right,]{\color{textcolor}\rmfamily\fontsize{10.000000}{12.000000}\selectfont \(\displaystyle {500}\)}%
\end{pgfscope}%
\begin{pgfscope}%
\definecolor{textcolor}{rgb}{0.000000,0.000000,0.000000}%
\pgfsetstrokecolor{textcolor}%
\pgfsetfillcolor{textcolor}%
\pgftext[x=0.489196in,y=3.000000in,,bottom,rotate=90.000000]{\color{textcolor}\rmfamily\fontsize{12.000000}{14.400000}\selectfont \(\displaystyle \theta\ (rad)\)}%
\end{pgfscope}%
\begin{pgfscope}%
\definecolor{textcolor}{rgb}{0.000000,0.000000,0.000000}%
\pgfsetstrokecolor{textcolor}%
\pgfsetfillcolor{textcolor}%
\pgftext[x=4.100000in,y=5.469444in,,base]{\color{textcolor}\rmfamily\fontsize{12.000000}{14.400000}\selectfont \(\displaystyle Simple\ pendulum\ solution\ (time\ step = 20\ (s))\)}%
\end{pgfscope}%
\begin{pgfscope}%
\pgfsetbuttcap%
\pgfsetmiterjoin%
\definecolor{currentfill}{rgb}{1.000000,1.000000,1.000000}%
\pgfsetfillcolor{currentfill}%
\pgfsetlinewidth{1.003750pt}%
\definecolor{currentstroke}{rgb}{0.000000,0.000000,0.000000}%
\pgfsetstrokecolor{currentstroke}%
\pgfsetdash{}{0pt}%
\pgfpathmoveto{\pgfqpoint{5.106890in}{4.569445in}}%
\pgfpathlineto{\pgfqpoint{7.116667in}{4.569445in}}%
\pgfpathlineto{\pgfqpoint{7.116667in}{5.316667in}}%
\pgfpathlineto{\pgfqpoint{5.106890in}{5.316667in}}%
\pgfpathlineto{\pgfqpoint{5.106890in}{4.569445in}}%
\pgfpathclose%
\pgfusepath{stroke,fill}%
\end{pgfscope}%
\begin{pgfscope}%
\pgfsetrectcap%
\pgfsetroundjoin%
\pgfsetlinewidth{1.003750pt}%
\definecolor{currentstroke}{rgb}{1.000000,0.000000,0.000000}%
\pgfsetstrokecolor{currentstroke}%
\pgfsetdash{}{0pt}%
\pgfpathmoveto{\pgfqpoint{5.223556in}{5.191667in}}%
\pgfpathlineto{\pgfqpoint{5.456890in}{5.191667in}}%
\pgfusepath{stroke}%
\end{pgfscope}%
\begin{pgfscope}%
\definecolor{textcolor}{rgb}{0.000000,0.000000,0.000000}%
\pgfsetstrokecolor{textcolor}%
\pgfsetfillcolor{textcolor}%
\pgftext[x=5.640223in,y=5.133333in,left,base]{\color{textcolor}\rmfamily\fontsize{12.000000}{14.400000}\selectfont \(\displaystyle euler\ explicit\)}%
\end{pgfscope}%
\begin{pgfscope}%
\pgfsetrectcap%
\pgfsetroundjoin%
\pgfsetlinewidth{1.003750pt}%
\definecolor{currentstroke}{rgb}{0.000000,0.000000,1.000000}%
\pgfsetstrokecolor{currentstroke}%
\pgfsetdash{}{0pt}%
\pgfpathmoveto{\pgfqpoint{5.223556in}{4.959260in}}%
\pgfpathlineto{\pgfqpoint{5.456890in}{4.959260in}}%
\pgfusepath{stroke}%
\end{pgfscope}%
\begin{pgfscope}%
\definecolor{textcolor}{rgb}{0.000000,0.000000,0.000000}%
\pgfsetstrokecolor{textcolor}%
\pgfsetfillcolor{textcolor}%
\pgftext[x=5.640223in,y=4.900926in,left,base]{\color{textcolor}\rmfamily\fontsize{12.000000}{14.400000}\selectfont \(\displaystyle euler\ implicit\)}%
\end{pgfscope}%
\begin{pgfscope}%
\pgfsetrectcap%
\pgfsetroundjoin%
\pgfsetlinewidth{1.003750pt}%
\definecolor{currentstroke}{rgb}{0.000000,0.000000,0.000000}%
\pgfsetstrokecolor{currentstroke}%
\pgfsetdash{}{0pt}%
\pgfpathmoveto{\pgfqpoint{5.223556in}{4.726852in}}%
\pgfpathlineto{\pgfqpoint{5.456890in}{4.726852in}}%
\pgfusepath{stroke}%
\end{pgfscope}%
\begin{pgfscope}%
\definecolor{textcolor}{rgb}{0.000000,0.000000,0.000000}%
\pgfsetstrokecolor{textcolor}%
\pgfsetfillcolor{textcolor}%
\pgftext[x=5.640223in,y=4.668519in,left,base]{\color{textcolor}\rmfamily\fontsize{12.000000}{14.400000}\selectfont \(\displaystyle trapezoidal\ scheme\)}%
\end{pgfscope}%
\end{pgfpicture}%
\makeatother%
\endgroup%
}
    \end{figure}

\end{enumerate}
\end{document}
