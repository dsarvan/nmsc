\documentclass{article}
\usepackage{hyperref}
\usepackage[table]{xcolor}
\usepackage{listings}
\usepackage{lmodern}
\usepackage[left=0.25in, right=0.25in, top=0.75in, bottom=0.75in]{geometry}
\usepackage{graphicx}
\usepackage{amsmath,amssymb}
\usepackage{tikz}
\usepackage{pgfplots}
\usepackage{subfigure}
\usepackage{enumerate}
\usepackage{tcolorbox}
\usepackage{fancyhdr}
\usepackage{cancel}
\usepackage{placeins}
\usepackage{multirow}
\usepackage{algorithm2e}
\pagecolor{white}
\color{black}

\pagestyle{fancy}

\hypersetup{%
  colorlinks=true,% hyperlinks will be black
  linkbordercolor=red,% hyperlink borders will be red
  pdfborderstyle={/S/U/W 1}% border style will be underline of width 1pt
}

\newcommand{\soln}{\\ \textbf{Solution}: }
\newcommand{\bkt}[1]{\left(#1\right)}

\lhead{MA5892: Numerical Methods and Scientific Computing}
\chead{Assignment: 6}
\rhead{Rollnumber: PH15M015}

\begin{document}
\begin{enumerate}
\item A common problem in mathematical physics is that of solving the Fredholm integral
equation

\begin{equation*}
f(x) = \phi(x) + \displaystyle \int_{a}^{b} K(x,t) \phi(t) dt
\end{equation*}

where the function $f(x)$ and $K(x,t)$ are given and the problem is to obtain $\phi(x)$.

\begin{itemize}
\item Describe a numerical method for solving the above equation
\item Solve the following equation

\begin{equation*}
\phi(x) = \pi x^{2} + \displaystyle \int_{0}^{\pi} 3 (0.5 sin(3x) - tx^{2}) \phi(t) dt
\end{equation*}

Obtain the exact solution of the above and compare your numerical solution with it.
\end{itemize}

\item Evaluate $I = \displaystyle \int_{0}^{1} \frac{e^{-x}}{\sqrt{x}} dx$ by subdividing
the domain into $ n \in \{5, 10, 20, 50, 100, 200, 500, 1000, 2000, 5000, 10000 \}$ panels.

\begin{itemize}
    \item Using rectangular rule.
    \item Make a change of variable $x = t^{2}$ and use rectangular rule.
\end{itemize}

Compare the two methods above in terms of accuracy and cost. Explain the difference, if
any.

\item The Householder's method is a generalization of the Newton method and the sequence
of iterates is given by

\begin{equation*}
x_{n+1} = x_{n} + d \frac{(1/f)^{d-1} (x_{n})}{(1/f)^{d} (x_{n})}
\end{equation*}

where $(1/f)^{k} (x_{n})$ is the $k^{th}$ derivative of the function $1/f$ evaluated at 
$x_{n}$. Note that taking $d = 1$, we obtain the Newton method. Prove that if $f(x)$ is 
$d + 1$ times continuously differentiable function, i.e., $f^{(d+1)}$ exists and is
continuous, and if the sequence of iterates converge to a root $a$, then we have

\begin{equation*}
|x_{n+1} - a| \leq K |x_{n} - a|^{d+1} for some K > 0 eventually
\end{equation*}

The above statement means that the order of convergence of the above method is $d+1$.

\item Lef $f(x)$ be a twice differentiable strictly convex function with a single simple
(i.e., multiplicity of the root is one) root at $x = a$. Prove that the Newton method 
converges to the root irrespective of the initial guess.

\item Prove that the function $w(x) = xe^{x} - a$ has only one real root for $a > 0$.
\begin{itemize}
\item Write a program to obtain the root of the above using (i) bisection (ii) Newton
method (iii) Secant method.
\item Explain in detail why, when and for what initial guess does each of the method
converge.
\item What happens when $a < 0$? Perform a complete analysis on the convergence for $a <
0$ as well.
\end{itemize}

\end{enumerate}
\end{document}
