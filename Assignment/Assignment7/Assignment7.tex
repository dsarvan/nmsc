\documentclass{article}
\usepackage{hyperref}
\usepackage[table]{xcolor}
\usepackage{listings}
\usepackage{lmodern}
\usepackage[left=0.25in, right=0.25in, top=0.75in, bottom=0.75in]{geometry}
\usepackage{graphicx}
\usepackage{amsmath,amssymb}
\usepackage{tikz}
\usepackage{pgfplots}
\usepackage{subfigure}
\usepackage{enumerate}
\usepackage{tcolorbox}
\usepackage{fancyhdr}
\usepackage{cancel}
\usepackage{placeins}
\usepackage{multirow}
\usepackage{algorithm2e}
\pagecolor{black}
\color{white}

\pagestyle{fancy}


\hypersetup{%
  colorlinks=true,% hyperlinks will be black
  linkbordercolor=red,% hyperlink borders will be red
  pdfborderstyle={/S/U/W 1}% border style will be underline of width 1pt
}

\newcommand{\soln}{\\ \textbf{Solution}: }
\newcommand{\bkt}[1]{\left(#1\right)}
\lhead{MA5892: Numerical Methods and Scientific Computing}

\chead{Assignment: 1}

\rhead{Rollnumber: MA14M099}

\begin{document}
	\begin{enumerate}
		\item
		Does there exist irrational numbers $u$ and $v$ such that $u^v$ is rational?
		\soln
		Yes. Consider the number $x=\sqrt2^{\sqrt2}$.
		\begin{itemize}
			\item
			If $x$ is rational, we are done by picking $u=v = \sqrt2$.
			\item
			Else, let $u=x=\sqrt2^{\sqrt2}$ and $v=\sqrt2$. We now have $u^v = \bkt{\sqrt2^{\sqrt2}}^{\sqrt2} = \bkt{\sqrt2}^2 = 2$ and now we are done.
		\end{itemize}
	\end{enumerate}
\end{document}