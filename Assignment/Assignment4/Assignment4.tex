\documentclass{article}
\usepackage{hyperref}
\usepackage[table]{xcolor}
\usepackage{listings}
\usepackage{lmodern}
\usepackage[left=0.25in, right=0.25in, top=0.75in, bottom=0.75in]{geometry}
\usepackage{graphicx}
\usepackage{amsmath,amssymb}
\usepackage{tikz}
\usepackage{pgfplots}
\usepackage{subfigure}
\usepackage{enumerate}
\usepackage{tcolorbox}
\usepackage{fancyhdr}
\usepackage{cancel}
\usepackage{placeins}
\usepackage{multirow}
\usepackage{algorithm2e}
\usepackage{booktabs}
\usepackage{bbding}
\pagecolor{white}
\color{black}

\pagestyle{fancy}

\hypersetup{%
  colorlinks=true,% hyperlinks will be black
  linkbordercolor=red,% hyperlink borders will be red
  pdfborderstyle={/S/U/W 1}% border style will be underline of width 1pt
}

\newcommand{\soln}{\\ \textbf{Solution}: }
\newcommand{\bkt}[1]{\left(#1\right)}

\lhead{MA5892: Numerical Methods and Scientific Computing}
\chead{Assignment: 4}
\rhead{Roll number: PH15M015}

\begin{document}
\begin{enumerate}
\item Compute $\displaystyle \int_{0}^{1} e^{x^{2}} \ dx$ using the trapezoidal rule and 
trapezoidal rule with end corrections using the first and third derivatives. Perform this
by subdividing $[0, 1]$ into $N \in \{2, 5, 10, 20, 50, 100, 200, 500, 1000\}$ panels and 
plot the decay of the absolute error using the \textbf{three methods}. The value of the 
integral accurate upto $16$ digits is $1.4626517459071816$.

\item Use the Euler-Macluarin to obtain
\begin{equation*}
\log (n!) = \log \displaystyle \left(C \left(\frac{n}{e} \right)^{n} \sqrt{n} \right) 
    + \mathcal{O}(1/n)
\end{equation*}
where $C$ is some constant.

\item We will now determine $C$ in the above question as follows.
\begin{itemize}
    \item Use integration by parts to obtain an expression for $I_{k} = \displaystyle
        \int_{0}^{\pi/2} sin^{k}(x)\ dx$ (It might be easier to look at the even and odd
        cases separately)

    \item Prove that $I_{k}$ is a monotone decreasing sequence.

    \item Show that
        \begin{equation*}
            \lim_{m \to \infty} \frac{I_{2m - 1}}{I_{2m + 1}} = 1
        \end{equation*}

    \item Conclude that
        \begin{equation*}
            \lim_{m \to \infty} \frac{I_{2m}}{I_{2m + 1}} = 1
        \end{equation*}

    \item Hence, infer that the central binomial coefficient is asymptotically given by
        \begin{equation*}
            \binom{2m}{m} \sim \frac{4^{m}}{\sqrt{m\pi}}
        \end{equation*}
        where $f(m) \sim g(m) \Longrightarrow \lim_{m \to \infty} \displaystyle
        \frac{f(m)}{g(m)} = 1$

    \item Conclude that $C$ in the above question is $\sqrt{2\pi}$

    \item Hence, obtain the Stirling formula:
        \begin{equation*}
            n! \sim \left(\frac{n}{e}\right)^{n} \sqrt{2\pi n}
        \end{equation*}

    \item Obtain the relative error in $n!$ using the Stirling formula for $n \in \{20,50\}$

    \item Obtain a better estimate for $n!$, which is accurate upto $\mathcal{O}(1/n^{3})$
\end{itemize}

\end{enumerate}
\end{document}
