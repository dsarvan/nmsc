\documentclass{article}
\usepackage{hyperref}
\usepackage[table]{xcolor}
\usepackage{listings}
\usepackage{lmodern}
\usepackage[left=0.25in, right=0.25in, top=0.75in, bottom=0.75in]{geometry}
\usepackage{graphicx}
\usepackage{amsmath,amssymb}
\usepackage{tikz}
\usepackage{pgfplots}
\usepackage{subfigure}
\usepackage{enumerate}
\usepackage{tcolorbox}
\usepackage{fancyhdr}
\usepackage{cancel}
\usepackage{placeins}
\usepackage{multirow}
\usepackage{algorithm2e}
\usepackage{booktabs}
\usepackage{bbding}
\pagecolor{white}
\color{black}

\pagestyle{fancy}

\hypersetup{%
  colorlinks=true,% hyperlinks will be black
  linkbordercolor=red,% hyperlink borders will be red
  pdfborderstyle={/S/U/W 1}% border style will be underline of width 1pt
}

\newcommand{\soln}{\\ \textbf{Solution}: }
\newcommand{\bkt}[1]{\left(#1\right)}

\lhead{MA5892: Numerical Methods and Scientific Computing}
\chead{Assignment: 3}
\rhead{Rollnumber: PH15M015}

\begin{document}
\begin{enumerate}
\item In numerical solution of boundary value problems in differential equations, we can
sometimes use the physics of the problem not only to enforce boundary conditions but also
to maintain high-order accuracy near the boundary. For example, we may know the heat flux
through a surface or displacement of a beam specified at one end. We can use this
information to produce better estimates of the derivatives near the boundary.

Suppose we want to numerically solve the following boundary value problem with Neumann
boundary conditions:

\begin{equation*}
    \frac{d^{2} y}{dx^{2}} + y = x^{3}, \ 0 \leq x \leq 1
\end{equation*}

with $y^{'}(0) = y^{'}(1) = 0$. We discretize the domain using grid points $x_{i} = 
(i - 0.5)h, \ i \in \{1,2,...,N\}$. In this problem, $y_{i}$ is the numerical estimate of 
$y$ at $x_{i}$. By using a finite difference scheme, we can estimate $y_{i}^{''}$ in terms 
of linear combinations of $y_{i}$'s and transform the ODE into a linear system of equations.

\begin{itemize}

    \item Derive a fourth order Pade approximation for the second derivative at the
        $i^{th}$ node involving only its neighbors $i \pm 1$, i.e., obtain $y_{i}^{''}$
        involving $y_{i \pm 1}$, $y_{i}$ and $y_{i \pm 1}^{''}$. Note that this is
        applicable only at $i \in \{2,3,...,N-1\}$

    \item For the left boundary, derive a third order Pade scheme to approximate
        $y_{1}^{''}$ in the following form:
        \begin{equation*}
            y_{1}^{''} + b_{2} y_{2}^{''} = a_{1}y_{1} + a_{2}y_{2} + a_{3}y_{3} +
            a_{4}y^{'}(0) + O(h^{3})
        \end{equation*}

    \item Repeat the above for the left boundary.

    \item Use the finite difference formulae derived above, to obtain a linear system for
        $y_{i}^{''}$. Explicitly write down the entries in the matrix and the right hand
        side.

    \item Compare the numerical and exact solution by varying $n \in
        \{10,20,50,100,200,500,1000\}$. Plot the rror (computed using the max-norm as a
        function of n) on a log-log plot. Discuss your result.

    \item How are the Neumann boundary conditions enforced into the discretized boundary
        value problem?

\end{itemize}

\end{enumerate}
\end{document}
